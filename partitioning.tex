\chapter{Partitioning hexagonal toruses}
	
	\label{sec:partitioning}
	
	The nodes in super computer networks are usually relatively small, for
	example in SpiNNaker each node is a single chip. To allow several nodes to
	share resources such as power supplies and simplify construction, tens of
	nodes are typically packed together into a single unit such as a circuit
	board or server blade \cite{gilge14,ajima12}. In commercial super computers
	built on non-hexagonal torus topologies, units usually represent a hypercube
	partition of the overall topology as illustrated in figure
	\ref{fig:apdx-hypercube-partitioning} \cite{chen11,ajima12}.
	
	\begin{figure}
		\center
		\begin{subfigure}[b]{0.45\textwidth}
			\center
			\buildfig{figures/hypercube-partitioning.tex}
			\caption{2D hypercube partitioning}
			\label{fig:apdx-hypercube-partitioning}
		\end{subfigure}
		\begin{subfigure}[b]{0.45\textwidth}
			\center
			\buildfig{figures/parallelogram-partitioning.tex}
			\caption{Parallelogram partitioning}
			\label{fig:apdx-parallelogram-partitioning}
		\end{subfigure}
		
		\caption{Partitioning schemes for 2D and hexagonal torus topologies.}
		\label{fig:apdx-partitioning-options}
	\end{figure}
	
	
	The analogue of this scheme in a hexagonal torus topology is a parallelogram
	as illustrated in figure~\ref{fig:apdx-parallelogram-partitioning}.  Each
	partition connects to six neighbouring partitions and, unlike hypercube
	partitions, the number of connections to each is imbalanced.  Specifically
	the partitions above-right and below-left are connected by only one link each
	while others are connected by many. As a consequence multiple interconnect
	technologies may be needed if connections between partitions are concentrated
	into single connections as in SpiNNaker (see chapter~\ref{sec:background}).
	This adds both to design complexity and system cost.
	
	In this appendix, I describe how, and why, the `wrapped triple' partitioning
	scheme devised by Davidson for SpiNNaker works \cite{davidsonWiring}.
	
	\section{Tiling}
		
		For a particular configuration of nodes to form a valid partition, it must
		be possible to use this configuration to `tile' a hexagonal torus. In this
		thesis, a `pointy-topped' hexagon is used to represent a chip in a
		hexagonal torus which is drawn (by convention) in a parallelogram-shaped
		arrangement.
		
		\begin{figure}
			\center
			\buildfig{figures/tiling-a-torus.tex}
			
			\caption{Tiling a hexagonal torus with parallelograms.}
			\label{fig:tiling-a-torus}
		 \end{figure}
		 
		 \begin{figure}
			\center
			\buildfig{figures/parallelogram-tiling.tex}
			
			\caption[A parallelogram tiles a hexagonal torus.]%
			{Visual proof that a parallelogram shares translational symmetry
			with a pointy-topped hexagon.}
			\label{fig:parallelogram-tiling}
		\end{figure}
		
		Any partition which shares the `translational symmetry' of a pointy-topped
		hexagon must also tile a hexagonal torus. For example, in
		figure~\ref{fig:tiling-a-torus} we can see that a $2\times2$ parallelograms
		can tile a $9\times9$ hexagonal torus topology.  In
		figure~\ref{fig:parallelogram-tiling}, I will demonstrate that this
		parallelogram partition shares its translational symmetry with a
		pointy-topped hexagon. In figure~\ref{fig:parallelogram-tiling}a, the
		$2\times2$ partition is shown with each node shaded differently and in
		figure~\ref{fig:parallelogram-tiling}b, a pointy-topped hexagon is
		superimposed. By tiling several copes of the partition
		(figure~\ref{fig:parallelogram-tiling}c), we can see that repeating pattern
		of the parallelogram matches that of the pointy-topped hexagon. I refer to
		this property as the parallelogram sharing the translational symmetry of a
		pointy-topped hexagon.
		
		Notice that the parallelogram partition can be redrawn such that all parts
		protruding from the overlaid pointy-topped hexagon wrap around to fill the
		gaps on opposite sides producing a pointy-topped hexagon shaped tile as
		shown in figure~\ref{fig:parallelogram-tiling}d.
		
	
	\section{How \emph{not} to tile a hexagonal torus}
		
		\begin{figure}[h]
			\center
			\buildfig{figures/wrapped-hexagon-tiling.tex}
			
			\caption[A wraped hexagon does not tile a hexagonal torus.]%
			{Visual proof that a wrapped hexagon does not share translational
			symmetry with a pointy-topped hexagon.}
			\label{fig:wrapped-hexagon-tiling}
		\end{figure}
	
		An `obvious' partitioning scheme for a hexagonal topology which evenly
		distributes links between six sides is to use a hexagonal partition. Such a
		partition might na\"ively be formed by wrapping `layers' of hexagons around
		a central hexagon. Figure~\ref{fig:wrapped-hexagon-tiling}a illustrates a
		simple partition with a single layer of hexagons.
	
		While this type of partition exposes six equally-sized edges (satisfying
		the requirement that connections between partitions should have a balanced
		number of connections) this partition does not share translational symmetry
		with a pointy-topped hexagon. Figure~\ref{fig:wrapped-hexagon-tiling}b
		shows a best-fitting pointy-topped hexagon superimposed on the partition.
		In figure~\ref{fig:wrapped-hexagon-tiling}c, we can see that when tiled,
		the partition leaves gaps between the superimposed pointy-topped hexagon
		indicating it does not share translational symmetry.
		
		Consequently, it is not possible to construct a hexagonal torus from
		partitions of this shape. As an aside, it is possible to construct a
		H-torus topology \cite{zhao08}, a related, but different, topology
		mentioned briefly in \S\ref{sec:hex-vs-h-torus}.
	
	\section{Triads of triples}
		
		\begin{figure}
			\center
			\buildfig{figures/wrapped-triple-tiling.tex}
			
			\caption[Triples share translational symetry with flat-topped hexagons.]%
			{Visual proof that a triple shares the same translational
			symmetry as a flat-topped hexagon.}
			\label{fig:wrapped-triple-tiling}
		\end{figure}
		
		A `triple' is a partition made up of three nodes arranged as in
		figure~\ref{fig:wrapped-triple-tiling}a. This partition's edges may be
		broken into six groups with an equal number of connections, meeting the
		requirements set out at the beginning of this appendix. A triple, however,
		does not share translational symmetry with a pointy-topped hexagon but
		\emph{does} share it with a `flat-topped' hexagon (the
		\SI{30}{\degree}-rotated cousin of the pointy-topped hexagon) as
		demonstrated in figures~\ref{fig:wrapped-triple-tiling}b and
		\ref{fig:wrapped-triple-tiling}c.
		
		\begin{figure}
			\center
			\buildfig{figures/triad-tiling.tex}
			
			\caption[Triads of triples tile a hexagonal torus.]%
			{Demonstration that a triad shares translational symmetry with a
			pointy-topped hexagon.}
			\label{fig:triad-tiling}
		\end{figure}
		
		Because a triple made up of pointy-topped hexagons tiles like a flat-topped
		hexagon, a triple made up of flat-topped hexagons must share translational
		symmetry with a pointy-topped hexagon (turn
		figure~\ref{fig:wrapped-triple-tiling} \SI{30}{\degree} to verify this).
		It follows that three triples arranged as in figure~\ref{fig:triad-tiling}
		-- a `triad' -- share translational symmetry with pointy-topped hexagons.
		Consequently, triads of triples may be used to tile a hexagonal torus
		topology 
	
	\section{Wrapped triples}
		
		\begin{figure}
			\center
			\buildfig{figures/wrapped-triple.tex}
			
			\caption[A wrapped triple with four layers.]%
			{A wrapped triple with four layers. Layers are shaded alternately.}
			\label{fig:wrapped-triple}
		\end{figure}
		
		A triple forms a partition which can be used to tile a hexagonal torus when
		tiled using triads of triples. To increase the number of nodes in the
		partition, layers of hexagons may be wrapped around a triple to form a
		`wrapped triple' as in figure~\ref{fig:wrapped-triple}. Wrapped triples
		have the same balanced communications as triples and triads of
		wrapped-triples also share translational symmetry with a pointy topped
		hexagon. Therefore, triads of wrapped triples may be used to tile a
		hexagonal torus topology.
		
		In SpiNNaker, each board partitions the network into a four-layer wrapped
		triple containing forty~eight chips (nodes) each.
