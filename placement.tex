\chapter{Placing applications in large SpiNNaker machines}
	
	In the previous chapter I tackled the problem of scale in generating routes
	for very large networks such as SpiNNaker. In this work the centroid traffic
	pattern was used as an approximation of the expected network traffic
	generated by `well behaved' neural network simulation software running on
	SpiNNaker. The traffic produced largely exhibits strong locality, that is
	most communication occurs between either nearby nodes or clusters of nodes.
	In reality, neural simulation applications are not specified geometrically
	but rather as abstract graphs of communicating neurons
	\cite{davison08,eliasmith13}. Applications must then \emph{place} these
	neurons onto nodes in a SpiNNaker system, attempting maximise communication
	locality.
	
	Unfortunately, constructing optimal placements is, in general, an NP-complete
	problem and intractable beyond a few tens of nodes leading researchers to
	develop numerous heuristic-driven algorithms for finding `good-enough'
	placement solutions, a number of which I will introduce in the `related work'
	section of this chapter. The placement problem is far from unique to neural
	applications and is widely studied in the context of super computer job
	scheduling though, as I will show, these techniques target applications
	smaller in scale than those intended for SpiNNaker. In the field of chip
	design the placement problem also appears in the form of placing billions of
	transistors while controlling the length of wires between them. In this
	chapter I adapt the techniques used for chip placement for the placement of
	applications for SpiNNaker and achieve XXX?
	
	\section{Related work}
		
		Placement in super computers or high-performance computing systems is often
		referred to as task scheduling since it involves dealing with temporal
		issues and handling queues of independent jobs. In SpiNNaker this is not
		done so instead we explore the subset of one-task placement algorithms
		used.
		
		\subsection{Super computer application placement algorithms}
			
			A number of algorithms are used when placing super computer applications.
			Indeed many researchers have indicated the significance of the impact of
			this task on common super computer loads. Surprisingly little is done in
			industry, despite efforts in academia.
			
			\subsubsection{Application-specific approaches (manual placement)}
				
				In the case of some applications, such as finite element modelling
				\cite{clausen10}, the computation proceeds by breaking a physical space
				into regular pieces and simulating each piece on an individual compute
				node. The compute nodes then communicate with nodes simulating
				neighbouring pieces to allow the model to proceed as one. Such
				approaches have straight-forward mappings onto popular super computer
				topologies such as toruses which are essentially optimal.
				
				TODO: FIGURE SHOWING PLACEMENT FOR FINITE ELEMENT MODELLING
				
				This approach is ideal for applications with obvious mappings but for
				many applications this simply isn't possible. Neural simulations are
				especially non-obvious to place.
			
			\subsubsection{Sequential placement}
				
				Super computer management software such as SLURM \cite{yoo03} and Blue
				Gene's system software \cite{gilge14} default to using straight-forward
				sequential allocation of resources.
				
				When
				network throughput requirements are low or when communication is random
				or broadcast-based this is also sufficient (since no placement is
				better than any other). When networks are small this is also probably
				sufficient. Existing SpiNNaker software uses this approach.
			
			\subsubsection{Greedy Placement}
				
				Some manufacturers such as Cray supply extensions to SLURM which
				allocate applications along a Hilbert curve and following a greedy
				algorithm to decide the order in which applications are placed on the
				curve \cite{yoo03}.
				
				\begin{figure}
					\center
					\buildfig{figures/hilbert-placement.tex}
					
					\caption{A Hilbert curve.}
					\label{fig:hilbert-placement}
				\end{figure}
				
				Hilbert curves map a 1D spaces into 2D spaces, attempting to preserve
				the property that things which are close in 1D will also be close in
				2D. As such this means that by iterating over the nodes in the graph in
				a sensible (linear) order, a sensible 2D placement should arise. This
				approach can work well but is easily misled and is essentially ignoring
				the 2D nature of the network topology.
			
			\subsubsection{Optimisation-based Placement}
				
				% Citations from short report about optimisation in placement...
				% \cite{chen06,jeannot14} and \cite{jeannot10} ("subsets of apps")
				
				In the academic community, a number of attempts have been made to use
				various optimisation algorithms for the placement of applications.
				Unfortunately, many of these simply aren't suited to the scale of
				neural applications running on SpiNNaker (e.g. only cope with tens of
				nodes while SpiNNaker may contain hundreds of thousands).
				
				Additionally, a number of algorithms have been developed which make
				assumptions about the topologies of the problem or network. Tree match
				for example attempts to map tree-shaped problems to tree-shaped
				networks. Such algorithms can be highly effective but again do not
				apply to SpiNNaker or its neural applications.
		
		\subsection{Chip placement algorithms}
			
			The chip-design industry has, for many years, dealt with problems
			analogous to the task of placing super computer jobs in a way suited to
			SpiNNaker. Modern CPUs have millions or billions of components with
			strictly fixed connectivity. CPU designers must place each of these onto
			a chip such that the connection lengths are controlled to reduce
			congestion and increase performance. As such, these algorithms are
			ideally suited to future super computer placement work since they already
			operate at the scales required \cite{nam07}.
			
			\subsubsection{Cost functions}
				
				HPWL is popular but a bit crap for high fan-outs. It is, however, quite
				simple.
				
				TODO: SELECT A BETTER COST FUNCTION...
			
			\subsubsection{Simulated annealing}
				
				One of the oldest techniques used for circuit placement is simulated
				annealing and this remains popular today thanks to its sheer
				versatility (see VPR, other open FPGA tools).
				
				SA works by analogy with the physical process of annealing.
				The simulated annealing algorithm works by selecting random pairs of
				components on a chip, swapping them and evaluating some cost function.
				If the swap reduces the cost function, it is kept, if not, depending on
				a function of the current temperature and the cost introduced by the
				swap.
				
				TODO: ILLUSTRATION OF SIMULATED ANNEALING SWAP OPERATION
				
				By occasionally allowing costly swaps, the annealing algorithm avoids
				becoming trapped in local minima. As the algorithm proceeds, the
				temperature is slowly reduced and with it the proportion of costly
				swaps which are retained. This causes the placement to move from
				exploration early on towards refinement later on.
				
				The temperature schedule of an annealing algorithm is critical to its
				success. In general these schedules are computed based on the
				performance of the algorithm as it runs. In VPR the following schedule
				is used.
				
				TODO: DESCRIBE VPR'S SCHEDULE
				
				TODO: FIND AND DESCRIBE ALTERNATIVE SCHEDULE?
				
				Unfortunately, SA is very difficult to parallelise, especially in the
				case of placement. As a result, its scalability has been limited and
				resulted in significantly reduced usage in recent work.
			
			\subsubsection{Partitioning placement}
				
				Partitioning based placement solves the placement problem using
				graph-partitioning recursively on the problem graph to assign each part
				of the circuit to some area in the super chip. Though a number of
				algorithms have proven successful in academic placement contests over
				the years, they are not popular in industrial settings.
			
			\subsubsection{Analytical placement}
				
				In analytical placement, cost function for the circuit graph is
				approximated in a form which is amenable to solutions with standard
				numerical or symbolic algebraic techniques. Using these techniques,
				exact minimum cost (in terms of the approximation) configurations can
				be obtained.
				
				Quadratic placement is a popular analytical placement technique which
				approximates the cost of a placement as the sum of the squares of the
				distances between connected circuit elements.
				
				TODO: FIGURE EXAMPLE QUADRATIC PLACEMENT PROBLEM AND SOLUTION
				
				As such this gives a quadratic cost function like so which we must
				minimise.
				
				TODO: QUADRATIC COST EQN
				
				To minimise the function we differentiate and solve using simple
				symbolic manipulation.
				
				TODO: QUADRATIC COST SOLUTION
				
				Unfortunately, quadratic placement doesn't contain any congestion
				relief by default so various schemes exist. For example, extra anchor
				nodes are inserted which gently pull the circuit components apart from
				each other. As a result, the algorithm generally proceeds by iterating,
				regenerating anchors each time.
				
				Other non-quadratic analytical methods exist too with numerical
				solutions. The approaches are often similar.
			
			\subsubsection{Hierarchical clustering}
				
				Many placement algorithms scale super-linearly with problem size and so
				larger problems become increasingly problematic to handle. To solve
				this problem clustering techniques are first applied to first simplify
				the placement problem. A solution is then found at the coarse level and
				then hierarchically fleshed out.
				
				Various clustering algorithms are in use.
				
				TODO: TALK ABOUT CLUSTERING IN PLACEMENT...
				
				TODO: DESCRIBE THE ALGORITHM I IMPLEMENTED.
	
	\section{Using chip-placement algorithms for application placement}
		
		As described above, chip and FPGA placement algorithms are already able to
		cope with the scale of the application placement problem but there are a
		few differences which must be considered.
		
		\subsection{Topological restrictions}
			
			Chip placement occurs in 2D Euclidean space: the surface of a chip.
			Unfortunately, super computer network topologies do not often exist in
			such mundane forms. In the case of SpiNNaker we have a torus network
			which causes great problems for many analytical techniques. This is
			because distances in a torus do not increase or decrease monotonically as
			two points are moved in opposite directions. As a result, solution
			methods frequently fall apart.
			
			TODO: FIGURE ILLUSTRATING DISTANCE DISCONTINUITY
		
		\subsection{Placement constraints}
			
			At the simplest level, there is usually a non-overlapping constraint for
			placement of circuit components. In FPGAs there may also be rules about
			which parts of the chip a particular component may be placed (e.g. block
			RAMS can only go in certain places...).
			
			In application placement there are multiple independent resource
			constraints on a given chip. For example, CPU resource and memory
			resource. As a result we're left with a bin packing problem not found in
			circuit placement.
			
			The soft constraints on the placement solution are also similar in that
			routing resources must be minimised and congestion avoided. Modern
			placers may also have timing based constraints which are not generally
			present in software applications due to strongly the non-deterministic
			nature of network latency.
	
	\section{Application placement by simulated annealing}
		
		I have implemented a simplified SA based application placement algorithm
		based on the approach used in the popular VPR place and route tool chain.
		The algorithm is written in C and is optimised for experimentation rather
		than performance but is production-ready. It has been integrated into the
		`Rig' SpiNNaker software tools and has been used to place very large
		simulations. More on that later.
		
		\subsection{Representation}
			
			Model each chip as having a quantity of various resources (e.g. Cores,
			SDRAM) available. The application graph consists of vertices which each
			consume some quantity of these resources. Vertices must be placed on a
			single chip such that the resources required on a given chip do not
			exceed those available. Vertices are then interconnected by 1:N nets with
			weights which act as hints. The nets are treated as a soft constraint:
			vertices connected via a net will, ideally, be placed near to each other,
			with priority being given to nets with higher weights. Additionally there
			will be a list of placement constraints (see later).
			
			A key observation is that while vertices in an application may frequently
			have a 1:1 correspondence with application cores, this need-not be the
			case. For example, a vertex may represent a block of SDRAM which is
			shared. A vertex may also represent some other resource, for example,
			external IO availability. By making these resource types user-defined,
			applications programmers can express flexible hard-constraints on their
			application.
			
			Another observation is that generic soft constraints can be expressed may
			be expressed using a net with an appropriate weight.
			
			As a result of these facilities, application programmers can easily
			express their own application-specific hard and soft placement
			constraints without having to modify the algorithm. This representation
			has become a de-facto standard for placement problem interchange for
			SpiNNaker applications.
		
		\subsection{Cost function}
			
			At present I've used HPWL despite this being really bad for high-fan-out
			multicast and totally ignorant to the hexagonal nature of SpiNNaker...
			
			To compute bounding boxes for tori I use the following approach. For each
			dimension, sort the points on that dimension and find the largest gap
			between them on a ring. The bounding box goes the other way.
			
			TODO: FIGURE ILLUSTRATING BOUNDING BOX COMPUTATION FOR TORI.
		
		\subsection{Annealing schedule}
			
			The annealing schedule is that used by VPR. Despite being for circuit
			placement, it seems to work jolly well.
			
			TODO: DESCRIBE AND RATIONALISE THE SCHEDULE
		
		\subsection{Constraint handling}
			
			Various hard and soft constraints may be expressed by software
			approaches. For each we explain how they may be handled by the placement
			algorithm:
			
			\subsubsection{Location Constraint}
				
				The vertex is placed on a chip and removed from the set of movement
				candidates.
			
			\subsubsection{Same-chip constraint}
				
				When two vertices must always be placed on the same chip they are
				simply combined into one vertex which consumes the sum of their
				resources. Placement then treats them as one chip and thus is forced to
				atomically place the vertices.
			
			\subsubsection{Reserve resource constraint}
				
				Simply reduce resource availability on that chip.
			
			\subsubsection{Keep near Ethernet}
				
				Simply add a net.
	
	\section{Hierarchical placement}
		
		I have implemented the <TODO WHICH ALGO?> clustering algorithm and used
		this as part of an experimental placement system. Maybe I'll remove this
		section...
		
		\subsection{Clustering}
		
		\subsection{Refinement}
			
			Though it is conventional to allow communication between partitions
			during placement, I've just ignored that and wish to see how it goes...
		
		\subsection{Parallelism}
			
			Run each refinement in parallel on SpiNNaker. Uniquely the amount of
			parallelism available will exactly match that available in the problem,
			by definition!
	
	\section{Evaluation}
		
		Though benchmarks exist for super computer loads and chip placement tasks,
		such things don't exist for neural applications. As a result I use a
		selection of real applications for SpiNNaker along with some synthetic
		benchmarks based on biological data.
		
		\subsection{Benchmark networks}
			
			First some real networks.
			
			Some nengo networks: SPAUN: `The world's largest functional brain model'.
			Word-net network from Jamie: Example of some learning.
			
			TODO: DESCRIBE SHAPE OF NENGO NETWORKS
			
			Some PyNN networks: Microcortical column model from PyNN. Note almost
			broadcast connectivity but varying weights. Try and extract a vision
			netlist from Anna. Maybe try and get a netlist for Tom's barrel cortex.
			
			Now for some artificial networks. Pipeline, noisy pipeline, mesh,
			Gaussian 2D.
		
		\subsection{Experiments}
			
			We compare random, linear, greedy and annealing based placement
			approaches to placement. We compare static metrics (such as mean/max
			congestion, table usage) along with experiments based on simulated
			network traffic in real hardware. Network Tester generates artificial
			traffic in proportion with the weights given for each model. We compare
			the relative level of traffic sustainable. We also consider use of
			machines of various sizes.
		
		\subsection{Results}
			
			SA placement is slow but rather effective, especially for some networks.
			Generally worth doing. Will need to be sped up for very large machines...
			
			TODO: EXPERIMENTS!
	
