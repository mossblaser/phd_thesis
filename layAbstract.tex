{
	\prefacesection{Lay abstract}
	
	% Single line spacing for the lay abstract page
	\setstretch{1.0}
	
	
	\vfill
	
	% Standard thesis information
	\begin{center}
		\textsc{\large\thesistitle}
		
		\vspace{0.5em}
		
		\thesisauthor
		
		\vspace{0.5em}
		
		A thesis submitted to the University of Manchester\\
		for the degree of Doctor of Philosophy, 2016
	\end{center}
	
	\vfill
	
	% The lay abstract
	
	SpiNNaker is a super computer designed to simulate neural networks like those
	found in the brain. Unlike biological experiments, computer simulations make
	it easy for neuroscientists to see what is going on in a neural network and
	easily test new theories. Though it is extremely unlikely that SpiNNaker will
	ever `think', it may give researchers a better understanding of how different
	parts of the brain work, and what makes them go wrong.
	
	Like most super computers, SpiNNaker is made up of many smaller computers
	which all work together on the same problem. When completed SpiNNaker, will
	contain over one million computer processors, each responsible for simulating
	several hundred neurons at once.
	
	This thesis makes three contributions towards tackling the challenge of
	building a full-sized SpiNNaker machine. Firstly I devised a new way of
	organising the many computers that make up SpiNNaker such that only short
	cables are required to connect everything together, even in big machines.
	These techniques make SpiNNaker cheaper and easier to build.  Secondly, I
	developed a method that allows the individual computers in SpiNNaker to
	communicate reliably, even if some of the connections between them break --
	an unavoidable situation in super computers. Finally, I adapted a technique
	normally used to design computer chips to decide how to assign neurons to
	SpiNNaker's processors. This method keeps neurons which are connected to each
	other close together inside SpiNNaker. By doing this, the signals exchanged
	between neurons have to travel less distance through SpiNNaker's computer
	network and do not get in each other's way. This makes it possible to
	simulate bigger and more complex neural networks.
	
	% Required to ensure single line spacing is used for this whole block
	\par%
}
