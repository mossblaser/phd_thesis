{
	\prefacesection{Lay abstract}
	
	% Single line spacing for the lay abstract page
	\setstretch{1.0}
	
	
	\vfill
	
	% Standard thesis information
	\begin{center}
		\textsc{\large\thesistitle}
		
		\vspace{0.5em}
		
		\thesisauthor
		
		\vspace{0.5em}
		
		A thesis submitted to the University of Manchester\\
		for the degree of Doctor of Philosophy, 2016.
	\end{center}
	
	\vfill
	
	% The lay abstract
	
	SpiNNaker is a super computer designed to simulate neural networks, such as
	those that make up the brain. Unlike biological experiments, simulations
	running on SpiNNaker allow neuroscientists to control and observe the
	behaviour of these networks while avoiding the need for animal experiments.
	Though it is extremely unlikely that SpiNNaker will ever `think', it may lead
	to a better understanding of the functions and failure modes of the brain.
	
	Like most super computers, SpiNNaker is made up of many smaller computers
	interconnected by a network. When the largest planned machine is completed,
	SpiNNaker, will contain over one million computer processors -- each
	responsible for simulating several hundred neurons -- capable of simulating
	neural networks similar in scale to a cat's brain.
	
	This thesis makes three contributions towards the construction and operation
	of full-sized SpiNNaker machines. Firstly I devised a new way of organising
	the physical components which make up a large SpiNNaker machine so that only
	short cables are required to construct its network making them cheaper and
	easier to build. Secondly, I developed a method that makes it possible for
	computers in SpiNNaker's network to communicate reliably, even when some
	connections are faulty -- an unavoidable situation in practice.  Finally, I
	adapted a technique normally used to design computer chips to assign neural
	models to SpiNNaker's processors. This method ensures connected neurons are
	assigned to nearby computers inside SpiNNaker's network meaning that signals
	between neurons have to travel less distance and are not as likely to get in
	each other's way. This makes it possible to simulate larger and more complex
	neural networks.
	
	% Required to ensure single line spacing is used for this whole block
	\par%
}
