\documentclass[12pt,twoside,openright,a4paper]{report}

\newcommand{\thesistitle}{Building and operating large-scale SpiNNaker machines}
\newcommand{\thesisauthor}{Jonathan Heathcote}
\newcommand{\thesisyear}{2016}

%%%%%%%%%%%%%%%%%%%%%%%%%%%%%%%%%%%%%%%%%%%%%%%%%%%%%%%%%%%%%%%%%%%%%%%%%%%%%%%%
% Set up document options
%%%%%%%%%%%%%%%%%%%%%%%%%%%%%%%%%%%%%%%%%%%%%%%%%%%%%%%%%%%%%%%%%%%%%%%%%%%%%%%%

% Set up all packages
% This \input-able file should be included in the preamble to use and configure
% all LaTeX packages used.

%%%%%%%%%%%%%%%%%%%%%%%%%%%%%%%%%%%%%%%%%%%%%%%%%%%%%%%%%%%%%%%%%%%%%%%%%%%%%%%%
% Used packages
%%%%%%%%%%%%%%%%%%%%%%%%%%%%%%%%%%%%%%%%%%%%%%%%%%%%%%%%%%%%%%%%%%%%%%%%%%%%%%%%

% More advanced mathematical typesetting
\usepackage{amsmath}

% Nice printing of URLs
\usepackage{url}

% Actually not tear-your-eyes-out-ugly tables
\usepackage{booktabs}

% Support for multiple-row table cells
\usepackage{multirow}

% Adjust linespacing for localised parts of the paper (e.g. abstract)
\usepackage{setspace}

% For the \ifthenelse macro
\usepackage{ifthen}

% For the \degree macro
\usepackage{gensymb}

% For subfigure support
\usepackage{caption}
\usepackage{subcaption}

% SI unit and number formatting
\usepackage{siunitx}

% For printing of 'code' etc.
\usepackage{verbatim}

% Used to draw labels with a white outline to make them stand-out in diagrams
\usepackage[outline]{contour}

% TikZ + PGF Plots for diagram/plot drawing
\usepackage{tikz}
\usepackage{tikz3d}
\usepackage{pgfplots}
\usetikzlibrary{ hexagon
               , calc
               , backgrounds
               , positioning
               , decorations.pathreplacing
               , decorations.markings
               , arrows
               , positioning
               , automata
               , shadows
               , fit
               , shapes
               , arrows
               , patterns
               , spy
               }
\usepgfplotslibrary{statistics}

%%%%%%%%%%%%%%%%%%%%%%%%%%%%%%%%%%%%%%%%%%%%%%%%%%%%%%%%%%%%%%%%%%%%%%%%%%%%%%%%
% Environment settings
%%%%%%%%%%%%%%%%%%%%%%%%%%%%%%%%%%%%%%%%%%%%%%%%%%%%%%%%%%%%%%%%%%%%%%%%%%%%%%%%

% Specifies the thickness of the contour added by the \contour macro.
\contourlength{1.5pt}

% Define a few layers for TikZ to allow easier layering
\pgfdeclarelayer{bg}
\pgfdeclarelayer{fg}
\pgfsetlayers{bg,main,fg}

%%%%%%%%%%%%%%%%%%%%%%%%%%%%%%%%%%%%%%%%%%%%%%%%%%%%%%%%%%%%%%%%%%%%%%%%%%%%%%%%
% Definitions
%%%%%%%%%%%%%%%%%%%%%%%%%%%%%%%%%%%%%%%%%%%%%%%%%%%%%%%%%%%%%%%%%%%%%%%%%%%%%%%%

% Build (and input) a LaTeX figure, recompiling only if necessary
\newcommand{\buildfig}[1]{%
	\input{|"python buildfig.py #1"}%
}

% Build a LaTeX script, recompiling if necessary. Does not include the compiled
% document. This may be used, e.g., for running analysis scripts on raw data.
\newcommand{\builddata}[1]{%
	\input{|"python buildfig.py --no-includegraphics #1"}%
}

% A colour to use when drawing the back of a sheet of paper
\colorlet{paperback}{black!10!white}


% 1.5 linespacing (as required by university)
\renewcommand{\baselinestretch}{1.5}

% Used in place of \chapter for preface sections. Prevents numbering but
% includes the chapter in the ToC
\newcommand{\prefacesection}[1]{
	\chapter*{#1}
	\addcontentsline{toc}{chapter}{#1}
}

% Reverses the usual LaTeX margins, placing the wider margin on the left on
% odd-numbered pages. When printed double-sided with the title first, the
% larger margin will be on the binding side. Though this goes against
% typographical conventions, it is what the University requires.
%
% When professionally printing and binding a book, extra-wide paper is used and
% an extra space is left to allow for binding. Since theses usually aren't
% professionally printed and bound (but rather done by a hack-job at a
% university print shop), normal A4 paper is used and the extra space to allow
% for binding must be included as part of the margins.
\let\tmp\oddsidemargin
\let\oddsidemargin\evensidemargin
\let\evensidemargin\tmp
\reversemarginpar


%%%%%%%%%%%%%%%%%%%%%%%%%%%%%%%%%%%%%%%%%%%%%%%%%%%%%%%%%%%%%%%%%%%%%%%%%%%%%%%%
% Show version/draft status in header
%%%%%%%%%%%%%%%%%%%%%%%%%%%%%%%%%%%%%%%%%%%%%%%%%%%%%%%%%%%%%%%%%%%%%%%%%%%%%%%%

% The background package is used to add things to the header
\usepackage{background}

% Package defines extra nodes relative to the page to allow nice positioning of
% the notice.
\usepackage{tikzpagenodes}

% Get commit information
\makeatletter
\immediate\write18{git show -s --format="\@percentchar h" > thesis.hash}
\immediate\write18{git show -s --format="\@percentchar aD" | cut -f 1 -d + > thesis.date}
\immediate\write18{./git_change_summary.sh > thesis.gitchanges}
\makeatother

% Display at the bottom of the page (comment out to disable)
\backgroundsetup{pages=all,
                 angle=0, scale=1,
                 color=black, opacity=1, contents={
	\begin{tikzpicture}[overlay,remember picture,help lines]
		% Main commit/version
		\node (commit notice)
		      [ font=\footnotesize
		      , inner sep=4pt
		      , above=of current page text area.south |- current page.south
		      ]
		      {Draft.
		       Revision \texttt{\documentclass[12pt,twoside]{report}

\newcommand{\thesistitle}{Building and operating large-scale SpiNNaker machines}
\newcommand{\thesisauthor}{Jonathan Heathcote}
\newcommand{\thesisyear}{2016}

%%%%%%%%%%%%%%%%%%%%%%%%%%%%%%%%%%%%%%%%%%%%%%%%%%%%%%%%%%%%%%%%%%%%%%%%%%%%%%%%
% Used packages
%%%%%%%%%%%%%%%%%%%%%%%%%%%%%%%%%%%%%%%%%%%%%%%%%%%%%%%%%%%%%%%%%%%%%%%%%%%%%%%%

% Nice printing of URLs
\usepackage{url}

% Actually not tear-your-eyes-out-ugly tables
\usepackage{booktabs}

% Adjust linespacing for localised parts of the paper (e.g. abstract)
\usepackage{setspace}

% For the \ifthenelse macro
\usepackage{ifthen}

% For the \degree macro
\usepackage{gensymb}

% For subfigure support
\usepackage{caption}
\usepackage{subcaption}

% SI unit and number formatting
\usepackage{siunitx}

% Used to draw labels with a white outline to make them stand-out in diagrams
\usepackage[outline]{contour}

% TikZ + PGF Plots for diagram/plot drawing
\usepackage{tikz}
\usepackage{tikz3d}
\usepackage{pgfplots}
\usetikzlibrary{ hexagon
               , calc
               , backgrounds
               , positioning
               , decorations.pathreplacing
               , decorations.markings
               , arrows
               , positioning
               , automata
               , shadows
               , fit
               , shapes
               , arrows
               , patterns
               , spy
               }
\usepgfplotslibrary{statistics}

%%%%%%%%%%%%%%%%%%%%%%%%%%%%%%%%%%%%%%%%%%%%%%%%%%%%%%%%%%%%%%%%%%%%%%%%%%%%%%%%
% Environment settings
%%%%%%%%%%%%%%%%%%%%%%%%%%%%%%%%%%%%%%%%%%%%%%%%%%%%%%%%%%%%%%%%%%%%%%%%%%%%%%%%

% 1.5 linespacing (as required by university)
\renewcommand{\baselinestretch}{1.5}


% Specifies the thickness of the contour added by the \contour macro.
\contourlength{1.5pt}

% Define a few layers for TikZ to allow easier layering
\pgfdeclarelayer{bg}
\pgfdeclarelayer{fg}
\pgfsetlayers{bg,main,fg}

%%%%%%%%%%%%%%%%%%%%%%%%%%%%%%%%%%%%%%%%%%%%%%%%%%%%%%%%%%%%%%%%%%%%%%%%%%%%%%%%
% Definitions
%%%%%%%%%%%%%%%%%%%%%%%%%%%%%%%%%%%%%%%%%%%%%%%%%%%%%%%%%%%%%%%%%%%%%%%%%%%%%%%%

% Used in place of \chapter for preface sections. Prevents numbering but
% includes the chapter in the ToC
\newcommand{\prefacesection}[1]{
	\chapter*{#1}
	\addcontentsline{toc}{chapter}{#1}
}

% Adds 'discard if' and  'discard if not' options for \addplot to enable
% filtering of data. Taken from
% http://tex.stackexchange.com/questions/58548/is-it-possible-to-change-the-color-of-a-single-bar-when-the-bar-plot-is-based-on
\pgfplotsset{
    discard if/.style 2 args={
        x filter/.code={
            \edef\tempa{\thisrow{#1}}
            \edef\tempb{#2}
            \ifx\tempa\tempb
                \def\pgfmathresult{inf}
            \fi
        }
    },
    discard if not/.style 2 args={
        x filter/.code={
            \edef\tempa{\thisrow{#1}}
            \edef\tempb{#2}
            \ifx\tempa\tempb
            \else
                \def\pgfmathresult{inf}
            \fi
        }
    }
}

% Make PGFplots Treat "NA" (regardless of letter case) as "nan". From:
% http://tex.stackexchange.com/questions/110441/skip-specific-string-in-a-numeric-column-while-using-pgfplots
\makeatletter
\expandafter\def\csname pgffltA@N\endcsname{\pgfflt@readundef}
\expandafter\def\csname pgffltA@n\endcsname{\pgfflt@readundef}
\def\pgfflt@readundef #1{%
    \def\pgfflt@readnan@ok{1}%
    \if#1a\else\if#1A\else\def\pgfflt@readnan@ok{0}\fi\fi
    \if\pgfflt@readnan@ok1%
        \pgfmathfloat@a@S=3\relax%
        \pgfmathfloat@a@Mtok={0.0}%
        \pgfmathfloat@a@E=0%
        \expandafter\pgfflt@finish
    \else
        \def\pgfflt@readnan@{\pgfflt@error #1}%
        \expandafter\pgfflt@readnan@
    \fi
}
\makeatother

%%%%%%%%%%%%%%%%%%%%%%%%%%%%%%%%%%%%%%%%%%%%%%%%%%%%%%%%%%%%%%%%%%%%%%%%%%%%%%%%
% Document body
%%%%%%%%%%%%%%%%%%%%%%%%%%%%%%%%%%%%%%%%%%%%%%%%%%%%%%%%%%%%%%%%%%%%%%%%%%%%%%%%
\begin{document}
	
	% The title page
	\begin{titlepage}
	
	
	\begin{center}
		
		\vspace*{1.0in}
		
		{\LARGE\textbf{\thesistitle}}
		
		\vfill
		
		\textsc{A thesis submitted to the University of Manchester\\for the degree of Doctor
		of Philosophy\\in the Faculty of Science and Engineering.}
		
		\vfill
		
		\thesisyear
		
		\vfill
		
		\thesisauthor
		
		\vfill
		
		School of Computer Science
		
		\vfill
		
		\color{gray}{
			{\tiny{}Revision \texttt{\documentclass[12pt,twoside]{report}

\newcommand{\thesistitle}{Building and operating large-scale SpiNNaker machines}
\newcommand{\thesisauthor}{Jonathan Heathcote}
\newcommand{\thesisyear}{2016}

%%%%%%%%%%%%%%%%%%%%%%%%%%%%%%%%%%%%%%%%%%%%%%%%%%%%%%%%%%%%%%%%%%%%%%%%%%%%%%%%
% Used packages
%%%%%%%%%%%%%%%%%%%%%%%%%%%%%%%%%%%%%%%%%%%%%%%%%%%%%%%%%%%%%%%%%%%%%%%%%%%%%%%%

% Nice printing of URLs
\usepackage{url}

% Actually not tear-your-eyes-out-ugly tables
\usepackage{booktabs}

% Adjust linespacing for localised parts of the paper (e.g. abstract)
\usepackage{setspace}

% For the \ifthenelse macro
\usepackage{ifthen}

% For the \degree macro
\usepackage{gensymb}

% For subfigure support
\usepackage{caption}
\usepackage{subcaption}

% SI unit and number formatting
\usepackage{siunitx}

% Used to draw labels with a white outline to make them stand-out in diagrams
\usepackage[outline]{contour}

% TikZ + PGF Plots for diagram/plot drawing
\usepackage{tikz}
\usepackage{tikz3d}
\usepackage{pgfplots}
\usetikzlibrary{ hexagon
               , calc
               , backgrounds
               , positioning
               , decorations.pathreplacing
               , decorations.markings
               , arrows
               , positioning
               , automata
               , shadows
               , fit
               , shapes
               , arrows
               , patterns
               , spy
               }
\usepgfplotslibrary{statistics}

%%%%%%%%%%%%%%%%%%%%%%%%%%%%%%%%%%%%%%%%%%%%%%%%%%%%%%%%%%%%%%%%%%%%%%%%%%%%%%%%
% Environment settings
%%%%%%%%%%%%%%%%%%%%%%%%%%%%%%%%%%%%%%%%%%%%%%%%%%%%%%%%%%%%%%%%%%%%%%%%%%%%%%%%

% 1.5 linespacing (as required by university)
\renewcommand{\baselinestretch}{1.5}


% Specifies the thickness of the contour added by the \contour macro.
\contourlength{1.5pt}

% Define a few layers for TikZ to allow easier layering
\pgfdeclarelayer{bg}
\pgfdeclarelayer{fg}
\pgfsetlayers{bg,main,fg}

%%%%%%%%%%%%%%%%%%%%%%%%%%%%%%%%%%%%%%%%%%%%%%%%%%%%%%%%%%%%%%%%%%%%%%%%%%%%%%%%
% Definitions
%%%%%%%%%%%%%%%%%%%%%%%%%%%%%%%%%%%%%%%%%%%%%%%%%%%%%%%%%%%%%%%%%%%%%%%%%%%%%%%%

% Used in place of \chapter for preface sections. Prevents numbering but
% includes the chapter in the ToC
\newcommand{\prefacesection}[1]{
	\chapter*{#1}
	\addcontentsline{toc}{chapter}{#1}
}

% Adds 'discard if' and  'discard if not' options for \addplot to enable
% filtering of data. Taken from
% http://tex.stackexchange.com/questions/58548/is-it-possible-to-change-the-color-of-a-single-bar-when-the-bar-plot-is-based-on
\pgfplotsset{
    discard if/.style 2 args={
        x filter/.code={
            \edef\tempa{\thisrow{#1}}
            \edef\tempb{#2}
            \ifx\tempa\tempb
                \def\pgfmathresult{inf}
            \fi
        }
    },
    discard if not/.style 2 args={
        x filter/.code={
            \edef\tempa{\thisrow{#1}}
            \edef\tempb{#2}
            \ifx\tempa\tempb
            \else
                \def\pgfmathresult{inf}
            \fi
        }
    }
}

% Make PGFplots Treat "NA" (regardless of letter case) as "nan". From:
% http://tex.stackexchange.com/questions/110441/skip-specific-string-in-a-numeric-column-while-using-pgfplots
\makeatletter
\expandafter\def\csname pgffltA@N\endcsname{\pgfflt@readundef}
\expandafter\def\csname pgffltA@n\endcsname{\pgfflt@readundef}
\def\pgfflt@readundef #1{%
    \def\pgfflt@readnan@ok{1}%
    \if#1a\else\if#1A\else\def\pgfflt@readnan@ok{0}\fi\fi
    \if\pgfflt@readnan@ok1%
        \pgfmathfloat@a@S=3\relax%
        \pgfmathfloat@a@Mtok={0.0}%
        \pgfmathfloat@a@E=0%
        \expandafter\pgfflt@finish
    \else
        \def\pgfflt@readnan@{\pgfflt@error #1}%
        \expandafter\pgfflt@readnan@
    \fi
}
\makeatother

%%%%%%%%%%%%%%%%%%%%%%%%%%%%%%%%%%%%%%%%%%%%%%%%%%%%%%%%%%%%%%%%%%%%%%%%%%%%%%%%
% Document body
%%%%%%%%%%%%%%%%%%%%%%%%%%%%%%%%%%%%%%%%%%%%%%%%%%%%%%%%%%%%%%%%%%%%%%%%%%%%%%%%
\begin{document}
	
	% The title page
	\input{titlepage}
	
	% The table of contents which, per university regulations, is followed by a
	% total wordcount.
	\tableofcontents
	\vfill
	\noindent This thesis contains
		\immediate\write18{texcount -1 -sum -inc thesis.tex > thesis.wordcount}%
		\input{thesis.wordcount}words.
	
	\clearpage
	\listoffigures
	
	\clearpage
	\listoftables
	
	% Abstract
	\input{abstract}
	
	% Declaration of non-submission elsewhere
	\input{declaration}
	
	% University-prescribed copyright statement...
	\input{copyright}
	
	% Acknowledgements
	\input{acknowledgements}
	
	% Main body
	\input{introduction.tex}
	\input{background.tex}
	\input{building.tex}
	\input{shortestPaths.tex}
	\input{routing.tex}
	\input{placement.tex}
	\input{discussion.tex}
	\input{future.tex}
	\input{conclusions.tex}
	
	% Bibliography
	\bibliography{references}
	\bibliographystyle{alpha}
	
\end{document}
}\documentclass[12pt,twoside]{report}

\newcommand{\thesistitle}{Building and operating large-scale SpiNNaker machines}
\newcommand{\thesisauthor}{Jonathan Heathcote}
\newcommand{\thesisyear}{2016}

%%%%%%%%%%%%%%%%%%%%%%%%%%%%%%%%%%%%%%%%%%%%%%%%%%%%%%%%%%%%%%%%%%%%%%%%%%%%%%%%
% Used packages
%%%%%%%%%%%%%%%%%%%%%%%%%%%%%%%%%%%%%%%%%%%%%%%%%%%%%%%%%%%%%%%%%%%%%%%%%%%%%%%%

% Nice printing of URLs
\usepackage{url}

% Actually not tear-your-eyes-out-ugly tables
\usepackage{booktabs}

% Adjust linespacing for localised parts of the paper (e.g. abstract)
\usepackage{setspace}

% For the \ifthenelse macro
\usepackage{ifthen}

% For the \degree macro
\usepackage{gensymb}

% For subfigure support
\usepackage{caption}
\usepackage{subcaption}

% SI unit and number formatting
\usepackage{siunitx}

% Used to draw labels with a white outline to make them stand-out in diagrams
\usepackage[outline]{contour}

% TikZ + PGF Plots for diagram/plot drawing
\usepackage{tikz}
\usepackage{tikz3d}
\usepackage{pgfplots}
\usetikzlibrary{ hexagon
               , calc
               , backgrounds
               , positioning
               , decorations.pathreplacing
               , decorations.markings
               , arrows
               , positioning
               , automata
               , shadows
               , fit
               , shapes
               , arrows
               , patterns
               , spy
               }
\usepgfplotslibrary{statistics}

%%%%%%%%%%%%%%%%%%%%%%%%%%%%%%%%%%%%%%%%%%%%%%%%%%%%%%%%%%%%%%%%%%%%%%%%%%%%%%%%
% Environment settings
%%%%%%%%%%%%%%%%%%%%%%%%%%%%%%%%%%%%%%%%%%%%%%%%%%%%%%%%%%%%%%%%%%%%%%%%%%%%%%%%

% 1.5 linespacing (as required by university)
\renewcommand{\baselinestretch}{1.5}


% Specifies the thickness of the contour added by the \contour macro.
\contourlength{1.5pt}

% Define a few layers for TikZ to allow easier layering
\pgfdeclarelayer{bg}
\pgfdeclarelayer{fg}
\pgfsetlayers{bg,main,fg}

%%%%%%%%%%%%%%%%%%%%%%%%%%%%%%%%%%%%%%%%%%%%%%%%%%%%%%%%%%%%%%%%%%%%%%%%%%%%%%%%
% Definitions
%%%%%%%%%%%%%%%%%%%%%%%%%%%%%%%%%%%%%%%%%%%%%%%%%%%%%%%%%%%%%%%%%%%%%%%%%%%%%%%%

% Used in place of \chapter for preface sections. Prevents numbering but
% includes the chapter in the ToC
\newcommand{\prefacesection}[1]{
	\chapter*{#1}
	\addcontentsline{toc}{chapter}{#1}
}

% Adds 'discard if' and  'discard if not' options for \addplot to enable
% filtering of data. Taken from
% http://tex.stackexchange.com/questions/58548/is-it-possible-to-change-the-color-of-a-single-bar-when-the-bar-plot-is-based-on
\pgfplotsset{
    discard if/.style 2 args={
        x filter/.code={
            \edef\tempa{\thisrow{#1}}
            \edef\tempb{#2}
            \ifx\tempa\tempb
                \def\pgfmathresult{inf}
            \fi
        }
    },
    discard if not/.style 2 args={
        x filter/.code={
            \edef\tempa{\thisrow{#1}}
            \edef\tempb{#2}
            \ifx\tempa\tempb
            \else
                \def\pgfmathresult{inf}
            \fi
        }
    }
}

% Make PGFplots Treat "NA" (regardless of letter case) as "nan". From:
% http://tex.stackexchange.com/questions/110441/skip-specific-string-in-a-numeric-column-while-using-pgfplots
\makeatletter
\expandafter\def\csname pgffltA@N\endcsname{\pgfflt@readundef}
\expandafter\def\csname pgffltA@n\endcsname{\pgfflt@readundef}
\def\pgfflt@readundef #1{%
    \def\pgfflt@readnan@ok{1}%
    \if#1a\else\if#1A\else\def\pgfflt@readnan@ok{0}\fi\fi
    \if\pgfflt@readnan@ok1%
        \pgfmathfloat@a@S=3\relax%
        \pgfmathfloat@a@Mtok={0.0}%
        \pgfmathfloat@a@E=0%
        \expandafter\pgfflt@finish
    \else
        \def\pgfflt@readnan@{\pgfflt@error #1}%
        \expandafter\pgfflt@readnan@
    \fi
}
\makeatother

%%%%%%%%%%%%%%%%%%%%%%%%%%%%%%%%%%%%%%%%%%%%%%%%%%%%%%%%%%%%%%%%%%%%%%%%%%%%%%%%
% Document body
%%%%%%%%%%%%%%%%%%%%%%%%%%%%%%%%%%%%%%%%%%%%%%%%%%%%%%%%%%%%%%%%%%%%%%%%%%%%%%%%
\begin{document}
	
	% The title page
	\input{titlepage}
	
	% The table of contents which, per university regulations, is followed by a
	% total wordcount.
	\tableofcontents
	\vfill
	\noindent This thesis contains
		\immediate\write18{texcount -1 -sum -inc thesis.tex > thesis.wordcount}%
		\input{thesis.wordcount}words.
	
	\clearpage
	\listoffigures
	
	\clearpage
	\listoftables
	
	% Abstract
	\input{abstract}
	
	% Declaration of non-submission elsewhere
	\input{declaration}
	
	% University-prescribed copyright statement...
	\input{copyright}
	
	% Acknowledgements
	\input{acknowledgements}
	
	% Main body
	\input{introduction.tex}
	\input{background.tex}
	\input{building.tex}
	\input{shortestPaths.tex}
	\input{routing.tex}
	\input{placement.tex}
	\input{discussion.tex}
	\input{future.tex}
	\input{conclusions.tex}
	
	% Bibliography
	\bibliography{references}
	\bibliographystyle{alpha}
	
\end{document}
}
		}
		
	\end{center}
	
\end{titlepage}

	
	% The table of contents which, per university regulations, is followed by a
	% total wordcount.
	\tableofcontents
	\vfill
	\noindent This thesis contains
		\immediate\write18{texcount -1 -sum -inc thesis.tex > thesis.wordcount}%
		\documentclass[12pt,twoside]{report}

\newcommand{\thesistitle}{Building and operating large-scale SpiNNaker machines}
\newcommand{\thesisauthor}{Jonathan Heathcote}
\newcommand{\thesisyear}{2016}

%%%%%%%%%%%%%%%%%%%%%%%%%%%%%%%%%%%%%%%%%%%%%%%%%%%%%%%%%%%%%%%%%%%%%%%%%%%%%%%%
% Used packages
%%%%%%%%%%%%%%%%%%%%%%%%%%%%%%%%%%%%%%%%%%%%%%%%%%%%%%%%%%%%%%%%%%%%%%%%%%%%%%%%

% Nice printing of URLs
\usepackage{url}

% Actually not tear-your-eyes-out-ugly tables
\usepackage{booktabs}

% Adjust linespacing for localised parts of the paper (e.g. abstract)
\usepackage{setspace}

% For the \ifthenelse macro
\usepackage{ifthen}

% For the \degree macro
\usepackage{gensymb}

% For subfigure support
\usepackage{caption}
\usepackage{subcaption}

% SI unit and number formatting
\usepackage{siunitx}

% Used to draw labels with a white outline to make them stand-out in diagrams
\usepackage[outline]{contour}

% TikZ + PGF Plots for diagram/plot drawing
\usepackage{tikz}
\usepackage{tikz3d}
\usepackage{pgfplots}
\usetikzlibrary{ hexagon
               , calc
               , backgrounds
               , positioning
               , decorations.pathreplacing
               , decorations.markings
               , arrows
               , positioning
               , automata
               , shadows
               , fit
               , shapes
               , arrows
               , patterns
               , spy
               }
\usepgfplotslibrary{statistics}

%%%%%%%%%%%%%%%%%%%%%%%%%%%%%%%%%%%%%%%%%%%%%%%%%%%%%%%%%%%%%%%%%%%%%%%%%%%%%%%%
% Environment settings
%%%%%%%%%%%%%%%%%%%%%%%%%%%%%%%%%%%%%%%%%%%%%%%%%%%%%%%%%%%%%%%%%%%%%%%%%%%%%%%%

% 1.5 linespacing (as required by university)
\renewcommand{\baselinestretch}{1.5}


% Specifies the thickness of the contour added by the \contour macro.
\contourlength{1.5pt}

% Define a few layers for TikZ to allow easier layering
\pgfdeclarelayer{bg}
\pgfdeclarelayer{fg}
\pgfsetlayers{bg,main,fg}

%%%%%%%%%%%%%%%%%%%%%%%%%%%%%%%%%%%%%%%%%%%%%%%%%%%%%%%%%%%%%%%%%%%%%%%%%%%%%%%%
% Definitions
%%%%%%%%%%%%%%%%%%%%%%%%%%%%%%%%%%%%%%%%%%%%%%%%%%%%%%%%%%%%%%%%%%%%%%%%%%%%%%%%

% Used in place of \chapter for preface sections. Prevents numbering but
% includes the chapter in the ToC
\newcommand{\prefacesection}[1]{
	\chapter*{#1}
	\addcontentsline{toc}{chapter}{#1}
}

% Adds 'discard if' and  'discard if not' options for \addplot to enable
% filtering of data. Taken from
% http://tex.stackexchange.com/questions/58548/is-it-possible-to-change-the-color-of-a-single-bar-when-the-bar-plot-is-based-on
\pgfplotsset{
    discard if/.style 2 args={
        x filter/.code={
            \edef\tempa{\thisrow{#1}}
            \edef\tempb{#2}
            \ifx\tempa\tempb
                \def\pgfmathresult{inf}
            \fi
        }
    },
    discard if not/.style 2 args={
        x filter/.code={
            \edef\tempa{\thisrow{#1}}
            \edef\tempb{#2}
            \ifx\tempa\tempb
            \else
                \def\pgfmathresult{inf}
            \fi
        }
    }
}

% Make PGFplots Treat "NA" (regardless of letter case) as "nan". From:
% http://tex.stackexchange.com/questions/110441/skip-specific-string-in-a-numeric-column-while-using-pgfplots
\makeatletter
\expandafter\def\csname pgffltA@N\endcsname{\pgfflt@readundef}
\expandafter\def\csname pgffltA@n\endcsname{\pgfflt@readundef}
\def\pgfflt@readundef #1{%
    \def\pgfflt@readnan@ok{1}%
    \if#1a\else\if#1A\else\def\pgfflt@readnan@ok{0}\fi\fi
    \if\pgfflt@readnan@ok1%
        \pgfmathfloat@a@S=3\relax%
        \pgfmathfloat@a@Mtok={0.0}%
        \pgfmathfloat@a@E=0%
        \expandafter\pgfflt@finish
    \else
        \def\pgfflt@readnan@{\pgfflt@error #1}%
        \expandafter\pgfflt@readnan@
    \fi
}
\makeatother

%%%%%%%%%%%%%%%%%%%%%%%%%%%%%%%%%%%%%%%%%%%%%%%%%%%%%%%%%%%%%%%%%%%%%%%%%%%%%%%%
% Document body
%%%%%%%%%%%%%%%%%%%%%%%%%%%%%%%%%%%%%%%%%%%%%%%%%%%%%%%%%%%%%%%%%%%%%%%%%%%%%%%%
\begin{document}
	
	% The title page
	\begin{titlepage}
	
	
	\begin{center}
		
		\vspace*{1.0in}
		
		{\LARGE\textbf{\thesistitle}}
		
		\vfill
		
		\textsc{A thesis submitted to the University of Manchester\\for the degree of Doctor
		of Philosophy\\in the Faculty of Science and Engineering.}
		
		\vfill
		
		\thesisyear
		
		\vfill
		
		\thesisauthor
		
		\vfill
		
		School of Computer Science
		
		\vfill
		
		\color{gray}{
			{\tiny{}Revision \texttt{\input{thesis.fullhash}}\input{thesis.date}}
		}
		
	\end{center}
	
\end{titlepage}

	
	% The table of contents which, per university regulations, is followed by a
	% total wordcount.
	\tableofcontents
	\vfill
	\noindent This thesis contains
		\immediate\write18{texcount -1 -sum -inc thesis.tex > thesis.wordcount}%
		\documentclass[12pt,twoside]{report}

\newcommand{\thesistitle}{Building and operating large-scale SpiNNaker machines}
\newcommand{\thesisauthor}{Jonathan Heathcote}
\newcommand{\thesisyear}{2016}

%%%%%%%%%%%%%%%%%%%%%%%%%%%%%%%%%%%%%%%%%%%%%%%%%%%%%%%%%%%%%%%%%%%%%%%%%%%%%%%%
% Used packages
%%%%%%%%%%%%%%%%%%%%%%%%%%%%%%%%%%%%%%%%%%%%%%%%%%%%%%%%%%%%%%%%%%%%%%%%%%%%%%%%

% Nice printing of URLs
\usepackage{url}

% Actually not tear-your-eyes-out-ugly tables
\usepackage{booktabs}

% Adjust linespacing for localised parts of the paper (e.g. abstract)
\usepackage{setspace}

% For the \ifthenelse macro
\usepackage{ifthen}

% For the \degree macro
\usepackage{gensymb}

% For subfigure support
\usepackage{caption}
\usepackage{subcaption}

% SI unit and number formatting
\usepackage{siunitx}

% Used to draw labels with a white outline to make them stand-out in diagrams
\usepackage[outline]{contour}

% TikZ + PGF Plots for diagram/plot drawing
\usepackage{tikz}
\usepackage{tikz3d}
\usepackage{pgfplots}
\usetikzlibrary{ hexagon
               , calc
               , backgrounds
               , positioning
               , decorations.pathreplacing
               , decorations.markings
               , arrows
               , positioning
               , automata
               , shadows
               , fit
               , shapes
               , arrows
               , patterns
               , spy
               }
\usepgfplotslibrary{statistics}

%%%%%%%%%%%%%%%%%%%%%%%%%%%%%%%%%%%%%%%%%%%%%%%%%%%%%%%%%%%%%%%%%%%%%%%%%%%%%%%%
% Environment settings
%%%%%%%%%%%%%%%%%%%%%%%%%%%%%%%%%%%%%%%%%%%%%%%%%%%%%%%%%%%%%%%%%%%%%%%%%%%%%%%%

% 1.5 linespacing (as required by university)
\renewcommand{\baselinestretch}{1.5}


% Specifies the thickness of the contour added by the \contour macro.
\contourlength{1.5pt}

% Define a few layers for TikZ to allow easier layering
\pgfdeclarelayer{bg}
\pgfdeclarelayer{fg}
\pgfsetlayers{bg,main,fg}

%%%%%%%%%%%%%%%%%%%%%%%%%%%%%%%%%%%%%%%%%%%%%%%%%%%%%%%%%%%%%%%%%%%%%%%%%%%%%%%%
% Definitions
%%%%%%%%%%%%%%%%%%%%%%%%%%%%%%%%%%%%%%%%%%%%%%%%%%%%%%%%%%%%%%%%%%%%%%%%%%%%%%%%

% Used in place of \chapter for preface sections. Prevents numbering but
% includes the chapter in the ToC
\newcommand{\prefacesection}[1]{
	\chapter*{#1}
	\addcontentsline{toc}{chapter}{#1}
}

% Adds 'discard if' and  'discard if not' options for \addplot to enable
% filtering of data. Taken from
% http://tex.stackexchange.com/questions/58548/is-it-possible-to-change-the-color-of-a-single-bar-when-the-bar-plot-is-based-on
\pgfplotsset{
    discard if/.style 2 args={
        x filter/.code={
            \edef\tempa{\thisrow{#1}}
            \edef\tempb{#2}
            \ifx\tempa\tempb
                \def\pgfmathresult{inf}
            \fi
        }
    },
    discard if not/.style 2 args={
        x filter/.code={
            \edef\tempa{\thisrow{#1}}
            \edef\tempb{#2}
            \ifx\tempa\tempb
            \else
                \def\pgfmathresult{inf}
            \fi
        }
    }
}

% Make PGFplots Treat "NA" (regardless of letter case) as "nan". From:
% http://tex.stackexchange.com/questions/110441/skip-specific-string-in-a-numeric-column-while-using-pgfplots
\makeatletter
\expandafter\def\csname pgffltA@N\endcsname{\pgfflt@readundef}
\expandafter\def\csname pgffltA@n\endcsname{\pgfflt@readundef}
\def\pgfflt@readundef #1{%
    \def\pgfflt@readnan@ok{1}%
    \if#1a\else\if#1A\else\def\pgfflt@readnan@ok{0}\fi\fi
    \if\pgfflt@readnan@ok1%
        \pgfmathfloat@a@S=3\relax%
        \pgfmathfloat@a@Mtok={0.0}%
        \pgfmathfloat@a@E=0%
        \expandafter\pgfflt@finish
    \else
        \def\pgfflt@readnan@{\pgfflt@error #1}%
        \expandafter\pgfflt@readnan@
    \fi
}
\makeatother

%%%%%%%%%%%%%%%%%%%%%%%%%%%%%%%%%%%%%%%%%%%%%%%%%%%%%%%%%%%%%%%%%%%%%%%%%%%%%%%%
% Document body
%%%%%%%%%%%%%%%%%%%%%%%%%%%%%%%%%%%%%%%%%%%%%%%%%%%%%%%%%%%%%%%%%%%%%%%%%%%%%%%%
\begin{document}
	
	% The title page
	\input{titlepage}
	
	% The table of contents which, per university regulations, is followed by a
	% total wordcount.
	\tableofcontents
	\vfill
	\noindent This thesis contains
		\immediate\write18{texcount -1 -sum -inc thesis.tex > thesis.wordcount}%
		\input{thesis.wordcount}words.
	
	\clearpage
	\listoffigures
	
	\clearpage
	\listoftables
	
	% Abstract
	\input{abstract}
	
	% Declaration of non-submission elsewhere
	\input{declaration}
	
	% University-prescribed copyright statement...
	\input{copyright}
	
	% Acknowledgements
	\input{acknowledgements}
	
	% Main body
	\input{introduction.tex}
	\input{background.tex}
	\input{building.tex}
	\input{shortestPaths.tex}
	\input{routing.tex}
	\input{placement.tex}
	\input{discussion.tex}
	\input{future.tex}
	\input{conclusions.tex}
	
	% Bibliography
	\bibliography{references}
	\bibliographystyle{alpha}
	
\end{document}
words.
	
	\clearpage
	\listoffigures
	
	\clearpage
	\listoftables
	
	% Abstract
	{
	\prefacesection{Abstract}
	
	% Single line spacing for the abstract page
	\setstretch{1.0}
	
	
	\vfill
	
	% Standard thesis information
	\begin{center}
		\textsc{\large\thesistitle}
		
		\vspace{0.5em}
		
		\thesisauthor
		
		\vspace{0.5em}
		
		A thesis submitted to the University of Manchester\\
		for the degree of Doctor of Philosophy, 2016
	\end{center}
	
	\vfill
	
	% The abstract
	
	SpiNNaker is an unconventional super computer architecture designed to
	simulate up to one billion biologically realistic neurons in real-time. To
	achieve this goal, SpiNNaker employs a novel network architecture which poses
	a number of practical problems in scaling up from desktop prototypes to
	machine room filling installations.
	
	SpiNNaker's hexagonal torus network topology has received mostly theoretical
	treatment in the literature. This thesis tackles some of the challenges
	encountered when building `real-world' systems.  Firstly, a scheme is devised
	for physically laying out hexagonal torus topologies in machine rooms which
	avoids long cables; this is demonstrated on a half-million core SpiNNaker
	prototype.  Secondly, to improve the performance of existing routing
	algorithms, a more efficient process is proposed for finding (logically)
	short paths through hexagonal torus topologies. This is complemented by a
	formula which provides routing algorithms greater flexibility when finding
	paths, potentially resulting in a more balanced network utilisation.
	
	The scale of SpiNNaker's network and the models intended for it also present
	their own challenges. Placement and routing algorithms are developed which
	assign processes to nodes and generate paths through SpiNNaker's network.
	These algorithms reduce congestion and tolerate network faults. The proposed
	placement algorithm is inspired by techniques used in chip design and is
	shown to enable larger applications to run on SpiNNaker -- with good
	performance -- than the previous state-of-the-art. Likewise the routing
	algorithm developed is able to tolerate network faults, inevitably present in
	large scale systems, with little performance overhead.
	
	
	% Required to ensure single line spacing is used for this whole block
	\par%
}

	
	% Declaration of non-submission elsewhere
	\prefacesection{Declaration}

% Single line spacing for the declaration
{
	\setstretch{1.0}
	No portion of the work referred to in this thesis has been submitted in support
	of an application for another degree or qualification of this or any other
	university or other institute of learning.
	
	\par%
}


	
	% University-prescribed copyright statement...
	\input{copyright}
	
	% Acknowledgements
	{
	\prefacesection{Acknowledgements}
	
	% Single line spacing
	\setstretch{1.0}
	
	It is often said that it is not \emph{what} you know but \emph{who} you know.
	Throughout the course of my PhD I've been exceptionally lucky to have been
	helped along by a great number of people.
	
	Both my supervisor, Jim Garside, and co-supervisor, Steve Furber, have each
	spent countless hours patiently discussing and describing all manner of
	things with me while giving me great freedom in my project. Jim's office door
	has always been open to my unexpected interruptions be it about work, writing
	or walking.  Likewise, Steve has always managed to find time for both
	technical and frivolous endeavours alike. I'm also hugely grateful to Luis
	Plana who has been a rich source of sage advice, insightful questions
	patiently suffered many a foolish question.
	
	Various parts of the work in this thesis (and numerous side projects) would
	not have been possible if not for the multitude of discussions,
	collaborations and even sheer physical hard work of Steve Temple, Javier
	Navaridas, Simon Davidson and Dave Clark. I'm also indebted to Andrew Mundy
	and Jamie Knight, both of whom have donated so much time and effort towards
	verifying and using software implementations of the ideas in this thesis.
	
	The injection of lunchtime silliness by Andrew and Jamie, along with Amanieu
	d'Antras and Andrew Webb and the other CDT members has always brightened my
	day. So to has the friendly and stimulating environment of the School of
	Computer Science and its many staff and students. Of course, I am also very
	grateful for the funding the school has provided for my research.
	
	I cannot thank my wonderful wife, Ann-Marie, enough for being by my side. She
	has given me so much kindness, love and patience and endured a lifetime's
	quota of conversations about hexagons. Finally, thanks too to rest of my
	family, especially my parents, who are to blame for starting me down this
	path and co-suffering drafts and endless rants about this document.
	
	% Required to ensure single line spacing is used for this whole block
	\par%
}

	
	% Main body
	\chapter{Introduction}

\label{sec:introduction}

%Problem area
%
%* Network construction and exploitation
%  * Cabling: Build it cheaply in terms of tech cost and install cost
%  * Routing: Get around it cheaply and reliably
%  * Placement: Use it efficiently

The Spiking Neural Network Architecture (SpiNNaker) is a novel super computer
architecture designed to simulate biologically realistic models of brains in
real time \cite{furber07}. Though neurons, the building blocks of the brain,
are relatively well understood, their complex interactions remain mysterious.
Just as understanding the workings of a transistor is insufficient to
understand a modern microprocessor, neuroscientists believe that understanding
the neurons in isolation cannot explain the brain and that understanding their
connectivity is key \cite{eliasmith13,eliasmith14}. Experiments on real brains,
however, are fraught with difficulty. Variations between individuals can be
significant and it is only possible to record tens or hundreds of the trillions
of signals in the brain, and even then only with limited control over which
signals are recorded. Computer simulations of models of large neural networks,
however, enable researchers to develop repeatable experiments and gain complete
visibility of any signal and any neuron. Models such as SPAUN
\cite{eliasmith12}, built from millions of simulated neurons, have shown great
promise in expanding our understanding of higher level brain functions such as
memory and simple problem solving.  Unfortunately these neural models are
expensive to simulate, requiring hours of compute time to simulate each second
of neural activity. As well as being inconvenient, this precludes the use of
robotics to immerse these models in real world environments and also limits
studies of long-term behaviours such as learning.

SpiNNaker is designed to enable the real time simulation of models containing
up to one billion neurons -- approximately \SI{1}{\percent} of a human brain or
ten mouse brains \cite{furber06}. To achieve this goal, the largest planned
SpiNNaker machine will contain over one million low-powered computer processors
interconnected by a bespoke network architecture.

SpiNNaker's large processor count matches the current trend in super computers
where processor counts are growing exponentially \cite{meuer16j}, mimicking the
growth of the number of components in the processors themselves predicted by
Gordon Moore's famous `law' \cite{moore75}. As a result of this growth, the
interconnection networks which enable these processors to work together have
grown in importance \cite{dally04}.  Network designers must carefully balance
performance against practicality and financial cost.  SpiNNaker's network is no
exception to this rule and, as the systems scale up from desktop prototypes to
machine-room scale installations, the reality of building and exploiting these
machines presents an array of challenges.

As in all super computers, SpiNNaker's network interconnects its processors in
a particular network topology which defines how different processors may
communicate with each other. Unlike the tree and $N$-dimensional torus
topologies found in contemporary super computers \cite{dally04}, SpiNNaker
employs a `hexagonal torus topology'. In this topology, nodes in SpiNNaker's
network fit together in a honeycomb-like pattern where messages may `hop' from
node to node to reach their destination. As we will see in
chapter~\ref{sec:background}, the hexagonal torus topology, in theory, sits at
a `sweet spot' in terms of network performance and practicality. As the first
known large-scale installation of the hexagonal torus topology, however, there
remain a number of practical challenges for large spinnaker machines arising
from this choice.

As super computer networks have grown in scale to millions of processors the
task of dealing with previously rare faults has grown.  Though fault rates in
networks remain consistently low, architectures such as SpiNNaker may have
hundreds of thousands of links meaning even fault rates of a fraction of a
percent will impact tens or hundreds of links. To enable reliable operation,
networks must be able to adapt the routes taken by messages through the network
to avoid faulty links and nodes. The techniques employed are often closely tied
to a particular network architecture and consequently SpiNNaker's novel network
architecture demands its own approach.

Another challenge introduced by the growing scale of super computers is making
\emph{efficient} use of network resources. Communicating processes should be
located on logically `nearby' nodes to reduce network load. The neural models
for which SpiNNaker is designed are often described abstractly, rather than
geometrically, using modelling languages such as PyNN~\cite{davison08} and
Nengo~\cite{eliasmith04}.  Because of this, the communication requirements of
simulations can be highly irregular making an efficient placement of processes
onto processors in the machine non-trivial.

%Contributions
%
%* Cabling scheme for hexagonal toruses without long cables
%* Efficient installation technique for dense systems
%* Exhaustive and efficient route calculation in hex toruses
%* Fault tolerant routing scheme exploiting SpiNNaker's odd router
%* Placement based on SA a: works very well and b: suggests circuit placement is
%  a good source of inspiration.

This thesis addresses the practical challenges of scaling up the SpiNNaker
architecture in a real-world setting summarised by these research questions:

\begin{enumerate}
	
	\item Can the hexagonal torus topology be deployed and used in real, large
	scale systems?
	
	\item Does SpiNNaker's router architecture help, or hinder fault tolerance?
	
	\item How can the parts of a neural simulation be placed onto a large
	hexagonal torus topology to reduce network load?
	
\end{enumerate}

%Structure
%
%* Chapter 2: Background: detailed dive into what's in SpiNNaker, why its
%  really so unusual. Also looks at what applications run on SpiNNaker and how
%  they work.
%* Chapter 3: How to build a really big SpiNNaker machine.
%* Chapter 4: How to find your way around that machine.
%* Chapter 5: How to find your way around that machine even when its broken.
%* Chapter 6: Now you can walk, time to run.
%* Chapter 7: Wrapping up.
%* Appendices: Hard-to-come-by theoretical and practical details useful if
%  you're about to continue where this research left off but be useful but
%  otherwise hard to come by, especially in one place.

Chapter~\ref{sec:background} introduces the SpiNNaker architecture and, in
particular, describes its hexagonal torus topology and network architecture.

In chapter~\ref{sec:building}, I develop a cabling scheme for large hexagonal
torus topologies which enables arbitrarily large networks to be constructed
using only short, inexpensive cables. This theoretical work is then evaluated
through the construction of a range of prototype SpiNNaker systems. The largest
of these prototypes contains over half a million processor cores and spans
several machine room cabinets. In addition, I propose the use of built-in
diagnostic facilities to assist technicians performing network installation and
maintenance. This technique is found to greatly reduce the effort required and
the number of mistakes made.

In chapters~\ref{sec:shortestPaths}~and~\ref{sec:routing} I develop new routing
techniques for SpiNNaker's network. Chapter~\ref{sec:shortestPaths} develops a
new approach to finding the shortest paths through hexagonal torus topologies,
an integral part of many routing algorithms. This newly proposed approach is
cheaper to compute than the state of the art and, unlike previous efforts, is
able to discover all valid short paths through the topology. This theoretical
advance brings hexagonal torus topologies in line with conventional topologies
by providing routing algorithms with complete information about the paths
available to them. In chapter \ref{sec:routing} I propose a fault tolerant
routing algorithm for SpiNNaker which is able to avoid arbitrary static fault
patterns with minimal performance overhead. A key finding of this chapter is
that the flexibility afforded to fault tolerant routing algorithms by
SpiNNaker's unconventional router architecture is what facilities the low
overheads reported in this chapter.

Finally, in chapter~\ref{sec:placement}, I explore the problem of application
placement in SpiNNaker's network. As in other networks and applications, neural
simulations should be arranged such that communication occurs primarily between
processors close together in the network to control network load. Due to the
irregular connectivity and large scale of the neural models expected to run on
SpiNNaker, an automated approach is necessary. I develop a novel placement
algorithm based on algorithms used for circuit layout in computer chips. My
algorithm is found to allow some larger neural models to run on SpiNNaker for
the first time while enabling other applications to run at greater speeds. In
addition, synthetic benchmarks containing over one million processes indicate
that this algorithm should handle the anticipated demands of the neural models
expected to run on large-scale SpiNNaker installations.

	\chapter{The SpiNNaker Architecture}
	
	\label{sec:background}
	
	SpiNNaker is a massively parallel computer architecture designed to simulate
	biologically realistic neural models \cite{furber07}. In this chapter we will
	explore this unconventional architecture in detail, starting with its purpose
	before focusing on its most unconventional feature: its network.
	
	% * Purpose
	%   * Spiking neural simulations
	%     * Neural modelling: PyNN, Nengo...
	%     * Parallelisation + communication
	
	\section{Neural simulation}
		
		Human brains contain billions of neurons connected together by trillions of
		`synapses'. Neurons communicate by transmitting and receiving `spikes'
		through their synapses. Each spike is `valueless' in that a spike's only
		significant features are when it arrives and where it has come from.
		
		\begin{figure}
			\center
			\buildfig{figures/lif-neuron.tex}
			
			\caption{A Leaky Integrate-and-Fire (LIF) neuron.}
			\label{fig:lif-neuron}
		\end{figure}
		
		Though some detailed models of the electrochemical processes occurring
		inside neurons are computationally intensive, simplified models such as the
		Leaky Integrate-and-Fire (LIF) model can be implemented in just a handful
		of CPU instructions \cite{vainbrand11}. Figure~\ref{fig:lif-neuron}
		illustrates a simple LIF neuron in which incoming spikes cause charge to
		build up (integrated) which over time, leaks away. If an incoming spike
		causes the charge to rise above a certain threshold, the neuron `fires'
		producing an outgoing spike. Despite the simplicity of this model, large
		neural networks such as Spaun \cite{eliasmith12} -- built entirely from LIF
		neurons -- exhibit complex behaviours such as fine motor control and
		problem solving.
		
		The computational expense of large scale neural simulations does not arise
		from the cost of modelling neurons but instead from distributing spikes. In
		biology, neurons produce spikes at an average rate of \SI{10}{\hertz} and
		synapses connect each neuron's output to (order) \num{1000}~neurons
		\cite{navaridas09}. Consider an example neural model with $7\times10^7$
		neurons, approximately the number in a house mouse and
		$\nicefrac{1}{10}^\textrm{th}$ of the design target of SpiNNaker. This
		network might produce $7\times10^8$~spikes per second. Because each neuron
		connects to many others, this equates to $7\times10^{11}$ spikes being
		received per second. If each spike were transmitted as a UDP datagram
		containing a single \SI{32}{\bit} payload, the total network throughput
		required for this simulation would be \SI{179.2}{\tera\bit\per\second}. At
		the time of writing, this is more than double the bisection bandwidth (the
		theoretical worst-case throughput) of the world's most powerful super
		computer \cite{dongarra16}.
	
	\section{Network architecture}
		
		Architectures such as IBM's Blue Gene \cite{chiu11} and Cray's XK7
		\cite{ornl16} employ powerful compute nodes connected together using
		networks designed to transfer multi-kilobyte blocks of data between nodes.
		Since neural models have relatively light computational requirements and
		communications are based on small pieces of data (spikes), this type of
		architecture is poorly suited to the task.
		
		SpiNNaker's architectural target is to support realtime simulations of up
		to one billion neurons. Since neural models such as LIF are inexpensive to
		model and many neurons can be simulated independently in parallel,
		SpiNNaker employs many small, energy efficient ARM processors
		\cite{furber07}. To support the unusual communication requirements of
		neural simulations, a bespoke interconnection network is used which is the
		background to this thesis.
		
	%   * SpiNNaker chip
	%     * Cores
	%     * SDRAM
	%     * NoC
	%     * Router
		
		\begin{figure}
			\center
			%\includegraphics[width=19mm]{figures/spinnakerChip.jpg}
			\buildfig{figures/hex-chips.tex}
			
			\caption[SpiNNaker chips connected to their six neighbours.]%
			{SpiNNaker chips (actual size) connected to their six neighbours.}
			\label{fig:spinnakerChip}
		\end{figure}
		
		The fundamental building block of the SpiNNaker architecture is the
		SpiNNaker chip (figure \ref{fig:spinnakerChip}) \cite{furber13}. Each chip
		contains eighteen low power ARM 968 processor cores each capable of
		simulating between \num{200} and \num{2000} LIF neurons in real time
		\cite{mundy15}.  Each core has a total of \SI{96}{\kilo\byte} of private
		Tightly-Coupled Memory (TCM) and shares access to \SI{128}{\mega\byte} of
		on-chip SDRAM with other cores on the same chip. Finally, each chip
		contains a programmable router which routes network packets to and from the
		local cores and six neighbouring SpiNNaker chips. SpiNNaker machines are
		constructed by combining many SpiNNaker chips.
		
		\begin{figure}
			\center
			\buildfig{figures/spinnaker-packet.tex}
			
			\caption{SpiNNaker's \SI{40}{\bit} and \SI{72}{\bit} multicast packet
			format.}
			\label{fig:spinnaker-packet}
		\end{figure}
		
		Processor cores can communicate by sending and receiving network packets
		forwarded by routers through the network. Since SpiNNaker's network is
		designed to transmit neural spike events efficiently, individual network
		packets are small, either \SI{40}{\bit} or \SI{72}{\bit} compared with tens
		or hundreds of byte packets in typical network architectures.
		
		In a real-time simulation, the time at which a spike is produced is
		implicitly indicated by the time it is received -- since at biological
		timescales a computer network delivers packets `instantaneously'.
		Consequently, the only information which must be explicitly encoded is the
		identity of the neuron which produced the spike. In SpiNNaker, a spike may
		be encoded by using a single \SI{40}{\bit} `multicast packet' whose format
		is illustrated in figure~\ref{fig:spinnaker-packet}.  The \SI{8}{\bit}
		header is used by SpiNNaker's routers to determine the type of packet and
		the \SI{32}{\bit} `routing key' is used to identify the neuron which
		produced the packet. The routing key is also used by SpiNNaker's routers to
		determine how the packet should be directed through the network.
		
		The optional \SI{32}{\bit} payload is not used by conventional spiking
		neural simulations \cite{galluppi10} but has been exploited to enable more
		efficient simulation of a particular class of neural models \cite{mundy15}.
	
	\section{The SpiNNaker router}
		
		The SpiNNaker router employs an unconventional design which, despite its
		compact size and small energy requirements, implements a flexible multicast
		routing scheme. Unlike conventional routers which often employ hard-coded
		routing rules \cite[chapter~8]{dally04}, the SpiNNaker router uses a
		programmable `routing table' to determine how packets should be forwarded.
		In addition, to avoid deadlocks, SpiNNaker's router employs a simple,
		timeout-based mechanism which exploits the ability of neural networks to
		tolerate occasional missing packets. As we will see in chapter
		\ref{sec:routing}, this mechanism greatly simplifies the task of routing in
		SpiNNaker's network. In this section we'll look at these features in
		greater detail.
		
		\subsection{Routing tables}
		
			When a multicast packet arrives at a SpiNNaker router (either from a
			local core or a neighbouring chip), the router looks up the routing key
			in its routing table. This table consists of \num{1024} programmable
			table entries, each specifying a routing key bit pattern and mask to
			match and a set of routes.  When a multicast packet's key is matched by a
			routing entry the packet is forwarded along every route specified by that
			entry, potentially duplicating the packet. This `multicast' technique
			allows packets to be transmitted once but received in a number of places
			while making efficient use of the network \cite{navaridas12}.
			
			Though routing table entries are in finite supply (\num{1024} entries per
			router), it is still possible for many thousands of traffic flows to be
			routed through a single router. The bit pattern and mask in each routing
			entry matches against the 32~bits of a routing key as either
			`\texttt{1}', `\texttt{0}' or `\texttt{X}' (don't care).  This means that
			a single routing entry may, for example, be used to match all routing
			keys with a certain prefix. If a routing key is not matched by any entry
			in the routing table then the packet is `default routed' in a straight
			line. For example if a packet with an unmatched key is received from the
			chip to the left, the packet will be default routed to the chip on the
			right. By assigning routing keys such that neurons whose spikes are sent
			to similar destinations share a similar prefix, the number of routing
			entries required by a simulation is greatly reduced \cite{davies12}.
			
			\begin{figure}
				\center
				\buildfig{figures/routing-example.tex}
				
				\caption[Multicast routing example.]%
				{Multicast routing example with \SI{4}{\bit} routing keys. Each
				box represents a SpiNNaker chip whose router has been programmed with
				the routing entries shown. Grey lines mark connections between chips.}
				\label{fig:routing-example}
			\end{figure}
			
			Consider the simplified example in figure~\ref{fig:routing-example} in
			which a number of (\SI{4}{\bit}) routing table entries have been
			configured in the routers of a small SpiNNaker network. If a packet with
			the routing key \texttt{1011} is transmitted by a core in the chip
			labelled $(0, 0, 0)$, this will match the first routing table entry on
			that chip and will be routed to chip $(1, 0, 0)$. On chip $(1, 0, 0)$,
			the packet once again matches the first routing entry and is routed to
			chip $(1, 0, -1)$. On $(1, 0, -1)$, no match is made so the packet is
			default routed to $(1, 0, -2)$. On this chip, the packet matches a
			routing entry which routes the packet to core~7. In this example, default
			routing allows only three routing table entries to direct a packet
			through four chips.
			
			As a second example, if a packet with the routing key \texttt{0010} is
			transmitted by a core on chip $(0, 0, 0)$, this key will be matched by
			the second routing entry since \texttt{X}s in the table entry will match
			both \texttt{1}s and \texttt{0}s in the corresponding bits of the routing
			key. When the packet arrives at chip $(0, 0, -1)$ the matching routing
			entry forwards the packet to both $(0, 1, -1)$ and $(1, 0, -1)$
			simultaneously. The copy of the packet arriving at $(0, 1, -1)$ is routed
			to core~5 on that chip.  Meanwhile, the copy forwarded to $(1, 0, -1)$ is
			duplicated again with one copy being routed to core~11 and another being
			routed to chip $(1, 0, -2)$. Here the packet is finally delivered to
			core~6. In this example, the ability of the router to multicast
			(duplicate) packets as they pass through the network meant that sending
			one copy of the packet was sufficient to reach three destination cores.
			In addition, by using \texttt{X}s in the routing table entry, the same
			routing entries are sufficient to route packets with the keys
			\texttt{0000}, \texttt{0001}, \texttt{0010} and \texttt{0011}.
			
			In spite of these mechanisms, it is still possible for an application to
			run out of routing table entries. As we will see in
			chapter~\ref{sec:placement} by arranging applications appropriately
			within SpiNNaker's network, routing table usage can be reduced. In
			addition, other behaviours of SpiNNaker's router may be exploited to
			compress an applications routing tables further, however the techniques
			employed are beyond the scope of this thesis \cite{mundy16}.
		
		\subsection{Timeouts}
			
			SpiNNaker's router is built on a pipeline architecture. As shown in
			figure~\ref{fig:router-architecture}, the router is fed packets by an
			arbiter which serialises packets arriving from other chips and local
			cores. Every (\SI{100}{\mega\hertz}) clock cycle, the router pipeline
			accepts one packet from the arbiter and routes a packet to one or several
			output links. If any of the required output ports are busy then the
			packet is not forwarded to any output link and the pipeline stalls. Once
			a packet has been blocked for a programmable timeout, it is dropped
			(discarded) and routing continues as usual for next packet in the
			pipeline. Links become blocked while transmitting packets or waiting for
			the remote receiver to become ready. For example, a receiving processor
			core may be busy performing some computation or a receiving router may be
			blocked waiting for some of its outputs to become ready.
			
			\begin{figure}
				\center
				\buildfig{figures/router-architecture.tex}
				
				\caption{SpiNNaker router architecture}
				\label{fig:router-architecture}
			\end{figure}
			
			The timeout-based packet dropping mechanism is designed to defuse
			deadlocks in the network. For example, if two routers are trying to send
			each other a packet at the same time they may become deadlocked, each
			waiting for the other router to accept a packet before continuing.
			SpiNNaker's timeout mechanism breaks deadlocks by dropping packets which
			have been blocked for some time and therefore may be in a deadlock.  Once
			a packet has been dropped it is left to software to either tolerate the
			missing packet or trigger a retransmission. In neural simulations, as in
			biology, the loss of a single spike is unlikely to have a significant
			impact on the behaviour of a neural model and therefore these simulations
			are inherently tolerant of occasional dropped packets. During application
			loading and other system tasks, a higher level, software driven protocol
			based on acknowledgements and retransmissions is used to ensure
			guaranteed delivery.
			
			% TODO: MENTION TIMEOUT VALUE USED?
			% Router timeouts must be configured to be long enough that delays in
			% packet transmission, for example due to the time taken for packets to
			% traverse a link, do not trigger packet dropping. Conversely, the timeout
			% should be as short as possible to reduce the time the router is
			% blocked and maximise network throughput.
	
	\section{The hexagonal torus topology}
		
		Each SpiNNaker chip is a node in a `hexagonal torus topology' as
		illustrated in figure~\ref{fig:hexagonalTorusTopology}. Network packets
		sent by SpiNNaker's processor cores may `hop' through several nodes in the
		network to reach their intended destination. In each hop, a packet may
		advance one node along one of the three axes of the topology. For example,
		a packet sent by the node labelled $\alpha$ (in the bottom-left corner) to
		the node labelled $\beta$, might take the following sequence of hops:
		X$^+$, X$^+$, Z$^-$. Packets sent from $\alpha$ to $\gamma$ might take the
		route: X$^-$, X$^-$, Y$^+$, Y$^+$. The first hop of this route `wraps
		around' from the bottom-left node to the bottom-right node in a single hop.
		
		\begin{figure}
			\center
			\buildfig{figures/hexagonalTorusTopology.tex}
			
			\caption[A hexagonal torus topology.]%
			{A hexagonal torus topology. Each hexagon represents a node (i.e.
			a SpiNNaker chip). Touching nodes are directly connected. Nodes on edges
			$a$, $b$ and $c$ are also directly connected to the corresponding nodes
			on edges $a'$, $b'$ and $c'$, respectively. The three axes of the
			hexagonal torus topology, `X', `Y' and `Z' are also shown.}
			\label{fig:hexagonalTorusTopology}
		\end{figure}
		
		\begin{figure}
			\center
			\begin{subfigure}{0.39\linewidth}
				\center
				\includegraphics[width=\linewidth]{figures/torus-3d-flat.pdf}
				\caption{}
				\label{fig:torus-3d-flat}
			\end{subfigure}
			~~
			\begin{subfigure}{0.26\linewidth}
				\center
				\includegraphics[width=\linewidth]{figures/torus-3d-tube.pdf}
				\caption{}
				\label{fig:torus-3d-tube}
			\end{subfigure}
			~~
			\begin{subfigure}{0.23\linewidth}
				\center
				\includegraphics[width=\linewidth]{figures/torus-3d-torus.pdf}
				\caption{}
				\label{fig:torus-3d-torus}
			\end{subfigure}
			
			\caption{Visualisation of a hexagonal torus topology as a torus.}
			\label{fig:torus-3d}
		\end{figure}
		
		The wrap around connections in the topology are what give it the `torus'
		part of its name. Figure~\ref{fig:torus-3d-flat} shows a hexagonal torus
		topology drawn flat as in the previous figure. If the topology is rolled up
		into a tube such that the top and bottom nodes become directly adjacent, a
		tube is formed as in figure~\ref{fig:torus-3d-tube}. This tube can then be
		bent to bring together the nodes at the ends of the tube to form a torus as
		shown in figure~\ref{fig:torus-3d-torus}.
		
		A hexagonal torus topology is typically defined in terms of its width and
		height along the X and Y axes respectively. For example,
		figure~\ref{fig:hexagonalTorusTopology} shows a $10\times10$ hexagonal
		torus.  The nodes in a hexagonal torus topology are addressed using
		hexagonal coordinates of the form $(x, y, z)$ \cite{patel15}. The bottom
		left node (labelled $\alpha$ in the figure) has the coordinate $(0, 0, 0)$
		and other nodes are assigned coordinates according to the number of hops
		along each dimension from $(0, 0, 0)$, for example node $\beta$ has the
		coordinate $(2, 0, -1)$. Because the hexagonal torus topology's axes are
		non-orthogonal, it is possible to define several coordinates for the same
		location. For example $(3, 1, 0)$ and $(1, -1, -2)$ are also valid
		coordinates for node $\beta$. These dual coordinates emerge from the fact
		that adding $(1, 1, 1)$ to a coordinate produces an equivalent, but
		different, coordinate. This phenomenon is explained in detail in
		appendix~\ref{app:minimal-hex-coordinates} and related phenomena will be
		discussed in chapter~\ref{sec:shortestPaths}.
		
		The hexagonal torus topology was chosen over a more conventional network
		topology -- such as a 2D or 3D torus (sometimes known as a 2-ary $N$-cube
		or 3-ary $N$-cube respectively) \cite[chapters~3~and~5]{dally04} -- due to
		its balance of theoretical performance and practicality. The bisection
		bandwidth of a topology indicates the theoretical worst-case total
		throughput the network is able to sustain \cite[chapter~1]{dally04}.  In
		networks with homogeneous link throughput, bisection bandwidth is
		determined by the number of links cut by a balanced bisection of the
		network.  Figure~\ref{fig:bisection-bandwidth} illustrates the bisections
		of several torus topologies.
		
		\begin{figure}
			\center
			\begin{subfigure}[b]{0.3\linewidth}
				\center
				\buildfig{figures/bisection-bandwidth-2d.tex}
				
				\caption{2D Torus}
				\label{fig:bisection-bandwidth-2d}
			\end{subfigure}
			\begin{subfigure}[b]{0.3\linewidth}
				\center
				\buildfig{figures/bisection-bandwidth-hex.tex}
				
				\caption{Hexagonal Torus}
				\label{fig:bisection-bandwidth-hex}
			\end{subfigure}
			\begin{subfigure}[b]{0.3\linewidth}
				\center
				\buildfig{figures/bisection-bandwidth-3d.tex}
				
				\caption{3D Torus}
				\label{fig:bisection-bandwidth-3d}
			\end{subfigure}
			
			\caption[Bisections of torus topologies.]%
			{Bisections of torus topologies. Connections cut by the bisection
			are drawn as lines.}
			\label{fig:bisection-bandwidth}
		\end{figure}
		
		In a $N \times N$ 2D torus topology, the bisection bandwidth is $2N$~links
		and each node requires four links. The hexagonal torus topology requires
		six links per node but provides double bisection bandwidth ($4N$~links).
		The 3D torus topology also requires six links per node but by connecting
		the nodes differently achieves a bisection bandwidth of $8N$~links.  The 3D
		torus topology, however, comes at a price -- when immersed into the
		(approximately) 2D space provided by a large machine room or row of server
		cabinets, some connections require long cables. By contrast, the 2D and
		hexagonal torus topologies are both inherently two dimensional and
		consequently do not suffer from this effect. The hexagonal torus topology,
		therefore, shares the practicality of construction of a 2D torus while
		still gaining some of the performance of a 3D torus topology. In addition,
		because nodes in a hexagonal torus topology have a greater number of links,
		greater redundancy is available in the network to tolerate faults.
		
		Most torus topologies, including hexagonal, 2D and 3D toruses, have a
		related `mesh' topology. These mesh topologies maintain the same general
		connectivity structure as their torus topologies but omit wrap-around
		links. In practice, this saves a small number of links at the expense of
		halving the network's bisection bandwidth.  Because of their poorer
		performance, mesh networks are rarely used as the basis of a network
		architecture. Mesh networks, however, are occasionally formed when a
		network is partitioned into several smaller sub-networks to allow multiple
		users to share a system \cite{spalloc16}.
		
		\begin{figure}
			\center
			\begin{subfigure}[b]{0.45\linewidth}
				\center
				\buildfig{figures/hexagonal-torus.tex}
				\caption{Hexagonal torus}
				\label{fig:topo-compare-hexagonal-torus}
			\end{subfigure}
			\begin{subfigure}[b]{0.45\linewidth}
				\center
				\buildfig{figures/h-torus.tex}
				\caption{H-torus}
				\label{fig:topo-compare-h-torus}
			\end{subfigure}
			
			\caption[Hexagonal torus vs. H-torus topology.]%
			{Hexagonal torus vs. H-torus topology. Each numbered hexagon
			represents a node. The thick outline indicates the bounds of the
			topology after which the network repeats. In each topology, the path
			taken by advancing in the Y$^+$ direction from the node labelled `0' is
			shown.}
			\label{fig:topo-compare}
		\end{figure}
		
		\label{sec:hex-vs-h-torus}
		
		The hexagonal torus topology is not to be confused with the `H-torus'
		topology. This topology also uses a hexagonal tiling of nodes and even
		wraps this tiling into a torus-like topology \cite{zhao08}. However,
		H-torus topologies have very different characteristics to the hexagonal
		torus topology and are related to `twisted torus' topologies
		\cite{camara10}. For example, figure~\ref{fig:topo-compare} illustrates one
		major difference in the way paths wrap around the peripheries of both
		topologies.
	
	\section{Scaling-up SpiNNaker machines}
		
		To build large SpiNNaker systems comprising of tens of thousands of
		SpiNNaker chips, groups of forty-eight chips are mounted onto circuit
		boards as illustrated in figure~\ref{fig:spinnakerBoard}. These boards may
		be connected together to form larger systems.  Figure~\ref{fig:threeboard}
		shows a prototype three board system. Though the chips are
		\emph{physically} arranged in a (nearly) $7\times7$ grid on each SpiNNaker
		board, they logically form a hexagonal `wrapped triple'
		\cite{davidsonWiring} (see appendix~\ref{sec:partitioning}) which logically
		fit together as illustrated in figure~\ref{fig:threeboard-separate}. The
		labelled exposed corners of the three forty-eight chip boards connect
		together to form a $12\times12$ hexagonal torus topology as illustrated in
		figure~\ref{fig:threeboard-wrapped}. Larger SpiNNaker machines are
		assembled by combining more boards.
		
		\begin{figure}
			\center
			\begin{subfigure}[b]{0.45\linewidth}
				\center
				\includegraphics[width=\linewidth]{figures/spinnakerBoard.jpg}
				
				\caption{A SpiNNaker board}
				\label{fig:spinnakerBoard}
			\end{subfigure}
			~~~
			\begin{subfigure}[b]{0.45\linewidth}
				\center
				\includegraphics[width=\linewidth]{figures/threeboard.jpg}
				
				\caption{Three board prototype}
				\label{fig:threeboard}
			\end{subfigure}
			
			\vspace*{1em}
			
			\begin{subfigure}[b]{0.45\linewidth}
				\center
				\buildfig{figures/threeboard-separate.tex}
				
				\caption{Three board topology}
				\label{fig:threeboard-separate}
			\end{subfigure}
			~~~
			\begin{subfigure}[b]{0.45\linewidth}
				\center
				\buildfig{figures/threeboard-wrapped.tex}
				
				\caption{\ldots{}as a parallelogram}
				\label{fig:threeboard-wrapped}
			\end{subfigure}
			
			\caption{SpiNNaker boards and their topology.}
			\label{fig:spinnaker-boards}
		\end{figure}
		
		
		SpiNNaker chips on the same circuit board connect using low power links
		requiring sixteen wires each.  If this link technology were used to connect
		chips on neighbouring boards, each pair of boards would need to be
		connected with a 128~wire cable.  Cables and connectors supporting this
		many signals are expensive, unreliable and physically large. Instead,
		chip-to-chip connections between boards are multiplexed and demultiplexed
		onto a single High-Speed Serial (HSS) link \cite{athavale05} carried via
		commodity S-ATA cables which are often used to connect hard disks in
		desktop computers and servers \cite{sata3spec}. The six high-speed links
		are implemented by three onboard FPGAs (the three large chips at the top of
		the SpiNNaker board) and are logically transparent to the underlying
		network. The underlying technology and the choice of S-ATA cables limits
		each board-to-board connection to spanning at most one metre gaps. In
		chapter~\ref{sec:building} I present a cabling scheme for hexagonal torus
		topologies which enable large SpiNNaker systems to be assembled using only
		short cables between boards.
		
	\section{Conclusions}
		
		The SpiNNaker architecture has been designed to enable the simulation of
		large biologically realistic neural models in real time. To support this,
		its network architecture takes on an unconventional design based on a
		custom router and hexagonal torus topology. In the remainder of this
		thesis, I will tackle a number of the challenges in scaling up the
		SpiNNaker architecture outlined in this chapter.

	\chapter{Building large SpiNNaker machines}
	
	Like any super computer, physically putting together a large SpiNNaker
	machine poses many challenges in terms of organisation, assembly and
	maintainance. One of the key tasks in this process is the installation of
	network cables such that a desired overall network topology is constructed.
	The largest planned SpiNNaker machine will use \num{3600} S-ATA
	\cite{sata3spec} cables to interconnect its \num{1200} circuit boards,
	creating a hexagonal torus topology. Since the machine will be installed
	within standard server room cabinets (which are not available in a
	giant-doughnut form-factor) a mapping from a board's logical location in the
	network topology to its physical location must be constructed. In addition,
	the interconnect technology employed by SpiNNaker restricts the length of
	S-ATA cables used to $\le$~\SI{1}{\meter}, constraining the possible mappings
	used. In addition the practical issues of installation complexity and
	maintainance must be considered since all \num{3600} cables must ultimately
	be installed and maintained by human operators.
	
	In this chapter I describe a novel technique for physically laying out
	machines configured in hexagonal torus topologies, such as SpiNNaker, in
	commercial machine rooms, building on the techniques used in more
	conventional torus topologies. In addition, I also propose a new methodology
	for installing and maintaining super computer cabling which which exploits
	existing diagnostic features of the SpiNNaker hardware to interactively guide
	and validate cable installation. Finally, I demonstrate how these new
	techniques have been used successfully to interconnect a prototype
	\num{518400} core SpiNNaker machine in substantially less time than the
	industry norm.
	
	In this chapter, the term \emph{unit} refers to the smallest physical
	ecomponent between which connections connections are to be made. For example,
	in a SpiNNaker machine a unit is a 48-chip board while in data center, a unit
	might be a server blade.
	
	\section{Related work}
		
		In this section I describe the techniques conventionally employed when
		laying out and interconnecting the units within super computers. Due to
		SpiNNaker's hexagonal torus topology and dense physical packing of units,
		these existing techniques are found to be insufficient. In the remainder of
		the chapter we will explore solutions to the limitations exposed below.
		
		\subsection{Avoiding long cables}
			
			Na\"ive arrangements of torus topologies, including hexagonal torus
			topologies, feature long `wrap-around' connections which connect units at
			the peripheries of the system. These connections can be problematic for
			numerous reasons:
			
			\begin{description}
				
				\item[Performance] Signal quality diminishes as cables get longer,
				requiring the use of slower signalling speeds, increased error
				correction overhead or more complex hardware.
				
				\item[Energy] Longer cables require higher drive strengths and/or
				buffering to maintain signal integrity.
				
				\item[Cost] Cost Shorter cables are cheaper than long ones.  Longer
				cables imply more wire in a given space making the tasks of routing or
				cable installation more difficult increasing labour costs by as much as
				$5\times$ \cite{curtis12}.
				
			\end{description}
			
			In conventional torus topologies the need for long cables is eliminated
			by folding and interleaving units of the network \cite{dally04}. For
			example, for a 1D torus topology (a ring network), one long connection
			exists to connect the two opposite sides of the system. To remove these
			long connections, half the units are `folded' on top of the others and
			then this arrangement of units is interleaved as illustrated in figure
			\ref{fig:ring-folding}.
			
			\begin{figure}
				\center
				\begin{subfigure}[b]{0.39\linewidth}
					\center
					\buildfig{figures/ring-folding-row.tex}
					\caption{A ring network}
					\label{fig:ring-folding-row}
				\end{subfigure}
				\begin{subfigure}[b]{0.24\linewidth}
					\center
					\buildfig{figures/ring-folding-folded.tex}
					\caption{Folded}
					\label{fig:ring-folding-folded}
				\end{subfigure}
				\begin{subfigure}[b]{0.35\linewidth}
					\center
					\buildfig{figures/ring-folding-interleaved.tex}
					\caption{Folded and interleaved}
					\label{fig:ring-folding-interleaved}
				\end{subfigure}
				
				\caption{Folding and interleaving a ring network to reduce maximum wire
				length.}
				\label{fig:ring-folding}
			\end{figure}
			
			Folding and interleaving has the effect of approximately doubling the
			average cable length but also eliminates the need for a cable spanning
			the entire system. Since the mean cable length is typically already
			short, doubling it in exchange for a substantially reduced maximum cable
			length is often preferable.
			
			The folding and interleaving process may be extended to $N$-dimensional
			torus topologies by folding each dimension in turn. Since all dimensions
			are orthogonal, the folding process only moves units in the dimension
			being folded. In the hexagonal torus topology, however, the three
			dimensions are non-orthogonal and thus folding in one dimension also
			moves units in other dimensions, preventing the edges of the torus
			meeting as illustrated in figure \ref{fig:failing-to-fold-hex-toruses}.
			
			\begin{figure}
				\center
				\begin{subfigure}[b]{0.24\linewidth}
					\center
					\buildfig{figures/failing-to-fold-hex-toruses-none.tex}
					\caption{Not folded}
					\label{fig:failing-to-fold-hex-toruses-none}
				\end{subfigure}
				\begin{subfigure}[b]{0.24\linewidth}
					\center
					\buildfig{figures/failing-to-fold-hex-toruses-x.tex}
					\caption{X}
					\label{fig:failing-to-fold-hex-toruses-x}
				\end{subfigure}
				\begin{subfigure}[b]{0.24\linewidth}
					\center
					\buildfig{figures/failing-to-fold-hex-toruses-y.tex}
					\caption{Y}
					\label{fig:failing-to-fold-hex-toruses-y}
				\end{subfigure}
				\begin{subfigure}[b]{0.24\linewidth}
					\center
					\buildfig{figures/failing-to-fold-hex-toruses-z.tex}
					\caption{Z}
					\label{fig:failing-to-fold-hex-toruses-z}
				\end{subfigure}
				
				\caption{Schematics showing hexagonal torus topologies folded along
				each of their non-orthogonal dimensions. Note that folding along
				the Z axis brings the \emph{wrong} edges closer together.}
				\label{fig:failing-to-fold-hex-toruses}
			\end{figure}
		
		\subsection{Cabling installation}
			
			Existing machine room installations feature very repetitive cabling
			patterns which can easily be memorised by a human technician. For example
			in BlueGene super computers the connectivity between units is highly
			regular \cite{lakner07} while in data centre networks cabling often
			centres around a small number of high-port-count switches
			\cite{cisco07,csernai15}. Cable installation is usually only aided by
			the labelling of connectors and sockets in a standardised manner
			\cite{tia2006} such as in figure \ref{fig:bgWiring}.
			
			\begin{figure}
				\center
				\begin{subfigure}[t]{0.5\textwidth}
					\begin{tikzpicture}
						\node (cables) [inner sep=0]
						      {\includegraphics[width=\textwidth]{figures/bgCables.png}};
						\node (sockets) [inner sep=0, below=1.0em of cables]
						      {\includegraphics[width=\textwidth]{figures/bgSockets.png}};
						
						% Point at label on cable
						\draw [white, <-, line width=0.4em]
						      ([shift={(0.7cm, -0.3cm)}]cables.center)
						      -- ++(45:1cm);
						
						% Point at label on socket
						\draw [white, <-, line width=0.4em]
						      ([shift={(-1.0cm, 1.1cm)}]sockets.center)
						      -- ++(-45:1cm);
					\end{tikzpicture}
					
					\caption{Pre-labelled cables and sockets}
					\label{fig:bgWiringLabels}
				\end{subfigure}
				~
				\begin{subfigure}[t]{0.30\textwidth}
					\includegraphics[height=6.15cm]{figures/bgWiring.jpg}
					
					\caption{Installation of cables}
					\label{fig:bgWiringInstallation}
				\end{subfigure}
				
				\caption{BlueGene/Q cable installation \cite{cscs13}}
				\label{fig:bgWiring}
			\end{figure}
			
			Despite the regularity and careful labelling of cables, the cost of
			installation and maintenance alone can be significant with costs in the
			range of \$45-95 per \SI{1}{\meter} cable run and \$100-400 for runs of
			\SI{10}{\meter} reported in the literature \cite{mudigonda11}. Much of
			this cost is due to the care required during installation to avoid
			miswiring and ensure that cooling airflow is not hampered by cable runs
			\cite{cisco07}.
			
			Many researchers have attempted to control cable installation costs by
			trying to reduce the number or length of cables required by developing
			alternative network topologies \cite{curtis12, popa10, mudigonda11}.
			Unfortunately, these techniques do not apply to SpiNNaker since its
			network topology is fixed.
			
			Some super computers make use of large custom `midplane` PCBs in place of
			cables to interconnect units within a cabinet and thus simplify the task
			of cable installation \cite{prickett10}. This scheme can greatly reduce
			wiring complexity since only coarser-grain cabinet-to-cabinet
			connectivity is provided by cables. Unfortunately this technique is
			expensive and also constrains the dimensions of the network topology
			supported by the machine. Since the SpiNNaker platform is designed to
			scale from desktop machines to machine-room installations, this scheme is
			not practical.
	
	\section{Folding \& interleaving hexagonal toruses}
		
		The first step towards a practical machine-room installation of a large
		machine using a hexagonal torus topology is to find an arrangement of
		boards between which cable lengths are minimised. In this section I
		describe a sequence of transformations which map the positions of units in
		a hexagonal torus topology onto a regular rectangular grid which may be
		folded and interleaved to eliminate long wires. It is worth emphasising
		that this transformation only affects the \emph{physical} positions of
		units and \emph{not} their connectivity.
		
		As described earlier in \S\ref{sec:parititioning} (page
		\pageref{sec:parititioning}), hexagonal torus topologies may be partitioned
		into units containing wrapped-triples of nodes. For example, in SpiNNaker,
		chips (nodes) are partitioned into circuit boards (units) containing 48
		chips. For completeness, this section describes the process of folding both
		systems whose units are individual nodes and those whose units are
		wrapped-triples.
		
		The transformation process is divided into two parts, each described
		separately in this section.
		
		\begin{description}
			
			\item[Parallelogram to rectangle] Units of the system are transformed
			from a parallelogram shape to a rectangular shape.
			
			\item[Uncrinkle] Units within the rectangle are moved such that they all
			lie on a regular (and fully packed) 2D grid.
			
		\end{description}
		
		\subsection{Parallelogram to rectangle}
			
			The hexagonal torus topology is most naturally drawn as a parallelogram
			as illustrated in figures \ref{fig:hex-to-plane-node-native} and
			\ref{fig:hex-to-plane-native}. Two transformations are presented which
			transform these arangements of units into a rectangular form: shearing
			and slicing.
			
			A \SI{30}{\degree} shear transformation distorts networks such that the X
			and Y axes become orthogonal leading to a rectangular arrangement of
			units as illustrated in figures \ref{fig:hex-to-plane-node-shear} and
			\ref{fig:hex-to-plane-shear}.
			
			The slice transformation slices the units protruding from the
			left-hand-side of the parallelogram and moves them into the matching gap
			on the opposite side of the parallelogram as illustrated in figures
			\ref{fig:hex-to-plane-node-slice} and \ref{fig:hex-to-plane-slice}.
			 
			While the shear transformation introduces some distortion causing cables
			in the Z dimension to become $\sqrt{2}\times$ longer it leaves the
			pattern of wrap-around connections remains unchanged. By contrast, the
			slice transformation does not elongate any cables but changes the pattern
			of wrap-around connections. The exact pattern wrap-around connections
			produced when slicing depends on the aspect ratio of the network as
			illustrated in \ref{fig:slicing-examples} and influences the choice of
			folding technique applied as described later.
			
			\begin{figure}
				\center
				\begin{subfigure}[b]{0.32\linewidth}
					\center
					\buildfig{figures/hex-to-plane-node-native.tex}
					
					\caption{$7 \times 7$ node torus}
					\label{fig:hex-to-plane-node-native}
				\end{subfigure}
				\begin{subfigure}[b]{0.32\linewidth}
					\center
					\buildfig{figures/hex-to-plane-node-shear.tex}
					
					\caption{Sheared}
					\label{fig:hex-to-plane-node-shear}
				\end{subfigure}
				\begin{subfigure}[b]{0.32\linewidth}
					\center
					\buildfig{figures/hex-to-plane-node-slice.tex}
					
					\caption{Sliced}
					\label{fig:hex-to-plane-node-slice}
				\end{subfigure}
				
				\caption{Transformations of hexagonal toruses of nodes into a
				rectangular form. Thin lines show wrap-around links. Pointy-topped
				hexagons represent individual nodes.}
				\label{fig:hex-to-plane-node}
			\end{figure}
			
			\begin{figure}
				
				\begin{subfigure}[b]{0.32\linewidth}
					\center
					\buildfig{figures/hex-to-plane-native.tex}
					
					\caption{$4 \times 4$ triad torus}
					\label{fig:hex-to-plane-native}
				\end{subfigure}
				\begin{subfigure}[b]{0.32\linewidth}
					\center
					\buildfig{figures/hex-to-plane-shear.tex}
					
					\caption{Sheared}
					\label{fig:hex-to-plane-shear}
				\end{subfigure}
				\begin{subfigure}[b]{0.32\linewidth}
					\center
					\buildfig{figures/hex-to-plane-slice.tex}
					
					\caption{Sliced}
					\label{fig:hex-to-plane-slice}
				\end{subfigure}
				
				\caption{Transformations of hexagonal toruses of wrapped triples into a
				rectangular form.  Thin lines show wrap-around links. Flat-topped
				hexagons represent a wrapped triple of nodes.}
				\label{fig:hex-to-plane}
			\end{figure}
			
			\begin{figure}
				\center
				\buildfig{figures/slicing-examples.tex}
				\caption{Patterns of wiring in sliced systems of various sizes.}
				\label{fig:slicing-examples}
			\end{figure}
			
		\subsection{Uncrinkling}
			
			Though the transformmation step yields rectangular arrangements of units,
			these arrangements do not fall onto a regular 2D grid, with the exception
			of the shear transform on individual nodes. Figure \ref{fig:uncrinkling}
			illustrates how the various arrangements of hexagons may be moved to
			`uncrinkle' the units into a regular grid.
			
			\begin{figure}
				\center
				\begin{subfigure}[b]{0.44\linewidth}
					\center
					\buildfig{figures/uncrinkling-node-sheared.tex}
					
					\caption{$7 \times 7$ nodes, sheared}
					\label{fig:uncrinkling-node-sheared}
				\end{subfigure}
				\begin{subfigure}[b]{0.44\linewidth}
					\center
					\buildfig{figures/uncrinkling-node-sliced.tex}
					
					\caption{$7 \times 7$ nodes, sliced}
					\label{fig:uncrinkling-node-sliced}
				\end{subfigure}
				
				\vspace{1cm}
				
				\begin{subfigure}[b]{0.44\linewidth}
					\center
					\buildfig{figures/uncrinkling-sheared.tex}
					
					\caption{$4 \times 4$ triples, sheared}
					\label{fig:uncrinkling-sheared}
				\end{subfigure}
				\begin{subfigure}[b]{0.44\linewidth}
					\center
					\buildfig{figures/uncrinkling-sliced.tex}
					
					\caption{$4 \times 4$ triples, sliced}
					\label{fig:uncrinkling-sliced}
				\end{subfigure}
				
				\vspace{1em}
				
				\caption{Mapping rectangular arrangements of units into a square grid.
				Thick lines show how layers of units are uncrinkled.  Annotations show
				how the relative positions of nodes and wrapped triples change after
				uncrinkling.}
				\label{fig:uncrinkling}
			\end{figure}
			
			In the figure, the numbered units enumerate the different positions on
			the crinkle and those labelled alphabetically are those that immediately
			surround them. From this we can observe that uncrinkling largely
			preserves spatial locality but some distortion is introduced, separating
			previously neighbouring units. For example, in figure
			\ref{fig:uncrinkling-sheared}, the units labelled `1' and `i' are
			neighbours before uncrinkling but are separated by a (Euclidean) distance
			of $\sqrt{5}$ afterwards. Note that the distortion introduced depends on
			what part of the crinkle is considered, for example `2' and `a' have
			distance 2 but are logically connected in the same way.
		
		\subsection{Folding and Interleaving}
			
			Once a regular grid of units has been formed, this may be folded in the
			conventional way, eliminating long cables crossing from left-to-right and
			top-to-bottom as illustrated in \ref{fig:folding-sheared}.
			
			Unfortunately, for sliced systems whose dimensions are not of the ratio
			$1:2$, the pattern of wrap-around cables may also include some cables
			which do not cross directly to the opposite side of the system (refer
			back to figure \ref{fig:slicing-examples}). As a result of these
			connections, folding does not successfully eliminate all long
			connections. An exception to this rule is sliced systems whose dimensions
			are in the ratio $1:1$ where folding twice along the Y axis may
			successfully eliminate all wrap-around connections as illustrated in
			\ref{fig:folding-sliced}.
			
			\begin{figure}
				\begin{subfigure}{\linewidth}
					\center
					\buildfig{figures/folding-sheared.tex}
					\caption{$N \times M$ sheared systems and $N \times 2N$ sliced systems}
					\label{fig:folding-sheared}
				\end{subfigure}
				
				\vspace{1em}
				
				\begin{subfigure}{\linewidth}
					\center
					\buildfig{figures/folding-sliced.tex}
					\caption{$N \times N$ sliced systems}
					\label{fig:folding-sliced}
				\end{subfigure}
				
				\caption{Schematic illustrating elimination of long wrap-around links
				during folding. In each example a single link has been highlighted to
				aid in following the process.}
				\label{fig:folding}
			\end{figure}
			
			Once folded, the 2D grid is straight-forwardly interleaved as illustrated
			previously in figure \ref{fig:ring-folding}. The interleaving process
			introduces some additional distortion to the layout of units and causes
			most connections to become twice as long. For sliced $1:1$ systems, the
			additional fold results in additional overhead during interleaving since
			four layers of the system are interleaved.
		
		\subsection{Mapping to Cabinets}
			
			In the final step of the process is to map the 2D grid of units into
			positions in machine room cabinets as illustrated in figure
			\ref{fig:million-core-machine}. As illustrated in figure
			\ref{fig:cabinetisation}, first the grid of units is partitioned into
			groups of columns, one per cabinet, then groups of rows one per frame per
			cabinet. The units in each group are then allocated to slots within a
			frame, interleaving the rows of the groups as shown.
			
			\begin{figure}
				\center
				\buildfig{figures/cabinet-units.tex}
				
				\caption{An illustration of the physical construction of a
				multi-cabinet SpiNNaker system. (Note: network cables \emph{not}
				installed.)}
				\label{fig:cabinet-units}
			\end{figure}
			
			\begin{figure}
				\center
				\buildfig{figures/cabinetisation.tex}
				
				\caption{Mapping from 2D space to cabinets, frames and boards.}
				\label{fig:cabinetisation}
			\end{figure}
		
	\section{Cable installation}
		
		Cable installation is performed by a team of (human) technicians who must
		ensure that all network cables are correctly installed. As illustrated in
		previously in figure \ref{fig:cabinet-units}, the density of SpiNNaker's
		units, combined with the nature of the hexagonal torus topology, poses a
		challenge. To address this challenge I propose a semi-automated approach to
		cable installation which exploits diagnostic facilities available in the
		majority of super computers in order to guide technicians through the
		cabling process, interactively guiding installation and maintenance.
		
		\subsection{Interactive technician guidance and validation}
			
			While automated systems for validating cabling correctness are
			commonplace, these systems are typically used only after cabling has been
			completed \cite{lakner07}. As with other large-scale machines, SpiNNaker
			includes a low-bandwidth system management bus which may be used to
			interrogate network hardware and control diagnostic LEDs prior to the
			installation of the main SpiNNaker network interconnect.  Using these
			facilities I have constructed a tool called SpiNNer which interactively
			guides a technician, or team of technicians, through the cable
			installation process, validating each connection in real-time.
			
			Diagnostic LEDs mounted on each SpiNNaker board (figure
			\ref{fig:interactive-wiring-guide-leds}) are used to indicate the
			endpoints of the cable currently being installed. Simultaneously a
			Text-To-Speech (TTS) system gives an audible indication of which cable
			type is to be used and location of each connection.  Additionally, a GUI
			via a computer display (figure \ref{fig:interactive-wiring-guide-gui}).
			The centre of the display shows a `big-picture' perspective of the
			locations of the boards to be connected. The detailed views on the left
			and right indicate which of the six sockets on each board the cables
			should connect.
			
			\begin{figure}
				\center
				\begin{subfigure}[b]{0.40\textwidth}
					\begin{tikzpicture}
						\node (leds) [inner sep=0]
						      {\includegraphics[width=\textwidth]{figures/leds.jpg}};
						% Point at left LED
						\draw [white, <-, line width=0.4em]
						      ([shift={(-0.0cm, -0.6cm)}]leds.center)
						      -- ++(225:1cm);
						% Point at right LED
						\draw [white, <-, line width=0.4em]
						      ([shift={(1.1cm, -1.1cm)}]leds.center)
						      -- ++(225:1cm);
					\end{tikzpicture}
					
					\caption{Diagnostic LEDs}
					\label{fig:interactive-wiring-guide-leds}
				\end{subfigure}
				~
				\begin{subfigure}[b]{0.546\textwidth}
					\begin{tikzpicture}[thin, black!20!white]
						\node (screen) [inner sep=0]
						      {\includegraphics[width=\textwidth]{figures/wiring_guide_screenshot.png}};
						\draw (screen.south west) rectangle (screen.north east);
					\end{tikzpicture}
					
					\caption{Interactive wiring guide GUI}
					\label{fig:interactive-wiring-guide-gui}
				\end{subfigure}
				
				\caption{The SpiNNer interactive wiring guide uses a GUI,
				text-to-speech and diagnostic LEDs to assist during cable
				installation.}
				\label{fig:interactive-wiring-guide}
			\end{figure}
			
			SpiNNer also validates the connectivity of the system in real-time by
			polling the diagnostic interfaces of the network hardware at the
			endpoints of the cable being installed to determine if they are correctly
			connected. If a miswiring occurs, this is immediately detected and
			announced via TTS enabling the technician to immediately correct the
			error. Once a cable has been installed correctly, the software
			automatically advances to the next cable meaning direct interaction with
			the software by the technician is minimal. In practice, it is rarely
			necessary to refer to the GUI.
		
			SpiNNer presents the cables in an order intended to maximise ease of
			installation. Cables are installed in three groups with intra-frame
			cables being installed first, followed by intra-cabinet cables and
			inter-cabinet cables. Within each group, the tightest cables are
			installed first resulting in slacker cables naturally being installed
			over the top of already installed cables. By grouping cables in this
			manner, multiple technicians may work independently on the wiring within
			individual frames and cabinets.
			
			SpiNNer may also be used to repair or replace cables in the system.
			During maintenance, obstructing cables may be blindly removed alongside
			any cable being replaced. At the conclusion of the process, the wiring
			guide may be used to interactively guide re-installation of all removed
			cables.
		
		\subsection{Cable selection}
			
			Controlling slack is critical to ensuring reliable and maintainable
			cabling installations. If cables are too tight, cables and connectors can
			become easily damaged and when too slack, the excess cable obstructs
			other cables and can easily become tangled and damaged \cite{cisco07}. It
			has been observed that when ready-made cables are employed technicians
			frequently over-estimate the cable lengths required preferring to use
			overly long cables for all connections \cite{mazaris97}. To solve this
			problem, the SpiNNer wiring guide software dictates the cable lengths to
			be used by an installer based the rule of (three-)thumbs according to
			Mazaris \cite{mazaris97}. This rule suggests that an ideal amount of
			slack is approximately that which can be wrapped around three fingers.
			Specifically, the shortest available cable is selected which ensures at
			least \SI{5}{\centi\meter} of slack.
			
			The SpiNNer tool allocates cables assuming all cables take a Euclidean
			straight-line path between the endpoints of the connection. The result is
			that wiring is not routed through dedicated cable management structures
			but is simply suspended by its connectors in front of the machine. As
			demonstrated later, this unconventional approach leads neither to cooling
			problems nor increased maintenance effort.
	
	\section{Results and Evaluation}
		
		This stuff has been used and works. In this section I'll go over the
		overheads introduced by the various transformations and
		folding/interleaving steps and show a wiring scheme for a large machine
		which uses only short cables. I'll then show how SpiNNer was used to
		install this wiring plan into a very large machine without foobaring the
		cooling and in very little time. I'll also report on difficulty of
		maintenance.
		
		\subsection{Cable length}
			
			The transformation from regular hexagonal torus to a folded and
			interleaved form introduces some overhead to the cable lengths required.
			Using figure \ref{fig:uncrinkling} (page \pageref{fig:uncrinkling}), it
			is possible to compute the exact overhead introduced when each type of
			transformation proposed.
			
			For example, to compute the mean overhead introduced by the slicing
			technique when applied to units composed of wrapped triples, consider
			figure \ref{fig:uncrinkling-sliced}. The uncrinkling pattern used to
			transform this topology is a repeating pattern of two units, a pair of
			which have been labelled $1$ and $2$ respectively. Unit $1$ is
			immediately surrounded by six units labelled $a$, $b$, $c$, $2$, $g$ and
			$h$. Similarly, unit $2$ is surrounded by units $1$, $c$, $d$, $e$, $f$
			and $g$. Before the transformation, the distances, $D$, to each of these
			units is $1$ but after the transformation is applied, this is not always
			the case. Additionally, folding and interleaving introduce additional
			overhead. In this example, if the system is folded into $f_x$ columns and
			$f_y$ rows, the distances between previously neighbouring units become:
			
			\begin{equation*}
				\begin{aligned}[c]
					D_{1\,\leftrightarrow{}\,a} &= \sqrt{f_x^2 + f_y^2} \\
					D_{1\,\leftrightarrow{}\,b} &= f_y \\
					D_{1\,\leftrightarrow{}\,c} &= \sqrt{f_x^2 + f_y^2} \\
					D_{1\,\leftrightarrow{}\,2} &= f_x \\
					D_{1\,\leftrightarrow{}\,g} &= f_y \\
					D_{1\,\leftrightarrow{}\,h} &= f_x
				\end{aligned}
				\hspace{2cm}
				\begin{aligned}[c]
					D_{2\,\leftrightarrow{}\,1} &= f_x \\
					D_{2\,\leftrightarrow{}\,c} &= f_y \\
					D_{2\,\leftrightarrow{}\,d} &= f_x \\
					D_{2\,\leftrightarrow{}\,e} &= \sqrt{f_x^2 + f_y^2} \\
					D_{2\,\leftrightarrow{}\,f} &= f_y \\
					D_{2\,\leftrightarrow{}\,g} &= \sqrt{f_x^2 + f_y^2}
				\end{aligned}
			\end{equation*}
			
			From these values, the mean and maximum connection distances after
			folding and interleaving may be computed. Table
			\ref{tab:transform-overhead} gives the mean and maximum connection
			distances for each of the four transformations described in this chapter.
			
			\begin{table}
				\begin{subtable}[b]{\linewidth}
					\center
					\begin{tabular}{l c c}
						\toprule
						& Shear & Slice \\
						\addlinespace
						Nodes &
							$\frac{f_x + f_y + \sqrt{f_x^2 + f_y^2}}{3}$ &
							$\frac{f_x + f_y + \sqrt{f_x^2 + f_y^2}}{3}$ \\
						\addlinespace
						Triples &
							$\frac{7f_x + 3\sqrt{f_x^2 + f_y^2} + \sqrt{(2f_x)^2 + f_y^2}}{9}$ &
							$\frac{f_x + f_y + \sqrt{f_x^2 + f_y^2}}{3}$ \\
						\bottomrule
					\end{tabular}
					
					\caption{Mean}
					\label{tab:transform-overhead-mean}
				\end{subtable}
				
				\vspace{1em}
				
				\begin{subtable}[b]{\linewidth}
					\center
					\begin{tabular}{l c c}
						\toprule
						& Shear & Slice \\
						\addlinespace
						Nodes &
							$\sqrt{f_x^2 + f_y^2}$ &
							$\sqrt{f_x^2 + f_y^2}$ \\
						\addlinespace
						Triples &
							$\sqrt{(2f_x)^2 + f_y^2}$ &
							$\sqrt{f_x^2 + f_y^2}$ \\
						\bottomrule
					\end{tabular}
					
					\caption{Maximum}
					\label{tab:transform-overhead-max}
				\end{subtable}
				
				\caption{Overheads introduced when transforming unit positions onto a
				regular grid.}
				\label{tab:transform-overhead}
			\end{table}
			
			From these results it is evident that shearing and slicing networks
			whose units are nodes result in identical mean and maximum overhead in
			cable length when folded similarly. Since sliced networks may require
			folding more than once along each axis the shearing approach is
			preferable in general.
			
			For networks constructed from units of wrapped triples, the slicing
			approach suffers the same mean and maximum overhead has networks of
			nodes, and less overhead than shearing for the same number of folds. For
			systems with an aspect ratio of $1:2$ (where both slicing and shearing
			require $f_x = f_y = 2$), the slicing transformation yields lower mean
			and maximum overhead than shearing. For all other aspect ratios (where
			slicing requires a greater degree of folding) the shearing technique
			produces lower overhead. The recommended transformations for a given
			machine are thus given in table \ref{tab:transform-recommended}.
			
			\begin{table}
				\center
				\begin{tabular}{lcc}
					\toprule
					                         & $1:2$  & Other \\
					\addlinespace
					\multirow{2}{*}{Nodes}   & Either & Shear\\
					                         & \footnotesize $\mu\approx2.28 \quad \vee\approx2.83$
					                         & \footnotesize $\mu\approx2.28 \quad \vee\approx2.83$\\
					\addlinespace
					\multirow{2}{*}{Triples} & Slice  & Shear\\
					                         & \footnotesize $\mu\approx2.28 \quad \vee\approx2.83$
					                         & \footnotesize $\mu\approx3.00 \quad \vee\approx4.47$\\
					\bottomrule
				\end{tabular}
				
				\caption{Recommended transformation and folding scheme for different
				system types. $\mu$ and $\vee$ give the mean and maximum wire
				distortion introduced, respectively.}
				\label{tab:transform-recommended}
			\end{table}
			
			\begin{figure}
				\center
				\buildfig{figures/million-core-machine.tex}
				
				\caption{Cabling plan for a \num{1036800} core SpiNNaker
				machine's \num{3600} cables.}
				\label{fig:million-core-machine}
			\end{figure}
			
			Following folding and mapping to physical locations, the cabling plans
			for large machines require no large gaps to be spanned.  The largest
			planned SpiNNaker machine, illustrated in figure
			\ref{fig:million-core-machine}, will be \SI{6}{\meter} wide but the
			largest gap any cable must span is \SI{66}{\centi\meter}. This distance
			is well within the \SI{1}{\meter} allowed by the hardware and cables.
			
		\subsection{Installation practicality}
			
			\begin{table}
				\center
				\begin{tabular}{lrr@{$\,$}l}
					\toprule
						System & Number of Cables & \multicolumn{2}{r}{Installation time} \\
					\midrule
						24 boards  & \num{72}   & \num{10} & \si{\minute}         \\
						1 cabinet  & \num{360}  & \num{4}  & \si{\hour}$^\dagger$ \\
						2 cabinets & \num{720}  & \num{2}  & \si{\hour}           \\
						5 cabinets & \num{1800} & ?        &                      \\
					\bottomrule
				\end{tabular}
				
				\caption{Installation times for various sizes of machine.
				$\dagger$~This machine was installed without real-time validation of
				connectivity.}
				\label{tab:install-time}
			\end{table}
			
			A number of SpiNNaker machines of various scales have been assembled
			using the techniques described in this chapter ranging from single frames
			of 24 boards to a half-scale 5 cabinet machine. Table
			\ref{tab:install-time} gives the reported installation times of each of
			these machines.
			
			The single cabinet machine's installation time is notably
			disproportionate to its size. When this system was assembled, real-time
			connection validation was not yet available. As a result, though cable
			installation was rapid correcting errors was extremely costly, requiring
			careful retracing of many installation steps.
			
			TODO: TALK ABOUT MULTI-PERSON-WIRING IN PRACTICE ON FIVE CABINET MACHINE.
			
			\begin{figure}
				
				\center
				\buildfig{figures/wire-length-histogram.tex}
				
				\caption{Histogram of connection distances in a ten-cabinet,
				one-million core SpiNNaker machine annotated with the suggested cable
				length.}
				\label{fig:wire-length-histogram}
				
			\end{figure}
			
			FIGURE \ref{fig:wire-length-histogram} SHOWS THE DISTRIBUTION OF CABLE
			LENGTHS REQUIRED. IN PRACTICE THE SLACK ALLOCATED PROVED ADEQUATE. AS
			SHOWN IN FIGURE \ref{fig:install-histogram}, THE MOST IMPORTANT FACTOR IS
			WHETHER LEAVING THE FRAME OR NOT. LEAVING THE FRAME TAKES THE LONGEST.
			
			\begin{figure}
				\builddata{data/build_connection_log.tex}
				\buildfig{figures/install-histogram.tex}
				
				\caption{Histogram of cable installation times}
				\label{fig:install-histogram}
			\end{figure}
			
			TODO: COMPARE DIRECTLY WITH INSTALL TIMES REPORTED IN LITERATURE.
		
		\subsection{Thermal Impact}
			
			TODO: SHOW HOW TEMPERATURE IS CHANGED
			
		\subsection{Maintenance}
			
			TOOD: QUANTIFY CABLE REMOVALS REQUIRED. EXPERIMENT: REMOVE/REPLACE RANDOM
			BOARDS AND MEASURE TIME TAKEN, CABLES REMOVED. COMPARE WITH STANDARD DATA
			CENTRE WIRING

	\chapter{Finding shortest path vectors in SpiNNaker's network}
	
	Once a SpiNNaker machine has been constructed as described in the previous
	chapter, its network forms a large hexagonal torus topology. To exploit this
	network routing algorithms must be used to generate routes for packets to
	follow between nodes. As well as ensuring that packets arrive at the correct
	destination, routing algorithms often attempt to produce routes which make
	efficient use of the network. This often involves attempting to reduce
	congestion by ensuring packets do not travel further through the network than
	absolutely necessary.
	
	Many popular routing algorithms for torus topologies, including all published
	algorithms designed for SpiNNaker's hexagonal torus topology
	\cite{davies12,navaridas14}, internally function by computing shortest path
	vectors and generating routes from them. Existing methods of calculating
	shortest path vectors in hexagonal torus topologies are unable to generate
	all possible shortest path vectors and, as a result, reduces the diversity of
	routes produced by routing algorithms, potentially worsening network
	contention.
	
	In this chapter I describe a novel technique for computing shortest path
	vectors in hexagonal torus topologies which yields \emph{all} possible
	shortest path vectors for any pair of nodes. Further, implementations of this
	new technique execute an order of magnitude faster than the existing
	alternatives.
	
	\section{Related work}
		
		TODO: INTRODUCE SECTION
		
		\begin{figure}
			\center
			
			\begin{subfigure}{\linewidth}
				\center
				\buildfig{figures/distance-map-mesh.tex}
				\caption{2D mesh topology}
				\label{fig:distance-map-mesh}
			\end{subfigure}
			
			\vspace{1em}
			
			\begin{subfigure}{\linewidth}
				\center
				\buildfig{figures/distance-map-torus.tex}
				\caption{2D torus topology}
				\label{fig:distance-map-torus}
			\end{subfigure}
			
			\vspace{1em}
			
			\begin{subfigure}{\linewidth}
				\center
				\buildfig{figures/distance-map-hex-mesh.tex}
				\caption{Hexagonal mesh topology}
				\label{fig:distance-map-hex-mesh}
			\end{subfigure}
			
			\vspace{1em}
			
			\begin{subfigure}{\linewidth}
				\center
				\buildfig{figures/distance-map-hex-torus.tex}
				\caption{Hexagonal torus topology}
				\label{fig:distance-map-hex-torus}
			\end{subfigure}
			
			\caption{Plots showing distance from various locations marked
			         {\color{red}$\times$}. Darker areas are further away. Contour
			         lines show equidistant points.}
			\label{fig:distance-map}
		\end{figure}
		
		\subsection{Mesh Networks}
			
			In a (non-hexagonal) mesh network topology, shortest path vectors are
			computed by taking the element-wise difference between the source and
			destination nodes' coordinates.
			
			\begin{figure}
				\center
				\buildfig{figures/mesh-topology-coordinates.tex}
				\caption{An example 2D mesh network with example shortest-path routes
				from `A' to `B' and `B' to `C'.}
				\label{fig:mesh-topology-coordinates}
			\end{figure}
			
			For example, figure \ref{fig:mesh-topology-coordinates} illustrates a 2D
			mesh topology. In this topology, the nodes labelled `A', `B' and `C' have
			position vectors $(1, 2)$, $(4, 5)$ and $(6, 1)$ respectively. The
			shortest path vector from node `A' to `B' is thus simply $(4, 5) - (1, 2)
			= (3, 3)$ and from `B' to `C' is $(6, 1) - (4, 5) = (2, -4)$.
			
			A route may be produced from a shortest path vector by advancing the
			number of hops specified for each dimension in the vector. For example
			any permutation of the hops X$^+\,$X$^+\,$X$^+\,$Y$^+\,$Y$^+\,$Y$^+$, an
			example of which is included in the figure. Likewise a route from `B' to
			`C' may be constructed from any permutation of
			X$^+\,$X$^+\,$Y$^-\,$Y$^-\,$Y$^-\,$Y$^-$.
			
			Many popular routing algorithms such as Dimension Order Routing (DOR),
			Right-Turn Only Routing (RTOR) and Longest Dimension First Routing (LDFR)
			\cite{dally04,davies12} directly follow the above procedure and just
			prescribe a specific permutation of hop order. For example, DOR produces
			routes with X hops first, Y hops second and so on.
			
			The length of routes produced from a shortest path vector have a number
			of hops proportional to the magnitude of the vector, thus a shortest path
			vector yields a route with the minimum number of hops. For a two
			dimensional vector $(a, b)$ the magnitude is given as:
			%
			\begin{equation}
				\| (a, b) \| = \lvert a \rvert + \lvert b \rvert
			\end{equation}
		
		\subsection{Torus Networks}
			
			Computing shortest path vectors in (non-hexagonal) torus topologies is
			also straight forward. As an example, lets find the shortest path vector
			from node `A' to `B' in the 2D torus topology shown in figure
			\ref{fig:torus-shortest-path-example}. First, both nodes are translated
			such that the source node, `A', is at the centre of the network (figure
			\ref{fig:torus-shortest-path-translate}). Note that this translation may
			result in the destination node `wrapping around' the network. After
			translation, the shortest path vector is computed as in a mesh topology.
			As illustrated in \ref{fig:torus-shortest-path-routed}, the computed
			shortest path vector may be used to produce routes between the two nodes
			in their original positions.
			
			\begin{figure}
				\center
				\begin{subfigure}{0.3\linewidth}
					\center
					\buildfig{figures/torus-shortest-path-example.tex}
					\caption{Original}
					\label{fig:torus-shortest-path-example}
				\end{subfigure}
				\begin{subfigure}{0.3\linewidth}
					\center
					\buildfig{figures/torus-shortest-path-translate.tex}
					\caption{Translated}
					\label{fig:torus-shortest-path-translate}
				\end{subfigure}
				\begin{subfigure}{0.3\linewidth}
					\center
					\buildfig{figures/torus-shortest-path-routed.tex}
					\caption{Routed}
					\label{fig:torus-shortest-path-routed}
				\end{subfigure}
				
				\caption{Finding shortest paths in a 2D torus topology.}
				\label{fig:torus-shortest-path}
			\end{figure}
			
			This process works because vectors from the centre (though not other
			locations) of a torus topology are identical to those in mesh topologies
			(see figures \ref{fig:distance-map-mesh} and
			\ref{fig:distance-map-torus}).
		
		\subsection{Hexagonal Mesh Networks}
			
			In hexagonal mesh topologies it is conventional to define three `axes' X,
			Y and Z as shown in figure \ref{fig:hex-mesh-topology-coordinates}
			\cite{patel15}. In this example, the three labelled nodes `A', `B' and
			`C' may be given position vectors such as $(1, 1, 0)$, $(3, 2, 0)$ and
			$(0, 0, -7)$ respectively. As in other mesh networks, a vector between
			two nodes is found by subtracting the nodes' vectors. For example, a
			vector from `A' to `B' is $(3, 2, 0) - (1, 1, 0) = (2, 1, 0)$. This
			vector can then be converted into a route in the same way as a mesh
			network by taking any permutation of the three hops  X$^+\,$X$^+\,$Y$^+$.
			
			\begin{figure}
				\center
				\buildfig{figures/hex-mesh-topology-coordinates.tex}
				\caption{An example hexagonal mesh network topology.}
				\label{fig:hex-mesh-topology-coordinates}
			\end{figure}
			
			As explained in detail in appendix \ref{app:minimal-hex-coordinates},
			there are an infinite number of vectors between any two points. For
			example, the vectors $(1, 0, -1)$ and $(3, 2, 1)$ also reach node `B'
			from `A' in the example. However, for a given pair of nodes, there is
			always a single, unique vector whose magnitude is minimal which is
			given by the function:
			%
			\begin{equation}
				\operatorname{minimiseVector}(x,y,z)
					= (x,y,z) - \operatorname{median}(x,y,z) \cdot (1,1,1)
			\end{equation}
			%
			An important side-effect of this function is that a minimised vector will
			always contain at least one zero element meaning that shortest path
			routes will use at most two of the three available dimensions.
			
			To aid the reader's intuition, figure \ref{fig:distance-map-hex-mesh}
			illustrates how distances grow in a hexagonal mesh topology.
		
		\subsection{Hexagonal Torus Networks}
			
			Unfortunately, unlike non-hexagonal torus topologies, the translation
			technique cannot be used to compute shortest path vectors. As illustrated
			in figures \ref{fig:distance-map-hex-mesh} and
			\ref{fig:distance-map-hex-torus}, shortest path vectors from the center
			of a hexagonal mesh network are not equivalent to those of a hexagonal
			torus network.
			
			Prior research into routing in SpiNNaker's network has been based on the
			INSEE \cite{navaridas09,ghasempour15} interconnect simulator. Internally
			INSEE tries a set of twelve candidate vectors and picks the shortest as
			the shortest path vector to use for routing.
			
			\begin{figure}
				\center
				\begin{subfigure}{0.45\linewidth}
					\center
					\buildfig{figures/insee-vector-candidates-no-wrap.tex}
					\caption{$(\Delta_\textrm{X}, \Delta_\textrm{Y}) = (5,3)$}
					\label{fig:insee-vector-candidates-no-wrap}
				\end{subfigure}
				\begin{subfigure}{0.45\linewidth}
					\center
					\buildfig{figures/insee-vector-candidates-wrap-x.tex}
					\caption{$(\Delta'_\textrm{X}, \Delta_\textrm{Y}) = (-3,3)$}
					\label{fig:insee-vector-candidates-wrap-x}
				\end{subfigure}
				
				\vspace{1em}
				
				\begin{subfigure}{0.45\linewidth}
					\center
					\buildfig{figures/insee-vector-candidates-wrap-y.tex}
					\caption{$(\Delta_\textrm{X}, \Delta'_\textrm{Y}) = (5,-5)$}
					\label{fig:insee-vector-candidates-wrap-y}
				\end{subfigure}
				\begin{subfigure}{0.45\linewidth}
					\center
					\buildfig{figures/insee-vector-candidates-wrap.tex}
					\caption{$(\Delta'_\textrm{X}, \Delta'_\textrm{Y}) = (-3,-5)$}
					\label{fig:insee-vector-candidates-wrap}
				\end{subfigure}
				
				\vspace{1em}
				
				% Key
				\begin{tikzpicture}[thick]
					\coordinate (last);
					
					% #1 colour
					% #2 label
					\newcommand{\colourkeyentry}[2]{
						\node [#1] [right=of last, fill, rectangle, minimum size=1em] (last) {};
						\node [right=0 of last] (last) {#2};
					}
					
					\colourkeyentry{cb3class0}{$(\textrm{X}, \textrm{Y}, 0)$}
					\colourkeyentry{cb3class1}{$(\textrm{X} - \textrm{Y}, 0, - \textrm{Y})$}
					\colourkeyentry{cb3class2}{$(0, \textrm{Y} - \textrm{X}, - \textrm{X})$}
					
				\end{tikzpicture}
				
				\caption{The twelve candidate shortest-path vectors considered by INSEE
				represented as dimension-order routes. $W=H=8$,
				$(\Delta_\textrm{X},\Delta_\textrm{Y}) = (5, 3)$ and
				$(\Delta'_\textrm{X},\Delta'_\textrm{Y}) = (-3, -5)$.}
				\label{fig:insee-vector-candidates}
			\end{figure}
			
			The twelve vectors considered are constructed as follows.
			
			First a shortest path vector from the source to target node are
			constructed as if the network was a 2D mesh yielding a vector
			$(\Delta_\textrm{X},\Delta_\textrm{Y})$. From this, another vector
			$(\Delta'_\textrm{X},\Delta'_\textrm{Y})$, is defined:
			%
			\begin{align}
				\Delta'_\textrm{X} &= \Delta_\textrm{X} - \operatorname{sign}(\Delta_\textrm{X})W
				\\
				\Delta'_\textrm{Y} &= \Delta_\textrm{Y} - \operatorname{sign}(\Delta_\textrm{Y})H
			\end{align}
			%
			Where $W$ and $H$ are the width and height of the network respectively
			(in nodes). This new vector yields routes from the source to destination
			node that in a torus topology that \emph{always} wrap around the `X' and
			`Y' dimensions.
			
			From the pair of vectors defined, four possible 2D vectors can be
			produced: $(\Delta_\textrm{X},\Delta_\textrm{Y})$,
			$(\Delta'_\textrm{X},\Delta_\textrm{Y})$,
			$(\Delta_\textrm{X},\Delta'_\textrm{Y})$ and
			$(\Delta'_\textrm{X},\Delta'_\textrm{Y})$. Further, each 2D vector may be
			converted into one of three 3D vectors where either X, Y or Z are zero
			for a total of twelve candidate vectors.  Figure
			\ref{fig:insee-vector-candidates} illustrates all twelve candidate
			vectors for an example pair of nodes.
			
			\begin{figure}
				\center
				\buildfig{figures/xyz-protocol-regions.tex}
				
				\caption{The four regions defined by the XYZ-protocol.}
				\label{fig:xyz-protocol-regions}
			\end{figure}
			
			A more efficient technique is proposed by Hoffmann and D\'es\'erable
			called the XYZ-Protocol \cite{hoffmann15,hoffmann11}. If the source and
			destination nodes are translated such that the source node lies at the
			center of the topolgoy, the destination will lie in one of four regions
			illustrated in figure \ref{fig:xyz-protocol-regions}.
			
			If the destination lies in regions 1 or 4, a route may be constructed as
			if in a hexagonal mesh topology.
			
			Alternatively, if the destination lies in regions 2 or 3, the algorithm
			tests whether taking a mesh-like route within the region or
			wrapping-around either the X or Y dimension yields the shorter vector.
			The shortest of these vectors is then chosen.
			
			TODO DESCRIBE SPIRAL ROUTES.
			
			TODO DESCRIBE RTOR AND LDFR.
		
	\section{Dimension order routing in hexagonal torus topologies}
		
		So, existing solutions have two problems: trying 12 options and picking one
		is a bit kludgey and there are actually more options than that.
		
		\subsection{Simpler minimal hexagonal torus vectors}
			
			If you redraw your route such that it is sourced from bottom left corner
			(which we'll now call (0, 0)), there are four possible ways this route
			could wrap.
			
			TODO: DESCRIBE WAYS OF WRAPPING
			
			For each of these wrappings, all the possible routes we can take are
			strictly limited in terms of the dimensions used since we're stuck in a
			corner.
			
			In each case, the function computing the minimal hex vector function
			simplifies to a much simpler operation.
			
			TODO: DESCRIBE MINIMUM VECTOR LENGTH FUNCTIONS FOR EACH CASE
			
			This gives us a cheap way to compute which of the four possible wrappings
			are shortest. Based on this we can pick one of at most two (is this
			easily provable?) vectors in some fair manner to reduce load. This vector
			can then be minimised in the usual way.
			
			This also leads to confirming a theoretical result giving the length of a
			shortest path in a hexagonal torus topology.
			
			TODO: DESCRIBE HOW TO GET LENGTH FUNCTION AND COMPARE WITH \cite{xiao04}
		
		\subsection{Generating spiralling routes}
			
			In non-hexagonal torus topologies the previous technique would reveal all
			possible shortest vectors (e.g. in cases where you can wrap more than one
			way). Unfortunately, due to the addition of a non-orthogonal axes,
			hexagonal toruses also have other cases when the width and height do not
			match.
			
			TODO: TORUS SPIRALLING EXAMPLE
			
			It is possible to calculate the maximum number of spirals thus:
			
			TODO: DESCRIBE HOW MAX NUMBER OF SPIRALS IS COMPUTED
			
			Given a number of spirals, the vector can be updated this (note that the
			change does not add a multiple of (1, 1, 1) but also does not result in
			the vector changing length and thus becoming non-minimal!).
			
			TODO: DESCRIBE TRANSFORMATION
			
			TODO: PROVE THAT MINIMALITY IS MAINTAINED
		
		\subsection{Proof of completeness}
		
			TODO: PROOF OF COMPLETENESS BY EXHAUSTIVE SEARCH
	
		\subsection{Conclusions}
			
			This approach is simpler, smaller, has sounder theoretical basis, and
			finds more routes than alternatives. This is good for load balancing and
			fault avoidance and also good for completeness.


	\chapter{Routing packets in large SpiNNaker machines}
	
	\label{sec:routing}
	
	So far, this thesis has focused on tackling the practical challenges
	resulting from SpiNNaker's hexagonal torus network topology. In this chapter,
	I adjust my focus towards the practical challenges resulting from SpiNNaker's
	large scale. Faults in large systems are inevitable and in the half-million
	core, \num{28800} chip SpiNNaker machine recently completed at the University
	of Manchester, around \SI{1}{\percent} of chips exhibited faults\footnote{Of
	the faulty chips discovered, the vast majority of faults are attributed to a
	currently unknown SDRAM failure}. These faults result in gaps and broken
	links in the network topology which routing algorithms must avoid in order to
	ensure correct system operation.
	
	In this chapter I tackle the problem of extending existing routing algorithms
	for SpiNNaker's network to enable them to route-around known, static faults.
	Though dynamic or transient faults may also occur, in this work such faults
	are ignored and other techniques, such as protocol-level fault tolerance, are
	relied on instead.
	
	Numerous heuristic-based fault-tolerant routing algorithms exist which target
	different network topologies and router architectures. Unfortunately as I
	will show, these algorithms are not portable and rely on or attempt to work
	around specific features of their target network architecture. In particular,
	existing work is dominated by the challenge of developing routing schemes
	which avoid or defuse network deadlocks. Due to SpiNNaker's unconventional
	use of timeout-based flow-control, it is not subject to the routing
	restrictions present in other architectures intended to cope with deadlocks.
	
	In this chapter I introduce a graph-search based post-processing step for
	non-fault-tolerant routing algorithms which guarantees routability in
	SpiNNaker systems without disconnected subregions. I also demonstrate that
	this technique introduces both negligible computational overhead to the
	routing algorithm runtime and resulting network performance.
	
	TODO: NOTE THE FAULT RATES ENCOUNTERED IN PRACTICE...
	
	\section{Related work}
		
		Existing work on routing in SpiNNaker's network has ignored the challenge
		of avoiding faults and instead focused on producing efficient multicast
		routes. As a result this section is broken into two halves. In the first
		half I survey the existing non-fault-tolerant approaches to routing used in
		SpiNNaker to-date. In the second I discuss the approaches to fault tolerant
		routing taken in other systems.
		
		\subsection{Multicast routing in SpiNNaker}
			
			Various fault-intolerant multicast routing algorithms exist for many
			networks and a number have been proposed and evaluated specifically in the
			context of SpiNNaker.
			
			In 2012, Davies \emph{et al.} evaluated the use of three common torus
			routing algorithms in SpiNNaker and found that simple oblivious routing is
			suitable for typical neural applications \cite{davies12}. The three
			routing techniques are:
			
			\begin{description}
				
				\item[Dimension Order Routing (DOR)] Packets are routed along each
				dimension (e.g. $X$, $Y$ and $Z$) in turn until no further hops are
				available in that direction.  The order in which the dimensions are
				traversed is fixed.
				
				\item[Right Turn Only Routing (RTOR)] As in DOR except the dimension
				order is chosen such that routes only contain right-turns.
				
				\item[Longest Dimension First Routing (LDFR)] As in DOR except the
				dimension order is chosen in descending order of number of hops in each
				dimension.
				
			\end{description}
			
			These unicast routing techniques are converted into a multicast routing
			algorithm by merging together the routes produced between the source node
			and each destination node as illustrated in figure
			\ref{fig:simple-routers}.
			
			\begin{figure}
				\center
				\begin{subfigure}{0.3\linewidth}
					\center
					\buildfig{figures/simple-routers-dor.tex}
					
					\caption{DOR}
					\label{fig:simple-routers-dor}
				\end{subfigure}
				\begin{subfigure}{0.3\linewidth}
					\center
					\buildfig{figures/simple-routers-rtor.tex}
					
					\caption{RTOR}
					\label{fig:simple-routers-dor}
				\end{subfigure}
				\begin{subfigure}{0.3\linewidth}
					\center
					\buildfig{figures/simple-routers-ldfr.tex}
					
					\caption{LDFR}
					\label{fig:simple-routers-dor}
				\end{subfigure}
				
				\caption{Example multicast routes produced by merging together unicast
				routes from a central source node to each destination node.}
				\label{fig:simple-routers}
			\end{figure}
			
			In 2014, Navaridas \emph{et al.} introduced two new algorithms, `Enhanced
			Shortest Path Routing' (ESPR) and `Neighbourhood Exploring Routing' (NER)
			which produce multicast routing trees with fewer total hops
			\cite{navaridas14}. In both algorithms, routes are generated sequentially
			for each of the destinations of a route using LDFR. Unlike LDFR, however,
			these algorithms search a limited area of the network for other,
			already-connected destination nodes to which LDFR routes may be
			constructed. If no suitable destination is found, a LDFR route is
			constructed to the source node. Figure \ref{fig:search-regions} illustrates
			the shape of the searched regions of each algorithm. ESPR searches the
			trapezoidal region between the source and destination nodes while NER
			searches a fixed radius out from the destination node.
			
			\begin{figure}
				\center
				\begin{subfigure}{0.45\linewidth}
					\center
					\buildfig{figures/search-regions-espr.tex}
					
					\caption{ESPR}
					\label{fig:search-regions-espr}
				\end{subfigure}
				\begin{subfigure}{0.45\linewidth}
					\center
					\buildfig{figures/search-regions-ner.tex}
					
					\caption{NER}
					\label{fig:search-regions-espr}
				\end{subfigure}
				
				\caption{The ESPR and NER algorithms attempt to connect the node marked
				`D' to the closest node in the shaded region which is connected to the
				source node, `S'. If no connected node is found in the shaded region, the
				LDFR route is taken to `S'. The dotted line indicates the route chosen
				from `D'.}
				\label{fig:search-regions}
			\end{figure}
			
			Unfortunately none of these routing algorithms make any allowance for the
			avoidance of network faults. As a result their utility in real-world
			systems is limited.
		
		\subsection{Fault-tolerant routing}
			
			Numerous fault-tolerant routing algorithms have been proposed for
			super-computer networks. These algorithms are largely constrained by the
			need to maintain deadlock freedom. Since SpiNNaker's routers employ a
			timeout based deadlock-breaking strategy, much of this effort is
			unnecessary in SpiNNaker. As described below, this frequently renders
			existing fault-tolerant routing algorithms unnecessarily complex and
			inflexible.
			
			Deadlocks occur in a network if a cyclic dependency exists on any limited
			resource in the network. For example, as illustrated in figure
			\ref{fig:ring-deadlock}, in a ring network a deadlock may form when every
			node is waiting on the next node to accept a packet before accepting new
			packets from the previous node.
			
			\begin{figure}
				\center
				\buildfig{figures/ring-deadlock.tex}
				
				\caption{A deadlock in a ring network where each node is waiting for
				the next to accept a packet before accepting any further packets.}
				\label{fig:ring-deadlock}
			\end{figure}
			
			To prevent deadlocks, combinations of router microarchitectural features
			and routing restrictions may be employed. For example, a simple
			deadlock-free routing algorithm for mesh and torus networks mandates the
			use of DOR \cite{dally93}. Packets travelling in a -ve direction along
			each axis take priority over those travelling in a +ve direction. Packets
			travelling along the Y axis take priority over those travelling along the
			X dimension. Given these rules it is possible to define a total ordering
			on all hops in the network. Figure \ref{fig:deadlock-free-dor}
			illustrates a $3\times3$ mesh network whose hops have been numbered
			according to the total ordering defined above.  Any `X-then-Y' DOR route
			through this network results in the use of hops labelled with strictly
			increasing numbers. As a result, no cyclic dependencies (and thus no
			deadlocks) may occur.
			
			\begin{figure}
				\center
				\buildfig{figures/deadlock-free-dor.tex}
			
				\caption{Deadlock-free routing of two example routes using DOR in a 2D
				mesh topology. The numbers of the hops taken by each route are given on
				the right.}
				\label{fig:deadlock-free-dor}
			\end{figure}
			
			Unfortunately, the routing restrictions imposed to ensure deadlock
			freedom can result in fault-intolerant routing. In the example above, if
			the node at the bottom-right corner of the figure was faulty, the dotted
			example route would be blocked as no alternative routes are allowed.
			
			In practice, the routing rules used may be more relaxed, for example
			requiring that any route whose length is equal to a DOR must exist to
			guarantee routability \cite{rodrigo09}.
			
			Alternative routing strategies take a hybrid approach whereby an
			efficient but fault-intollerant routing algorithm is used where possible
			and in the presence of faults a less efficient but more robust strategy
			is employed. For example, the Immucube network architecture employs three
			virtual networks which operate independently over the same physical links
			\cite{puente07}. Initially messages are routed using a high-performance
			but potentially-deadlockable routing scheme in the first virtual network.
			If a deadlock is occurs, the deadlocked packet is dropped into the second
			virtual network in which packets are routed using a less efficient but
			deadlock-free but fault-intolerant routing algorithm. Finally, upon
			encountering a fault, packets are dropped onto the third virtual network
			which forms a ring network routing packets to every node in the network.
			
			Releated approaches \cite{mejia06,boppana95} divide the network into
			regions in which different routing rules are enforced to ensure deadlock
			freedom and, when required, fault tolerance.
			
			TODO FIGURE?
			
			The BlueGene/L supercomputer \cite{adiga02} uses DOR for its torus
			network and implements fault-tolerance by sacrificing otherwise
			functioning `lamb' nodes to ensure no route passes through a known dead
			link \cite{ho04}. In figure \ref{fig:lamb-nodes} an example scenario is
			shown where a single dead node is present and all nodes in the same row
			or column as the dead node have been made into lamb nodes. The lamb nodes
			may not be used in an application except as a through-route for other
			traffic. This pattern of lamb nodes guarantees that all dimension-order
			routes between all pairs of non-lamb nodes are not obstructed by the
			faulty node. This approach trades use of higher performance routing
			logic for wasted resources. This type of approach is most appropriate
			when algorithmic routing is used and routing rules are inflexible.
			
			\begin{figure}
				\center
				\buildfig{figures/lamb-nodes.tex}
				
				\caption{`Lamb' nodes may be disabled to ensure DOR will never
				encounter a fault.}
				\label{fig:lamb-nodes}
			\end{figure}
			
			Other algorithms proposed for the BlueGene architecture attempt to avoid
			the need for lamb nodes by generating routes which reach their destination
			via a `proxy' node \cite{gomez04}. By appropriately selecting the location
			of such a proxy, the existing routing algorithm used by the system can be
			guaranteed to select a route free of faults.
			
			TODO: EXAMPLE OF PROXY ROUTING TO AVOID FAULT
			
			Finally, many algorithms in in the field are distributed and use only local
			information along with limited information from their peers to generate
			routes \cite{fick09b}. In SpiNNaker, route generation is conventionally
			carried out centrally since no special on-chip hardware facilities exist
			for route generation. Centralised route generation also enables the routing
			algorithm to consider all available routes. As a result, there is little
			incentive for the use of distributed routing algorithms on SpiNNaker since
			global system information could be compactly shared for one-off routing
			passes.
			
			Algorithms for other architectures such as IP networks tend to be poor fits
			for static, regular network topologies since they use expensive graph-based
			algorithms for route discovery which aren't necessary here. They also tend
			to heavily feature graph topology discovery etc. which aren't needed here.
			
			Work on fault-tolerance in data centre networks does exploit the regularity
			of the network topology in routing algorithms \cite{guo08,liao12}.
			Unfortunately, the approaches used are not general enough to be applied to
			mesh-like topologies such as the one in SpiNNaker.
			
			Outside the field of computer networks, routing algorithms used to route
			wires across the surfaces of chips are required to solve similar problems
			to fault-tolerant network routing problems in mesh networks. Like mesh
			networks, the routes are defined within a regular Manhattan geometry and
			congested areas, rather than faults must be avoided by the algorithms
			\cite{kahng11}.  Unfortunately, these algorithms are designed for
			occasional batch operation prior to the multi-month process of chip
			manufacturing and so runtimes of hours or days are commonplace
			\cite{nam08}. As such these algorithms would be inappropriate for use
			with applications such as SpiNNaker where users' applications tend to be
			short-lived and thus routing should not be allowed to dominate runtime.
	
	\section{Partial graph search repair}
		
		In this section I introduce a novel post-processing algorithm, Partial
		Graph Search (PGS) repair, for routes produced by non-fault-tolerant
		routing algorithms.
		
		PGS repair guarantees routability for networks with no disconnected
		subregions by using a graph search algorithm to route around faults in the
		original route.  General-purpose graph search algorithms such as Breadth
		First Search (BFS), Dijkstra's Algorithm and A* are guaranteed to find
		shortest-path routes between pairs of points in arbitrary graphs. Such
		algorithms are generally a poor choice in highly regular network topologies
		such as meshes and toruses due to their high computational cost. In PGS
		repair, graph searching is only used for \emph{part} of the routing
		problem: to repair gaps in routes generated by more efficient routing
		algorithms.
		
		Real world super computer architectures are designed to ensure that faults
		are isolated \cite{gara05,alverson12} and thus tend to only impact a
		localised region of the network. Since PGS repair is only needed to route
		around these isolated faults, the space searched by the graph search
		algorithm should be very small in practice resulting in only short
		runtimes. In addition since faults are rare in real-world systems, the
		graph search process will only rarely be invoked.
		
		The PGS repair post-processing technique starts with a route produced by a
		non-fault-tolerant routing algorithm such as ESPR or NER. If this route is
		not obstructed by a fault, the algorithm terminates immediately without
		modifying the route. If the route attempts to use a faulty link, the
		algorithm proceeds as follows.
		
		The routing tree produced by the underlying routing algorithm is broken
		into subtrees wherever it attempts to route through a broken link and
		each subtree is assigned a unique colour, as illustrated in figure
		\ref{fig:pgs-repair-colouring}. From each disconnected subtree's root
		node in turn, a graph search is performed to find a short, fault-free
		route to a subtree node of a different colour. The subtree is then
		attached to the tree discovered by the graph search and re-coloured to
		match the tree it is connected to.
		
		\begin{figure}
			\center
			\begin{subfigure}{0.32\linewidth}
				\hspace*{-1.5em}
				\buildfig{figures/pgs-repair-colouring.tex}
				
				\caption{}
				\label{fig:pgs-repair-colouring}
			\end{subfigure}
			\begin{subfigure}{0.32\linewidth}
				\hspace*{-1.5em}
				\buildfig{figures/pgs-repair-colouring-fix1.tex}
				
				\caption{}
				\label{fig:pgs-repair-colouring-fix1}
			\end{subfigure}
			\begin{subfigure}{0.32\linewidth}
				\hspace*{-1.5em}
				\buildfig{figures/pgs-repair-colouring-fix2.tex}
				
				\caption{}
				\label{fig:pgs-repair-colouring-fix2}
			\end{subfigure}
			
			\caption{PGS repair process example showing a disconnected multicast
			route from A to B, C, D, E and F. $\times$ indicates a broken link.}
			\label{fig:pgs-repair-colouring-steps}
		\end{figure}
		
		For example in figure \ref{fig:pgs-repair-colouring-fix1} a path from the
		root of the subtree containing nodes E and F is found which connects it to
		the subtree rooted at A. Similarly in figure
		\ref{fig:pgs-repair-colouring-fix2} a path is also found connecting the
		subtree containing nodes C and D back to the subtree rooted at node A.
		
		If the routing tree was broken into $N+1$ subtrees by faults there will be
		$N$ subtrees disconnected from the root node. Each of the $N$ iterations of
		the algorithm connect a disconnected subtree to another subtree reducing
		the number of subtrees by $1$ each time. After $N$ iterations, therefore,
		exactly $1$ subtree remains which connects every node in the original
		routing tree without traversing faulty links.
		
		TODO: EXPLAIN THE FIDDLINESS HERE TO ENSURE WE DON'T CREATE LOOPS.
		
	\section{Evaluation \& Results}
		
		The PGS repair technique, by design, is able to work around all possible
		fault patterns which don't completely disconnect parts of the network. This
		result this evaluation focuses on the impact on performance PGS repair
		imposes. The metrics of interest in this evaluation are:
		
		\begin{itemize}
			\item Algorithm runtime
			\item Network congestion
			\item Routing table utilisation
		\end{itemize}
		
		\subsection{Traffic Patterns}
			
			In this evaluation, two standard benchmark multicast traffic patterns are
			used which have been used in previous research into SpiNNaker's network:
			
			\begin{figure}
				\center
				\buildfig{figures/traffic-distribution-centroids.tex}
				
				\caption{An example 4-centroid distribution with four centroids. The
				$\times$ marks the location of the origin node. Lighter colours
				indicate greater likelihood of a connection.}
				\label{fig:traffic-distribution-centroids}
			\end{figure}
			
			\begin{description}
				
				\item[Uniform] Destinations are chosen with uniform probability
				anywhere in the machine.
				
				\item[$N$-Centroids] Destinations are clustered around one of $N$
				randomly chosen `centroids' as illustrated in figure
				\ref{fig:traffic-distribution-centroids}.
				
			\end{description}
			
			The uniform traffic pattern is widely used in networks research
			\cite{dally04,davies12} while the centroids model was developed
			specifically to reproduce the traffic patterns found in the neural
			applications SpiNNaker is designed for \cite{navaridas14}. In this work
			we consider 3 centroids.
		
		\subsection{Fault model}
			
			In addition two different fault models are used which are representative of
			the faults found in real SpiNNaker systems:
			
			\begin{figure}
				\center
				\begin{subfigure}{0.48\linewidth}
					\hspace*{-1.5cm}
					\buildfig{figures/fault-example-uniform.tex}
					
					\caption{Uniform}
					\label{fig:fault-example-uniform}
				\end{subfigure}
				\begin{subfigure}{0.48\linewidth}
					\hspace*{-1.5cm}
					\buildfig{figures/fault-example-hss.tex}
					
					\caption{HSS Link}
					\label{fig:fault-example-hss}
				\end{subfigure}
				
				\caption{The two link fault models considered.}
				\label{fig:fault-example}
			\end{figure}
			
			\begin{description}
				
				\item[Uniform] Links are selected and disabled at random (figure
				\ref{fig:fault-example-uniform}).
				
				\item[HSS Link] Groups of links corresponding with randomly selected
				single High-Speed Serial (HSS) link between SpiNNaker boards are disabled
				together (figure \ref{fig:fault-example-uniform}).
				
			\end{description}
			
			The uniform link failure model models isolated failures resulting from
			isolated manufacturing defects in individual links. The HSS Link failure
			model models faults arising from failing or disconnected board-to-board
			links which carry several chip-to-chip traffic flows via a single cable in
			SpiNNaker systems. Though SpiNNaker-specific, the later fault model is
			analogous to failure modes arising in other architectures where a single
			fault may render several links impassable in a single area.
			
			A range of failure rates are explored in this section. My measurements of
			current large-scale SpiNNaker installations the link failure rate is about
			\SI{0.03}{\percent} with failures due to both individual chip-to-chip links
			and board-to-board HSS links. Exact link failure statistics for commercial
			super computer installations are not widely available, however, published
			Mean-Time-Between-Failure (MTBF) statistics place an upper bound on link
			failure rates at a similar \SI{0.03}{\percent} in one-year-old BlueGene/Q
			systems \cite{chiu11}.
			
			Unfortunately presently undiagnosed problem with the SDRAM packaged with
			approximately \SI{1}{\percent} of SpiNNaker chips has rendered these chips
			unusable for most applications. The gaps in the network resulting from the
			loss of these chips currently dominate true link failures leaving just over
			\SI{1}{\percent} of links inoperable.
			
			Surprisingly, research into fault tolerant routing in super computers
			appears to focus on benchmarks with even higher fault rates ranging from
			\SI{3}{\percent} to as high as \SI{7}{\percent}
			\cite{ho04,gomez04,mejia06}.
			
			In this evaluation, fault rates ranging from \SI{0.01}{\percent} to
			\SI{5}{\percent} are considered to cover both realistic fault levels
			along with the more extreme cases considered in related work.
		
		\subsection{Base routing algorithm}
			
			Since the PGS repair process is routing algorithm agnostic all
			experiments use the NER algorithm which has been found to be appropriate
			for SpiNNaker applications \cite{navaridas14}.
		
		\subsection{Algorithm runtime}
			
			To assess the impact of the PGS repair process on routing algorithm
			runtime, the algorithm was used to process a large number of randomly
			generated routing problems and the runtime recorded.
			
			\num{10000} one-to-sixteen multicast routing problems were generated in a
			$256\times256$ hexagonal torus topology, the largest size possible for a
			SpiNNaker system. Other quantities of multicast destinations were also
			evaluated but are omitted for brevity since the pattern of results are
			similar to those outlined here.
			
			TODO: APPENDIX WITH OTHER RUNS?
			
			The NER and PGS repair algorithms were written in C and compiled with GCC
			4.8.3 with \verb|-O2| level optimisations and executed on a cluster of
			idle workstations with 3.10 GHz Intel Core-i5-2400 CPUs.
			
			\begin{figure}
				\center
				\buildrplot{figures/routing-runtimes.R}
				
				\caption{Mean runtime of routing and PGS repair overhead. PGS repair
				overhead is stacked above the routing runtime (i.e. bars do not
				overlap). Error bars indicate 95\% confidence interval. Note different
				Y-scale for HSS link and uniform fault models.}
				\label{fig:routing-runtimes}
			\end{figure}
			
			Figure \ref{fig:routing-runtimes} shows the average runtimes recorded for
			both the NER and PGS repair algorithms. In fault-free networks the
			PGS-repair post-processing step is not required and incurs no penalty
			while the runtime of the algorithm grows with the fault rate for both
			fault and traffic models.
			
			Notably the HSS fault model results in longer runtimes for the PGS repair
			process compared with an equivalent fault-density of uniform faults.
			Because the HSS fault model produces contiguous lines of faults the PGS
			repair algorithm must construct a longer path to avoid the fault.  Since
			the space explored by a graph algorithm typically grows with $O(H^2)$
			with respect to the hops in the discovered route, $H$, this increase in
			search distance has a large impact on the runtime of the PGS repair
			process.
			
			The runtime of the PGS repair algorithm remains roughly in proportion to
			the runtime of the underlying routing algorithm with respect to different
			traffic models. The centroid traffic pattern tends to result in routes
			with fewer hops than a uniform traffic pattern with the same number of
			destination nodes as segments of routes are often shared between
			destination nodes. Since the NER algorithm's runtime is strongly related
			to the number of hops in the output route the runtime of the algorithm is
			greater for uniform traffic. Likewise the probability of PGS repair being
			required increases with the number of hops in route and hence the runtime
			of the PGS repair algorithm increases roughly in proportion.
		
		\subsection{Routing table usage}
			
			In order to gain a realistic measure of routing table usage it is
			necessary to determine the effect of many routes being generated for a
			single set of faults. To enable a sufficiently large number of sample to
			be collected the experimental setup considered previously is reduced to a
			network containing $48\times48$ nodes.
			
			\num{1000} $48\times48$ node network models are produced according to the
			HSS link and uniform fault models. For each of these models
			$48\times48\times16=$~\num{36864} one-to-sixteen routes are generated using
			the centroid and uniform traffic models. This corresponds to one
			multicast route per application core. As is convention in SpiNNaker,
			routing table entries are inserted for each route at the source of the
			route, at each destination and at each corner or fork. The number of
			routing table entries at each node in the model is counted and the
			maximum number of entries in a single node is reported for each network
			model.  The \emph{maximum} number of routing entries of any router was
			chosen since the number of entries available per SpiNNaker router is
			bounded by hardware.
			
			\begin{figure}
				\center
				\buildrplot{figures/routing-entries.R}
				
				\caption{Violin plot showing the distribution of maximum table sizes
				for \num{1000} random networks. The red line at \num{1024} entries
				indicates the size of SpiNNaker's routing tables.}
				\label{fig:routing-entries}
			\end{figure}
			
			
			Figure \ref{fig:routing-entries} shows the distributions of the largest
			routing table sizes for each fault and traffic model.
			
			\begin{figure}
				\center
				\begin{subfigure}{0.48\linewidth}
					\center
					\buildfig{figures/hss-link-routing-table-usage.tex}
					
					\caption{Routing table entries}
					\label{fig:hss-link-routing-table-usage}
				\end{subfigure}
				\begin{subfigure}{0.48\linewidth}
					\center
					\buildfig{figures/hss-link-resource-usage.tex}
					
					\caption{Routes passing through chip}
					\label{fig:hss-link-resource-usage}
				\end{subfigure}
				
				\caption{The impact of a HSS link fault on routing table usage and
				congestion. Each hexagon represents a single chip, the red line
				indicates the chip-to-chip connections broken by the HSS link fault.}
				\label{fig:hss-link-usage}
			\end{figure}
			
			The HSS link failure model has a much greater impact on peak routing
			table resource usage than uniform link failures for a given fault rate.
			This is because HSS link faults result in a large concentration of routes
			being disrupted and then re-routed around the same obstacle in a single
			location. Figure \ref{fig:hss-link-routing-table-usage} shows how routing
			table usage varies around a HSS link fault in one instance of the
			experiment. There are clear peaks in routing table usage around the ends
			of the line of faults which result from routes produced by PGS repair
			finding shortest paths around the edge of the faults.
		
		\subsection{Network congestion}
			
			To measure the impact of PGS repair on network congestion, two
			experiments were performed, one using the same model used to measure
			routing table usage and one based on tests run on SpiNNaker hardware.
			
			For each of the network fault and traffic pattern described previously,
			the paths taken for the \num{36864} one-to-sixteen multicast routes
			generated are used to compute the number of times each link in the
			network is used. The number of routes passing through the most-used link
			is then recorded, giving an indication of the level of congestion in the
			network.
			
			\begin{figure}
				\center
				\buildrplot{figures/routing-resource.R}
				
				\caption{Violin plot showing the distribution of maximum
				routes-per-chip for \num{1000} random networks.}
				\label{fig:routing-resource}
			\end{figure}
			
			The results are presented in figure \ref{fig:routing-resource} and follow
			the same trends as the results previously shown for routing table usage.
			Again, HSS link faults result in routes with the greatest congestion due
			to the concentration of routes finding shortest paths around an obstacle
			(see \ref{fig:hss-link-resource-usage}).
			
			To verify that the results above, an additional experiment has been
			carried out which attempts to mimic the model used previously in actual
			SpiNNaker hardware. In these experiments a large SpiNNaker machine is
			divided into independent 48-board (2304-chip) sections. Because the
			48-board systems used in these experiments are cut out of a larger
			machine, they lack wrap-around links and thus form hexagonal mesh
			topologies, rather than hexagonal toruses.
			
			Due to the SDRAM issue described above, fault rates below
			\SI{1}{\percent} cannot be modelled.  To simulate higher fault rates,
			additional links are disabled in software according to the fault models
			described used previously. Since some faults are due to genuine hardware
			faults, these faults cannot be placed randomly in each experiment. To
			reduce, bias each combination of fault rate, fault model and traffic
			pattern is repeated XXX times across randomly chosen physical machines.
			
			XXX 1-to-XXX routes are generated in both uniform and XXX-centroid
			distributions as used throughout this evaluation. Synthetic network
			traffic is generated at the source of each route following a Bernoulli
			distribution. Traffic consumers running on all destination nodes accept
			packets as quickly as possible from the network and log their arrival.
			The Bernoulli probability is set the same for every route's traffic
			generator and increased in steps of XXX and the number of packets dropped
			in an XXX second period logged. The network is considered saturated once
			less than \SI{99}{\percent} of packets successfully arrive at their
			destination.
			
			Figure \ref{XXX} shows the distributions of the saturation points for
			each experimental configuration.
			
			TODO: ANALYSIS
		
	\section{Conclusions}
		
		In this chapter I described how SpiNNaker's unconventional network and
		router architecture render existing fault tolerant routing algorithms
		unsuitable. I introduced PGS repair, a post-processing technique for
		existing non-fault tolerant routing algorithms designed for SpiNNaker such
		as NER.
		
		Unlike some other fault tolerant routing algorithms for other
		architectures, PGS repair is able to work-around arbitrary fault patterns
		by exploiting SpiNNaker's inbuilt deadlock avoidance mechanisms. In the
		presence of realistic failure rates of up to \SI{1}{\percent}, only small
		overheads of up to XXX, XXX and XXX for in algorithm runtime, routing table
		usage and network performance are incurred respectively. This low
		performance overhead makes PGS repair appropriate for use in real
		applications. At the time of writing the algorithm has been successfully
		used in a number of neural and non-neural SpiNNaker applications.
		
		At more extreme fault rates not expected in real-world systems, the
		algorithm still functions correctly but the results incur much greater
		routing table and congestion overheads, particularly when faults are
		concentrated. Future extensions to this algorithm might aim to reduce this
		overhead by producing longer and more varied routes around faults to even
		out the load.

	\chapter{Placing applications in large SpiNNaker machines}
	
	In the previous chapter I tackled the problem of scale in generating routes
	for very large networks such as SpiNNaker. In this work the centroid traffic
	pattern was used as an approximation of the expected network traffic
	generated by `well behaved' neural network simulation software running on
	SpiNNaker. The traffic produced largely exhibits strong locality, that is
	most communication occurs between either nearby nodes or clusters of nodes.
	In reality, neural simulation applications are not specified geometrically
	but rather as abstract graphs of communicating neurons
	\cite{davison08,eliasmith13}. Applications must then \emph{place} these
	neurons onto nodes in a SpiNNaker system, attempting maximise communication
	locality.
	
	In this chapter I re-evaluate the suitability of simulated annealing as a
	technique for finding high quality placements for large parallel
	applications. Though this technique had fallen out of fashion in the field of
	application placement by the early 1990s, it has found wide use for placing
	components in computer chip and FPGA designs. In the intervening years,
	placement problems in super computers have grown in size from tens or
	hundreds of nodes to millions, a scale at which chip placement techniques
	were operating in the mid 1990s. I adapt the simulated annealing algorithm
	used by the VPR academic circuit placement software to produce placements for
	applications running on SpiNNaker. In that in a range of real and synthetic
	benchmarks simulated annealing produces high quality placements enabling
	efficient use of SpiNNaker's network resources.
	
	
	%In the field of chip design, Moore's `Law' \cite{moore65,moore75} observes a
	%similar exponential growth in the number of components within a single chip.
	%Today modern processors contain billions of components and an analagous
	%placement problem exists in attempting to place interconnected components
	%near to eachother. In this chapter I explore the techniques used for circuit
	%placement and adapt one such technique, Simulated Annealing (SA)
	%\cite{kirkpatrick83}, for use in application placement. Despite some early
	%interest in SA for application placement in the 1980s and early 1990s, the
	%technique has since fallen out of favour. I find that at the scales of modern
	%placement problems SA-based placement is able to produce solutions of
	%superiour quality to contemporary methods.
	%
	%TODO: SUMMARISE RESULTS...
	
	\section{Related work}
		
		The placement problem has been tackled independently in the literature by
		researchers in both the application and chip placement communities. In this
		survey I cover application and chip placement separately as these two
		communities have remained largely isolated from one another. First I
		explore the techniques applied to application placement before moving on to
		contrast this with the techniques used in circuit placement.
		
		In the application placement literature, the placement problem is often
		referred under the umbrella term `mapping'. Unfortunately term is often
		used more broadly to include other tasks such as routing and application
		partitioning. To avoid ambiguity I use the term `placement', as preferred
		by the chip and FPGA design communities, to refer specifically to the
		problem of assigning nodes in an application's communication graph to nodes
		in a machine's connectivity graph.
		
		\subsection{Application placement algorithms}
			
			TODO: GENERAL INTRO
			
			\subsubsection{Application-specific approaches (manual placement)}
				
				In the case of some applications such as finite element modelling
				\cite{bermejo13}, the structure of the problem itself leads to a
				natural placement of the computation on nodes in a machine. For example
				when simulating a 3D volume in an node super computer with a $3 \times
				4 \times 2$ 3D torus or mesh topology network, the modelled volume
				might be divided into as in figure \ref{fig:fem-partitioning}. Each
				cuboid in the model is then assigned to the corresponding node in the
				network topology.
				
				\begin{figure}
					\center
					\buildfig{figures/fem-partitioning.tex}
					
					\caption{Example partitioning of a 3D space to fit into a super
					computer with a $3\times4\times2$ torus or mesh topology.}
					\label{fig:fem-partitioning}
				\end{figure}
				
				When the number of dimensions in a problem do not match that of the
				underlying network architecture, the common solution is to either
				divide only along a subset of the axes or to divide into additional
				pieces on the existing axes \cite{gilge14}.
			
			\subsubsection{Sequential placement}
				
				In the case where a placement solution is non-obvious one of the
				simplest and most popular strategies is to apply a simple sequential
				placement algorithm. Sequential placement algorithms function by
				iterating over the vertices in the application's communication graph
				and assigning them to a free node in the target machine. Sequential
				placement algorithms are differentiated by the order in which they
				iterate over vertices in the communication graph and fill nodes in the
				target machine. A number of widely used orderings are described below.
				
				\begin{figure}
					\center
					\begin{subfigure}{0.32\linewidth}
						\center
						\buildfig{figures/sequential-row-order.tex}
						\caption{Row-order}
						\label{fig:sequential-row-order}
					\end{subfigure}
					\begin{subfigure}{0.32\linewidth}
						\center
						\buildfig{figures/sequential-alternating.tex}
						\caption{Alternating}
						\label{fig:sequential-alternating}
					\end{subfigure}
					\begin{subfigure}{0.32\linewidth}
						\center
						\buildfig{figures/sequential-hilbert.tex}
						\caption{Hilbert curve}
						\label{fig:sequential-hilbert}
					\end{subfigure}
					
					\caption{Space-filling curves in 2D mesh and torus topologies.}
					\label{fig:sequential}
				\end{figure}
				
				Super computer management software such as SLURM \cite{yoo03} and Blue
				Gene's system software \cite{gilge14} by default na\"ively iterate over
				vertices in an application communication graph in the order they are
				provided. The nodes in the target machine are then iterated over in a
				simple space-filling curve through the network topology. Figure
				\ref{fig:hilbert-placement} illustrates the default patterns available
				in these software packages. The row-order (figure
				\ref{fig:sequential-row-order}) and alternating (figure
				\ref{fig:sequential-alternating}) curves correspond with 2D versions of
				the default node assignment orders used in SLURM and BlueGene systems.
				
				\begin{figure}
					\center
					\buildfig{figures/hilbert-placement.tex}
					
					\caption{A Hilbert curve, coloured from blue to red.}
					\label{fig:hilbert-placement}
				\end{figure}
				
				The Cray extensions to SLURM software provide a Hilbert curve
				\cite{hilbert91} (figure \ref{fig:sequential-hilbert}) node assignment
				order. Unlike the row-order and alternating space filling curves the
				Hilbert curve ensures that pairs of vertices close together in the node
				iteration order are also close together in the target machine's network
				\cite{moon01, zumbusch99}. Figure \ref{fig:hilbert-placement} shows a
				5$^\textrm{th}$-order Hilbert curve where each point in the curve is
				coloured according to its position along the curve. In this figure it
				is possible to see that nearby positions in the curve (which share
				similar colours) are also close in 2D space.
				
				When the proximity of vertices in the vertex-ordering supplied by an
				application is a good estimator of those vertices communication
				requirements, the sequential assignment schemes discussed above can be
				very effective. These techniques have also proven adequate in
				small-scale and densely connected applications such as early neural
				simulations running on prototype SpiNNaker machines with tens of nodes
				\cite{galluppi10} but growing beyond this scale has proven problematic.
				
				\begin{figure}
					\center
					\begin{subfigure}{0.45\linewidth}
						\center
						\buildfig{figures/rcm-initial.tex}
						
						\caption{Original permutation}
						\label{fig:rcm-initial}
					\end{subfigure}
					\begin{subfigure}{0.45\linewidth}
						\center
						\buildfig{figures/rcm-sorted.tex}
						
						\caption{RCM permutation}
						\label{fig:rcm-sorted}
					\end{subfigure}
					
					\caption{Adjacency matrix representation of a graph before and after
					permutation by the RCM algorithm.}
					\label{fig:rcm}
				\end{figure}
				
				A number of algorithms have been proposed for automatically selecting
				good vertex iteration orders, typically using a graph-traversal based
				heuristic. A typical method, described by Hoefler \emph{et al.}
				\cite{hoefler11} exploits the Reverse-Cuthill-McKee (RCM) algorithm
				\cite{cuthill69}. An application's communication matrix is represented
				as an adjacency matrix, $M$, where $M_{i,j}$ is 1 if node $i$ is
				connected by an edge to node $j$ and 0 otherwise. An example matrix is
				illustrated in figure \ref{fig:rcm-initial}. The RCM algorithm uses a
				simple heuristic to permute the matrix (i.e. renumber the nodes in the
				graph) in order to reduce the bandwidth of the matrix. Figure
				\ref{fig:rcm-sorted} shows the RCM-permuted version of the example
				adjacency matrix. When a graph's vertices are ordered as in a
				bandwidth-reduced sparse matrix, vertices close together in the
				ordering are likely to communicate while those further apart tend not
				to communicate.
				
			\subsubsection{Optimisation-based Placement}
				
				% Citations from short report about optimisation in placement...
				% \cite{chen06,jeannot14} and \cite{jeannot10} ("subsets of apps")
				
				In the academic community, a number of attempts have been made to use
				more sophisticated optimisation algorithms for the placement of
				applications. In 1985, Steele \cite{steele85} proposed the use of
				simulated annealing for placing applications in the 6D torus topology
				of the 64 node `Caltech Cosmic Cube' machine. Simulated annealing,
				originally developed by Kirkpatrick \emph{et al.} \cite{kirkpatrick83},
				is a general-purpose optimisation algorithm which works by analogy to
				the physical process of annealing. In brief simulated annealing
				functions by randomly swapping vertices in a candidate placement
				solution, accepting swaps which move connected vertices closer together
				and rejecting some proportion of swaps which move connected vertices
				further apart. The simulated annealing algorithm is described in detail
				later in this chapter.
				
				Towards the end of the 1980s, application placement appeared to be
				becoming less important as super computer network architectures
				improved:
				%
				\begin{displayquote}
					``Careful placement was necessary because of the slow communication
					and non-uniform addressing of early concurrent computers. However,
					the development of message passing machines with fast communications
					and a uniform global address space  has made placement less of an
					issue. In such machines a random placement performs nearly as well as
					an optimum placement.''
					
					\noindent --- W. Dally, 1987 \cite{dally87}
				\end{displayquote}
				%
				In addition, network and problem sizes remained small, so small in fact
				that linear-programming based optimal placement still appeared in
				benchmarks comparing placement algorithms \cite{xu91}. In this
				environment, simpler sequential placement algorithms gained favour over
				more computationally expensive algorithms such as simulated annealing.
				
				As problem and machine sizes have grown and network utilisation has
				once again become an important factor in application performance
				\cite{navaridas09b} more complex optimisation algorithms have
				reappeared in the literature. One popular approach employs graph
				partitioning algorithms such as METIS \cite{karypis98} to perform
				recursive bipartitioning based placement
				\cite{phillips14,hoefler11,pellegrini96}.  This placement process is
				illustrated in figure \ref{fig:partitioning}.
				
				In the first step, the application communication graph and machine
				connectivity graph are bipartitioned such that the number of edges
				between partitions is minimised. Each half of the communication graph
				is associated with one of the halves of the machine connectivity graph.
				The partitioning process is then repeated recursively on each of the
				two communication and connectivity graph pairs. The process halts when
				the graphs can no longer be partitioned at which point the vertices in
				the communication graph are placed on their associated node.
				
				\begin{figure}
					\center
					\buildfig{figures/partitioning.tex}
					
					\caption{Illustration of application placement by recursive
					partitioning.}
					\label{fig:partitioning}
				\end{figure}
				
				TODO: PARTITIONING IS GREAT AND ALL BUT QUALITY ISN'T ALWAYS GREAT AND
				IT DOESN'T DEAL WELL WITH MULTI-CONSTRAINT SCENARIOS E.G. PROCESSOR AND
				MEMORY RESTRICTIONS.
				
				Unfortunately, many of these simply aren't suited to the scale of
				neural applications running on SpiNNaker (e.g. only cope with tens of
				nodes while SpiNNaker may contain hundreds of thousands).
				
				Additionally, a number of algorithms have been developed which make
				assumptions about the topologies of the problem or network. Tree match
				for example attempts to map tree-shaped problems to tree-shaped
				networks. Such algorithms can be highly effective but again do not
				apply to SpiNNaker or its neural applications.
		
		\subsection{Chip placement algorithms}
			
			The chip-design industry has, for many years, dealt with problems
			analogous to the task of placing super computer jobs in a way suited to
			SpiNNaker. Modern CPUs have millions or billions of components with
			strictly fixed connectivity. CPU designers must place each of these onto
			a chip such that the connection lengths are controlled to reduce
			congestion and increase performance. As such, these algorithms are
			ideally suited to future super computer placement work since they already
			operate at the scales required \cite{nam07}.
			
			\subsubsection{Cost functions}
				
				HPWL is popular but a bit crap for high fan-outs. It is, however, quite
				simple.
				
				TODO: SELECT A BETTER COST FUNCTION...
			
			\subsubsection{Simulated annealing}
				
				One of the oldest techniques used for circuit placement is simulated
				annealing and this remains popular today thanks to its sheer
				versatility (see VPR, other open FPGA tools).
				
				SA works by analogy with the physical process of annealing.
				The simulated annealing algorithm works by selecting random pairs of
				components on a chip, swapping them and evaluating some cost function.
				If the swap reduces the cost function, it is kept, if not, depending on
				a function of the current temperature and the cost introduced by the
				swap.
				
				TODO: ILLUSTRATION OF SIMULATED ANNEALING SWAP OPERATION
				
				By occasionally allowing costly swaps, the annealing algorithm avoids
				becoming trapped in local minima. As the algorithm proceeds, the
				temperature is slowly reduced and with it the proportion of costly
				swaps which are retained. This causes the placement to move from
				exploration early on towards refinement later on.
				
				The temperature schedule of an annealing algorithm is critical to its
				success. In general these schedules are computed based on the
				performance of the algorithm as it runs. In VPR the following schedule
				is used.
				
				TODO: DESCRIBE VPR'S SCHEDULE
				
				TODO: FIND AND DESCRIBE ALTERNATIVE SCHEDULE?
				
				Unfortunately, SA is very difficult to parallelise, especially in the
				case of placement. As a result, its scalability has been limited and
				resulted in significantly reduced usage in recent work.
			
			\subsubsection{Partitioning placement}
				
				Partitioning based placement solves the placement problem using
				graph-partitioning recursively on the problem graph to assign each part
				of the circuit to some area in the super chip. Though a number of
				algorithms have proven successful in academic placement contests over
				the years, they are not popular in industrial settings.
			
			\subsubsection{Analytical placement}
				
				In analytical placement, cost function for the circuit graph is
				approximated in a form which is amenable to solutions with standard
				numerical or symbolic algebraic techniques. Using these techniques,
				exact minimum cost (in terms of the approximation) configurations can
				be obtained.
				
				Quadratic placement is a popular analytical placement technique which
				approximates the cost of a placement as the sum of the squares of the
				distances between connected circuit elements.
				
				TODO: FIGURE EXAMPLE QUADRATIC PLACEMENT PROBLEM AND SOLUTION
				
				As such this gives a quadratic cost function like so which we must
				minimise.
				
				TODO: QUADRATIC COST EQN
				
				To minimise the function we differentiate and solve using simple
				symbolic manipulation.
				
				TODO: QUADRATIC COST SOLUTION
				
				Unfortunately, quadratic placement doesn't contain any congestion
				relief by default so various schemes exist. For example, extra anchor
				nodes are inserted which gently pull the circuit components apart from
				each other. As a result, the algorithm generally proceeds by iterating,
				regenerating anchors each time.
				
				Other non-quadratic analytical methods exist too with numerical
				solutions. The approaches are often similar.
			
			\subsubsection{Hierarchical clustering}
				
				Many placement algorithms scale super-linearly with problem size and so
				larger problems become increasingly problematic to handle. To solve
				this problem clustering techniques are first applied to first simplify
				the placement problem. A solution is then found at the coarse level and
				then hierarchically fleshed out.
				
				Various clustering algorithms are in use.
				
				TODO: TALK ABOUT CLUSTERING IN PLACEMENT...
				
				TODO: DESCRIBE THE ALGORITHM I IMPLEMENTED.
	
	\section{Application placement by simulated annealing}
		
		\label{sec:placement-by-annealing}	
		
		I have implemented a simplified SA based application placement algorithm
		based on the approach used in the popular VPR place and route tool chain.
		The algorithm is written in C and is optimised for experimentation rather
		than performance but is production-ready. It has been integrated into the
		`Rig' SpiNNaker software tools and has been used to place very large
		simulations. More on that later.
		
		\subsection{Representation}
			
			Model each chip as having a quantity of various resources (e.g. Cores,
			SDRAM) available. The application graph consists of vertices which each
			consume some quantity of these resources. Vertices must be placed on a
			single chip such that the resources required on a given chip do not
			exceed those available. Vertices are then interconnected by 1:N nets with
			weights which act as hints. The nets are treated as a soft constraint:
			vertices connected via a net will, ideally, be placed near to each other,
			with priority being given to nets with higher weights. Additionally there
			will be a list of placement constraints (see later).
			
			A key observation is that while vertices in an application may frequently
			have a 1:1 correspondence with application cores, this need-not be the
			case. For example, a vertex may represent a block of SDRAM which is
			shared. A vertex may also represent some other resource, for example,
			external IO availability. By making these resource types user-defined,
			applications programmers can express flexible hard-constraints on their
			application.
			
			Another observation is that generic soft constraints can be expressed may
			be expressed using a net with an appropriate weight.
			
			As a result of these facilities, application programmers can easily
			express their own application-specific hard and soft placement
			constraints without having to modify the algorithm. This representation
			has become a de-facto standard for placement problem interchange for
			SpiNNaker applications.
		
		\subsection{Cost function}
			
			At present I've used HPWL despite this being really bad for high-fan-out
			multicast and totally ignorant to the hexagonal nature of SpiNNaker...
			
			To compute bounding boxes for tori I use the following approach. For each
			dimension, sort the points on that dimension and find the largest gap
			between them on a ring. The bounding box goes the other way.
			
			TODO: FIGURE ILLUSTRATING BOUNDING BOX COMPUTATION FOR TORI.
		
		\subsection{Annealing schedule}
			
			The annealing schedule is that used by VPR. Despite being for circuit
			placement, it seems to work jolly well.
			
			TODO: DESCRIBE AND RATIONALISE THE SCHEDULE
		
		\subsection{Constraint handling}
			
			Various hard and soft constraints may be expressed by software
			approaches. For each we explain how they may be handled by the placement
			algorithm:
			
			\subsubsection{Location Constraint}
				
				The vertex is placed on a chip and removed from the set of movement
				candidates.
			
			\subsubsection{Same-chip constraint}
				
				When two vertices must always be placed on the same chip they are
				simply combined into one vertex which consumes the sum of their
				resources. Placement then treats them as one chip and thus is forced to
				atomically place the vertices.
			
			\subsubsection{Reserve resource constraint}
				
				Simply reduce resource availability on that chip.
			
			\subsubsection{Keep near Ethernet}
				
				Simply add a net.
	
	\section{Evaluation}
		
		\label{sec:placement-results}
		
		Though benchmarks exist for super computer loads and chip placement tasks,
		such things don't exist for neural applications. As a result I use a
		selection of real applications for SpiNNaker along with some synthetic
		benchmarks based on biological data.
		
		\subsection{Benchmark networks}
			
			First some real networks.
			
			Some nengo networks: SPAUN: `The world's largest functional brain model'.
			Word-net network from Jamie: Example of some learning.
			
			TODO: DESCRIBE SHAPE OF NENGO NETWORKS
			
			Some PyNN networks: Microcortical column model from PyNN. Note almost
			broadcast connectivity but varying weights. Try and extract a vision
			netlist from Anna. Maybe try and get a netlist for Tom's barrel cortex.
			
			Now for some artificial networks. Pipeline, noisy pipeline, mesh,
			Gaussian 2D.
		
		\subsection{Experiments}
			
			We compare random, linear, greedy and annealing based placement
			approaches to placement. We compare static metrics (such as mean/max
			congestion, table usage) along with experiments based on simulated
			network traffic in real hardware. Network Tester generates artificial
			traffic in proportion with the weights given for each model. We compare
			the relative level of traffic sustainable. We also consider use of
			machines of various sizes.
		
		\subsection{Results}
			
			SA placement is slow but rather effective, especially for some networks.
			Generally worth doing. Will need to be sped up for very large machines...
			
			TODO: EXPERIMENTS!
	

	\chapter{Discussion}

\section{Suitability of the hexagonal torus topology}
	\subsection{Physical scalability}
	\subsection{Routability}
	\subsection{Placeability}

\section{Suitability of the SpiNNaker router}
	\subsection{Deadlock avoidance}
	\subsection{Routing table size}

\section{Suitability of circuit placers for application placement}
	\subsection{Quality}
	\subsection{Runtime}
	\subsection{Routing resources}
	\subsection{Flexibility}
	\subsection{Scalability}


	\chapter{Future research}
	
	In this thesis I have presented a number of new techniques which have made it
	possible to assemble and operate the SpiNNaker super computer. This work
	opens up a range of possibie lines of research to extend this work to future
	architectures and applications. In this chapter I focus on two anticipated
	challenges of future systems: growing scale and greater dynamicism in
	applications.
	
	\section{Scaling up}
		
		TODO: INTRO
		
		\subsection{Grid machine room layouts}
			
			In chapter XXX, I developed a machine room layout for hexagonal torus
			topologies which allowed machines occupying a row of standard
			machine-room cabinets to scale up without the need for long
			interconnecting cables. For larger installations, however, it will be
			necessary to employ multiple rows of cabinets in a 2D arrangement.
		
		\subsection{Routing congestion control}
		
		\subsection{Parallel place and route}
	
	\section{Structural plasticity and dynamic fault-tolerance}
		\subsection{Plasticity models}
		\subsection{Incremental placement}
		\subsection{Incremental routing}
		\subsection{Hot-spare routes}

	\chapter{Conclusions and future research}
	
	The SpiNNaker architecture was designed to tackle the challenges presented by
	the simulation of biologically realistic neural networks. One of its
	distinguishing features is its network architecture which employs both an
	unconventional network topology and multicast router architecture. The
	hexagonal torus topology used by SpiNNaker was chosen to enable greater
	performance while maintaining ease of construction and scalability compared
	with conventional network topologies. SpiNNaker's router design centres
	around packets mimicking the neural `spike' signals they are designed to
	convey by being small, multicast and not guaranteed to arrive at their
	destination.  This novel design, though largely complete before this work
	began, left a number of open problems which this thesis has attempted to
	address.
	
	In this concluding chapter I begin by summarising the answers to the research
	questions raised in chapter~\ref{sec:introduction}. This is followed by a
	discussion of new research topics which have been uncovered by this work.
	
	\section{Answers to research questions}
		
		Each of the three research questions are answered below.
		
		\subsubsection{1. Can the hexagonal torus topology be deployed and used in
		real, large-scale systems?}
		
		In chapter~\ref{sec:building}, I introduced a cabling scheme and assembly
		technique which has been used successfully to build a prototype SpiNNaker
		system with over half a million processor cores using the hexagonal torus
		topology. The techniques shown are expected to enable a final SpiNNaker
		machine of double this size to be built, filling a six metre long row of
		machine-room cabinets.
		
		Though SpiNNaker's processor-count places it amongst some of the world's
		largest supercomputers (see figure \ref{fig:top500-num-processors} on page
		\pageref{fig:top500-num-processors}), it is comparatively compact, filling
		one row of cabinets compared with the warehouse-scale installations found
		in commercial systems. In spite of this, the folding and interleaving
		techniques described allow hexagonal torus topologies to scale to
		arbitrarily large installations without cables which span the machine.
		
		Chapter~\ref{sec:shortestPaths} described an efficient and general
		technique for finding, and enumerating shortest path vectors in hexagonal
		torus topologies. These developments bring the hexagonal torus topology in
		line with other topologies by enabling routing algorithms to exploit all
		possible paths in a network. Further, chapter~\ref{sec:placement}
		demonstrated that placement algorithms are also adaptable to hexagonal
		torus topologies thanks to their similarity to 2D toruses.
		
		Though, as this thesis highlights, hexagonal toruses lack many of the
		intuitive properties enjoyed by other topologies, it is still possible to
		reason about them with only limited computational effort.  Now that the
		practicality and scalability of the topology has also been demonstrated in
		practice, it represents a credible option for future network architectures.
		
		\subsubsection{2. Does SpiNNaker's router architecture help, or hinder
		fault tolerance?}
		
		SpiNNaker's unconventional use of packet dropping to avoid deadlocks
		greatly simplifies the router architecture, part of the motivation for this
		design. In chapter~\ref{sec:routing} this feature is used to the advantage
		of PGS repair to add fault tolerance to existing routing algorithms.
		Compared with the often complex and wasteful methods used to tolerate
		faults in other networks, PGS repair incurs very little performance
		overhead in the presence of static faults.
		
		Routing table usage does increase in the presence of faults, however, which
		may be a concern for applications which already require many routing table
		entries. Routing table usage, as well as other overheads, were most
		significantly increased in the presence of contiguous groups of network
		faults. This is because the PGS repair algorithm produces routes which pass
		tightly around the corners of faults, resulting in concentrations of
		routing table entries in those areas.  Though the symptoms of this problem
		can be attributed to the design of SpiNNaker's multicast routing mechanism,
		the responsibility lies with the behaviour of the PGS repair algorithm.
		Potential improvements to the PGS repair algorithm are discussed later in
		\S\ref{sec:pgs-repair-improvements}.
		
		The overall answer to this research question, therefore, is that the
		flexibility provided to routing algorithms in SpiNNaker's architecture is
		of great benefit, enabling arbitrary fault patterns to be inexpensively
		avoided.
		
		\subsubsection{3. How can the parts of a neural simulation be placed onto a
		large hexagonal torus topology to reduce network load?}
		
		In chapter~\ref{sec:placement}, I explored a number of contemporary
		approaches to the problem of placing applications with irregular
		communication patterns into network topologies. I observed that researchers
		working on circuit placement for chips and FPGAs are tackling similar
		problems and working at scales as large, or larger than, those faced in
		application placement. Based on this I developed a
		simulated annealing based placement algorithm inspired by the techniques
		used in circuit placement, with specific adaptations for use in application
		placement and SpiNNaker's network topology.
		
		The simulated annealing based placement algorithm consistently outperforms
		pre-existing placement algorithms included in benchmarks in terms of
		placement quality.  In the case of one benchmark, simulated annealing based
		placement made it possible to run that neural simulation in real-time for
		the first time.  At larger scales, simulated annealing was also found to be
		able to produce good quality placements in benchmarks containing over one
		million processes -- the largest size supported by the SpiNNaker
		architecture.
		
		The major shortcoming of simulated annealing based placement is its
		execution speed. Though its execution time grows in proportion to the size
		of the problem, the implementation used took over 12~hours to place a
		synthetic problem for the largest planned SpiNNaker machine. Though
		tractable -- particularly given the relative output quality compared with
		the prior state-of-the-art -- the algorithm is unlikely to function
		comfortably as-is on larger problems.
		
		The conclusion to be drawn from this result, however, is not just that
		simulated annealing is a good solution for today's placement problems but
		that circuit placement techniques in general could be successfully adapted
		to fulfil this role. The placement problems faced by chip designers are
		growing at roughly the same exponential rate as the size of super computers
		but circuit designs hold the lead in terms of problem size. Consequently,
		as approaches are retired by chip placement researchers, they may find new
		life in the field of application placement.
		
	\section{Future research}
		
		Though the goals of this study have largely been met, there also remain
		some important limitations which future work may hope to address.
		Furthermore, this work has uncovered a number of new research areas
		warranting future enquiry. This section outlines a number of future lines
		of research.
		
		\subsection{Warehouse-scale cabling}
			
			In chapter~\ref{sec:building} I developed and implemented a number of
			cabling schemes for the SpiNNaker architecture spanning up to a six metre
			row of machine-room cabinets -- a relatively small installation by
			current standards. In SpiNNaker, the cabling exists in a 2D plane (i.e.
			across the faces of the cabinets) but as the system is scaled up, a
			single row of cabinets will tend towards a 1D line. Since embedding a 2D
			structure in a 1D space necessarily results in long connections, this
			cannot scale indefinitely.
			
			\begin{figure}
				\center
				\buildfig{figures/multi-row-cabling.tex}
				
				\caption{Multiple rows of interconnected cabinets.}
				\label{fig:multi-row-cabling}
			\end{figure}
			
			In conventional large-scale super computer installations, nodes are
			installed in rows of cabinets as illustrated in
			figure~\ref{fig:multi-row-cabling}.  From a `bird's-eye' view, the system
			approximates a 2D space, spread across the floor of a machine-room.
			Therefore, in principle, the folding and interleaving techniques
			described in chapter~\ref{sec:building} still apply. Unfortunately for
			SpiNNaker, cables connecting between rows of cabinets would be longer
			than the one metre limit imposed by its hardware because of the spacing
			between rows of cabinets.  Future SpiNNaker systems will need to consider
			alternative link technologies.  For example, a hybrid system could be
			used in which intra-cabinet connections continue to use the current HSS
			link technology while inter-cabinet links might use optical connections.
			This type of architecture could be supported by the use of pluggable
			`SFP+' transceiver modules~\cite{sff01}.
		
		\subsection{Cabling assistance for other architectures}
			
			A secondary result of the construction of prototype SpiNNaker systems in
			chapter~\ref{sec:building} was the use of real-time guidance and feedback
			to assist cable installation. I am not aware of this technique's use by
			existing architectures and, following the success experienced in this
			project, it is possible that the technique may also be useful in
			conventional systems.
			
			During the construction of prototype SpiNNaker machines, each cable took
			seconds to install compared with the minutes reported for existing
			systems~\cite{mudigonda11}. Part of this increase in efficiency appears
			to result from the immediate identification of mistakes made during
			cabling, saving time-consuming backtracking later on.
			
			In many real-world network installations, units are less densely packed
			than in SpiNNaker and so longer cables are often required. As a
			consequence, cabling errors may become more likely as cabling patterns
			are spread over a larger area making them more difficult to visually
			verify. Like SpiNNaker, conventional networking hardware is often
			equipped with a generous range of indicator LEDs and diagnostic
			facilities which might be used to implement real-time installation
			guidance. Future work could explore the use of this technique in the
			construction of other large-scale networks, such as data centres.
		
		\subsection{Congestion mitigation}
			
			\label{sec:wiggly-board-allocations}
			
			In chapter~\ref{sec:routing} I found that contiguous network faults cause
			hot-spots of congestion and routing table depletion where the PGS repair
			algorithm routed many paths around the edges of faults.  However, it is
			not just faults which can cause contiguous blockages in the network
			topology. In reality, researchers do not always require a full-sized
			SpiNNaker system to perform their experiments so large SpiNNaker systems
			are soft-partitioned on demand into many smaller
			machines~\cite{spalloc16}. To ensure isolation between partitioned
			sub-machines, HSS links between boards in different partitions are
			disabled. Because of SpiNNaker's `wrapped triple' partitioning scheme,
			the resulting sub-machines have hexagonal \emph{mesh} topologies (i.e.
			without wrap-around links) with irregular boundaries as in
			figure~\ref{fig:spalloc-mesh}.
			
			\begin{figure}
				\center
				\buildfig{figures/spalloc-mesh.tex}
				
				\caption[Irregular edges of a partitioned SpiNNaker system.]%
				{Irregular edges in a SpiNNaker system comprised of 24~boards
				partitioned from a larger machine.  Each hexagon represents a SpiNNaker
				chip. No wrap-around connections are present.}
				\label{fig:spalloc-mesh}
			\end{figure}
			
			In partitioned systems, the `tooth'-like gaps on the periphery of the
			network result in similar congestion to the HSS link failures considered
			in chapter~\ref{sec:routing}. When a route is generated between nodes on
			opposite sides of a gap, the PGS repair process will produce a
			shortest-path route around it. Since many routes may be blocked by a
			single gap, a hot-spot may develop around the corners of the gap.
			
			In chapter~\ref{sec:placement}, the `CConv' benchmark application was
			found to run correctly the majority of the time when placed by the
			simulated annealing algorithm but would occasionally fail by a
			significant margin. Preliminary experiments suggest these occasional
			failures are caused by placement solutions which place heavily
			communicating parts of the application on opposite sides of gaps along
			the perimeter of the network. Two possible approaches which future work
			may consider are presented below.
			
			\subsubsection{Avoiding hotspots with PGS repair}
				
				\label{sec:pgs-repair-improvements}	
				
				Network congestion around faults and network irregularities could be
				reduced by forcing the PGS repair process to take more varied routes
				around faults. For example, in circuit routing algorithms, routers
				avoid congestion by increasing the cost of routes which pass through
				congested areas~\cite{kahng11}. A similar technique could be used in
				PGS repair to spread the routes it produces.
				
				An alternative approach would be to adapt the base routing algorithms
				used prior to PGS repair to, for example, attempt alternative dimension
				order routes which may avoid blockages due to faulty links.
			
			\subsubsection{Fault and irregularity aware placement}
				
				One of the shortcomings of the simulated annealing based placer
				developed in chapter~\ref{sec:placement} is that it does not account
				for network faults, or irregularities, when estimating the cost of
				placement solutions.  Future work may exploit techniques used in
				congestion-aware circuit placement which could be adapted for
				application placement~\cite{viswanathan07}.
		
		\subsection{Reducing placement execution time}
			
			The simulated annealing based placer presented in
			chapter~\ref{sec:placement} produced good quality placements but its
			execution time limits its use beyond one million vertex placement
			problems. Future work should explore possibilities for improving the
			performance and scalability of this technique.
			
			In addition to considering alternative placement algorithms based on
			other methods, one possible approach is to attempt to reduce the execution
			time of simulated annealing based placement by shrinking the application
			graph being placed.
			
			For example, graph clustering~\cite{schaeffer07} may be used to group
			together strongly connected vertices which would then be placed as a
			single unit.  Unfortunately, clustering can suffer from the same problems
			as graph-partitioning-based placement: vertices may be grouped together
			in ways which, in practice, cannot be packed together into a given portion
			of a machine.  A possible solution to this problem is to use a two-phase
			placement approach~\cite{kahng11}. In a `global' placement phase,
			solutions are permitted which can slightly over-allocate resources but
			overall achieve good placement quality. In the `detailed' placement phase
			which follows, the solution is `legalised' by making small changes to the
			global placement to eliminate over allocation.
			
			An alternative approach suited to SpiNNaker could be to limit the
			clustering process to clusters which fit on a single SpiNNaker chip. In
			typical SpiNNaker application graphs, clustering to this level may reduce
			placement problem sizes by an order of magnitude and, consequently,
			reduce execution times by the same ratio. Preliminary experiments suggest
			that this approach might result in little placement quality loss for
			large placement problems whilst substantially reducing overall execution
			time.
		
		\subsection{Benchmarking}
			
			One of the most significant limitations of this study has been the
			unavailability of large-scale SpiNNaker applications for use as
			benchmarks. As a consequence, much of the scalability experimentation
			performed has relied on simple synthetic benchmarks based on projections
			of future application behaviour.
			
			In the short term, more sophisticated synthetic benchmark generation
			techniques used by the circuit placement community~\cite{nam07} may offer
			alternative benchmarks for future work. In the longer term, however, it
			is hoped that the availability of large SpiNNaker systems -- and
			placement and routing algorithms better suited to exploit them -- will
			lead to larger scale applications being developed. Hopefully these
			applications will lead to more interesting and representative benchmarks
			for use in future work.
	
	\section{Closing remarks}
		
		One of the primary outcomes of this work is that a number of the practical
		challenges faced in scaling up the SpiNNaker architecture have been
		addressed leading to the construction of large-scale SpiNNaker machines.
		The development of an effective placement algorithm for SpiNNaker
		applications has been shown to enable some neural simulations to exploit
		SpiNNaker's architecture for the first time. The availability of larger
		SpiNNaker machines paves the way for future large-scale neural modelling
		work built on much larger models such as Spaun, `the world's largest
		functional brain model'~\cite{eliasmith12}.
		
		Beyond the SpiNNaker project, the hexagonal torus topology has also been
		validated as a scalable and practical candidate for future network
		architectures. As super computers become ever larger, the physical
		scalability afforded by the 2D nature of the hexagonal torus topology may
		make it a compelling option. In addition, the finding that circuit
		placement techniques can be adapted to support placement of SpiNNaker
		software indicates that these algorithms may also be applicable to other
		applications. Indeed, if this is the case, circuit placement may offer a
		long-term source of placement algorithms able to handle the demands of
		future applications.
		
		% This thesis has explored and tackled a number of the challenges posed in
		% scaling up the unconventional SpiNNaker architecture. Along the way I have
		% demonstrated that the hexagonal torus topology may be a practical choice in
		% future applications which can scale up to the physical dimensions expected
		% of future super computers. I have also developed new efficient and
		% effective methods of placing and routing neural simulation software on
		% SpiNNaker which -- it is hoped -- will enable a new generation of large
		% scale neural simulations on spinnaker.
		
		Although this work stops short of demonstrating truly large-scale
		neuroscientific simulations running at the scale of newly completed
		SpiNNaker machines (largely because such simulations do not yet exist) a
		number of smaller-scale neural simulations have been made possible for the
		first time. The algorithms and techniques devised in this work have
		subsequently been incorporated into various software libraries and tools
		now being used by researchers building SpiNNaker applications, vindicating
		the efforts of this thesis (see appendix~\ref{sec:software}). A successor
		to the SpiNNaker architecture is also in the early stages of design and is
		building on experience of the existing architecture. The current intention
		is to retain the hexagonal torus topology used by SpiNNaker, a decision
		supported by the findings of this thesis.
		
		With SpiNNaker's hardware architecture now operating at scales close to its
		architectural limits, it is hoped that the contributions of this work will
		enable researchers to develop larger and more detailed neural models for
		this unique architecture.

	
	% Bibliography
	\bibliography{references}
	\bibliographystyle{alpha}
	
\end{document}
words.
	
	\clearpage
	\listoffigures
	
	\clearpage
	\listoftables
	
	% Abstract
	{
	\prefacesection{Abstract}
	
	% Single line spacing for the abstract page
	\setstretch{1.0}
	
	
	\vfill
	
	% Standard thesis information
	\begin{center}
		\textsc{\large\thesistitle}
		
		\vspace{0.5em}
		
		\thesisauthor
		
		\vspace{0.5em}
		
		A thesis submitted to the University of Manchester\\
		for the degree of Doctor of Philosophy, 2016
	\end{center}
	
	\vfill
	
	% The abstract
	
	SpiNNaker is an unconventional super computer architecture designed to
	simulate up to one billion biologically realistic neurons in real-time. To
	achieve this goal, SpiNNaker employs a novel network architecture which poses
	a number of practical problems in scaling up from desktop prototypes to
	machine room filling installations.
	
	SpiNNaker's hexagonal torus network topology has received mostly theoretical
	treatment in the literature. This thesis tackles some of the challenges
	encountered when building `real-world' systems.  Firstly, a scheme is devised
	for physically laying out hexagonal torus topologies in machine rooms which
	avoids long cables; this is demonstrated on a half-million core SpiNNaker
	prototype.  Secondly, to improve the performance of existing routing
	algorithms, a more efficient process is proposed for finding (logically)
	short paths through hexagonal torus topologies. This is complemented by a
	formula which provides routing algorithms greater flexibility when finding
	paths, potentially resulting in a more balanced network utilisation.
	
	The scale of SpiNNaker's network and the models intended for it also present
	their own challenges. Placement and routing algorithms are developed which
	assign processes to nodes and generate paths through SpiNNaker's network.
	These algorithms reduce congestion and tolerate network faults. The proposed
	placement algorithm is inspired by techniques used in chip design and is
	shown to enable larger applications to run on SpiNNaker -- with good
	performance -- than the previous state-of-the-art. Likewise the routing
	algorithm developed is able to tolerate network faults, inevitably present in
	large scale systems, with little performance overhead.
	
	
	% Required to ensure single line spacing is used for this whole block
	\par%
}

	
	% Declaration of non-submission elsewhere
	\prefacesection{Declaration}

% Single line spacing for the declaration
{
	\setstretch{1.0}
	No portion of the work referred to in this thesis has been submitted in support
	of an application for another degree or qualification of this or any other
	university or other institute of learning.
	
	\par%
}


	
	% University-prescribed copyright statement...
	\input{copyright}
	
	% Acknowledgements
	{
	\prefacesection{Acknowledgements}
	
	% Single line spacing
	\setstretch{1.0}
	
	It is often said that it is not \emph{what} you know but \emph{who} you know.
	Throughout the course of my PhD I've been exceptionally lucky to have been
	helped along by a great number of people.
	
	Both my supervisor, Jim Garside, and co-supervisor, Steve Furber, have each
	spent countless hours patiently discussing and describing all manner of
	things with me while giving me great freedom in my project. Jim's office door
	has always been open to my unexpected interruptions be it about work, writing
	or walking.  Likewise, Steve has always managed to find time for both
	technical and frivolous endeavours alike. I'm also hugely grateful to Luis
	Plana who has been a rich source of sage advice, insightful questions
	patiently suffered many a foolish question.
	
	Various parts of the work in this thesis (and numerous side projects) would
	not have been possible if not for the multitude of discussions,
	collaborations and even sheer physical hard work of Steve Temple, Javier
	Navaridas, Simon Davidson and Dave Clark. I'm also indebted to Andrew Mundy
	and Jamie Knight, both of whom have donated so much time and effort towards
	verifying and using software implementations of the ideas in this thesis.
	
	The injection of lunchtime silliness by Andrew and Jamie, along with Amanieu
	d'Antras and Andrew Webb and the other CDT members has always brightened my
	day. So to has the friendly and stimulating environment of the School of
	Computer Science and its many staff and students. Of course, I am also very
	grateful for the funding the school has provided for my research.
	
	I cannot thank my wonderful wife, Ann-Marie, enough for being by my side. She
	has given me so much kindness, love and patience and endured a lifetime's
	quota of conversations about hexagons. Finally, thanks too to rest of my
	family, especially my parents, who are to blame for starting me down this
	path and co-suffering drafts and endless rants about this document.
	
	% Required to ensure single line spacing is used for this whole block
	\par%
}

	
	% Main body
	\chapter{Introduction}

\label{sec:introduction}

%Problem area
%
%* Network construction and exploitation
%  * Cabling: Build it cheaply in terms of tech cost and install cost
%  * Routing: Get around it cheaply and reliably
%  * Placement: Use it efficiently

The Spiking Neural Network Architecture (SpiNNaker) is a novel super computer
architecture designed to simulate biologically realistic models of brains in
real time \cite{furber07}. Though neurons, the building blocks of the brain,
are relatively well understood, their complex interactions remain mysterious.
Just as understanding the workings of a transistor is insufficient to
understand a modern microprocessor, neuroscientists believe that understanding
the neurons in isolation cannot explain the brain and that understanding their
connectivity is key \cite{eliasmith13,eliasmith14}. Experiments on real brains,
however, are fraught with difficulty. Variations between individuals can be
significant and it is only possible to record tens or hundreds of the trillions
of signals in the brain, and even then only with limited control over which
signals are recorded. Computer simulations of models of large neural networks,
however, enable researchers to develop repeatable experiments and gain complete
visibility of any signal and any neuron. Models such as SPAUN
\cite{eliasmith12}, built from millions of simulated neurons, have shown great
promise in expanding our understanding of higher level brain functions such as
memory and simple problem solving.  Unfortunately these neural models are
expensive to simulate, requiring hours of compute time to simulate each second
of neural activity. As well as being inconvenient, this precludes the use of
robotics to immerse these models in real world environments and also limits
studies of long-term behaviours such as learning.

SpiNNaker is designed to enable the real time simulation of models containing
up to one billion neurons -- approximately \SI{1}{\percent} of a human brain or
ten mouse brains \cite{furber06}. To achieve this goal, the largest planned
SpiNNaker machine will contain over one million low-powered computer processors
interconnected by a bespoke network architecture.

SpiNNaker's large processor count matches the current trend in super computers
where processor counts are growing exponentially \cite{meuer16j}, mimicking the
growth of the number of components in the processors themselves predicted by
Gordon Moore's famous `law' \cite{moore75}. As a result of this growth, the
interconnection networks which enable these processors to work together have
grown in importance \cite{dally04}.  Network designers must carefully balance
performance against practicality and financial cost.  SpiNNaker's network is no
exception to this rule and, as the systems scale up from desktop prototypes to
machine-room scale installations, the reality of building and exploiting these
machines presents an array of challenges.

As in all super computers, SpiNNaker's network interconnects its processors in
a particular network topology which defines how different processors may
communicate with each other. Unlike the tree and $N$-dimensional torus
topologies found in contemporary super computers \cite{dally04}, SpiNNaker
employs a `hexagonal torus topology'. In this topology, nodes in SpiNNaker's
network fit together in a honeycomb-like pattern where messages may `hop' from
node to node to reach their destination. As we will see in
chapter~\ref{sec:background}, the hexagonal torus topology, in theory, sits at
a `sweet spot' in terms of network performance and practicality. As the first
known large-scale installation of the hexagonal torus topology, however, there
remain a number of practical challenges for large spinnaker machines arising
from this choice.

As super computer networks have grown in scale to millions of processors the
task of dealing with previously rare faults has grown.  Though fault rates in
networks remain consistently low, architectures such as SpiNNaker may have
hundreds of thousands of links meaning even fault rates of a fraction of a
percent will impact tens or hundreds of links. To enable reliable operation,
networks must be able to adapt the routes taken by messages through the network
to avoid faulty links and nodes. The techniques employed are often closely tied
to a particular network architecture and consequently SpiNNaker's novel network
architecture demands its own approach.

Another challenge introduced by the growing scale of super computers is making
\emph{efficient} use of network resources. Communicating processes should be
located on logically `nearby' nodes to reduce network load. The neural models
for which SpiNNaker is designed are often described abstractly, rather than
geometrically, using modelling languages such as PyNN~\cite{davison08} and
Nengo~\cite{eliasmith04}.  Because of this, the communication requirements of
simulations can be highly irregular making an efficient placement of processes
onto processors in the machine non-trivial.

%Contributions
%
%* Cabling scheme for hexagonal toruses without long cables
%* Efficient installation technique for dense systems
%* Exhaustive and efficient route calculation in hex toruses
%* Fault tolerant routing scheme exploiting SpiNNaker's odd router
%* Placement based on SA a: works very well and b: suggests circuit placement is
%  a good source of inspiration.

This thesis addresses the practical challenges of scaling up the SpiNNaker
architecture in a real-world setting summarised by these research questions:

\begin{enumerate}
	
	\item Can the hexagonal torus topology be deployed and used in real, large
	scale systems?
	
	\item Does SpiNNaker's router architecture help, or hinder fault tolerance?
	
	\item How can the parts of a neural simulation be placed onto a large
	hexagonal torus topology to reduce network load?
	
\end{enumerate}

%Structure
%
%* Chapter 2: Background: detailed dive into what's in SpiNNaker, why its
%  really so unusual. Also looks at what applications run on SpiNNaker and how
%  they work.
%* Chapter 3: How to build a really big SpiNNaker machine.
%* Chapter 4: How to find your way around that machine.
%* Chapter 5: How to find your way around that machine even when its broken.
%* Chapter 6: Now you can walk, time to run.
%* Chapter 7: Wrapping up.
%* Appendices: Hard-to-come-by theoretical and practical details useful if
%  you're about to continue where this research left off but be useful but
%  otherwise hard to come by, especially in one place.

Chapter~\ref{sec:background} introduces the SpiNNaker architecture and, in
particular, describes its hexagonal torus topology and network architecture.

In chapter~\ref{sec:building}, I develop a cabling scheme for large hexagonal
torus topologies which enables arbitrarily large networks to be constructed
using only short, inexpensive cables. This theoretical work is then evaluated
through the construction of a range of prototype SpiNNaker systems. The largest
of these prototypes contains over half a million processor cores and spans
several machine room cabinets. In addition, I propose the use of built-in
diagnostic facilities to assist technicians performing network installation and
maintenance. This technique is found to greatly reduce the effort required and
the number of mistakes made.

In chapters~\ref{sec:shortestPaths}~and~\ref{sec:routing} I develop new routing
techniques for SpiNNaker's network. Chapter~\ref{sec:shortestPaths} develops a
new approach to finding the shortest paths through hexagonal torus topologies,
an integral part of many routing algorithms. This newly proposed approach is
cheaper to compute than the state of the art and, unlike previous efforts, is
able to discover all valid short paths through the topology. This theoretical
advance brings hexagonal torus topologies in line with conventional topologies
by providing routing algorithms with complete information about the paths
available to them. In chapter \ref{sec:routing} I propose a fault tolerant
routing algorithm for SpiNNaker which is able to avoid arbitrary static fault
patterns with minimal performance overhead. A key finding of this chapter is
that the flexibility afforded to fault tolerant routing algorithms by
SpiNNaker's unconventional router architecture is what facilities the low
overheads reported in this chapter.

Finally, in chapter~\ref{sec:placement}, I explore the problem of application
placement in SpiNNaker's network. As in other networks and applications, neural
simulations should be arranged such that communication occurs primarily between
processors close together in the network to control network load. Due to the
irregular connectivity and large scale of the neural models expected to run on
SpiNNaker, an automated approach is necessary. I develop a novel placement
algorithm based on algorithms used for circuit layout in computer chips. My
algorithm is found to allow some larger neural models to run on SpiNNaker for
the first time while enabling other applications to run at greater speeds. In
addition, synthetic benchmarks containing over one million processes indicate
that this algorithm should handle the anticipated demands of the neural models
expected to run on large-scale SpiNNaker installations.

	\chapter{The SpiNNaker Architecture}
	
	\label{sec:background}
	
	SpiNNaker is a massively parallel computer architecture designed to simulate
	biologically realistic neural models \cite{furber07}. In this chapter we will
	explore this unconventional architecture in detail, starting with its purpose
	before focusing on its most unconventional feature: its network.
	
	% * Purpose
	%   * Spiking neural simulations
	%     * Neural modelling: PyNN, Nengo...
	%     * Parallelisation + communication
	
	\section{Neural simulation}
		
		Human brains contain billions of neurons connected together by trillions of
		`synapses'. Neurons communicate by transmitting and receiving `spikes'
		through their synapses. Each spike is `valueless' in that a spike's only
		significant features are when it arrives and where it has come from.
		
		\begin{figure}
			\center
			\buildfig{figures/lif-neuron.tex}
			
			\caption{A Leaky Integrate-and-Fire (LIF) neuron.}
			\label{fig:lif-neuron}
		\end{figure}
		
		Though some detailed models of the electrochemical processes occurring
		inside neurons are computationally intensive, simplified models such as the
		Leaky Integrate-and-Fire (LIF) model can be implemented in just a handful
		of CPU instructions \cite{vainbrand11}. Figure~\ref{fig:lif-neuron}
		illustrates a simple LIF neuron in which incoming spikes cause charge to
		build up (integrated) which over time, leaks away. If an incoming spike
		causes the charge to rise above a certain threshold, the neuron `fires'
		producing an outgoing spike. Despite the simplicity of this model, large
		neural networks such as Spaun \cite{eliasmith12} -- built entirely from LIF
		neurons -- exhibit complex behaviours such as fine motor control and
		problem solving.
		
		The computational expense of large scale neural simulations does not arise
		from the cost of modelling neurons but instead from distributing spikes. In
		biology, neurons produce spikes at an average rate of \SI{10}{\hertz} and
		synapses connect each neuron's output to (order) \num{1000}~neurons
		\cite{navaridas09}. Consider an example neural model with $7\times10^7$
		neurons, approximately the number in a house mouse and
		$\nicefrac{1}{10}^\textrm{th}$ of the design target of SpiNNaker. This
		network might produce $7\times10^8$~spikes per second. Because each neuron
		connects to many others, this equates to $7\times10^{11}$ spikes being
		received per second. If each spike were transmitted as a UDP datagram
		containing a single \SI{32}{\bit} payload, the total network throughput
		required for this simulation would be \SI{179.2}{\tera\bit\per\second}. At
		the time of writing, this is more than double the bisection bandwidth (the
		theoretical worst-case throughput) of the world's most powerful super
		computer \cite{dongarra16}.
	
	\section{Network architecture}
		
		Architectures such as IBM's Blue Gene \cite{chiu11} and Cray's XK7
		\cite{ornl16} employ powerful compute nodes connected together using
		networks designed to transfer multi-kilobyte blocks of data between nodes.
		Since neural models have relatively light computational requirements and
		communications are based on small pieces of data (spikes), this type of
		architecture is poorly suited to the task.
		
		SpiNNaker's architectural target is to support realtime simulations of up
		to one billion neurons. Since neural models such as LIF are inexpensive to
		model and many neurons can be simulated independently in parallel,
		SpiNNaker employs many small, energy efficient ARM processors
		\cite{furber07}. To support the unusual communication requirements of
		neural simulations, a bespoke interconnection network is used which is the
		background to this thesis.
		
	%   * SpiNNaker chip
	%     * Cores
	%     * SDRAM
	%     * NoC
	%     * Router
		
		\begin{figure}
			\center
			%\includegraphics[width=19mm]{figures/spinnakerChip.jpg}
			\buildfig{figures/hex-chips.tex}
			
			\caption[SpiNNaker chips connected to their six neighbours.]%
			{SpiNNaker chips (actual size) connected to their six neighbours.}
			\label{fig:spinnakerChip}
		\end{figure}
		
		The fundamental building block of the SpiNNaker architecture is the
		SpiNNaker chip (figure \ref{fig:spinnakerChip}) \cite{furber13}. Each chip
		contains eighteen low power ARM 968 processor cores each capable of
		simulating between \num{200} and \num{2000} LIF neurons in real time
		\cite{mundy15}.  Each core has a total of \SI{96}{\kilo\byte} of private
		Tightly-Coupled Memory (TCM) and shares access to \SI{128}{\mega\byte} of
		on-chip SDRAM with other cores on the same chip. Finally, each chip
		contains a programmable router which routes network packets to and from the
		local cores and six neighbouring SpiNNaker chips. SpiNNaker machines are
		constructed by combining many SpiNNaker chips.
		
		\begin{figure}
			\center
			\buildfig{figures/spinnaker-packet.tex}
			
			\caption{SpiNNaker's \SI{40}{\bit} and \SI{72}{\bit} multicast packet
			format.}
			\label{fig:spinnaker-packet}
		\end{figure}
		
		Processor cores can communicate by sending and receiving network packets
		forwarded by routers through the network. Since SpiNNaker's network is
		designed to transmit neural spike events efficiently, individual network
		packets are small, either \SI{40}{\bit} or \SI{72}{\bit} compared with tens
		or hundreds of byte packets in typical network architectures.
		
		In a real-time simulation, the time at which a spike is produced is
		implicitly indicated by the time it is received -- since at biological
		timescales a computer network delivers packets `instantaneously'.
		Consequently, the only information which must be explicitly encoded is the
		identity of the neuron which produced the spike. In SpiNNaker, a spike may
		be encoded by using a single \SI{40}{\bit} `multicast packet' whose format
		is illustrated in figure~\ref{fig:spinnaker-packet}.  The \SI{8}{\bit}
		header is used by SpiNNaker's routers to determine the type of packet and
		the \SI{32}{\bit} `routing key' is used to identify the neuron which
		produced the packet. The routing key is also used by SpiNNaker's routers to
		determine how the packet should be directed through the network.
		
		The optional \SI{32}{\bit} payload is not used by conventional spiking
		neural simulations \cite{galluppi10} but has been exploited to enable more
		efficient simulation of a particular class of neural models \cite{mundy15}.
	
	\section{The SpiNNaker router}
		
		The SpiNNaker router employs an unconventional design which, despite its
		compact size and small energy requirements, implements a flexible multicast
		routing scheme. Unlike conventional routers which often employ hard-coded
		routing rules \cite[chapter~8]{dally04}, the SpiNNaker router uses a
		programmable `routing table' to determine how packets should be forwarded.
		In addition, to avoid deadlocks, SpiNNaker's router employs a simple,
		timeout-based mechanism which exploits the ability of neural networks to
		tolerate occasional missing packets. As we will see in chapter
		\ref{sec:routing}, this mechanism greatly simplifies the task of routing in
		SpiNNaker's network. In this section we'll look at these features in
		greater detail.
		
		\subsection{Routing tables}
		
			When a multicast packet arrives at a SpiNNaker router (either from a
			local core or a neighbouring chip), the router looks up the routing key
			in its routing table. This table consists of \num{1024} programmable
			table entries, each specifying a routing key bit pattern and mask to
			match and a set of routes.  When a multicast packet's key is matched by a
			routing entry the packet is forwarded along every route specified by that
			entry, potentially duplicating the packet. This `multicast' technique
			allows packets to be transmitted once but received in a number of places
			while making efficient use of the network \cite{navaridas12}.
			
			Though routing table entries are in finite supply (\num{1024} entries per
			router), it is still possible for many thousands of traffic flows to be
			routed through a single router. The bit pattern and mask in each routing
			entry matches against the 32~bits of a routing key as either
			`\texttt{1}', `\texttt{0}' or `\texttt{X}' (don't care).  This means that
			a single routing entry may, for example, be used to match all routing
			keys with a certain prefix. If a routing key is not matched by any entry
			in the routing table then the packet is `default routed' in a straight
			line. For example if a packet with an unmatched key is received from the
			chip to the left, the packet will be default routed to the chip on the
			right. By assigning routing keys such that neurons whose spikes are sent
			to similar destinations share a similar prefix, the number of routing
			entries required by a simulation is greatly reduced \cite{davies12}.
			
			\begin{figure}
				\center
				\buildfig{figures/routing-example.tex}
				
				\caption[Multicast routing example.]%
				{Multicast routing example with \SI{4}{\bit} routing keys. Each
				box represents a SpiNNaker chip whose router has been programmed with
				the routing entries shown. Grey lines mark connections between chips.}
				\label{fig:routing-example}
			\end{figure}
			
			Consider the simplified example in figure~\ref{fig:routing-example} in
			which a number of (\SI{4}{\bit}) routing table entries have been
			configured in the routers of a small SpiNNaker network. If a packet with
			the routing key \texttt{1011} is transmitted by a core in the chip
			labelled $(0, 0, 0)$, this will match the first routing table entry on
			that chip and will be routed to chip $(1, 0, 0)$. On chip $(1, 0, 0)$,
			the packet once again matches the first routing entry and is routed to
			chip $(1, 0, -1)$. On $(1, 0, -1)$, no match is made so the packet is
			default routed to $(1, 0, -2)$. On this chip, the packet matches a
			routing entry which routes the packet to core~7. In this example, default
			routing allows only three routing table entries to direct a packet
			through four chips.
			
			As a second example, if a packet with the routing key \texttt{0010} is
			transmitted by a core on chip $(0, 0, 0)$, this key will be matched by
			the second routing entry since \texttt{X}s in the table entry will match
			both \texttt{1}s and \texttt{0}s in the corresponding bits of the routing
			key. When the packet arrives at chip $(0, 0, -1)$ the matching routing
			entry forwards the packet to both $(0, 1, -1)$ and $(1, 0, -1)$
			simultaneously. The copy of the packet arriving at $(0, 1, -1)$ is routed
			to core~5 on that chip.  Meanwhile, the copy forwarded to $(1, 0, -1)$ is
			duplicated again with one copy being routed to core~11 and another being
			routed to chip $(1, 0, -2)$. Here the packet is finally delivered to
			core~6. In this example, the ability of the router to multicast
			(duplicate) packets as they pass through the network meant that sending
			one copy of the packet was sufficient to reach three destination cores.
			In addition, by using \texttt{X}s in the routing table entry, the same
			routing entries are sufficient to route packets with the keys
			\texttt{0000}, \texttt{0001}, \texttt{0010} and \texttt{0011}.
			
			In spite of these mechanisms, it is still possible for an application to
			run out of routing table entries. As we will see in
			chapter~\ref{sec:placement} by arranging applications appropriately
			within SpiNNaker's network, routing table usage can be reduced. In
			addition, other behaviours of SpiNNaker's router may be exploited to
			compress an applications routing tables further, however the techniques
			employed are beyond the scope of this thesis \cite{mundy16}.
		
		\subsection{Timeouts}
			
			SpiNNaker's router is built on a pipeline architecture. As shown in
			figure~\ref{fig:router-architecture}, the router is fed packets by an
			arbiter which serialises packets arriving from other chips and local
			cores. Every (\SI{100}{\mega\hertz}) clock cycle, the router pipeline
			accepts one packet from the arbiter and routes a packet to one or several
			output links. If any of the required output ports are busy then the
			packet is not forwarded to any output link and the pipeline stalls. Once
			a packet has been blocked for a programmable timeout, it is dropped
			(discarded) and routing continues as usual for next packet in the
			pipeline. Links become blocked while transmitting packets or waiting for
			the remote receiver to become ready. For example, a receiving processor
			core may be busy performing some computation or a receiving router may be
			blocked waiting for some of its outputs to become ready.
			
			\begin{figure}
				\center
				\buildfig{figures/router-architecture.tex}
				
				\caption{SpiNNaker router architecture}
				\label{fig:router-architecture}
			\end{figure}
			
			The timeout-based packet dropping mechanism is designed to defuse
			deadlocks in the network. For example, if two routers are trying to send
			each other a packet at the same time they may become deadlocked, each
			waiting for the other router to accept a packet before continuing.
			SpiNNaker's timeout mechanism breaks deadlocks by dropping packets which
			have been blocked for some time and therefore may be in a deadlock.  Once
			a packet has been dropped it is left to software to either tolerate the
			missing packet or trigger a retransmission. In neural simulations, as in
			biology, the loss of a single spike is unlikely to have a significant
			impact on the behaviour of a neural model and therefore these simulations
			are inherently tolerant of occasional dropped packets. During application
			loading and other system tasks, a higher level, software driven protocol
			based on acknowledgements and retransmissions is used to ensure
			guaranteed delivery.
			
			% TODO: MENTION TIMEOUT VALUE USED?
			% Router timeouts must be configured to be long enough that delays in
			% packet transmission, for example due to the time taken for packets to
			% traverse a link, do not trigger packet dropping. Conversely, the timeout
			% should be as short as possible to reduce the time the router is
			% blocked and maximise network throughput.
	
	\section{The hexagonal torus topology}
		
		Each SpiNNaker chip is a node in a `hexagonal torus topology' as
		illustrated in figure~\ref{fig:hexagonalTorusTopology}. Network packets
		sent by SpiNNaker's processor cores may `hop' through several nodes in the
		network to reach their intended destination. In each hop, a packet may
		advance one node along one of the three axes of the topology. For example,
		a packet sent by the node labelled $\alpha$ (in the bottom-left corner) to
		the node labelled $\beta$, might take the following sequence of hops:
		X$^+$, X$^+$, Z$^-$. Packets sent from $\alpha$ to $\gamma$ might take the
		route: X$^-$, X$^-$, Y$^+$, Y$^+$. The first hop of this route `wraps
		around' from the bottom-left node to the bottom-right node in a single hop.
		
		\begin{figure}
			\center
			\buildfig{figures/hexagonalTorusTopology.tex}
			
			\caption[A hexagonal torus topology.]%
			{A hexagonal torus topology. Each hexagon represents a node (i.e.
			a SpiNNaker chip). Touching nodes are directly connected. Nodes on edges
			$a$, $b$ and $c$ are also directly connected to the corresponding nodes
			on edges $a'$, $b'$ and $c'$, respectively. The three axes of the
			hexagonal torus topology, `X', `Y' and `Z' are also shown.}
			\label{fig:hexagonalTorusTopology}
		\end{figure}
		
		\begin{figure}
			\center
			\begin{subfigure}{0.39\linewidth}
				\center
				\includegraphics[width=\linewidth]{figures/torus-3d-flat.pdf}
				\caption{}
				\label{fig:torus-3d-flat}
			\end{subfigure}
			~~
			\begin{subfigure}{0.26\linewidth}
				\center
				\includegraphics[width=\linewidth]{figures/torus-3d-tube.pdf}
				\caption{}
				\label{fig:torus-3d-tube}
			\end{subfigure}
			~~
			\begin{subfigure}{0.23\linewidth}
				\center
				\includegraphics[width=\linewidth]{figures/torus-3d-torus.pdf}
				\caption{}
				\label{fig:torus-3d-torus}
			\end{subfigure}
			
			\caption{Visualisation of a hexagonal torus topology as a torus.}
			\label{fig:torus-3d}
		\end{figure}
		
		The wrap around connections in the topology are what give it the `torus'
		part of its name. Figure~\ref{fig:torus-3d-flat} shows a hexagonal torus
		topology drawn flat as in the previous figure. If the topology is rolled up
		into a tube such that the top and bottom nodes become directly adjacent, a
		tube is formed as in figure~\ref{fig:torus-3d-tube}. This tube can then be
		bent to bring together the nodes at the ends of the tube to form a torus as
		shown in figure~\ref{fig:torus-3d-torus}.
		
		A hexagonal torus topology is typically defined in terms of its width and
		height along the X and Y axes respectively. For example,
		figure~\ref{fig:hexagonalTorusTopology} shows a $10\times10$ hexagonal
		torus.  The nodes in a hexagonal torus topology are addressed using
		hexagonal coordinates of the form $(x, y, z)$ \cite{patel15}. The bottom
		left node (labelled $\alpha$ in the figure) has the coordinate $(0, 0, 0)$
		and other nodes are assigned coordinates according to the number of hops
		along each dimension from $(0, 0, 0)$, for example node $\beta$ has the
		coordinate $(2, 0, -1)$. Because the hexagonal torus topology's axes are
		non-orthogonal, it is possible to define several coordinates for the same
		location. For example $(3, 1, 0)$ and $(1, -1, -2)$ are also valid
		coordinates for node $\beta$. These dual coordinates emerge from the fact
		that adding $(1, 1, 1)$ to a coordinate produces an equivalent, but
		different, coordinate. This phenomenon is explained in detail in
		appendix~\ref{app:minimal-hex-coordinates} and related phenomena will be
		discussed in chapter~\ref{sec:shortestPaths}.
		
		The hexagonal torus topology was chosen over a more conventional network
		topology -- such as a 2D or 3D torus (sometimes known as a 2-ary $N$-cube
		or 3-ary $N$-cube respectively) \cite[chapters~3~and~5]{dally04} -- due to
		its balance of theoretical performance and practicality. The bisection
		bandwidth of a topology indicates the theoretical worst-case total
		throughput the network is able to sustain \cite[chapter~1]{dally04}.  In
		networks with homogeneous link throughput, bisection bandwidth is
		determined by the number of links cut by a balanced bisection of the
		network.  Figure~\ref{fig:bisection-bandwidth} illustrates the bisections
		of several torus topologies.
		
		\begin{figure}
			\center
			\begin{subfigure}[b]{0.3\linewidth}
				\center
				\buildfig{figures/bisection-bandwidth-2d.tex}
				
				\caption{2D Torus}
				\label{fig:bisection-bandwidth-2d}
			\end{subfigure}
			\begin{subfigure}[b]{0.3\linewidth}
				\center
				\buildfig{figures/bisection-bandwidth-hex.tex}
				
				\caption{Hexagonal Torus}
				\label{fig:bisection-bandwidth-hex}
			\end{subfigure}
			\begin{subfigure}[b]{0.3\linewidth}
				\center
				\buildfig{figures/bisection-bandwidth-3d.tex}
				
				\caption{3D Torus}
				\label{fig:bisection-bandwidth-3d}
			\end{subfigure}
			
			\caption[Bisections of torus topologies.]%
			{Bisections of torus topologies. Connections cut by the bisection
			are drawn as lines.}
			\label{fig:bisection-bandwidth}
		\end{figure}
		
		In a $N \times N$ 2D torus topology, the bisection bandwidth is $2N$~links
		and each node requires four links. The hexagonal torus topology requires
		six links per node but provides double bisection bandwidth ($4N$~links).
		The 3D torus topology also requires six links per node but by connecting
		the nodes differently achieves a bisection bandwidth of $8N$~links.  The 3D
		torus topology, however, comes at a price -- when immersed into the
		(approximately) 2D space provided by a large machine room or row of server
		cabinets, some connections require long cables. By contrast, the 2D and
		hexagonal torus topologies are both inherently two dimensional and
		consequently do not suffer from this effect. The hexagonal torus topology,
		therefore, shares the practicality of construction of a 2D torus while
		still gaining some of the performance of a 3D torus topology. In addition,
		because nodes in a hexagonal torus topology have a greater number of links,
		greater redundancy is available in the network to tolerate faults.
		
		Most torus topologies, including hexagonal, 2D and 3D toruses, have a
		related `mesh' topology. These mesh topologies maintain the same general
		connectivity structure as their torus topologies but omit wrap-around
		links. In practice, this saves a small number of links at the expense of
		halving the network's bisection bandwidth.  Because of their poorer
		performance, mesh networks are rarely used as the basis of a network
		architecture. Mesh networks, however, are occasionally formed when a
		network is partitioned into several smaller sub-networks to allow multiple
		users to share a system \cite{spalloc16}.
		
		\begin{figure}
			\center
			\begin{subfigure}[b]{0.45\linewidth}
				\center
				\buildfig{figures/hexagonal-torus.tex}
				\caption{Hexagonal torus}
				\label{fig:topo-compare-hexagonal-torus}
			\end{subfigure}
			\begin{subfigure}[b]{0.45\linewidth}
				\center
				\buildfig{figures/h-torus.tex}
				\caption{H-torus}
				\label{fig:topo-compare-h-torus}
			\end{subfigure}
			
			\caption[Hexagonal torus vs. H-torus topology.]%
			{Hexagonal torus vs. H-torus topology. Each numbered hexagon
			represents a node. The thick outline indicates the bounds of the
			topology after which the network repeats. In each topology, the path
			taken by advancing in the Y$^+$ direction from the node labelled `0' is
			shown.}
			\label{fig:topo-compare}
		\end{figure}
		
		\label{sec:hex-vs-h-torus}
		
		The hexagonal torus topology is not to be confused with the `H-torus'
		topology. This topology also uses a hexagonal tiling of nodes and even
		wraps this tiling into a torus-like topology \cite{zhao08}. However,
		H-torus topologies have very different characteristics to the hexagonal
		torus topology and are related to `twisted torus' topologies
		\cite{camara10}. For example, figure~\ref{fig:topo-compare} illustrates one
		major difference in the way paths wrap around the peripheries of both
		topologies.
	
	\section{Scaling-up SpiNNaker machines}
		
		To build large SpiNNaker systems comprising of tens of thousands of
		SpiNNaker chips, groups of forty-eight chips are mounted onto circuit
		boards as illustrated in figure~\ref{fig:spinnakerBoard}. These boards may
		be connected together to form larger systems.  Figure~\ref{fig:threeboard}
		shows a prototype three board system. Though the chips are
		\emph{physically} arranged in a (nearly) $7\times7$ grid on each SpiNNaker
		board, they logically form a hexagonal `wrapped triple'
		\cite{davidsonWiring} (see appendix~\ref{sec:partitioning}) which logically
		fit together as illustrated in figure~\ref{fig:threeboard-separate}. The
		labelled exposed corners of the three forty-eight chip boards connect
		together to form a $12\times12$ hexagonal torus topology as illustrated in
		figure~\ref{fig:threeboard-wrapped}. Larger SpiNNaker machines are
		assembled by combining more boards.
		
		\begin{figure}
			\center
			\begin{subfigure}[b]{0.45\linewidth}
				\center
				\includegraphics[width=\linewidth]{figures/spinnakerBoard.jpg}
				
				\caption{A SpiNNaker board}
				\label{fig:spinnakerBoard}
			\end{subfigure}
			~~~
			\begin{subfigure}[b]{0.45\linewidth}
				\center
				\includegraphics[width=\linewidth]{figures/threeboard.jpg}
				
				\caption{Three board prototype}
				\label{fig:threeboard}
			\end{subfigure}
			
			\vspace*{1em}
			
			\begin{subfigure}[b]{0.45\linewidth}
				\center
				\buildfig{figures/threeboard-separate.tex}
				
				\caption{Three board topology}
				\label{fig:threeboard-separate}
			\end{subfigure}
			~~~
			\begin{subfigure}[b]{0.45\linewidth}
				\center
				\buildfig{figures/threeboard-wrapped.tex}
				
				\caption{\ldots{}as a parallelogram}
				\label{fig:threeboard-wrapped}
			\end{subfigure}
			
			\caption{SpiNNaker boards and their topology.}
			\label{fig:spinnaker-boards}
		\end{figure}
		
		
		SpiNNaker chips on the same circuit board connect using low power links
		requiring sixteen wires each.  If this link technology were used to connect
		chips on neighbouring boards, each pair of boards would need to be
		connected with a 128~wire cable.  Cables and connectors supporting this
		many signals are expensive, unreliable and physically large. Instead,
		chip-to-chip connections between boards are multiplexed and demultiplexed
		onto a single High-Speed Serial (HSS) link \cite{athavale05} carried via
		commodity S-ATA cables which are often used to connect hard disks in
		desktop computers and servers \cite{sata3spec}. The six high-speed links
		are implemented by three onboard FPGAs (the three large chips at the top of
		the SpiNNaker board) and are logically transparent to the underlying
		network. The underlying technology and the choice of S-ATA cables limits
		each board-to-board connection to spanning at most one metre gaps. In
		chapter~\ref{sec:building} I present a cabling scheme for hexagonal torus
		topologies which enable large SpiNNaker systems to be assembled using only
		short cables between boards.
		
	\section{Conclusions}
		
		The SpiNNaker architecture has been designed to enable the simulation of
		large biologically realistic neural models in real time. To support this,
		its network architecture takes on an unconventional design based on a
		custom router and hexagonal torus topology. In the remainder of this
		thesis, I will tackle a number of the challenges in scaling up the
		SpiNNaker architecture outlined in this chapter.

	\chapter{Building large SpiNNaker machines}
	
	Like any super computer, physically putting together a large SpiNNaker
	machine poses many challenges in terms of organisation, assembly and
	maintainance. One of the key tasks in this process is the installation of
	network cables such that a desired overall network topology is constructed.
	The largest planned SpiNNaker machine will use \num{3600} S-ATA
	\cite{sata3spec} cables to interconnect its \num{1200} circuit boards,
	creating a hexagonal torus topology. Since the machine will be installed
	within standard server room cabinets (which are not available in a
	giant-doughnut form-factor) a mapping from a board's logical location in the
	network topology to its physical location must be constructed. In addition,
	the interconnect technology employed by SpiNNaker restricts the length of
	S-ATA cables used to $\le$~\SI{1}{\meter}, constraining the possible mappings
	used. In addition the practical issues of installation complexity and
	maintainance must be considered since all \num{3600} cables must ultimately
	be installed and maintained by human operators.
	
	In this chapter I describe a novel technique for physically laying out
	machines configured in hexagonal torus topologies, such as SpiNNaker, in
	commercial machine rooms, building on the techniques used in more
	conventional torus topologies. In addition, I also propose a new methodology
	for installing and maintaining super computer cabling which which exploits
	existing diagnostic features of the SpiNNaker hardware to interactively guide
	and validate cable installation. Finally, I demonstrate how these new
	techniques have been used successfully to interconnect a prototype
	\num{518400} core SpiNNaker machine in substantially less time than the
	industry norm.
	
	In this chapter, the term \emph{unit} refers to the smallest physical
	ecomponent between which connections connections are to be made. For example,
	in a SpiNNaker machine a unit is a 48-chip board while in data center, a unit
	might be a server blade.
	
	\section{Related work}
		
		In this section I describe the techniques conventionally employed when
		laying out and interconnecting the units within super computers. Due to
		SpiNNaker's hexagonal torus topology and dense physical packing of units,
		these existing techniques are found to be insufficient. In the remainder of
		the chapter we will explore solutions to the limitations exposed below.
		
		\subsection{Avoiding long cables}
			
			Na\"ive arrangements of torus topologies, including hexagonal torus
			topologies, feature long `wrap-around' connections which connect units at
			the peripheries of the system. These connections can be problematic for
			numerous reasons:
			
			\begin{description}
				
				\item[Performance] Signal quality diminishes as cables get longer,
				requiring the use of slower signalling speeds, increased error
				correction overhead or more complex hardware.
				
				\item[Energy] Longer cables require higher drive strengths and/or
				buffering to maintain signal integrity.
				
				\item[Cost] Cost Shorter cables are cheaper than long ones.  Longer
				cables imply more wire in a given space making the tasks of routing or
				cable installation more difficult increasing labour costs by as much as
				$5\times$ \cite{curtis12}.
				
			\end{description}
			
			In conventional torus topologies the need for long cables is eliminated
			by folding and interleaving units of the network \cite{dally04}. For
			example, for a 1D torus topology (a ring network), one long connection
			exists to connect the two opposite sides of the system. To remove these
			long connections, half the units are `folded' on top of the others and
			then this arrangement of units is interleaved as illustrated in figure
			\ref{fig:ring-folding}.
			
			\begin{figure}
				\center
				\begin{subfigure}[b]{0.39\linewidth}
					\center
					\buildfig{figures/ring-folding-row.tex}
					\caption{A ring network}
					\label{fig:ring-folding-row}
				\end{subfigure}
				\begin{subfigure}[b]{0.24\linewidth}
					\center
					\buildfig{figures/ring-folding-folded.tex}
					\caption{Folded}
					\label{fig:ring-folding-folded}
				\end{subfigure}
				\begin{subfigure}[b]{0.35\linewidth}
					\center
					\buildfig{figures/ring-folding-interleaved.tex}
					\caption{Folded and interleaved}
					\label{fig:ring-folding-interleaved}
				\end{subfigure}
				
				\caption{Folding and interleaving a ring network to reduce maximum wire
				length.}
				\label{fig:ring-folding}
			\end{figure}
			
			Folding and interleaving has the effect of approximately doubling the
			average cable length but also eliminates the need for a cable spanning
			the entire system. Since the mean cable length is typically already
			short, doubling it in exchange for a substantially reduced maximum cable
			length is often preferable.
			
			The folding and interleaving process may be extended to $N$-dimensional
			torus topologies by folding each dimension in turn. Since all dimensions
			are orthogonal, the folding process only moves units in the dimension
			being folded. In the hexagonal torus topology, however, the three
			dimensions are non-orthogonal and thus folding in one dimension also
			moves units in other dimensions, preventing the edges of the torus
			meeting as illustrated in figure \ref{fig:failing-to-fold-hex-toruses}.
			
			\begin{figure}
				\center
				\begin{subfigure}[b]{0.24\linewidth}
					\center
					\buildfig{figures/failing-to-fold-hex-toruses-none.tex}
					\caption{Not folded}
					\label{fig:failing-to-fold-hex-toruses-none}
				\end{subfigure}
				\begin{subfigure}[b]{0.24\linewidth}
					\center
					\buildfig{figures/failing-to-fold-hex-toruses-x.tex}
					\caption{X}
					\label{fig:failing-to-fold-hex-toruses-x}
				\end{subfigure}
				\begin{subfigure}[b]{0.24\linewidth}
					\center
					\buildfig{figures/failing-to-fold-hex-toruses-y.tex}
					\caption{Y}
					\label{fig:failing-to-fold-hex-toruses-y}
				\end{subfigure}
				\begin{subfigure}[b]{0.24\linewidth}
					\center
					\buildfig{figures/failing-to-fold-hex-toruses-z.tex}
					\caption{Z}
					\label{fig:failing-to-fold-hex-toruses-z}
				\end{subfigure}
				
				\caption{Schematics showing hexagonal torus topologies folded along
				each of their non-orthogonal dimensions. Note that folding along
				the Z axis brings the \emph{wrong} edges closer together.}
				\label{fig:failing-to-fold-hex-toruses}
			\end{figure}
		
		\subsection{Cabling installation}
			
			Existing machine room installations feature very repetitive cabling
			patterns which can easily be memorised by a human technician. For example
			in BlueGene super computers the connectivity between units is highly
			regular \cite{lakner07} while in data centre networks cabling often
			centres around a small number of high-port-count switches
			\cite{cisco07,csernai15}. Cable installation is usually only aided by
			the labelling of connectors and sockets in a standardised manner
			\cite{tia2006} such as in figure \ref{fig:bgWiring}.
			
			\begin{figure}
				\center
				\begin{subfigure}[t]{0.5\textwidth}
					\begin{tikzpicture}
						\node (cables) [inner sep=0]
						      {\includegraphics[width=\textwidth]{figures/bgCables.png}};
						\node (sockets) [inner sep=0, below=1.0em of cables]
						      {\includegraphics[width=\textwidth]{figures/bgSockets.png}};
						
						% Point at label on cable
						\draw [white, <-, line width=0.4em]
						      ([shift={(0.7cm, -0.3cm)}]cables.center)
						      -- ++(45:1cm);
						
						% Point at label on socket
						\draw [white, <-, line width=0.4em]
						      ([shift={(-1.0cm, 1.1cm)}]sockets.center)
						      -- ++(-45:1cm);
					\end{tikzpicture}
					
					\caption{Pre-labelled cables and sockets}
					\label{fig:bgWiringLabels}
				\end{subfigure}
				~
				\begin{subfigure}[t]{0.30\textwidth}
					\includegraphics[height=6.15cm]{figures/bgWiring.jpg}
					
					\caption{Installation of cables}
					\label{fig:bgWiringInstallation}
				\end{subfigure}
				
				\caption{BlueGene/Q cable installation \cite{cscs13}}
				\label{fig:bgWiring}
			\end{figure}
			
			Despite the regularity and careful labelling of cables, the cost of
			installation and maintenance alone can be significant with costs in the
			range of \$45-95 per \SI{1}{\meter} cable run and \$100-400 for runs of
			\SI{10}{\meter} reported in the literature \cite{mudigonda11}. Much of
			this cost is due to the care required during installation to avoid
			miswiring and ensure that cooling airflow is not hampered by cable runs
			\cite{cisco07}.
			
			Many researchers have attempted to control cable installation costs by
			trying to reduce the number or length of cables required by developing
			alternative network topologies \cite{curtis12, popa10, mudigonda11}.
			Unfortunately, these techniques do not apply to SpiNNaker since its
			network topology is fixed.
			
			Some super computers make use of large custom `midplane` PCBs in place of
			cables to interconnect units within a cabinet and thus simplify the task
			of cable installation \cite{prickett10}. This scheme can greatly reduce
			wiring complexity since only coarser-grain cabinet-to-cabinet
			connectivity is provided by cables. Unfortunately this technique is
			expensive and also constrains the dimensions of the network topology
			supported by the machine. Since the SpiNNaker platform is designed to
			scale from desktop machines to machine-room installations, this scheme is
			not practical.
	
	\section{Folding \& interleaving hexagonal toruses}
		
		The first step towards a practical machine-room installation of a large
		machine using a hexagonal torus topology is to find an arrangement of
		boards between which cable lengths are minimised. In this section I
		describe a sequence of transformations which map the positions of units in
		a hexagonal torus topology onto a regular rectangular grid which may be
		folded and interleaved to eliminate long wires. It is worth emphasising
		that this transformation only affects the \emph{physical} positions of
		units and \emph{not} their connectivity.
		
		As described earlier in \S\ref{sec:parititioning} (page
		\pageref{sec:parititioning}), hexagonal torus topologies may be partitioned
		into units containing wrapped-triples of nodes. For example, in SpiNNaker,
		chips (nodes) are partitioned into circuit boards (units) containing 48
		chips. For completeness, this section describes the process of folding both
		systems whose units are individual nodes and those whose units are
		wrapped-triples.
		
		The transformation process is divided into two parts, each described
		separately in this section.
		
		\begin{description}
			
			\item[Parallelogram to rectangle] Units of the system are transformed
			from a parallelogram shape to a rectangular shape.
			
			\item[Uncrinkle] Units within the rectangle are moved such that they all
			lie on a regular (and fully packed) 2D grid.
			
		\end{description}
		
		\subsection{Parallelogram to rectangle}
			
			The hexagonal torus topology is most naturally drawn as a parallelogram
			as illustrated in figures \ref{fig:hex-to-plane-node-native} and
			\ref{fig:hex-to-plane-native}. Two transformations are presented which
			transform these arangements of units into a rectangular form: shearing
			and slicing.
			
			A \SI{30}{\degree} shear transformation distorts networks such that the X
			and Y axes become orthogonal leading to a rectangular arrangement of
			units as illustrated in figures \ref{fig:hex-to-plane-node-shear} and
			\ref{fig:hex-to-plane-shear}.
			
			The slice transformation slices the units protruding from the
			left-hand-side of the parallelogram and moves them into the matching gap
			on the opposite side of the parallelogram as illustrated in figures
			\ref{fig:hex-to-plane-node-slice} and \ref{fig:hex-to-plane-slice}.
			 
			While the shear transformation introduces some distortion causing cables
			in the Z dimension to become $\sqrt{2}\times$ longer it leaves the
			pattern of wrap-around connections remains unchanged. By contrast, the
			slice transformation does not elongate any cables but changes the pattern
			of wrap-around connections. The exact pattern wrap-around connections
			produced when slicing depends on the aspect ratio of the network as
			illustrated in \ref{fig:slicing-examples} and influences the choice of
			folding technique applied as described later.
			
			\begin{figure}
				\center
				\begin{subfigure}[b]{0.32\linewidth}
					\center
					\buildfig{figures/hex-to-plane-node-native.tex}
					
					\caption{$7 \times 7$ node torus}
					\label{fig:hex-to-plane-node-native}
				\end{subfigure}
				\begin{subfigure}[b]{0.32\linewidth}
					\center
					\buildfig{figures/hex-to-plane-node-shear.tex}
					
					\caption{Sheared}
					\label{fig:hex-to-plane-node-shear}
				\end{subfigure}
				\begin{subfigure}[b]{0.32\linewidth}
					\center
					\buildfig{figures/hex-to-plane-node-slice.tex}
					
					\caption{Sliced}
					\label{fig:hex-to-plane-node-slice}
				\end{subfigure}
				
				\caption{Transformations of hexagonal toruses of nodes into a
				rectangular form. Thin lines show wrap-around links. Pointy-topped
				hexagons represent individual nodes.}
				\label{fig:hex-to-plane-node}
			\end{figure}
			
			\begin{figure}
				
				\begin{subfigure}[b]{0.32\linewidth}
					\center
					\buildfig{figures/hex-to-plane-native.tex}
					
					\caption{$4 \times 4$ triad torus}
					\label{fig:hex-to-plane-native}
				\end{subfigure}
				\begin{subfigure}[b]{0.32\linewidth}
					\center
					\buildfig{figures/hex-to-plane-shear.tex}
					
					\caption{Sheared}
					\label{fig:hex-to-plane-shear}
				\end{subfigure}
				\begin{subfigure}[b]{0.32\linewidth}
					\center
					\buildfig{figures/hex-to-plane-slice.tex}
					
					\caption{Sliced}
					\label{fig:hex-to-plane-slice}
				\end{subfigure}
				
				\caption{Transformations of hexagonal toruses of wrapped triples into a
				rectangular form.  Thin lines show wrap-around links. Flat-topped
				hexagons represent a wrapped triple of nodes.}
				\label{fig:hex-to-plane}
			\end{figure}
			
			\begin{figure}
				\center
				\buildfig{figures/slicing-examples.tex}
				\caption{Patterns of wiring in sliced systems of various sizes.}
				\label{fig:slicing-examples}
			\end{figure}
			
		\subsection{Uncrinkling}
			
			Though the transformmation step yields rectangular arrangements of units,
			these arrangements do not fall onto a regular 2D grid, with the exception
			of the shear transform on individual nodes. Figure \ref{fig:uncrinkling}
			illustrates how the various arrangements of hexagons may be moved to
			`uncrinkle' the units into a regular grid.
			
			\begin{figure}
				\center
				\begin{subfigure}[b]{0.44\linewidth}
					\center
					\buildfig{figures/uncrinkling-node-sheared.tex}
					
					\caption{$7 \times 7$ nodes, sheared}
					\label{fig:uncrinkling-node-sheared}
				\end{subfigure}
				\begin{subfigure}[b]{0.44\linewidth}
					\center
					\buildfig{figures/uncrinkling-node-sliced.tex}
					
					\caption{$7 \times 7$ nodes, sliced}
					\label{fig:uncrinkling-node-sliced}
				\end{subfigure}
				
				\vspace{1cm}
				
				\begin{subfigure}[b]{0.44\linewidth}
					\center
					\buildfig{figures/uncrinkling-sheared.tex}
					
					\caption{$4 \times 4$ triples, sheared}
					\label{fig:uncrinkling-sheared}
				\end{subfigure}
				\begin{subfigure}[b]{0.44\linewidth}
					\center
					\buildfig{figures/uncrinkling-sliced.tex}
					
					\caption{$4 \times 4$ triples, sliced}
					\label{fig:uncrinkling-sliced}
				\end{subfigure}
				
				\vspace{1em}
				
				\caption{Mapping rectangular arrangements of units into a square grid.
				Thick lines show how layers of units are uncrinkled.  Annotations show
				how the relative positions of nodes and wrapped triples change after
				uncrinkling.}
				\label{fig:uncrinkling}
			\end{figure}
			
			In the figure, the numbered units enumerate the different positions on
			the crinkle and those labelled alphabetically are those that immediately
			surround them. From this we can observe that uncrinkling largely
			preserves spatial locality but some distortion is introduced, separating
			previously neighbouring units. For example, in figure
			\ref{fig:uncrinkling-sheared}, the units labelled `1' and `i' are
			neighbours before uncrinkling but are separated by a (Euclidean) distance
			of $\sqrt{5}$ afterwards. Note that the distortion introduced depends on
			what part of the crinkle is considered, for example `2' and `a' have
			distance 2 but are logically connected in the same way.
		
		\subsection{Folding and Interleaving}
			
			Once a regular grid of units has been formed, this may be folded in the
			conventional way, eliminating long cables crossing from left-to-right and
			top-to-bottom as illustrated in \ref{fig:folding-sheared}.
			
			Unfortunately, for sliced systems whose dimensions are not of the ratio
			$1:2$, the pattern of wrap-around cables may also include some cables
			which do not cross directly to the opposite side of the system (refer
			back to figure \ref{fig:slicing-examples}). As a result of these
			connections, folding does not successfully eliminate all long
			connections. An exception to this rule is sliced systems whose dimensions
			are in the ratio $1:1$ where folding twice along the Y axis may
			successfully eliminate all wrap-around connections as illustrated in
			\ref{fig:folding-sliced}.
			
			\begin{figure}
				\begin{subfigure}{\linewidth}
					\center
					\buildfig{figures/folding-sheared.tex}
					\caption{$N \times M$ sheared systems and $N \times 2N$ sliced systems}
					\label{fig:folding-sheared}
				\end{subfigure}
				
				\vspace{1em}
				
				\begin{subfigure}{\linewidth}
					\center
					\buildfig{figures/folding-sliced.tex}
					\caption{$N \times N$ sliced systems}
					\label{fig:folding-sliced}
				\end{subfigure}
				
				\caption{Schematic illustrating elimination of long wrap-around links
				during folding. In each example a single link has been highlighted to
				aid in following the process.}
				\label{fig:folding}
			\end{figure}
			
			Once folded, the 2D grid is straight-forwardly interleaved as illustrated
			previously in figure \ref{fig:ring-folding}. The interleaving process
			introduces some additional distortion to the layout of units and causes
			most connections to become twice as long. For sliced $1:1$ systems, the
			additional fold results in additional overhead during interleaving since
			four layers of the system are interleaved.
		
		\subsection{Mapping to Cabinets}
			
			In the final step of the process is to map the 2D grid of units into
			positions in machine room cabinets as illustrated in figure
			\ref{fig:million-core-machine}. As illustrated in figure
			\ref{fig:cabinetisation}, first the grid of units is partitioned into
			groups of columns, one per cabinet, then groups of rows one per frame per
			cabinet. The units in each group are then allocated to slots within a
			frame, interleaving the rows of the groups as shown.
			
			\begin{figure}
				\center
				\buildfig{figures/cabinet-units.tex}
				
				\caption{An illustration of the physical construction of a
				multi-cabinet SpiNNaker system. (Note: network cables \emph{not}
				installed.)}
				\label{fig:cabinet-units}
			\end{figure}
			
			\begin{figure}
				\center
				\buildfig{figures/cabinetisation.tex}
				
				\caption{Mapping from 2D space to cabinets, frames and boards.}
				\label{fig:cabinetisation}
			\end{figure}
		
	\section{Cable installation}
		
		Cable installation is performed by a team of (human) technicians who must
		ensure that all network cables are correctly installed. As illustrated in
		previously in figure \ref{fig:cabinet-units}, the density of SpiNNaker's
		units, combined with the nature of the hexagonal torus topology, poses a
		challenge. To address this challenge I propose a semi-automated approach to
		cable installation which exploits diagnostic facilities available in the
		majority of super computers in order to guide technicians through the
		cabling process, interactively guiding installation and maintenance.
		
		\subsection{Interactive technician guidance and validation}
			
			While automated systems for validating cabling correctness are
			commonplace, these systems are typically used only after cabling has been
			completed \cite{lakner07}. As with other large-scale machines, SpiNNaker
			includes a low-bandwidth system management bus which may be used to
			interrogate network hardware and control diagnostic LEDs prior to the
			installation of the main SpiNNaker network interconnect.  Using these
			facilities I have constructed a tool called SpiNNer which interactively
			guides a technician, or team of technicians, through the cable
			installation process, validating each connection in real-time.
			
			Diagnostic LEDs mounted on each SpiNNaker board (figure
			\ref{fig:interactive-wiring-guide-leds}) are used to indicate the
			endpoints of the cable currently being installed. Simultaneously a
			Text-To-Speech (TTS) system gives an audible indication of which cable
			type is to be used and location of each connection.  Additionally, a GUI
			via a computer display (figure \ref{fig:interactive-wiring-guide-gui}).
			The centre of the display shows a `big-picture' perspective of the
			locations of the boards to be connected. The detailed views on the left
			and right indicate which of the six sockets on each board the cables
			should connect.
			
			\begin{figure}
				\center
				\begin{subfigure}[b]{0.40\textwidth}
					\begin{tikzpicture}
						\node (leds) [inner sep=0]
						      {\includegraphics[width=\textwidth]{figures/leds.jpg}};
						% Point at left LED
						\draw [white, <-, line width=0.4em]
						      ([shift={(-0.0cm, -0.6cm)}]leds.center)
						      -- ++(225:1cm);
						% Point at right LED
						\draw [white, <-, line width=0.4em]
						      ([shift={(1.1cm, -1.1cm)}]leds.center)
						      -- ++(225:1cm);
					\end{tikzpicture}
					
					\caption{Diagnostic LEDs}
					\label{fig:interactive-wiring-guide-leds}
				\end{subfigure}
				~
				\begin{subfigure}[b]{0.546\textwidth}
					\begin{tikzpicture}[thin, black!20!white]
						\node (screen) [inner sep=0]
						      {\includegraphics[width=\textwidth]{figures/wiring_guide_screenshot.png}};
						\draw (screen.south west) rectangle (screen.north east);
					\end{tikzpicture}
					
					\caption{Interactive wiring guide GUI}
					\label{fig:interactive-wiring-guide-gui}
				\end{subfigure}
				
				\caption{The SpiNNer interactive wiring guide uses a GUI,
				text-to-speech and diagnostic LEDs to assist during cable
				installation.}
				\label{fig:interactive-wiring-guide}
			\end{figure}
			
			SpiNNer also validates the connectivity of the system in real-time by
			polling the diagnostic interfaces of the network hardware at the
			endpoints of the cable being installed to determine if they are correctly
			connected. If a miswiring occurs, this is immediately detected and
			announced via TTS enabling the technician to immediately correct the
			error. Once a cable has been installed correctly, the software
			automatically advances to the next cable meaning direct interaction with
			the software by the technician is minimal. In practice, it is rarely
			necessary to refer to the GUI.
		
			SpiNNer presents the cables in an order intended to maximise ease of
			installation. Cables are installed in three groups with intra-frame
			cables being installed first, followed by intra-cabinet cables and
			inter-cabinet cables. Within each group, the tightest cables are
			installed first resulting in slacker cables naturally being installed
			over the top of already installed cables. By grouping cables in this
			manner, multiple technicians may work independently on the wiring within
			individual frames and cabinets.
			
			SpiNNer may also be used to repair or replace cables in the system.
			During maintenance, obstructing cables may be blindly removed alongside
			any cable being replaced. At the conclusion of the process, the wiring
			guide may be used to interactively guide re-installation of all removed
			cables.
		
		\subsection{Cable selection}
			
			Controlling slack is critical to ensuring reliable and maintainable
			cabling installations. If cables are too tight, cables and connectors can
			become easily damaged and when too slack, the excess cable obstructs
			other cables and can easily become tangled and damaged \cite{cisco07}. It
			has been observed that when ready-made cables are employed technicians
			frequently over-estimate the cable lengths required preferring to use
			overly long cables for all connections \cite{mazaris97}. To solve this
			problem, the SpiNNer wiring guide software dictates the cable lengths to
			be used by an installer based the rule of (three-)thumbs according to
			Mazaris \cite{mazaris97}. This rule suggests that an ideal amount of
			slack is approximately that which can be wrapped around three fingers.
			Specifically, the shortest available cable is selected which ensures at
			least \SI{5}{\centi\meter} of slack.
			
			The SpiNNer tool allocates cables assuming all cables take a Euclidean
			straight-line path between the endpoints of the connection. The result is
			that wiring is not routed through dedicated cable management structures
			but is simply suspended by its connectors in front of the machine. As
			demonstrated later, this unconventional approach leads neither to cooling
			problems nor increased maintenance effort.
	
	\section{Results and Evaluation}
		
		This stuff has been used and works. In this section I'll go over the
		overheads introduced by the various transformations and
		folding/interleaving steps and show a wiring scheme for a large machine
		which uses only short cables. I'll then show how SpiNNer was used to
		install this wiring plan into a very large machine without foobaring the
		cooling and in very little time. I'll also report on difficulty of
		maintenance.
		
		\subsection{Cable length}
			
			The transformation from regular hexagonal torus to a folded and
			interleaved form introduces some overhead to the cable lengths required.
			Using figure \ref{fig:uncrinkling} (page \pageref{fig:uncrinkling}), it
			is possible to compute the exact overhead introduced when each type of
			transformation proposed.
			
			For example, to compute the mean overhead introduced by the slicing
			technique when applied to units composed of wrapped triples, consider
			figure \ref{fig:uncrinkling-sliced}. The uncrinkling pattern used to
			transform this topology is a repeating pattern of two units, a pair of
			which have been labelled $1$ and $2$ respectively. Unit $1$ is
			immediately surrounded by six units labelled $a$, $b$, $c$, $2$, $g$ and
			$h$. Similarly, unit $2$ is surrounded by units $1$, $c$, $d$, $e$, $f$
			and $g$. Before the transformation, the distances, $D$, to each of these
			units is $1$ but after the transformation is applied, this is not always
			the case. Additionally, folding and interleaving introduce additional
			overhead. In this example, if the system is folded into $f_x$ columns and
			$f_y$ rows, the distances between previously neighbouring units become:
			
			\begin{equation*}
				\begin{aligned}[c]
					D_{1\,\leftrightarrow{}\,a} &= \sqrt{f_x^2 + f_y^2} \\
					D_{1\,\leftrightarrow{}\,b} &= f_y \\
					D_{1\,\leftrightarrow{}\,c} &= \sqrt{f_x^2 + f_y^2} \\
					D_{1\,\leftrightarrow{}\,2} &= f_x \\
					D_{1\,\leftrightarrow{}\,g} &= f_y \\
					D_{1\,\leftrightarrow{}\,h} &= f_x
				\end{aligned}
				\hspace{2cm}
				\begin{aligned}[c]
					D_{2\,\leftrightarrow{}\,1} &= f_x \\
					D_{2\,\leftrightarrow{}\,c} &= f_y \\
					D_{2\,\leftrightarrow{}\,d} &= f_x \\
					D_{2\,\leftrightarrow{}\,e} &= \sqrt{f_x^2 + f_y^2} \\
					D_{2\,\leftrightarrow{}\,f} &= f_y \\
					D_{2\,\leftrightarrow{}\,g} &= \sqrt{f_x^2 + f_y^2}
				\end{aligned}
			\end{equation*}
			
			From these values, the mean and maximum connection distances after
			folding and interleaving may be computed. Table
			\ref{tab:transform-overhead} gives the mean and maximum connection
			distances for each of the four transformations described in this chapter.
			
			\begin{table}
				\begin{subtable}[b]{\linewidth}
					\center
					\begin{tabular}{l c c}
						\toprule
						& Shear & Slice \\
						\addlinespace
						Nodes &
							$\frac{f_x + f_y + \sqrt{f_x^2 + f_y^2}}{3}$ &
							$\frac{f_x + f_y + \sqrt{f_x^2 + f_y^2}}{3}$ \\
						\addlinespace
						Triples &
							$\frac{7f_x + 3\sqrt{f_x^2 + f_y^2} + \sqrt{(2f_x)^2 + f_y^2}}{9}$ &
							$\frac{f_x + f_y + \sqrt{f_x^2 + f_y^2}}{3}$ \\
						\bottomrule
					\end{tabular}
					
					\caption{Mean}
					\label{tab:transform-overhead-mean}
				\end{subtable}
				
				\vspace{1em}
				
				\begin{subtable}[b]{\linewidth}
					\center
					\begin{tabular}{l c c}
						\toprule
						& Shear & Slice \\
						\addlinespace
						Nodes &
							$\sqrt{f_x^2 + f_y^2}$ &
							$\sqrt{f_x^2 + f_y^2}$ \\
						\addlinespace
						Triples &
							$\sqrt{(2f_x)^2 + f_y^2}$ &
							$\sqrt{f_x^2 + f_y^2}$ \\
						\bottomrule
					\end{tabular}
					
					\caption{Maximum}
					\label{tab:transform-overhead-max}
				\end{subtable}
				
				\caption{Overheads introduced when transforming unit positions onto a
				regular grid.}
				\label{tab:transform-overhead}
			\end{table}
			
			From these results it is evident that shearing and slicing networks
			whose units are nodes result in identical mean and maximum overhead in
			cable length when folded similarly. Since sliced networks may require
			folding more than once along each axis the shearing approach is
			preferable in general.
			
			For networks constructed from units of wrapped triples, the slicing
			approach suffers the same mean and maximum overhead has networks of
			nodes, and less overhead than shearing for the same number of folds. For
			systems with an aspect ratio of $1:2$ (where both slicing and shearing
			require $f_x = f_y = 2$), the slicing transformation yields lower mean
			and maximum overhead than shearing. For all other aspect ratios (where
			slicing requires a greater degree of folding) the shearing technique
			produces lower overhead. The recommended transformations for a given
			machine are thus given in table \ref{tab:transform-recommended}.
			
			\begin{table}
				\center
				\begin{tabular}{lcc}
					\toprule
					                         & $1:2$  & Other \\
					\addlinespace
					\multirow{2}{*}{Nodes}   & Either & Shear\\
					                         & \footnotesize $\mu\approx2.28 \quad \vee\approx2.83$
					                         & \footnotesize $\mu\approx2.28 \quad \vee\approx2.83$\\
					\addlinespace
					\multirow{2}{*}{Triples} & Slice  & Shear\\
					                         & \footnotesize $\mu\approx2.28 \quad \vee\approx2.83$
					                         & \footnotesize $\mu\approx3.00 \quad \vee\approx4.47$\\
					\bottomrule
				\end{tabular}
				
				\caption{Recommended transformation and folding scheme for different
				system types. $\mu$ and $\vee$ give the mean and maximum wire
				distortion introduced, respectively.}
				\label{tab:transform-recommended}
			\end{table}
			
			\begin{figure}
				\center
				\buildfig{figures/million-core-machine.tex}
				
				\caption{Cabling plan for a \num{1036800} core SpiNNaker
				machine's \num{3600} cables.}
				\label{fig:million-core-machine}
			\end{figure}
			
			Following folding and mapping to physical locations, the cabling plans
			for large machines require no large gaps to be spanned.  The largest
			planned SpiNNaker machine, illustrated in figure
			\ref{fig:million-core-machine}, will be \SI{6}{\meter} wide but the
			largest gap any cable must span is \SI{66}{\centi\meter}. This distance
			is well within the \SI{1}{\meter} allowed by the hardware and cables.
			
		\subsection{Installation practicality}
			
			\begin{table}
				\center
				\begin{tabular}{lrr@{$\,$}l}
					\toprule
						System & Number of Cables & \multicolumn{2}{r}{Installation time} \\
					\midrule
						24 boards  & \num{72}   & \num{10} & \si{\minute}         \\
						1 cabinet  & \num{360}  & \num{4}  & \si{\hour}$^\dagger$ \\
						2 cabinets & \num{720}  & \num{2}  & \si{\hour}           \\
						5 cabinets & \num{1800} & ?        &                      \\
					\bottomrule
				\end{tabular}
				
				\caption{Installation times for various sizes of machine.
				$\dagger$~This machine was installed without real-time validation of
				connectivity.}
				\label{tab:install-time}
			\end{table}
			
			A number of SpiNNaker machines of various scales have been assembled
			using the techniques described in this chapter ranging from single frames
			of 24 boards to a half-scale 5 cabinet machine. Table
			\ref{tab:install-time} gives the reported installation times of each of
			these machines.
			
			The single cabinet machine's installation time is notably
			disproportionate to its size. When this system was assembled, real-time
			connection validation was not yet available. As a result, though cable
			installation was rapid correcting errors was extremely costly, requiring
			careful retracing of many installation steps.
			
			TODO: TALK ABOUT MULTI-PERSON-WIRING IN PRACTICE ON FIVE CABINET MACHINE.
			
			\begin{figure}
				
				\center
				\buildfig{figures/wire-length-histogram.tex}
				
				\caption{Histogram of connection distances in a ten-cabinet,
				one-million core SpiNNaker machine annotated with the suggested cable
				length.}
				\label{fig:wire-length-histogram}
				
			\end{figure}
			
			FIGURE \ref{fig:wire-length-histogram} SHOWS THE DISTRIBUTION OF CABLE
			LENGTHS REQUIRED. IN PRACTICE THE SLACK ALLOCATED PROVED ADEQUATE. AS
			SHOWN IN FIGURE \ref{fig:install-histogram}, THE MOST IMPORTANT FACTOR IS
			WHETHER LEAVING THE FRAME OR NOT. LEAVING THE FRAME TAKES THE LONGEST.
			
			\begin{figure}
				\builddata{data/build_connection_log.tex}
				\buildfig{figures/install-histogram.tex}
				
				\caption{Histogram of cable installation times}
				\label{fig:install-histogram}
			\end{figure}
			
			TODO: COMPARE DIRECTLY WITH INSTALL TIMES REPORTED IN LITERATURE.
		
		\subsection{Thermal Impact}
			
			TODO: SHOW HOW TEMPERATURE IS CHANGED
			
		\subsection{Maintenance}
			
			TOOD: QUANTIFY CABLE REMOVALS REQUIRED. EXPERIMENT: REMOVE/REPLACE RANDOM
			BOARDS AND MEASURE TIME TAKEN, CABLES REMOVED. COMPARE WITH STANDARD DATA
			CENTRE WIRING

	\chapter{Finding shortest path vectors in SpiNNaker's network}
	
	Once a SpiNNaker machine has been constructed as described in the previous
	chapter, its network forms a large hexagonal torus topology. To exploit this
	network routing algorithms must be used to generate routes for packets to
	follow between nodes. As well as ensuring that packets arrive at the correct
	destination, routing algorithms often attempt to produce routes which make
	efficient use of the network. This often involves attempting to reduce
	congestion by ensuring packets do not travel further through the network than
	absolutely necessary.
	
	Many popular routing algorithms for torus topologies, including all published
	algorithms designed for SpiNNaker's hexagonal torus topology
	\cite{davies12,navaridas14}, internally function by computing shortest path
	vectors and generating routes from them. Existing methods of calculating
	shortest path vectors in hexagonal torus topologies are unable to generate
	all possible shortest path vectors and, as a result, reduces the diversity of
	routes produced by routing algorithms, potentially worsening network
	contention.
	
	In this chapter I describe a novel technique for computing shortest path
	vectors in hexagonal torus topologies which yields \emph{all} possible
	shortest path vectors for any pair of nodes. Further, implementations of this
	new technique execute an order of magnitude faster than the existing
	alternatives.
	
	\section{Related work}
		
		TODO: INTRODUCE SECTION
		
		\begin{figure}
			\center
			
			\begin{subfigure}{\linewidth}
				\center
				\buildfig{figures/distance-map-mesh.tex}
				\caption{2D mesh topology}
				\label{fig:distance-map-mesh}
			\end{subfigure}
			
			\vspace{1em}
			
			\begin{subfigure}{\linewidth}
				\center
				\buildfig{figures/distance-map-torus.tex}
				\caption{2D torus topology}
				\label{fig:distance-map-torus}
			\end{subfigure}
			
			\vspace{1em}
			
			\begin{subfigure}{\linewidth}
				\center
				\buildfig{figures/distance-map-hex-mesh.tex}
				\caption{Hexagonal mesh topology}
				\label{fig:distance-map-hex-mesh}
			\end{subfigure}
			
			\vspace{1em}
			
			\begin{subfigure}{\linewidth}
				\center
				\buildfig{figures/distance-map-hex-torus.tex}
				\caption{Hexagonal torus topology}
				\label{fig:distance-map-hex-torus}
			\end{subfigure}
			
			\caption{Plots showing distance from various locations marked
			         {\color{red}$\times$}. Darker areas are further away. Contour
			         lines show equidistant points.}
			\label{fig:distance-map}
		\end{figure}
		
		\subsection{Mesh Networks}
			
			In a (non-hexagonal) mesh network topology, shortest path vectors are
			computed by taking the element-wise difference between the source and
			destination nodes' coordinates.
			
			\begin{figure}
				\center
				\buildfig{figures/mesh-topology-coordinates.tex}
				\caption{An example 2D mesh network with example shortest-path routes
				from `A' to `B' and `B' to `C'.}
				\label{fig:mesh-topology-coordinates}
			\end{figure}
			
			For example, figure \ref{fig:mesh-topology-coordinates} illustrates a 2D
			mesh topology. In this topology, the nodes labelled `A', `B' and `C' have
			position vectors $(1, 2)$, $(4, 5)$ and $(6, 1)$ respectively. The
			shortest path vector from node `A' to `B' is thus simply $(4, 5) - (1, 2)
			= (3, 3)$ and from `B' to `C' is $(6, 1) - (4, 5) = (2, -4)$.
			
			A route may be produced from a shortest path vector by advancing the
			number of hops specified for each dimension in the vector. For example
			any permutation of the hops X$^+\,$X$^+\,$X$^+\,$Y$^+\,$Y$^+\,$Y$^+$, an
			example of which is included in the figure. Likewise a route from `B' to
			`C' may be constructed from any permutation of
			X$^+\,$X$^+\,$Y$^-\,$Y$^-\,$Y$^-\,$Y$^-$.
			
			Many popular routing algorithms such as Dimension Order Routing (DOR),
			Right-Turn Only Routing (RTOR) and Longest Dimension First Routing (LDFR)
			\cite{dally04,davies12} directly follow the above procedure and just
			prescribe a specific permutation of hop order. For example, DOR produces
			routes with X hops first, Y hops second and so on.
			
			The length of routes produced from a shortest path vector have a number
			of hops proportional to the magnitude of the vector, thus a shortest path
			vector yields a route with the minimum number of hops. For a two
			dimensional vector $(a, b)$ the magnitude is given as:
			%
			\begin{equation}
				\| (a, b) \| = \lvert a \rvert + \lvert b \rvert
			\end{equation}
		
		\subsection{Torus Networks}
			
			Computing shortest path vectors in (non-hexagonal) torus topologies is
			also straight forward. As an example, lets find the shortest path vector
			from node `A' to `B' in the 2D torus topology shown in figure
			\ref{fig:torus-shortest-path-example}. First, both nodes are translated
			such that the source node, `A', is at the centre of the network (figure
			\ref{fig:torus-shortest-path-translate}). Note that this translation may
			result in the destination node `wrapping around' the network. After
			translation, the shortest path vector is computed as in a mesh topology.
			As illustrated in \ref{fig:torus-shortest-path-routed}, the computed
			shortest path vector may be used to produce routes between the two nodes
			in their original positions.
			
			\begin{figure}
				\center
				\begin{subfigure}{0.3\linewidth}
					\center
					\buildfig{figures/torus-shortest-path-example.tex}
					\caption{Original}
					\label{fig:torus-shortest-path-example}
				\end{subfigure}
				\begin{subfigure}{0.3\linewidth}
					\center
					\buildfig{figures/torus-shortest-path-translate.tex}
					\caption{Translated}
					\label{fig:torus-shortest-path-translate}
				\end{subfigure}
				\begin{subfigure}{0.3\linewidth}
					\center
					\buildfig{figures/torus-shortest-path-routed.tex}
					\caption{Routed}
					\label{fig:torus-shortest-path-routed}
				\end{subfigure}
				
				\caption{Finding shortest paths in a 2D torus topology.}
				\label{fig:torus-shortest-path}
			\end{figure}
			
			This process works because vectors from the centre (though not other
			locations) of a torus topology are identical to those in mesh topologies
			(see figures \ref{fig:distance-map-mesh} and
			\ref{fig:distance-map-torus}).
		
		\subsection{Hexagonal Mesh Networks}
			
			In hexagonal mesh topologies it is conventional to define three `axes' X,
			Y and Z as shown in figure \ref{fig:hex-mesh-topology-coordinates}
			\cite{patel15}. In this example, the three labelled nodes `A', `B' and
			`C' may be given position vectors such as $(1, 1, 0)$, $(3, 2, 0)$ and
			$(0, 0, -7)$ respectively. As in other mesh networks, a vector between
			two nodes is found by subtracting the nodes' vectors. For example, a
			vector from `A' to `B' is $(3, 2, 0) - (1, 1, 0) = (2, 1, 0)$. This
			vector can then be converted into a route in the same way as a mesh
			network by taking any permutation of the three hops  X$^+\,$X$^+\,$Y$^+$.
			
			\begin{figure}
				\center
				\buildfig{figures/hex-mesh-topology-coordinates.tex}
				\caption{An example hexagonal mesh network topology.}
				\label{fig:hex-mesh-topology-coordinates}
			\end{figure}
			
			As explained in detail in appendix \ref{app:minimal-hex-coordinates},
			there are an infinite number of vectors between any two points. For
			example, the vectors $(1, 0, -1)$ and $(3, 2, 1)$ also reach node `B'
			from `A' in the example. However, for a given pair of nodes, there is
			always a single, unique vector whose magnitude is minimal which is
			given by the function:
			%
			\begin{equation}
				\operatorname{minimiseVector}(x,y,z)
					= (x,y,z) - \operatorname{median}(x,y,z) \cdot (1,1,1)
			\end{equation}
			%
			An important side-effect of this function is that a minimised vector will
			always contain at least one zero element meaning that shortest path
			routes will use at most two of the three available dimensions.
			
			To aid the reader's intuition, figure \ref{fig:distance-map-hex-mesh}
			illustrates how distances grow in a hexagonal mesh topology.
		
		\subsection{Hexagonal Torus Networks}
			
			Unfortunately, unlike non-hexagonal torus topologies, the translation
			technique cannot be used to compute shortest path vectors. As illustrated
			in figures \ref{fig:distance-map-hex-mesh} and
			\ref{fig:distance-map-hex-torus}, shortest path vectors from the center
			of a hexagonal mesh network are not equivalent to those of a hexagonal
			torus network.
			
			Prior research into routing in SpiNNaker's network has been based on the
			INSEE \cite{navaridas09,ghasempour15} interconnect simulator. Internally
			INSEE tries a set of twelve candidate vectors and picks the shortest as
			the shortest path vector to use for routing.
			
			\begin{figure}
				\center
				\begin{subfigure}{0.45\linewidth}
					\center
					\buildfig{figures/insee-vector-candidates-no-wrap.tex}
					\caption{$(\Delta_\textrm{X}, \Delta_\textrm{Y}) = (5,3)$}
					\label{fig:insee-vector-candidates-no-wrap}
				\end{subfigure}
				\begin{subfigure}{0.45\linewidth}
					\center
					\buildfig{figures/insee-vector-candidates-wrap-x.tex}
					\caption{$(\Delta'_\textrm{X}, \Delta_\textrm{Y}) = (-3,3)$}
					\label{fig:insee-vector-candidates-wrap-x}
				\end{subfigure}
				
				\vspace{1em}
				
				\begin{subfigure}{0.45\linewidth}
					\center
					\buildfig{figures/insee-vector-candidates-wrap-y.tex}
					\caption{$(\Delta_\textrm{X}, \Delta'_\textrm{Y}) = (5,-5)$}
					\label{fig:insee-vector-candidates-wrap-y}
				\end{subfigure}
				\begin{subfigure}{0.45\linewidth}
					\center
					\buildfig{figures/insee-vector-candidates-wrap.tex}
					\caption{$(\Delta'_\textrm{X}, \Delta'_\textrm{Y}) = (-3,-5)$}
					\label{fig:insee-vector-candidates-wrap}
				\end{subfigure}
				
				\vspace{1em}
				
				% Key
				\begin{tikzpicture}[thick]
					\coordinate (last);
					
					% #1 colour
					% #2 label
					\newcommand{\colourkeyentry}[2]{
						\node [#1] [right=of last, fill, rectangle, minimum size=1em] (last) {};
						\node [right=0 of last] (last) {#2};
					}
					
					\colourkeyentry{cb3class0}{$(\textrm{X}, \textrm{Y}, 0)$}
					\colourkeyentry{cb3class1}{$(\textrm{X} - \textrm{Y}, 0, - \textrm{Y})$}
					\colourkeyentry{cb3class2}{$(0, \textrm{Y} - \textrm{X}, - \textrm{X})$}
					
				\end{tikzpicture}
				
				\caption{The twelve candidate shortest-path vectors considered by INSEE
				represented as dimension-order routes. $W=H=8$,
				$(\Delta_\textrm{X},\Delta_\textrm{Y}) = (5, 3)$ and
				$(\Delta'_\textrm{X},\Delta'_\textrm{Y}) = (-3, -5)$.}
				\label{fig:insee-vector-candidates}
			\end{figure}
			
			The twelve vectors considered are constructed as follows.
			
			First a shortest path vector from the source to target node are
			constructed as if the network was a 2D mesh yielding a vector
			$(\Delta_\textrm{X},\Delta_\textrm{Y})$. From this, another vector
			$(\Delta'_\textrm{X},\Delta'_\textrm{Y})$, is defined:
			%
			\begin{align}
				\Delta'_\textrm{X} &= \Delta_\textrm{X} - \operatorname{sign}(\Delta_\textrm{X})W
				\\
				\Delta'_\textrm{Y} &= \Delta_\textrm{Y} - \operatorname{sign}(\Delta_\textrm{Y})H
			\end{align}
			%
			Where $W$ and $H$ are the width and height of the network respectively
			(in nodes). This new vector yields routes from the source to destination
			node that in a torus topology that \emph{always} wrap around the `X' and
			`Y' dimensions.
			
			From the pair of vectors defined, four possible 2D vectors can be
			produced: $(\Delta_\textrm{X},\Delta_\textrm{Y})$,
			$(\Delta'_\textrm{X},\Delta_\textrm{Y})$,
			$(\Delta_\textrm{X},\Delta'_\textrm{Y})$ and
			$(\Delta'_\textrm{X},\Delta'_\textrm{Y})$. Further, each 2D vector may be
			converted into one of three 3D vectors where either X, Y or Z are zero
			for a total of twelve candidate vectors.  Figure
			\ref{fig:insee-vector-candidates} illustrates all twelve candidate
			vectors for an example pair of nodes.
			
			\begin{figure}
				\center
				\buildfig{figures/xyz-protocol-regions.tex}
				
				\caption{The four regions defined by the XYZ-protocol.}
				\label{fig:xyz-protocol-regions}
			\end{figure}
			
			A more efficient technique is proposed by Hoffmann and D\'es\'erable
			called the XYZ-Protocol \cite{hoffmann15,hoffmann11}. If the source and
			destination nodes are translated such that the source node lies at the
			center of the topolgoy, the destination will lie in one of four regions
			illustrated in figure \ref{fig:xyz-protocol-regions}.
			
			If the destination lies in regions 1 or 4, a route may be constructed as
			if in a hexagonal mesh topology.
			
			Alternatively, if the destination lies in regions 2 or 3, the algorithm
			tests whether taking a mesh-like route within the region or
			wrapping-around either the X or Y dimension yields the shorter vector.
			The shortest of these vectors is then chosen.
			
			TODO DESCRIBE SPIRAL ROUTES.
			
			TODO DESCRIBE RTOR AND LDFR.
		
	\section{Dimension order routing in hexagonal torus topologies}
		
		So, existing solutions have two problems: trying 12 options and picking one
		is a bit kludgey and there are actually more options than that.
		
		\subsection{Simpler minimal hexagonal torus vectors}
			
			If you redraw your route such that it is sourced from bottom left corner
			(which we'll now call (0, 0)), there are four possible ways this route
			could wrap.
			
			TODO: DESCRIBE WAYS OF WRAPPING
			
			For each of these wrappings, all the possible routes we can take are
			strictly limited in terms of the dimensions used since we're stuck in a
			corner.
			
			In each case, the function computing the minimal hex vector function
			simplifies to a much simpler operation.
			
			TODO: DESCRIBE MINIMUM VECTOR LENGTH FUNCTIONS FOR EACH CASE
			
			This gives us a cheap way to compute which of the four possible wrappings
			are shortest. Based on this we can pick one of at most two (is this
			easily provable?) vectors in some fair manner to reduce load. This vector
			can then be minimised in the usual way.
			
			This also leads to confirming a theoretical result giving the length of a
			shortest path in a hexagonal torus topology.
			
			TODO: DESCRIBE HOW TO GET LENGTH FUNCTION AND COMPARE WITH \cite{xiao04}
		
		\subsection{Generating spiralling routes}
			
			In non-hexagonal torus topologies the previous technique would reveal all
			possible shortest vectors (e.g. in cases where you can wrap more than one
			way). Unfortunately, due to the addition of a non-orthogonal axes,
			hexagonal toruses also have other cases when the width and height do not
			match.
			
			TODO: TORUS SPIRALLING EXAMPLE
			
			It is possible to calculate the maximum number of spirals thus:
			
			TODO: DESCRIBE HOW MAX NUMBER OF SPIRALS IS COMPUTED
			
			Given a number of spirals, the vector can be updated this (note that the
			change does not add a multiple of (1, 1, 1) but also does not result in
			the vector changing length and thus becoming non-minimal!).
			
			TODO: DESCRIBE TRANSFORMATION
			
			TODO: PROVE THAT MINIMALITY IS MAINTAINED
		
		\subsection{Proof of completeness}
		
			TODO: PROOF OF COMPLETENESS BY EXHAUSTIVE SEARCH
	
		\subsection{Conclusions}
			
			This approach is simpler, smaller, has sounder theoretical basis, and
			finds more routes than alternatives. This is good for load balancing and
			fault avoidance and also good for completeness.


	\chapter{Routing packets in large SpiNNaker machines}
	
	\label{sec:routing}
	
	So far, this thesis has focused on tackling the practical challenges
	resulting from SpiNNaker's hexagonal torus network topology. In this chapter,
	I adjust my focus towards the practical challenges resulting from SpiNNaker's
	large scale. Faults in large systems are inevitable and in the half-million
	core, \num{28800} chip SpiNNaker machine recently completed at the University
	of Manchester, around \SI{1}{\percent} of chips exhibited faults\footnote{Of
	the faulty chips discovered, the vast majority of faults are attributed to a
	currently unknown SDRAM failure}. These faults result in gaps and broken
	links in the network topology which routing algorithms must avoid in order to
	ensure correct system operation.
	
	In this chapter I tackle the problem of extending existing routing algorithms
	for SpiNNaker's network to enable them to route-around known, static faults.
	Though dynamic or transient faults may also occur, in this work such faults
	are ignored and other techniques, such as protocol-level fault tolerance, are
	relied on instead.
	
	Numerous heuristic-based fault-tolerant routing algorithms exist which target
	different network topologies and router architectures. Unfortunately as I
	will show, these algorithms are not portable and rely on or attempt to work
	around specific features of their target network architecture. In particular,
	existing work is dominated by the challenge of developing routing schemes
	which avoid or defuse network deadlocks. Due to SpiNNaker's unconventional
	use of timeout-based flow-control, it is not subject to the routing
	restrictions present in other architectures intended to cope with deadlocks.
	
	In this chapter I introduce a graph-search based post-processing step for
	non-fault-tolerant routing algorithms which guarantees routability in
	SpiNNaker systems without disconnected subregions. I also demonstrate that
	this technique introduces both negligible computational overhead to the
	routing algorithm runtime and resulting network performance.
	
	TODO: NOTE THE FAULT RATES ENCOUNTERED IN PRACTICE...
	
	\section{Related work}
		
		Existing work on routing in SpiNNaker's network has ignored the challenge
		of avoiding faults and instead focused on producing efficient multicast
		routes. As a result this section is broken into two halves. In the first
		half I survey the existing non-fault-tolerant approaches to routing used in
		SpiNNaker to-date. In the second I discuss the approaches to fault tolerant
		routing taken in other systems.
		
		\subsection{Multicast routing in SpiNNaker}
			
			Various fault-intolerant multicast routing algorithms exist for many
			networks and a number have been proposed and evaluated specifically in the
			context of SpiNNaker.
			
			In 2012, Davies \emph{et al.} evaluated the use of three common torus
			routing algorithms in SpiNNaker and found that simple oblivious routing is
			suitable for typical neural applications \cite{davies12}. The three
			routing techniques are:
			
			\begin{description}
				
				\item[Dimension Order Routing (DOR)] Packets are routed along each
				dimension (e.g. $X$, $Y$ and $Z$) in turn until no further hops are
				available in that direction.  The order in which the dimensions are
				traversed is fixed.
				
				\item[Right Turn Only Routing (RTOR)] As in DOR except the dimension
				order is chosen such that routes only contain right-turns.
				
				\item[Longest Dimension First Routing (LDFR)] As in DOR except the
				dimension order is chosen in descending order of number of hops in each
				dimension.
				
			\end{description}
			
			These unicast routing techniques are converted into a multicast routing
			algorithm by merging together the routes produced between the source node
			and each destination node as illustrated in figure
			\ref{fig:simple-routers}.
			
			\begin{figure}
				\center
				\begin{subfigure}{0.3\linewidth}
					\center
					\buildfig{figures/simple-routers-dor.tex}
					
					\caption{DOR}
					\label{fig:simple-routers-dor}
				\end{subfigure}
				\begin{subfigure}{0.3\linewidth}
					\center
					\buildfig{figures/simple-routers-rtor.tex}
					
					\caption{RTOR}
					\label{fig:simple-routers-dor}
				\end{subfigure}
				\begin{subfigure}{0.3\linewidth}
					\center
					\buildfig{figures/simple-routers-ldfr.tex}
					
					\caption{LDFR}
					\label{fig:simple-routers-dor}
				\end{subfigure}
				
				\caption{Example multicast routes produced by merging together unicast
				routes from a central source node to each destination node.}
				\label{fig:simple-routers}
			\end{figure}
			
			In 2014, Navaridas \emph{et al.} introduced two new algorithms, `Enhanced
			Shortest Path Routing' (ESPR) and `Neighbourhood Exploring Routing' (NER)
			which produce multicast routing trees with fewer total hops
			\cite{navaridas14}. In both algorithms, routes are generated sequentially
			for each of the destinations of a route using LDFR. Unlike LDFR, however,
			these algorithms search a limited area of the network for other,
			already-connected destination nodes to which LDFR routes may be
			constructed. If no suitable destination is found, a LDFR route is
			constructed to the source node. Figure \ref{fig:search-regions} illustrates
			the shape of the searched regions of each algorithm. ESPR searches the
			trapezoidal region between the source and destination nodes while NER
			searches a fixed radius out from the destination node.
			
			\begin{figure}
				\center
				\begin{subfigure}{0.45\linewidth}
					\center
					\buildfig{figures/search-regions-espr.tex}
					
					\caption{ESPR}
					\label{fig:search-regions-espr}
				\end{subfigure}
				\begin{subfigure}{0.45\linewidth}
					\center
					\buildfig{figures/search-regions-ner.tex}
					
					\caption{NER}
					\label{fig:search-regions-espr}
				\end{subfigure}
				
				\caption{The ESPR and NER algorithms attempt to connect the node marked
				`D' to the closest node in the shaded region which is connected to the
				source node, `S'. If no connected node is found in the shaded region, the
				LDFR route is taken to `S'. The dotted line indicates the route chosen
				from `D'.}
				\label{fig:search-regions}
			\end{figure}
			
			Unfortunately none of these routing algorithms make any allowance for the
			avoidance of network faults. As a result their utility in real-world
			systems is limited.
		
		\subsection{Fault-tolerant routing}
			
			Numerous fault-tolerant routing algorithms have been proposed for
			super-computer networks. These algorithms are largely constrained by the
			need to maintain deadlock freedom. Since SpiNNaker's routers employ a
			timeout based deadlock-breaking strategy, much of this effort is
			unnecessary in SpiNNaker. As described below, this frequently renders
			existing fault-tolerant routing algorithms unnecessarily complex and
			inflexible.
			
			Deadlocks occur in a network if a cyclic dependency exists on any limited
			resource in the network. For example, as illustrated in figure
			\ref{fig:ring-deadlock}, in a ring network a deadlock may form when every
			node is waiting on the next node to accept a packet before accepting new
			packets from the previous node.
			
			\begin{figure}
				\center
				\buildfig{figures/ring-deadlock.tex}
				
				\caption{A deadlock in a ring network where each node is waiting for
				the next to accept a packet before accepting any further packets.}
				\label{fig:ring-deadlock}
			\end{figure}
			
			To prevent deadlocks, combinations of router microarchitectural features
			and routing restrictions may be employed. For example, a simple
			deadlock-free routing algorithm for mesh and torus networks mandates the
			use of DOR \cite{dally93}. Packets travelling in a -ve direction along
			each axis take priority over those travelling in a +ve direction. Packets
			travelling along the Y axis take priority over those travelling along the
			X dimension. Given these rules it is possible to define a total ordering
			on all hops in the network. Figure \ref{fig:deadlock-free-dor}
			illustrates a $3\times3$ mesh network whose hops have been numbered
			according to the total ordering defined above.  Any `X-then-Y' DOR route
			through this network results in the use of hops labelled with strictly
			increasing numbers. As a result, no cyclic dependencies (and thus no
			deadlocks) may occur.
			
			\begin{figure}
				\center
				\buildfig{figures/deadlock-free-dor.tex}
			
				\caption{Deadlock-free routing of two example routes using DOR in a 2D
				mesh topology. The numbers of the hops taken by each route are given on
				the right.}
				\label{fig:deadlock-free-dor}
			\end{figure}
			
			Unfortunately, the routing restrictions imposed to ensure deadlock
			freedom can result in fault-intolerant routing. In the example above, if
			the node at the bottom-right corner of the figure was faulty, the dotted
			example route would be blocked as no alternative routes are allowed.
			
			In practice, the routing rules used may be more relaxed, for example
			requiring that any route whose length is equal to a DOR must exist to
			guarantee routability \cite{rodrigo09}.
			
			Alternative routing strategies take a hybrid approach whereby an
			efficient but fault-intollerant routing algorithm is used where possible
			and in the presence of faults a less efficient but more robust strategy
			is employed. For example, the Immucube network architecture employs three
			virtual networks which operate independently over the same physical links
			\cite{puente07}. Initially messages are routed using a high-performance
			but potentially-deadlockable routing scheme in the first virtual network.
			If a deadlock is occurs, the deadlocked packet is dropped into the second
			virtual network in which packets are routed using a less efficient but
			deadlock-free but fault-intolerant routing algorithm. Finally, upon
			encountering a fault, packets are dropped onto the third virtual network
			which forms a ring network routing packets to every node in the network.
			
			Releated approaches \cite{mejia06,boppana95} divide the network into
			regions in which different routing rules are enforced to ensure deadlock
			freedom and, when required, fault tolerance.
			
			TODO FIGURE?
			
			The BlueGene/L supercomputer \cite{adiga02} uses DOR for its torus
			network and implements fault-tolerance by sacrificing otherwise
			functioning `lamb' nodes to ensure no route passes through a known dead
			link \cite{ho04}. In figure \ref{fig:lamb-nodes} an example scenario is
			shown where a single dead node is present and all nodes in the same row
			or column as the dead node have been made into lamb nodes. The lamb nodes
			may not be used in an application except as a through-route for other
			traffic. This pattern of lamb nodes guarantees that all dimension-order
			routes between all pairs of non-lamb nodes are not obstructed by the
			faulty node. This approach trades use of higher performance routing
			logic for wasted resources. This type of approach is most appropriate
			when algorithmic routing is used and routing rules are inflexible.
			
			\begin{figure}
				\center
				\buildfig{figures/lamb-nodes.tex}
				
				\caption{`Lamb' nodes may be disabled to ensure DOR will never
				encounter a fault.}
				\label{fig:lamb-nodes}
			\end{figure}
			
			Other algorithms proposed for the BlueGene architecture attempt to avoid
			the need for lamb nodes by generating routes which reach their destination
			via a `proxy' node \cite{gomez04}. By appropriately selecting the location
			of such a proxy, the existing routing algorithm used by the system can be
			guaranteed to select a route free of faults.
			
			TODO: EXAMPLE OF PROXY ROUTING TO AVOID FAULT
			
			Finally, many algorithms in in the field are distributed and use only local
			information along with limited information from their peers to generate
			routes \cite{fick09b}. In SpiNNaker, route generation is conventionally
			carried out centrally since no special on-chip hardware facilities exist
			for route generation. Centralised route generation also enables the routing
			algorithm to consider all available routes. As a result, there is little
			incentive for the use of distributed routing algorithms on SpiNNaker since
			global system information could be compactly shared for one-off routing
			passes.
			
			Algorithms for other architectures such as IP networks tend to be poor fits
			for static, regular network topologies since they use expensive graph-based
			algorithms for route discovery which aren't necessary here. They also tend
			to heavily feature graph topology discovery etc. which aren't needed here.
			
			Work on fault-tolerance in data centre networks does exploit the regularity
			of the network topology in routing algorithms \cite{guo08,liao12}.
			Unfortunately, the approaches used are not general enough to be applied to
			mesh-like topologies such as the one in SpiNNaker.
			
			Outside the field of computer networks, routing algorithms used to route
			wires across the surfaces of chips are required to solve similar problems
			to fault-tolerant network routing problems in mesh networks. Like mesh
			networks, the routes are defined within a regular Manhattan geometry and
			congested areas, rather than faults must be avoided by the algorithms
			\cite{kahng11}.  Unfortunately, these algorithms are designed for
			occasional batch operation prior to the multi-month process of chip
			manufacturing and so runtimes of hours or days are commonplace
			\cite{nam08}. As such these algorithms would be inappropriate for use
			with applications such as SpiNNaker where users' applications tend to be
			short-lived and thus routing should not be allowed to dominate runtime.
	
	\section{Partial graph search repair}
		
		In this section I introduce a novel post-processing algorithm, Partial
		Graph Search (PGS) repair, for routes produced by non-fault-tolerant
		routing algorithms.
		
		PGS repair guarantees routability for networks with no disconnected
		subregions by using a graph search algorithm to route around faults in the
		original route.  General-purpose graph search algorithms such as Breadth
		First Search (BFS), Dijkstra's Algorithm and A* are guaranteed to find
		shortest-path routes between pairs of points in arbitrary graphs. Such
		algorithms are generally a poor choice in highly regular network topologies
		such as meshes and toruses due to their high computational cost. In PGS
		repair, graph searching is only used for \emph{part} of the routing
		problem: to repair gaps in routes generated by more efficient routing
		algorithms.
		
		Real world super computer architectures are designed to ensure that faults
		are isolated \cite{gara05,alverson12} and thus tend to only impact a
		localised region of the network. Since PGS repair is only needed to route
		around these isolated faults, the space searched by the graph search
		algorithm should be very small in practice resulting in only short
		runtimes. In addition since faults are rare in real-world systems, the
		graph search process will only rarely be invoked.
		
		The PGS repair post-processing technique starts with a route produced by a
		non-fault-tolerant routing algorithm such as ESPR or NER. If this route is
		not obstructed by a fault, the algorithm terminates immediately without
		modifying the route. If the route attempts to use a faulty link, the
		algorithm proceeds as follows.
		
		The routing tree produced by the underlying routing algorithm is broken
		into subtrees wherever it attempts to route through a broken link and
		each subtree is assigned a unique colour, as illustrated in figure
		\ref{fig:pgs-repair-colouring}. From each disconnected subtree's root
		node in turn, a graph search is performed to find a short, fault-free
		route to a subtree node of a different colour. The subtree is then
		attached to the tree discovered by the graph search and re-coloured to
		match the tree it is connected to.
		
		\begin{figure}
			\center
			\begin{subfigure}{0.32\linewidth}
				\hspace*{-1.5em}
				\buildfig{figures/pgs-repair-colouring.tex}
				
				\caption{}
				\label{fig:pgs-repair-colouring}
			\end{subfigure}
			\begin{subfigure}{0.32\linewidth}
				\hspace*{-1.5em}
				\buildfig{figures/pgs-repair-colouring-fix1.tex}
				
				\caption{}
				\label{fig:pgs-repair-colouring-fix1}
			\end{subfigure}
			\begin{subfigure}{0.32\linewidth}
				\hspace*{-1.5em}
				\buildfig{figures/pgs-repair-colouring-fix2.tex}
				
				\caption{}
				\label{fig:pgs-repair-colouring-fix2}
			\end{subfigure}
			
			\caption{PGS repair process example showing a disconnected multicast
			route from A to B, C, D, E and F. $\times$ indicates a broken link.}
			\label{fig:pgs-repair-colouring-steps}
		\end{figure}
		
		For example in figure \ref{fig:pgs-repair-colouring-fix1} a path from the
		root of the subtree containing nodes E and F is found which connects it to
		the subtree rooted at A. Similarly in figure
		\ref{fig:pgs-repair-colouring-fix2} a path is also found connecting the
		subtree containing nodes C and D back to the subtree rooted at node A.
		
		If the routing tree was broken into $N+1$ subtrees by faults there will be
		$N$ subtrees disconnected from the root node. Each of the $N$ iterations of
		the algorithm connect a disconnected subtree to another subtree reducing
		the number of subtrees by $1$ each time. After $N$ iterations, therefore,
		exactly $1$ subtree remains which connects every node in the original
		routing tree without traversing faulty links.
		
		TODO: EXPLAIN THE FIDDLINESS HERE TO ENSURE WE DON'T CREATE LOOPS.
		
	\section{Evaluation \& Results}
		
		The PGS repair technique, by design, is able to work around all possible
		fault patterns which don't completely disconnect parts of the network. This
		result this evaluation focuses on the impact on performance PGS repair
		imposes. The metrics of interest in this evaluation are:
		
		\begin{itemize}
			\item Algorithm runtime
			\item Network congestion
			\item Routing table utilisation
		\end{itemize}
		
		\subsection{Traffic Patterns}
			
			In this evaluation, two standard benchmark multicast traffic patterns are
			used which have been used in previous research into SpiNNaker's network:
			
			\begin{figure}
				\center
				\buildfig{figures/traffic-distribution-centroids.tex}
				
				\caption{An example 4-centroid distribution with four centroids. The
				$\times$ marks the location of the origin node. Lighter colours
				indicate greater likelihood of a connection.}
				\label{fig:traffic-distribution-centroids}
			\end{figure}
			
			\begin{description}
				
				\item[Uniform] Destinations are chosen with uniform probability
				anywhere in the machine.
				
				\item[$N$-Centroids] Destinations are clustered around one of $N$
				randomly chosen `centroids' as illustrated in figure
				\ref{fig:traffic-distribution-centroids}.
				
			\end{description}
			
			The uniform traffic pattern is widely used in networks research
			\cite{dally04,davies12} while the centroids model was developed
			specifically to reproduce the traffic patterns found in the neural
			applications SpiNNaker is designed for \cite{navaridas14}. In this work
			we consider 3 centroids.
		
		\subsection{Fault model}
			
			In addition two different fault models are used which are representative of
			the faults found in real SpiNNaker systems:
			
			\begin{figure}
				\center
				\begin{subfigure}{0.48\linewidth}
					\hspace*{-1.5cm}
					\buildfig{figures/fault-example-uniform.tex}
					
					\caption{Uniform}
					\label{fig:fault-example-uniform}
				\end{subfigure}
				\begin{subfigure}{0.48\linewidth}
					\hspace*{-1.5cm}
					\buildfig{figures/fault-example-hss.tex}
					
					\caption{HSS Link}
					\label{fig:fault-example-hss}
				\end{subfigure}
				
				\caption{The two link fault models considered.}
				\label{fig:fault-example}
			\end{figure}
			
			\begin{description}
				
				\item[Uniform] Links are selected and disabled at random (figure
				\ref{fig:fault-example-uniform}).
				
				\item[HSS Link] Groups of links corresponding with randomly selected
				single High-Speed Serial (HSS) link between SpiNNaker boards are disabled
				together (figure \ref{fig:fault-example-uniform}).
				
			\end{description}
			
			The uniform link failure model models isolated failures resulting from
			isolated manufacturing defects in individual links. The HSS Link failure
			model models faults arising from failing or disconnected board-to-board
			links which carry several chip-to-chip traffic flows via a single cable in
			SpiNNaker systems. Though SpiNNaker-specific, the later fault model is
			analogous to failure modes arising in other architectures where a single
			fault may render several links impassable in a single area.
			
			A range of failure rates are explored in this section. My measurements of
			current large-scale SpiNNaker installations the link failure rate is about
			\SI{0.03}{\percent} with failures due to both individual chip-to-chip links
			and board-to-board HSS links. Exact link failure statistics for commercial
			super computer installations are not widely available, however, published
			Mean-Time-Between-Failure (MTBF) statistics place an upper bound on link
			failure rates at a similar \SI{0.03}{\percent} in one-year-old BlueGene/Q
			systems \cite{chiu11}.
			
			Unfortunately presently undiagnosed problem with the SDRAM packaged with
			approximately \SI{1}{\percent} of SpiNNaker chips has rendered these chips
			unusable for most applications. The gaps in the network resulting from the
			loss of these chips currently dominate true link failures leaving just over
			\SI{1}{\percent} of links inoperable.
			
			Surprisingly, research into fault tolerant routing in super computers
			appears to focus on benchmarks with even higher fault rates ranging from
			\SI{3}{\percent} to as high as \SI{7}{\percent}
			\cite{ho04,gomez04,mejia06}.
			
			In this evaluation, fault rates ranging from \SI{0.01}{\percent} to
			\SI{5}{\percent} are considered to cover both realistic fault levels
			along with the more extreme cases considered in related work.
		
		\subsection{Base routing algorithm}
			
			Since the PGS repair process is routing algorithm agnostic all
			experiments use the NER algorithm which has been found to be appropriate
			for SpiNNaker applications \cite{navaridas14}.
		
		\subsection{Algorithm runtime}
			
			To assess the impact of the PGS repair process on routing algorithm
			runtime, the algorithm was used to process a large number of randomly
			generated routing problems and the runtime recorded.
			
			\num{10000} one-to-sixteen multicast routing problems were generated in a
			$256\times256$ hexagonal torus topology, the largest size possible for a
			SpiNNaker system. Other quantities of multicast destinations were also
			evaluated but are omitted for brevity since the pattern of results are
			similar to those outlined here.
			
			TODO: APPENDIX WITH OTHER RUNS?
			
			The NER and PGS repair algorithms were written in C and compiled with GCC
			4.8.3 with \verb|-O2| level optimisations and executed on a cluster of
			idle workstations with 3.10 GHz Intel Core-i5-2400 CPUs.
			
			\begin{figure}
				\center
				\buildrplot{figures/routing-runtimes.R}
				
				\caption{Mean runtime of routing and PGS repair overhead. PGS repair
				overhead is stacked above the routing runtime (i.e. bars do not
				overlap). Error bars indicate 95\% confidence interval. Note different
				Y-scale for HSS link and uniform fault models.}
				\label{fig:routing-runtimes}
			\end{figure}
			
			Figure \ref{fig:routing-runtimes} shows the average runtimes recorded for
			both the NER and PGS repair algorithms. In fault-free networks the
			PGS-repair post-processing step is not required and incurs no penalty
			while the runtime of the algorithm grows with the fault rate for both
			fault and traffic models.
			
			Notably the HSS fault model results in longer runtimes for the PGS repair
			process compared with an equivalent fault-density of uniform faults.
			Because the HSS fault model produces contiguous lines of faults the PGS
			repair algorithm must construct a longer path to avoid the fault.  Since
			the space explored by a graph algorithm typically grows with $O(H^2)$
			with respect to the hops in the discovered route, $H$, this increase in
			search distance has a large impact on the runtime of the PGS repair
			process.
			
			The runtime of the PGS repair algorithm remains roughly in proportion to
			the runtime of the underlying routing algorithm with respect to different
			traffic models. The centroid traffic pattern tends to result in routes
			with fewer hops than a uniform traffic pattern with the same number of
			destination nodes as segments of routes are often shared between
			destination nodes. Since the NER algorithm's runtime is strongly related
			to the number of hops in the output route the runtime of the algorithm is
			greater for uniform traffic. Likewise the probability of PGS repair being
			required increases with the number of hops in route and hence the runtime
			of the PGS repair algorithm increases roughly in proportion.
		
		\subsection{Routing table usage}
			
			In order to gain a realistic measure of routing table usage it is
			necessary to determine the effect of many routes being generated for a
			single set of faults. To enable a sufficiently large number of sample to
			be collected the experimental setup considered previously is reduced to a
			network containing $48\times48$ nodes.
			
			\num{1000} $48\times48$ node network models are produced according to the
			HSS link and uniform fault models. For each of these models
			$48\times48\times16=$~\num{36864} one-to-sixteen routes are generated using
			the centroid and uniform traffic models. This corresponds to one
			multicast route per application core. As is convention in SpiNNaker,
			routing table entries are inserted for each route at the source of the
			route, at each destination and at each corner or fork. The number of
			routing table entries at each node in the model is counted and the
			maximum number of entries in a single node is reported for each network
			model.  The \emph{maximum} number of routing entries of any router was
			chosen since the number of entries available per SpiNNaker router is
			bounded by hardware.
			
			\begin{figure}
				\center
				\buildrplot{figures/routing-entries.R}
				
				\caption{Violin plot showing the distribution of maximum table sizes
				for \num{1000} random networks. The red line at \num{1024} entries
				indicates the size of SpiNNaker's routing tables.}
				\label{fig:routing-entries}
			\end{figure}
			
			
			Figure \ref{fig:routing-entries} shows the distributions of the largest
			routing table sizes for each fault and traffic model.
			
			\begin{figure}
				\center
				\begin{subfigure}{0.48\linewidth}
					\center
					\buildfig{figures/hss-link-routing-table-usage.tex}
					
					\caption{Routing table entries}
					\label{fig:hss-link-routing-table-usage}
				\end{subfigure}
				\begin{subfigure}{0.48\linewidth}
					\center
					\buildfig{figures/hss-link-resource-usage.tex}
					
					\caption{Routes passing through chip}
					\label{fig:hss-link-resource-usage}
				\end{subfigure}
				
				\caption{The impact of a HSS link fault on routing table usage and
				congestion. Each hexagon represents a single chip, the red line
				indicates the chip-to-chip connections broken by the HSS link fault.}
				\label{fig:hss-link-usage}
			\end{figure}
			
			The HSS link failure model has a much greater impact on peak routing
			table resource usage than uniform link failures for a given fault rate.
			This is because HSS link faults result in a large concentration of routes
			being disrupted and then re-routed around the same obstacle in a single
			location. Figure \ref{fig:hss-link-routing-table-usage} shows how routing
			table usage varies around a HSS link fault in one instance of the
			experiment. There are clear peaks in routing table usage around the ends
			of the line of faults which result from routes produced by PGS repair
			finding shortest paths around the edge of the faults.
		
		\subsection{Network congestion}
			
			To measure the impact of PGS repair on network congestion, two
			experiments were performed, one using the same model used to measure
			routing table usage and one based on tests run on SpiNNaker hardware.
			
			For each of the network fault and traffic pattern described previously,
			the paths taken for the \num{36864} one-to-sixteen multicast routes
			generated are used to compute the number of times each link in the
			network is used. The number of routes passing through the most-used link
			is then recorded, giving an indication of the level of congestion in the
			network.
			
			\begin{figure}
				\center
				\buildrplot{figures/routing-resource.R}
				
				\caption{Violin plot showing the distribution of maximum
				routes-per-chip for \num{1000} random networks.}
				\label{fig:routing-resource}
			\end{figure}
			
			The results are presented in figure \ref{fig:routing-resource} and follow
			the same trends as the results previously shown for routing table usage.
			Again, HSS link faults result in routes with the greatest congestion due
			to the concentration of routes finding shortest paths around an obstacle
			(see \ref{fig:hss-link-resource-usage}).
			
			To verify that the results above, an additional experiment has been
			carried out which attempts to mimic the model used previously in actual
			SpiNNaker hardware. In these experiments a large SpiNNaker machine is
			divided into independent 48-board (2304-chip) sections. Because the
			48-board systems used in these experiments are cut out of a larger
			machine, they lack wrap-around links and thus form hexagonal mesh
			topologies, rather than hexagonal toruses.
			
			Due to the SDRAM issue described above, fault rates below
			\SI{1}{\percent} cannot be modelled.  To simulate higher fault rates,
			additional links are disabled in software according to the fault models
			described used previously. Since some faults are due to genuine hardware
			faults, these faults cannot be placed randomly in each experiment. To
			reduce, bias each combination of fault rate, fault model and traffic
			pattern is repeated XXX times across randomly chosen physical machines.
			
			XXX 1-to-XXX routes are generated in both uniform and XXX-centroid
			distributions as used throughout this evaluation. Synthetic network
			traffic is generated at the source of each route following a Bernoulli
			distribution. Traffic consumers running on all destination nodes accept
			packets as quickly as possible from the network and log their arrival.
			The Bernoulli probability is set the same for every route's traffic
			generator and increased in steps of XXX and the number of packets dropped
			in an XXX second period logged. The network is considered saturated once
			less than \SI{99}{\percent} of packets successfully arrive at their
			destination.
			
			Figure \ref{XXX} shows the distributions of the saturation points for
			each experimental configuration.
			
			TODO: ANALYSIS
		
	\section{Conclusions}
		
		In this chapter I described how SpiNNaker's unconventional network and
		router architecture render existing fault tolerant routing algorithms
		unsuitable. I introduced PGS repair, a post-processing technique for
		existing non-fault tolerant routing algorithms designed for SpiNNaker such
		as NER.
		
		Unlike some other fault tolerant routing algorithms for other
		architectures, PGS repair is able to work-around arbitrary fault patterns
		by exploiting SpiNNaker's inbuilt deadlock avoidance mechanisms. In the
		presence of realistic failure rates of up to \SI{1}{\percent}, only small
		overheads of up to XXX, XXX and XXX for in algorithm runtime, routing table
		usage and network performance are incurred respectively. This low
		performance overhead makes PGS repair appropriate for use in real
		applications. At the time of writing the algorithm has been successfully
		used in a number of neural and non-neural SpiNNaker applications.
		
		At more extreme fault rates not expected in real-world systems, the
		algorithm still functions correctly but the results incur much greater
		routing table and congestion overheads, particularly when faults are
		concentrated. Future extensions to this algorithm might aim to reduce this
		overhead by producing longer and more varied routes around faults to even
		out the load.

	\chapter{Placing applications in large SpiNNaker machines}
	
	In the previous chapter I tackled the problem of scale in generating routes
	for very large networks such as SpiNNaker. In this work the centroid traffic
	pattern was used as an approximation of the expected network traffic
	generated by `well behaved' neural network simulation software running on
	SpiNNaker. The traffic produced largely exhibits strong locality, that is
	most communication occurs between either nearby nodes or clusters of nodes.
	In reality, neural simulation applications are not specified geometrically
	but rather as abstract graphs of communicating neurons
	\cite{davison08,eliasmith13}. Applications must then \emph{place} these
	neurons onto nodes in a SpiNNaker system, attempting maximise communication
	locality.
	
	In this chapter I re-evaluate the suitability of simulated annealing as a
	technique for finding high quality placements for large parallel
	applications. Though this technique had fallen out of fashion in the field of
	application placement by the early 1990s, it has found wide use for placing
	components in computer chip and FPGA designs. In the intervening years,
	placement problems in super computers have grown in size from tens or
	hundreds of nodes to millions, a scale at which chip placement techniques
	were operating in the mid 1990s. I adapt the simulated annealing algorithm
	used by the VPR academic circuit placement software to produce placements for
	applications running on SpiNNaker. In that in a range of real and synthetic
	benchmarks simulated annealing produces high quality placements enabling
	efficient use of SpiNNaker's network resources.
	
	
	%In the field of chip design, Moore's `Law' \cite{moore65,moore75} observes a
	%similar exponential growth in the number of components within a single chip.
	%Today modern processors contain billions of components and an analagous
	%placement problem exists in attempting to place interconnected components
	%near to eachother. In this chapter I explore the techniques used for circuit
	%placement and adapt one such technique, Simulated Annealing (SA)
	%\cite{kirkpatrick83}, for use in application placement. Despite some early
	%interest in SA for application placement in the 1980s and early 1990s, the
	%technique has since fallen out of favour. I find that at the scales of modern
	%placement problems SA-based placement is able to produce solutions of
	%superiour quality to contemporary methods.
	%
	%TODO: SUMMARISE RESULTS...
	
	\section{Related work}
		
		The placement problem has been tackled independently in the literature by
		researchers in both the application and chip placement communities. In this
		survey I cover application and chip placement separately as these two
		communities have remained largely isolated from one another. First I
		explore the techniques applied to application placement before moving on to
		contrast this with the techniques used in circuit placement.
		
		In the application placement literature, the placement problem is often
		referred under the umbrella term `mapping'. Unfortunately term is often
		used more broadly to include other tasks such as routing and application
		partitioning. To avoid ambiguity I use the term `placement', as preferred
		by the chip and FPGA design communities, to refer specifically to the
		problem of assigning nodes in an application's communication graph to nodes
		in a machine's connectivity graph.
		
		\subsection{Application placement algorithms}
			
			TODO: GENERAL INTRO
			
			\subsubsection{Application-specific approaches (manual placement)}
				
				In the case of some applications such as finite element modelling
				\cite{bermejo13}, the structure of the problem itself leads to a
				natural placement of the computation on nodes in a machine. For example
				when simulating a 3D volume in an node super computer with a $3 \times
				4 \times 2$ 3D torus or mesh topology network, the modelled volume
				might be divided into as in figure \ref{fig:fem-partitioning}. Each
				cuboid in the model is then assigned to the corresponding node in the
				network topology.
				
				\begin{figure}
					\center
					\buildfig{figures/fem-partitioning.tex}
					
					\caption{Example partitioning of a 3D space to fit into a super
					computer with a $3\times4\times2$ torus or mesh topology.}
					\label{fig:fem-partitioning}
				\end{figure}
				
				When the number of dimensions in a problem do not match that of the
				underlying network architecture, the common solution is to either
				divide only along a subset of the axes or to divide into additional
				pieces on the existing axes \cite{gilge14}.
			
			\subsubsection{Sequential placement}
				
				In the case where a placement solution is non-obvious one of the
				simplest and most popular strategies is to apply a simple sequential
				placement algorithm. Sequential placement algorithms function by
				iterating over the vertices in the application's communication graph
				and assigning them to a free node in the target machine. Sequential
				placement algorithms are differentiated by the order in which they
				iterate over vertices in the communication graph and fill nodes in the
				target machine. A number of widely used orderings are described below.
				
				\begin{figure}
					\center
					\begin{subfigure}{0.32\linewidth}
						\center
						\buildfig{figures/sequential-row-order.tex}
						\caption{Row-order}
						\label{fig:sequential-row-order}
					\end{subfigure}
					\begin{subfigure}{0.32\linewidth}
						\center
						\buildfig{figures/sequential-alternating.tex}
						\caption{Alternating}
						\label{fig:sequential-alternating}
					\end{subfigure}
					\begin{subfigure}{0.32\linewidth}
						\center
						\buildfig{figures/sequential-hilbert.tex}
						\caption{Hilbert curve}
						\label{fig:sequential-hilbert}
					\end{subfigure}
					
					\caption{Space-filling curves in 2D mesh and torus topologies.}
					\label{fig:sequential}
				\end{figure}
				
				Super computer management software such as SLURM \cite{yoo03} and Blue
				Gene's system software \cite{gilge14} by default na\"ively iterate over
				vertices in an application communication graph in the order they are
				provided. The nodes in the target machine are then iterated over in a
				simple space-filling curve through the network topology. Figure
				\ref{fig:hilbert-placement} illustrates the default patterns available
				in these software packages. The row-order (figure
				\ref{fig:sequential-row-order}) and alternating (figure
				\ref{fig:sequential-alternating}) curves correspond with 2D versions of
				the default node assignment orders used in SLURM and BlueGene systems.
				
				\begin{figure}
					\center
					\buildfig{figures/hilbert-placement.tex}
					
					\caption{A Hilbert curve, coloured from blue to red.}
					\label{fig:hilbert-placement}
				\end{figure}
				
				The Cray extensions to SLURM software provide a Hilbert curve
				\cite{hilbert91} (figure \ref{fig:sequential-hilbert}) node assignment
				order. Unlike the row-order and alternating space filling curves the
				Hilbert curve ensures that pairs of vertices close together in the node
				iteration order are also close together in the target machine's network
				\cite{moon01, zumbusch99}. Figure \ref{fig:hilbert-placement} shows a
				5$^\textrm{th}$-order Hilbert curve where each point in the curve is
				coloured according to its position along the curve. In this figure it
				is possible to see that nearby positions in the curve (which share
				similar colours) are also close in 2D space.
				
				When the proximity of vertices in the vertex-ordering supplied by an
				application is a good estimator of those vertices communication
				requirements, the sequential assignment schemes discussed above can be
				very effective. These techniques have also proven adequate in
				small-scale and densely connected applications such as early neural
				simulations running on prototype SpiNNaker machines with tens of nodes
				\cite{galluppi10} but growing beyond this scale has proven problematic.
				
				\begin{figure}
					\center
					\begin{subfigure}{0.45\linewidth}
						\center
						\buildfig{figures/rcm-initial.tex}
						
						\caption{Original permutation}
						\label{fig:rcm-initial}
					\end{subfigure}
					\begin{subfigure}{0.45\linewidth}
						\center
						\buildfig{figures/rcm-sorted.tex}
						
						\caption{RCM permutation}
						\label{fig:rcm-sorted}
					\end{subfigure}
					
					\caption{Adjacency matrix representation of a graph before and after
					permutation by the RCM algorithm.}
					\label{fig:rcm}
				\end{figure}
				
				A number of algorithms have been proposed for automatically selecting
				good vertex iteration orders, typically using a graph-traversal based
				heuristic. A typical method, described by Hoefler \emph{et al.}
				\cite{hoefler11} exploits the Reverse-Cuthill-McKee (RCM) algorithm
				\cite{cuthill69}. An application's communication matrix is represented
				as an adjacency matrix, $M$, where $M_{i,j}$ is 1 if node $i$ is
				connected by an edge to node $j$ and 0 otherwise. An example matrix is
				illustrated in figure \ref{fig:rcm-initial}. The RCM algorithm uses a
				simple heuristic to permute the matrix (i.e. renumber the nodes in the
				graph) in order to reduce the bandwidth of the matrix. Figure
				\ref{fig:rcm-sorted} shows the RCM-permuted version of the example
				adjacency matrix. When a graph's vertices are ordered as in a
				bandwidth-reduced sparse matrix, vertices close together in the
				ordering are likely to communicate while those further apart tend not
				to communicate.
				
			\subsubsection{Optimisation-based Placement}
				
				% Citations from short report about optimisation in placement...
				% \cite{chen06,jeannot14} and \cite{jeannot10} ("subsets of apps")
				
				In the academic community, a number of attempts have been made to use
				more sophisticated optimisation algorithms for the placement of
				applications. In 1985, Steele \cite{steele85} proposed the use of
				simulated annealing for placing applications in the 6D torus topology
				of the 64 node `Caltech Cosmic Cube' machine. Simulated annealing,
				originally developed by Kirkpatrick \emph{et al.} \cite{kirkpatrick83},
				is a general-purpose optimisation algorithm which works by analogy to
				the physical process of annealing. In brief simulated annealing
				functions by randomly swapping vertices in a candidate placement
				solution, accepting swaps which move connected vertices closer together
				and rejecting some proportion of swaps which move connected vertices
				further apart. The simulated annealing algorithm is described in detail
				later in this chapter.
				
				Towards the end of the 1980s, application placement appeared to be
				becoming less important as super computer network architectures
				improved:
				%
				\begin{displayquote}
					``Careful placement was necessary because of the slow communication
					and non-uniform addressing of early concurrent computers. However,
					the development of message passing machines with fast communications
					and a uniform global address space  has made placement less of an
					issue. In such machines a random placement performs nearly as well as
					an optimum placement.''
					
					\noindent --- W. Dally, 1987 \cite{dally87}
				\end{displayquote}
				%
				In addition, network and problem sizes remained small, so small in fact
				that linear-programming based optimal placement still appeared in
				benchmarks comparing placement algorithms \cite{xu91}. In this
				environment, simpler sequential placement algorithms gained favour over
				more computationally expensive algorithms such as simulated annealing.
				
				As problem and machine sizes have grown and network utilisation has
				once again become an important factor in application performance
				\cite{navaridas09b} more complex optimisation algorithms have
				reappeared in the literature. One popular approach employs graph
				partitioning algorithms such as METIS \cite{karypis98} to perform
				recursive bipartitioning based placement
				\cite{phillips14,hoefler11,pellegrini96}.  This placement process is
				illustrated in figure \ref{fig:partitioning}.
				
				In the first step, the application communication graph and machine
				connectivity graph are bipartitioned such that the number of edges
				between partitions is minimised. Each half of the communication graph
				is associated with one of the halves of the machine connectivity graph.
				The partitioning process is then repeated recursively on each of the
				two communication and connectivity graph pairs. The process halts when
				the graphs can no longer be partitioned at which point the vertices in
				the communication graph are placed on their associated node.
				
				\begin{figure}
					\center
					\buildfig{figures/partitioning.tex}
					
					\caption{Illustration of application placement by recursive
					partitioning.}
					\label{fig:partitioning}
				\end{figure}
				
				TODO: PARTITIONING IS GREAT AND ALL BUT QUALITY ISN'T ALWAYS GREAT AND
				IT DOESN'T DEAL WELL WITH MULTI-CONSTRAINT SCENARIOS E.G. PROCESSOR AND
				MEMORY RESTRICTIONS.
				
				Unfortunately, many of these simply aren't suited to the scale of
				neural applications running on SpiNNaker (e.g. only cope with tens of
				nodes while SpiNNaker may contain hundreds of thousands).
				
				Additionally, a number of algorithms have been developed which make
				assumptions about the topologies of the problem or network. Tree match
				for example attempts to map tree-shaped problems to tree-shaped
				networks. Such algorithms can be highly effective but again do not
				apply to SpiNNaker or its neural applications.
		
		\subsection{Chip placement algorithms}
			
			The chip-design industry has, for many years, dealt with problems
			analogous to the task of placing super computer jobs in a way suited to
			SpiNNaker. Modern CPUs have millions or billions of components with
			strictly fixed connectivity. CPU designers must place each of these onto
			a chip such that the connection lengths are controlled to reduce
			congestion and increase performance. As such, these algorithms are
			ideally suited to future super computer placement work since they already
			operate at the scales required \cite{nam07}.
			
			\subsubsection{Cost functions}
				
				HPWL is popular but a bit crap for high fan-outs. It is, however, quite
				simple.
				
				TODO: SELECT A BETTER COST FUNCTION...
			
			\subsubsection{Simulated annealing}
				
				One of the oldest techniques used for circuit placement is simulated
				annealing and this remains popular today thanks to its sheer
				versatility (see VPR, other open FPGA tools).
				
				SA works by analogy with the physical process of annealing.
				The simulated annealing algorithm works by selecting random pairs of
				components on a chip, swapping them and evaluating some cost function.
				If the swap reduces the cost function, it is kept, if not, depending on
				a function of the current temperature and the cost introduced by the
				swap.
				
				TODO: ILLUSTRATION OF SIMULATED ANNEALING SWAP OPERATION
				
				By occasionally allowing costly swaps, the annealing algorithm avoids
				becoming trapped in local minima. As the algorithm proceeds, the
				temperature is slowly reduced and with it the proportion of costly
				swaps which are retained. This causes the placement to move from
				exploration early on towards refinement later on.
				
				The temperature schedule of an annealing algorithm is critical to its
				success. In general these schedules are computed based on the
				performance of the algorithm as it runs. In VPR the following schedule
				is used.
				
				TODO: DESCRIBE VPR'S SCHEDULE
				
				TODO: FIND AND DESCRIBE ALTERNATIVE SCHEDULE?
				
				Unfortunately, SA is very difficult to parallelise, especially in the
				case of placement. As a result, its scalability has been limited and
				resulted in significantly reduced usage in recent work.
			
			\subsubsection{Partitioning placement}
				
				Partitioning based placement solves the placement problem using
				graph-partitioning recursively on the problem graph to assign each part
				of the circuit to some area in the super chip. Though a number of
				algorithms have proven successful in academic placement contests over
				the years, they are not popular in industrial settings.
			
			\subsubsection{Analytical placement}
				
				In analytical placement, cost function for the circuit graph is
				approximated in a form which is amenable to solutions with standard
				numerical or symbolic algebraic techniques. Using these techniques,
				exact minimum cost (in terms of the approximation) configurations can
				be obtained.
				
				Quadratic placement is a popular analytical placement technique which
				approximates the cost of a placement as the sum of the squares of the
				distances between connected circuit elements.
				
				TODO: FIGURE EXAMPLE QUADRATIC PLACEMENT PROBLEM AND SOLUTION
				
				As such this gives a quadratic cost function like so which we must
				minimise.
				
				TODO: QUADRATIC COST EQN
				
				To minimise the function we differentiate and solve using simple
				symbolic manipulation.
				
				TODO: QUADRATIC COST SOLUTION
				
				Unfortunately, quadratic placement doesn't contain any congestion
				relief by default so various schemes exist. For example, extra anchor
				nodes are inserted which gently pull the circuit components apart from
				each other. As a result, the algorithm generally proceeds by iterating,
				regenerating anchors each time.
				
				Other non-quadratic analytical methods exist too with numerical
				solutions. The approaches are often similar.
			
			\subsubsection{Hierarchical clustering}
				
				Many placement algorithms scale super-linearly with problem size and so
				larger problems become increasingly problematic to handle. To solve
				this problem clustering techniques are first applied to first simplify
				the placement problem. A solution is then found at the coarse level and
				then hierarchically fleshed out.
				
				Various clustering algorithms are in use.
				
				TODO: TALK ABOUT CLUSTERING IN PLACEMENT...
				
				TODO: DESCRIBE THE ALGORITHM I IMPLEMENTED.
	
	\section{Application placement by simulated annealing}
		
		\label{sec:placement-by-annealing}	
		
		I have implemented a simplified SA based application placement algorithm
		based on the approach used in the popular VPR place and route tool chain.
		The algorithm is written in C and is optimised for experimentation rather
		than performance but is production-ready. It has been integrated into the
		`Rig' SpiNNaker software tools and has been used to place very large
		simulations. More on that later.
		
		\subsection{Representation}
			
			Model each chip as having a quantity of various resources (e.g. Cores,
			SDRAM) available. The application graph consists of vertices which each
			consume some quantity of these resources. Vertices must be placed on a
			single chip such that the resources required on a given chip do not
			exceed those available. Vertices are then interconnected by 1:N nets with
			weights which act as hints. The nets are treated as a soft constraint:
			vertices connected via a net will, ideally, be placed near to each other,
			with priority being given to nets with higher weights. Additionally there
			will be a list of placement constraints (see later).
			
			A key observation is that while vertices in an application may frequently
			have a 1:1 correspondence with application cores, this need-not be the
			case. For example, a vertex may represent a block of SDRAM which is
			shared. A vertex may also represent some other resource, for example,
			external IO availability. By making these resource types user-defined,
			applications programmers can express flexible hard-constraints on their
			application.
			
			Another observation is that generic soft constraints can be expressed may
			be expressed using a net with an appropriate weight.
			
			As a result of these facilities, application programmers can easily
			express their own application-specific hard and soft placement
			constraints without having to modify the algorithm. This representation
			has become a de-facto standard for placement problem interchange for
			SpiNNaker applications.
		
		\subsection{Cost function}
			
			At present I've used HPWL despite this being really bad for high-fan-out
			multicast and totally ignorant to the hexagonal nature of SpiNNaker...
			
			To compute bounding boxes for tori I use the following approach. For each
			dimension, sort the points on that dimension and find the largest gap
			between them on a ring. The bounding box goes the other way.
			
			TODO: FIGURE ILLUSTRATING BOUNDING BOX COMPUTATION FOR TORI.
		
		\subsection{Annealing schedule}
			
			The annealing schedule is that used by VPR. Despite being for circuit
			placement, it seems to work jolly well.
			
			TODO: DESCRIBE AND RATIONALISE THE SCHEDULE
		
		\subsection{Constraint handling}
			
			Various hard and soft constraints may be expressed by software
			approaches. For each we explain how they may be handled by the placement
			algorithm:
			
			\subsubsection{Location Constraint}
				
				The vertex is placed on a chip and removed from the set of movement
				candidates.
			
			\subsubsection{Same-chip constraint}
				
				When two vertices must always be placed on the same chip they are
				simply combined into one vertex which consumes the sum of their
				resources. Placement then treats them as one chip and thus is forced to
				atomically place the vertices.
			
			\subsubsection{Reserve resource constraint}
				
				Simply reduce resource availability on that chip.
			
			\subsubsection{Keep near Ethernet}
				
				Simply add a net.
	
	\section{Evaluation}
		
		\label{sec:placement-results}
		
		Though benchmarks exist for super computer loads and chip placement tasks,
		such things don't exist for neural applications. As a result I use a
		selection of real applications for SpiNNaker along with some synthetic
		benchmarks based on biological data.
		
		\subsection{Benchmark networks}
			
			First some real networks.
			
			Some nengo networks: SPAUN: `The world's largest functional brain model'.
			Word-net network from Jamie: Example of some learning.
			
			TODO: DESCRIBE SHAPE OF NENGO NETWORKS
			
			Some PyNN networks: Microcortical column model from PyNN. Note almost
			broadcast connectivity but varying weights. Try and extract a vision
			netlist from Anna. Maybe try and get a netlist for Tom's barrel cortex.
			
			Now for some artificial networks. Pipeline, noisy pipeline, mesh,
			Gaussian 2D.
		
		\subsection{Experiments}
			
			We compare random, linear, greedy and annealing based placement
			approaches to placement. We compare static metrics (such as mean/max
			congestion, table usage) along with experiments based on simulated
			network traffic in real hardware. Network Tester generates artificial
			traffic in proportion with the weights given for each model. We compare
			the relative level of traffic sustainable. We also consider use of
			machines of various sizes.
		
		\subsection{Results}
			
			SA placement is slow but rather effective, especially for some networks.
			Generally worth doing. Will need to be sped up for very large machines...
			
			TODO: EXPERIMENTS!
	

	\chapter{Discussion}

\section{Suitability of the hexagonal torus topology}
	\subsection{Physical scalability}
	\subsection{Routability}
	\subsection{Placeability}

\section{Suitability of the SpiNNaker router}
	\subsection{Deadlock avoidance}
	\subsection{Routing table size}

\section{Suitability of circuit placers for application placement}
	\subsection{Quality}
	\subsection{Runtime}
	\subsection{Routing resources}
	\subsection{Flexibility}
	\subsection{Scalability}


	\chapter{Future research}
	
	In this thesis I have presented a number of new techniques which have made it
	possible to assemble and operate the SpiNNaker super computer. This work
	opens up a range of possibie lines of research to extend this work to future
	architectures and applications. In this chapter I focus on two anticipated
	challenges of future systems: growing scale and greater dynamicism in
	applications.
	
	\section{Scaling up}
		
		TODO: INTRO
		
		\subsection{Grid machine room layouts}
			
			In chapter XXX, I developed a machine room layout for hexagonal torus
			topologies which allowed machines occupying a row of standard
			machine-room cabinets to scale up without the need for long
			interconnecting cables. For larger installations, however, it will be
			necessary to employ multiple rows of cabinets in a 2D arrangement.
		
		\subsection{Routing congestion control}
		
		\subsection{Parallel place and route}
	
	\section{Structural plasticity and dynamic fault-tolerance}
		\subsection{Plasticity models}
		\subsection{Incremental placement}
		\subsection{Incremental routing}
		\subsection{Hot-spare routes}

	\chapter{Conclusions and future research}
	
	The SpiNNaker architecture was designed to tackle the challenges presented by
	the simulation of biologically realistic neural networks. One of its
	distinguishing features is its network architecture which employs both an
	unconventional network topology and multicast router architecture. The
	hexagonal torus topology used by SpiNNaker was chosen to enable greater
	performance while maintaining ease of construction and scalability compared
	with conventional network topologies. SpiNNaker's router design centres
	around packets mimicking the neural `spike' signals they are designed to
	convey by being small, multicast and not guaranteed to arrive at their
	destination.  This novel design, though largely complete before this work
	began, left a number of open problems which this thesis has attempted to
	address.
	
	In this concluding chapter I begin by summarising the answers to the research
	questions raised in chapter~\ref{sec:introduction}. This is followed by a
	discussion of new research topics which have been uncovered by this work.
	
	\section{Answers to research questions}
		
		Each of the three research questions are answered below.
		
		\subsubsection{1. Can the hexagonal torus topology be deployed and used in
		real, large-scale systems?}
		
		In chapter~\ref{sec:building}, I introduced a cabling scheme and assembly
		technique which has been used successfully to build a prototype SpiNNaker
		system with over half a million processor cores using the hexagonal torus
		topology. The techniques shown are expected to enable a final SpiNNaker
		machine of double this size to be built, filling a six metre long row of
		machine-room cabinets.
		
		Though SpiNNaker's processor-count places it amongst some of the world's
		largest supercomputers (see figure \ref{fig:top500-num-processors} on page
		\pageref{fig:top500-num-processors}), it is comparatively compact, filling
		one row of cabinets compared with the warehouse-scale installations found
		in commercial systems. In spite of this, the folding and interleaving
		techniques described allow hexagonal torus topologies to scale to
		arbitrarily large installations without cables which span the machine.
		
		Chapter~\ref{sec:shortestPaths} described an efficient and general
		technique for finding, and enumerating shortest path vectors in hexagonal
		torus topologies. These developments bring the hexagonal torus topology in
		line with other topologies by enabling routing algorithms to exploit all
		possible paths in a network. Further, chapter~\ref{sec:placement}
		demonstrated that placement algorithms are also adaptable to hexagonal
		torus topologies thanks to their similarity to 2D toruses.
		
		Though, as this thesis highlights, hexagonal toruses lack many of the
		intuitive properties enjoyed by other topologies, it is still possible to
		reason about them with only limited computational effort.  Now that the
		practicality and scalability of the topology has also been demonstrated in
		practice, it represents a credible option for future network architectures.
		
		\subsubsection{2. Does SpiNNaker's router architecture help, or hinder
		fault tolerance?}
		
		SpiNNaker's unconventional use of packet dropping to avoid deadlocks
		greatly simplifies the router architecture, part of the motivation for this
		design. In chapter~\ref{sec:routing} this feature is used to the advantage
		of PGS repair to add fault tolerance to existing routing algorithms.
		Compared with the often complex and wasteful methods used to tolerate
		faults in other networks, PGS repair incurs very little performance
		overhead in the presence of static faults.
		
		Routing table usage does increase in the presence of faults, however, which
		may be a concern for applications which already require many routing table
		entries. Routing table usage, as well as other overheads, were most
		significantly increased in the presence of contiguous groups of network
		faults. This is because the PGS repair algorithm produces routes which pass
		tightly around the corners of faults, resulting in concentrations of
		routing table entries in those areas.  Though the symptoms of this problem
		can be attributed to the design of SpiNNaker's multicast routing mechanism,
		the responsibility lies with the behaviour of the PGS repair algorithm.
		Potential improvements to the PGS repair algorithm are discussed later in
		\S\ref{sec:pgs-repair-improvements}.
		
		The overall answer to this research question, therefore, is that the
		flexibility provided to routing algorithms in SpiNNaker's architecture is
		of great benefit, enabling arbitrary fault patterns to be inexpensively
		avoided.
		
		\subsubsection{3. How can the parts of a neural simulation be placed onto a
		large hexagonal torus topology to reduce network load?}
		
		In chapter~\ref{sec:placement}, I explored a number of contemporary
		approaches to the problem of placing applications with irregular
		communication patterns into network topologies. I observed that researchers
		working on circuit placement for chips and FPGAs are tackling similar
		problems and working at scales as large, or larger than, those faced in
		application placement. Based on this I developed a
		simulated annealing based placement algorithm inspired by the techniques
		used in circuit placement, with specific adaptations for use in application
		placement and SpiNNaker's network topology.
		
		The simulated annealing based placement algorithm consistently outperforms
		pre-existing placement algorithms included in benchmarks in terms of
		placement quality.  In the case of one benchmark, simulated annealing based
		placement made it possible to run that neural simulation in real-time for
		the first time.  At larger scales, simulated annealing was also found to be
		able to produce good quality placements in benchmarks containing over one
		million processes -- the largest size supported by the SpiNNaker
		architecture.
		
		The major shortcoming of simulated annealing based placement is its
		execution speed. Though its execution time grows in proportion to the size
		of the problem, the implementation used took over 12~hours to place a
		synthetic problem for the largest planned SpiNNaker machine. Though
		tractable -- particularly given the relative output quality compared with
		the prior state-of-the-art -- the algorithm is unlikely to function
		comfortably as-is on larger problems.
		
		The conclusion to be drawn from this result, however, is not just that
		simulated annealing is a good solution for today's placement problems but
		that circuit placement techniques in general could be successfully adapted
		to fulfil this role. The placement problems faced by chip designers are
		growing at roughly the same exponential rate as the size of super computers
		but circuit designs hold the lead in terms of problem size. Consequently,
		as approaches are retired by chip placement researchers, they may find new
		life in the field of application placement.
		
	\section{Future research}
		
		Though the goals of this study have largely been met, there also remain
		some important limitations which future work may hope to address.
		Furthermore, this work has uncovered a number of new research areas
		warranting future enquiry. This section outlines a number of future lines
		of research.
		
		\subsection{Warehouse-scale cabling}
			
			In chapter~\ref{sec:building} I developed and implemented a number of
			cabling schemes for the SpiNNaker architecture spanning up to a six metre
			row of machine-room cabinets -- a relatively small installation by
			current standards. In SpiNNaker, the cabling exists in a 2D plane (i.e.
			across the faces of the cabinets) but as the system is scaled up, a
			single row of cabinets will tend towards a 1D line. Since embedding a 2D
			structure in a 1D space necessarily results in long connections, this
			cannot scale indefinitely.
			
			\begin{figure}
				\center
				\buildfig{figures/multi-row-cabling.tex}
				
				\caption{Multiple rows of interconnected cabinets.}
				\label{fig:multi-row-cabling}
			\end{figure}
			
			In conventional large-scale super computer installations, nodes are
			installed in rows of cabinets as illustrated in
			figure~\ref{fig:multi-row-cabling}.  From a `bird's-eye' view, the system
			approximates a 2D space, spread across the floor of a machine-room.
			Therefore, in principle, the folding and interleaving techniques
			described in chapter~\ref{sec:building} still apply. Unfortunately for
			SpiNNaker, cables connecting between rows of cabinets would be longer
			than the one metre limit imposed by its hardware because of the spacing
			between rows of cabinets.  Future SpiNNaker systems will need to consider
			alternative link technologies.  For example, a hybrid system could be
			used in which intra-cabinet connections continue to use the current HSS
			link technology while inter-cabinet links might use optical connections.
			This type of architecture could be supported by the use of pluggable
			`SFP+' transceiver modules~\cite{sff01}.
		
		\subsection{Cabling assistance for other architectures}
			
			A secondary result of the construction of prototype SpiNNaker systems in
			chapter~\ref{sec:building} was the use of real-time guidance and feedback
			to assist cable installation. I am not aware of this technique's use by
			existing architectures and, following the success experienced in this
			project, it is possible that the technique may also be useful in
			conventional systems.
			
			During the construction of prototype SpiNNaker machines, each cable took
			seconds to install compared with the minutes reported for existing
			systems~\cite{mudigonda11}. Part of this increase in efficiency appears
			to result from the immediate identification of mistakes made during
			cabling, saving time-consuming backtracking later on.
			
			In many real-world network installations, units are less densely packed
			than in SpiNNaker and so longer cables are often required. As a
			consequence, cabling errors may become more likely as cabling patterns
			are spread over a larger area making them more difficult to visually
			verify. Like SpiNNaker, conventional networking hardware is often
			equipped with a generous range of indicator LEDs and diagnostic
			facilities which might be used to implement real-time installation
			guidance. Future work could explore the use of this technique in the
			construction of other large-scale networks, such as data centres.
		
		\subsection{Congestion mitigation}
			
			\label{sec:wiggly-board-allocations}
			
			In chapter~\ref{sec:routing} I found that contiguous network faults cause
			hot-spots of congestion and routing table depletion where the PGS repair
			algorithm routed many paths around the edges of faults.  However, it is
			not just faults which can cause contiguous blockages in the network
			topology. In reality, researchers do not always require a full-sized
			SpiNNaker system to perform their experiments so large SpiNNaker systems
			are soft-partitioned on demand into many smaller
			machines~\cite{spalloc16}. To ensure isolation between partitioned
			sub-machines, HSS links between boards in different partitions are
			disabled. Because of SpiNNaker's `wrapped triple' partitioning scheme,
			the resulting sub-machines have hexagonal \emph{mesh} topologies (i.e.
			without wrap-around links) with irregular boundaries as in
			figure~\ref{fig:spalloc-mesh}.
			
			\begin{figure}
				\center
				\buildfig{figures/spalloc-mesh.tex}
				
				\caption[Irregular edges of a partitioned SpiNNaker system.]%
				{Irregular edges in a SpiNNaker system comprised of 24~boards
				partitioned from a larger machine.  Each hexagon represents a SpiNNaker
				chip. No wrap-around connections are present.}
				\label{fig:spalloc-mesh}
			\end{figure}
			
			In partitioned systems, the `tooth'-like gaps on the periphery of the
			network result in similar congestion to the HSS link failures considered
			in chapter~\ref{sec:routing}. When a route is generated between nodes on
			opposite sides of a gap, the PGS repair process will produce a
			shortest-path route around it. Since many routes may be blocked by a
			single gap, a hot-spot may develop around the corners of the gap.
			
			In chapter~\ref{sec:placement}, the `CConv' benchmark application was
			found to run correctly the majority of the time when placed by the
			simulated annealing algorithm but would occasionally fail by a
			significant margin. Preliminary experiments suggest these occasional
			failures are caused by placement solutions which place heavily
			communicating parts of the application on opposite sides of gaps along
			the perimeter of the network. Two possible approaches which future work
			may consider are presented below.
			
			\subsubsection{Avoiding hotspots with PGS repair}
				
				\label{sec:pgs-repair-improvements}	
				
				Network congestion around faults and network irregularities could be
				reduced by forcing the PGS repair process to take more varied routes
				around faults. For example, in circuit routing algorithms, routers
				avoid congestion by increasing the cost of routes which pass through
				congested areas~\cite{kahng11}. A similar technique could be used in
				PGS repair to spread the routes it produces.
				
				An alternative approach would be to adapt the base routing algorithms
				used prior to PGS repair to, for example, attempt alternative dimension
				order routes which may avoid blockages due to faulty links.
			
			\subsubsection{Fault and irregularity aware placement}
				
				One of the shortcomings of the simulated annealing based placer
				developed in chapter~\ref{sec:placement} is that it does not account
				for network faults, or irregularities, when estimating the cost of
				placement solutions.  Future work may exploit techniques used in
				congestion-aware circuit placement which could be adapted for
				application placement~\cite{viswanathan07}.
		
		\subsection{Reducing placement execution time}
			
			The simulated annealing based placer presented in
			chapter~\ref{sec:placement} produced good quality placements but its
			execution time limits its use beyond one million vertex placement
			problems. Future work should explore possibilities for improving the
			performance and scalability of this technique.
			
			In addition to considering alternative placement algorithms based on
			other methods, one possible approach is to attempt to reduce the execution
			time of simulated annealing based placement by shrinking the application
			graph being placed.
			
			For example, graph clustering~\cite{schaeffer07} may be used to group
			together strongly connected vertices which would then be placed as a
			single unit.  Unfortunately, clustering can suffer from the same problems
			as graph-partitioning-based placement: vertices may be grouped together
			in ways which, in practice, cannot be packed together into a given portion
			of a machine.  A possible solution to this problem is to use a two-phase
			placement approach~\cite{kahng11}. In a `global' placement phase,
			solutions are permitted which can slightly over-allocate resources but
			overall achieve good placement quality. In the `detailed' placement phase
			which follows, the solution is `legalised' by making small changes to the
			global placement to eliminate over allocation.
			
			An alternative approach suited to SpiNNaker could be to limit the
			clustering process to clusters which fit on a single SpiNNaker chip. In
			typical SpiNNaker application graphs, clustering to this level may reduce
			placement problem sizes by an order of magnitude and, consequently,
			reduce execution times by the same ratio. Preliminary experiments suggest
			that this approach might result in little placement quality loss for
			large placement problems whilst substantially reducing overall execution
			time.
		
		\subsection{Benchmarking}
			
			One of the most significant limitations of this study has been the
			unavailability of large-scale SpiNNaker applications for use as
			benchmarks. As a consequence, much of the scalability experimentation
			performed has relied on simple synthetic benchmarks based on projections
			of future application behaviour.
			
			In the short term, more sophisticated synthetic benchmark generation
			techniques used by the circuit placement community~\cite{nam07} may offer
			alternative benchmarks for future work. In the longer term, however, it
			is hoped that the availability of large SpiNNaker systems -- and
			placement and routing algorithms better suited to exploit them -- will
			lead to larger scale applications being developed. Hopefully these
			applications will lead to more interesting and representative benchmarks
			for use in future work.
	
	\section{Closing remarks}
		
		One of the primary outcomes of this work is that a number of the practical
		challenges faced in scaling up the SpiNNaker architecture have been
		addressed leading to the construction of large-scale SpiNNaker machines.
		The development of an effective placement algorithm for SpiNNaker
		applications has been shown to enable some neural simulations to exploit
		SpiNNaker's architecture for the first time. The availability of larger
		SpiNNaker machines paves the way for future large-scale neural modelling
		work built on much larger models such as Spaun, `the world's largest
		functional brain model'~\cite{eliasmith12}.
		
		Beyond the SpiNNaker project, the hexagonal torus topology has also been
		validated as a scalable and practical candidate for future network
		architectures. As super computers become ever larger, the physical
		scalability afforded by the 2D nature of the hexagonal torus topology may
		make it a compelling option. In addition, the finding that circuit
		placement techniques can be adapted to support placement of SpiNNaker
		software indicates that these algorithms may also be applicable to other
		applications. Indeed, if this is the case, circuit placement may offer a
		long-term source of placement algorithms able to handle the demands of
		future applications.
		
		% This thesis has explored and tackled a number of the challenges posed in
		% scaling up the unconventional SpiNNaker architecture. Along the way I have
		% demonstrated that the hexagonal torus topology may be a practical choice in
		% future applications which can scale up to the physical dimensions expected
		% of future super computers. I have also developed new efficient and
		% effective methods of placing and routing neural simulation software on
		% SpiNNaker which -- it is hoped -- will enable a new generation of large
		% scale neural simulations on spinnaker.
		
		Although this work stops short of demonstrating truly large-scale
		neuroscientific simulations running at the scale of newly completed
		SpiNNaker machines (largely because such simulations do not yet exist) a
		number of smaller-scale neural simulations have been made possible for the
		first time. The algorithms and techniques devised in this work have
		subsequently been incorporated into various software libraries and tools
		now being used by researchers building SpiNNaker applications, vindicating
		the efforts of this thesis (see appendix~\ref{sec:software}). A successor
		to the SpiNNaker architecture is also in the early stages of design and is
		building on experience of the existing architecture. The current intention
		is to retain the hexagonal torus topology used by SpiNNaker, a decision
		supported by the findings of this thesis.
		
		With SpiNNaker's hardware architecture now operating at scales close to its
		architectural limits, it is hoped that the contributions of this work will
		enable researchers to develop larger and more detailed neural models for
		this unique architecture.

	
	% Bibliography
	\bibliography{references}
	\bibliographystyle{alpha}
	
\end{document}
\unskip}
		       committed on \documentclass[12pt,twoside]{report}

\newcommand{\thesistitle}{Building and operating large-scale SpiNNaker machines}
\newcommand{\thesisauthor}{Jonathan Heathcote}
\newcommand{\thesisyear}{2016}

%%%%%%%%%%%%%%%%%%%%%%%%%%%%%%%%%%%%%%%%%%%%%%%%%%%%%%%%%%%%%%%%%%%%%%%%%%%%%%%%
% Used packages
%%%%%%%%%%%%%%%%%%%%%%%%%%%%%%%%%%%%%%%%%%%%%%%%%%%%%%%%%%%%%%%%%%%%%%%%%%%%%%%%

% Nice printing of URLs
\usepackage{url}

% Actually not tear-your-eyes-out-ugly tables
\usepackage{booktabs}

% Adjust linespacing for localised parts of the paper (e.g. abstract)
\usepackage{setspace}

% For the \ifthenelse macro
\usepackage{ifthen}

% For the \degree macro
\usepackage{gensymb}

% For subfigure support
\usepackage{caption}
\usepackage{subcaption}

% SI unit and number formatting
\usepackage{siunitx}

% Used to draw labels with a white outline to make them stand-out in diagrams
\usepackage[outline]{contour}

% TikZ + PGF Plots for diagram/plot drawing
\usepackage{tikz}
\usepackage{tikz3d}
\usepackage{pgfplots}
\usetikzlibrary{ hexagon
               , calc
               , backgrounds
               , positioning
               , decorations.pathreplacing
               , decorations.markings
               , arrows
               , positioning
               , automata
               , shadows
               , fit
               , shapes
               , arrows
               , patterns
               , spy
               }
\usepgfplotslibrary{statistics}

%%%%%%%%%%%%%%%%%%%%%%%%%%%%%%%%%%%%%%%%%%%%%%%%%%%%%%%%%%%%%%%%%%%%%%%%%%%%%%%%
% Environment settings
%%%%%%%%%%%%%%%%%%%%%%%%%%%%%%%%%%%%%%%%%%%%%%%%%%%%%%%%%%%%%%%%%%%%%%%%%%%%%%%%

% 1.5 linespacing (as required by university)
\renewcommand{\baselinestretch}{1.5}


% Specifies the thickness of the contour added by the \contour macro.
\contourlength{1.5pt}

% Define a few layers for TikZ to allow easier layering
\pgfdeclarelayer{bg}
\pgfdeclarelayer{fg}
\pgfsetlayers{bg,main,fg}

%%%%%%%%%%%%%%%%%%%%%%%%%%%%%%%%%%%%%%%%%%%%%%%%%%%%%%%%%%%%%%%%%%%%%%%%%%%%%%%%
% Definitions
%%%%%%%%%%%%%%%%%%%%%%%%%%%%%%%%%%%%%%%%%%%%%%%%%%%%%%%%%%%%%%%%%%%%%%%%%%%%%%%%

% Used in place of \chapter for preface sections. Prevents numbering but
% includes the chapter in the ToC
\newcommand{\prefacesection}[1]{
	\chapter*{#1}
	\addcontentsline{toc}{chapter}{#1}
}

% Adds 'discard if' and  'discard if not' options for \addplot to enable
% filtering of data. Taken from
% http://tex.stackexchange.com/questions/58548/is-it-possible-to-change-the-color-of-a-single-bar-when-the-bar-plot-is-based-on
\pgfplotsset{
    discard if/.style 2 args={
        x filter/.code={
            \edef\tempa{\thisrow{#1}}
            \edef\tempb{#2}
            \ifx\tempa\tempb
                \def\pgfmathresult{inf}
            \fi
        }
    },
    discard if not/.style 2 args={
        x filter/.code={
            \edef\tempa{\thisrow{#1}}
            \edef\tempb{#2}
            \ifx\tempa\tempb
            \else
                \def\pgfmathresult{inf}
            \fi
        }
    }
}

% Make PGFplots Treat "NA" (regardless of letter case) as "nan". From:
% http://tex.stackexchange.com/questions/110441/skip-specific-string-in-a-numeric-column-while-using-pgfplots
\makeatletter
\expandafter\def\csname pgffltA@N\endcsname{\pgfflt@readundef}
\expandafter\def\csname pgffltA@n\endcsname{\pgfflt@readundef}
\def\pgfflt@readundef #1{%
    \def\pgfflt@readnan@ok{1}%
    \if#1a\else\if#1A\else\def\pgfflt@readnan@ok{0}\fi\fi
    \if\pgfflt@readnan@ok1%
        \pgfmathfloat@a@S=3\relax%
        \pgfmathfloat@a@Mtok={0.0}%
        \pgfmathfloat@a@E=0%
        \expandafter\pgfflt@finish
    \else
        \def\pgfflt@readnan@{\pgfflt@error #1}%
        \expandafter\pgfflt@readnan@
    \fi
}
\makeatother

%%%%%%%%%%%%%%%%%%%%%%%%%%%%%%%%%%%%%%%%%%%%%%%%%%%%%%%%%%%%%%%%%%%%%%%%%%%%%%%%
% Document body
%%%%%%%%%%%%%%%%%%%%%%%%%%%%%%%%%%%%%%%%%%%%%%%%%%%%%%%%%%%%%%%%%%%%%%%%%%%%%%%%
\begin{document}
	
	% The title page
	\begin{titlepage}
	
	
	\begin{center}
		
		\vspace*{1.0in}
		
		{\LARGE\textbf{\thesistitle}}
		
		\vfill
		
		\textsc{A thesis submitted to the University of Manchester\\for the degree of Doctor
		of Philosophy\\in the Faculty of Science and Engineering.}
		
		\vfill
		
		\thesisyear
		
		\vfill
		
		\thesisauthor
		
		\vfill
		
		School of Computer Science
		
		\vfill
		
		\color{gray}{
			{\tiny{}Revision \texttt{\documentclass[12pt,twoside]{report}

\newcommand{\thesistitle}{Building and operating large-scale SpiNNaker machines}
\newcommand{\thesisauthor}{Jonathan Heathcote}
\newcommand{\thesisyear}{2016}

%%%%%%%%%%%%%%%%%%%%%%%%%%%%%%%%%%%%%%%%%%%%%%%%%%%%%%%%%%%%%%%%%%%%%%%%%%%%%%%%
% Used packages
%%%%%%%%%%%%%%%%%%%%%%%%%%%%%%%%%%%%%%%%%%%%%%%%%%%%%%%%%%%%%%%%%%%%%%%%%%%%%%%%

% Nice printing of URLs
\usepackage{url}

% Actually not tear-your-eyes-out-ugly tables
\usepackage{booktabs}

% Adjust linespacing for localised parts of the paper (e.g. abstract)
\usepackage{setspace}

% For the \ifthenelse macro
\usepackage{ifthen}

% For the \degree macro
\usepackage{gensymb}

% For subfigure support
\usepackage{caption}
\usepackage{subcaption}

% SI unit and number formatting
\usepackage{siunitx}

% Used to draw labels with a white outline to make them stand-out in diagrams
\usepackage[outline]{contour}

% TikZ + PGF Plots for diagram/plot drawing
\usepackage{tikz}
\usepackage{tikz3d}
\usepackage{pgfplots}
\usetikzlibrary{ hexagon
               , calc
               , backgrounds
               , positioning
               , decorations.pathreplacing
               , decorations.markings
               , arrows
               , positioning
               , automata
               , shadows
               , fit
               , shapes
               , arrows
               , patterns
               , spy
               }
\usepgfplotslibrary{statistics}

%%%%%%%%%%%%%%%%%%%%%%%%%%%%%%%%%%%%%%%%%%%%%%%%%%%%%%%%%%%%%%%%%%%%%%%%%%%%%%%%
% Environment settings
%%%%%%%%%%%%%%%%%%%%%%%%%%%%%%%%%%%%%%%%%%%%%%%%%%%%%%%%%%%%%%%%%%%%%%%%%%%%%%%%

% 1.5 linespacing (as required by university)
\renewcommand{\baselinestretch}{1.5}


% Specifies the thickness of the contour added by the \contour macro.
\contourlength{1.5pt}

% Define a few layers for TikZ to allow easier layering
\pgfdeclarelayer{bg}
\pgfdeclarelayer{fg}
\pgfsetlayers{bg,main,fg}

%%%%%%%%%%%%%%%%%%%%%%%%%%%%%%%%%%%%%%%%%%%%%%%%%%%%%%%%%%%%%%%%%%%%%%%%%%%%%%%%
% Definitions
%%%%%%%%%%%%%%%%%%%%%%%%%%%%%%%%%%%%%%%%%%%%%%%%%%%%%%%%%%%%%%%%%%%%%%%%%%%%%%%%

% Used in place of \chapter for preface sections. Prevents numbering but
% includes the chapter in the ToC
\newcommand{\prefacesection}[1]{
	\chapter*{#1}
	\addcontentsline{toc}{chapter}{#1}
}

% Adds 'discard if' and  'discard if not' options for \addplot to enable
% filtering of data. Taken from
% http://tex.stackexchange.com/questions/58548/is-it-possible-to-change-the-color-of-a-single-bar-when-the-bar-plot-is-based-on
\pgfplotsset{
    discard if/.style 2 args={
        x filter/.code={
            \edef\tempa{\thisrow{#1}}
            \edef\tempb{#2}
            \ifx\tempa\tempb
                \def\pgfmathresult{inf}
            \fi
        }
    },
    discard if not/.style 2 args={
        x filter/.code={
            \edef\tempa{\thisrow{#1}}
            \edef\tempb{#2}
            \ifx\tempa\tempb
            \else
                \def\pgfmathresult{inf}
            \fi
        }
    }
}

% Make PGFplots Treat "NA" (regardless of letter case) as "nan". From:
% http://tex.stackexchange.com/questions/110441/skip-specific-string-in-a-numeric-column-while-using-pgfplots
\makeatletter
\expandafter\def\csname pgffltA@N\endcsname{\pgfflt@readundef}
\expandafter\def\csname pgffltA@n\endcsname{\pgfflt@readundef}
\def\pgfflt@readundef #1{%
    \def\pgfflt@readnan@ok{1}%
    \if#1a\else\if#1A\else\def\pgfflt@readnan@ok{0}\fi\fi
    \if\pgfflt@readnan@ok1%
        \pgfmathfloat@a@S=3\relax%
        \pgfmathfloat@a@Mtok={0.0}%
        \pgfmathfloat@a@E=0%
        \expandafter\pgfflt@finish
    \else
        \def\pgfflt@readnan@{\pgfflt@error #1}%
        \expandafter\pgfflt@readnan@
    \fi
}
\makeatother

%%%%%%%%%%%%%%%%%%%%%%%%%%%%%%%%%%%%%%%%%%%%%%%%%%%%%%%%%%%%%%%%%%%%%%%%%%%%%%%%
% Document body
%%%%%%%%%%%%%%%%%%%%%%%%%%%%%%%%%%%%%%%%%%%%%%%%%%%%%%%%%%%%%%%%%%%%%%%%%%%%%%%%
\begin{document}
	
	% The title page
	\input{titlepage}
	
	% The table of contents which, per university regulations, is followed by a
	% total wordcount.
	\tableofcontents
	\vfill
	\noindent This thesis contains
		\immediate\write18{texcount -1 -sum -inc thesis.tex > thesis.wordcount}%
		\input{thesis.wordcount}words.
	
	\clearpage
	\listoffigures
	
	\clearpage
	\listoftables
	
	% Abstract
	\input{abstract}
	
	% Declaration of non-submission elsewhere
	\input{declaration}
	
	% University-prescribed copyright statement...
	\input{copyright}
	
	% Acknowledgements
	\input{acknowledgements}
	
	% Main body
	\input{introduction.tex}
	\input{background.tex}
	\input{building.tex}
	\input{shortestPaths.tex}
	\input{routing.tex}
	\input{placement.tex}
	\input{discussion.tex}
	\input{future.tex}
	\input{conclusions.tex}
	
	% Bibliography
	\bibliography{references}
	\bibliographystyle{alpha}
	
\end{document}
}\documentclass[12pt,twoside]{report}

\newcommand{\thesistitle}{Building and operating large-scale SpiNNaker machines}
\newcommand{\thesisauthor}{Jonathan Heathcote}
\newcommand{\thesisyear}{2016}

%%%%%%%%%%%%%%%%%%%%%%%%%%%%%%%%%%%%%%%%%%%%%%%%%%%%%%%%%%%%%%%%%%%%%%%%%%%%%%%%
% Used packages
%%%%%%%%%%%%%%%%%%%%%%%%%%%%%%%%%%%%%%%%%%%%%%%%%%%%%%%%%%%%%%%%%%%%%%%%%%%%%%%%

% Nice printing of URLs
\usepackage{url}

% Actually not tear-your-eyes-out-ugly tables
\usepackage{booktabs}

% Adjust linespacing for localised parts of the paper (e.g. abstract)
\usepackage{setspace}

% For the \ifthenelse macro
\usepackage{ifthen}

% For the \degree macro
\usepackage{gensymb}

% For subfigure support
\usepackage{caption}
\usepackage{subcaption}

% SI unit and number formatting
\usepackage{siunitx}

% Used to draw labels with a white outline to make them stand-out in diagrams
\usepackage[outline]{contour}

% TikZ + PGF Plots for diagram/plot drawing
\usepackage{tikz}
\usepackage{tikz3d}
\usepackage{pgfplots}
\usetikzlibrary{ hexagon
               , calc
               , backgrounds
               , positioning
               , decorations.pathreplacing
               , decorations.markings
               , arrows
               , positioning
               , automata
               , shadows
               , fit
               , shapes
               , arrows
               , patterns
               , spy
               }
\usepgfplotslibrary{statistics}

%%%%%%%%%%%%%%%%%%%%%%%%%%%%%%%%%%%%%%%%%%%%%%%%%%%%%%%%%%%%%%%%%%%%%%%%%%%%%%%%
% Environment settings
%%%%%%%%%%%%%%%%%%%%%%%%%%%%%%%%%%%%%%%%%%%%%%%%%%%%%%%%%%%%%%%%%%%%%%%%%%%%%%%%

% 1.5 linespacing (as required by university)
\renewcommand{\baselinestretch}{1.5}


% Specifies the thickness of the contour added by the \contour macro.
\contourlength{1.5pt}

% Define a few layers for TikZ to allow easier layering
\pgfdeclarelayer{bg}
\pgfdeclarelayer{fg}
\pgfsetlayers{bg,main,fg}

%%%%%%%%%%%%%%%%%%%%%%%%%%%%%%%%%%%%%%%%%%%%%%%%%%%%%%%%%%%%%%%%%%%%%%%%%%%%%%%%
% Definitions
%%%%%%%%%%%%%%%%%%%%%%%%%%%%%%%%%%%%%%%%%%%%%%%%%%%%%%%%%%%%%%%%%%%%%%%%%%%%%%%%

% Used in place of \chapter for preface sections. Prevents numbering but
% includes the chapter in the ToC
\newcommand{\prefacesection}[1]{
	\chapter*{#1}
	\addcontentsline{toc}{chapter}{#1}
}

% Adds 'discard if' and  'discard if not' options for \addplot to enable
% filtering of data. Taken from
% http://tex.stackexchange.com/questions/58548/is-it-possible-to-change-the-color-of-a-single-bar-when-the-bar-plot-is-based-on
\pgfplotsset{
    discard if/.style 2 args={
        x filter/.code={
            \edef\tempa{\thisrow{#1}}
            \edef\tempb{#2}
            \ifx\tempa\tempb
                \def\pgfmathresult{inf}
            \fi
        }
    },
    discard if not/.style 2 args={
        x filter/.code={
            \edef\tempa{\thisrow{#1}}
            \edef\tempb{#2}
            \ifx\tempa\tempb
            \else
                \def\pgfmathresult{inf}
            \fi
        }
    }
}

% Make PGFplots Treat "NA" (regardless of letter case) as "nan". From:
% http://tex.stackexchange.com/questions/110441/skip-specific-string-in-a-numeric-column-while-using-pgfplots
\makeatletter
\expandafter\def\csname pgffltA@N\endcsname{\pgfflt@readundef}
\expandafter\def\csname pgffltA@n\endcsname{\pgfflt@readundef}
\def\pgfflt@readundef #1{%
    \def\pgfflt@readnan@ok{1}%
    \if#1a\else\if#1A\else\def\pgfflt@readnan@ok{0}\fi\fi
    \if\pgfflt@readnan@ok1%
        \pgfmathfloat@a@S=3\relax%
        \pgfmathfloat@a@Mtok={0.0}%
        \pgfmathfloat@a@E=0%
        \expandafter\pgfflt@finish
    \else
        \def\pgfflt@readnan@{\pgfflt@error #1}%
        \expandafter\pgfflt@readnan@
    \fi
}
\makeatother

%%%%%%%%%%%%%%%%%%%%%%%%%%%%%%%%%%%%%%%%%%%%%%%%%%%%%%%%%%%%%%%%%%%%%%%%%%%%%%%%
% Document body
%%%%%%%%%%%%%%%%%%%%%%%%%%%%%%%%%%%%%%%%%%%%%%%%%%%%%%%%%%%%%%%%%%%%%%%%%%%%%%%%
\begin{document}
	
	% The title page
	\input{titlepage}
	
	% The table of contents which, per university regulations, is followed by a
	% total wordcount.
	\tableofcontents
	\vfill
	\noindent This thesis contains
		\immediate\write18{texcount -1 -sum -inc thesis.tex > thesis.wordcount}%
		\input{thesis.wordcount}words.
	
	\clearpage
	\listoffigures
	
	\clearpage
	\listoftables
	
	% Abstract
	\input{abstract}
	
	% Declaration of non-submission elsewhere
	\input{declaration}
	
	% University-prescribed copyright statement...
	\input{copyright}
	
	% Acknowledgements
	\input{acknowledgements}
	
	% Main body
	\input{introduction.tex}
	\input{background.tex}
	\input{building.tex}
	\input{shortestPaths.tex}
	\input{routing.tex}
	\input{placement.tex}
	\input{discussion.tex}
	\input{future.tex}
	\input{conclusions.tex}
	
	% Bibliography
	\bibliography{references}
	\bibliographystyle{alpha}
	
\end{document}
}
		}
		
	\end{center}
	
\end{titlepage}

	
	% The table of contents which, per university regulations, is followed by a
	% total wordcount.
	\tableofcontents
	\vfill
	\noindent This thesis contains
		\immediate\write18{texcount -1 -sum -inc thesis.tex > thesis.wordcount}%
		\documentclass[12pt,twoside]{report}

\newcommand{\thesistitle}{Building and operating large-scale SpiNNaker machines}
\newcommand{\thesisauthor}{Jonathan Heathcote}
\newcommand{\thesisyear}{2016}

%%%%%%%%%%%%%%%%%%%%%%%%%%%%%%%%%%%%%%%%%%%%%%%%%%%%%%%%%%%%%%%%%%%%%%%%%%%%%%%%
% Used packages
%%%%%%%%%%%%%%%%%%%%%%%%%%%%%%%%%%%%%%%%%%%%%%%%%%%%%%%%%%%%%%%%%%%%%%%%%%%%%%%%

% Nice printing of URLs
\usepackage{url}

% Actually not tear-your-eyes-out-ugly tables
\usepackage{booktabs}

% Adjust linespacing for localised parts of the paper (e.g. abstract)
\usepackage{setspace}

% For the \ifthenelse macro
\usepackage{ifthen}

% For the \degree macro
\usepackage{gensymb}

% For subfigure support
\usepackage{caption}
\usepackage{subcaption}

% SI unit and number formatting
\usepackage{siunitx}

% Used to draw labels with a white outline to make them stand-out in diagrams
\usepackage[outline]{contour}

% TikZ + PGF Plots for diagram/plot drawing
\usepackage{tikz}
\usepackage{tikz3d}
\usepackage{pgfplots}
\usetikzlibrary{ hexagon
               , calc
               , backgrounds
               , positioning
               , decorations.pathreplacing
               , decorations.markings
               , arrows
               , positioning
               , automata
               , shadows
               , fit
               , shapes
               , arrows
               , patterns
               , spy
               }
\usepgfplotslibrary{statistics}

%%%%%%%%%%%%%%%%%%%%%%%%%%%%%%%%%%%%%%%%%%%%%%%%%%%%%%%%%%%%%%%%%%%%%%%%%%%%%%%%
% Environment settings
%%%%%%%%%%%%%%%%%%%%%%%%%%%%%%%%%%%%%%%%%%%%%%%%%%%%%%%%%%%%%%%%%%%%%%%%%%%%%%%%

% 1.5 linespacing (as required by university)
\renewcommand{\baselinestretch}{1.5}


% Specifies the thickness of the contour added by the \contour macro.
\contourlength{1.5pt}

% Define a few layers for TikZ to allow easier layering
\pgfdeclarelayer{bg}
\pgfdeclarelayer{fg}
\pgfsetlayers{bg,main,fg}

%%%%%%%%%%%%%%%%%%%%%%%%%%%%%%%%%%%%%%%%%%%%%%%%%%%%%%%%%%%%%%%%%%%%%%%%%%%%%%%%
% Definitions
%%%%%%%%%%%%%%%%%%%%%%%%%%%%%%%%%%%%%%%%%%%%%%%%%%%%%%%%%%%%%%%%%%%%%%%%%%%%%%%%

% Used in place of \chapter for preface sections. Prevents numbering but
% includes the chapter in the ToC
\newcommand{\prefacesection}[1]{
	\chapter*{#1}
	\addcontentsline{toc}{chapter}{#1}
}

% Adds 'discard if' and  'discard if not' options for \addplot to enable
% filtering of data. Taken from
% http://tex.stackexchange.com/questions/58548/is-it-possible-to-change-the-color-of-a-single-bar-when-the-bar-plot-is-based-on
\pgfplotsset{
    discard if/.style 2 args={
        x filter/.code={
            \edef\tempa{\thisrow{#1}}
            \edef\tempb{#2}
            \ifx\tempa\tempb
                \def\pgfmathresult{inf}
            \fi
        }
    },
    discard if not/.style 2 args={
        x filter/.code={
            \edef\tempa{\thisrow{#1}}
            \edef\tempb{#2}
            \ifx\tempa\tempb
            \else
                \def\pgfmathresult{inf}
            \fi
        }
    }
}

% Make PGFplots Treat "NA" (regardless of letter case) as "nan". From:
% http://tex.stackexchange.com/questions/110441/skip-specific-string-in-a-numeric-column-while-using-pgfplots
\makeatletter
\expandafter\def\csname pgffltA@N\endcsname{\pgfflt@readundef}
\expandafter\def\csname pgffltA@n\endcsname{\pgfflt@readundef}
\def\pgfflt@readundef #1{%
    \def\pgfflt@readnan@ok{1}%
    \if#1a\else\if#1A\else\def\pgfflt@readnan@ok{0}\fi\fi
    \if\pgfflt@readnan@ok1%
        \pgfmathfloat@a@S=3\relax%
        \pgfmathfloat@a@Mtok={0.0}%
        \pgfmathfloat@a@E=0%
        \expandafter\pgfflt@finish
    \else
        \def\pgfflt@readnan@{\pgfflt@error #1}%
        \expandafter\pgfflt@readnan@
    \fi
}
\makeatother

%%%%%%%%%%%%%%%%%%%%%%%%%%%%%%%%%%%%%%%%%%%%%%%%%%%%%%%%%%%%%%%%%%%%%%%%%%%%%%%%
% Document body
%%%%%%%%%%%%%%%%%%%%%%%%%%%%%%%%%%%%%%%%%%%%%%%%%%%%%%%%%%%%%%%%%%%%%%%%%%%%%%%%
\begin{document}
	
	% The title page
	\begin{titlepage}
	
	
	\begin{center}
		
		\vspace*{1.0in}
		
		{\LARGE\textbf{\thesistitle}}
		
		\vfill
		
		\textsc{A thesis submitted to the University of Manchester\\for the degree of Doctor
		of Philosophy\\in the Faculty of Science and Engineering.}
		
		\vfill
		
		\thesisyear
		
		\vfill
		
		\thesisauthor
		
		\vfill
		
		School of Computer Science
		
		\vfill
		
		\color{gray}{
			{\tiny{}Revision \texttt{\input{thesis.fullhash}}\input{thesis.date}}
		}
		
	\end{center}
	
\end{titlepage}

	
	% The table of contents which, per university regulations, is followed by a
	% total wordcount.
	\tableofcontents
	\vfill
	\noindent This thesis contains
		\immediate\write18{texcount -1 -sum -inc thesis.tex > thesis.wordcount}%
		\documentclass[12pt,twoside]{report}

\newcommand{\thesistitle}{Building and operating large-scale SpiNNaker machines}
\newcommand{\thesisauthor}{Jonathan Heathcote}
\newcommand{\thesisyear}{2016}

%%%%%%%%%%%%%%%%%%%%%%%%%%%%%%%%%%%%%%%%%%%%%%%%%%%%%%%%%%%%%%%%%%%%%%%%%%%%%%%%
% Used packages
%%%%%%%%%%%%%%%%%%%%%%%%%%%%%%%%%%%%%%%%%%%%%%%%%%%%%%%%%%%%%%%%%%%%%%%%%%%%%%%%

% Nice printing of URLs
\usepackage{url}

% Actually not tear-your-eyes-out-ugly tables
\usepackage{booktabs}

% Adjust linespacing for localised parts of the paper (e.g. abstract)
\usepackage{setspace}

% For the \ifthenelse macro
\usepackage{ifthen}

% For the \degree macro
\usepackage{gensymb}

% For subfigure support
\usepackage{caption}
\usepackage{subcaption}

% SI unit and number formatting
\usepackage{siunitx}

% Used to draw labels with a white outline to make them stand-out in diagrams
\usepackage[outline]{contour}

% TikZ + PGF Plots for diagram/plot drawing
\usepackage{tikz}
\usepackage{tikz3d}
\usepackage{pgfplots}
\usetikzlibrary{ hexagon
               , calc
               , backgrounds
               , positioning
               , decorations.pathreplacing
               , decorations.markings
               , arrows
               , positioning
               , automata
               , shadows
               , fit
               , shapes
               , arrows
               , patterns
               , spy
               }
\usepgfplotslibrary{statistics}

%%%%%%%%%%%%%%%%%%%%%%%%%%%%%%%%%%%%%%%%%%%%%%%%%%%%%%%%%%%%%%%%%%%%%%%%%%%%%%%%
% Environment settings
%%%%%%%%%%%%%%%%%%%%%%%%%%%%%%%%%%%%%%%%%%%%%%%%%%%%%%%%%%%%%%%%%%%%%%%%%%%%%%%%

% 1.5 linespacing (as required by university)
\renewcommand{\baselinestretch}{1.5}


% Specifies the thickness of the contour added by the \contour macro.
\contourlength{1.5pt}

% Define a few layers for TikZ to allow easier layering
\pgfdeclarelayer{bg}
\pgfdeclarelayer{fg}
\pgfsetlayers{bg,main,fg}

%%%%%%%%%%%%%%%%%%%%%%%%%%%%%%%%%%%%%%%%%%%%%%%%%%%%%%%%%%%%%%%%%%%%%%%%%%%%%%%%
% Definitions
%%%%%%%%%%%%%%%%%%%%%%%%%%%%%%%%%%%%%%%%%%%%%%%%%%%%%%%%%%%%%%%%%%%%%%%%%%%%%%%%

% Used in place of \chapter for preface sections. Prevents numbering but
% includes the chapter in the ToC
\newcommand{\prefacesection}[1]{
	\chapter*{#1}
	\addcontentsline{toc}{chapter}{#1}
}

% Adds 'discard if' and  'discard if not' options for \addplot to enable
% filtering of data. Taken from
% http://tex.stackexchange.com/questions/58548/is-it-possible-to-change-the-color-of-a-single-bar-when-the-bar-plot-is-based-on
\pgfplotsset{
    discard if/.style 2 args={
        x filter/.code={
            \edef\tempa{\thisrow{#1}}
            \edef\tempb{#2}
            \ifx\tempa\tempb
                \def\pgfmathresult{inf}
            \fi
        }
    },
    discard if not/.style 2 args={
        x filter/.code={
            \edef\tempa{\thisrow{#1}}
            \edef\tempb{#2}
            \ifx\tempa\tempb
            \else
                \def\pgfmathresult{inf}
            \fi
        }
    }
}

% Make PGFplots Treat "NA" (regardless of letter case) as "nan". From:
% http://tex.stackexchange.com/questions/110441/skip-specific-string-in-a-numeric-column-while-using-pgfplots
\makeatletter
\expandafter\def\csname pgffltA@N\endcsname{\pgfflt@readundef}
\expandafter\def\csname pgffltA@n\endcsname{\pgfflt@readundef}
\def\pgfflt@readundef #1{%
    \def\pgfflt@readnan@ok{1}%
    \if#1a\else\if#1A\else\def\pgfflt@readnan@ok{0}\fi\fi
    \if\pgfflt@readnan@ok1%
        \pgfmathfloat@a@S=3\relax%
        \pgfmathfloat@a@Mtok={0.0}%
        \pgfmathfloat@a@E=0%
        \expandafter\pgfflt@finish
    \else
        \def\pgfflt@readnan@{\pgfflt@error #1}%
        \expandafter\pgfflt@readnan@
    \fi
}
\makeatother

%%%%%%%%%%%%%%%%%%%%%%%%%%%%%%%%%%%%%%%%%%%%%%%%%%%%%%%%%%%%%%%%%%%%%%%%%%%%%%%%
% Document body
%%%%%%%%%%%%%%%%%%%%%%%%%%%%%%%%%%%%%%%%%%%%%%%%%%%%%%%%%%%%%%%%%%%%%%%%%%%%%%%%
\begin{document}
	
	% The title page
	\input{titlepage}
	
	% The table of contents which, per university regulations, is followed by a
	% total wordcount.
	\tableofcontents
	\vfill
	\noindent This thesis contains
		\immediate\write18{texcount -1 -sum -inc thesis.tex > thesis.wordcount}%
		\input{thesis.wordcount}words.
	
	\clearpage
	\listoffigures
	
	\clearpage
	\listoftables
	
	% Abstract
	\input{abstract}
	
	% Declaration of non-submission elsewhere
	\input{declaration}
	
	% University-prescribed copyright statement...
	\input{copyright}
	
	% Acknowledgements
	\input{acknowledgements}
	
	% Main body
	\input{introduction.tex}
	\input{background.tex}
	\input{building.tex}
	\input{shortestPaths.tex}
	\input{routing.tex}
	\input{placement.tex}
	\input{discussion.tex}
	\input{future.tex}
	\input{conclusions.tex}
	
	% Bibliography
	\bibliography{references}
	\bibliographystyle{alpha}
	
\end{document}
words.
	
	\clearpage
	\listoffigures
	
	\clearpage
	\listoftables
	
	% Abstract
	{
	\prefacesection{Abstract}
	
	% Single line spacing for the abstract page
	\setstretch{1.0}
	
	
	\vfill
	
	% Standard thesis information
	\begin{center}
		\textsc{\large\thesistitle}
		
		\vspace{0.5em}
		
		\thesisauthor
		
		\vspace{0.5em}
		
		A thesis submitted to the University of Manchester\\
		for the degree of Doctor of Philosophy, 2016
	\end{center}
	
	\vfill
	
	% The abstract
	
	SpiNNaker is an unconventional super computer architecture designed to
	simulate up to one billion biologically realistic neurons in real-time. To
	achieve this goal, SpiNNaker employs a novel network architecture which poses
	a number of practical problems in scaling up from desktop prototypes to
	machine room filling installations.
	
	SpiNNaker's hexagonal torus network topology has received mostly theoretical
	treatment in the literature. This thesis tackles some of the challenges
	encountered when building `real-world' systems.  Firstly, a scheme is devised
	for physically laying out hexagonal torus topologies in machine rooms which
	avoids long cables; this is demonstrated on a half-million core SpiNNaker
	prototype.  Secondly, to improve the performance of existing routing
	algorithms, a more efficient process is proposed for finding (logically)
	short paths through hexagonal torus topologies. This is complemented by a
	formula which provides routing algorithms greater flexibility when finding
	paths, potentially resulting in a more balanced network utilisation.
	
	The scale of SpiNNaker's network and the models intended for it also present
	their own challenges. Placement and routing algorithms are developed which
	assign processes to nodes and generate paths through SpiNNaker's network.
	These algorithms reduce congestion and tolerate network faults. The proposed
	placement algorithm is inspired by techniques used in chip design and is
	shown to enable larger applications to run on SpiNNaker -- with good
	performance -- than the previous state-of-the-art. Likewise the routing
	algorithm developed is able to tolerate network faults, inevitably present in
	large scale systems, with little performance overhead.
	
	
	% Required to ensure single line spacing is used for this whole block
	\par%
}

	
	% Declaration of non-submission elsewhere
	\prefacesection{Declaration}

% Single line spacing for the declaration
{
	\setstretch{1.0}
	No portion of the work referred to in this thesis has been submitted in support
	of an application for another degree or qualification of this or any other
	university or other institute of learning.
	
	\par%
}


	
	% University-prescribed copyright statement...
	\input{copyright}
	
	% Acknowledgements
	{
	\prefacesection{Acknowledgements}
	
	% Single line spacing
	\setstretch{1.0}
	
	It is often said that it is not \emph{what} you know but \emph{who} you know.
	Throughout the course of my PhD I've been exceptionally lucky to have been
	helped along by a great number of people.
	
	Both my supervisor, Jim Garside, and co-supervisor, Steve Furber, have each
	spent countless hours patiently discussing and describing all manner of
	things with me while giving me great freedom in my project. Jim's office door
	has always been open to my unexpected interruptions be it about work, writing
	or walking.  Likewise, Steve has always managed to find time for both
	technical and frivolous endeavours alike. I'm also hugely grateful to Luis
	Plana who has been a rich source of sage advice, insightful questions
	patiently suffered many a foolish question.
	
	Various parts of the work in this thesis (and numerous side projects) would
	not have been possible if not for the multitude of discussions,
	collaborations and even sheer physical hard work of Steve Temple, Javier
	Navaridas, Simon Davidson and Dave Clark. I'm also indebted to Andrew Mundy
	and Jamie Knight, both of whom have donated so much time and effort towards
	verifying and using software implementations of the ideas in this thesis.
	
	The injection of lunchtime silliness by Andrew and Jamie, along with Amanieu
	d'Antras and Andrew Webb and the other CDT members has always brightened my
	day. So to has the friendly and stimulating environment of the School of
	Computer Science and its many staff and students. Of course, I am also very
	grateful for the funding the school has provided for my research.
	
	I cannot thank my wonderful wife, Ann-Marie, enough for being by my side. She
	has given me so much kindness, love and patience and endured a lifetime's
	quota of conversations about hexagons. Finally, thanks too to rest of my
	family, especially my parents, who are to blame for starting me down this
	path and co-suffering drafts and endless rants about this document.
	
	% Required to ensure single line spacing is used for this whole block
	\par%
}

	
	% Main body
	\chapter{Introduction}

\label{sec:introduction}

%Problem area
%
%* Network construction and exploitation
%  * Cabling: Build it cheaply in terms of tech cost and install cost
%  * Routing: Get around it cheaply and reliably
%  * Placement: Use it efficiently

The Spiking Neural Network Architecture (SpiNNaker) is a novel super computer
architecture designed to simulate biologically realistic models of brains in
real time \cite{furber07}. Though neurons, the building blocks of the brain,
are relatively well understood, their complex interactions remain mysterious.
Just as understanding the workings of a transistor is insufficient to
understand a modern microprocessor, neuroscientists believe that understanding
the neurons in isolation cannot explain the brain and that understanding their
connectivity is key \cite{eliasmith13,eliasmith14}. Experiments on real brains,
however, are fraught with difficulty. Variations between individuals can be
significant and it is only possible to record tens or hundreds of the trillions
of signals in the brain, and even then only with limited control over which
signals are recorded. Computer simulations of models of large neural networks,
however, enable researchers to develop repeatable experiments and gain complete
visibility of any signal and any neuron. Models such as SPAUN
\cite{eliasmith12}, built from millions of simulated neurons, have shown great
promise in expanding our understanding of higher level brain functions such as
memory and simple problem solving.  Unfortunately these neural models are
expensive to simulate, requiring hours of compute time to simulate each second
of neural activity. As well as being inconvenient, this precludes the use of
robotics to immerse these models in real world environments and also limits
studies of long-term behaviours such as learning.

SpiNNaker is designed to enable the real time simulation of models containing
up to one billion neurons -- approximately \SI{1}{\percent} of a human brain or
ten mouse brains \cite{furber06}. To achieve this goal, the largest planned
SpiNNaker machine will contain over one million low-powered computer processors
interconnected by a bespoke network architecture.

SpiNNaker's large processor count matches the current trend in super computers
where processor counts are growing exponentially \cite{meuer16j}, mimicking the
growth of the number of components in the processors themselves predicted by
Gordon Moore's famous `law' \cite{moore75}. As a result of this growth, the
interconnection networks which enable these processors to work together have
grown in importance \cite{dally04}.  Network designers must carefully balance
performance against practicality and financial cost.  SpiNNaker's network is no
exception to this rule and, as the systems scale up from desktop prototypes to
machine-room scale installations, the reality of building and exploiting these
machines presents an array of challenges.

As in all super computers, SpiNNaker's network interconnects its processors in
a particular network topology which defines how different processors may
communicate with each other. Unlike the tree and $N$-dimensional torus
topologies found in contemporary super computers \cite{dally04}, SpiNNaker
employs a `hexagonal torus topology'. In this topology, nodes in SpiNNaker's
network fit together in a honeycomb-like pattern where messages may `hop' from
node to node to reach their destination. As we will see in
chapter~\ref{sec:background}, the hexagonal torus topology, in theory, sits at
a `sweet spot' in terms of network performance and practicality. As the first
known large-scale installation of the hexagonal torus topology, however, there
remain a number of practical challenges for large spinnaker machines arising
from this choice.

As super computer networks have grown in scale to millions of processors the
task of dealing with previously rare faults has grown.  Though fault rates in
networks remain consistently low, architectures such as SpiNNaker may have
hundreds of thousands of links meaning even fault rates of a fraction of a
percent will impact tens or hundreds of links. To enable reliable operation,
networks must be able to adapt the routes taken by messages through the network
to avoid faulty links and nodes. The techniques employed are often closely tied
to a particular network architecture and consequently SpiNNaker's novel network
architecture demands its own approach.

Another challenge introduced by the growing scale of super computers is making
\emph{efficient} use of network resources. Communicating processes should be
located on logically `nearby' nodes to reduce network load. The neural models
for which SpiNNaker is designed are often described abstractly, rather than
geometrically, using modelling languages such as PyNN~\cite{davison08} and
Nengo~\cite{eliasmith04}.  Because of this, the communication requirements of
simulations can be highly irregular making an efficient placement of processes
onto processors in the machine non-trivial.

%Contributions
%
%* Cabling scheme for hexagonal toruses without long cables
%* Efficient installation technique for dense systems
%* Exhaustive and efficient route calculation in hex toruses
%* Fault tolerant routing scheme exploiting SpiNNaker's odd router
%* Placement based on SA a: works very well and b: suggests circuit placement is
%  a good source of inspiration.

This thesis addresses the practical challenges of scaling up the SpiNNaker
architecture in a real-world setting summarised by these research questions:

\begin{enumerate}
	
	\item Can the hexagonal torus topology be deployed and used in real, large
	scale systems?
	
	\item Does SpiNNaker's router architecture help, or hinder fault tolerance?
	
	\item How can the parts of a neural simulation be placed onto a large
	hexagonal torus topology to reduce network load?
	
\end{enumerate}

%Structure
%
%* Chapter 2: Background: detailed dive into what's in SpiNNaker, why its
%  really so unusual. Also looks at what applications run on SpiNNaker and how
%  they work.
%* Chapter 3: How to build a really big SpiNNaker machine.
%* Chapter 4: How to find your way around that machine.
%* Chapter 5: How to find your way around that machine even when its broken.
%* Chapter 6: Now you can walk, time to run.
%* Chapter 7: Wrapping up.
%* Appendices: Hard-to-come-by theoretical and practical details useful if
%  you're about to continue where this research left off but be useful but
%  otherwise hard to come by, especially in one place.

Chapter~\ref{sec:background} introduces the SpiNNaker architecture and, in
particular, describes its hexagonal torus topology and network architecture.

In chapter~\ref{sec:building}, I develop a cabling scheme for large hexagonal
torus topologies which enables arbitrarily large networks to be constructed
using only short, inexpensive cables. This theoretical work is then evaluated
through the construction of a range of prototype SpiNNaker systems. The largest
of these prototypes contains over half a million processor cores and spans
several machine room cabinets. In addition, I propose the use of built-in
diagnostic facilities to assist technicians performing network installation and
maintenance. This technique is found to greatly reduce the effort required and
the number of mistakes made.

In chapters~\ref{sec:shortestPaths}~and~\ref{sec:routing} I develop new routing
techniques for SpiNNaker's network. Chapter~\ref{sec:shortestPaths} develops a
new approach to finding the shortest paths through hexagonal torus topologies,
an integral part of many routing algorithms. This newly proposed approach is
cheaper to compute than the state of the art and, unlike previous efforts, is
able to discover all valid short paths through the topology. This theoretical
advance brings hexagonal torus topologies in line with conventional topologies
by providing routing algorithms with complete information about the paths
available to them. In chapter \ref{sec:routing} I propose a fault tolerant
routing algorithm for SpiNNaker which is able to avoid arbitrary static fault
patterns with minimal performance overhead. A key finding of this chapter is
that the flexibility afforded to fault tolerant routing algorithms by
SpiNNaker's unconventional router architecture is what facilities the low
overheads reported in this chapter.

Finally, in chapter~\ref{sec:placement}, I explore the problem of application
placement in SpiNNaker's network. As in other networks and applications, neural
simulations should be arranged such that communication occurs primarily between
processors close together in the network to control network load. Due to the
irregular connectivity and large scale of the neural models expected to run on
SpiNNaker, an automated approach is necessary. I develop a novel placement
algorithm based on algorithms used for circuit layout in computer chips. My
algorithm is found to allow some larger neural models to run on SpiNNaker for
the first time while enabling other applications to run at greater speeds. In
addition, synthetic benchmarks containing over one million processes indicate
that this algorithm should handle the anticipated demands of the neural models
expected to run on large-scale SpiNNaker installations.

	\chapter{The SpiNNaker Architecture}
	
	\label{sec:background}
	
	SpiNNaker is a massively parallel computer architecture designed to simulate
	biologically realistic neural models \cite{furber07}. In this chapter we will
	explore this unconventional architecture in detail, starting with its purpose
	before focusing on its most unconventional feature: its network.
	
	% * Purpose
	%   * Spiking neural simulations
	%     * Neural modelling: PyNN, Nengo...
	%     * Parallelisation + communication
	
	\section{Neural simulation}
		
		Human brains contain billions of neurons connected together by trillions of
		`synapses'. Neurons communicate by transmitting and receiving `spikes'
		through their synapses. Each spike is `valueless' in that a spike's only
		significant features are when it arrives and where it has come from.
		
		\begin{figure}
			\center
			\buildfig{figures/lif-neuron.tex}
			
			\caption{A Leaky Integrate-and-Fire (LIF) neuron.}
			\label{fig:lif-neuron}
		\end{figure}
		
		Though some detailed models of the electrochemical processes occurring
		inside neurons are computationally intensive, simplified models such as the
		Leaky Integrate-and-Fire (LIF) model can be implemented in just a handful
		of CPU instructions \cite{vainbrand11}. Figure~\ref{fig:lif-neuron}
		illustrates a simple LIF neuron in which incoming spikes cause charge to
		build up (integrated) which over time, leaks away. If an incoming spike
		causes the charge to rise above a certain threshold, the neuron `fires'
		producing an outgoing spike. Despite the simplicity of this model, large
		neural networks such as Spaun \cite{eliasmith12} -- built entirely from LIF
		neurons -- exhibit complex behaviours such as fine motor control and
		problem solving.
		
		The computational expense of large scale neural simulations does not arise
		from the cost of modelling neurons but instead from distributing spikes. In
		biology, neurons produce spikes at an average rate of \SI{10}{\hertz} and
		synapses connect each neuron's output to (order) \num{1000}~neurons
		\cite{navaridas09}. Consider an example neural model with $7\times10^7$
		neurons, approximately the number in a house mouse and
		$\nicefrac{1}{10}^\textrm{th}$ of the design target of SpiNNaker. This
		network might produce $7\times10^8$~spikes per second. Because each neuron
		connects to many others, this equates to $7\times10^{11}$ spikes being
		received per second. If each spike were transmitted as a UDP datagram
		containing a single \SI{32}{\bit} payload, the total network throughput
		required for this simulation would be \SI{179.2}{\tera\bit\per\second}. At
		the time of writing, this is more than double the bisection bandwidth (the
		theoretical worst-case throughput) of the world's most powerful super
		computer \cite{dongarra16}.
	
	\section{Network architecture}
		
		Architectures such as IBM's Blue Gene \cite{chiu11} and Cray's XK7
		\cite{ornl16} employ powerful compute nodes connected together using
		networks designed to transfer multi-kilobyte blocks of data between nodes.
		Since neural models have relatively light computational requirements and
		communications are based on small pieces of data (spikes), this type of
		architecture is poorly suited to the task.
		
		SpiNNaker's architectural target is to support realtime simulations of up
		to one billion neurons. Since neural models such as LIF are inexpensive to
		model and many neurons can be simulated independently in parallel,
		SpiNNaker employs many small, energy efficient ARM processors
		\cite{furber07}. To support the unusual communication requirements of
		neural simulations, a bespoke interconnection network is used which is the
		background to this thesis.
		
	%   * SpiNNaker chip
	%     * Cores
	%     * SDRAM
	%     * NoC
	%     * Router
		
		\begin{figure}
			\center
			%\includegraphics[width=19mm]{figures/spinnakerChip.jpg}
			\buildfig{figures/hex-chips.tex}
			
			\caption[SpiNNaker chips connected to their six neighbours.]%
			{SpiNNaker chips (actual size) connected to their six neighbours.}
			\label{fig:spinnakerChip}
		\end{figure}
		
		The fundamental building block of the SpiNNaker architecture is the
		SpiNNaker chip (figure \ref{fig:spinnakerChip}) \cite{furber13}. Each chip
		contains eighteen low power ARM 968 processor cores each capable of
		simulating between \num{200} and \num{2000} LIF neurons in real time
		\cite{mundy15}.  Each core has a total of \SI{96}{\kilo\byte} of private
		Tightly-Coupled Memory (TCM) and shares access to \SI{128}{\mega\byte} of
		on-chip SDRAM with other cores on the same chip. Finally, each chip
		contains a programmable router which routes network packets to and from the
		local cores and six neighbouring SpiNNaker chips. SpiNNaker machines are
		constructed by combining many SpiNNaker chips.
		
		\begin{figure}
			\center
			\buildfig{figures/spinnaker-packet.tex}
			
			\caption{SpiNNaker's \SI{40}{\bit} and \SI{72}{\bit} multicast packet
			format.}
			\label{fig:spinnaker-packet}
		\end{figure}
		
		Processor cores can communicate by sending and receiving network packets
		forwarded by routers through the network. Since SpiNNaker's network is
		designed to transmit neural spike events efficiently, individual network
		packets are small, either \SI{40}{\bit} or \SI{72}{\bit} compared with tens
		or hundreds of byte packets in typical network architectures.
		
		In a real-time simulation, the time at which a spike is produced is
		implicitly indicated by the time it is received -- since at biological
		timescales a computer network delivers packets `instantaneously'.
		Consequently, the only information which must be explicitly encoded is the
		identity of the neuron which produced the spike. In SpiNNaker, a spike may
		be encoded by using a single \SI{40}{\bit} `multicast packet' whose format
		is illustrated in figure~\ref{fig:spinnaker-packet}.  The \SI{8}{\bit}
		header is used by SpiNNaker's routers to determine the type of packet and
		the \SI{32}{\bit} `routing key' is used to identify the neuron which
		produced the packet. The routing key is also used by SpiNNaker's routers to
		determine how the packet should be directed through the network.
		
		The optional \SI{32}{\bit} payload is not used by conventional spiking
		neural simulations \cite{galluppi10} but has been exploited to enable more
		efficient simulation of a particular class of neural models \cite{mundy15}.
	
	\section{The SpiNNaker router}
		
		The SpiNNaker router employs an unconventional design which, despite its
		compact size and small energy requirements, implements a flexible multicast
		routing scheme. Unlike conventional routers which often employ hard-coded
		routing rules \cite[chapter~8]{dally04}, the SpiNNaker router uses a
		programmable `routing table' to determine how packets should be forwarded.
		In addition, to avoid deadlocks, SpiNNaker's router employs a simple,
		timeout-based mechanism which exploits the ability of neural networks to
		tolerate occasional missing packets. As we will see in chapter
		\ref{sec:routing}, this mechanism greatly simplifies the task of routing in
		SpiNNaker's network. In this section we'll look at these features in
		greater detail.
		
		\subsection{Routing tables}
		
			When a multicast packet arrives at a SpiNNaker router (either from a
			local core or a neighbouring chip), the router looks up the routing key
			in its routing table. This table consists of \num{1024} programmable
			table entries, each specifying a routing key bit pattern and mask to
			match and a set of routes.  When a multicast packet's key is matched by a
			routing entry the packet is forwarded along every route specified by that
			entry, potentially duplicating the packet. This `multicast' technique
			allows packets to be transmitted once but received in a number of places
			while making efficient use of the network \cite{navaridas12}.
			
			Though routing table entries are in finite supply (\num{1024} entries per
			router), it is still possible for many thousands of traffic flows to be
			routed through a single router. The bit pattern and mask in each routing
			entry matches against the 32~bits of a routing key as either
			`\texttt{1}', `\texttt{0}' or `\texttt{X}' (don't care).  This means that
			a single routing entry may, for example, be used to match all routing
			keys with a certain prefix. If a routing key is not matched by any entry
			in the routing table then the packet is `default routed' in a straight
			line. For example if a packet with an unmatched key is received from the
			chip to the left, the packet will be default routed to the chip on the
			right. By assigning routing keys such that neurons whose spikes are sent
			to similar destinations share a similar prefix, the number of routing
			entries required by a simulation is greatly reduced \cite{davies12}.
			
			\begin{figure}
				\center
				\buildfig{figures/routing-example.tex}
				
				\caption[Multicast routing example.]%
				{Multicast routing example with \SI{4}{\bit} routing keys. Each
				box represents a SpiNNaker chip whose router has been programmed with
				the routing entries shown. Grey lines mark connections between chips.}
				\label{fig:routing-example}
			\end{figure}
			
			Consider the simplified example in figure~\ref{fig:routing-example} in
			which a number of (\SI{4}{\bit}) routing table entries have been
			configured in the routers of a small SpiNNaker network. If a packet with
			the routing key \texttt{1011} is transmitted by a core in the chip
			labelled $(0, 0, 0)$, this will match the first routing table entry on
			that chip and will be routed to chip $(1, 0, 0)$. On chip $(1, 0, 0)$,
			the packet once again matches the first routing entry and is routed to
			chip $(1, 0, -1)$. On $(1, 0, -1)$, no match is made so the packet is
			default routed to $(1, 0, -2)$. On this chip, the packet matches a
			routing entry which routes the packet to core~7. In this example, default
			routing allows only three routing table entries to direct a packet
			through four chips.
			
			As a second example, if a packet with the routing key \texttt{0010} is
			transmitted by a core on chip $(0, 0, 0)$, this key will be matched by
			the second routing entry since \texttt{X}s in the table entry will match
			both \texttt{1}s and \texttt{0}s in the corresponding bits of the routing
			key. When the packet arrives at chip $(0, 0, -1)$ the matching routing
			entry forwards the packet to both $(0, 1, -1)$ and $(1, 0, -1)$
			simultaneously. The copy of the packet arriving at $(0, 1, -1)$ is routed
			to core~5 on that chip.  Meanwhile, the copy forwarded to $(1, 0, -1)$ is
			duplicated again with one copy being routed to core~11 and another being
			routed to chip $(1, 0, -2)$. Here the packet is finally delivered to
			core~6. In this example, the ability of the router to multicast
			(duplicate) packets as they pass through the network meant that sending
			one copy of the packet was sufficient to reach three destination cores.
			In addition, by using \texttt{X}s in the routing table entry, the same
			routing entries are sufficient to route packets with the keys
			\texttt{0000}, \texttt{0001}, \texttt{0010} and \texttt{0011}.
			
			In spite of these mechanisms, it is still possible for an application to
			run out of routing table entries. As we will see in
			chapter~\ref{sec:placement} by arranging applications appropriately
			within SpiNNaker's network, routing table usage can be reduced. In
			addition, other behaviours of SpiNNaker's router may be exploited to
			compress an applications routing tables further, however the techniques
			employed are beyond the scope of this thesis \cite{mundy16}.
		
		\subsection{Timeouts}
			
			SpiNNaker's router is built on a pipeline architecture. As shown in
			figure~\ref{fig:router-architecture}, the router is fed packets by an
			arbiter which serialises packets arriving from other chips and local
			cores. Every (\SI{100}{\mega\hertz}) clock cycle, the router pipeline
			accepts one packet from the arbiter and routes a packet to one or several
			output links. If any of the required output ports are busy then the
			packet is not forwarded to any output link and the pipeline stalls. Once
			a packet has been blocked for a programmable timeout, it is dropped
			(discarded) and routing continues as usual for next packet in the
			pipeline. Links become blocked while transmitting packets or waiting for
			the remote receiver to become ready. For example, a receiving processor
			core may be busy performing some computation or a receiving router may be
			blocked waiting for some of its outputs to become ready.
			
			\begin{figure}
				\center
				\buildfig{figures/router-architecture.tex}
				
				\caption{SpiNNaker router architecture}
				\label{fig:router-architecture}
			\end{figure}
			
			The timeout-based packet dropping mechanism is designed to defuse
			deadlocks in the network. For example, if two routers are trying to send
			each other a packet at the same time they may become deadlocked, each
			waiting for the other router to accept a packet before continuing.
			SpiNNaker's timeout mechanism breaks deadlocks by dropping packets which
			have been blocked for some time and therefore may be in a deadlock.  Once
			a packet has been dropped it is left to software to either tolerate the
			missing packet or trigger a retransmission. In neural simulations, as in
			biology, the loss of a single spike is unlikely to have a significant
			impact on the behaviour of a neural model and therefore these simulations
			are inherently tolerant of occasional dropped packets. During application
			loading and other system tasks, a higher level, software driven protocol
			based on acknowledgements and retransmissions is used to ensure
			guaranteed delivery.
			
			% TODO: MENTION TIMEOUT VALUE USED?
			% Router timeouts must be configured to be long enough that delays in
			% packet transmission, for example due to the time taken for packets to
			% traverse a link, do not trigger packet dropping. Conversely, the timeout
			% should be as short as possible to reduce the time the router is
			% blocked and maximise network throughput.
	
	\section{The hexagonal torus topology}
		
		Each SpiNNaker chip is a node in a `hexagonal torus topology' as
		illustrated in figure~\ref{fig:hexagonalTorusTopology}. Network packets
		sent by SpiNNaker's processor cores may `hop' through several nodes in the
		network to reach their intended destination. In each hop, a packet may
		advance one node along one of the three axes of the topology. For example,
		a packet sent by the node labelled $\alpha$ (in the bottom-left corner) to
		the node labelled $\beta$, might take the following sequence of hops:
		X$^+$, X$^+$, Z$^-$. Packets sent from $\alpha$ to $\gamma$ might take the
		route: X$^-$, X$^-$, Y$^+$, Y$^+$. The first hop of this route `wraps
		around' from the bottom-left node to the bottom-right node in a single hop.
		
		\begin{figure}
			\center
			\buildfig{figures/hexagonalTorusTopology.tex}
			
			\caption[A hexagonal torus topology.]%
			{A hexagonal torus topology. Each hexagon represents a node (i.e.
			a SpiNNaker chip). Touching nodes are directly connected. Nodes on edges
			$a$, $b$ and $c$ are also directly connected to the corresponding nodes
			on edges $a'$, $b'$ and $c'$, respectively. The three axes of the
			hexagonal torus topology, `X', `Y' and `Z' are also shown.}
			\label{fig:hexagonalTorusTopology}
		\end{figure}
		
		\begin{figure}
			\center
			\begin{subfigure}{0.39\linewidth}
				\center
				\includegraphics[width=\linewidth]{figures/torus-3d-flat.pdf}
				\caption{}
				\label{fig:torus-3d-flat}
			\end{subfigure}
			~~
			\begin{subfigure}{0.26\linewidth}
				\center
				\includegraphics[width=\linewidth]{figures/torus-3d-tube.pdf}
				\caption{}
				\label{fig:torus-3d-tube}
			\end{subfigure}
			~~
			\begin{subfigure}{0.23\linewidth}
				\center
				\includegraphics[width=\linewidth]{figures/torus-3d-torus.pdf}
				\caption{}
				\label{fig:torus-3d-torus}
			\end{subfigure}
			
			\caption{Visualisation of a hexagonal torus topology as a torus.}
			\label{fig:torus-3d}
		\end{figure}
		
		The wrap around connections in the topology are what give it the `torus'
		part of its name. Figure~\ref{fig:torus-3d-flat} shows a hexagonal torus
		topology drawn flat as in the previous figure. If the topology is rolled up
		into a tube such that the top and bottom nodes become directly adjacent, a
		tube is formed as in figure~\ref{fig:torus-3d-tube}. This tube can then be
		bent to bring together the nodes at the ends of the tube to form a torus as
		shown in figure~\ref{fig:torus-3d-torus}.
		
		A hexagonal torus topology is typically defined in terms of its width and
		height along the X and Y axes respectively. For example,
		figure~\ref{fig:hexagonalTorusTopology} shows a $10\times10$ hexagonal
		torus.  The nodes in a hexagonal torus topology are addressed using
		hexagonal coordinates of the form $(x, y, z)$ \cite{patel15}. The bottom
		left node (labelled $\alpha$ in the figure) has the coordinate $(0, 0, 0)$
		and other nodes are assigned coordinates according to the number of hops
		along each dimension from $(0, 0, 0)$, for example node $\beta$ has the
		coordinate $(2, 0, -1)$. Because the hexagonal torus topology's axes are
		non-orthogonal, it is possible to define several coordinates for the same
		location. For example $(3, 1, 0)$ and $(1, -1, -2)$ are also valid
		coordinates for node $\beta$. These dual coordinates emerge from the fact
		that adding $(1, 1, 1)$ to a coordinate produces an equivalent, but
		different, coordinate. This phenomenon is explained in detail in
		appendix~\ref{app:minimal-hex-coordinates} and related phenomena will be
		discussed in chapter~\ref{sec:shortestPaths}.
		
		The hexagonal torus topology was chosen over a more conventional network
		topology -- such as a 2D or 3D torus (sometimes known as a 2-ary $N$-cube
		or 3-ary $N$-cube respectively) \cite[chapters~3~and~5]{dally04} -- due to
		its balance of theoretical performance and practicality. The bisection
		bandwidth of a topology indicates the theoretical worst-case total
		throughput the network is able to sustain \cite[chapter~1]{dally04}.  In
		networks with homogeneous link throughput, bisection bandwidth is
		determined by the number of links cut by a balanced bisection of the
		network.  Figure~\ref{fig:bisection-bandwidth} illustrates the bisections
		of several torus topologies.
		
		\begin{figure}
			\center
			\begin{subfigure}[b]{0.3\linewidth}
				\center
				\buildfig{figures/bisection-bandwidth-2d.tex}
				
				\caption{2D Torus}
				\label{fig:bisection-bandwidth-2d}
			\end{subfigure}
			\begin{subfigure}[b]{0.3\linewidth}
				\center
				\buildfig{figures/bisection-bandwidth-hex.tex}
				
				\caption{Hexagonal Torus}
				\label{fig:bisection-bandwidth-hex}
			\end{subfigure}
			\begin{subfigure}[b]{0.3\linewidth}
				\center
				\buildfig{figures/bisection-bandwidth-3d.tex}
				
				\caption{3D Torus}
				\label{fig:bisection-bandwidth-3d}
			\end{subfigure}
			
			\caption[Bisections of torus topologies.]%
			{Bisections of torus topologies. Connections cut by the bisection
			are drawn as lines.}
			\label{fig:bisection-bandwidth}
		\end{figure}
		
		In a $N \times N$ 2D torus topology, the bisection bandwidth is $2N$~links
		and each node requires four links. The hexagonal torus topology requires
		six links per node but provides double bisection bandwidth ($4N$~links).
		The 3D torus topology also requires six links per node but by connecting
		the nodes differently achieves a bisection bandwidth of $8N$~links.  The 3D
		torus topology, however, comes at a price -- when immersed into the
		(approximately) 2D space provided by a large machine room or row of server
		cabinets, some connections require long cables. By contrast, the 2D and
		hexagonal torus topologies are both inherently two dimensional and
		consequently do not suffer from this effect. The hexagonal torus topology,
		therefore, shares the practicality of construction of a 2D torus while
		still gaining some of the performance of a 3D torus topology. In addition,
		because nodes in a hexagonal torus topology have a greater number of links,
		greater redundancy is available in the network to tolerate faults.
		
		Most torus topologies, including hexagonal, 2D and 3D toruses, have a
		related `mesh' topology. These mesh topologies maintain the same general
		connectivity structure as their torus topologies but omit wrap-around
		links. In practice, this saves a small number of links at the expense of
		halving the network's bisection bandwidth.  Because of their poorer
		performance, mesh networks are rarely used as the basis of a network
		architecture. Mesh networks, however, are occasionally formed when a
		network is partitioned into several smaller sub-networks to allow multiple
		users to share a system \cite{spalloc16}.
		
		\begin{figure}
			\center
			\begin{subfigure}[b]{0.45\linewidth}
				\center
				\buildfig{figures/hexagonal-torus.tex}
				\caption{Hexagonal torus}
				\label{fig:topo-compare-hexagonal-torus}
			\end{subfigure}
			\begin{subfigure}[b]{0.45\linewidth}
				\center
				\buildfig{figures/h-torus.tex}
				\caption{H-torus}
				\label{fig:topo-compare-h-torus}
			\end{subfigure}
			
			\caption[Hexagonal torus vs. H-torus topology.]%
			{Hexagonal torus vs. H-torus topology. Each numbered hexagon
			represents a node. The thick outline indicates the bounds of the
			topology after which the network repeats. In each topology, the path
			taken by advancing in the Y$^+$ direction from the node labelled `0' is
			shown.}
			\label{fig:topo-compare}
		\end{figure}
		
		\label{sec:hex-vs-h-torus}
		
		The hexagonal torus topology is not to be confused with the `H-torus'
		topology. This topology also uses a hexagonal tiling of nodes and even
		wraps this tiling into a torus-like topology \cite{zhao08}. However,
		H-torus topologies have very different characteristics to the hexagonal
		torus topology and are related to `twisted torus' topologies
		\cite{camara10}. For example, figure~\ref{fig:topo-compare} illustrates one
		major difference in the way paths wrap around the peripheries of both
		topologies.
	
	\section{Scaling-up SpiNNaker machines}
		
		To build large SpiNNaker systems comprising of tens of thousands of
		SpiNNaker chips, groups of forty-eight chips are mounted onto circuit
		boards as illustrated in figure~\ref{fig:spinnakerBoard}. These boards may
		be connected together to form larger systems.  Figure~\ref{fig:threeboard}
		shows a prototype three board system. Though the chips are
		\emph{physically} arranged in a (nearly) $7\times7$ grid on each SpiNNaker
		board, they logically form a hexagonal `wrapped triple'
		\cite{davidsonWiring} (see appendix~\ref{sec:partitioning}) which logically
		fit together as illustrated in figure~\ref{fig:threeboard-separate}. The
		labelled exposed corners of the three forty-eight chip boards connect
		together to form a $12\times12$ hexagonal torus topology as illustrated in
		figure~\ref{fig:threeboard-wrapped}. Larger SpiNNaker machines are
		assembled by combining more boards.
		
		\begin{figure}
			\center
			\begin{subfigure}[b]{0.45\linewidth}
				\center
				\includegraphics[width=\linewidth]{figures/spinnakerBoard.jpg}
				
				\caption{A SpiNNaker board}
				\label{fig:spinnakerBoard}
			\end{subfigure}
			~~~
			\begin{subfigure}[b]{0.45\linewidth}
				\center
				\includegraphics[width=\linewidth]{figures/threeboard.jpg}
				
				\caption{Three board prototype}
				\label{fig:threeboard}
			\end{subfigure}
			
			\vspace*{1em}
			
			\begin{subfigure}[b]{0.45\linewidth}
				\center
				\buildfig{figures/threeboard-separate.tex}
				
				\caption{Three board topology}
				\label{fig:threeboard-separate}
			\end{subfigure}
			~~~
			\begin{subfigure}[b]{0.45\linewidth}
				\center
				\buildfig{figures/threeboard-wrapped.tex}
				
				\caption{\ldots{}as a parallelogram}
				\label{fig:threeboard-wrapped}
			\end{subfigure}
			
			\caption{SpiNNaker boards and their topology.}
			\label{fig:spinnaker-boards}
		\end{figure}
		
		
		SpiNNaker chips on the same circuit board connect using low power links
		requiring sixteen wires each.  If this link technology were used to connect
		chips on neighbouring boards, each pair of boards would need to be
		connected with a 128~wire cable.  Cables and connectors supporting this
		many signals are expensive, unreliable and physically large. Instead,
		chip-to-chip connections between boards are multiplexed and demultiplexed
		onto a single High-Speed Serial (HSS) link \cite{athavale05} carried via
		commodity S-ATA cables which are often used to connect hard disks in
		desktop computers and servers \cite{sata3spec}. The six high-speed links
		are implemented by three onboard FPGAs (the three large chips at the top of
		the SpiNNaker board) and are logically transparent to the underlying
		network. The underlying technology and the choice of S-ATA cables limits
		each board-to-board connection to spanning at most one metre gaps. In
		chapter~\ref{sec:building} I present a cabling scheme for hexagonal torus
		topologies which enable large SpiNNaker systems to be assembled using only
		short cables between boards.
		
	\section{Conclusions}
		
		The SpiNNaker architecture has been designed to enable the simulation of
		large biologically realistic neural models in real time. To support this,
		its network architecture takes on an unconventional design based on a
		custom router and hexagonal torus topology. In the remainder of this
		thesis, I will tackle a number of the challenges in scaling up the
		SpiNNaker architecture outlined in this chapter.

	\chapter{Building large SpiNNaker machines}
	
	Like any super computer, physically putting together a large SpiNNaker
	machine poses many challenges in terms of organisation, assembly and
	maintainance. One of the key tasks in this process is the installation of
	network cables such that a desired overall network topology is constructed.
	The largest planned SpiNNaker machine will use \num{3600} S-ATA
	\cite{sata3spec} cables to interconnect its \num{1200} circuit boards,
	creating a hexagonal torus topology. Since the machine will be installed
	within standard server room cabinets (which are not available in a
	giant-doughnut form-factor) a mapping from a board's logical location in the
	network topology to its physical location must be constructed. In addition,
	the interconnect technology employed by SpiNNaker restricts the length of
	S-ATA cables used to $\le$~\SI{1}{\meter}, constraining the possible mappings
	used. In addition the practical issues of installation complexity and
	maintainance must be considered since all \num{3600} cables must ultimately
	be installed and maintained by human operators.
	
	In this chapter I describe a novel technique for physically laying out
	machines configured in hexagonal torus topologies, such as SpiNNaker, in
	commercial machine rooms, building on the techniques used in more
	conventional torus topologies. In addition, I also propose a new methodology
	for installing and maintaining super computer cabling which which exploits
	existing diagnostic features of the SpiNNaker hardware to interactively guide
	and validate cable installation. Finally, I demonstrate how these new
	techniques have been used successfully to interconnect a prototype
	\num{518400} core SpiNNaker machine in substantially less time than the
	industry norm.
	
	In this chapter, the term \emph{unit} refers to the smallest physical
	ecomponent between which connections connections are to be made. For example,
	in a SpiNNaker machine a unit is a 48-chip board while in data center, a unit
	might be a server blade.
	
	\section{Related work}
		
		In this section I describe the techniques conventionally employed when
		laying out and interconnecting the units within super computers. Due to
		SpiNNaker's hexagonal torus topology and dense physical packing of units,
		these existing techniques are found to be insufficient. In the remainder of
		the chapter we will explore solutions to the limitations exposed below.
		
		\subsection{Avoiding long cables}
			
			Na\"ive arrangements of torus topologies, including hexagonal torus
			topologies, feature long `wrap-around' connections which connect units at
			the peripheries of the system. These connections can be problematic for
			numerous reasons:
			
			\begin{description}
				
				\item[Performance] Signal quality diminishes as cables get longer,
				requiring the use of slower signalling speeds, increased error
				correction overhead or more complex hardware.
				
				\item[Energy] Longer cables require higher drive strengths and/or
				buffering to maintain signal integrity.
				
				\item[Cost] Cost Shorter cables are cheaper than long ones.  Longer
				cables imply more wire in a given space making the tasks of routing or
				cable installation more difficult increasing labour costs by as much as
				$5\times$ \cite{curtis12}.
				
			\end{description}
			
			In conventional torus topologies the need for long cables is eliminated
			by folding and interleaving units of the network \cite{dally04}. For
			example, for a 1D torus topology (a ring network), one long connection
			exists to connect the two opposite sides of the system. To remove these
			long connections, half the units are `folded' on top of the others and
			then this arrangement of units is interleaved as illustrated in figure
			\ref{fig:ring-folding}.
			
			\begin{figure}
				\center
				\begin{subfigure}[b]{0.39\linewidth}
					\center
					\buildfig{figures/ring-folding-row.tex}
					\caption{A ring network}
					\label{fig:ring-folding-row}
				\end{subfigure}
				\begin{subfigure}[b]{0.24\linewidth}
					\center
					\buildfig{figures/ring-folding-folded.tex}
					\caption{Folded}
					\label{fig:ring-folding-folded}
				\end{subfigure}
				\begin{subfigure}[b]{0.35\linewidth}
					\center
					\buildfig{figures/ring-folding-interleaved.tex}
					\caption{Folded and interleaved}
					\label{fig:ring-folding-interleaved}
				\end{subfigure}
				
				\caption{Folding and interleaving a ring network to reduce maximum wire
				length.}
				\label{fig:ring-folding}
			\end{figure}
			
			Folding and interleaving has the effect of approximately doubling the
			average cable length but also eliminates the need for a cable spanning
			the entire system. Since the mean cable length is typically already
			short, doubling it in exchange for a substantially reduced maximum cable
			length is often preferable.
			
			The folding and interleaving process may be extended to $N$-dimensional
			torus topologies by folding each dimension in turn. Since all dimensions
			are orthogonal, the folding process only moves units in the dimension
			being folded. In the hexagonal torus topology, however, the three
			dimensions are non-orthogonal and thus folding in one dimension also
			moves units in other dimensions, preventing the edges of the torus
			meeting as illustrated in figure \ref{fig:failing-to-fold-hex-toruses}.
			
			\begin{figure}
				\center
				\begin{subfigure}[b]{0.24\linewidth}
					\center
					\buildfig{figures/failing-to-fold-hex-toruses-none.tex}
					\caption{Not folded}
					\label{fig:failing-to-fold-hex-toruses-none}
				\end{subfigure}
				\begin{subfigure}[b]{0.24\linewidth}
					\center
					\buildfig{figures/failing-to-fold-hex-toruses-x.tex}
					\caption{X}
					\label{fig:failing-to-fold-hex-toruses-x}
				\end{subfigure}
				\begin{subfigure}[b]{0.24\linewidth}
					\center
					\buildfig{figures/failing-to-fold-hex-toruses-y.tex}
					\caption{Y}
					\label{fig:failing-to-fold-hex-toruses-y}
				\end{subfigure}
				\begin{subfigure}[b]{0.24\linewidth}
					\center
					\buildfig{figures/failing-to-fold-hex-toruses-z.tex}
					\caption{Z}
					\label{fig:failing-to-fold-hex-toruses-z}
				\end{subfigure}
				
				\caption{Schematics showing hexagonal torus topologies folded along
				each of their non-orthogonal dimensions. Note that folding along
				the Z axis brings the \emph{wrong} edges closer together.}
				\label{fig:failing-to-fold-hex-toruses}
			\end{figure}
		
		\subsection{Cabling installation}
			
			Existing machine room installations feature very repetitive cabling
			patterns which can easily be memorised by a human technician. For example
			in BlueGene super computers the connectivity between units is highly
			regular \cite{lakner07} while in data centre networks cabling often
			centres around a small number of high-port-count switches
			\cite{cisco07,csernai15}. Cable installation is usually only aided by
			the labelling of connectors and sockets in a standardised manner
			\cite{tia2006} such as in figure \ref{fig:bgWiring}.
			
			\begin{figure}
				\center
				\begin{subfigure}[t]{0.5\textwidth}
					\begin{tikzpicture}
						\node (cables) [inner sep=0]
						      {\includegraphics[width=\textwidth]{figures/bgCables.png}};
						\node (sockets) [inner sep=0, below=1.0em of cables]
						      {\includegraphics[width=\textwidth]{figures/bgSockets.png}};
						
						% Point at label on cable
						\draw [white, <-, line width=0.4em]
						      ([shift={(0.7cm, -0.3cm)}]cables.center)
						      -- ++(45:1cm);
						
						% Point at label on socket
						\draw [white, <-, line width=0.4em]
						      ([shift={(-1.0cm, 1.1cm)}]sockets.center)
						      -- ++(-45:1cm);
					\end{tikzpicture}
					
					\caption{Pre-labelled cables and sockets}
					\label{fig:bgWiringLabels}
				\end{subfigure}
				~
				\begin{subfigure}[t]{0.30\textwidth}
					\includegraphics[height=6.15cm]{figures/bgWiring.jpg}
					
					\caption{Installation of cables}
					\label{fig:bgWiringInstallation}
				\end{subfigure}
				
				\caption{BlueGene/Q cable installation \cite{cscs13}}
				\label{fig:bgWiring}
			\end{figure}
			
			Despite the regularity and careful labelling of cables, the cost of
			installation and maintenance alone can be significant with costs in the
			range of \$45-95 per \SI{1}{\meter} cable run and \$100-400 for runs of
			\SI{10}{\meter} reported in the literature \cite{mudigonda11}. Much of
			this cost is due to the care required during installation to avoid
			miswiring and ensure that cooling airflow is not hampered by cable runs
			\cite{cisco07}.
			
			Many researchers have attempted to control cable installation costs by
			trying to reduce the number or length of cables required by developing
			alternative network topologies \cite{curtis12, popa10, mudigonda11}.
			Unfortunately, these techniques do not apply to SpiNNaker since its
			network topology is fixed.
			
			Some super computers make use of large custom `midplane` PCBs in place of
			cables to interconnect units within a cabinet and thus simplify the task
			of cable installation \cite{prickett10}. This scheme can greatly reduce
			wiring complexity since only coarser-grain cabinet-to-cabinet
			connectivity is provided by cables. Unfortunately this technique is
			expensive and also constrains the dimensions of the network topology
			supported by the machine. Since the SpiNNaker platform is designed to
			scale from desktop machines to machine-room installations, this scheme is
			not practical.
	
	\section{Folding \& interleaving hexagonal toruses}
		
		The first step towards a practical machine-room installation of a large
		machine using a hexagonal torus topology is to find an arrangement of
		boards between which cable lengths are minimised. In this section I
		describe a sequence of transformations which map the positions of units in
		a hexagonal torus topology onto a regular rectangular grid which may be
		folded and interleaved to eliminate long wires. It is worth emphasising
		that this transformation only affects the \emph{physical} positions of
		units and \emph{not} their connectivity.
		
		As described earlier in \S\ref{sec:parititioning} (page
		\pageref{sec:parititioning}), hexagonal torus topologies may be partitioned
		into units containing wrapped-triples of nodes. For example, in SpiNNaker,
		chips (nodes) are partitioned into circuit boards (units) containing 48
		chips. For completeness, this section describes the process of folding both
		systems whose units are individual nodes and those whose units are
		wrapped-triples.
		
		The transformation process is divided into two parts, each described
		separately in this section.
		
		\begin{description}
			
			\item[Parallelogram to rectangle] Units of the system are transformed
			from a parallelogram shape to a rectangular shape.
			
			\item[Uncrinkle] Units within the rectangle are moved such that they all
			lie on a regular (and fully packed) 2D grid.
			
		\end{description}
		
		\subsection{Parallelogram to rectangle}
			
			The hexagonal torus topology is most naturally drawn as a parallelogram
			as illustrated in figures \ref{fig:hex-to-plane-node-native} and
			\ref{fig:hex-to-plane-native}. Two transformations are presented which
			transform these arangements of units into a rectangular form: shearing
			and slicing.
			
			A \SI{30}{\degree} shear transformation distorts networks such that the X
			and Y axes become orthogonal leading to a rectangular arrangement of
			units as illustrated in figures \ref{fig:hex-to-plane-node-shear} and
			\ref{fig:hex-to-plane-shear}.
			
			The slice transformation slices the units protruding from the
			left-hand-side of the parallelogram and moves them into the matching gap
			on the opposite side of the parallelogram as illustrated in figures
			\ref{fig:hex-to-plane-node-slice} and \ref{fig:hex-to-plane-slice}.
			 
			While the shear transformation introduces some distortion causing cables
			in the Z dimension to become $\sqrt{2}\times$ longer it leaves the
			pattern of wrap-around connections remains unchanged. By contrast, the
			slice transformation does not elongate any cables but changes the pattern
			of wrap-around connections. The exact pattern wrap-around connections
			produced when slicing depends on the aspect ratio of the network as
			illustrated in \ref{fig:slicing-examples} and influences the choice of
			folding technique applied as described later.
			
			\begin{figure}
				\center
				\begin{subfigure}[b]{0.32\linewidth}
					\center
					\buildfig{figures/hex-to-plane-node-native.tex}
					
					\caption{$7 \times 7$ node torus}
					\label{fig:hex-to-plane-node-native}
				\end{subfigure}
				\begin{subfigure}[b]{0.32\linewidth}
					\center
					\buildfig{figures/hex-to-plane-node-shear.tex}
					
					\caption{Sheared}
					\label{fig:hex-to-plane-node-shear}
				\end{subfigure}
				\begin{subfigure}[b]{0.32\linewidth}
					\center
					\buildfig{figures/hex-to-plane-node-slice.tex}
					
					\caption{Sliced}
					\label{fig:hex-to-plane-node-slice}
				\end{subfigure}
				
				\caption{Transformations of hexagonal toruses of nodes into a
				rectangular form. Thin lines show wrap-around links. Pointy-topped
				hexagons represent individual nodes.}
				\label{fig:hex-to-plane-node}
			\end{figure}
			
			\begin{figure}
				
				\begin{subfigure}[b]{0.32\linewidth}
					\center
					\buildfig{figures/hex-to-plane-native.tex}
					
					\caption{$4 \times 4$ triad torus}
					\label{fig:hex-to-plane-native}
				\end{subfigure}
				\begin{subfigure}[b]{0.32\linewidth}
					\center
					\buildfig{figures/hex-to-plane-shear.tex}
					
					\caption{Sheared}
					\label{fig:hex-to-plane-shear}
				\end{subfigure}
				\begin{subfigure}[b]{0.32\linewidth}
					\center
					\buildfig{figures/hex-to-plane-slice.tex}
					
					\caption{Sliced}
					\label{fig:hex-to-plane-slice}
				\end{subfigure}
				
				\caption{Transformations of hexagonal toruses of wrapped triples into a
				rectangular form.  Thin lines show wrap-around links. Flat-topped
				hexagons represent a wrapped triple of nodes.}
				\label{fig:hex-to-plane}
			\end{figure}
			
			\begin{figure}
				\center
				\buildfig{figures/slicing-examples.tex}
				\caption{Patterns of wiring in sliced systems of various sizes.}
				\label{fig:slicing-examples}
			\end{figure}
			
		\subsection{Uncrinkling}
			
			Though the transformmation step yields rectangular arrangements of units,
			these arrangements do not fall onto a regular 2D grid, with the exception
			of the shear transform on individual nodes. Figure \ref{fig:uncrinkling}
			illustrates how the various arrangements of hexagons may be moved to
			`uncrinkle' the units into a regular grid.
			
			\begin{figure}
				\center
				\begin{subfigure}[b]{0.44\linewidth}
					\center
					\buildfig{figures/uncrinkling-node-sheared.tex}
					
					\caption{$7 \times 7$ nodes, sheared}
					\label{fig:uncrinkling-node-sheared}
				\end{subfigure}
				\begin{subfigure}[b]{0.44\linewidth}
					\center
					\buildfig{figures/uncrinkling-node-sliced.tex}
					
					\caption{$7 \times 7$ nodes, sliced}
					\label{fig:uncrinkling-node-sliced}
				\end{subfigure}
				
				\vspace{1cm}
				
				\begin{subfigure}[b]{0.44\linewidth}
					\center
					\buildfig{figures/uncrinkling-sheared.tex}
					
					\caption{$4 \times 4$ triples, sheared}
					\label{fig:uncrinkling-sheared}
				\end{subfigure}
				\begin{subfigure}[b]{0.44\linewidth}
					\center
					\buildfig{figures/uncrinkling-sliced.tex}
					
					\caption{$4 \times 4$ triples, sliced}
					\label{fig:uncrinkling-sliced}
				\end{subfigure}
				
				\vspace{1em}
				
				\caption{Mapping rectangular arrangements of units into a square grid.
				Thick lines show how layers of units are uncrinkled.  Annotations show
				how the relative positions of nodes and wrapped triples change after
				uncrinkling.}
				\label{fig:uncrinkling}
			\end{figure}
			
			In the figure, the numbered units enumerate the different positions on
			the crinkle and those labelled alphabetically are those that immediately
			surround them. From this we can observe that uncrinkling largely
			preserves spatial locality but some distortion is introduced, separating
			previously neighbouring units. For example, in figure
			\ref{fig:uncrinkling-sheared}, the units labelled `1' and `i' are
			neighbours before uncrinkling but are separated by a (Euclidean) distance
			of $\sqrt{5}$ afterwards. Note that the distortion introduced depends on
			what part of the crinkle is considered, for example `2' and `a' have
			distance 2 but are logically connected in the same way.
		
		\subsection{Folding and Interleaving}
			
			Once a regular grid of units has been formed, this may be folded in the
			conventional way, eliminating long cables crossing from left-to-right and
			top-to-bottom as illustrated in \ref{fig:folding-sheared}.
			
			Unfortunately, for sliced systems whose dimensions are not of the ratio
			$1:2$, the pattern of wrap-around cables may also include some cables
			which do not cross directly to the opposite side of the system (refer
			back to figure \ref{fig:slicing-examples}). As a result of these
			connections, folding does not successfully eliminate all long
			connections. An exception to this rule is sliced systems whose dimensions
			are in the ratio $1:1$ where folding twice along the Y axis may
			successfully eliminate all wrap-around connections as illustrated in
			\ref{fig:folding-sliced}.
			
			\begin{figure}
				\begin{subfigure}{\linewidth}
					\center
					\buildfig{figures/folding-sheared.tex}
					\caption{$N \times M$ sheared systems and $N \times 2N$ sliced systems}
					\label{fig:folding-sheared}
				\end{subfigure}
				
				\vspace{1em}
				
				\begin{subfigure}{\linewidth}
					\center
					\buildfig{figures/folding-sliced.tex}
					\caption{$N \times N$ sliced systems}
					\label{fig:folding-sliced}
				\end{subfigure}
				
				\caption{Schematic illustrating elimination of long wrap-around links
				during folding. In each example a single link has been highlighted to
				aid in following the process.}
				\label{fig:folding}
			\end{figure}
			
			Once folded, the 2D grid is straight-forwardly interleaved as illustrated
			previously in figure \ref{fig:ring-folding}. The interleaving process
			introduces some additional distortion to the layout of units and causes
			most connections to become twice as long. For sliced $1:1$ systems, the
			additional fold results in additional overhead during interleaving since
			four layers of the system are interleaved.
		
		\subsection{Mapping to Cabinets}
			
			In the final step of the process is to map the 2D grid of units into
			positions in machine room cabinets as illustrated in figure
			\ref{fig:million-core-machine}. As illustrated in figure
			\ref{fig:cabinetisation}, first the grid of units is partitioned into
			groups of columns, one per cabinet, then groups of rows one per frame per
			cabinet. The units in each group are then allocated to slots within a
			frame, interleaving the rows of the groups as shown.
			
			\begin{figure}
				\center
				\buildfig{figures/cabinet-units.tex}
				
				\caption{An illustration of the physical construction of a
				multi-cabinet SpiNNaker system. (Note: network cables \emph{not}
				installed.)}
				\label{fig:cabinet-units}
			\end{figure}
			
			\begin{figure}
				\center
				\buildfig{figures/cabinetisation.tex}
				
				\caption{Mapping from 2D space to cabinets, frames and boards.}
				\label{fig:cabinetisation}
			\end{figure}
		
	\section{Cable installation}
		
		Cable installation is performed by a team of (human) technicians who must
		ensure that all network cables are correctly installed. As illustrated in
		previously in figure \ref{fig:cabinet-units}, the density of SpiNNaker's
		units, combined with the nature of the hexagonal torus topology, poses a
		challenge. To address this challenge I propose a semi-automated approach to
		cable installation which exploits diagnostic facilities available in the
		majority of super computers in order to guide technicians through the
		cabling process, interactively guiding installation and maintenance.
		
		\subsection{Interactive technician guidance and validation}
			
			While automated systems for validating cabling correctness are
			commonplace, these systems are typically used only after cabling has been
			completed \cite{lakner07}. As with other large-scale machines, SpiNNaker
			includes a low-bandwidth system management bus which may be used to
			interrogate network hardware and control diagnostic LEDs prior to the
			installation of the main SpiNNaker network interconnect.  Using these
			facilities I have constructed a tool called SpiNNer which interactively
			guides a technician, or team of technicians, through the cable
			installation process, validating each connection in real-time.
			
			Diagnostic LEDs mounted on each SpiNNaker board (figure
			\ref{fig:interactive-wiring-guide-leds}) are used to indicate the
			endpoints of the cable currently being installed. Simultaneously a
			Text-To-Speech (TTS) system gives an audible indication of which cable
			type is to be used and location of each connection.  Additionally, a GUI
			via a computer display (figure \ref{fig:interactive-wiring-guide-gui}).
			The centre of the display shows a `big-picture' perspective of the
			locations of the boards to be connected. The detailed views on the left
			and right indicate which of the six sockets on each board the cables
			should connect.
			
			\begin{figure}
				\center
				\begin{subfigure}[b]{0.40\textwidth}
					\begin{tikzpicture}
						\node (leds) [inner sep=0]
						      {\includegraphics[width=\textwidth]{figures/leds.jpg}};
						% Point at left LED
						\draw [white, <-, line width=0.4em]
						      ([shift={(-0.0cm, -0.6cm)}]leds.center)
						      -- ++(225:1cm);
						% Point at right LED
						\draw [white, <-, line width=0.4em]
						      ([shift={(1.1cm, -1.1cm)}]leds.center)
						      -- ++(225:1cm);
					\end{tikzpicture}
					
					\caption{Diagnostic LEDs}
					\label{fig:interactive-wiring-guide-leds}
				\end{subfigure}
				~
				\begin{subfigure}[b]{0.546\textwidth}
					\begin{tikzpicture}[thin, black!20!white]
						\node (screen) [inner sep=0]
						      {\includegraphics[width=\textwidth]{figures/wiring_guide_screenshot.png}};
						\draw (screen.south west) rectangle (screen.north east);
					\end{tikzpicture}
					
					\caption{Interactive wiring guide GUI}
					\label{fig:interactive-wiring-guide-gui}
				\end{subfigure}
				
				\caption{The SpiNNer interactive wiring guide uses a GUI,
				text-to-speech and diagnostic LEDs to assist during cable
				installation.}
				\label{fig:interactive-wiring-guide}
			\end{figure}
			
			SpiNNer also validates the connectivity of the system in real-time by
			polling the diagnostic interfaces of the network hardware at the
			endpoints of the cable being installed to determine if they are correctly
			connected. If a miswiring occurs, this is immediately detected and
			announced via TTS enabling the technician to immediately correct the
			error. Once a cable has been installed correctly, the software
			automatically advances to the next cable meaning direct interaction with
			the software by the technician is minimal. In practice, it is rarely
			necessary to refer to the GUI.
		
			SpiNNer presents the cables in an order intended to maximise ease of
			installation. Cables are installed in three groups with intra-frame
			cables being installed first, followed by intra-cabinet cables and
			inter-cabinet cables. Within each group, the tightest cables are
			installed first resulting in slacker cables naturally being installed
			over the top of already installed cables. By grouping cables in this
			manner, multiple technicians may work independently on the wiring within
			individual frames and cabinets.
			
			SpiNNer may also be used to repair or replace cables in the system.
			During maintenance, obstructing cables may be blindly removed alongside
			any cable being replaced. At the conclusion of the process, the wiring
			guide may be used to interactively guide re-installation of all removed
			cables.
		
		\subsection{Cable selection}
			
			Controlling slack is critical to ensuring reliable and maintainable
			cabling installations. If cables are too tight, cables and connectors can
			become easily damaged and when too slack, the excess cable obstructs
			other cables and can easily become tangled and damaged \cite{cisco07}. It
			has been observed that when ready-made cables are employed technicians
			frequently over-estimate the cable lengths required preferring to use
			overly long cables for all connections \cite{mazaris97}. To solve this
			problem, the SpiNNer wiring guide software dictates the cable lengths to
			be used by an installer based the rule of (three-)thumbs according to
			Mazaris \cite{mazaris97}. This rule suggests that an ideal amount of
			slack is approximately that which can be wrapped around three fingers.
			Specifically, the shortest available cable is selected which ensures at
			least \SI{5}{\centi\meter} of slack.
			
			The SpiNNer tool allocates cables assuming all cables take a Euclidean
			straight-line path between the endpoints of the connection. The result is
			that wiring is not routed through dedicated cable management structures
			but is simply suspended by its connectors in front of the machine. As
			demonstrated later, this unconventional approach leads neither to cooling
			problems nor increased maintenance effort.
	
	\section{Results and Evaluation}
		
		This stuff has been used and works. In this section I'll go over the
		overheads introduced by the various transformations and
		folding/interleaving steps and show a wiring scheme for a large machine
		which uses only short cables. I'll then show how SpiNNer was used to
		install this wiring plan into a very large machine without foobaring the
		cooling and in very little time. I'll also report on difficulty of
		maintenance.
		
		\subsection{Cable length}
			
			The transformation from regular hexagonal torus to a folded and
			interleaved form introduces some overhead to the cable lengths required.
			Using figure \ref{fig:uncrinkling} (page \pageref{fig:uncrinkling}), it
			is possible to compute the exact overhead introduced when each type of
			transformation proposed.
			
			For example, to compute the mean overhead introduced by the slicing
			technique when applied to units composed of wrapped triples, consider
			figure \ref{fig:uncrinkling-sliced}. The uncrinkling pattern used to
			transform this topology is a repeating pattern of two units, a pair of
			which have been labelled $1$ and $2$ respectively. Unit $1$ is
			immediately surrounded by six units labelled $a$, $b$, $c$, $2$, $g$ and
			$h$. Similarly, unit $2$ is surrounded by units $1$, $c$, $d$, $e$, $f$
			and $g$. Before the transformation, the distances, $D$, to each of these
			units is $1$ but after the transformation is applied, this is not always
			the case. Additionally, folding and interleaving introduce additional
			overhead. In this example, if the system is folded into $f_x$ columns and
			$f_y$ rows, the distances between previously neighbouring units become:
			
			\begin{equation*}
				\begin{aligned}[c]
					D_{1\,\leftrightarrow{}\,a} &= \sqrt{f_x^2 + f_y^2} \\
					D_{1\,\leftrightarrow{}\,b} &= f_y \\
					D_{1\,\leftrightarrow{}\,c} &= \sqrt{f_x^2 + f_y^2} \\
					D_{1\,\leftrightarrow{}\,2} &= f_x \\
					D_{1\,\leftrightarrow{}\,g} &= f_y \\
					D_{1\,\leftrightarrow{}\,h} &= f_x
				\end{aligned}
				\hspace{2cm}
				\begin{aligned}[c]
					D_{2\,\leftrightarrow{}\,1} &= f_x \\
					D_{2\,\leftrightarrow{}\,c} &= f_y \\
					D_{2\,\leftrightarrow{}\,d} &= f_x \\
					D_{2\,\leftrightarrow{}\,e} &= \sqrt{f_x^2 + f_y^2} \\
					D_{2\,\leftrightarrow{}\,f} &= f_y \\
					D_{2\,\leftrightarrow{}\,g} &= \sqrt{f_x^2 + f_y^2}
				\end{aligned}
			\end{equation*}
			
			From these values, the mean and maximum connection distances after
			folding and interleaving may be computed. Table
			\ref{tab:transform-overhead} gives the mean and maximum connection
			distances for each of the four transformations described in this chapter.
			
			\begin{table}
				\begin{subtable}[b]{\linewidth}
					\center
					\begin{tabular}{l c c}
						\toprule
						& Shear & Slice \\
						\addlinespace
						Nodes &
							$\frac{f_x + f_y + \sqrt{f_x^2 + f_y^2}}{3}$ &
							$\frac{f_x + f_y + \sqrt{f_x^2 + f_y^2}}{3}$ \\
						\addlinespace
						Triples &
							$\frac{7f_x + 3\sqrt{f_x^2 + f_y^2} + \sqrt{(2f_x)^2 + f_y^2}}{9}$ &
							$\frac{f_x + f_y + \sqrt{f_x^2 + f_y^2}}{3}$ \\
						\bottomrule
					\end{tabular}
					
					\caption{Mean}
					\label{tab:transform-overhead-mean}
				\end{subtable}
				
				\vspace{1em}
				
				\begin{subtable}[b]{\linewidth}
					\center
					\begin{tabular}{l c c}
						\toprule
						& Shear & Slice \\
						\addlinespace
						Nodes &
							$\sqrt{f_x^2 + f_y^2}$ &
							$\sqrt{f_x^2 + f_y^2}$ \\
						\addlinespace
						Triples &
							$\sqrt{(2f_x)^2 + f_y^2}$ &
							$\sqrt{f_x^2 + f_y^2}$ \\
						\bottomrule
					\end{tabular}
					
					\caption{Maximum}
					\label{tab:transform-overhead-max}
				\end{subtable}
				
				\caption{Overheads introduced when transforming unit positions onto a
				regular grid.}
				\label{tab:transform-overhead}
			\end{table}
			
			From these results it is evident that shearing and slicing networks
			whose units are nodes result in identical mean and maximum overhead in
			cable length when folded similarly. Since sliced networks may require
			folding more than once along each axis the shearing approach is
			preferable in general.
			
			For networks constructed from units of wrapped triples, the slicing
			approach suffers the same mean and maximum overhead has networks of
			nodes, and less overhead than shearing for the same number of folds. For
			systems with an aspect ratio of $1:2$ (where both slicing and shearing
			require $f_x = f_y = 2$), the slicing transformation yields lower mean
			and maximum overhead than shearing. For all other aspect ratios (where
			slicing requires a greater degree of folding) the shearing technique
			produces lower overhead. The recommended transformations for a given
			machine are thus given in table \ref{tab:transform-recommended}.
			
			\begin{table}
				\center
				\begin{tabular}{lcc}
					\toprule
					                         & $1:2$  & Other \\
					\addlinespace
					\multirow{2}{*}{Nodes}   & Either & Shear\\
					                         & \footnotesize $\mu\approx2.28 \quad \vee\approx2.83$
					                         & \footnotesize $\mu\approx2.28 \quad \vee\approx2.83$\\
					\addlinespace
					\multirow{2}{*}{Triples} & Slice  & Shear\\
					                         & \footnotesize $\mu\approx2.28 \quad \vee\approx2.83$
					                         & \footnotesize $\mu\approx3.00 \quad \vee\approx4.47$\\
					\bottomrule
				\end{tabular}
				
				\caption{Recommended transformation and folding scheme for different
				system types. $\mu$ and $\vee$ give the mean and maximum wire
				distortion introduced, respectively.}
				\label{tab:transform-recommended}
			\end{table}
			
			\begin{figure}
				\center
				\buildfig{figures/million-core-machine.tex}
				
				\caption{Cabling plan for a \num{1036800} core SpiNNaker
				machine's \num{3600} cables.}
				\label{fig:million-core-machine}
			\end{figure}
			
			Following folding and mapping to physical locations, the cabling plans
			for large machines require no large gaps to be spanned.  The largest
			planned SpiNNaker machine, illustrated in figure
			\ref{fig:million-core-machine}, will be \SI{6}{\meter} wide but the
			largest gap any cable must span is \SI{66}{\centi\meter}. This distance
			is well within the \SI{1}{\meter} allowed by the hardware and cables.
			
		\subsection{Installation practicality}
			
			\begin{table}
				\center
				\begin{tabular}{lrr@{$\,$}l}
					\toprule
						System & Number of Cables & \multicolumn{2}{r}{Installation time} \\
					\midrule
						24 boards  & \num{72}   & \num{10} & \si{\minute}         \\
						1 cabinet  & \num{360}  & \num{4}  & \si{\hour}$^\dagger$ \\
						2 cabinets & \num{720}  & \num{2}  & \si{\hour}           \\
						5 cabinets & \num{1800} & ?        &                      \\
					\bottomrule
				\end{tabular}
				
				\caption{Installation times for various sizes of machine.
				$\dagger$~This machine was installed without real-time validation of
				connectivity.}
				\label{tab:install-time}
			\end{table}
			
			A number of SpiNNaker machines of various scales have been assembled
			using the techniques described in this chapter ranging from single frames
			of 24 boards to a half-scale 5 cabinet machine. Table
			\ref{tab:install-time} gives the reported installation times of each of
			these machines.
			
			The single cabinet machine's installation time is notably
			disproportionate to its size. When this system was assembled, real-time
			connection validation was not yet available. As a result, though cable
			installation was rapid correcting errors was extremely costly, requiring
			careful retracing of many installation steps.
			
			TODO: TALK ABOUT MULTI-PERSON-WIRING IN PRACTICE ON FIVE CABINET MACHINE.
			
			\begin{figure}
				
				\center
				\buildfig{figures/wire-length-histogram.tex}
				
				\caption{Histogram of connection distances in a ten-cabinet,
				one-million core SpiNNaker machine annotated with the suggested cable
				length.}
				\label{fig:wire-length-histogram}
				
			\end{figure}
			
			FIGURE \ref{fig:wire-length-histogram} SHOWS THE DISTRIBUTION OF CABLE
			LENGTHS REQUIRED. IN PRACTICE THE SLACK ALLOCATED PROVED ADEQUATE. AS
			SHOWN IN FIGURE \ref{fig:install-histogram}, THE MOST IMPORTANT FACTOR IS
			WHETHER LEAVING THE FRAME OR NOT. LEAVING THE FRAME TAKES THE LONGEST.
			
			\begin{figure}
				\builddata{data/build_connection_log.tex}
				\buildfig{figures/install-histogram.tex}
				
				\caption{Histogram of cable installation times}
				\label{fig:install-histogram}
			\end{figure}
			
			TODO: COMPARE DIRECTLY WITH INSTALL TIMES REPORTED IN LITERATURE.
		
		\subsection{Thermal Impact}
			
			TODO: SHOW HOW TEMPERATURE IS CHANGED
			
		\subsection{Maintenance}
			
			TOOD: QUANTIFY CABLE REMOVALS REQUIRED. EXPERIMENT: REMOVE/REPLACE RANDOM
			BOARDS AND MEASURE TIME TAKEN, CABLES REMOVED. COMPARE WITH STANDARD DATA
			CENTRE WIRING

	\chapter{Finding shortest path vectors in SpiNNaker's network}
	
	Once a SpiNNaker machine has been constructed as described in the previous
	chapter, its network forms a large hexagonal torus topology. To exploit this
	network routing algorithms must be used to generate routes for packets to
	follow between nodes. As well as ensuring that packets arrive at the correct
	destination, routing algorithms often attempt to produce routes which make
	efficient use of the network. This often involves attempting to reduce
	congestion by ensuring packets do not travel further through the network than
	absolutely necessary.
	
	Many popular routing algorithms for torus topologies, including all published
	algorithms designed for SpiNNaker's hexagonal torus topology
	\cite{davies12,navaridas14}, internally function by computing shortest path
	vectors and generating routes from them. Existing methods of calculating
	shortest path vectors in hexagonal torus topologies are unable to generate
	all possible shortest path vectors and, as a result, reduces the diversity of
	routes produced by routing algorithms, potentially worsening network
	contention.
	
	In this chapter I describe a novel technique for computing shortest path
	vectors in hexagonal torus topologies which yields \emph{all} possible
	shortest path vectors for any pair of nodes. Further, implementations of this
	new technique execute an order of magnitude faster than the existing
	alternatives.
	
	\section{Related work}
		
		TODO: INTRODUCE SECTION
		
		\begin{figure}
			\center
			
			\begin{subfigure}{\linewidth}
				\center
				\buildfig{figures/distance-map-mesh.tex}
				\caption{2D mesh topology}
				\label{fig:distance-map-mesh}
			\end{subfigure}
			
			\vspace{1em}
			
			\begin{subfigure}{\linewidth}
				\center
				\buildfig{figures/distance-map-torus.tex}
				\caption{2D torus topology}
				\label{fig:distance-map-torus}
			\end{subfigure}
			
			\vspace{1em}
			
			\begin{subfigure}{\linewidth}
				\center
				\buildfig{figures/distance-map-hex-mesh.tex}
				\caption{Hexagonal mesh topology}
				\label{fig:distance-map-hex-mesh}
			\end{subfigure}
			
			\vspace{1em}
			
			\begin{subfigure}{\linewidth}
				\center
				\buildfig{figures/distance-map-hex-torus.tex}
				\caption{Hexagonal torus topology}
				\label{fig:distance-map-hex-torus}
			\end{subfigure}
			
			\caption{Plots showing distance from various locations marked
			         {\color{red}$\times$}. Darker areas are further away. Contour
			         lines show equidistant points.}
			\label{fig:distance-map}
		\end{figure}
		
		\subsection{Mesh Networks}
			
			In a (non-hexagonal) mesh network topology, shortest path vectors are
			computed by taking the element-wise difference between the source and
			destination nodes' coordinates.
			
			\begin{figure}
				\center
				\buildfig{figures/mesh-topology-coordinates.tex}
				\caption{An example 2D mesh network with example shortest-path routes
				from `A' to `B' and `B' to `C'.}
				\label{fig:mesh-topology-coordinates}
			\end{figure}
			
			For example, figure \ref{fig:mesh-topology-coordinates} illustrates a 2D
			mesh topology. In this topology, the nodes labelled `A', `B' and `C' have
			position vectors $(1, 2)$, $(4, 5)$ and $(6, 1)$ respectively. The
			shortest path vector from node `A' to `B' is thus simply $(4, 5) - (1, 2)
			= (3, 3)$ and from `B' to `C' is $(6, 1) - (4, 5) = (2, -4)$.
			
			A route may be produced from a shortest path vector by advancing the
			number of hops specified for each dimension in the vector. For example
			any permutation of the hops X$^+\,$X$^+\,$X$^+\,$Y$^+\,$Y$^+\,$Y$^+$, an
			example of which is included in the figure. Likewise a route from `B' to
			`C' may be constructed from any permutation of
			X$^+\,$X$^+\,$Y$^-\,$Y$^-\,$Y$^-\,$Y$^-$.
			
			Many popular routing algorithms such as Dimension Order Routing (DOR),
			Right-Turn Only Routing (RTOR) and Longest Dimension First Routing (LDFR)
			\cite{dally04,davies12} directly follow the above procedure and just
			prescribe a specific permutation of hop order. For example, DOR produces
			routes with X hops first, Y hops second and so on.
			
			The length of routes produced from a shortest path vector have a number
			of hops proportional to the magnitude of the vector, thus a shortest path
			vector yields a route with the minimum number of hops. For a two
			dimensional vector $(a, b)$ the magnitude is given as:
			%
			\begin{equation}
				\| (a, b) \| = \lvert a \rvert + \lvert b \rvert
			\end{equation}
		
		\subsection{Torus Networks}
			
			Computing shortest path vectors in (non-hexagonal) torus topologies is
			also straight forward. As an example, lets find the shortest path vector
			from node `A' to `B' in the 2D torus topology shown in figure
			\ref{fig:torus-shortest-path-example}. First, both nodes are translated
			such that the source node, `A', is at the centre of the network (figure
			\ref{fig:torus-shortest-path-translate}). Note that this translation may
			result in the destination node `wrapping around' the network. After
			translation, the shortest path vector is computed as in a mesh topology.
			As illustrated in \ref{fig:torus-shortest-path-routed}, the computed
			shortest path vector may be used to produce routes between the two nodes
			in their original positions.
			
			\begin{figure}
				\center
				\begin{subfigure}{0.3\linewidth}
					\center
					\buildfig{figures/torus-shortest-path-example.tex}
					\caption{Original}
					\label{fig:torus-shortest-path-example}
				\end{subfigure}
				\begin{subfigure}{0.3\linewidth}
					\center
					\buildfig{figures/torus-shortest-path-translate.tex}
					\caption{Translated}
					\label{fig:torus-shortest-path-translate}
				\end{subfigure}
				\begin{subfigure}{0.3\linewidth}
					\center
					\buildfig{figures/torus-shortest-path-routed.tex}
					\caption{Routed}
					\label{fig:torus-shortest-path-routed}
				\end{subfigure}
				
				\caption{Finding shortest paths in a 2D torus topology.}
				\label{fig:torus-shortest-path}
			\end{figure}
			
			This process works because vectors from the centre (though not other
			locations) of a torus topology are identical to those in mesh topologies
			(see figures \ref{fig:distance-map-mesh} and
			\ref{fig:distance-map-torus}).
		
		\subsection{Hexagonal Mesh Networks}
			
			In hexagonal mesh topologies it is conventional to define three `axes' X,
			Y and Z as shown in figure \ref{fig:hex-mesh-topology-coordinates}
			\cite{patel15}. In this example, the three labelled nodes `A', `B' and
			`C' may be given position vectors such as $(1, 1, 0)$, $(3, 2, 0)$ and
			$(0, 0, -7)$ respectively. As in other mesh networks, a vector between
			two nodes is found by subtracting the nodes' vectors. For example, a
			vector from `A' to `B' is $(3, 2, 0) - (1, 1, 0) = (2, 1, 0)$. This
			vector can then be converted into a route in the same way as a mesh
			network by taking any permutation of the three hops  X$^+\,$X$^+\,$Y$^+$.
			
			\begin{figure}
				\center
				\buildfig{figures/hex-mesh-topology-coordinates.tex}
				\caption{An example hexagonal mesh network topology.}
				\label{fig:hex-mesh-topology-coordinates}
			\end{figure}
			
			As explained in detail in appendix \ref{app:minimal-hex-coordinates},
			there are an infinite number of vectors between any two points. For
			example, the vectors $(1, 0, -1)$ and $(3, 2, 1)$ also reach node `B'
			from `A' in the example. However, for a given pair of nodes, there is
			always a single, unique vector whose magnitude is minimal which is
			given by the function:
			%
			\begin{equation}
				\operatorname{minimiseVector}(x,y,z)
					= (x,y,z) - \operatorname{median}(x,y,z) \cdot (1,1,1)
			\end{equation}
			%
			An important side-effect of this function is that a minimised vector will
			always contain at least one zero element meaning that shortest path
			routes will use at most two of the three available dimensions.
			
			To aid the reader's intuition, figure \ref{fig:distance-map-hex-mesh}
			illustrates how distances grow in a hexagonal mesh topology.
		
		\subsection{Hexagonal Torus Networks}
			
			Unfortunately, unlike non-hexagonal torus topologies, the translation
			technique cannot be used to compute shortest path vectors. As illustrated
			in figures \ref{fig:distance-map-hex-mesh} and
			\ref{fig:distance-map-hex-torus}, shortest path vectors from the center
			of a hexagonal mesh network are not equivalent to those of a hexagonal
			torus network.
			
			Prior research into routing in SpiNNaker's network has been based on the
			INSEE \cite{navaridas09,ghasempour15} interconnect simulator. Internally
			INSEE tries a set of twelve candidate vectors and picks the shortest as
			the shortest path vector to use for routing.
			
			\begin{figure}
				\center
				\begin{subfigure}{0.45\linewidth}
					\center
					\buildfig{figures/insee-vector-candidates-no-wrap.tex}
					\caption{$(\Delta_\textrm{X}, \Delta_\textrm{Y}) = (5,3)$}
					\label{fig:insee-vector-candidates-no-wrap}
				\end{subfigure}
				\begin{subfigure}{0.45\linewidth}
					\center
					\buildfig{figures/insee-vector-candidates-wrap-x.tex}
					\caption{$(\Delta'_\textrm{X}, \Delta_\textrm{Y}) = (-3,3)$}
					\label{fig:insee-vector-candidates-wrap-x}
				\end{subfigure}
				
				\vspace{1em}
				
				\begin{subfigure}{0.45\linewidth}
					\center
					\buildfig{figures/insee-vector-candidates-wrap-y.tex}
					\caption{$(\Delta_\textrm{X}, \Delta'_\textrm{Y}) = (5,-5)$}
					\label{fig:insee-vector-candidates-wrap-y}
				\end{subfigure}
				\begin{subfigure}{0.45\linewidth}
					\center
					\buildfig{figures/insee-vector-candidates-wrap.tex}
					\caption{$(\Delta'_\textrm{X}, \Delta'_\textrm{Y}) = (-3,-5)$}
					\label{fig:insee-vector-candidates-wrap}
				\end{subfigure}
				
				\vspace{1em}
				
				% Key
				\begin{tikzpicture}[thick]
					\coordinate (last);
					
					% #1 colour
					% #2 label
					\newcommand{\colourkeyentry}[2]{
						\node [#1] [right=of last, fill, rectangle, minimum size=1em] (last) {};
						\node [right=0 of last] (last) {#2};
					}
					
					\colourkeyentry{cb3class0}{$(\textrm{X}, \textrm{Y}, 0)$}
					\colourkeyentry{cb3class1}{$(\textrm{X} - \textrm{Y}, 0, - \textrm{Y})$}
					\colourkeyentry{cb3class2}{$(0, \textrm{Y} - \textrm{X}, - \textrm{X})$}
					
				\end{tikzpicture}
				
				\caption{The twelve candidate shortest-path vectors considered by INSEE
				represented as dimension-order routes. $W=H=8$,
				$(\Delta_\textrm{X},\Delta_\textrm{Y}) = (5, 3)$ and
				$(\Delta'_\textrm{X},\Delta'_\textrm{Y}) = (-3, -5)$.}
				\label{fig:insee-vector-candidates}
			\end{figure}
			
			The twelve vectors considered are constructed as follows.
			
			First a shortest path vector from the source to target node are
			constructed as if the network was a 2D mesh yielding a vector
			$(\Delta_\textrm{X},\Delta_\textrm{Y})$. From this, another vector
			$(\Delta'_\textrm{X},\Delta'_\textrm{Y})$, is defined:
			%
			\begin{align}
				\Delta'_\textrm{X} &= \Delta_\textrm{X} - \operatorname{sign}(\Delta_\textrm{X})W
				\\
				\Delta'_\textrm{Y} &= \Delta_\textrm{Y} - \operatorname{sign}(\Delta_\textrm{Y})H
			\end{align}
			%
			Where $W$ and $H$ are the width and height of the network respectively
			(in nodes). This new vector yields routes from the source to destination
			node that in a torus topology that \emph{always} wrap around the `X' and
			`Y' dimensions.
			
			From the pair of vectors defined, four possible 2D vectors can be
			produced: $(\Delta_\textrm{X},\Delta_\textrm{Y})$,
			$(\Delta'_\textrm{X},\Delta_\textrm{Y})$,
			$(\Delta_\textrm{X},\Delta'_\textrm{Y})$ and
			$(\Delta'_\textrm{X},\Delta'_\textrm{Y})$. Further, each 2D vector may be
			converted into one of three 3D vectors where either X, Y or Z are zero
			for a total of twelve candidate vectors.  Figure
			\ref{fig:insee-vector-candidates} illustrates all twelve candidate
			vectors for an example pair of nodes.
			
			\begin{figure}
				\center
				\buildfig{figures/xyz-protocol-regions.tex}
				
				\caption{The four regions defined by the XYZ-protocol.}
				\label{fig:xyz-protocol-regions}
			\end{figure}
			
			A more efficient technique is proposed by Hoffmann and D\'es\'erable
			called the XYZ-Protocol \cite{hoffmann15,hoffmann11}. If the source and
			destination nodes are translated such that the source node lies at the
			center of the topolgoy, the destination will lie in one of four regions
			illustrated in figure \ref{fig:xyz-protocol-regions}.
			
			If the destination lies in regions 1 or 4, a route may be constructed as
			if in a hexagonal mesh topology.
			
			Alternatively, if the destination lies in regions 2 or 3, the algorithm
			tests whether taking a mesh-like route within the region or
			wrapping-around either the X or Y dimension yields the shorter vector.
			The shortest of these vectors is then chosen.
			
			TODO DESCRIBE SPIRAL ROUTES.
			
			TODO DESCRIBE RTOR AND LDFR.
		
	\section{Dimension order routing in hexagonal torus topologies}
		
		So, existing solutions have two problems: trying 12 options and picking one
		is a bit kludgey and there are actually more options than that.
		
		\subsection{Simpler minimal hexagonal torus vectors}
			
			If you redraw your route such that it is sourced from bottom left corner
			(which we'll now call (0, 0)), there are four possible ways this route
			could wrap.
			
			TODO: DESCRIBE WAYS OF WRAPPING
			
			For each of these wrappings, all the possible routes we can take are
			strictly limited in terms of the dimensions used since we're stuck in a
			corner.
			
			In each case, the function computing the minimal hex vector function
			simplifies to a much simpler operation.
			
			TODO: DESCRIBE MINIMUM VECTOR LENGTH FUNCTIONS FOR EACH CASE
			
			This gives us a cheap way to compute which of the four possible wrappings
			are shortest. Based on this we can pick one of at most two (is this
			easily provable?) vectors in some fair manner to reduce load. This vector
			can then be minimised in the usual way.
			
			This also leads to confirming a theoretical result giving the length of a
			shortest path in a hexagonal torus topology.
			
			TODO: DESCRIBE HOW TO GET LENGTH FUNCTION AND COMPARE WITH \cite{xiao04}
		
		\subsection{Generating spiralling routes}
			
			In non-hexagonal torus topologies the previous technique would reveal all
			possible shortest vectors (e.g. in cases where you can wrap more than one
			way). Unfortunately, due to the addition of a non-orthogonal axes,
			hexagonal toruses also have other cases when the width and height do not
			match.
			
			TODO: TORUS SPIRALLING EXAMPLE
			
			It is possible to calculate the maximum number of spirals thus:
			
			TODO: DESCRIBE HOW MAX NUMBER OF SPIRALS IS COMPUTED
			
			Given a number of spirals, the vector can be updated this (note that the
			change does not add a multiple of (1, 1, 1) but also does not result in
			the vector changing length and thus becoming non-minimal!).
			
			TODO: DESCRIBE TRANSFORMATION
			
			TODO: PROVE THAT MINIMALITY IS MAINTAINED
		
		\subsection{Proof of completeness}
		
			TODO: PROOF OF COMPLETENESS BY EXHAUSTIVE SEARCH
	
		\subsection{Conclusions}
			
			This approach is simpler, smaller, has sounder theoretical basis, and
			finds more routes than alternatives. This is good for load balancing and
			fault avoidance and also good for completeness.


	\chapter{Routing packets in large SpiNNaker machines}
	
	\label{sec:routing}
	
	So far, this thesis has focused on tackling the practical challenges
	resulting from SpiNNaker's hexagonal torus network topology. In this chapter,
	I adjust my focus towards the practical challenges resulting from SpiNNaker's
	large scale. Faults in large systems are inevitable and in the half-million
	core, \num{28800} chip SpiNNaker machine recently completed at the University
	of Manchester, around \SI{1}{\percent} of chips exhibited faults\footnote{Of
	the faulty chips discovered, the vast majority of faults are attributed to a
	currently unknown SDRAM failure}. These faults result in gaps and broken
	links in the network topology which routing algorithms must avoid in order to
	ensure correct system operation.
	
	In this chapter I tackle the problem of extending existing routing algorithms
	for SpiNNaker's network to enable them to route-around known, static faults.
	Though dynamic or transient faults may also occur, in this work such faults
	are ignored and other techniques, such as protocol-level fault tolerance, are
	relied on instead.
	
	Numerous heuristic-based fault-tolerant routing algorithms exist which target
	different network topologies and router architectures. Unfortunately as I
	will show, these algorithms are not portable and rely on or attempt to work
	around specific features of their target network architecture. In particular,
	existing work is dominated by the challenge of developing routing schemes
	which avoid or defuse network deadlocks. Due to SpiNNaker's unconventional
	use of timeout-based flow-control, it is not subject to the routing
	restrictions present in other architectures intended to cope with deadlocks.
	
	In this chapter I introduce a graph-search based post-processing step for
	non-fault-tolerant routing algorithms which guarantees routability in
	SpiNNaker systems without disconnected subregions. I also demonstrate that
	this technique introduces both negligible computational overhead to the
	routing algorithm runtime and resulting network performance.
	
	TODO: NOTE THE FAULT RATES ENCOUNTERED IN PRACTICE...
	
	\section{Related work}
		
		Existing work on routing in SpiNNaker's network has ignored the challenge
		of avoiding faults and instead focused on producing efficient multicast
		routes. As a result this section is broken into two halves. In the first
		half I survey the existing non-fault-tolerant approaches to routing used in
		SpiNNaker to-date. In the second I discuss the approaches to fault tolerant
		routing taken in other systems.
		
		\subsection{Multicast routing in SpiNNaker}
			
			Various fault-intolerant multicast routing algorithms exist for many
			networks and a number have been proposed and evaluated specifically in the
			context of SpiNNaker.
			
			In 2012, Davies \emph{et al.} evaluated the use of three common torus
			routing algorithms in SpiNNaker and found that simple oblivious routing is
			suitable for typical neural applications \cite{davies12}. The three
			routing techniques are:
			
			\begin{description}
				
				\item[Dimension Order Routing (DOR)] Packets are routed along each
				dimension (e.g. $X$, $Y$ and $Z$) in turn until no further hops are
				available in that direction.  The order in which the dimensions are
				traversed is fixed.
				
				\item[Right Turn Only Routing (RTOR)] As in DOR except the dimension
				order is chosen such that routes only contain right-turns.
				
				\item[Longest Dimension First Routing (LDFR)] As in DOR except the
				dimension order is chosen in descending order of number of hops in each
				dimension.
				
			\end{description}
			
			These unicast routing techniques are converted into a multicast routing
			algorithm by merging together the routes produced between the source node
			and each destination node as illustrated in figure
			\ref{fig:simple-routers}.
			
			\begin{figure}
				\center
				\begin{subfigure}{0.3\linewidth}
					\center
					\buildfig{figures/simple-routers-dor.tex}
					
					\caption{DOR}
					\label{fig:simple-routers-dor}
				\end{subfigure}
				\begin{subfigure}{0.3\linewidth}
					\center
					\buildfig{figures/simple-routers-rtor.tex}
					
					\caption{RTOR}
					\label{fig:simple-routers-dor}
				\end{subfigure}
				\begin{subfigure}{0.3\linewidth}
					\center
					\buildfig{figures/simple-routers-ldfr.tex}
					
					\caption{LDFR}
					\label{fig:simple-routers-dor}
				\end{subfigure}
				
				\caption{Example multicast routes produced by merging together unicast
				routes from a central source node to each destination node.}
				\label{fig:simple-routers}
			\end{figure}
			
			In 2014, Navaridas \emph{et al.} introduced two new algorithms, `Enhanced
			Shortest Path Routing' (ESPR) and `Neighbourhood Exploring Routing' (NER)
			which produce multicast routing trees with fewer total hops
			\cite{navaridas14}. In both algorithms, routes are generated sequentially
			for each of the destinations of a route using LDFR. Unlike LDFR, however,
			these algorithms search a limited area of the network for other,
			already-connected destination nodes to which LDFR routes may be
			constructed. If no suitable destination is found, a LDFR route is
			constructed to the source node. Figure \ref{fig:search-regions} illustrates
			the shape of the searched regions of each algorithm. ESPR searches the
			trapezoidal region between the source and destination nodes while NER
			searches a fixed radius out from the destination node.
			
			\begin{figure}
				\center
				\begin{subfigure}{0.45\linewidth}
					\center
					\buildfig{figures/search-regions-espr.tex}
					
					\caption{ESPR}
					\label{fig:search-regions-espr}
				\end{subfigure}
				\begin{subfigure}{0.45\linewidth}
					\center
					\buildfig{figures/search-regions-ner.tex}
					
					\caption{NER}
					\label{fig:search-regions-espr}
				\end{subfigure}
				
				\caption{The ESPR and NER algorithms attempt to connect the node marked
				`D' to the closest node in the shaded region which is connected to the
				source node, `S'. If no connected node is found in the shaded region, the
				LDFR route is taken to `S'. The dotted line indicates the route chosen
				from `D'.}
				\label{fig:search-regions}
			\end{figure}
			
			Unfortunately none of these routing algorithms make any allowance for the
			avoidance of network faults. As a result their utility in real-world
			systems is limited.
		
		\subsection{Fault-tolerant routing}
			
			Numerous fault-tolerant routing algorithms have been proposed for
			super-computer networks. These algorithms are largely constrained by the
			need to maintain deadlock freedom. Since SpiNNaker's routers employ a
			timeout based deadlock-breaking strategy, much of this effort is
			unnecessary in SpiNNaker. As described below, this frequently renders
			existing fault-tolerant routing algorithms unnecessarily complex and
			inflexible.
			
			Deadlocks occur in a network if a cyclic dependency exists on any limited
			resource in the network. For example, as illustrated in figure
			\ref{fig:ring-deadlock}, in a ring network a deadlock may form when every
			node is waiting on the next node to accept a packet before accepting new
			packets from the previous node.
			
			\begin{figure}
				\center
				\buildfig{figures/ring-deadlock.tex}
				
				\caption{A deadlock in a ring network where each node is waiting for
				the next to accept a packet before accepting any further packets.}
				\label{fig:ring-deadlock}
			\end{figure}
			
			To prevent deadlocks, combinations of router microarchitectural features
			and routing restrictions may be employed. For example, a simple
			deadlock-free routing algorithm for mesh and torus networks mandates the
			use of DOR \cite{dally93}. Packets travelling in a -ve direction along
			each axis take priority over those travelling in a +ve direction. Packets
			travelling along the Y axis take priority over those travelling along the
			X dimension. Given these rules it is possible to define a total ordering
			on all hops in the network. Figure \ref{fig:deadlock-free-dor}
			illustrates a $3\times3$ mesh network whose hops have been numbered
			according to the total ordering defined above.  Any `X-then-Y' DOR route
			through this network results in the use of hops labelled with strictly
			increasing numbers. As a result, no cyclic dependencies (and thus no
			deadlocks) may occur.
			
			\begin{figure}
				\center
				\buildfig{figures/deadlock-free-dor.tex}
			
				\caption{Deadlock-free routing of two example routes using DOR in a 2D
				mesh topology. The numbers of the hops taken by each route are given on
				the right.}
				\label{fig:deadlock-free-dor}
			\end{figure}
			
			Unfortunately, the routing restrictions imposed to ensure deadlock
			freedom can result in fault-intolerant routing. In the example above, if
			the node at the bottom-right corner of the figure was faulty, the dotted
			example route would be blocked as no alternative routes are allowed.
			
			In practice, the routing rules used may be more relaxed, for example
			requiring that any route whose length is equal to a DOR must exist to
			guarantee routability \cite{rodrigo09}.
			
			Alternative routing strategies take a hybrid approach whereby an
			efficient but fault-intollerant routing algorithm is used where possible
			and in the presence of faults a less efficient but more robust strategy
			is employed. For example, the Immucube network architecture employs three
			virtual networks which operate independently over the same physical links
			\cite{puente07}. Initially messages are routed using a high-performance
			but potentially-deadlockable routing scheme in the first virtual network.
			If a deadlock is occurs, the deadlocked packet is dropped into the second
			virtual network in which packets are routed using a less efficient but
			deadlock-free but fault-intolerant routing algorithm. Finally, upon
			encountering a fault, packets are dropped onto the third virtual network
			which forms a ring network routing packets to every node in the network.
			
			Releated approaches \cite{mejia06,boppana95} divide the network into
			regions in which different routing rules are enforced to ensure deadlock
			freedom and, when required, fault tolerance.
			
			TODO FIGURE?
			
			The BlueGene/L supercomputer \cite{adiga02} uses DOR for its torus
			network and implements fault-tolerance by sacrificing otherwise
			functioning `lamb' nodes to ensure no route passes through a known dead
			link \cite{ho04}. In figure \ref{fig:lamb-nodes} an example scenario is
			shown where a single dead node is present and all nodes in the same row
			or column as the dead node have been made into lamb nodes. The lamb nodes
			may not be used in an application except as a through-route for other
			traffic. This pattern of lamb nodes guarantees that all dimension-order
			routes between all pairs of non-lamb nodes are not obstructed by the
			faulty node. This approach trades use of higher performance routing
			logic for wasted resources. This type of approach is most appropriate
			when algorithmic routing is used and routing rules are inflexible.
			
			\begin{figure}
				\center
				\buildfig{figures/lamb-nodes.tex}
				
				\caption{`Lamb' nodes may be disabled to ensure DOR will never
				encounter a fault.}
				\label{fig:lamb-nodes}
			\end{figure}
			
			Other algorithms proposed for the BlueGene architecture attempt to avoid
			the need for lamb nodes by generating routes which reach their destination
			via a `proxy' node \cite{gomez04}. By appropriately selecting the location
			of such a proxy, the existing routing algorithm used by the system can be
			guaranteed to select a route free of faults.
			
			TODO: EXAMPLE OF PROXY ROUTING TO AVOID FAULT
			
			Finally, many algorithms in in the field are distributed and use only local
			information along with limited information from their peers to generate
			routes \cite{fick09b}. In SpiNNaker, route generation is conventionally
			carried out centrally since no special on-chip hardware facilities exist
			for route generation. Centralised route generation also enables the routing
			algorithm to consider all available routes. As a result, there is little
			incentive for the use of distributed routing algorithms on SpiNNaker since
			global system information could be compactly shared for one-off routing
			passes.
			
			Algorithms for other architectures such as IP networks tend to be poor fits
			for static, regular network topologies since they use expensive graph-based
			algorithms for route discovery which aren't necessary here. They also tend
			to heavily feature graph topology discovery etc. which aren't needed here.
			
			Work on fault-tolerance in data centre networks does exploit the regularity
			of the network topology in routing algorithms \cite{guo08,liao12}.
			Unfortunately, the approaches used are not general enough to be applied to
			mesh-like topologies such as the one in SpiNNaker.
			
			Outside the field of computer networks, routing algorithms used to route
			wires across the surfaces of chips are required to solve similar problems
			to fault-tolerant network routing problems in mesh networks. Like mesh
			networks, the routes are defined within a regular Manhattan geometry and
			congested areas, rather than faults must be avoided by the algorithms
			\cite{kahng11}.  Unfortunately, these algorithms are designed for
			occasional batch operation prior to the multi-month process of chip
			manufacturing and so runtimes of hours or days are commonplace
			\cite{nam08}. As such these algorithms would be inappropriate for use
			with applications such as SpiNNaker where users' applications tend to be
			short-lived and thus routing should not be allowed to dominate runtime.
	
	\section{Partial graph search repair}
		
		In this section I introduce a novel post-processing algorithm, Partial
		Graph Search (PGS) repair, for routes produced by non-fault-tolerant
		routing algorithms.
		
		PGS repair guarantees routability for networks with no disconnected
		subregions by using a graph search algorithm to route around faults in the
		original route.  General-purpose graph search algorithms such as Breadth
		First Search (BFS), Dijkstra's Algorithm and A* are guaranteed to find
		shortest-path routes between pairs of points in arbitrary graphs. Such
		algorithms are generally a poor choice in highly regular network topologies
		such as meshes and toruses due to their high computational cost. In PGS
		repair, graph searching is only used for \emph{part} of the routing
		problem: to repair gaps in routes generated by more efficient routing
		algorithms.
		
		Real world super computer architectures are designed to ensure that faults
		are isolated \cite{gara05,alverson12} and thus tend to only impact a
		localised region of the network. Since PGS repair is only needed to route
		around these isolated faults, the space searched by the graph search
		algorithm should be very small in practice resulting in only short
		runtimes. In addition since faults are rare in real-world systems, the
		graph search process will only rarely be invoked.
		
		The PGS repair post-processing technique starts with a route produced by a
		non-fault-tolerant routing algorithm such as ESPR or NER. If this route is
		not obstructed by a fault, the algorithm terminates immediately without
		modifying the route. If the route attempts to use a faulty link, the
		algorithm proceeds as follows.
		
		The routing tree produced by the underlying routing algorithm is broken
		into subtrees wherever it attempts to route through a broken link and
		each subtree is assigned a unique colour, as illustrated in figure
		\ref{fig:pgs-repair-colouring}. From each disconnected subtree's root
		node in turn, a graph search is performed to find a short, fault-free
		route to a subtree node of a different colour. The subtree is then
		attached to the tree discovered by the graph search and re-coloured to
		match the tree it is connected to.
		
		\begin{figure}
			\center
			\begin{subfigure}{0.32\linewidth}
				\hspace*{-1.5em}
				\buildfig{figures/pgs-repair-colouring.tex}
				
				\caption{}
				\label{fig:pgs-repair-colouring}
			\end{subfigure}
			\begin{subfigure}{0.32\linewidth}
				\hspace*{-1.5em}
				\buildfig{figures/pgs-repair-colouring-fix1.tex}
				
				\caption{}
				\label{fig:pgs-repair-colouring-fix1}
			\end{subfigure}
			\begin{subfigure}{0.32\linewidth}
				\hspace*{-1.5em}
				\buildfig{figures/pgs-repair-colouring-fix2.tex}
				
				\caption{}
				\label{fig:pgs-repair-colouring-fix2}
			\end{subfigure}
			
			\caption{PGS repair process example showing a disconnected multicast
			route from A to B, C, D, E and F. $\times$ indicates a broken link.}
			\label{fig:pgs-repair-colouring-steps}
		\end{figure}
		
		For example in figure \ref{fig:pgs-repair-colouring-fix1} a path from the
		root of the subtree containing nodes E and F is found which connects it to
		the subtree rooted at A. Similarly in figure
		\ref{fig:pgs-repair-colouring-fix2} a path is also found connecting the
		subtree containing nodes C and D back to the subtree rooted at node A.
		
		If the routing tree was broken into $N+1$ subtrees by faults there will be
		$N$ subtrees disconnected from the root node. Each of the $N$ iterations of
		the algorithm connect a disconnected subtree to another subtree reducing
		the number of subtrees by $1$ each time. After $N$ iterations, therefore,
		exactly $1$ subtree remains which connects every node in the original
		routing tree without traversing faulty links.
		
		TODO: EXPLAIN THE FIDDLINESS HERE TO ENSURE WE DON'T CREATE LOOPS.
		
	\section{Evaluation \& Results}
		
		The PGS repair technique, by design, is able to work around all possible
		fault patterns which don't completely disconnect parts of the network. This
		result this evaluation focuses on the impact on performance PGS repair
		imposes. The metrics of interest in this evaluation are:
		
		\begin{itemize}
			\item Algorithm runtime
			\item Network congestion
			\item Routing table utilisation
		\end{itemize}
		
		\subsection{Traffic Patterns}
			
			In this evaluation, two standard benchmark multicast traffic patterns are
			used which have been used in previous research into SpiNNaker's network:
			
			\begin{figure}
				\center
				\buildfig{figures/traffic-distribution-centroids.tex}
				
				\caption{An example 4-centroid distribution with four centroids. The
				$\times$ marks the location of the origin node. Lighter colours
				indicate greater likelihood of a connection.}
				\label{fig:traffic-distribution-centroids}
			\end{figure}
			
			\begin{description}
				
				\item[Uniform] Destinations are chosen with uniform probability
				anywhere in the machine.
				
				\item[$N$-Centroids] Destinations are clustered around one of $N$
				randomly chosen `centroids' as illustrated in figure
				\ref{fig:traffic-distribution-centroids}.
				
			\end{description}
			
			The uniform traffic pattern is widely used in networks research
			\cite{dally04,davies12} while the centroids model was developed
			specifically to reproduce the traffic patterns found in the neural
			applications SpiNNaker is designed for \cite{navaridas14}. In this work
			we consider 3 centroids.
		
		\subsection{Fault model}
			
			In addition two different fault models are used which are representative of
			the faults found in real SpiNNaker systems:
			
			\begin{figure}
				\center
				\begin{subfigure}{0.48\linewidth}
					\hspace*{-1.5cm}
					\buildfig{figures/fault-example-uniform.tex}
					
					\caption{Uniform}
					\label{fig:fault-example-uniform}
				\end{subfigure}
				\begin{subfigure}{0.48\linewidth}
					\hspace*{-1.5cm}
					\buildfig{figures/fault-example-hss.tex}
					
					\caption{HSS Link}
					\label{fig:fault-example-hss}
				\end{subfigure}
				
				\caption{The two link fault models considered.}
				\label{fig:fault-example}
			\end{figure}
			
			\begin{description}
				
				\item[Uniform] Links are selected and disabled at random (figure
				\ref{fig:fault-example-uniform}).
				
				\item[HSS Link] Groups of links corresponding with randomly selected
				single High-Speed Serial (HSS) link between SpiNNaker boards are disabled
				together (figure \ref{fig:fault-example-uniform}).
				
			\end{description}
			
			The uniform link failure model models isolated failures resulting from
			isolated manufacturing defects in individual links. The HSS Link failure
			model models faults arising from failing or disconnected board-to-board
			links which carry several chip-to-chip traffic flows via a single cable in
			SpiNNaker systems. Though SpiNNaker-specific, the later fault model is
			analogous to failure modes arising in other architectures where a single
			fault may render several links impassable in a single area.
			
			A range of failure rates are explored in this section. My measurements of
			current large-scale SpiNNaker installations the link failure rate is about
			\SI{0.03}{\percent} with failures due to both individual chip-to-chip links
			and board-to-board HSS links. Exact link failure statistics for commercial
			super computer installations are not widely available, however, published
			Mean-Time-Between-Failure (MTBF) statistics place an upper bound on link
			failure rates at a similar \SI{0.03}{\percent} in one-year-old BlueGene/Q
			systems \cite{chiu11}.
			
			Unfortunately presently undiagnosed problem with the SDRAM packaged with
			approximately \SI{1}{\percent} of SpiNNaker chips has rendered these chips
			unusable for most applications. The gaps in the network resulting from the
			loss of these chips currently dominate true link failures leaving just over
			\SI{1}{\percent} of links inoperable.
			
			Surprisingly, research into fault tolerant routing in super computers
			appears to focus on benchmarks with even higher fault rates ranging from
			\SI{3}{\percent} to as high as \SI{7}{\percent}
			\cite{ho04,gomez04,mejia06}.
			
			In this evaluation, fault rates ranging from \SI{0.01}{\percent} to
			\SI{5}{\percent} are considered to cover both realistic fault levels
			along with the more extreme cases considered in related work.
		
		\subsection{Base routing algorithm}
			
			Since the PGS repair process is routing algorithm agnostic all
			experiments use the NER algorithm which has been found to be appropriate
			for SpiNNaker applications \cite{navaridas14}.
		
		\subsection{Algorithm runtime}
			
			To assess the impact of the PGS repair process on routing algorithm
			runtime, the algorithm was used to process a large number of randomly
			generated routing problems and the runtime recorded.
			
			\num{10000} one-to-sixteen multicast routing problems were generated in a
			$256\times256$ hexagonal torus topology, the largest size possible for a
			SpiNNaker system. Other quantities of multicast destinations were also
			evaluated but are omitted for brevity since the pattern of results are
			similar to those outlined here.
			
			TODO: APPENDIX WITH OTHER RUNS?
			
			The NER and PGS repair algorithms were written in C and compiled with GCC
			4.8.3 with \verb|-O2| level optimisations and executed on a cluster of
			idle workstations with 3.10 GHz Intel Core-i5-2400 CPUs.
			
			\begin{figure}
				\center
				\buildrplot{figures/routing-runtimes.R}
				
				\caption{Mean runtime of routing and PGS repair overhead. PGS repair
				overhead is stacked above the routing runtime (i.e. bars do not
				overlap). Error bars indicate 95\% confidence interval. Note different
				Y-scale for HSS link and uniform fault models.}
				\label{fig:routing-runtimes}
			\end{figure}
			
			Figure \ref{fig:routing-runtimes} shows the average runtimes recorded for
			both the NER and PGS repair algorithms. In fault-free networks the
			PGS-repair post-processing step is not required and incurs no penalty
			while the runtime of the algorithm grows with the fault rate for both
			fault and traffic models.
			
			Notably the HSS fault model results in longer runtimes for the PGS repair
			process compared with an equivalent fault-density of uniform faults.
			Because the HSS fault model produces contiguous lines of faults the PGS
			repair algorithm must construct a longer path to avoid the fault.  Since
			the space explored by a graph algorithm typically grows with $O(H^2)$
			with respect to the hops in the discovered route, $H$, this increase in
			search distance has a large impact on the runtime of the PGS repair
			process.
			
			The runtime of the PGS repair algorithm remains roughly in proportion to
			the runtime of the underlying routing algorithm with respect to different
			traffic models. The centroid traffic pattern tends to result in routes
			with fewer hops than a uniform traffic pattern with the same number of
			destination nodes as segments of routes are often shared between
			destination nodes. Since the NER algorithm's runtime is strongly related
			to the number of hops in the output route the runtime of the algorithm is
			greater for uniform traffic. Likewise the probability of PGS repair being
			required increases with the number of hops in route and hence the runtime
			of the PGS repair algorithm increases roughly in proportion.
		
		\subsection{Routing table usage}
			
			In order to gain a realistic measure of routing table usage it is
			necessary to determine the effect of many routes being generated for a
			single set of faults. To enable a sufficiently large number of sample to
			be collected the experimental setup considered previously is reduced to a
			network containing $48\times48$ nodes.
			
			\num{1000} $48\times48$ node network models are produced according to the
			HSS link and uniform fault models. For each of these models
			$48\times48\times16=$~\num{36864} one-to-sixteen routes are generated using
			the centroid and uniform traffic models. This corresponds to one
			multicast route per application core. As is convention in SpiNNaker,
			routing table entries are inserted for each route at the source of the
			route, at each destination and at each corner or fork. The number of
			routing table entries at each node in the model is counted and the
			maximum number of entries in a single node is reported for each network
			model.  The \emph{maximum} number of routing entries of any router was
			chosen since the number of entries available per SpiNNaker router is
			bounded by hardware.
			
			\begin{figure}
				\center
				\buildrplot{figures/routing-entries.R}
				
				\caption{Violin plot showing the distribution of maximum table sizes
				for \num{1000} random networks. The red line at \num{1024} entries
				indicates the size of SpiNNaker's routing tables.}
				\label{fig:routing-entries}
			\end{figure}
			
			
			Figure \ref{fig:routing-entries} shows the distributions of the largest
			routing table sizes for each fault and traffic model.
			
			\begin{figure}
				\center
				\begin{subfigure}{0.48\linewidth}
					\center
					\buildfig{figures/hss-link-routing-table-usage.tex}
					
					\caption{Routing table entries}
					\label{fig:hss-link-routing-table-usage}
				\end{subfigure}
				\begin{subfigure}{0.48\linewidth}
					\center
					\buildfig{figures/hss-link-resource-usage.tex}
					
					\caption{Routes passing through chip}
					\label{fig:hss-link-resource-usage}
				\end{subfigure}
				
				\caption{The impact of a HSS link fault on routing table usage and
				congestion. Each hexagon represents a single chip, the red line
				indicates the chip-to-chip connections broken by the HSS link fault.}
				\label{fig:hss-link-usage}
			\end{figure}
			
			The HSS link failure model has a much greater impact on peak routing
			table resource usage than uniform link failures for a given fault rate.
			This is because HSS link faults result in a large concentration of routes
			being disrupted and then re-routed around the same obstacle in a single
			location. Figure \ref{fig:hss-link-routing-table-usage} shows how routing
			table usage varies around a HSS link fault in one instance of the
			experiment. There are clear peaks in routing table usage around the ends
			of the line of faults which result from routes produced by PGS repair
			finding shortest paths around the edge of the faults.
		
		\subsection{Network congestion}
			
			To measure the impact of PGS repair on network congestion, two
			experiments were performed, one using the same model used to measure
			routing table usage and one based on tests run on SpiNNaker hardware.
			
			For each of the network fault and traffic pattern described previously,
			the paths taken for the \num{36864} one-to-sixteen multicast routes
			generated are used to compute the number of times each link in the
			network is used. The number of routes passing through the most-used link
			is then recorded, giving an indication of the level of congestion in the
			network.
			
			\begin{figure}
				\center
				\buildrplot{figures/routing-resource.R}
				
				\caption{Violin plot showing the distribution of maximum
				routes-per-chip for \num{1000} random networks.}
				\label{fig:routing-resource}
			\end{figure}
			
			The results are presented in figure \ref{fig:routing-resource} and follow
			the same trends as the results previously shown for routing table usage.
			Again, HSS link faults result in routes with the greatest congestion due
			to the concentration of routes finding shortest paths around an obstacle
			(see \ref{fig:hss-link-resource-usage}).
			
			To verify that the results above, an additional experiment has been
			carried out which attempts to mimic the model used previously in actual
			SpiNNaker hardware. In these experiments a large SpiNNaker machine is
			divided into independent 48-board (2304-chip) sections. Because the
			48-board systems used in these experiments are cut out of a larger
			machine, they lack wrap-around links and thus form hexagonal mesh
			topologies, rather than hexagonal toruses.
			
			Due to the SDRAM issue described above, fault rates below
			\SI{1}{\percent} cannot be modelled.  To simulate higher fault rates,
			additional links are disabled in software according to the fault models
			described used previously. Since some faults are due to genuine hardware
			faults, these faults cannot be placed randomly in each experiment. To
			reduce, bias each combination of fault rate, fault model and traffic
			pattern is repeated XXX times across randomly chosen physical machines.
			
			XXX 1-to-XXX routes are generated in both uniform and XXX-centroid
			distributions as used throughout this evaluation. Synthetic network
			traffic is generated at the source of each route following a Bernoulli
			distribution. Traffic consumers running on all destination nodes accept
			packets as quickly as possible from the network and log their arrival.
			The Bernoulli probability is set the same for every route's traffic
			generator and increased in steps of XXX and the number of packets dropped
			in an XXX second period logged. The network is considered saturated once
			less than \SI{99}{\percent} of packets successfully arrive at their
			destination.
			
			Figure \ref{XXX} shows the distributions of the saturation points for
			each experimental configuration.
			
			TODO: ANALYSIS
		
	\section{Conclusions}
		
		In this chapter I described how SpiNNaker's unconventional network and
		router architecture render existing fault tolerant routing algorithms
		unsuitable. I introduced PGS repair, a post-processing technique for
		existing non-fault tolerant routing algorithms designed for SpiNNaker such
		as NER.
		
		Unlike some other fault tolerant routing algorithms for other
		architectures, PGS repair is able to work-around arbitrary fault patterns
		by exploiting SpiNNaker's inbuilt deadlock avoidance mechanisms. In the
		presence of realistic failure rates of up to \SI{1}{\percent}, only small
		overheads of up to XXX, XXX and XXX for in algorithm runtime, routing table
		usage and network performance are incurred respectively. This low
		performance overhead makes PGS repair appropriate for use in real
		applications. At the time of writing the algorithm has been successfully
		used in a number of neural and non-neural SpiNNaker applications.
		
		At more extreme fault rates not expected in real-world systems, the
		algorithm still functions correctly but the results incur much greater
		routing table and congestion overheads, particularly when faults are
		concentrated. Future extensions to this algorithm might aim to reduce this
		overhead by producing longer and more varied routes around faults to even
		out the load.

	\chapter{Placing applications in large SpiNNaker machines}
	
	In the previous chapter I tackled the problem of scale in generating routes
	for very large networks such as SpiNNaker. In this work the centroid traffic
	pattern was used as an approximation of the expected network traffic
	generated by `well behaved' neural network simulation software running on
	SpiNNaker. The traffic produced largely exhibits strong locality, that is
	most communication occurs between either nearby nodes or clusters of nodes.
	In reality, neural simulation applications are not specified geometrically
	but rather as abstract graphs of communicating neurons
	\cite{davison08,eliasmith13}. Applications must then \emph{place} these
	neurons onto nodes in a SpiNNaker system, attempting maximise communication
	locality.
	
	In this chapter I re-evaluate the suitability of simulated annealing as a
	technique for finding high quality placements for large parallel
	applications. Though this technique had fallen out of fashion in the field of
	application placement by the early 1990s, it has found wide use for placing
	components in computer chip and FPGA designs. In the intervening years,
	placement problems in super computers have grown in size from tens or
	hundreds of nodes to millions, a scale at which chip placement techniques
	were operating in the mid 1990s. I adapt the simulated annealing algorithm
	used by the VPR academic circuit placement software to produce placements for
	applications running on SpiNNaker. In that in a range of real and synthetic
	benchmarks simulated annealing produces high quality placements enabling
	efficient use of SpiNNaker's network resources.
	
	
	%In the field of chip design, Moore's `Law' \cite{moore65,moore75} observes a
	%similar exponential growth in the number of components within a single chip.
	%Today modern processors contain billions of components and an analagous
	%placement problem exists in attempting to place interconnected components
	%near to eachother. In this chapter I explore the techniques used for circuit
	%placement and adapt one such technique, Simulated Annealing (SA)
	%\cite{kirkpatrick83}, for use in application placement. Despite some early
	%interest in SA for application placement in the 1980s and early 1990s, the
	%technique has since fallen out of favour. I find that at the scales of modern
	%placement problems SA-based placement is able to produce solutions of
	%superiour quality to contemporary methods.
	%
	%TODO: SUMMARISE RESULTS...
	
	\section{Related work}
		
		The placement problem has been tackled independently in the literature by
		researchers in both the application and chip placement communities. In this
		survey I cover application and chip placement separately as these two
		communities have remained largely isolated from one another. First I
		explore the techniques applied to application placement before moving on to
		contrast this with the techniques used in circuit placement.
		
		In the application placement literature, the placement problem is often
		referred under the umbrella term `mapping'. Unfortunately term is often
		used more broadly to include other tasks such as routing and application
		partitioning. To avoid ambiguity I use the term `placement', as preferred
		by the chip and FPGA design communities, to refer specifically to the
		problem of assigning nodes in an application's communication graph to nodes
		in a machine's connectivity graph.
		
		\subsection{Application placement algorithms}
			
			TODO: GENERAL INTRO
			
			\subsubsection{Application-specific approaches (manual placement)}
				
				In the case of some applications such as finite element modelling
				\cite{bermejo13}, the structure of the problem itself leads to a
				natural placement of the computation on nodes in a machine. For example
				when simulating a 3D volume in an node super computer with a $3 \times
				4 \times 2$ 3D torus or mesh topology network, the modelled volume
				might be divided into as in figure \ref{fig:fem-partitioning}. Each
				cuboid in the model is then assigned to the corresponding node in the
				network topology.
				
				\begin{figure}
					\center
					\buildfig{figures/fem-partitioning.tex}
					
					\caption{Example partitioning of a 3D space to fit into a super
					computer with a $3\times4\times2$ torus or mesh topology.}
					\label{fig:fem-partitioning}
				\end{figure}
				
				When the number of dimensions in a problem do not match that of the
				underlying network architecture, the common solution is to either
				divide only along a subset of the axes or to divide into additional
				pieces on the existing axes \cite{gilge14}.
			
			\subsubsection{Sequential placement}
				
				In the case where a placement solution is non-obvious one of the
				simplest and most popular strategies is to apply a simple sequential
				placement algorithm. Sequential placement algorithms function by
				iterating over the vertices in the application's communication graph
				and assigning them to a free node in the target machine. Sequential
				placement algorithms are differentiated by the order in which they
				iterate over vertices in the communication graph and fill nodes in the
				target machine. A number of widely used orderings are described below.
				
				\begin{figure}
					\center
					\begin{subfigure}{0.32\linewidth}
						\center
						\buildfig{figures/sequential-row-order.tex}
						\caption{Row-order}
						\label{fig:sequential-row-order}
					\end{subfigure}
					\begin{subfigure}{0.32\linewidth}
						\center
						\buildfig{figures/sequential-alternating.tex}
						\caption{Alternating}
						\label{fig:sequential-alternating}
					\end{subfigure}
					\begin{subfigure}{0.32\linewidth}
						\center
						\buildfig{figures/sequential-hilbert.tex}
						\caption{Hilbert curve}
						\label{fig:sequential-hilbert}
					\end{subfigure}
					
					\caption{Space-filling curves in 2D mesh and torus topologies.}
					\label{fig:sequential}
				\end{figure}
				
				Super computer management software such as SLURM \cite{yoo03} and Blue
				Gene's system software \cite{gilge14} by default na\"ively iterate over
				vertices in an application communication graph in the order they are
				provided. The nodes in the target machine are then iterated over in a
				simple space-filling curve through the network topology. Figure
				\ref{fig:hilbert-placement} illustrates the default patterns available
				in these software packages. The row-order (figure
				\ref{fig:sequential-row-order}) and alternating (figure
				\ref{fig:sequential-alternating}) curves correspond with 2D versions of
				the default node assignment orders used in SLURM and BlueGene systems.
				
				\begin{figure}
					\center
					\buildfig{figures/hilbert-placement.tex}
					
					\caption{A Hilbert curve, coloured from blue to red.}
					\label{fig:hilbert-placement}
				\end{figure}
				
				The Cray extensions to SLURM software provide a Hilbert curve
				\cite{hilbert91} (figure \ref{fig:sequential-hilbert}) node assignment
				order. Unlike the row-order and alternating space filling curves the
				Hilbert curve ensures that pairs of vertices close together in the node
				iteration order are also close together in the target machine's network
				\cite{moon01, zumbusch99}. Figure \ref{fig:hilbert-placement} shows a
				5$^\textrm{th}$-order Hilbert curve where each point in the curve is
				coloured according to its position along the curve. In this figure it
				is possible to see that nearby positions in the curve (which share
				similar colours) are also close in 2D space.
				
				When the proximity of vertices in the vertex-ordering supplied by an
				application is a good estimator of those vertices communication
				requirements, the sequential assignment schemes discussed above can be
				very effective. These techniques have also proven adequate in
				small-scale and densely connected applications such as early neural
				simulations running on prototype SpiNNaker machines with tens of nodes
				\cite{galluppi10} but growing beyond this scale has proven problematic.
				
				\begin{figure}
					\center
					\begin{subfigure}{0.45\linewidth}
						\center
						\buildfig{figures/rcm-initial.tex}
						
						\caption{Original permutation}
						\label{fig:rcm-initial}
					\end{subfigure}
					\begin{subfigure}{0.45\linewidth}
						\center
						\buildfig{figures/rcm-sorted.tex}
						
						\caption{RCM permutation}
						\label{fig:rcm-sorted}
					\end{subfigure}
					
					\caption{Adjacency matrix representation of a graph before and after
					permutation by the RCM algorithm.}
					\label{fig:rcm}
				\end{figure}
				
				A number of algorithms have been proposed for automatically selecting
				good vertex iteration orders, typically using a graph-traversal based
				heuristic. A typical method, described by Hoefler \emph{et al.}
				\cite{hoefler11} exploits the Reverse-Cuthill-McKee (RCM) algorithm
				\cite{cuthill69}. An application's communication matrix is represented
				as an adjacency matrix, $M$, where $M_{i,j}$ is 1 if node $i$ is
				connected by an edge to node $j$ and 0 otherwise. An example matrix is
				illustrated in figure \ref{fig:rcm-initial}. The RCM algorithm uses a
				simple heuristic to permute the matrix (i.e. renumber the nodes in the
				graph) in order to reduce the bandwidth of the matrix. Figure
				\ref{fig:rcm-sorted} shows the RCM-permuted version of the example
				adjacency matrix. When a graph's vertices are ordered as in a
				bandwidth-reduced sparse matrix, vertices close together in the
				ordering are likely to communicate while those further apart tend not
				to communicate.
				
			\subsubsection{Optimisation-based Placement}
				
				% Citations from short report about optimisation in placement...
				% \cite{chen06,jeannot14} and \cite{jeannot10} ("subsets of apps")
				
				In the academic community, a number of attempts have been made to use
				more sophisticated optimisation algorithms for the placement of
				applications. In 1985, Steele \cite{steele85} proposed the use of
				simulated annealing for placing applications in the 6D torus topology
				of the 64 node `Caltech Cosmic Cube' machine. Simulated annealing,
				originally developed by Kirkpatrick \emph{et al.} \cite{kirkpatrick83},
				is a general-purpose optimisation algorithm which works by analogy to
				the physical process of annealing. In brief simulated annealing
				functions by randomly swapping vertices in a candidate placement
				solution, accepting swaps which move connected vertices closer together
				and rejecting some proportion of swaps which move connected vertices
				further apart. The simulated annealing algorithm is described in detail
				later in this chapter.
				
				Towards the end of the 1980s, application placement appeared to be
				becoming less important as super computer network architectures
				improved:
				%
				\begin{displayquote}
					``Careful placement was necessary because of the slow communication
					and non-uniform addressing of early concurrent computers. However,
					the development of message passing machines with fast communications
					and a uniform global address space  has made placement less of an
					issue. In such machines a random placement performs nearly as well as
					an optimum placement.''
					
					\noindent --- W. Dally, 1987 \cite{dally87}
				\end{displayquote}
				%
				In addition, network and problem sizes remained small, so small in fact
				that linear-programming based optimal placement still appeared in
				benchmarks comparing placement algorithms \cite{xu91}. In this
				environment, simpler sequential placement algorithms gained favour over
				more computationally expensive algorithms such as simulated annealing.
				
				As problem and machine sizes have grown and network utilisation has
				once again become an important factor in application performance
				\cite{navaridas09b} more complex optimisation algorithms have
				reappeared in the literature. One popular approach employs graph
				partitioning algorithms such as METIS \cite{karypis98} to perform
				recursive bipartitioning based placement
				\cite{phillips14,hoefler11,pellegrini96}.  This placement process is
				illustrated in figure \ref{fig:partitioning}.
				
				In the first step, the application communication graph and machine
				connectivity graph are bipartitioned such that the number of edges
				between partitions is minimised. Each half of the communication graph
				is associated with one of the halves of the machine connectivity graph.
				The partitioning process is then repeated recursively on each of the
				two communication and connectivity graph pairs. The process halts when
				the graphs can no longer be partitioned at which point the vertices in
				the communication graph are placed on their associated node.
				
				\begin{figure}
					\center
					\buildfig{figures/partitioning.tex}
					
					\caption{Illustration of application placement by recursive
					partitioning.}
					\label{fig:partitioning}
				\end{figure}
				
				TODO: PARTITIONING IS GREAT AND ALL BUT QUALITY ISN'T ALWAYS GREAT AND
				IT DOESN'T DEAL WELL WITH MULTI-CONSTRAINT SCENARIOS E.G. PROCESSOR AND
				MEMORY RESTRICTIONS.
				
				Unfortunately, many of these simply aren't suited to the scale of
				neural applications running on SpiNNaker (e.g. only cope with tens of
				nodes while SpiNNaker may contain hundreds of thousands).
				
				Additionally, a number of algorithms have been developed which make
				assumptions about the topologies of the problem or network. Tree match
				for example attempts to map tree-shaped problems to tree-shaped
				networks. Such algorithms can be highly effective but again do not
				apply to SpiNNaker or its neural applications.
		
		\subsection{Chip placement algorithms}
			
			The chip-design industry has, for many years, dealt with problems
			analogous to the task of placing super computer jobs in a way suited to
			SpiNNaker. Modern CPUs have millions or billions of components with
			strictly fixed connectivity. CPU designers must place each of these onto
			a chip such that the connection lengths are controlled to reduce
			congestion and increase performance. As such, these algorithms are
			ideally suited to future super computer placement work since they already
			operate at the scales required \cite{nam07}.
			
			\subsubsection{Cost functions}
				
				HPWL is popular but a bit crap for high fan-outs. It is, however, quite
				simple.
				
				TODO: SELECT A BETTER COST FUNCTION...
			
			\subsubsection{Simulated annealing}
				
				One of the oldest techniques used for circuit placement is simulated
				annealing and this remains popular today thanks to its sheer
				versatility (see VPR, other open FPGA tools).
				
				SA works by analogy with the physical process of annealing.
				The simulated annealing algorithm works by selecting random pairs of
				components on a chip, swapping them and evaluating some cost function.
				If the swap reduces the cost function, it is kept, if not, depending on
				a function of the current temperature and the cost introduced by the
				swap.
				
				TODO: ILLUSTRATION OF SIMULATED ANNEALING SWAP OPERATION
				
				By occasionally allowing costly swaps, the annealing algorithm avoids
				becoming trapped in local minima. As the algorithm proceeds, the
				temperature is slowly reduced and with it the proportion of costly
				swaps which are retained. This causes the placement to move from
				exploration early on towards refinement later on.
				
				The temperature schedule of an annealing algorithm is critical to its
				success. In general these schedules are computed based on the
				performance of the algorithm as it runs. In VPR the following schedule
				is used.
				
				TODO: DESCRIBE VPR'S SCHEDULE
				
				TODO: FIND AND DESCRIBE ALTERNATIVE SCHEDULE?
				
				Unfortunately, SA is very difficult to parallelise, especially in the
				case of placement. As a result, its scalability has been limited and
				resulted in significantly reduced usage in recent work.
			
			\subsubsection{Partitioning placement}
				
				Partitioning based placement solves the placement problem using
				graph-partitioning recursively on the problem graph to assign each part
				of the circuit to some area in the super chip. Though a number of
				algorithms have proven successful in academic placement contests over
				the years, they are not popular in industrial settings.
			
			\subsubsection{Analytical placement}
				
				In analytical placement, cost function for the circuit graph is
				approximated in a form which is amenable to solutions with standard
				numerical or symbolic algebraic techniques. Using these techniques,
				exact minimum cost (in terms of the approximation) configurations can
				be obtained.
				
				Quadratic placement is a popular analytical placement technique which
				approximates the cost of a placement as the sum of the squares of the
				distances between connected circuit elements.
				
				TODO: FIGURE EXAMPLE QUADRATIC PLACEMENT PROBLEM AND SOLUTION
				
				As such this gives a quadratic cost function like so which we must
				minimise.
				
				TODO: QUADRATIC COST EQN
				
				To minimise the function we differentiate and solve using simple
				symbolic manipulation.
				
				TODO: QUADRATIC COST SOLUTION
				
				Unfortunately, quadratic placement doesn't contain any congestion
				relief by default so various schemes exist. For example, extra anchor
				nodes are inserted which gently pull the circuit components apart from
				each other. As a result, the algorithm generally proceeds by iterating,
				regenerating anchors each time.
				
				Other non-quadratic analytical methods exist too with numerical
				solutions. The approaches are often similar.
			
			\subsubsection{Hierarchical clustering}
				
				Many placement algorithms scale super-linearly with problem size and so
				larger problems become increasingly problematic to handle. To solve
				this problem clustering techniques are first applied to first simplify
				the placement problem. A solution is then found at the coarse level and
				then hierarchically fleshed out.
				
				Various clustering algorithms are in use.
				
				TODO: TALK ABOUT CLUSTERING IN PLACEMENT...
				
				TODO: DESCRIBE THE ALGORITHM I IMPLEMENTED.
	
	\section{Application placement by simulated annealing}
		
		\label{sec:placement-by-annealing}	
		
		I have implemented a simplified SA based application placement algorithm
		based on the approach used in the popular VPR place and route tool chain.
		The algorithm is written in C and is optimised for experimentation rather
		than performance but is production-ready. It has been integrated into the
		`Rig' SpiNNaker software tools and has been used to place very large
		simulations. More on that later.
		
		\subsection{Representation}
			
			Model each chip as having a quantity of various resources (e.g. Cores,
			SDRAM) available. The application graph consists of vertices which each
			consume some quantity of these resources. Vertices must be placed on a
			single chip such that the resources required on a given chip do not
			exceed those available. Vertices are then interconnected by 1:N nets with
			weights which act as hints. The nets are treated as a soft constraint:
			vertices connected via a net will, ideally, be placed near to each other,
			with priority being given to nets with higher weights. Additionally there
			will be a list of placement constraints (see later).
			
			A key observation is that while vertices in an application may frequently
			have a 1:1 correspondence with application cores, this need-not be the
			case. For example, a vertex may represent a block of SDRAM which is
			shared. A vertex may also represent some other resource, for example,
			external IO availability. By making these resource types user-defined,
			applications programmers can express flexible hard-constraints on their
			application.
			
			Another observation is that generic soft constraints can be expressed may
			be expressed using a net with an appropriate weight.
			
			As a result of these facilities, application programmers can easily
			express their own application-specific hard and soft placement
			constraints without having to modify the algorithm. This representation
			has become a de-facto standard for placement problem interchange for
			SpiNNaker applications.
		
		\subsection{Cost function}
			
			At present I've used HPWL despite this being really bad for high-fan-out
			multicast and totally ignorant to the hexagonal nature of SpiNNaker...
			
			To compute bounding boxes for tori I use the following approach. For each
			dimension, sort the points on that dimension and find the largest gap
			between them on a ring. The bounding box goes the other way.
			
			TODO: FIGURE ILLUSTRATING BOUNDING BOX COMPUTATION FOR TORI.
		
		\subsection{Annealing schedule}
			
			The annealing schedule is that used by VPR. Despite being for circuit
			placement, it seems to work jolly well.
			
			TODO: DESCRIBE AND RATIONALISE THE SCHEDULE
		
		\subsection{Constraint handling}
			
			Various hard and soft constraints may be expressed by software
			approaches. For each we explain how they may be handled by the placement
			algorithm:
			
			\subsubsection{Location Constraint}
				
				The vertex is placed on a chip and removed from the set of movement
				candidates.
			
			\subsubsection{Same-chip constraint}
				
				When two vertices must always be placed on the same chip they are
				simply combined into one vertex which consumes the sum of their
				resources. Placement then treats them as one chip and thus is forced to
				atomically place the vertices.
			
			\subsubsection{Reserve resource constraint}
				
				Simply reduce resource availability on that chip.
			
			\subsubsection{Keep near Ethernet}
				
				Simply add a net.
	
	\section{Evaluation}
		
		\label{sec:placement-results}
		
		Though benchmarks exist for super computer loads and chip placement tasks,
		such things don't exist for neural applications. As a result I use a
		selection of real applications for SpiNNaker along with some synthetic
		benchmarks based on biological data.
		
		\subsection{Benchmark networks}
			
			First some real networks.
			
			Some nengo networks: SPAUN: `The world's largest functional brain model'.
			Word-net network from Jamie: Example of some learning.
			
			TODO: DESCRIBE SHAPE OF NENGO NETWORKS
			
			Some PyNN networks: Microcortical column model from PyNN. Note almost
			broadcast connectivity but varying weights. Try and extract a vision
			netlist from Anna. Maybe try and get a netlist for Tom's barrel cortex.
			
			Now for some artificial networks. Pipeline, noisy pipeline, mesh,
			Gaussian 2D.
		
		\subsection{Experiments}
			
			We compare random, linear, greedy and annealing based placement
			approaches to placement. We compare static metrics (such as mean/max
			congestion, table usage) along with experiments based on simulated
			network traffic in real hardware. Network Tester generates artificial
			traffic in proportion with the weights given for each model. We compare
			the relative level of traffic sustainable. We also consider use of
			machines of various sizes.
		
		\subsection{Results}
			
			SA placement is slow but rather effective, especially for some networks.
			Generally worth doing. Will need to be sped up for very large machines...
			
			TODO: EXPERIMENTS!
	

	\chapter{Discussion}

\section{Suitability of the hexagonal torus topology}
	\subsection{Physical scalability}
	\subsection{Routability}
	\subsection{Placeability}

\section{Suitability of the SpiNNaker router}
	\subsection{Deadlock avoidance}
	\subsection{Routing table size}

\section{Suitability of circuit placers for application placement}
	\subsection{Quality}
	\subsection{Runtime}
	\subsection{Routing resources}
	\subsection{Flexibility}
	\subsection{Scalability}


	\chapter{Future research}
	
	In this thesis I have presented a number of new techniques which have made it
	possible to assemble and operate the SpiNNaker super computer. This work
	opens up a range of possibie lines of research to extend this work to future
	architectures and applications. In this chapter I focus on two anticipated
	challenges of future systems: growing scale and greater dynamicism in
	applications.
	
	\section{Scaling up}
		
		TODO: INTRO
		
		\subsection{Grid machine room layouts}
			
			In chapter XXX, I developed a machine room layout for hexagonal torus
			topologies which allowed machines occupying a row of standard
			machine-room cabinets to scale up without the need for long
			interconnecting cables. For larger installations, however, it will be
			necessary to employ multiple rows of cabinets in a 2D arrangement.
		
		\subsection{Routing congestion control}
		
		\subsection{Parallel place and route}
	
	\section{Structural plasticity and dynamic fault-tolerance}
		\subsection{Plasticity models}
		\subsection{Incremental placement}
		\subsection{Incremental routing}
		\subsection{Hot-spare routes}

	\chapter{Conclusions and future research}
	
	The SpiNNaker architecture was designed to tackle the challenges presented by
	the simulation of biologically realistic neural networks. One of its
	distinguishing features is its network architecture which employs both an
	unconventional network topology and multicast router architecture. The
	hexagonal torus topology used by SpiNNaker was chosen to enable greater
	performance while maintaining ease of construction and scalability compared
	with conventional network topologies. SpiNNaker's router design centres
	around packets mimicking the neural `spike' signals they are designed to
	convey by being small, multicast and not guaranteed to arrive at their
	destination.  This novel design, though largely complete before this work
	began, left a number of open problems which this thesis has attempted to
	address.
	
	In this concluding chapter I begin by summarising the answers to the research
	questions raised in chapter~\ref{sec:introduction}. This is followed by a
	discussion of new research topics which have been uncovered by this work.
	
	\section{Answers to research questions}
		
		Each of the three research questions are answered below.
		
		\subsubsection{1. Can the hexagonal torus topology be deployed and used in
		real, large-scale systems?}
		
		In chapter~\ref{sec:building}, I introduced a cabling scheme and assembly
		technique which has been used successfully to build a prototype SpiNNaker
		system with over half a million processor cores using the hexagonal torus
		topology. The techniques shown are expected to enable a final SpiNNaker
		machine of double this size to be built, filling a six metre long row of
		machine-room cabinets.
		
		Though SpiNNaker's processor-count places it amongst some of the world's
		largest supercomputers (see figure \ref{fig:top500-num-processors} on page
		\pageref{fig:top500-num-processors}), it is comparatively compact, filling
		one row of cabinets compared with the warehouse-scale installations found
		in commercial systems. In spite of this, the folding and interleaving
		techniques described allow hexagonal torus topologies to scale to
		arbitrarily large installations without cables which span the machine.
		
		Chapter~\ref{sec:shortestPaths} described an efficient and general
		technique for finding, and enumerating shortest path vectors in hexagonal
		torus topologies. These developments bring the hexagonal torus topology in
		line with other topologies by enabling routing algorithms to exploit all
		possible paths in a network. Further, chapter~\ref{sec:placement}
		demonstrated that placement algorithms are also adaptable to hexagonal
		torus topologies thanks to their similarity to 2D toruses.
		
		Though, as this thesis highlights, hexagonal toruses lack many of the
		intuitive properties enjoyed by other topologies, it is still possible to
		reason about them with only limited computational effort.  Now that the
		practicality and scalability of the topology has also been demonstrated in
		practice, it represents a credible option for future network architectures.
		
		\subsubsection{2. Does SpiNNaker's router architecture help, or hinder
		fault tolerance?}
		
		SpiNNaker's unconventional use of packet dropping to avoid deadlocks
		greatly simplifies the router architecture, part of the motivation for this
		design. In chapter~\ref{sec:routing} this feature is used to the advantage
		of PGS repair to add fault tolerance to existing routing algorithms.
		Compared with the often complex and wasteful methods used to tolerate
		faults in other networks, PGS repair incurs very little performance
		overhead in the presence of static faults.
		
		Routing table usage does increase in the presence of faults, however, which
		may be a concern for applications which already require many routing table
		entries. Routing table usage, as well as other overheads, were most
		significantly increased in the presence of contiguous groups of network
		faults. This is because the PGS repair algorithm produces routes which pass
		tightly around the corners of faults, resulting in concentrations of
		routing table entries in those areas.  Though the symptoms of this problem
		can be attributed to the design of SpiNNaker's multicast routing mechanism,
		the responsibility lies with the behaviour of the PGS repair algorithm.
		Potential improvements to the PGS repair algorithm are discussed later in
		\S\ref{sec:pgs-repair-improvements}.
		
		The overall answer to this research question, therefore, is that the
		flexibility provided to routing algorithms in SpiNNaker's architecture is
		of great benefit, enabling arbitrary fault patterns to be inexpensively
		avoided.
		
		\subsubsection{3. How can the parts of a neural simulation be placed onto a
		large hexagonal torus topology to reduce network load?}
		
		In chapter~\ref{sec:placement}, I explored a number of contemporary
		approaches to the problem of placing applications with irregular
		communication patterns into network topologies. I observed that researchers
		working on circuit placement for chips and FPGAs are tackling similar
		problems and working at scales as large, or larger than, those faced in
		application placement. Based on this I developed a
		simulated annealing based placement algorithm inspired by the techniques
		used in circuit placement, with specific adaptations for use in application
		placement and SpiNNaker's network topology.
		
		The simulated annealing based placement algorithm consistently outperforms
		pre-existing placement algorithms included in benchmarks in terms of
		placement quality.  In the case of one benchmark, simulated annealing based
		placement made it possible to run that neural simulation in real-time for
		the first time.  At larger scales, simulated annealing was also found to be
		able to produce good quality placements in benchmarks containing over one
		million processes -- the largest size supported by the SpiNNaker
		architecture.
		
		The major shortcoming of simulated annealing based placement is its
		execution speed. Though its execution time grows in proportion to the size
		of the problem, the implementation used took over 12~hours to place a
		synthetic problem for the largest planned SpiNNaker machine. Though
		tractable -- particularly given the relative output quality compared with
		the prior state-of-the-art -- the algorithm is unlikely to function
		comfortably as-is on larger problems.
		
		The conclusion to be drawn from this result, however, is not just that
		simulated annealing is a good solution for today's placement problems but
		that circuit placement techniques in general could be successfully adapted
		to fulfil this role. The placement problems faced by chip designers are
		growing at roughly the same exponential rate as the size of super computers
		but circuit designs hold the lead in terms of problem size. Consequently,
		as approaches are retired by chip placement researchers, they may find new
		life in the field of application placement.
		
	\section{Future research}
		
		Though the goals of this study have largely been met, there also remain
		some important limitations which future work may hope to address.
		Furthermore, this work has uncovered a number of new research areas
		warranting future enquiry. This section outlines a number of future lines
		of research.
		
		\subsection{Warehouse-scale cabling}
			
			In chapter~\ref{sec:building} I developed and implemented a number of
			cabling schemes for the SpiNNaker architecture spanning up to a six metre
			row of machine-room cabinets -- a relatively small installation by
			current standards. In SpiNNaker, the cabling exists in a 2D plane (i.e.
			across the faces of the cabinets) but as the system is scaled up, a
			single row of cabinets will tend towards a 1D line. Since embedding a 2D
			structure in a 1D space necessarily results in long connections, this
			cannot scale indefinitely.
			
			\begin{figure}
				\center
				\buildfig{figures/multi-row-cabling.tex}
				
				\caption{Multiple rows of interconnected cabinets.}
				\label{fig:multi-row-cabling}
			\end{figure}
			
			In conventional large-scale super computer installations, nodes are
			installed in rows of cabinets as illustrated in
			figure~\ref{fig:multi-row-cabling}.  From a `bird's-eye' view, the system
			approximates a 2D space, spread across the floor of a machine-room.
			Therefore, in principle, the folding and interleaving techniques
			described in chapter~\ref{sec:building} still apply. Unfortunately for
			SpiNNaker, cables connecting between rows of cabinets would be longer
			than the one metre limit imposed by its hardware because of the spacing
			between rows of cabinets.  Future SpiNNaker systems will need to consider
			alternative link technologies.  For example, a hybrid system could be
			used in which intra-cabinet connections continue to use the current HSS
			link technology while inter-cabinet links might use optical connections.
			This type of architecture could be supported by the use of pluggable
			`SFP+' transceiver modules~\cite{sff01}.
		
		\subsection{Cabling assistance for other architectures}
			
			A secondary result of the construction of prototype SpiNNaker systems in
			chapter~\ref{sec:building} was the use of real-time guidance and feedback
			to assist cable installation. I am not aware of this technique's use by
			existing architectures and, following the success experienced in this
			project, it is possible that the technique may also be useful in
			conventional systems.
			
			During the construction of prototype SpiNNaker machines, each cable took
			seconds to install compared with the minutes reported for existing
			systems~\cite{mudigonda11}. Part of this increase in efficiency appears
			to result from the immediate identification of mistakes made during
			cabling, saving time-consuming backtracking later on.
			
			In many real-world network installations, units are less densely packed
			than in SpiNNaker and so longer cables are often required. As a
			consequence, cabling errors may become more likely as cabling patterns
			are spread over a larger area making them more difficult to visually
			verify. Like SpiNNaker, conventional networking hardware is often
			equipped with a generous range of indicator LEDs and diagnostic
			facilities which might be used to implement real-time installation
			guidance. Future work could explore the use of this technique in the
			construction of other large-scale networks, such as data centres.
		
		\subsection{Congestion mitigation}
			
			\label{sec:wiggly-board-allocations}
			
			In chapter~\ref{sec:routing} I found that contiguous network faults cause
			hot-spots of congestion and routing table depletion where the PGS repair
			algorithm routed many paths around the edges of faults.  However, it is
			not just faults which can cause contiguous blockages in the network
			topology. In reality, researchers do not always require a full-sized
			SpiNNaker system to perform their experiments so large SpiNNaker systems
			are soft-partitioned on demand into many smaller
			machines~\cite{spalloc16}. To ensure isolation between partitioned
			sub-machines, HSS links between boards in different partitions are
			disabled. Because of SpiNNaker's `wrapped triple' partitioning scheme,
			the resulting sub-machines have hexagonal \emph{mesh} topologies (i.e.
			without wrap-around links) with irregular boundaries as in
			figure~\ref{fig:spalloc-mesh}.
			
			\begin{figure}
				\center
				\buildfig{figures/spalloc-mesh.tex}
				
				\caption[Irregular edges of a partitioned SpiNNaker system.]%
				{Irregular edges in a SpiNNaker system comprised of 24~boards
				partitioned from a larger machine.  Each hexagon represents a SpiNNaker
				chip. No wrap-around connections are present.}
				\label{fig:spalloc-mesh}
			\end{figure}
			
			In partitioned systems, the `tooth'-like gaps on the periphery of the
			network result in similar congestion to the HSS link failures considered
			in chapter~\ref{sec:routing}. When a route is generated between nodes on
			opposite sides of a gap, the PGS repair process will produce a
			shortest-path route around it. Since many routes may be blocked by a
			single gap, a hot-spot may develop around the corners of the gap.
			
			In chapter~\ref{sec:placement}, the `CConv' benchmark application was
			found to run correctly the majority of the time when placed by the
			simulated annealing algorithm but would occasionally fail by a
			significant margin. Preliminary experiments suggest these occasional
			failures are caused by placement solutions which place heavily
			communicating parts of the application on opposite sides of gaps along
			the perimeter of the network. Two possible approaches which future work
			may consider are presented below.
			
			\subsubsection{Avoiding hotspots with PGS repair}
				
				\label{sec:pgs-repair-improvements}	
				
				Network congestion around faults and network irregularities could be
				reduced by forcing the PGS repair process to take more varied routes
				around faults. For example, in circuit routing algorithms, routers
				avoid congestion by increasing the cost of routes which pass through
				congested areas~\cite{kahng11}. A similar technique could be used in
				PGS repair to spread the routes it produces.
				
				An alternative approach would be to adapt the base routing algorithms
				used prior to PGS repair to, for example, attempt alternative dimension
				order routes which may avoid blockages due to faulty links.
			
			\subsubsection{Fault and irregularity aware placement}
				
				One of the shortcomings of the simulated annealing based placer
				developed in chapter~\ref{sec:placement} is that it does not account
				for network faults, or irregularities, when estimating the cost of
				placement solutions.  Future work may exploit techniques used in
				congestion-aware circuit placement which could be adapted for
				application placement~\cite{viswanathan07}.
		
		\subsection{Reducing placement execution time}
			
			The simulated annealing based placer presented in
			chapter~\ref{sec:placement} produced good quality placements but its
			execution time limits its use beyond one million vertex placement
			problems. Future work should explore possibilities for improving the
			performance and scalability of this technique.
			
			In addition to considering alternative placement algorithms based on
			other methods, one possible approach is to attempt to reduce the execution
			time of simulated annealing based placement by shrinking the application
			graph being placed.
			
			For example, graph clustering~\cite{schaeffer07} may be used to group
			together strongly connected vertices which would then be placed as a
			single unit.  Unfortunately, clustering can suffer from the same problems
			as graph-partitioning-based placement: vertices may be grouped together
			in ways which, in practice, cannot be packed together into a given portion
			of a machine.  A possible solution to this problem is to use a two-phase
			placement approach~\cite{kahng11}. In a `global' placement phase,
			solutions are permitted which can slightly over-allocate resources but
			overall achieve good placement quality. In the `detailed' placement phase
			which follows, the solution is `legalised' by making small changes to the
			global placement to eliminate over allocation.
			
			An alternative approach suited to SpiNNaker could be to limit the
			clustering process to clusters which fit on a single SpiNNaker chip. In
			typical SpiNNaker application graphs, clustering to this level may reduce
			placement problem sizes by an order of magnitude and, consequently,
			reduce execution times by the same ratio. Preliminary experiments suggest
			that this approach might result in little placement quality loss for
			large placement problems whilst substantially reducing overall execution
			time.
		
		\subsection{Benchmarking}
			
			One of the most significant limitations of this study has been the
			unavailability of large-scale SpiNNaker applications for use as
			benchmarks. As a consequence, much of the scalability experimentation
			performed has relied on simple synthetic benchmarks based on projections
			of future application behaviour.
			
			In the short term, more sophisticated synthetic benchmark generation
			techniques used by the circuit placement community~\cite{nam07} may offer
			alternative benchmarks for future work. In the longer term, however, it
			is hoped that the availability of large SpiNNaker systems -- and
			placement and routing algorithms better suited to exploit them -- will
			lead to larger scale applications being developed. Hopefully these
			applications will lead to more interesting and representative benchmarks
			for use in future work.
	
	\section{Closing remarks}
		
		One of the primary outcomes of this work is that a number of the practical
		challenges faced in scaling up the SpiNNaker architecture have been
		addressed leading to the construction of large-scale SpiNNaker machines.
		The development of an effective placement algorithm for SpiNNaker
		applications has been shown to enable some neural simulations to exploit
		SpiNNaker's architecture for the first time. The availability of larger
		SpiNNaker machines paves the way for future large-scale neural modelling
		work built on much larger models such as Spaun, `the world's largest
		functional brain model'~\cite{eliasmith12}.
		
		Beyond the SpiNNaker project, the hexagonal torus topology has also been
		validated as a scalable and practical candidate for future network
		architectures. As super computers become ever larger, the physical
		scalability afforded by the 2D nature of the hexagonal torus topology may
		make it a compelling option. In addition, the finding that circuit
		placement techniques can be adapted to support placement of SpiNNaker
		software indicates that these algorithms may also be applicable to other
		applications. Indeed, if this is the case, circuit placement may offer a
		long-term source of placement algorithms able to handle the demands of
		future applications.
		
		% This thesis has explored and tackled a number of the challenges posed in
		% scaling up the unconventional SpiNNaker architecture. Along the way I have
		% demonstrated that the hexagonal torus topology may be a practical choice in
		% future applications which can scale up to the physical dimensions expected
		% of future super computers. I have also developed new efficient and
		% effective methods of placing and routing neural simulation software on
		% SpiNNaker which -- it is hoped -- will enable a new generation of large
		% scale neural simulations on spinnaker.
		
		Although this work stops short of demonstrating truly large-scale
		neuroscientific simulations running at the scale of newly completed
		SpiNNaker machines (largely because such simulations do not yet exist) a
		number of smaller-scale neural simulations have been made possible for the
		first time. The algorithms and techniques devised in this work have
		subsequently been incorporated into various software libraries and tools
		now being used by researchers building SpiNNaker applications, vindicating
		the efforts of this thesis (see appendix~\ref{sec:software}). A successor
		to the SpiNNaker architecture is also in the early stages of design and is
		building on experience of the existing architecture. The current intention
		is to retain the hexagonal torus topology used by SpiNNaker, a decision
		supported by the findings of this thesis.
		
		With SpiNNaker's hardware architecture now operating at scales close to its
		architectural limits, it is hoped that the contributions of this work will
		enable researchers to develop larger and more detailed neural models for
		this unique architecture.

	
	% Bibliography
	\bibliography{references}
	\bibliographystyle{alpha}
	
\end{document}
words.
	
	\clearpage
	\listoffigures
	
	\clearpage
	\listoftables
	
	% Abstract
	{
	\prefacesection{Abstract}
	
	% Single line spacing for the abstract page
	\setstretch{1.0}
	
	
	\vfill
	
	% Standard thesis information
	\begin{center}
		\textsc{\large\thesistitle}
		
		\vspace{0.5em}
		
		\thesisauthor
		
		\vspace{0.5em}
		
		A thesis submitted to the University of Manchester\\
		for the degree of Doctor of Philosophy, 2016
	\end{center}
	
	\vfill
	
	% The abstract
	
	SpiNNaker is an unconventional super computer architecture designed to
	simulate up to one billion biologically realistic neurons in real-time. To
	achieve this goal, SpiNNaker employs a novel network architecture which poses
	a number of practical problems in scaling up from desktop prototypes to
	machine room filling installations.
	
	SpiNNaker's hexagonal torus network topology has received mostly theoretical
	treatment in the literature. This thesis tackles some of the challenges
	encountered when building `real-world' systems.  Firstly, a scheme is devised
	for physically laying out hexagonal torus topologies in machine rooms which
	avoids long cables; this is demonstrated on a half-million core SpiNNaker
	prototype.  Secondly, to improve the performance of existing routing
	algorithms, a more efficient process is proposed for finding (logically)
	short paths through hexagonal torus topologies. This is complemented by a
	formula which provides routing algorithms greater flexibility when finding
	paths, potentially resulting in a more balanced network utilisation.
	
	The scale of SpiNNaker's network and the models intended for it also present
	their own challenges. Placement and routing algorithms are developed which
	assign processes to nodes and generate paths through SpiNNaker's network.
	These algorithms reduce congestion and tolerate network faults. The proposed
	placement algorithm is inspired by techniques used in chip design and is
	shown to enable larger applications to run on SpiNNaker -- with good
	performance -- than the previous state-of-the-art. Likewise the routing
	algorithm developed is able to tolerate network faults, inevitably present in
	large scale systems, with little performance overhead.
	
	
	% Required to ensure single line spacing is used for this whole block
	\par%
}

	
	% Declaration of non-submission elsewhere
	\prefacesection{Declaration}

% Single line spacing for the declaration
{
	\setstretch{1.0}
	No portion of the work referred to in this thesis has been submitted in support
	of an application for another degree or qualification of this or any other
	university or other institute of learning.
	
	\par%
}


	
	% University-prescribed copyright statement...
	\input{copyright}
	
	% Acknowledgements
	{
	\prefacesection{Acknowledgements}
	
	% Single line spacing
	\setstretch{1.0}
	
	It is often said that it is not \emph{what} you know but \emph{who} you know.
	Throughout the course of my PhD I've been exceptionally lucky to have been
	helped along by a great number of people.
	
	Both my supervisor, Jim Garside, and co-supervisor, Steve Furber, have each
	spent countless hours patiently discussing and describing all manner of
	things with me while giving me great freedom in my project. Jim's office door
	has always been open to my unexpected interruptions be it about work, writing
	or walking.  Likewise, Steve has always managed to find time for both
	technical and frivolous endeavours alike. I'm also hugely grateful to Luis
	Plana who has been a rich source of sage advice, insightful questions
	patiently suffered many a foolish question.
	
	Various parts of the work in this thesis (and numerous side projects) would
	not have been possible if not for the multitude of discussions,
	collaborations and even sheer physical hard work of Steve Temple, Javier
	Navaridas, Simon Davidson and Dave Clark. I'm also indebted to Andrew Mundy
	and Jamie Knight, both of whom have donated so much time and effort towards
	verifying and using software implementations of the ideas in this thesis.
	
	The injection of lunchtime silliness by Andrew and Jamie, along with Amanieu
	d'Antras and Andrew Webb and the other CDT members has always brightened my
	day. So to has the friendly and stimulating environment of the School of
	Computer Science and its many staff and students. Of course, I am also very
	grateful for the funding the school has provided for my research.
	
	I cannot thank my wonderful wife, Ann-Marie, enough for being by my side. She
	has given me so much kindness, love and patience and endured a lifetime's
	quota of conversations about hexagons. Finally, thanks too to rest of my
	family, especially my parents, who are to blame for starting me down this
	path and co-suffering drafts and endless rants about this document.
	
	% Required to ensure single line spacing is used for this whole block
	\par%
}

	
	% Main body
	\chapter{Introduction}

\label{sec:introduction}

%Problem area
%
%* Network construction and exploitation
%  * Cabling: Build it cheaply in terms of tech cost and install cost
%  * Routing: Get around it cheaply and reliably
%  * Placement: Use it efficiently

The Spiking Neural Network Architecture (SpiNNaker) is a novel super computer
architecture designed to simulate biologically realistic models of brains in
real time \cite{furber07}. Though neurons, the building blocks of the brain,
are relatively well understood, their complex interactions remain mysterious.
Just as understanding the workings of a transistor is insufficient to
understand a modern microprocessor, neuroscientists believe that understanding
the neurons in isolation cannot explain the brain and that understanding their
connectivity is key \cite{eliasmith13,eliasmith14}. Experiments on real brains,
however, are fraught with difficulty. Variations between individuals can be
significant and it is only possible to record tens or hundreds of the trillions
of signals in the brain, and even then only with limited control over which
signals are recorded. Computer simulations of models of large neural networks,
however, enable researchers to develop repeatable experiments and gain complete
visibility of any signal and any neuron. Models such as SPAUN
\cite{eliasmith12}, built from millions of simulated neurons, have shown great
promise in expanding our understanding of higher level brain functions such as
memory and simple problem solving.  Unfortunately these neural models are
expensive to simulate, requiring hours of compute time to simulate each second
of neural activity. As well as being inconvenient, this precludes the use of
robotics to immerse these models in real world environments and also limits
studies of long-term behaviours such as learning.

SpiNNaker is designed to enable the real time simulation of models containing
up to one billion neurons -- approximately \SI{1}{\percent} of a human brain or
ten mouse brains \cite{furber06}. To achieve this goal, the largest planned
SpiNNaker machine will contain over one million low-powered computer processors
interconnected by a bespoke network architecture.

SpiNNaker's large processor count matches the current trend in super computers
where processor counts are growing exponentially \cite{meuer16j}, mimicking the
growth of the number of components in the processors themselves predicted by
Gordon Moore's famous `law' \cite{moore75}. As a result of this growth, the
interconnection networks which enable these processors to work together have
grown in importance \cite{dally04}.  Network designers must carefully balance
performance against practicality and financial cost.  SpiNNaker's network is no
exception to this rule and, as the systems scale up from desktop prototypes to
machine-room scale installations, the reality of building and exploiting these
machines presents an array of challenges.

As in all super computers, SpiNNaker's network interconnects its processors in
a particular network topology which defines how different processors may
communicate with each other. Unlike the tree and $N$-dimensional torus
topologies found in contemporary super computers \cite{dally04}, SpiNNaker
employs a `hexagonal torus topology'. In this topology, nodes in SpiNNaker's
network fit together in a honeycomb-like pattern where messages may `hop' from
node to node to reach their destination. As we will see in
chapter~\ref{sec:background}, the hexagonal torus topology, in theory, sits at
a `sweet spot' in terms of network performance and practicality. As the first
known large-scale installation of the hexagonal torus topology, however, there
remain a number of practical challenges for large spinnaker machines arising
from this choice.

As super computer networks have grown in scale to millions of processors the
task of dealing with previously rare faults has grown.  Though fault rates in
networks remain consistently low, architectures such as SpiNNaker may have
hundreds of thousands of links meaning even fault rates of a fraction of a
percent will impact tens or hundreds of links. To enable reliable operation,
networks must be able to adapt the routes taken by messages through the network
to avoid faulty links and nodes. The techniques employed are often closely tied
to a particular network architecture and consequently SpiNNaker's novel network
architecture demands its own approach.

Another challenge introduced by the growing scale of super computers is making
\emph{efficient} use of network resources. Communicating processes should be
located on logically `nearby' nodes to reduce network load. The neural models
for which SpiNNaker is designed are often described abstractly, rather than
geometrically, using modelling languages such as PyNN~\cite{davison08} and
Nengo~\cite{eliasmith04}.  Because of this, the communication requirements of
simulations can be highly irregular making an efficient placement of processes
onto processors in the machine non-trivial.

%Contributions
%
%* Cabling scheme for hexagonal toruses without long cables
%* Efficient installation technique for dense systems
%* Exhaustive and efficient route calculation in hex toruses
%* Fault tolerant routing scheme exploiting SpiNNaker's odd router
%* Placement based on SA a: works very well and b: suggests circuit placement is
%  a good source of inspiration.

This thesis addresses the practical challenges of scaling up the SpiNNaker
architecture in a real-world setting summarised by these research questions:

\begin{enumerate}
	
	\item Can the hexagonal torus topology be deployed and used in real, large
	scale systems?
	
	\item Does SpiNNaker's router architecture help, or hinder fault tolerance?
	
	\item How can the parts of a neural simulation be placed onto a large
	hexagonal torus topology to reduce network load?
	
\end{enumerate}

%Structure
%
%* Chapter 2: Background: detailed dive into what's in SpiNNaker, why its
%  really so unusual. Also looks at what applications run on SpiNNaker and how
%  they work.
%* Chapter 3: How to build a really big SpiNNaker machine.
%* Chapter 4: How to find your way around that machine.
%* Chapter 5: How to find your way around that machine even when its broken.
%* Chapter 6: Now you can walk, time to run.
%* Chapter 7: Wrapping up.
%* Appendices: Hard-to-come-by theoretical and practical details useful if
%  you're about to continue where this research left off but be useful but
%  otherwise hard to come by, especially in one place.

Chapter~\ref{sec:background} introduces the SpiNNaker architecture and, in
particular, describes its hexagonal torus topology and network architecture.

In chapter~\ref{sec:building}, I develop a cabling scheme for large hexagonal
torus topologies which enables arbitrarily large networks to be constructed
using only short, inexpensive cables. This theoretical work is then evaluated
through the construction of a range of prototype SpiNNaker systems. The largest
of these prototypes contains over half a million processor cores and spans
several machine room cabinets. In addition, I propose the use of built-in
diagnostic facilities to assist technicians performing network installation and
maintenance. This technique is found to greatly reduce the effort required and
the number of mistakes made.

In chapters~\ref{sec:shortestPaths}~and~\ref{sec:routing} I develop new routing
techniques for SpiNNaker's network. Chapter~\ref{sec:shortestPaths} develops a
new approach to finding the shortest paths through hexagonal torus topologies,
an integral part of many routing algorithms. This newly proposed approach is
cheaper to compute than the state of the art and, unlike previous efforts, is
able to discover all valid short paths through the topology. This theoretical
advance brings hexagonal torus topologies in line with conventional topologies
by providing routing algorithms with complete information about the paths
available to them. In chapter \ref{sec:routing} I propose a fault tolerant
routing algorithm for SpiNNaker which is able to avoid arbitrary static fault
patterns with minimal performance overhead. A key finding of this chapter is
that the flexibility afforded to fault tolerant routing algorithms by
SpiNNaker's unconventional router architecture is what facilities the low
overheads reported in this chapter.

Finally, in chapter~\ref{sec:placement}, I explore the problem of application
placement in SpiNNaker's network. As in other networks and applications, neural
simulations should be arranged such that communication occurs primarily between
processors close together in the network to control network load. Due to the
irregular connectivity and large scale of the neural models expected to run on
SpiNNaker, an automated approach is necessary. I develop a novel placement
algorithm based on algorithms used for circuit layout in computer chips. My
algorithm is found to allow some larger neural models to run on SpiNNaker for
the first time while enabling other applications to run at greater speeds. In
addition, synthetic benchmarks containing over one million processes indicate
that this algorithm should handle the anticipated demands of the neural models
expected to run on large-scale SpiNNaker installations.

	\chapter{The SpiNNaker Architecture}
	
	\label{sec:background}
	
	SpiNNaker is a massively parallel computer architecture designed to simulate
	biologically realistic neural models \cite{furber07}. In this chapter we will
	explore this unconventional architecture in detail, starting with its purpose
	before focusing on its most unconventional feature: its network.
	
	% * Purpose
	%   * Spiking neural simulations
	%     * Neural modelling: PyNN, Nengo...
	%     * Parallelisation + communication
	
	\section{Neural simulation}
		
		Human brains contain billions of neurons connected together by trillions of
		`synapses'. Neurons communicate by transmitting and receiving `spikes'
		through their synapses. Each spike is `valueless' in that a spike's only
		significant features are when it arrives and where it has come from.
		
		\begin{figure}
			\center
			\buildfig{figures/lif-neuron.tex}
			
			\caption{A Leaky Integrate-and-Fire (LIF) neuron.}
			\label{fig:lif-neuron}
		\end{figure}
		
		Though some detailed models of the electrochemical processes occurring
		inside neurons are computationally intensive, simplified models such as the
		Leaky Integrate-and-Fire (LIF) model can be implemented in just a handful
		of CPU instructions \cite{vainbrand11}. Figure~\ref{fig:lif-neuron}
		illustrates a simple LIF neuron in which incoming spikes cause charge to
		build up (integrated) which over time, leaks away. If an incoming spike
		causes the charge to rise above a certain threshold, the neuron `fires'
		producing an outgoing spike. Despite the simplicity of this model, large
		neural networks such as Spaun \cite{eliasmith12} -- built entirely from LIF
		neurons -- exhibit complex behaviours such as fine motor control and
		problem solving.
		
		The computational expense of large scale neural simulations does not arise
		from the cost of modelling neurons but instead from distributing spikes. In
		biology, neurons produce spikes at an average rate of \SI{10}{\hertz} and
		synapses connect each neuron's output to (order) \num{1000}~neurons
		\cite{navaridas09}. Consider an example neural model with $7\times10^7$
		neurons, approximately the number in a house mouse and
		$\nicefrac{1}{10}^\textrm{th}$ of the design target of SpiNNaker. This
		network might produce $7\times10^8$~spikes per second. Because each neuron
		connects to many others, this equates to $7\times10^{11}$ spikes being
		received per second. If each spike were transmitted as a UDP datagram
		containing a single \SI{32}{\bit} payload, the total network throughput
		required for this simulation would be \SI{179.2}{\tera\bit\per\second}. At
		the time of writing, this is more than double the bisection bandwidth (the
		theoretical worst-case throughput) of the world's most powerful super
		computer \cite{dongarra16}.
	
	\section{Network architecture}
		
		Architectures such as IBM's Blue Gene \cite{chiu11} and Cray's XK7
		\cite{ornl16} employ powerful compute nodes connected together using
		networks designed to transfer multi-kilobyte blocks of data between nodes.
		Since neural models have relatively light computational requirements and
		communications are based on small pieces of data (spikes), this type of
		architecture is poorly suited to the task.
		
		SpiNNaker's architectural target is to support realtime simulations of up
		to one billion neurons. Since neural models such as LIF are inexpensive to
		model and many neurons can be simulated independently in parallel,
		SpiNNaker employs many small, energy efficient ARM processors
		\cite{furber07}. To support the unusual communication requirements of
		neural simulations, a bespoke interconnection network is used which is the
		background to this thesis.
		
	%   * SpiNNaker chip
	%     * Cores
	%     * SDRAM
	%     * NoC
	%     * Router
		
		\begin{figure}
			\center
			%\includegraphics[width=19mm]{figures/spinnakerChip.jpg}
			\buildfig{figures/hex-chips.tex}
			
			\caption[SpiNNaker chips connected to their six neighbours.]%
			{SpiNNaker chips (actual size) connected to their six neighbours.}
			\label{fig:spinnakerChip}
		\end{figure}
		
		The fundamental building block of the SpiNNaker architecture is the
		SpiNNaker chip (figure \ref{fig:spinnakerChip}) \cite{furber13}. Each chip
		contains eighteen low power ARM 968 processor cores each capable of
		simulating between \num{200} and \num{2000} LIF neurons in real time
		\cite{mundy15}.  Each core has a total of \SI{96}{\kilo\byte} of private
		Tightly-Coupled Memory (TCM) and shares access to \SI{128}{\mega\byte} of
		on-chip SDRAM with other cores on the same chip. Finally, each chip
		contains a programmable router which routes network packets to and from the
		local cores and six neighbouring SpiNNaker chips. SpiNNaker machines are
		constructed by combining many SpiNNaker chips.
		
		\begin{figure}
			\center
			\buildfig{figures/spinnaker-packet.tex}
			
			\caption{SpiNNaker's \SI{40}{\bit} and \SI{72}{\bit} multicast packet
			format.}
			\label{fig:spinnaker-packet}
		\end{figure}
		
		Processor cores can communicate by sending and receiving network packets
		forwarded by routers through the network. Since SpiNNaker's network is
		designed to transmit neural spike events efficiently, individual network
		packets are small, either \SI{40}{\bit} or \SI{72}{\bit} compared with tens
		or hundreds of byte packets in typical network architectures.
		
		In a real-time simulation, the time at which a spike is produced is
		implicitly indicated by the time it is received -- since at biological
		timescales a computer network delivers packets `instantaneously'.
		Consequently, the only information which must be explicitly encoded is the
		identity of the neuron which produced the spike. In SpiNNaker, a spike may
		be encoded by using a single \SI{40}{\bit} `multicast packet' whose format
		is illustrated in figure~\ref{fig:spinnaker-packet}.  The \SI{8}{\bit}
		header is used by SpiNNaker's routers to determine the type of packet and
		the \SI{32}{\bit} `routing key' is used to identify the neuron which
		produced the packet. The routing key is also used by SpiNNaker's routers to
		determine how the packet should be directed through the network.
		
		The optional \SI{32}{\bit} payload is not used by conventional spiking
		neural simulations \cite{galluppi10} but has been exploited to enable more
		efficient simulation of a particular class of neural models \cite{mundy15}.
	
	\section{The SpiNNaker router}
		
		The SpiNNaker router employs an unconventional design which, despite its
		compact size and small energy requirements, implements a flexible multicast
		routing scheme. Unlike conventional routers which often employ hard-coded
		routing rules \cite[chapter~8]{dally04}, the SpiNNaker router uses a
		programmable `routing table' to determine how packets should be forwarded.
		In addition, to avoid deadlocks, SpiNNaker's router employs a simple,
		timeout-based mechanism which exploits the ability of neural networks to
		tolerate occasional missing packets. As we will see in chapter
		\ref{sec:routing}, this mechanism greatly simplifies the task of routing in
		SpiNNaker's network. In this section we'll look at these features in
		greater detail.
		
		\subsection{Routing tables}
		
			When a multicast packet arrives at a SpiNNaker router (either from a
			local core or a neighbouring chip), the router looks up the routing key
			in its routing table. This table consists of \num{1024} programmable
			table entries, each specifying a routing key bit pattern and mask to
			match and a set of routes.  When a multicast packet's key is matched by a
			routing entry the packet is forwarded along every route specified by that
			entry, potentially duplicating the packet. This `multicast' technique
			allows packets to be transmitted once but received in a number of places
			while making efficient use of the network \cite{navaridas12}.
			
			Though routing table entries are in finite supply (\num{1024} entries per
			router), it is still possible for many thousands of traffic flows to be
			routed through a single router. The bit pattern and mask in each routing
			entry matches against the 32~bits of a routing key as either
			`\texttt{1}', `\texttt{0}' or `\texttt{X}' (don't care).  This means that
			a single routing entry may, for example, be used to match all routing
			keys with a certain prefix. If a routing key is not matched by any entry
			in the routing table then the packet is `default routed' in a straight
			line. For example if a packet with an unmatched key is received from the
			chip to the left, the packet will be default routed to the chip on the
			right. By assigning routing keys such that neurons whose spikes are sent
			to similar destinations share a similar prefix, the number of routing
			entries required by a simulation is greatly reduced \cite{davies12}.
			
			\begin{figure}
				\center
				\buildfig{figures/routing-example.tex}
				
				\caption[Multicast routing example.]%
				{Multicast routing example with \SI{4}{\bit} routing keys. Each
				box represents a SpiNNaker chip whose router has been programmed with
				the routing entries shown. Grey lines mark connections between chips.}
				\label{fig:routing-example}
			\end{figure}
			
			Consider the simplified example in figure~\ref{fig:routing-example} in
			which a number of (\SI{4}{\bit}) routing table entries have been
			configured in the routers of a small SpiNNaker network. If a packet with
			the routing key \texttt{1011} is transmitted by a core in the chip
			labelled $(0, 0, 0)$, this will match the first routing table entry on
			that chip and will be routed to chip $(1, 0, 0)$. On chip $(1, 0, 0)$,
			the packet once again matches the first routing entry and is routed to
			chip $(1, 0, -1)$. On $(1, 0, -1)$, no match is made so the packet is
			default routed to $(1, 0, -2)$. On this chip, the packet matches a
			routing entry which routes the packet to core~7. In this example, default
			routing allows only three routing table entries to direct a packet
			through four chips.
			
			As a second example, if a packet with the routing key \texttt{0010} is
			transmitted by a core on chip $(0, 0, 0)$, this key will be matched by
			the second routing entry since \texttt{X}s in the table entry will match
			both \texttt{1}s and \texttt{0}s in the corresponding bits of the routing
			key. When the packet arrives at chip $(0, 0, -1)$ the matching routing
			entry forwards the packet to both $(0, 1, -1)$ and $(1, 0, -1)$
			simultaneously. The copy of the packet arriving at $(0, 1, -1)$ is routed
			to core~5 on that chip.  Meanwhile, the copy forwarded to $(1, 0, -1)$ is
			duplicated again with one copy being routed to core~11 and another being
			routed to chip $(1, 0, -2)$. Here the packet is finally delivered to
			core~6. In this example, the ability of the router to multicast
			(duplicate) packets as they pass through the network meant that sending
			one copy of the packet was sufficient to reach three destination cores.
			In addition, by using \texttt{X}s in the routing table entry, the same
			routing entries are sufficient to route packets with the keys
			\texttt{0000}, \texttt{0001}, \texttt{0010} and \texttt{0011}.
			
			In spite of these mechanisms, it is still possible for an application to
			run out of routing table entries. As we will see in
			chapter~\ref{sec:placement} by arranging applications appropriately
			within SpiNNaker's network, routing table usage can be reduced. In
			addition, other behaviours of SpiNNaker's router may be exploited to
			compress an applications routing tables further, however the techniques
			employed are beyond the scope of this thesis \cite{mundy16}.
		
		\subsection{Timeouts}
			
			SpiNNaker's router is built on a pipeline architecture. As shown in
			figure~\ref{fig:router-architecture}, the router is fed packets by an
			arbiter which serialises packets arriving from other chips and local
			cores. Every (\SI{100}{\mega\hertz}) clock cycle, the router pipeline
			accepts one packet from the arbiter and routes a packet to one or several
			output links. If any of the required output ports are busy then the
			packet is not forwarded to any output link and the pipeline stalls. Once
			a packet has been blocked for a programmable timeout, it is dropped
			(discarded) and routing continues as usual for next packet in the
			pipeline. Links become blocked while transmitting packets or waiting for
			the remote receiver to become ready. For example, a receiving processor
			core may be busy performing some computation or a receiving router may be
			blocked waiting for some of its outputs to become ready.
			
			\begin{figure}
				\center
				\buildfig{figures/router-architecture.tex}
				
				\caption{SpiNNaker router architecture}
				\label{fig:router-architecture}
			\end{figure}
			
			The timeout-based packet dropping mechanism is designed to defuse
			deadlocks in the network. For example, if two routers are trying to send
			each other a packet at the same time they may become deadlocked, each
			waiting for the other router to accept a packet before continuing.
			SpiNNaker's timeout mechanism breaks deadlocks by dropping packets which
			have been blocked for some time and therefore may be in a deadlock.  Once
			a packet has been dropped it is left to software to either tolerate the
			missing packet or trigger a retransmission. In neural simulations, as in
			biology, the loss of a single spike is unlikely to have a significant
			impact on the behaviour of a neural model and therefore these simulations
			are inherently tolerant of occasional dropped packets. During application
			loading and other system tasks, a higher level, software driven protocol
			based on acknowledgements and retransmissions is used to ensure
			guaranteed delivery.
			
			% TODO: MENTION TIMEOUT VALUE USED?
			% Router timeouts must be configured to be long enough that delays in
			% packet transmission, for example due to the time taken for packets to
			% traverse a link, do not trigger packet dropping. Conversely, the timeout
			% should be as short as possible to reduce the time the router is
			% blocked and maximise network throughput.
	
	\section{The hexagonal torus topology}
		
		Each SpiNNaker chip is a node in a `hexagonal torus topology' as
		illustrated in figure~\ref{fig:hexagonalTorusTopology}. Network packets
		sent by SpiNNaker's processor cores may `hop' through several nodes in the
		network to reach their intended destination. In each hop, a packet may
		advance one node along one of the three axes of the topology. For example,
		a packet sent by the node labelled $\alpha$ (in the bottom-left corner) to
		the node labelled $\beta$, might take the following sequence of hops:
		X$^+$, X$^+$, Z$^-$. Packets sent from $\alpha$ to $\gamma$ might take the
		route: X$^-$, X$^-$, Y$^+$, Y$^+$. The first hop of this route `wraps
		around' from the bottom-left node to the bottom-right node in a single hop.
		
		\begin{figure}
			\center
			\buildfig{figures/hexagonalTorusTopology.tex}
			
			\caption[A hexagonal torus topology.]%
			{A hexagonal torus topology. Each hexagon represents a node (i.e.
			a SpiNNaker chip). Touching nodes are directly connected. Nodes on edges
			$a$, $b$ and $c$ are also directly connected to the corresponding nodes
			on edges $a'$, $b'$ and $c'$, respectively. The three axes of the
			hexagonal torus topology, `X', `Y' and `Z' are also shown.}
			\label{fig:hexagonalTorusTopology}
		\end{figure}
		
		\begin{figure}
			\center
			\begin{subfigure}{0.39\linewidth}
				\center
				\includegraphics[width=\linewidth]{figures/torus-3d-flat.pdf}
				\caption{}
				\label{fig:torus-3d-flat}
			\end{subfigure}
			~~
			\begin{subfigure}{0.26\linewidth}
				\center
				\includegraphics[width=\linewidth]{figures/torus-3d-tube.pdf}
				\caption{}
				\label{fig:torus-3d-tube}
			\end{subfigure}
			~~
			\begin{subfigure}{0.23\linewidth}
				\center
				\includegraphics[width=\linewidth]{figures/torus-3d-torus.pdf}
				\caption{}
				\label{fig:torus-3d-torus}
			\end{subfigure}
			
			\caption{Visualisation of a hexagonal torus topology as a torus.}
			\label{fig:torus-3d}
		\end{figure}
		
		The wrap around connections in the topology are what give it the `torus'
		part of its name. Figure~\ref{fig:torus-3d-flat} shows a hexagonal torus
		topology drawn flat as in the previous figure. If the topology is rolled up
		into a tube such that the top and bottom nodes become directly adjacent, a
		tube is formed as in figure~\ref{fig:torus-3d-tube}. This tube can then be
		bent to bring together the nodes at the ends of the tube to form a torus as
		shown in figure~\ref{fig:torus-3d-torus}.
		
		A hexagonal torus topology is typically defined in terms of its width and
		height along the X and Y axes respectively. For example,
		figure~\ref{fig:hexagonalTorusTopology} shows a $10\times10$ hexagonal
		torus.  The nodes in a hexagonal torus topology are addressed using
		hexagonal coordinates of the form $(x, y, z)$ \cite{patel15}. The bottom
		left node (labelled $\alpha$ in the figure) has the coordinate $(0, 0, 0)$
		and other nodes are assigned coordinates according to the number of hops
		along each dimension from $(0, 0, 0)$, for example node $\beta$ has the
		coordinate $(2, 0, -1)$. Because the hexagonal torus topology's axes are
		non-orthogonal, it is possible to define several coordinates for the same
		location. For example $(3, 1, 0)$ and $(1, -1, -2)$ are also valid
		coordinates for node $\beta$. These dual coordinates emerge from the fact
		that adding $(1, 1, 1)$ to a coordinate produces an equivalent, but
		different, coordinate. This phenomenon is explained in detail in
		appendix~\ref{app:minimal-hex-coordinates} and related phenomena will be
		discussed in chapter~\ref{sec:shortestPaths}.
		
		The hexagonal torus topology was chosen over a more conventional network
		topology -- such as a 2D or 3D torus (sometimes known as a 2-ary $N$-cube
		or 3-ary $N$-cube respectively) \cite[chapters~3~and~5]{dally04} -- due to
		its balance of theoretical performance and practicality. The bisection
		bandwidth of a topology indicates the theoretical worst-case total
		throughput the network is able to sustain \cite[chapter~1]{dally04}.  In
		networks with homogeneous link throughput, bisection bandwidth is
		determined by the number of links cut by a balanced bisection of the
		network.  Figure~\ref{fig:bisection-bandwidth} illustrates the bisections
		of several torus topologies.
		
		\begin{figure}
			\center
			\begin{subfigure}[b]{0.3\linewidth}
				\center
				\buildfig{figures/bisection-bandwidth-2d.tex}
				
				\caption{2D Torus}
				\label{fig:bisection-bandwidth-2d}
			\end{subfigure}
			\begin{subfigure}[b]{0.3\linewidth}
				\center
				\buildfig{figures/bisection-bandwidth-hex.tex}
				
				\caption{Hexagonal Torus}
				\label{fig:bisection-bandwidth-hex}
			\end{subfigure}
			\begin{subfigure}[b]{0.3\linewidth}
				\center
				\buildfig{figures/bisection-bandwidth-3d.tex}
				
				\caption{3D Torus}
				\label{fig:bisection-bandwidth-3d}
			\end{subfigure}
			
			\caption[Bisections of torus topologies.]%
			{Bisections of torus topologies. Connections cut by the bisection
			are drawn as lines.}
			\label{fig:bisection-bandwidth}
		\end{figure}
		
		In a $N \times N$ 2D torus topology, the bisection bandwidth is $2N$~links
		and each node requires four links. The hexagonal torus topology requires
		six links per node but provides double bisection bandwidth ($4N$~links).
		The 3D torus topology also requires six links per node but by connecting
		the nodes differently achieves a bisection bandwidth of $8N$~links.  The 3D
		torus topology, however, comes at a price -- when immersed into the
		(approximately) 2D space provided by a large machine room or row of server
		cabinets, some connections require long cables. By contrast, the 2D and
		hexagonal torus topologies are both inherently two dimensional and
		consequently do not suffer from this effect. The hexagonal torus topology,
		therefore, shares the practicality of construction of a 2D torus while
		still gaining some of the performance of a 3D torus topology. In addition,
		because nodes in a hexagonal torus topology have a greater number of links,
		greater redundancy is available in the network to tolerate faults.
		
		Most torus topologies, including hexagonal, 2D and 3D toruses, have a
		related `mesh' topology. These mesh topologies maintain the same general
		connectivity structure as their torus topologies but omit wrap-around
		links. In practice, this saves a small number of links at the expense of
		halving the network's bisection bandwidth.  Because of their poorer
		performance, mesh networks are rarely used as the basis of a network
		architecture. Mesh networks, however, are occasionally formed when a
		network is partitioned into several smaller sub-networks to allow multiple
		users to share a system \cite{spalloc16}.
		
		\begin{figure}
			\center
			\begin{subfigure}[b]{0.45\linewidth}
				\center
				\buildfig{figures/hexagonal-torus.tex}
				\caption{Hexagonal torus}
				\label{fig:topo-compare-hexagonal-torus}
			\end{subfigure}
			\begin{subfigure}[b]{0.45\linewidth}
				\center
				\buildfig{figures/h-torus.tex}
				\caption{H-torus}
				\label{fig:topo-compare-h-torus}
			\end{subfigure}
			
			\caption[Hexagonal torus vs. H-torus topology.]%
			{Hexagonal torus vs. H-torus topology. Each numbered hexagon
			represents a node. The thick outline indicates the bounds of the
			topology after which the network repeats. In each topology, the path
			taken by advancing in the Y$^+$ direction from the node labelled `0' is
			shown.}
			\label{fig:topo-compare}
		\end{figure}
		
		\label{sec:hex-vs-h-torus}
		
		The hexagonal torus topology is not to be confused with the `H-torus'
		topology. This topology also uses a hexagonal tiling of nodes and even
		wraps this tiling into a torus-like topology \cite{zhao08}. However,
		H-torus topologies have very different characteristics to the hexagonal
		torus topology and are related to `twisted torus' topologies
		\cite{camara10}. For example, figure~\ref{fig:topo-compare} illustrates one
		major difference in the way paths wrap around the peripheries of both
		topologies.
	
	\section{Scaling-up SpiNNaker machines}
		
		To build large SpiNNaker systems comprising of tens of thousands of
		SpiNNaker chips, groups of forty-eight chips are mounted onto circuit
		boards as illustrated in figure~\ref{fig:spinnakerBoard}. These boards may
		be connected together to form larger systems.  Figure~\ref{fig:threeboard}
		shows a prototype three board system. Though the chips are
		\emph{physically} arranged in a (nearly) $7\times7$ grid on each SpiNNaker
		board, they logically form a hexagonal `wrapped triple'
		\cite{davidsonWiring} (see appendix~\ref{sec:partitioning}) which logically
		fit together as illustrated in figure~\ref{fig:threeboard-separate}. The
		labelled exposed corners of the three forty-eight chip boards connect
		together to form a $12\times12$ hexagonal torus topology as illustrated in
		figure~\ref{fig:threeboard-wrapped}. Larger SpiNNaker machines are
		assembled by combining more boards.
		
		\begin{figure}
			\center
			\begin{subfigure}[b]{0.45\linewidth}
				\center
				\includegraphics[width=\linewidth]{figures/spinnakerBoard.jpg}
				
				\caption{A SpiNNaker board}
				\label{fig:spinnakerBoard}
			\end{subfigure}
			~~~
			\begin{subfigure}[b]{0.45\linewidth}
				\center
				\includegraphics[width=\linewidth]{figures/threeboard.jpg}
				
				\caption{Three board prototype}
				\label{fig:threeboard}
			\end{subfigure}
			
			\vspace*{1em}
			
			\begin{subfigure}[b]{0.45\linewidth}
				\center
				\buildfig{figures/threeboard-separate.tex}
				
				\caption{Three board topology}
				\label{fig:threeboard-separate}
			\end{subfigure}
			~~~
			\begin{subfigure}[b]{0.45\linewidth}
				\center
				\buildfig{figures/threeboard-wrapped.tex}
				
				\caption{\ldots{}as a parallelogram}
				\label{fig:threeboard-wrapped}
			\end{subfigure}
			
			\caption{SpiNNaker boards and their topology.}
			\label{fig:spinnaker-boards}
		\end{figure}
		
		
		SpiNNaker chips on the same circuit board connect using low power links
		requiring sixteen wires each.  If this link technology were used to connect
		chips on neighbouring boards, each pair of boards would need to be
		connected with a 128~wire cable.  Cables and connectors supporting this
		many signals are expensive, unreliable and physically large. Instead,
		chip-to-chip connections between boards are multiplexed and demultiplexed
		onto a single High-Speed Serial (HSS) link \cite{athavale05} carried via
		commodity S-ATA cables which are often used to connect hard disks in
		desktop computers and servers \cite{sata3spec}. The six high-speed links
		are implemented by three onboard FPGAs (the three large chips at the top of
		the SpiNNaker board) and are logically transparent to the underlying
		network. The underlying technology and the choice of S-ATA cables limits
		each board-to-board connection to spanning at most one metre gaps. In
		chapter~\ref{sec:building} I present a cabling scheme for hexagonal torus
		topologies which enable large SpiNNaker systems to be assembled using only
		short cables between boards.
		
	\section{Conclusions}
		
		The SpiNNaker architecture has been designed to enable the simulation of
		large biologically realistic neural models in real time. To support this,
		its network architecture takes on an unconventional design based on a
		custom router and hexagonal torus topology. In the remainder of this
		thesis, I will tackle a number of the challenges in scaling up the
		SpiNNaker architecture outlined in this chapter.

	\chapter{Building large SpiNNaker machines}
	
	Like any super computer, physically putting together a large SpiNNaker
	machine poses many challenges in terms of organisation, assembly and
	maintainance. One of the key tasks in this process is the installation of
	network cables such that a desired overall network topology is constructed.
	The largest planned SpiNNaker machine will use \num{3600} S-ATA
	\cite{sata3spec} cables to interconnect its \num{1200} circuit boards,
	creating a hexagonal torus topology. Since the machine will be installed
	within standard server room cabinets (which are not available in a
	giant-doughnut form-factor) a mapping from a board's logical location in the
	network topology to its physical location must be constructed. In addition,
	the interconnect technology employed by SpiNNaker restricts the length of
	S-ATA cables used to $\le$~\SI{1}{\meter}, constraining the possible mappings
	used. In addition the practical issues of installation complexity and
	maintainance must be considered since all \num{3600} cables must ultimately
	be installed and maintained by human operators.
	
	In this chapter I describe a novel technique for physically laying out
	machines configured in hexagonal torus topologies, such as SpiNNaker, in
	commercial machine rooms, building on the techniques used in more
	conventional torus topologies. In addition, I also propose a new methodology
	for installing and maintaining super computer cabling which which exploits
	existing diagnostic features of the SpiNNaker hardware to interactively guide
	and validate cable installation. Finally, I demonstrate how these new
	techniques have been used successfully to interconnect a prototype
	\num{518400} core SpiNNaker machine in substantially less time than the
	industry norm.
	
	In this chapter, the term \emph{unit} refers to the smallest physical
	ecomponent between which connections connections are to be made. For example,
	in a SpiNNaker machine a unit is a 48-chip board while in data center, a unit
	might be a server blade.
	
	\section{Related work}
		
		In this section I describe the techniques conventionally employed when
		laying out and interconnecting the units within super computers. Due to
		SpiNNaker's hexagonal torus topology and dense physical packing of units,
		these existing techniques are found to be insufficient. In the remainder of
		the chapter we will explore solutions to the limitations exposed below.
		
		\subsection{Avoiding long cables}
			
			Na\"ive arrangements of torus topologies, including hexagonal torus
			topologies, feature long `wrap-around' connections which connect units at
			the peripheries of the system. These connections can be problematic for
			numerous reasons:
			
			\begin{description}
				
				\item[Performance] Signal quality diminishes as cables get longer,
				requiring the use of slower signalling speeds, increased error
				correction overhead or more complex hardware.
				
				\item[Energy] Longer cables require higher drive strengths and/or
				buffering to maintain signal integrity.
				
				\item[Cost] Cost Shorter cables are cheaper than long ones.  Longer
				cables imply more wire in a given space making the tasks of routing or
				cable installation more difficult increasing labour costs by as much as
				$5\times$ \cite{curtis12}.
				
			\end{description}
			
			In conventional torus topologies the need for long cables is eliminated
			by folding and interleaving units of the network \cite{dally04}. For
			example, for a 1D torus topology (a ring network), one long connection
			exists to connect the two opposite sides of the system. To remove these
			long connections, half the units are `folded' on top of the others and
			then this arrangement of units is interleaved as illustrated in figure
			\ref{fig:ring-folding}.
			
			\begin{figure}
				\center
				\begin{subfigure}[b]{0.39\linewidth}
					\center
					\buildfig{figures/ring-folding-row.tex}
					\caption{A ring network}
					\label{fig:ring-folding-row}
				\end{subfigure}
				\begin{subfigure}[b]{0.24\linewidth}
					\center
					\buildfig{figures/ring-folding-folded.tex}
					\caption{Folded}
					\label{fig:ring-folding-folded}
				\end{subfigure}
				\begin{subfigure}[b]{0.35\linewidth}
					\center
					\buildfig{figures/ring-folding-interleaved.tex}
					\caption{Folded and interleaved}
					\label{fig:ring-folding-interleaved}
				\end{subfigure}
				
				\caption{Folding and interleaving a ring network to reduce maximum wire
				length.}
				\label{fig:ring-folding}
			\end{figure}
			
			Folding and interleaving has the effect of approximately doubling the
			average cable length but also eliminates the need for a cable spanning
			the entire system. Since the mean cable length is typically already
			short, doubling it in exchange for a substantially reduced maximum cable
			length is often preferable.
			
			The folding and interleaving process may be extended to $N$-dimensional
			torus topologies by folding each dimension in turn. Since all dimensions
			are orthogonal, the folding process only moves units in the dimension
			being folded. In the hexagonal torus topology, however, the three
			dimensions are non-orthogonal and thus folding in one dimension also
			moves units in other dimensions, preventing the edges of the torus
			meeting as illustrated in figure \ref{fig:failing-to-fold-hex-toruses}.
			
			\begin{figure}
				\center
				\begin{subfigure}[b]{0.24\linewidth}
					\center
					\buildfig{figures/failing-to-fold-hex-toruses-none.tex}
					\caption{Not folded}
					\label{fig:failing-to-fold-hex-toruses-none}
				\end{subfigure}
				\begin{subfigure}[b]{0.24\linewidth}
					\center
					\buildfig{figures/failing-to-fold-hex-toruses-x.tex}
					\caption{X}
					\label{fig:failing-to-fold-hex-toruses-x}
				\end{subfigure}
				\begin{subfigure}[b]{0.24\linewidth}
					\center
					\buildfig{figures/failing-to-fold-hex-toruses-y.tex}
					\caption{Y}
					\label{fig:failing-to-fold-hex-toruses-y}
				\end{subfigure}
				\begin{subfigure}[b]{0.24\linewidth}
					\center
					\buildfig{figures/failing-to-fold-hex-toruses-z.tex}
					\caption{Z}
					\label{fig:failing-to-fold-hex-toruses-z}
				\end{subfigure}
				
				\caption{Schematics showing hexagonal torus topologies folded along
				each of their non-orthogonal dimensions. Note that folding along
				the Z axis brings the \emph{wrong} edges closer together.}
				\label{fig:failing-to-fold-hex-toruses}
			\end{figure}
		
		\subsection{Cabling installation}
			
			Existing machine room installations feature very repetitive cabling
			patterns which can easily be memorised by a human technician. For example
			in BlueGene super computers the connectivity between units is highly
			regular \cite{lakner07} while in data centre networks cabling often
			centres around a small number of high-port-count switches
			\cite{cisco07,csernai15}. Cable installation is usually only aided by
			the labelling of connectors and sockets in a standardised manner
			\cite{tia2006} such as in figure \ref{fig:bgWiring}.
			
			\begin{figure}
				\center
				\begin{subfigure}[t]{0.5\textwidth}
					\begin{tikzpicture}
						\node (cables) [inner sep=0]
						      {\includegraphics[width=\textwidth]{figures/bgCables.png}};
						\node (sockets) [inner sep=0, below=1.0em of cables]
						      {\includegraphics[width=\textwidth]{figures/bgSockets.png}};
						
						% Point at label on cable
						\draw [white, <-, line width=0.4em]
						      ([shift={(0.7cm, -0.3cm)}]cables.center)
						      -- ++(45:1cm);
						
						% Point at label on socket
						\draw [white, <-, line width=0.4em]
						      ([shift={(-1.0cm, 1.1cm)}]sockets.center)
						      -- ++(-45:1cm);
					\end{tikzpicture}
					
					\caption{Pre-labelled cables and sockets}
					\label{fig:bgWiringLabels}
				\end{subfigure}
				~
				\begin{subfigure}[t]{0.30\textwidth}
					\includegraphics[height=6.15cm]{figures/bgWiring.jpg}
					
					\caption{Installation of cables}
					\label{fig:bgWiringInstallation}
				\end{subfigure}
				
				\caption{BlueGene/Q cable installation \cite{cscs13}}
				\label{fig:bgWiring}
			\end{figure}
			
			Despite the regularity and careful labelling of cables, the cost of
			installation and maintenance alone can be significant with costs in the
			range of \$45-95 per \SI{1}{\meter} cable run and \$100-400 for runs of
			\SI{10}{\meter} reported in the literature \cite{mudigonda11}. Much of
			this cost is due to the care required during installation to avoid
			miswiring and ensure that cooling airflow is not hampered by cable runs
			\cite{cisco07}.
			
			Many researchers have attempted to control cable installation costs by
			trying to reduce the number or length of cables required by developing
			alternative network topologies \cite{curtis12, popa10, mudigonda11}.
			Unfortunately, these techniques do not apply to SpiNNaker since its
			network topology is fixed.
			
			Some super computers make use of large custom `midplane` PCBs in place of
			cables to interconnect units within a cabinet and thus simplify the task
			of cable installation \cite{prickett10}. This scheme can greatly reduce
			wiring complexity since only coarser-grain cabinet-to-cabinet
			connectivity is provided by cables. Unfortunately this technique is
			expensive and also constrains the dimensions of the network topology
			supported by the machine. Since the SpiNNaker platform is designed to
			scale from desktop machines to machine-room installations, this scheme is
			not practical.
	
	\section{Folding \& interleaving hexagonal toruses}
		
		The first step towards a practical machine-room installation of a large
		machine using a hexagonal torus topology is to find an arrangement of
		boards between which cable lengths are minimised. In this section I
		describe a sequence of transformations which map the positions of units in
		a hexagonal torus topology onto a regular rectangular grid which may be
		folded and interleaved to eliminate long wires. It is worth emphasising
		that this transformation only affects the \emph{physical} positions of
		units and \emph{not} their connectivity.
		
		As described earlier in \S\ref{sec:parititioning} (page
		\pageref{sec:parititioning}), hexagonal torus topologies may be partitioned
		into units containing wrapped-triples of nodes. For example, in SpiNNaker,
		chips (nodes) are partitioned into circuit boards (units) containing 48
		chips. For completeness, this section describes the process of folding both
		systems whose units are individual nodes and those whose units are
		wrapped-triples.
		
		The transformation process is divided into two parts, each described
		separately in this section.
		
		\begin{description}
			
			\item[Parallelogram to rectangle] Units of the system are transformed
			from a parallelogram shape to a rectangular shape.
			
			\item[Uncrinkle] Units within the rectangle are moved such that they all
			lie on a regular (and fully packed) 2D grid.
			
		\end{description}
		
		\subsection{Parallelogram to rectangle}
			
			The hexagonal torus topology is most naturally drawn as a parallelogram
			as illustrated in figures \ref{fig:hex-to-plane-node-native} and
			\ref{fig:hex-to-plane-native}. Two transformations are presented which
			transform these arangements of units into a rectangular form: shearing
			and slicing.
			
			A \SI{30}{\degree} shear transformation distorts networks such that the X
			and Y axes become orthogonal leading to a rectangular arrangement of
			units as illustrated in figures \ref{fig:hex-to-plane-node-shear} and
			\ref{fig:hex-to-plane-shear}.
			
			The slice transformation slices the units protruding from the
			left-hand-side of the parallelogram and moves them into the matching gap
			on the opposite side of the parallelogram as illustrated in figures
			\ref{fig:hex-to-plane-node-slice} and \ref{fig:hex-to-plane-slice}.
			 
			While the shear transformation introduces some distortion causing cables
			in the Z dimension to become $\sqrt{2}\times$ longer it leaves the
			pattern of wrap-around connections remains unchanged. By contrast, the
			slice transformation does not elongate any cables but changes the pattern
			of wrap-around connections. The exact pattern wrap-around connections
			produced when slicing depends on the aspect ratio of the network as
			illustrated in \ref{fig:slicing-examples} and influences the choice of
			folding technique applied as described later.
			
			\begin{figure}
				\center
				\begin{subfigure}[b]{0.32\linewidth}
					\center
					\buildfig{figures/hex-to-plane-node-native.tex}
					
					\caption{$7 \times 7$ node torus}
					\label{fig:hex-to-plane-node-native}
				\end{subfigure}
				\begin{subfigure}[b]{0.32\linewidth}
					\center
					\buildfig{figures/hex-to-plane-node-shear.tex}
					
					\caption{Sheared}
					\label{fig:hex-to-plane-node-shear}
				\end{subfigure}
				\begin{subfigure}[b]{0.32\linewidth}
					\center
					\buildfig{figures/hex-to-plane-node-slice.tex}
					
					\caption{Sliced}
					\label{fig:hex-to-plane-node-slice}
				\end{subfigure}
				
				\caption{Transformations of hexagonal toruses of nodes into a
				rectangular form. Thin lines show wrap-around links. Pointy-topped
				hexagons represent individual nodes.}
				\label{fig:hex-to-plane-node}
			\end{figure}
			
			\begin{figure}
				
				\begin{subfigure}[b]{0.32\linewidth}
					\center
					\buildfig{figures/hex-to-plane-native.tex}
					
					\caption{$4 \times 4$ triad torus}
					\label{fig:hex-to-plane-native}
				\end{subfigure}
				\begin{subfigure}[b]{0.32\linewidth}
					\center
					\buildfig{figures/hex-to-plane-shear.tex}
					
					\caption{Sheared}
					\label{fig:hex-to-plane-shear}
				\end{subfigure}
				\begin{subfigure}[b]{0.32\linewidth}
					\center
					\buildfig{figures/hex-to-plane-slice.tex}
					
					\caption{Sliced}
					\label{fig:hex-to-plane-slice}
				\end{subfigure}
				
				\caption{Transformations of hexagonal toruses of wrapped triples into a
				rectangular form.  Thin lines show wrap-around links. Flat-topped
				hexagons represent a wrapped triple of nodes.}
				\label{fig:hex-to-plane}
			\end{figure}
			
			\begin{figure}
				\center
				\buildfig{figures/slicing-examples.tex}
				\caption{Patterns of wiring in sliced systems of various sizes.}
				\label{fig:slicing-examples}
			\end{figure}
			
		\subsection{Uncrinkling}
			
			Though the transformmation step yields rectangular arrangements of units,
			these arrangements do not fall onto a regular 2D grid, with the exception
			of the shear transform on individual nodes. Figure \ref{fig:uncrinkling}
			illustrates how the various arrangements of hexagons may be moved to
			`uncrinkle' the units into a regular grid.
			
			\begin{figure}
				\center
				\begin{subfigure}[b]{0.44\linewidth}
					\center
					\buildfig{figures/uncrinkling-node-sheared.tex}
					
					\caption{$7 \times 7$ nodes, sheared}
					\label{fig:uncrinkling-node-sheared}
				\end{subfigure}
				\begin{subfigure}[b]{0.44\linewidth}
					\center
					\buildfig{figures/uncrinkling-node-sliced.tex}
					
					\caption{$7 \times 7$ nodes, sliced}
					\label{fig:uncrinkling-node-sliced}
				\end{subfigure}
				
				\vspace{1cm}
				
				\begin{subfigure}[b]{0.44\linewidth}
					\center
					\buildfig{figures/uncrinkling-sheared.tex}
					
					\caption{$4 \times 4$ triples, sheared}
					\label{fig:uncrinkling-sheared}
				\end{subfigure}
				\begin{subfigure}[b]{0.44\linewidth}
					\center
					\buildfig{figures/uncrinkling-sliced.tex}
					
					\caption{$4 \times 4$ triples, sliced}
					\label{fig:uncrinkling-sliced}
				\end{subfigure}
				
				\vspace{1em}
				
				\caption{Mapping rectangular arrangements of units into a square grid.
				Thick lines show how layers of units are uncrinkled.  Annotations show
				how the relative positions of nodes and wrapped triples change after
				uncrinkling.}
				\label{fig:uncrinkling}
			\end{figure}
			
			In the figure, the numbered units enumerate the different positions on
			the crinkle and those labelled alphabetically are those that immediately
			surround them. From this we can observe that uncrinkling largely
			preserves spatial locality but some distortion is introduced, separating
			previously neighbouring units. For example, in figure
			\ref{fig:uncrinkling-sheared}, the units labelled `1' and `i' are
			neighbours before uncrinkling but are separated by a (Euclidean) distance
			of $\sqrt{5}$ afterwards. Note that the distortion introduced depends on
			what part of the crinkle is considered, for example `2' and `a' have
			distance 2 but are logically connected in the same way.
		
		\subsection{Folding and Interleaving}
			
			Once a regular grid of units has been formed, this may be folded in the
			conventional way, eliminating long cables crossing from left-to-right and
			top-to-bottom as illustrated in \ref{fig:folding-sheared}.
			
			Unfortunately, for sliced systems whose dimensions are not of the ratio
			$1:2$, the pattern of wrap-around cables may also include some cables
			which do not cross directly to the opposite side of the system (refer
			back to figure \ref{fig:slicing-examples}). As a result of these
			connections, folding does not successfully eliminate all long
			connections. An exception to this rule is sliced systems whose dimensions
			are in the ratio $1:1$ where folding twice along the Y axis may
			successfully eliminate all wrap-around connections as illustrated in
			\ref{fig:folding-sliced}.
			
			\begin{figure}
				\begin{subfigure}{\linewidth}
					\center
					\buildfig{figures/folding-sheared.tex}
					\caption{$N \times M$ sheared systems and $N \times 2N$ sliced systems}
					\label{fig:folding-sheared}
				\end{subfigure}
				
				\vspace{1em}
				
				\begin{subfigure}{\linewidth}
					\center
					\buildfig{figures/folding-sliced.tex}
					\caption{$N \times N$ sliced systems}
					\label{fig:folding-sliced}
				\end{subfigure}
				
				\caption{Schematic illustrating elimination of long wrap-around links
				during folding. In each example a single link has been highlighted to
				aid in following the process.}
				\label{fig:folding}
			\end{figure}
			
			Once folded, the 2D grid is straight-forwardly interleaved as illustrated
			previously in figure \ref{fig:ring-folding}. The interleaving process
			introduces some additional distortion to the layout of units and causes
			most connections to become twice as long. For sliced $1:1$ systems, the
			additional fold results in additional overhead during interleaving since
			four layers of the system are interleaved.
		
		\subsection{Mapping to Cabinets}
			
			In the final step of the process is to map the 2D grid of units into
			positions in machine room cabinets as illustrated in figure
			\ref{fig:million-core-machine}. As illustrated in figure
			\ref{fig:cabinetisation}, first the grid of units is partitioned into
			groups of columns, one per cabinet, then groups of rows one per frame per
			cabinet. The units in each group are then allocated to slots within a
			frame, interleaving the rows of the groups as shown.
			
			\begin{figure}
				\center
				\buildfig{figures/cabinet-units.tex}
				
				\caption{An illustration of the physical construction of a
				multi-cabinet SpiNNaker system. (Note: network cables \emph{not}
				installed.)}
				\label{fig:cabinet-units}
			\end{figure}
			
			\begin{figure}
				\center
				\buildfig{figures/cabinetisation.tex}
				
				\caption{Mapping from 2D space to cabinets, frames and boards.}
				\label{fig:cabinetisation}
			\end{figure}
		
	\section{Cable installation}
		
		Cable installation is performed by a team of (human) technicians who must
		ensure that all network cables are correctly installed. As illustrated in
		previously in figure \ref{fig:cabinet-units}, the density of SpiNNaker's
		units, combined with the nature of the hexagonal torus topology, poses a
		challenge. To address this challenge I propose a semi-automated approach to
		cable installation which exploits diagnostic facilities available in the
		majority of super computers in order to guide technicians through the
		cabling process, interactively guiding installation and maintenance.
		
		\subsection{Interactive technician guidance and validation}
			
			While automated systems for validating cabling correctness are
			commonplace, these systems are typically used only after cabling has been
			completed \cite{lakner07}. As with other large-scale machines, SpiNNaker
			includes a low-bandwidth system management bus which may be used to
			interrogate network hardware and control diagnostic LEDs prior to the
			installation of the main SpiNNaker network interconnect.  Using these
			facilities I have constructed a tool called SpiNNer which interactively
			guides a technician, or team of technicians, through the cable
			installation process, validating each connection in real-time.
			
			Diagnostic LEDs mounted on each SpiNNaker board (figure
			\ref{fig:interactive-wiring-guide-leds}) are used to indicate the
			endpoints of the cable currently being installed. Simultaneously a
			Text-To-Speech (TTS) system gives an audible indication of which cable
			type is to be used and location of each connection.  Additionally, a GUI
			via a computer display (figure \ref{fig:interactive-wiring-guide-gui}).
			The centre of the display shows a `big-picture' perspective of the
			locations of the boards to be connected. The detailed views on the left
			and right indicate which of the six sockets on each board the cables
			should connect.
			
			\begin{figure}
				\center
				\begin{subfigure}[b]{0.40\textwidth}
					\begin{tikzpicture}
						\node (leds) [inner sep=0]
						      {\includegraphics[width=\textwidth]{figures/leds.jpg}};
						% Point at left LED
						\draw [white, <-, line width=0.4em]
						      ([shift={(-0.0cm, -0.6cm)}]leds.center)
						      -- ++(225:1cm);
						% Point at right LED
						\draw [white, <-, line width=0.4em]
						      ([shift={(1.1cm, -1.1cm)}]leds.center)
						      -- ++(225:1cm);
					\end{tikzpicture}
					
					\caption{Diagnostic LEDs}
					\label{fig:interactive-wiring-guide-leds}
				\end{subfigure}
				~
				\begin{subfigure}[b]{0.546\textwidth}
					\begin{tikzpicture}[thin, black!20!white]
						\node (screen) [inner sep=0]
						      {\includegraphics[width=\textwidth]{figures/wiring_guide_screenshot.png}};
						\draw (screen.south west) rectangle (screen.north east);
					\end{tikzpicture}
					
					\caption{Interactive wiring guide GUI}
					\label{fig:interactive-wiring-guide-gui}
				\end{subfigure}
				
				\caption{The SpiNNer interactive wiring guide uses a GUI,
				text-to-speech and diagnostic LEDs to assist during cable
				installation.}
				\label{fig:interactive-wiring-guide}
			\end{figure}
			
			SpiNNer also validates the connectivity of the system in real-time by
			polling the diagnostic interfaces of the network hardware at the
			endpoints of the cable being installed to determine if they are correctly
			connected. If a miswiring occurs, this is immediately detected and
			announced via TTS enabling the technician to immediately correct the
			error. Once a cable has been installed correctly, the software
			automatically advances to the next cable meaning direct interaction with
			the software by the technician is minimal. In practice, it is rarely
			necessary to refer to the GUI.
		
			SpiNNer presents the cables in an order intended to maximise ease of
			installation. Cables are installed in three groups with intra-frame
			cables being installed first, followed by intra-cabinet cables and
			inter-cabinet cables. Within each group, the tightest cables are
			installed first resulting in slacker cables naturally being installed
			over the top of already installed cables. By grouping cables in this
			manner, multiple technicians may work independently on the wiring within
			individual frames and cabinets.
			
			SpiNNer may also be used to repair or replace cables in the system.
			During maintenance, obstructing cables may be blindly removed alongside
			any cable being replaced. At the conclusion of the process, the wiring
			guide may be used to interactively guide re-installation of all removed
			cables.
		
		\subsection{Cable selection}
			
			Controlling slack is critical to ensuring reliable and maintainable
			cabling installations. If cables are too tight, cables and connectors can
			become easily damaged and when too slack, the excess cable obstructs
			other cables and can easily become tangled and damaged \cite{cisco07}. It
			has been observed that when ready-made cables are employed technicians
			frequently over-estimate the cable lengths required preferring to use
			overly long cables for all connections \cite{mazaris97}. To solve this
			problem, the SpiNNer wiring guide software dictates the cable lengths to
			be used by an installer based the rule of (three-)thumbs according to
			Mazaris \cite{mazaris97}. This rule suggests that an ideal amount of
			slack is approximately that which can be wrapped around three fingers.
			Specifically, the shortest available cable is selected which ensures at
			least \SI{5}{\centi\meter} of slack.
			
			The SpiNNer tool allocates cables assuming all cables take a Euclidean
			straight-line path between the endpoints of the connection. The result is
			that wiring is not routed through dedicated cable management structures
			but is simply suspended by its connectors in front of the machine. As
			demonstrated later, this unconventional approach leads neither to cooling
			problems nor increased maintenance effort.
	
	\section{Results and Evaluation}
		
		This stuff has been used and works. In this section I'll go over the
		overheads introduced by the various transformations and
		folding/interleaving steps and show a wiring scheme for a large machine
		which uses only short cables. I'll then show how SpiNNer was used to
		install this wiring plan into a very large machine without foobaring the
		cooling and in very little time. I'll also report on difficulty of
		maintenance.
		
		\subsection{Cable length}
			
			The transformation from regular hexagonal torus to a folded and
			interleaved form introduces some overhead to the cable lengths required.
			Using figure \ref{fig:uncrinkling} (page \pageref{fig:uncrinkling}), it
			is possible to compute the exact overhead introduced when each type of
			transformation proposed.
			
			For example, to compute the mean overhead introduced by the slicing
			technique when applied to units composed of wrapped triples, consider
			figure \ref{fig:uncrinkling-sliced}. The uncrinkling pattern used to
			transform this topology is a repeating pattern of two units, a pair of
			which have been labelled $1$ and $2$ respectively. Unit $1$ is
			immediately surrounded by six units labelled $a$, $b$, $c$, $2$, $g$ and
			$h$. Similarly, unit $2$ is surrounded by units $1$, $c$, $d$, $e$, $f$
			and $g$. Before the transformation, the distances, $D$, to each of these
			units is $1$ but after the transformation is applied, this is not always
			the case. Additionally, folding and interleaving introduce additional
			overhead. In this example, if the system is folded into $f_x$ columns and
			$f_y$ rows, the distances between previously neighbouring units become:
			
			\begin{equation*}
				\begin{aligned}[c]
					D_{1\,\leftrightarrow{}\,a} &= \sqrt{f_x^2 + f_y^2} \\
					D_{1\,\leftrightarrow{}\,b} &= f_y \\
					D_{1\,\leftrightarrow{}\,c} &= \sqrt{f_x^2 + f_y^2} \\
					D_{1\,\leftrightarrow{}\,2} &= f_x \\
					D_{1\,\leftrightarrow{}\,g} &= f_y \\
					D_{1\,\leftrightarrow{}\,h} &= f_x
				\end{aligned}
				\hspace{2cm}
				\begin{aligned}[c]
					D_{2\,\leftrightarrow{}\,1} &= f_x \\
					D_{2\,\leftrightarrow{}\,c} &= f_y \\
					D_{2\,\leftrightarrow{}\,d} &= f_x \\
					D_{2\,\leftrightarrow{}\,e} &= \sqrt{f_x^2 + f_y^2} \\
					D_{2\,\leftrightarrow{}\,f} &= f_y \\
					D_{2\,\leftrightarrow{}\,g} &= \sqrt{f_x^2 + f_y^2}
				\end{aligned}
			\end{equation*}
			
			From these values, the mean and maximum connection distances after
			folding and interleaving may be computed. Table
			\ref{tab:transform-overhead} gives the mean and maximum connection
			distances for each of the four transformations described in this chapter.
			
			\begin{table}
				\begin{subtable}[b]{\linewidth}
					\center
					\begin{tabular}{l c c}
						\toprule
						& Shear & Slice \\
						\addlinespace
						Nodes &
							$\frac{f_x + f_y + \sqrt{f_x^2 + f_y^2}}{3}$ &
							$\frac{f_x + f_y + \sqrt{f_x^2 + f_y^2}}{3}$ \\
						\addlinespace
						Triples &
							$\frac{7f_x + 3\sqrt{f_x^2 + f_y^2} + \sqrt{(2f_x)^2 + f_y^2}}{9}$ &
							$\frac{f_x + f_y + \sqrt{f_x^2 + f_y^2}}{3}$ \\
						\bottomrule
					\end{tabular}
					
					\caption{Mean}
					\label{tab:transform-overhead-mean}
				\end{subtable}
				
				\vspace{1em}
				
				\begin{subtable}[b]{\linewidth}
					\center
					\begin{tabular}{l c c}
						\toprule
						& Shear & Slice \\
						\addlinespace
						Nodes &
							$\sqrt{f_x^2 + f_y^2}$ &
							$\sqrt{f_x^2 + f_y^2}$ \\
						\addlinespace
						Triples &
							$\sqrt{(2f_x)^2 + f_y^2}$ &
							$\sqrt{f_x^2 + f_y^2}$ \\
						\bottomrule
					\end{tabular}
					
					\caption{Maximum}
					\label{tab:transform-overhead-max}
				\end{subtable}
				
				\caption{Overheads introduced when transforming unit positions onto a
				regular grid.}
				\label{tab:transform-overhead}
			\end{table}
			
			From these results it is evident that shearing and slicing networks
			whose units are nodes result in identical mean and maximum overhead in
			cable length when folded similarly. Since sliced networks may require
			folding more than once along each axis the shearing approach is
			preferable in general.
			
			For networks constructed from units of wrapped triples, the slicing
			approach suffers the same mean and maximum overhead has networks of
			nodes, and less overhead than shearing for the same number of folds. For
			systems with an aspect ratio of $1:2$ (where both slicing and shearing
			require $f_x = f_y = 2$), the slicing transformation yields lower mean
			and maximum overhead than shearing. For all other aspect ratios (where
			slicing requires a greater degree of folding) the shearing technique
			produces lower overhead. The recommended transformations for a given
			machine are thus given in table \ref{tab:transform-recommended}.
			
			\begin{table}
				\center
				\begin{tabular}{lcc}
					\toprule
					                         & $1:2$  & Other \\
					\addlinespace
					\multirow{2}{*}{Nodes}   & Either & Shear\\
					                         & \footnotesize $\mu\approx2.28 \quad \vee\approx2.83$
					                         & \footnotesize $\mu\approx2.28 \quad \vee\approx2.83$\\
					\addlinespace
					\multirow{2}{*}{Triples} & Slice  & Shear\\
					                         & \footnotesize $\mu\approx2.28 \quad \vee\approx2.83$
					                         & \footnotesize $\mu\approx3.00 \quad \vee\approx4.47$\\
					\bottomrule
				\end{tabular}
				
				\caption{Recommended transformation and folding scheme for different
				system types. $\mu$ and $\vee$ give the mean and maximum wire
				distortion introduced, respectively.}
				\label{tab:transform-recommended}
			\end{table}
			
			\begin{figure}
				\center
				\buildfig{figures/million-core-machine.tex}
				
				\caption{Cabling plan for a \num{1036800} core SpiNNaker
				machine's \num{3600} cables.}
				\label{fig:million-core-machine}
			\end{figure}
			
			Following folding and mapping to physical locations, the cabling plans
			for large machines require no large gaps to be spanned.  The largest
			planned SpiNNaker machine, illustrated in figure
			\ref{fig:million-core-machine}, will be \SI{6}{\meter} wide but the
			largest gap any cable must span is \SI{66}{\centi\meter}. This distance
			is well within the \SI{1}{\meter} allowed by the hardware and cables.
			
		\subsection{Installation practicality}
			
			\begin{table}
				\center
				\begin{tabular}{lrr@{$\,$}l}
					\toprule
						System & Number of Cables & \multicolumn{2}{r}{Installation time} \\
					\midrule
						24 boards  & \num{72}   & \num{10} & \si{\minute}         \\
						1 cabinet  & \num{360}  & \num{4}  & \si{\hour}$^\dagger$ \\
						2 cabinets & \num{720}  & \num{2}  & \si{\hour}           \\
						5 cabinets & \num{1800} & ?        &                      \\
					\bottomrule
				\end{tabular}
				
				\caption{Installation times for various sizes of machine.
				$\dagger$~This machine was installed without real-time validation of
				connectivity.}
				\label{tab:install-time}
			\end{table}
			
			A number of SpiNNaker machines of various scales have been assembled
			using the techniques described in this chapter ranging from single frames
			of 24 boards to a half-scale 5 cabinet machine. Table
			\ref{tab:install-time} gives the reported installation times of each of
			these machines.
			
			The single cabinet machine's installation time is notably
			disproportionate to its size. When this system was assembled, real-time
			connection validation was not yet available. As a result, though cable
			installation was rapid correcting errors was extremely costly, requiring
			careful retracing of many installation steps.
			
			TODO: TALK ABOUT MULTI-PERSON-WIRING IN PRACTICE ON FIVE CABINET MACHINE.
			
			\begin{figure}
				
				\center
				\buildfig{figures/wire-length-histogram.tex}
				
				\caption{Histogram of connection distances in a ten-cabinet,
				one-million core SpiNNaker machine annotated with the suggested cable
				length.}
				\label{fig:wire-length-histogram}
				
			\end{figure}
			
			FIGURE \ref{fig:wire-length-histogram} SHOWS THE DISTRIBUTION OF CABLE
			LENGTHS REQUIRED. IN PRACTICE THE SLACK ALLOCATED PROVED ADEQUATE. AS
			SHOWN IN FIGURE \ref{fig:install-histogram}, THE MOST IMPORTANT FACTOR IS
			WHETHER LEAVING THE FRAME OR NOT. LEAVING THE FRAME TAKES THE LONGEST.
			
			\begin{figure}
				\builddata{data/build_connection_log.tex}
				\buildfig{figures/install-histogram.tex}
				
				\caption{Histogram of cable installation times}
				\label{fig:install-histogram}
			\end{figure}
			
			TODO: COMPARE DIRECTLY WITH INSTALL TIMES REPORTED IN LITERATURE.
		
		\subsection{Thermal Impact}
			
			TODO: SHOW HOW TEMPERATURE IS CHANGED
			
		\subsection{Maintenance}
			
			TOOD: QUANTIFY CABLE REMOVALS REQUIRED. EXPERIMENT: REMOVE/REPLACE RANDOM
			BOARDS AND MEASURE TIME TAKEN, CABLES REMOVED. COMPARE WITH STANDARD DATA
			CENTRE WIRING

	\chapter{Finding shortest path vectors in SpiNNaker's network}
	
	Once a SpiNNaker machine has been constructed as described in the previous
	chapter, its network forms a large hexagonal torus topology. To exploit this
	network routing algorithms must be used to generate routes for packets to
	follow between nodes. As well as ensuring that packets arrive at the correct
	destination, routing algorithms often attempt to produce routes which make
	efficient use of the network. This often involves attempting to reduce
	congestion by ensuring packets do not travel further through the network than
	absolutely necessary.
	
	Many popular routing algorithms for torus topologies, including all published
	algorithms designed for SpiNNaker's hexagonal torus topology
	\cite{davies12,navaridas14}, internally function by computing shortest path
	vectors and generating routes from them. Existing methods of calculating
	shortest path vectors in hexagonal torus topologies are unable to generate
	all possible shortest path vectors and, as a result, reduces the diversity of
	routes produced by routing algorithms, potentially worsening network
	contention.
	
	In this chapter I describe a novel technique for computing shortest path
	vectors in hexagonal torus topologies which yields \emph{all} possible
	shortest path vectors for any pair of nodes. Further, implementations of this
	new technique execute an order of magnitude faster than the existing
	alternatives.
	
	\section{Related work}
		
		TODO: INTRODUCE SECTION
		
		\begin{figure}
			\center
			
			\begin{subfigure}{\linewidth}
				\center
				\buildfig{figures/distance-map-mesh.tex}
				\caption{2D mesh topology}
				\label{fig:distance-map-mesh}
			\end{subfigure}
			
			\vspace{1em}
			
			\begin{subfigure}{\linewidth}
				\center
				\buildfig{figures/distance-map-torus.tex}
				\caption{2D torus topology}
				\label{fig:distance-map-torus}
			\end{subfigure}
			
			\vspace{1em}
			
			\begin{subfigure}{\linewidth}
				\center
				\buildfig{figures/distance-map-hex-mesh.tex}
				\caption{Hexagonal mesh topology}
				\label{fig:distance-map-hex-mesh}
			\end{subfigure}
			
			\vspace{1em}
			
			\begin{subfigure}{\linewidth}
				\center
				\buildfig{figures/distance-map-hex-torus.tex}
				\caption{Hexagonal torus topology}
				\label{fig:distance-map-hex-torus}
			\end{subfigure}
			
			\caption{Plots showing distance from various locations marked
			         {\color{red}$\times$}. Darker areas are further away. Contour
			         lines show equidistant points.}
			\label{fig:distance-map}
		\end{figure}
		
		\subsection{Mesh Networks}
			
			In a (non-hexagonal) mesh network topology, shortest path vectors are
			computed by taking the element-wise difference between the source and
			destination nodes' coordinates.
			
			\begin{figure}
				\center
				\buildfig{figures/mesh-topology-coordinates.tex}
				\caption{An example 2D mesh network with example shortest-path routes
				from `A' to `B' and `B' to `C'.}
				\label{fig:mesh-topology-coordinates}
			\end{figure}
			
			For example, figure \ref{fig:mesh-topology-coordinates} illustrates a 2D
			mesh topology. In this topology, the nodes labelled `A', `B' and `C' have
			position vectors $(1, 2)$, $(4, 5)$ and $(6, 1)$ respectively. The
			shortest path vector from node `A' to `B' is thus simply $(4, 5) - (1, 2)
			= (3, 3)$ and from `B' to `C' is $(6, 1) - (4, 5) = (2, -4)$.
			
			A route may be produced from a shortest path vector by advancing the
			number of hops specified for each dimension in the vector. For example
			any permutation of the hops X$^+\,$X$^+\,$X$^+\,$Y$^+\,$Y$^+\,$Y$^+$, an
			example of which is included in the figure. Likewise a route from `B' to
			`C' may be constructed from any permutation of
			X$^+\,$X$^+\,$Y$^-\,$Y$^-\,$Y$^-\,$Y$^-$.
			
			Many popular routing algorithms such as Dimension Order Routing (DOR),
			Right-Turn Only Routing (RTOR) and Longest Dimension First Routing (LDFR)
			\cite{dally04,davies12} directly follow the above procedure and just
			prescribe a specific permutation of hop order. For example, DOR produces
			routes with X hops first, Y hops second and so on.
			
			The length of routes produced from a shortest path vector have a number
			of hops proportional to the magnitude of the vector, thus a shortest path
			vector yields a route with the minimum number of hops. For a two
			dimensional vector $(a, b)$ the magnitude is given as:
			%
			\begin{equation}
				\| (a, b) \| = \lvert a \rvert + \lvert b \rvert
			\end{equation}
		
		\subsection{Torus Networks}
			
			Computing shortest path vectors in (non-hexagonal) torus topologies is
			also straight forward. As an example, lets find the shortest path vector
			from node `A' to `B' in the 2D torus topology shown in figure
			\ref{fig:torus-shortest-path-example}. First, both nodes are translated
			such that the source node, `A', is at the centre of the network (figure
			\ref{fig:torus-shortest-path-translate}). Note that this translation may
			result in the destination node `wrapping around' the network. After
			translation, the shortest path vector is computed as in a mesh topology.
			As illustrated in \ref{fig:torus-shortest-path-routed}, the computed
			shortest path vector may be used to produce routes between the two nodes
			in their original positions.
			
			\begin{figure}
				\center
				\begin{subfigure}{0.3\linewidth}
					\center
					\buildfig{figures/torus-shortest-path-example.tex}
					\caption{Original}
					\label{fig:torus-shortest-path-example}
				\end{subfigure}
				\begin{subfigure}{0.3\linewidth}
					\center
					\buildfig{figures/torus-shortest-path-translate.tex}
					\caption{Translated}
					\label{fig:torus-shortest-path-translate}
				\end{subfigure}
				\begin{subfigure}{0.3\linewidth}
					\center
					\buildfig{figures/torus-shortest-path-routed.tex}
					\caption{Routed}
					\label{fig:torus-shortest-path-routed}
				\end{subfigure}
				
				\caption{Finding shortest paths in a 2D torus topology.}
				\label{fig:torus-shortest-path}
			\end{figure}
			
			This process works because vectors from the centre (though not other
			locations) of a torus topology are identical to those in mesh topologies
			(see figures \ref{fig:distance-map-mesh} and
			\ref{fig:distance-map-torus}).
		
		\subsection{Hexagonal Mesh Networks}
			
			In hexagonal mesh topologies it is conventional to define three `axes' X,
			Y and Z as shown in figure \ref{fig:hex-mesh-topology-coordinates}
			\cite{patel15}. In this example, the three labelled nodes `A', `B' and
			`C' may be given position vectors such as $(1, 1, 0)$, $(3, 2, 0)$ and
			$(0, 0, -7)$ respectively. As in other mesh networks, a vector between
			two nodes is found by subtracting the nodes' vectors. For example, a
			vector from `A' to `B' is $(3, 2, 0) - (1, 1, 0) = (2, 1, 0)$. This
			vector can then be converted into a route in the same way as a mesh
			network by taking any permutation of the three hops  X$^+\,$X$^+\,$Y$^+$.
			
			\begin{figure}
				\center
				\buildfig{figures/hex-mesh-topology-coordinates.tex}
				\caption{An example hexagonal mesh network topology.}
				\label{fig:hex-mesh-topology-coordinates}
			\end{figure}
			
			As explained in detail in appendix \ref{app:minimal-hex-coordinates},
			there are an infinite number of vectors between any two points. For
			example, the vectors $(1, 0, -1)$ and $(3, 2, 1)$ also reach node `B'
			from `A' in the example. However, for a given pair of nodes, there is
			always a single, unique vector whose magnitude is minimal which is
			given by the function:
			%
			\begin{equation}
				\operatorname{minimiseVector}(x,y,z)
					= (x,y,z) - \operatorname{median}(x,y,z) \cdot (1,1,1)
			\end{equation}
			%
			An important side-effect of this function is that a minimised vector will
			always contain at least one zero element meaning that shortest path
			routes will use at most two of the three available dimensions.
			
			To aid the reader's intuition, figure \ref{fig:distance-map-hex-mesh}
			illustrates how distances grow in a hexagonal mesh topology.
		
		\subsection{Hexagonal Torus Networks}
			
			Unfortunately, unlike non-hexagonal torus topologies, the translation
			technique cannot be used to compute shortest path vectors. As illustrated
			in figures \ref{fig:distance-map-hex-mesh} and
			\ref{fig:distance-map-hex-torus}, shortest path vectors from the center
			of a hexagonal mesh network are not equivalent to those of a hexagonal
			torus network.
			
			Prior research into routing in SpiNNaker's network has been based on the
			INSEE \cite{navaridas09,ghasempour15} interconnect simulator. Internally
			INSEE tries a set of twelve candidate vectors and picks the shortest as
			the shortest path vector to use for routing.
			
			\begin{figure}
				\center
				\begin{subfigure}{0.45\linewidth}
					\center
					\buildfig{figures/insee-vector-candidates-no-wrap.tex}
					\caption{$(\Delta_\textrm{X}, \Delta_\textrm{Y}) = (5,3)$}
					\label{fig:insee-vector-candidates-no-wrap}
				\end{subfigure}
				\begin{subfigure}{0.45\linewidth}
					\center
					\buildfig{figures/insee-vector-candidates-wrap-x.tex}
					\caption{$(\Delta'_\textrm{X}, \Delta_\textrm{Y}) = (-3,3)$}
					\label{fig:insee-vector-candidates-wrap-x}
				\end{subfigure}
				
				\vspace{1em}
				
				\begin{subfigure}{0.45\linewidth}
					\center
					\buildfig{figures/insee-vector-candidates-wrap-y.tex}
					\caption{$(\Delta_\textrm{X}, \Delta'_\textrm{Y}) = (5,-5)$}
					\label{fig:insee-vector-candidates-wrap-y}
				\end{subfigure}
				\begin{subfigure}{0.45\linewidth}
					\center
					\buildfig{figures/insee-vector-candidates-wrap.tex}
					\caption{$(\Delta'_\textrm{X}, \Delta'_\textrm{Y}) = (-3,-5)$}
					\label{fig:insee-vector-candidates-wrap}
				\end{subfigure}
				
				\vspace{1em}
				
				% Key
				\begin{tikzpicture}[thick]
					\coordinate (last);
					
					% #1 colour
					% #2 label
					\newcommand{\colourkeyentry}[2]{
						\node [#1] [right=of last, fill, rectangle, minimum size=1em] (last) {};
						\node [right=0 of last] (last) {#2};
					}
					
					\colourkeyentry{cb3class0}{$(\textrm{X}, \textrm{Y}, 0)$}
					\colourkeyentry{cb3class1}{$(\textrm{X} - \textrm{Y}, 0, - \textrm{Y})$}
					\colourkeyentry{cb3class2}{$(0, \textrm{Y} - \textrm{X}, - \textrm{X})$}
					
				\end{tikzpicture}
				
				\caption{The twelve candidate shortest-path vectors considered by INSEE
				represented as dimension-order routes. $W=H=8$,
				$(\Delta_\textrm{X},\Delta_\textrm{Y}) = (5, 3)$ and
				$(\Delta'_\textrm{X},\Delta'_\textrm{Y}) = (-3, -5)$.}
				\label{fig:insee-vector-candidates}
			\end{figure}
			
			The twelve vectors considered are constructed as follows.
			
			First a shortest path vector from the source to target node are
			constructed as if the network was a 2D mesh yielding a vector
			$(\Delta_\textrm{X},\Delta_\textrm{Y})$. From this, another vector
			$(\Delta'_\textrm{X},\Delta'_\textrm{Y})$, is defined:
			%
			\begin{align}
				\Delta'_\textrm{X} &= \Delta_\textrm{X} - \operatorname{sign}(\Delta_\textrm{X})W
				\\
				\Delta'_\textrm{Y} &= \Delta_\textrm{Y} - \operatorname{sign}(\Delta_\textrm{Y})H
			\end{align}
			%
			Where $W$ and $H$ are the width and height of the network respectively
			(in nodes). This new vector yields routes from the source to destination
			node that in a torus topology that \emph{always} wrap around the `X' and
			`Y' dimensions.
			
			From the pair of vectors defined, four possible 2D vectors can be
			produced: $(\Delta_\textrm{X},\Delta_\textrm{Y})$,
			$(\Delta'_\textrm{X},\Delta_\textrm{Y})$,
			$(\Delta_\textrm{X},\Delta'_\textrm{Y})$ and
			$(\Delta'_\textrm{X},\Delta'_\textrm{Y})$. Further, each 2D vector may be
			converted into one of three 3D vectors where either X, Y or Z are zero
			for a total of twelve candidate vectors.  Figure
			\ref{fig:insee-vector-candidates} illustrates all twelve candidate
			vectors for an example pair of nodes.
			
			\begin{figure}
				\center
				\buildfig{figures/xyz-protocol-regions.tex}
				
				\caption{The four regions defined by the XYZ-protocol.}
				\label{fig:xyz-protocol-regions}
			\end{figure}
			
			A more efficient technique is proposed by Hoffmann and D\'es\'erable
			called the XYZ-Protocol \cite{hoffmann15,hoffmann11}. If the source and
			destination nodes are translated such that the source node lies at the
			center of the topolgoy, the destination will lie in one of four regions
			illustrated in figure \ref{fig:xyz-protocol-regions}.
			
			If the destination lies in regions 1 or 4, a route may be constructed as
			if in a hexagonal mesh topology.
			
			Alternatively, if the destination lies in regions 2 or 3, the algorithm
			tests whether taking a mesh-like route within the region or
			wrapping-around either the X or Y dimension yields the shorter vector.
			The shortest of these vectors is then chosen.
			
			TODO DESCRIBE SPIRAL ROUTES.
			
			TODO DESCRIBE RTOR AND LDFR.
		
	\section{Dimension order routing in hexagonal torus topologies}
		
		So, existing solutions have two problems: trying 12 options and picking one
		is a bit kludgey and there are actually more options than that.
		
		\subsection{Simpler minimal hexagonal torus vectors}
			
			If you redraw your route such that it is sourced from bottom left corner
			(which we'll now call (0, 0)), there are four possible ways this route
			could wrap.
			
			TODO: DESCRIBE WAYS OF WRAPPING
			
			For each of these wrappings, all the possible routes we can take are
			strictly limited in terms of the dimensions used since we're stuck in a
			corner.
			
			In each case, the function computing the minimal hex vector function
			simplifies to a much simpler operation.
			
			TODO: DESCRIBE MINIMUM VECTOR LENGTH FUNCTIONS FOR EACH CASE
			
			This gives us a cheap way to compute which of the four possible wrappings
			are shortest. Based on this we can pick one of at most two (is this
			easily provable?) vectors in some fair manner to reduce load. This vector
			can then be minimised in the usual way.
			
			This also leads to confirming a theoretical result giving the length of a
			shortest path in a hexagonal torus topology.
			
			TODO: DESCRIBE HOW TO GET LENGTH FUNCTION AND COMPARE WITH \cite{xiao04}
		
		\subsection{Generating spiralling routes}
			
			In non-hexagonal torus topologies the previous technique would reveal all
			possible shortest vectors (e.g. in cases where you can wrap more than one
			way). Unfortunately, due to the addition of a non-orthogonal axes,
			hexagonal toruses also have other cases when the width and height do not
			match.
			
			TODO: TORUS SPIRALLING EXAMPLE
			
			It is possible to calculate the maximum number of spirals thus:
			
			TODO: DESCRIBE HOW MAX NUMBER OF SPIRALS IS COMPUTED
			
			Given a number of spirals, the vector can be updated this (note that the
			change does not add a multiple of (1, 1, 1) but also does not result in
			the vector changing length and thus becoming non-minimal!).
			
			TODO: DESCRIBE TRANSFORMATION
			
			TODO: PROVE THAT MINIMALITY IS MAINTAINED
		
		\subsection{Proof of completeness}
		
			TODO: PROOF OF COMPLETENESS BY EXHAUSTIVE SEARCH
	
		\subsection{Conclusions}
			
			This approach is simpler, smaller, has sounder theoretical basis, and
			finds more routes than alternatives. This is good for load balancing and
			fault avoidance and also good for completeness.


	\chapter{Routing packets in large SpiNNaker machines}
	
	\label{sec:routing}
	
	So far, this thesis has focused on tackling the practical challenges
	resulting from SpiNNaker's hexagonal torus network topology. In this chapter,
	I adjust my focus towards the practical challenges resulting from SpiNNaker's
	large scale. Faults in large systems are inevitable and in the half-million
	core, \num{28800} chip SpiNNaker machine recently completed at the University
	of Manchester, around \SI{1}{\percent} of chips exhibited faults\footnote{Of
	the faulty chips discovered, the vast majority of faults are attributed to a
	currently unknown SDRAM failure}. These faults result in gaps and broken
	links in the network topology which routing algorithms must avoid in order to
	ensure correct system operation.
	
	In this chapter I tackle the problem of extending existing routing algorithms
	for SpiNNaker's network to enable them to route-around known, static faults.
	Though dynamic or transient faults may also occur, in this work such faults
	are ignored and other techniques, such as protocol-level fault tolerance, are
	relied on instead.
	
	Numerous heuristic-based fault-tolerant routing algorithms exist which target
	different network topologies and router architectures. Unfortunately as I
	will show, these algorithms are not portable and rely on or attempt to work
	around specific features of their target network architecture. In particular,
	existing work is dominated by the challenge of developing routing schemes
	which avoid or defuse network deadlocks. Due to SpiNNaker's unconventional
	use of timeout-based flow-control, it is not subject to the routing
	restrictions present in other architectures intended to cope with deadlocks.
	
	In this chapter I introduce a graph-search based post-processing step for
	non-fault-tolerant routing algorithms which guarantees routability in
	SpiNNaker systems without disconnected subregions. I also demonstrate that
	this technique introduces both negligible computational overhead to the
	routing algorithm runtime and resulting network performance.
	
	TODO: NOTE THE FAULT RATES ENCOUNTERED IN PRACTICE...
	
	\section{Related work}
		
		Existing work on routing in SpiNNaker's network has ignored the challenge
		of avoiding faults and instead focused on producing efficient multicast
		routes. As a result this section is broken into two halves. In the first
		half I survey the existing non-fault-tolerant approaches to routing used in
		SpiNNaker to-date. In the second I discuss the approaches to fault tolerant
		routing taken in other systems.
		
		\subsection{Multicast routing in SpiNNaker}
			
			Various fault-intolerant multicast routing algorithms exist for many
			networks and a number have been proposed and evaluated specifically in the
			context of SpiNNaker.
			
			In 2012, Davies \emph{et al.} evaluated the use of three common torus
			routing algorithms in SpiNNaker and found that simple oblivious routing is
			suitable for typical neural applications \cite{davies12}. The three
			routing techniques are:
			
			\begin{description}
				
				\item[Dimension Order Routing (DOR)] Packets are routed along each
				dimension (e.g. $X$, $Y$ and $Z$) in turn until no further hops are
				available in that direction.  The order in which the dimensions are
				traversed is fixed.
				
				\item[Right Turn Only Routing (RTOR)] As in DOR except the dimension
				order is chosen such that routes only contain right-turns.
				
				\item[Longest Dimension First Routing (LDFR)] As in DOR except the
				dimension order is chosen in descending order of number of hops in each
				dimension.
				
			\end{description}
			
			These unicast routing techniques are converted into a multicast routing
			algorithm by merging together the routes produced between the source node
			and each destination node as illustrated in figure
			\ref{fig:simple-routers}.
			
			\begin{figure}
				\center
				\begin{subfigure}{0.3\linewidth}
					\center
					\buildfig{figures/simple-routers-dor.tex}
					
					\caption{DOR}
					\label{fig:simple-routers-dor}
				\end{subfigure}
				\begin{subfigure}{0.3\linewidth}
					\center
					\buildfig{figures/simple-routers-rtor.tex}
					
					\caption{RTOR}
					\label{fig:simple-routers-dor}
				\end{subfigure}
				\begin{subfigure}{0.3\linewidth}
					\center
					\buildfig{figures/simple-routers-ldfr.tex}
					
					\caption{LDFR}
					\label{fig:simple-routers-dor}
				\end{subfigure}
				
				\caption{Example multicast routes produced by merging together unicast
				routes from a central source node to each destination node.}
				\label{fig:simple-routers}
			\end{figure}
			
			In 2014, Navaridas \emph{et al.} introduced two new algorithms, `Enhanced
			Shortest Path Routing' (ESPR) and `Neighbourhood Exploring Routing' (NER)
			which produce multicast routing trees with fewer total hops
			\cite{navaridas14}. In both algorithms, routes are generated sequentially
			for each of the destinations of a route using LDFR. Unlike LDFR, however,
			these algorithms search a limited area of the network for other,
			already-connected destination nodes to which LDFR routes may be
			constructed. If no suitable destination is found, a LDFR route is
			constructed to the source node. Figure \ref{fig:search-regions} illustrates
			the shape of the searched regions of each algorithm. ESPR searches the
			trapezoidal region between the source and destination nodes while NER
			searches a fixed radius out from the destination node.
			
			\begin{figure}
				\center
				\begin{subfigure}{0.45\linewidth}
					\center
					\buildfig{figures/search-regions-espr.tex}
					
					\caption{ESPR}
					\label{fig:search-regions-espr}
				\end{subfigure}
				\begin{subfigure}{0.45\linewidth}
					\center
					\buildfig{figures/search-regions-ner.tex}
					
					\caption{NER}
					\label{fig:search-regions-espr}
				\end{subfigure}
				
				\caption{The ESPR and NER algorithms attempt to connect the node marked
				`D' to the closest node in the shaded region which is connected to the
				source node, `S'. If no connected node is found in the shaded region, the
				LDFR route is taken to `S'. The dotted line indicates the route chosen
				from `D'.}
				\label{fig:search-regions}
			\end{figure}
			
			Unfortunately none of these routing algorithms make any allowance for the
			avoidance of network faults. As a result their utility in real-world
			systems is limited.
		
		\subsection{Fault-tolerant routing}
			
			Numerous fault-tolerant routing algorithms have been proposed for
			super-computer networks. These algorithms are largely constrained by the
			need to maintain deadlock freedom. Since SpiNNaker's routers employ a
			timeout based deadlock-breaking strategy, much of this effort is
			unnecessary in SpiNNaker. As described below, this frequently renders
			existing fault-tolerant routing algorithms unnecessarily complex and
			inflexible.
			
			Deadlocks occur in a network if a cyclic dependency exists on any limited
			resource in the network. For example, as illustrated in figure
			\ref{fig:ring-deadlock}, in a ring network a deadlock may form when every
			node is waiting on the next node to accept a packet before accepting new
			packets from the previous node.
			
			\begin{figure}
				\center
				\buildfig{figures/ring-deadlock.tex}
				
				\caption{A deadlock in a ring network where each node is waiting for
				the next to accept a packet before accepting any further packets.}
				\label{fig:ring-deadlock}
			\end{figure}
			
			To prevent deadlocks, combinations of router microarchitectural features
			and routing restrictions may be employed. For example, a simple
			deadlock-free routing algorithm for mesh and torus networks mandates the
			use of DOR \cite{dally93}. Packets travelling in a -ve direction along
			each axis take priority over those travelling in a +ve direction. Packets
			travelling along the Y axis take priority over those travelling along the
			X dimension. Given these rules it is possible to define a total ordering
			on all hops in the network. Figure \ref{fig:deadlock-free-dor}
			illustrates a $3\times3$ mesh network whose hops have been numbered
			according to the total ordering defined above.  Any `X-then-Y' DOR route
			through this network results in the use of hops labelled with strictly
			increasing numbers. As a result, no cyclic dependencies (and thus no
			deadlocks) may occur.
			
			\begin{figure}
				\center
				\buildfig{figures/deadlock-free-dor.tex}
			
				\caption{Deadlock-free routing of two example routes using DOR in a 2D
				mesh topology. The numbers of the hops taken by each route are given on
				the right.}
				\label{fig:deadlock-free-dor}
			\end{figure}
			
			Unfortunately, the routing restrictions imposed to ensure deadlock
			freedom can result in fault-intolerant routing. In the example above, if
			the node at the bottom-right corner of the figure was faulty, the dotted
			example route would be blocked as no alternative routes are allowed.
			
			In practice, the routing rules used may be more relaxed, for example
			requiring that any route whose length is equal to a DOR must exist to
			guarantee routability \cite{rodrigo09}.
			
			Alternative routing strategies take a hybrid approach whereby an
			efficient but fault-intollerant routing algorithm is used where possible
			and in the presence of faults a less efficient but more robust strategy
			is employed. For example, the Immucube network architecture employs three
			virtual networks which operate independently over the same physical links
			\cite{puente07}. Initially messages are routed using a high-performance
			but potentially-deadlockable routing scheme in the first virtual network.
			If a deadlock is occurs, the deadlocked packet is dropped into the second
			virtual network in which packets are routed using a less efficient but
			deadlock-free but fault-intolerant routing algorithm. Finally, upon
			encountering a fault, packets are dropped onto the third virtual network
			which forms a ring network routing packets to every node in the network.
			
			Releated approaches \cite{mejia06,boppana95} divide the network into
			regions in which different routing rules are enforced to ensure deadlock
			freedom and, when required, fault tolerance.
			
			TODO FIGURE?
			
			The BlueGene/L supercomputer \cite{adiga02} uses DOR for its torus
			network and implements fault-tolerance by sacrificing otherwise
			functioning `lamb' nodes to ensure no route passes through a known dead
			link \cite{ho04}. In figure \ref{fig:lamb-nodes} an example scenario is
			shown where a single dead node is present and all nodes in the same row
			or column as the dead node have been made into lamb nodes. The lamb nodes
			may not be used in an application except as a through-route for other
			traffic. This pattern of lamb nodes guarantees that all dimension-order
			routes between all pairs of non-lamb nodes are not obstructed by the
			faulty node. This approach trades use of higher performance routing
			logic for wasted resources. This type of approach is most appropriate
			when algorithmic routing is used and routing rules are inflexible.
			
			\begin{figure}
				\center
				\buildfig{figures/lamb-nodes.tex}
				
				\caption{`Lamb' nodes may be disabled to ensure DOR will never
				encounter a fault.}
				\label{fig:lamb-nodes}
			\end{figure}
			
			Other algorithms proposed for the BlueGene architecture attempt to avoid
			the need for lamb nodes by generating routes which reach their destination
			via a `proxy' node \cite{gomez04}. By appropriately selecting the location
			of such a proxy, the existing routing algorithm used by the system can be
			guaranteed to select a route free of faults.
			
			TODO: EXAMPLE OF PROXY ROUTING TO AVOID FAULT
			
			Finally, many algorithms in in the field are distributed and use only local
			information along with limited information from their peers to generate
			routes \cite{fick09b}. In SpiNNaker, route generation is conventionally
			carried out centrally since no special on-chip hardware facilities exist
			for route generation. Centralised route generation also enables the routing
			algorithm to consider all available routes. As a result, there is little
			incentive for the use of distributed routing algorithms on SpiNNaker since
			global system information could be compactly shared for one-off routing
			passes.
			
			Algorithms for other architectures such as IP networks tend to be poor fits
			for static, regular network topologies since they use expensive graph-based
			algorithms for route discovery which aren't necessary here. They also tend
			to heavily feature graph topology discovery etc. which aren't needed here.
			
			Work on fault-tolerance in data centre networks does exploit the regularity
			of the network topology in routing algorithms \cite{guo08,liao12}.
			Unfortunately, the approaches used are not general enough to be applied to
			mesh-like topologies such as the one in SpiNNaker.
			
			Outside the field of computer networks, routing algorithms used to route
			wires across the surfaces of chips are required to solve similar problems
			to fault-tolerant network routing problems in mesh networks. Like mesh
			networks, the routes are defined within a regular Manhattan geometry and
			congested areas, rather than faults must be avoided by the algorithms
			\cite{kahng11}.  Unfortunately, these algorithms are designed for
			occasional batch operation prior to the multi-month process of chip
			manufacturing and so runtimes of hours or days are commonplace
			\cite{nam08}. As such these algorithms would be inappropriate for use
			with applications such as SpiNNaker where users' applications tend to be
			short-lived and thus routing should not be allowed to dominate runtime.
	
	\section{Partial graph search repair}
		
		In this section I introduce a novel post-processing algorithm, Partial
		Graph Search (PGS) repair, for routes produced by non-fault-tolerant
		routing algorithms.
		
		PGS repair guarantees routability for networks with no disconnected
		subregions by using a graph search algorithm to route around faults in the
		original route.  General-purpose graph search algorithms such as Breadth
		First Search (BFS), Dijkstra's Algorithm and A* are guaranteed to find
		shortest-path routes between pairs of points in arbitrary graphs. Such
		algorithms are generally a poor choice in highly regular network topologies
		such as meshes and toruses due to their high computational cost. In PGS
		repair, graph searching is only used for \emph{part} of the routing
		problem: to repair gaps in routes generated by more efficient routing
		algorithms.
		
		Real world super computer architectures are designed to ensure that faults
		are isolated \cite{gara05,alverson12} and thus tend to only impact a
		localised region of the network. Since PGS repair is only needed to route
		around these isolated faults, the space searched by the graph search
		algorithm should be very small in practice resulting in only short
		runtimes. In addition since faults are rare in real-world systems, the
		graph search process will only rarely be invoked.
		
		The PGS repair post-processing technique starts with a route produced by a
		non-fault-tolerant routing algorithm such as ESPR or NER. If this route is
		not obstructed by a fault, the algorithm terminates immediately without
		modifying the route. If the route attempts to use a faulty link, the
		algorithm proceeds as follows.
		
		The routing tree produced by the underlying routing algorithm is broken
		into subtrees wherever it attempts to route through a broken link and
		each subtree is assigned a unique colour, as illustrated in figure
		\ref{fig:pgs-repair-colouring}. From each disconnected subtree's root
		node in turn, a graph search is performed to find a short, fault-free
		route to a subtree node of a different colour. The subtree is then
		attached to the tree discovered by the graph search and re-coloured to
		match the tree it is connected to.
		
		\begin{figure}
			\center
			\begin{subfigure}{0.32\linewidth}
				\hspace*{-1.5em}
				\buildfig{figures/pgs-repair-colouring.tex}
				
				\caption{}
				\label{fig:pgs-repair-colouring}
			\end{subfigure}
			\begin{subfigure}{0.32\linewidth}
				\hspace*{-1.5em}
				\buildfig{figures/pgs-repair-colouring-fix1.tex}
				
				\caption{}
				\label{fig:pgs-repair-colouring-fix1}
			\end{subfigure}
			\begin{subfigure}{0.32\linewidth}
				\hspace*{-1.5em}
				\buildfig{figures/pgs-repair-colouring-fix2.tex}
				
				\caption{}
				\label{fig:pgs-repair-colouring-fix2}
			\end{subfigure}
			
			\caption{PGS repair process example showing a disconnected multicast
			route from A to B, C, D, E and F. $\times$ indicates a broken link.}
			\label{fig:pgs-repair-colouring-steps}
		\end{figure}
		
		For example in figure \ref{fig:pgs-repair-colouring-fix1} a path from the
		root of the subtree containing nodes E and F is found which connects it to
		the subtree rooted at A. Similarly in figure
		\ref{fig:pgs-repair-colouring-fix2} a path is also found connecting the
		subtree containing nodes C and D back to the subtree rooted at node A.
		
		If the routing tree was broken into $N+1$ subtrees by faults there will be
		$N$ subtrees disconnected from the root node. Each of the $N$ iterations of
		the algorithm connect a disconnected subtree to another subtree reducing
		the number of subtrees by $1$ each time. After $N$ iterations, therefore,
		exactly $1$ subtree remains which connects every node in the original
		routing tree without traversing faulty links.
		
		TODO: EXPLAIN THE FIDDLINESS HERE TO ENSURE WE DON'T CREATE LOOPS.
		
	\section{Evaluation \& Results}
		
		The PGS repair technique, by design, is able to work around all possible
		fault patterns which don't completely disconnect parts of the network. This
		result this evaluation focuses on the impact on performance PGS repair
		imposes. The metrics of interest in this evaluation are:
		
		\begin{itemize}
			\item Algorithm runtime
			\item Network congestion
			\item Routing table utilisation
		\end{itemize}
		
		\subsection{Traffic Patterns}
			
			In this evaluation, two standard benchmark multicast traffic patterns are
			used which have been used in previous research into SpiNNaker's network:
			
			\begin{figure}
				\center
				\buildfig{figures/traffic-distribution-centroids.tex}
				
				\caption{An example 4-centroid distribution with four centroids. The
				$\times$ marks the location of the origin node. Lighter colours
				indicate greater likelihood of a connection.}
				\label{fig:traffic-distribution-centroids}
			\end{figure}
			
			\begin{description}
				
				\item[Uniform] Destinations are chosen with uniform probability
				anywhere in the machine.
				
				\item[$N$-Centroids] Destinations are clustered around one of $N$
				randomly chosen `centroids' as illustrated in figure
				\ref{fig:traffic-distribution-centroids}.
				
			\end{description}
			
			The uniform traffic pattern is widely used in networks research
			\cite{dally04,davies12} while the centroids model was developed
			specifically to reproduce the traffic patterns found in the neural
			applications SpiNNaker is designed for \cite{navaridas14}. In this work
			we consider 3 centroids.
		
		\subsection{Fault model}
			
			In addition two different fault models are used which are representative of
			the faults found in real SpiNNaker systems:
			
			\begin{figure}
				\center
				\begin{subfigure}{0.48\linewidth}
					\hspace*{-1.5cm}
					\buildfig{figures/fault-example-uniform.tex}
					
					\caption{Uniform}
					\label{fig:fault-example-uniform}
				\end{subfigure}
				\begin{subfigure}{0.48\linewidth}
					\hspace*{-1.5cm}
					\buildfig{figures/fault-example-hss.tex}
					
					\caption{HSS Link}
					\label{fig:fault-example-hss}
				\end{subfigure}
				
				\caption{The two link fault models considered.}
				\label{fig:fault-example}
			\end{figure}
			
			\begin{description}
				
				\item[Uniform] Links are selected and disabled at random (figure
				\ref{fig:fault-example-uniform}).
				
				\item[HSS Link] Groups of links corresponding with randomly selected
				single High-Speed Serial (HSS) link between SpiNNaker boards are disabled
				together (figure \ref{fig:fault-example-uniform}).
				
			\end{description}
			
			The uniform link failure model models isolated failures resulting from
			isolated manufacturing defects in individual links. The HSS Link failure
			model models faults arising from failing or disconnected board-to-board
			links which carry several chip-to-chip traffic flows via a single cable in
			SpiNNaker systems. Though SpiNNaker-specific, the later fault model is
			analogous to failure modes arising in other architectures where a single
			fault may render several links impassable in a single area.
			
			A range of failure rates are explored in this section. My measurements of
			current large-scale SpiNNaker installations the link failure rate is about
			\SI{0.03}{\percent} with failures due to both individual chip-to-chip links
			and board-to-board HSS links. Exact link failure statistics for commercial
			super computer installations are not widely available, however, published
			Mean-Time-Between-Failure (MTBF) statistics place an upper bound on link
			failure rates at a similar \SI{0.03}{\percent} in one-year-old BlueGene/Q
			systems \cite{chiu11}.
			
			Unfortunately presently undiagnosed problem with the SDRAM packaged with
			approximately \SI{1}{\percent} of SpiNNaker chips has rendered these chips
			unusable for most applications. The gaps in the network resulting from the
			loss of these chips currently dominate true link failures leaving just over
			\SI{1}{\percent} of links inoperable.
			
			Surprisingly, research into fault tolerant routing in super computers
			appears to focus on benchmarks with even higher fault rates ranging from
			\SI{3}{\percent} to as high as \SI{7}{\percent}
			\cite{ho04,gomez04,mejia06}.
			
			In this evaluation, fault rates ranging from \SI{0.01}{\percent} to
			\SI{5}{\percent} are considered to cover both realistic fault levels
			along with the more extreme cases considered in related work.
		
		\subsection{Base routing algorithm}
			
			Since the PGS repair process is routing algorithm agnostic all
			experiments use the NER algorithm which has been found to be appropriate
			for SpiNNaker applications \cite{navaridas14}.
		
		\subsection{Algorithm runtime}
			
			To assess the impact of the PGS repair process on routing algorithm
			runtime, the algorithm was used to process a large number of randomly
			generated routing problems and the runtime recorded.
			
			\num{10000} one-to-sixteen multicast routing problems were generated in a
			$256\times256$ hexagonal torus topology, the largest size possible for a
			SpiNNaker system. Other quantities of multicast destinations were also
			evaluated but are omitted for brevity since the pattern of results are
			similar to those outlined here.
			
			TODO: APPENDIX WITH OTHER RUNS?
			
			The NER and PGS repair algorithms were written in C and compiled with GCC
			4.8.3 with \verb|-O2| level optimisations and executed on a cluster of
			idle workstations with 3.10 GHz Intel Core-i5-2400 CPUs.
			
			\begin{figure}
				\center
				\buildrplot{figures/routing-runtimes.R}
				
				\caption{Mean runtime of routing and PGS repair overhead. PGS repair
				overhead is stacked above the routing runtime (i.e. bars do not
				overlap). Error bars indicate 95\% confidence interval. Note different
				Y-scale for HSS link and uniform fault models.}
				\label{fig:routing-runtimes}
			\end{figure}
			
			Figure \ref{fig:routing-runtimes} shows the average runtimes recorded for
			both the NER and PGS repair algorithms. In fault-free networks the
			PGS-repair post-processing step is not required and incurs no penalty
			while the runtime of the algorithm grows with the fault rate for both
			fault and traffic models.
			
			Notably the HSS fault model results in longer runtimes for the PGS repair
			process compared with an equivalent fault-density of uniform faults.
			Because the HSS fault model produces contiguous lines of faults the PGS
			repair algorithm must construct a longer path to avoid the fault.  Since
			the space explored by a graph algorithm typically grows with $O(H^2)$
			with respect to the hops in the discovered route, $H$, this increase in
			search distance has a large impact on the runtime of the PGS repair
			process.
			
			The runtime of the PGS repair algorithm remains roughly in proportion to
			the runtime of the underlying routing algorithm with respect to different
			traffic models. The centroid traffic pattern tends to result in routes
			with fewer hops than a uniform traffic pattern with the same number of
			destination nodes as segments of routes are often shared between
			destination nodes. Since the NER algorithm's runtime is strongly related
			to the number of hops in the output route the runtime of the algorithm is
			greater for uniform traffic. Likewise the probability of PGS repair being
			required increases with the number of hops in route and hence the runtime
			of the PGS repair algorithm increases roughly in proportion.
		
		\subsection{Routing table usage}
			
			In order to gain a realistic measure of routing table usage it is
			necessary to determine the effect of many routes being generated for a
			single set of faults. To enable a sufficiently large number of sample to
			be collected the experimental setup considered previously is reduced to a
			network containing $48\times48$ nodes.
			
			\num{1000} $48\times48$ node network models are produced according to the
			HSS link and uniform fault models. For each of these models
			$48\times48\times16=$~\num{36864} one-to-sixteen routes are generated using
			the centroid and uniform traffic models. This corresponds to one
			multicast route per application core. As is convention in SpiNNaker,
			routing table entries are inserted for each route at the source of the
			route, at each destination and at each corner or fork. The number of
			routing table entries at each node in the model is counted and the
			maximum number of entries in a single node is reported for each network
			model.  The \emph{maximum} number of routing entries of any router was
			chosen since the number of entries available per SpiNNaker router is
			bounded by hardware.
			
			\begin{figure}
				\center
				\buildrplot{figures/routing-entries.R}
				
				\caption{Violin plot showing the distribution of maximum table sizes
				for \num{1000} random networks. The red line at \num{1024} entries
				indicates the size of SpiNNaker's routing tables.}
				\label{fig:routing-entries}
			\end{figure}
			
			
			Figure \ref{fig:routing-entries} shows the distributions of the largest
			routing table sizes for each fault and traffic model.
			
			\begin{figure}
				\center
				\begin{subfigure}{0.48\linewidth}
					\center
					\buildfig{figures/hss-link-routing-table-usage.tex}
					
					\caption{Routing table entries}
					\label{fig:hss-link-routing-table-usage}
				\end{subfigure}
				\begin{subfigure}{0.48\linewidth}
					\center
					\buildfig{figures/hss-link-resource-usage.tex}
					
					\caption{Routes passing through chip}
					\label{fig:hss-link-resource-usage}
				\end{subfigure}
				
				\caption{The impact of a HSS link fault on routing table usage and
				congestion. Each hexagon represents a single chip, the red line
				indicates the chip-to-chip connections broken by the HSS link fault.}
				\label{fig:hss-link-usage}
			\end{figure}
			
			The HSS link failure model has a much greater impact on peak routing
			table resource usage than uniform link failures for a given fault rate.
			This is because HSS link faults result in a large concentration of routes
			being disrupted and then re-routed around the same obstacle in a single
			location. Figure \ref{fig:hss-link-routing-table-usage} shows how routing
			table usage varies around a HSS link fault in one instance of the
			experiment. There are clear peaks in routing table usage around the ends
			of the line of faults which result from routes produced by PGS repair
			finding shortest paths around the edge of the faults.
		
		\subsection{Network congestion}
			
			To measure the impact of PGS repair on network congestion, two
			experiments were performed, one using the same model used to measure
			routing table usage and one based on tests run on SpiNNaker hardware.
			
			For each of the network fault and traffic pattern described previously,
			the paths taken for the \num{36864} one-to-sixteen multicast routes
			generated are used to compute the number of times each link in the
			network is used. The number of routes passing through the most-used link
			is then recorded, giving an indication of the level of congestion in the
			network.
			
			\begin{figure}
				\center
				\buildrplot{figures/routing-resource.R}
				
				\caption{Violin plot showing the distribution of maximum
				routes-per-chip for \num{1000} random networks.}
				\label{fig:routing-resource}
			\end{figure}
			
			The results are presented in figure \ref{fig:routing-resource} and follow
			the same trends as the results previously shown for routing table usage.
			Again, HSS link faults result in routes with the greatest congestion due
			to the concentration of routes finding shortest paths around an obstacle
			(see \ref{fig:hss-link-resource-usage}).
			
			To verify that the results above, an additional experiment has been
			carried out which attempts to mimic the model used previously in actual
			SpiNNaker hardware. In these experiments a large SpiNNaker machine is
			divided into independent 48-board (2304-chip) sections. Because the
			48-board systems used in these experiments are cut out of a larger
			machine, they lack wrap-around links and thus form hexagonal mesh
			topologies, rather than hexagonal toruses.
			
			Due to the SDRAM issue described above, fault rates below
			\SI{1}{\percent} cannot be modelled.  To simulate higher fault rates,
			additional links are disabled in software according to the fault models
			described used previously. Since some faults are due to genuine hardware
			faults, these faults cannot be placed randomly in each experiment. To
			reduce, bias each combination of fault rate, fault model and traffic
			pattern is repeated XXX times across randomly chosen physical machines.
			
			XXX 1-to-XXX routes are generated in both uniform and XXX-centroid
			distributions as used throughout this evaluation. Synthetic network
			traffic is generated at the source of each route following a Bernoulli
			distribution. Traffic consumers running on all destination nodes accept
			packets as quickly as possible from the network and log their arrival.
			The Bernoulli probability is set the same for every route's traffic
			generator and increased in steps of XXX and the number of packets dropped
			in an XXX second period logged. The network is considered saturated once
			less than \SI{99}{\percent} of packets successfully arrive at their
			destination.
			
			Figure \ref{XXX} shows the distributions of the saturation points for
			each experimental configuration.
			
			TODO: ANALYSIS
		
	\section{Conclusions}
		
		In this chapter I described how SpiNNaker's unconventional network and
		router architecture render existing fault tolerant routing algorithms
		unsuitable. I introduced PGS repair, a post-processing technique for
		existing non-fault tolerant routing algorithms designed for SpiNNaker such
		as NER.
		
		Unlike some other fault tolerant routing algorithms for other
		architectures, PGS repair is able to work-around arbitrary fault patterns
		by exploiting SpiNNaker's inbuilt deadlock avoidance mechanisms. In the
		presence of realistic failure rates of up to \SI{1}{\percent}, only small
		overheads of up to XXX, XXX and XXX for in algorithm runtime, routing table
		usage and network performance are incurred respectively. This low
		performance overhead makes PGS repair appropriate for use in real
		applications. At the time of writing the algorithm has been successfully
		used in a number of neural and non-neural SpiNNaker applications.
		
		At more extreme fault rates not expected in real-world systems, the
		algorithm still functions correctly but the results incur much greater
		routing table and congestion overheads, particularly when faults are
		concentrated. Future extensions to this algorithm might aim to reduce this
		overhead by producing longer and more varied routes around faults to even
		out the load.

	\chapter{Placing applications in large SpiNNaker machines}
	
	In the previous chapter I tackled the problem of scale in generating routes
	for very large networks such as SpiNNaker. In this work the centroid traffic
	pattern was used as an approximation of the expected network traffic
	generated by `well behaved' neural network simulation software running on
	SpiNNaker. The traffic produced largely exhibits strong locality, that is
	most communication occurs between either nearby nodes or clusters of nodes.
	In reality, neural simulation applications are not specified geometrically
	but rather as abstract graphs of communicating neurons
	\cite{davison08,eliasmith13}. Applications must then \emph{place} these
	neurons onto nodes in a SpiNNaker system, attempting maximise communication
	locality.
	
	In this chapter I re-evaluate the suitability of simulated annealing as a
	technique for finding high quality placements for large parallel
	applications. Though this technique had fallen out of fashion in the field of
	application placement by the early 1990s, it has found wide use for placing
	components in computer chip and FPGA designs. In the intervening years,
	placement problems in super computers have grown in size from tens or
	hundreds of nodes to millions, a scale at which chip placement techniques
	were operating in the mid 1990s. I adapt the simulated annealing algorithm
	used by the VPR academic circuit placement software to produce placements for
	applications running on SpiNNaker. In that in a range of real and synthetic
	benchmarks simulated annealing produces high quality placements enabling
	efficient use of SpiNNaker's network resources.
	
	
	%In the field of chip design, Moore's `Law' \cite{moore65,moore75} observes a
	%similar exponential growth in the number of components within a single chip.
	%Today modern processors contain billions of components and an analagous
	%placement problem exists in attempting to place interconnected components
	%near to eachother. In this chapter I explore the techniques used for circuit
	%placement and adapt one such technique, Simulated Annealing (SA)
	%\cite{kirkpatrick83}, for use in application placement. Despite some early
	%interest in SA for application placement in the 1980s and early 1990s, the
	%technique has since fallen out of favour. I find that at the scales of modern
	%placement problems SA-based placement is able to produce solutions of
	%superiour quality to contemporary methods.
	%
	%TODO: SUMMARISE RESULTS...
	
	\section{Related work}
		
		The placement problem has been tackled independently in the literature by
		researchers in both the application and chip placement communities. In this
		survey I cover application and chip placement separately as these two
		communities have remained largely isolated from one another. First I
		explore the techniques applied to application placement before moving on to
		contrast this with the techniques used in circuit placement.
		
		In the application placement literature, the placement problem is often
		referred under the umbrella term `mapping'. Unfortunately term is often
		used more broadly to include other tasks such as routing and application
		partitioning. To avoid ambiguity I use the term `placement', as preferred
		by the chip and FPGA design communities, to refer specifically to the
		problem of assigning nodes in an application's communication graph to nodes
		in a machine's connectivity graph.
		
		\subsection{Application placement algorithms}
			
			TODO: GENERAL INTRO
			
			\subsubsection{Application-specific approaches (manual placement)}
				
				In the case of some applications such as finite element modelling
				\cite{bermejo13}, the structure of the problem itself leads to a
				natural placement of the computation on nodes in a machine. For example
				when simulating a 3D volume in an node super computer with a $3 \times
				4 \times 2$ 3D torus or mesh topology network, the modelled volume
				might be divided into as in figure \ref{fig:fem-partitioning}. Each
				cuboid in the model is then assigned to the corresponding node in the
				network topology.
				
				\begin{figure}
					\center
					\buildfig{figures/fem-partitioning.tex}
					
					\caption{Example partitioning of a 3D space to fit into a super
					computer with a $3\times4\times2$ torus or mesh topology.}
					\label{fig:fem-partitioning}
				\end{figure}
				
				When the number of dimensions in a problem do not match that of the
				underlying network architecture, the common solution is to either
				divide only along a subset of the axes or to divide into additional
				pieces on the existing axes \cite{gilge14}.
			
			\subsubsection{Sequential placement}
				
				In the case where a placement solution is non-obvious one of the
				simplest and most popular strategies is to apply a simple sequential
				placement algorithm. Sequential placement algorithms function by
				iterating over the vertices in the application's communication graph
				and assigning them to a free node in the target machine. Sequential
				placement algorithms are differentiated by the order in which they
				iterate over vertices in the communication graph and fill nodes in the
				target machine. A number of widely used orderings are described below.
				
				\begin{figure}
					\center
					\begin{subfigure}{0.32\linewidth}
						\center
						\buildfig{figures/sequential-row-order.tex}
						\caption{Row-order}
						\label{fig:sequential-row-order}
					\end{subfigure}
					\begin{subfigure}{0.32\linewidth}
						\center
						\buildfig{figures/sequential-alternating.tex}
						\caption{Alternating}
						\label{fig:sequential-alternating}
					\end{subfigure}
					\begin{subfigure}{0.32\linewidth}
						\center
						\buildfig{figures/sequential-hilbert.tex}
						\caption{Hilbert curve}
						\label{fig:sequential-hilbert}
					\end{subfigure}
					
					\caption{Space-filling curves in 2D mesh and torus topologies.}
					\label{fig:sequential}
				\end{figure}
				
				Super computer management software such as SLURM \cite{yoo03} and Blue
				Gene's system software \cite{gilge14} by default na\"ively iterate over
				vertices in an application communication graph in the order they are
				provided. The nodes in the target machine are then iterated over in a
				simple space-filling curve through the network topology. Figure
				\ref{fig:hilbert-placement} illustrates the default patterns available
				in these software packages. The row-order (figure
				\ref{fig:sequential-row-order}) and alternating (figure
				\ref{fig:sequential-alternating}) curves correspond with 2D versions of
				the default node assignment orders used in SLURM and BlueGene systems.
				
				\begin{figure}
					\center
					\buildfig{figures/hilbert-placement.tex}
					
					\caption{A Hilbert curve, coloured from blue to red.}
					\label{fig:hilbert-placement}
				\end{figure}
				
				The Cray extensions to SLURM software provide a Hilbert curve
				\cite{hilbert91} (figure \ref{fig:sequential-hilbert}) node assignment
				order. Unlike the row-order and alternating space filling curves the
				Hilbert curve ensures that pairs of vertices close together in the node
				iteration order are also close together in the target machine's network
				\cite{moon01, zumbusch99}. Figure \ref{fig:hilbert-placement} shows a
				5$^\textrm{th}$-order Hilbert curve where each point in the curve is
				coloured according to its position along the curve. In this figure it
				is possible to see that nearby positions in the curve (which share
				similar colours) are also close in 2D space.
				
				When the proximity of vertices in the vertex-ordering supplied by an
				application is a good estimator of those vertices communication
				requirements, the sequential assignment schemes discussed above can be
				very effective. These techniques have also proven adequate in
				small-scale and densely connected applications such as early neural
				simulations running on prototype SpiNNaker machines with tens of nodes
				\cite{galluppi10} but growing beyond this scale has proven problematic.
				
				\begin{figure}
					\center
					\begin{subfigure}{0.45\linewidth}
						\center
						\buildfig{figures/rcm-initial.tex}
						
						\caption{Original permutation}
						\label{fig:rcm-initial}
					\end{subfigure}
					\begin{subfigure}{0.45\linewidth}
						\center
						\buildfig{figures/rcm-sorted.tex}
						
						\caption{RCM permutation}
						\label{fig:rcm-sorted}
					\end{subfigure}
					
					\caption{Adjacency matrix representation of a graph before and after
					permutation by the RCM algorithm.}
					\label{fig:rcm}
				\end{figure}
				
				A number of algorithms have been proposed for automatically selecting
				good vertex iteration orders, typically using a graph-traversal based
				heuristic. A typical method, described by Hoefler \emph{et al.}
				\cite{hoefler11} exploits the Reverse-Cuthill-McKee (RCM) algorithm
				\cite{cuthill69}. An application's communication matrix is represented
				as an adjacency matrix, $M$, where $M_{i,j}$ is 1 if node $i$ is
				connected by an edge to node $j$ and 0 otherwise. An example matrix is
				illustrated in figure \ref{fig:rcm-initial}. The RCM algorithm uses a
				simple heuristic to permute the matrix (i.e. renumber the nodes in the
				graph) in order to reduce the bandwidth of the matrix. Figure
				\ref{fig:rcm-sorted} shows the RCM-permuted version of the example
				adjacency matrix. When a graph's vertices are ordered as in a
				bandwidth-reduced sparse matrix, vertices close together in the
				ordering are likely to communicate while those further apart tend not
				to communicate.
				
			\subsubsection{Optimisation-based Placement}
				
				% Citations from short report about optimisation in placement...
				% \cite{chen06,jeannot14} and \cite{jeannot10} ("subsets of apps")
				
				In the academic community, a number of attempts have been made to use
				more sophisticated optimisation algorithms for the placement of
				applications. In 1985, Steele \cite{steele85} proposed the use of
				simulated annealing for placing applications in the 6D torus topology
				of the 64 node `Caltech Cosmic Cube' machine. Simulated annealing,
				originally developed by Kirkpatrick \emph{et al.} \cite{kirkpatrick83},
				is a general-purpose optimisation algorithm which works by analogy to
				the physical process of annealing. In brief simulated annealing
				functions by randomly swapping vertices in a candidate placement
				solution, accepting swaps which move connected vertices closer together
				and rejecting some proportion of swaps which move connected vertices
				further apart. The simulated annealing algorithm is described in detail
				later in this chapter.
				
				Towards the end of the 1980s, application placement appeared to be
				becoming less important as super computer network architectures
				improved:
				%
				\begin{displayquote}
					``Careful placement was necessary because of the slow communication
					and non-uniform addressing of early concurrent computers. However,
					the development of message passing machines with fast communications
					and a uniform global address space  has made placement less of an
					issue. In such machines a random placement performs nearly as well as
					an optimum placement.''
					
					\noindent --- W. Dally, 1987 \cite{dally87}
				\end{displayquote}
				%
				In addition, network and problem sizes remained small, so small in fact
				that linear-programming based optimal placement still appeared in
				benchmarks comparing placement algorithms \cite{xu91}. In this
				environment, simpler sequential placement algorithms gained favour over
				more computationally expensive algorithms such as simulated annealing.
				
				As problem and machine sizes have grown and network utilisation has
				once again become an important factor in application performance
				\cite{navaridas09b} more complex optimisation algorithms have
				reappeared in the literature. One popular approach employs graph
				partitioning algorithms such as METIS \cite{karypis98} to perform
				recursive bipartitioning based placement
				\cite{phillips14,hoefler11,pellegrini96}.  This placement process is
				illustrated in figure \ref{fig:partitioning}.
				
				In the first step, the application communication graph and machine
				connectivity graph are bipartitioned such that the number of edges
				between partitions is minimised. Each half of the communication graph
				is associated with one of the halves of the machine connectivity graph.
				The partitioning process is then repeated recursively on each of the
				two communication and connectivity graph pairs. The process halts when
				the graphs can no longer be partitioned at which point the vertices in
				the communication graph are placed on their associated node.
				
				\begin{figure}
					\center
					\buildfig{figures/partitioning.tex}
					
					\caption{Illustration of application placement by recursive
					partitioning.}
					\label{fig:partitioning}
				\end{figure}
				
				TODO: PARTITIONING IS GREAT AND ALL BUT QUALITY ISN'T ALWAYS GREAT AND
				IT DOESN'T DEAL WELL WITH MULTI-CONSTRAINT SCENARIOS E.G. PROCESSOR AND
				MEMORY RESTRICTIONS.
				
				Unfortunately, many of these simply aren't suited to the scale of
				neural applications running on SpiNNaker (e.g. only cope with tens of
				nodes while SpiNNaker may contain hundreds of thousands).
				
				Additionally, a number of algorithms have been developed which make
				assumptions about the topologies of the problem or network. Tree match
				for example attempts to map tree-shaped problems to tree-shaped
				networks. Such algorithms can be highly effective but again do not
				apply to SpiNNaker or its neural applications.
		
		\subsection{Chip placement algorithms}
			
			The chip-design industry has, for many years, dealt with problems
			analogous to the task of placing super computer jobs in a way suited to
			SpiNNaker. Modern CPUs have millions or billions of components with
			strictly fixed connectivity. CPU designers must place each of these onto
			a chip such that the connection lengths are controlled to reduce
			congestion and increase performance. As such, these algorithms are
			ideally suited to future super computer placement work since they already
			operate at the scales required \cite{nam07}.
			
			\subsubsection{Cost functions}
				
				HPWL is popular but a bit crap for high fan-outs. It is, however, quite
				simple.
				
				TODO: SELECT A BETTER COST FUNCTION...
			
			\subsubsection{Simulated annealing}
				
				One of the oldest techniques used for circuit placement is simulated
				annealing and this remains popular today thanks to its sheer
				versatility (see VPR, other open FPGA tools).
				
				SA works by analogy with the physical process of annealing.
				The simulated annealing algorithm works by selecting random pairs of
				components on a chip, swapping them and evaluating some cost function.
				If the swap reduces the cost function, it is kept, if not, depending on
				a function of the current temperature and the cost introduced by the
				swap.
				
				TODO: ILLUSTRATION OF SIMULATED ANNEALING SWAP OPERATION
				
				By occasionally allowing costly swaps, the annealing algorithm avoids
				becoming trapped in local minima. As the algorithm proceeds, the
				temperature is slowly reduced and with it the proportion of costly
				swaps which are retained. This causes the placement to move from
				exploration early on towards refinement later on.
				
				The temperature schedule of an annealing algorithm is critical to its
				success. In general these schedules are computed based on the
				performance of the algorithm as it runs. In VPR the following schedule
				is used.
				
				TODO: DESCRIBE VPR'S SCHEDULE
				
				TODO: FIND AND DESCRIBE ALTERNATIVE SCHEDULE?
				
				Unfortunately, SA is very difficult to parallelise, especially in the
				case of placement. As a result, its scalability has been limited and
				resulted in significantly reduced usage in recent work.
			
			\subsubsection{Partitioning placement}
				
				Partitioning based placement solves the placement problem using
				graph-partitioning recursively on the problem graph to assign each part
				of the circuit to some area in the super chip. Though a number of
				algorithms have proven successful in academic placement contests over
				the years, they are not popular in industrial settings.
			
			\subsubsection{Analytical placement}
				
				In analytical placement, cost function for the circuit graph is
				approximated in a form which is amenable to solutions with standard
				numerical or symbolic algebraic techniques. Using these techniques,
				exact minimum cost (in terms of the approximation) configurations can
				be obtained.
				
				Quadratic placement is a popular analytical placement technique which
				approximates the cost of a placement as the sum of the squares of the
				distances between connected circuit elements.
				
				TODO: FIGURE EXAMPLE QUADRATIC PLACEMENT PROBLEM AND SOLUTION
				
				As such this gives a quadratic cost function like so which we must
				minimise.
				
				TODO: QUADRATIC COST EQN
				
				To minimise the function we differentiate and solve using simple
				symbolic manipulation.
				
				TODO: QUADRATIC COST SOLUTION
				
				Unfortunately, quadratic placement doesn't contain any congestion
				relief by default so various schemes exist. For example, extra anchor
				nodes are inserted which gently pull the circuit components apart from
				each other. As a result, the algorithm generally proceeds by iterating,
				regenerating anchors each time.
				
				Other non-quadratic analytical methods exist too with numerical
				solutions. The approaches are often similar.
			
			\subsubsection{Hierarchical clustering}
				
				Many placement algorithms scale super-linearly with problem size and so
				larger problems become increasingly problematic to handle. To solve
				this problem clustering techniques are first applied to first simplify
				the placement problem. A solution is then found at the coarse level and
				then hierarchically fleshed out.
				
				Various clustering algorithms are in use.
				
				TODO: TALK ABOUT CLUSTERING IN PLACEMENT...
				
				TODO: DESCRIBE THE ALGORITHM I IMPLEMENTED.
	
	\section{Application placement by simulated annealing}
		
		\label{sec:placement-by-annealing}	
		
		I have implemented a simplified SA based application placement algorithm
		based on the approach used in the popular VPR place and route tool chain.
		The algorithm is written in C and is optimised for experimentation rather
		than performance but is production-ready. It has been integrated into the
		`Rig' SpiNNaker software tools and has been used to place very large
		simulations. More on that later.
		
		\subsection{Representation}
			
			Model each chip as having a quantity of various resources (e.g. Cores,
			SDRAM) available. The application graph consists of vertices which each
			consume some quantity of these resources. Vertices must be placed on a
			single chip such that the resources required on a given chip do not
			exceed those available. Vertices are then interconnected by 1:N nets with
			weights which act as hints. The nets are treated as a soft constraint:
			vertices connected via a net will, ideally, be placed near to each other,
			with priority being given to nets with higher weights. Additionally there
			will be a list of placement constraints (see later).
			
			A key observation is that while vertices in an application may frequently
			have a 1:1 correspondence with application cores, this need-not be the
			case. For example, a vertex may represent a block of SDRAM which is
			shared. A vertex may also represent some other resource, for example,
			external IO availability. By making these resource types user-defined,
			applications programmers can express flexible hard-constraints on their
			application.
			
			Another observation is that generic soft constraints can be expressed may
			be expressed using a net with an appropriate weight.
			
			As a result of these facilities, application programmers can easily
			express their own application-specific hard and soft placement
			constraints without having to modify the algorithm. This representation
			has become a de-facto standard for placement problem interchange for
			SpiNNaker applications.
		
		\subsection{Cost function}
			
			At present I've used HPWL despite this being really bad for high-fan-out
			multicast and totally ignorant to the hexagonal nature of SpiNNaker...
			
			To compute bounding boxes for tori I use the following approach. For each
			dimension, sort the points on that dimension and find the largest gap
			between them on a ring. The bounding box goes the other way.
			
			TODO: FIGURE ILLUSTRATING BOUNDING BOX COMPUTATION FOR TORI.
		
		\subsection{Annealing schedule}
			
			The annealing schedule is that used by VPR. Despite being for circuit
			placement, it seems to work jolly well.
			
			TODO: DESCRIBE AND RATIONALISE THE SCHEDULE
		
		\subsection{Constraint handling}
			
			Various hard and soft constraints may be expressed by software
			approaches. For each we explain how they may be handled by the placement
			algorithm:
			
			\subsubsection{Location Constraint}
				
				The vertex is placed on a chip and removed from the set of movement
				candidates.
			
			\subsubsection{Same-chip constraint}
				
				When two vertices must always be placed on the same chip they are
				simply combined into one vertex which consumes the sum of their
				resources. Placement then treats them as one chip and thus is forced to
				atomically place the vertices.
			
			\subsubsection{Reserve resource constraint}
				
				Simply reduce resource availability on that chip.
			
			\subsubsection{Keep near Ethernet}
				
				Simply add a net.
	
	\section{Evaluation}
		
		\label{sec:placement-results}
		
		Though benchmarks exist for super computer loads and chip placement tasks,
		such things don't exist for neural applications. As a result I use a
		selection of real applications for SpiNNaker along with some synthetic
		benchmarks based on biological data.
		
		\subsection{Benchmark networks}
			
			First some real networks.
			
			Some nengo networks: SPAUN: `The world's largest functional brain model'.
			Word-net network from Jamie: Example of some learning.
			
			TODO: DESCRIBE SHAPE OF NENGO NETWORKS
			
			Some PyNN networks: Microcortical column model from PyNN. Note almost
			broadcast connectivity but varying weights. Try and extract a vision
			netlist from Anna. Maybe try and get a netlist for Tom's barrel cortex.
			
			Now for some artificial networks. Pipeline, noisy pipeline, mesh,
			Gaussian 2D.
		
		\subsection{Experiments}
			
			We compare random, linear, greedy and annealing based placement
			approaches to placement. We compare static metrics (such as mean/max
			congestion, table usage) along with experiments based on simulated
			network traffic in real hardware. Network Tester generates artificial
			traffic in proportion with the weights given for each model. We compare
			the relative level of traffic sustainable. We also consider use of
			machines of various sizes.
		
		\subsection{Results}
			
			SA placement is slow but rather effective, especially for some networks.
			Generally worth doing. Will need to be sped up for very large machines...
			
			TODO: EXPERIMENTS!
	

	\chapter{Discussion}

\section{Suitability of the hexagonal torus topology}
	\subsection{Physical scalability}
	\subsection{Routability}
	\subsection{Placeability}

\section{Suitability of the SpiNNaker router}
	\subsection{Deadlock avoidance}
	\subsection{Routing table size}

\section{Suitability of circuit placers for application placement}
	\subsection{Quality}
	\subsection{Runtime}
	\subsection{Routing resources}
	\subsection{Flexibility}
	\subsection{Scalability}


	\chapter{Future research}
	
	In this thesis I have presented a number of new techniques which have made it
	possible to assemble and operate the SpiNNaker super computer. This work
	opens up a range of possibie lines of research to extend this work to future
	architectures and applications. In this chapter I focus on two anticipated
	challenges of future systems: growing scale and greater dynamicism in
	applications.
	
	\section{Scaling up}
		
		TODO: INTRO
		
		\subsection{Grid machine room layouts}
			
			In chapter XXX, I developed a machine room layout for hexagonal torus
			topologies which allowed machines occupying a row of standard
			machine-room cabinets to scale up without the need for long
			interconnecting cables. For larger installations, however, it will be
			necessary to employ multiple rows of cabinets in a 2D arrangement.
		
		\subsection{Routing congestion control}
		
		\subsection{Parallel place and route}
	
	\section{Structural plasticity and dynamic fault-tolerance}
		\subsection{Plasticity models}
		\subsection{Incremental placement}
		\subsection{Incremental routing}
		\subsection{Hot-spare routes}

	\chapter{Conclusions and future research}
	
	The SpiNNaker architecture was designed to tackle the challenges presented by
	the simulation of biologically realistic neural networks. One of its
	distinguishing features is its network architecture which employs both an
	unconventional network topology and multicast router architecture. The
	hexagonal torus topology used by SpiNNaker was chosen to enable greater
	performance while maintaining ease of construction and scalability compared
	with conventional network topologies. SpiNNaker's router design centres
	around packets mimicking the neural `spike' signals they are designed to
	convey by being small, multicast and not guaranteed to arrive at their
	destination.  This novel design, though largely complete before this work
	began, left a number of open problems which this thesis has attempted to
	address.
	
	In this concluding chapter I begin by summarising the answers to the research
	questions raised in chapter~\ref{sec:introduction}. This is followed by a
	discussion of new research topics which have been uncovered by this work.
	
	\section{Answers to research questions}
		
		Each of the three research questions are answered below.
		
		\subsubsection{1. Can the hexagonal torus topology be deployed and used in
		real, large-scale systems?}
		
		In chapter~\ref{sec:building}, I introduced a cabling scheme and assembly
		technique which has been used successfully to build a prototype SpiNNaker
		system with over half a million processor cores using the hexagonal torus
		topology. The techniques shown are expected to enable a final SpiNNaker
		machine of double this size to be built, filling a six metre long row of
		machine-room cabinets.
		
		Though SpiNNaker's processor-count places it amongst some of the world's
		largest supercomputers (see figure \ref{fig:top500-num-processors} on page
		\pageref{fig:top500-num-processors}), it is comparatively compact, filling
		one row of cabinets compared with the warehouse-scale installations found
		in commercial systems. In spite of this, the folding and interleaving
		techniques described allow hexagonal torus topologies to scale to
		arbitrarily large installations without cables which span the machine.
		
		Chapter~\ref{sec:shortestPaths} described an efficient and general
		technique for finding, and enumerating shortest path vectors in hexagonal
		torus topologies. These developments bring the hexagonal torus topology in
		line with other topologies by enabling routing algorithms to exploit all
		possible paths in a network. Further, chapter~\ref{sec:placement}
		demonstrated that placement algorithms are also adaptable to hexagonal
		torus topologies thanks to their similarity to 2D toruses.
		
		Though, as this thesis highlights, hexagonal toruses lack many of the
		intuitive properties enjoyed by other topologies, it is still possible to
		reason about them with only limited computational effort.  Now that the
		practicality and scalability of the topology has also been demonstrated in
		practice, it represents a credible option for future network architectures.
		
		\subsubsection{2. Does SpiNNaker's router architecture help, or hinder
		fault tolerance?}
		
		SpiNNaker's unconventional use of packet dropping to avoid deadlocks
		greatly simplifies the router architecture, part of the motivation for this
		design. In chapter~\ref{sec:routing} this feature is used to the advantage
		of PGS repair to add fault tolerance to existing routing algorithms.
		Compared with the often complex and wasteful methods used to tolerate
		faults in other networks, PGS repair incurs very little performance
		overhead in the presence of static faults.
		
		Routing table usage does increase in the presence of faults, however, which
		may be a concern for applications which already require many routing table
		entries. Routing table usage, as well as other overheads, were most
		significantly increased in the presence of contiguous groups of network
		faults. This is because the PGS repair algorithm produces routes which pass
		tightly around the corners of faults, resulting in concentrations of
		routing table entries in those areas.  Though the symptoms of this problem
		can be attributed to the design of SpiNNaker's multicast routing mechanism,
		the responsibility lies with the behaviour of the PGS repair algorithm.
		Potential improvements to the PGS repair algorithm are discussed later in
		\S\ref{sec:pgs-repair-improvements}.
		
		The overall answer to this research question, therefore, is that the
		flexibility provided to routing algorithms in SpiNNaker's architecture is
		of great benefit, enabling arbitrary fault patterns to be inexpensively
		avoided.
		
		\subsubsection{3. How can the parts of a neural simulation be placed onto a
		large hexagonal torus topology to reduce network load?}
		
		In chapter~\ref{sec:placement}, I explored a number of contemporary
		approaches to the problem of placing applications with irregular
		communication patterns into network topologies. I observed that researchers
		working on circuit placement for chips and FPGAs are tackling similar
		problems and working at scales as large, or larger than, those faced in
		application placement. Based on this I developed a
		simulated annealing based placement algorithm inspired by the techniques
		used in circuit placement, with specific adaptations for use in application
		placement and SpiNNaker's network topology.
		
		The simulated annealing based placement algorithm consistently outperforms
		pre-existing placement algorithms included in benchmarks in terms of
		placement quality.  In the case of one benchmark, simulated annealing based
		placement made it possible to run that neural simulation in real-time for
		the first time.  At larger scales, simulated annealing was also found to be
		able to produce good quality placements in benchmarks containing over one
		million processes -- the largest size supported by the SpiNNaker
		architecture.
		
		The major shortcoming of simulated annealing based placement is its
		execution speed. Though its execution time grows in proportion to the size
		of the problem, the implementation used took over 12~hours to place a
		synthetic problem for the largest planned SpiNNaker machine. Though
		tractable -- particularly given the relative output quality compared with
		the prior state-of-the-art -- the algorithm is unlikely to function
		comfortably as-is on larger problems.
		
		The conclusion to be drawn from this result, however, is not just that
		simulated annealing is a good solution for today's placement problems but
		that circuit placement techniques in general could be successfully adapted
		to fulfil this role. The placement problems faced by chip designers are
		growing at roughly the same exponential rate as the size of super computers
		but circuit designs hold the lead in terms of problem size. Consequently,
		as approaches are retired by chip placement researchers, they may find new
		life in the field of application placement.
		
	\section{Future research}
		
		Though the goals of this study have largely been met, there also remain
		some important limitations which future work may hope to address.
		Furthermore, this work has uncovered a number of new research areas
		warranting future enquiry. This section outlines a number of future lines
		of research.
		
		\subsection{Warehouse-scale cabling}
			
			In chapter~\ref{sec:building} I developed and implemented a number of
			cabling schemes for the SpiNNaker architecture spanning up to a six metre
			row of machine-room cabinets -- a relatively small installation by
			current standards. In SpiNNaker, the cabling exists in a 2D plane (i.e.
			across the faces of the cabinets) but as the system is scaled up, a
			single row of cabinets will tend towards a 1D line. Since embedding a 2D
			structure in a 1D space necessarily results in long connections, this
			cannot scale indefinitely.
			
			\begin{figure}
				\center
				\buildfig{figures/multi-row-cabling.tex}
				
				\caption{Multiple rows of interconnected cabinets.}
				\label{fig:multi-row-cabling}
			\end{figure}
			
			In conventional large-scale super computer installations, nodes are
			installed in rows of cabinets as illustrated in
			figure~\ref{fig:multi-row-cabling}.  From a `bird's-eye' view, the system
			approximates a 2D space, spread across the floor of a machine-room.
			Therefore, in principle, the folding and interleaving techniques
			described in chapter~\ref{sec:building} still apply. Unfortunately for
			SpiNNaker, cables connecting between rows of cabinets would be longer
			than the one metre limit imposed by its hardware because of the spacing
			between rows of cabinets.  Future SpiNNaker systems will need to consider
			alternative link technologies.  For example, a hybrid system could be
			used in which intra-cabinet connections continue to use the current HSS
			link technology while inter-cabinet links might use optical connections.
			This type of architecture could be supported by the use of pluggable
			`SFP+' transceiver modules~\cite{sff01}.
		
		\subsection{Cabling assistance for other architectures}
			
			A secondary result of the construction of prototype SpiNNaker systems in
			chapter~\ref{sec:building} was the use of real-time guidance and feedback
			to assist cable installation. I am not aware of this technique's use by
			existing architectures and, following the success experienced in this
			project, it is possible that the technique may also be useful in
			conventional systems.
			
			During the construction of prototype SpiNNaker machines, each cable took
			seconds to install compared with the minutes reported for existing
			systems~\cite{mudigonda11}. Part of this increase in efficiency appears
			to result from the immediate identification of mistakes made during
			cabling, saving time-consuming backtracking later on.
			
			In many real-world network installations, units are less densely packed
			than in SpiNNaker and so longer cables are often required. As a
			consequence, cabling errors may become more likely as cabling patterns
			are spread over a larger area making them more difficult to visually
			verify. Like SpiNNaker, conventional networking hardware is often
			equipped with a generous range of indicator LEDs and diagnostic
			facilities which might be used to implement real-time installation
			guidance. Future work could explore the use of this technique in the
			construction of other large-scale networks, such as data centres.
		
		\subsection{Congestion mitigation}
			
			\label{sec:wiggly-board-allocations}
			
			In chapter~\ref{sec:routing} I found that contiguous network faults cause
			hot-spots of congestion and routing table depletion where the PGS repair
			algorithm routed many paths around the edges of faults.  However, it is
			not just faults which can cause contiguous blockages in the network
			topology. In reality, researchers do not always require a full-sized
			SpiNNaker system to perform their experiments so large SpiNNaker systems
			are soft-partitioned on demand into many smaller
			machines~\cite{spalloc16}. To ensure isolation between partitioned
			sub-machines, HSS links between boards in different partitions are
			disabled. Because of SpiNNaker's `wrapped triple' partitioning scheme,
			the resulting sub-machines have hexagonal \emph{mesh} topologies (i.e.
			without wrap-around links) with irregular boundaries as in
			figure~\ref{fig:spalloc-mesh}.
			
			\begin{figure}
				\center
				\buildfig{figures/spalloc-mesh.tex}
				
				\caption[Irregular edges of a partitioned SpiNNaker system.]%
				{Irregular edges in a SpiNNaker system comprised of 24~boards
				partitioned from a larger machine.  Each hexagon represents a SpiNNaker
				chip. No wrap-around connections are present.}
				\label{fig:spalloc-mesh}
			\end{figure}
			
			In partitioned systems, the `tooth'-like gaps on the periphery of the
			network result in similar congestion to the HSS link failures considered
			in chapter~\ref{sec:routing}. When a route is generated between nodes on
			opposite sides of a gap, the PGS repair process will produce a
			shortest-path route around it. Since many routes may be blocked by a
			single gap, a hot-spot may develop around the corners of the gap.
			
			In chapter~\ref{sec:placement}, the `CConv' benchmark application was
			found to run correctly the majority of the time when placed by the
			simulated annealing algorithm but would occasionally fail by a
			significant margin. Preliminary experiments suggest these occasional
			failures are caused by placement solutions which place heavily
			communicating parts of the application on opposite sides of gaps along
			the perimeter of the network. Two possible approaches which future work
			may consider are presented below.
			
			\subsubsection{Avoiding hotspots with PGS repair}
				
				\label{sec:pgs-repair-improvements}	
				
				Network congestion around faults and network irregularities could be
				reduced by forcing the PGS repair process to take more varied routes
				around faults. For example, in circuit routing algorithms, routers
				avoid congestion by increasing the cost of routes which pass through
				congested areas~\cite{kahng11}. A similar technique could be used in
				PGS repair to spread the routes it produces.
				
				An alternative approach would be to adapt the base routing algorithms
				used prior to PGS repair to, for example, attempt alternative dimension
				order routes which may avoid blockages due to faulty links.
			
			\subsubsection{Fault and irregularity aware placement}
				
				One of the shortcomings of the simulated annealing based placer
				developed in chapter~\ref{sec:placement} is that it does not account
				for network faults, or irregularities, when estimating the cost of
				placement solutions.  Future work may exploit techniques used in
				congestion-aware circuit placement which could be adapted for
				application placement~\cite{viswanathan07}.
		
		\subsection{Reducing placement execution time}
			
			The simulated annealing based placer presented in
			chapter~\ref{sec:placement} produced good quality placements but its
			execution time limits its use beyond one million vertex placement
			problems. Future work should explore possibilities for improving the
			performance and scalability of this technique.
			
			In addition to considering alternative placement algorithms based on
			other methods, one possible approach is to attempt to reduce the execution
			time of simulated annealing based placement by shrinking the application
			graph being placed.
			
			For example, graph clustering~\cite{schaeffer07} may be used to group
			together strongly connected vertices which would then be placed as a
			single unit.  Unfortunately, clustering can suffer from the same problems
			as graph-partitioning-based placement: vertices may be grouped together
			in ways which, in practice, cannot be packed together into a given portion
			of a machine.  A possible solution to this problem is to use a two-phase
			placement approach~\cite{kahng11}. In a `global' placement phase,
			solutions are permitted which can slightly over-allocate resources but
			overall achieve good placement quality. In the `detailed' placement phase
			which follows, the solution is `legalised' by making small changes to the
			global placement to eliminate over allocation.
			
			An alternative approach suited to SpiNNaker could be to limit the
			clustering process to clusters which fit on a single SpiNNaker chip. In
			typical SpiNNaker application graphs, clustering to this level may reduce
			placement problem sizes by an order of magnitude and, consequently,
			reduce execution times by the same ratio. Preliminary experiments suggest
			that this approach might result in little placement quality loss for
			large placement problems whilst substantially reducing overall execution
			time.
		
		\subsection{Benchmarking}
			
			One of the most significant limitations of this study has been the
			unavailability of large-scale SpiNNaker applications for use as
			benchmarks. As a consequence, much of the scalability experimentation
			performed has relied on simple synthetic benchmarks based on projections
			of future application behaviour.
			
			In the short term, more sophisticated synthetic benchmark generation
			techniques used by the circuit placement community~\cite{nam07} may offer
			alternative benchmarks for future work. In the longer term, however, it
			is hoped that the availability of large SpiNNaker systems -- and
			placement and routing algorithms better suited to exploit them -- will
			lead to larger scale applications being developed. Hopefully these
			applications will lead to more interesting and representative benchmarks
			for use in future work.
	
	\section{Closing remarks}
		
		One of the primary outcomes of this work is that a number of the practical
		challenges faced in scaling up the SpiNNaker architecture have been
		addressed leading to the construction of large-scale SpiNNaker machines.
		The development of an effective placement algorithm for SpiNNaker
		applications has been shown to enable some neural simulations to exploit
		SpiNNaker's architecture for the first time. The availability of larger
		SpiNNaker machines paves the way for future large-scale neural modelling
		work built on much larger models such as Spaun, `the world's largest
		functional brain model'~\cite{eliasmith12}.
		
		Beyond the SpiNNaker project, the hexagonal torus topology has also been
		validated as a scalable and practical candidate for future network
		architectures. As super computers become ever larger, the physical
		scalability afforded by the 2D nature of the hexagonal torus topology may
		make it a compelling option. In addition, the finding that circuit
		placement techniques can be adapted to support placement of SpiNNaker
		software indicates that these algorithms may also be applicable to other
		applications. Indeed, if this is the case, circuit placement may offer a
		long-term source of placement algorithms able to handle the demands of
		future applications.
		
		% This thesis has explored and tackled a number of the challenges posed in
		% scaling up the unconventional SpiNNaker architecture. Along the way I have
		% demonstrated that the hexagonal torus topology may be a practical choice in
		% future applications which can scale up to the physical dimensions expected
		% of future super computers. I have also developed new efficient and
		% effective methods of placing and routing neural simulation software on
		% SpiNNaker which -- it is hoped -- will enable a new generation of large
		% scale neural simulations on spinnaker.
		
		Although this work stops short of demonstrating truly large-scale
		neuroscientific simulations running at the scale of newly completed
		SpiNNaker machines (largely because such simulations do not yet exist) a
		number of smaller-scale neural simulations have been made possible for the
		first time. The algorithms and techniques devised in this work have
		subsequently been incorporated into various software libraries and tools
		now being used by researchers building SpiNNaker applications, vindicating
		the efforts of this thesis (see appendix~\ref{sec:software}). A successor
		to the SpiNNaker architecture is also in the early stages of design and is
		building on experience of the existing architecture. The current intention
		is to retain the hexagonal torus topology used by SpiNNaker, a decision
		supported by the findings of this thesis.
		
		With SpiNNaker's hardware architecture now operating at scales close to its
		architectural limits, it is hoped that the contributions of this work will
		enable researchers to develop larger and more detailed neural models for
		this unique architecture.

	
	% Bibliography
	\bibliography{references}
	\bibliographystyle{alpha}
	
\end{document}
\unskip.};
		% Box around commit message
		\draw [rounded corners, white!90!black]
		      (commit notice.south west)
		      rectangle
		      (commit notice.north east);
		% List changed files (if any)
		\node [ above=10pt of commit notice
		      , text width=\linewidth
		      , font=\tiny
		      , align=center
		      ]
		      {\documentclass[12pt,twoside]{report}

\newcommand{\thesistitle}{Building and operating large-scale SpiNNaker machines}
\newcommand{\thesisauthor}{Jonathan Heathcote}
\newcommand{\thesisyear}{2016}

%%%%%%%%%%%%%%%%%%%%%%%%%%%%%%%%%%%%%%%%%%%%%%%%%%%%%%%%%%%%%%%%%%%%%%%%%%%%%%%%
% Used packages
%%%%%%%%%%%%%%%%%%%%%%%%%%%%%%%%%%%%%%%%%%%%%%%%%%%%%%%%%%%%%%%%%%%%%%%%%%%%%%%%

% Nice printing of URLs
\usepackage{url}

% Actually not tear-your-eyes-out-ugly tables
\usepackage{booktabs}

% Adjust linespacing for localised parts of the paper (e.g. abstract)
\usepackage{setspace}

% For the \ifthenelse macro
\usepackage{ifthen}

% For the \degree macro
\usepackage{gensymb}

% For subfigure support
\usepackage{caption}
\usepackage{subcaption}

% SI unit and number formatting
\usepackage{siunitx}

% Used to draw labels with a white outline to make them stand-out in diagrams
\usepackage[outline]{contour}

% TikZ + PGF Plots for diagram/plot drawing
\usepackage{tikz}
\usepackage{tikz3d}
\usepackage{pgfplots}
\usetikzlibrary{ hexagon
               , calc
               , backgrounds
               , positioning
               , decorations.pathreplacing
               , decorations.markings
               , arrows
               , positioning
               , automata
               , shadows
               , fit
               , shapes
               , arrows
               , patterns
               , spy
               }
\usepgfplotslibrary{statistics}

%%%%%%%%%%%%%%%%%%%%%%%%%%%%%%%%%%%%%%%%%%%%%%%%%%%%%%%%%%%%%%%%%%%%%%%%%%%%%%%%
% Environment settings
%%%%%%%%%%%%%%%%%%%%%%%%%%%%%%%%%%%%%%%%%%%%%%%%%%%%%%%%%%%%%%%%%%%%%%%%%%%%%%%%

% 1.5 linespacing (as required by university)
\renewcommand{\baselinestretch}{1.5}


% Specifies the thickness of the contour added by the \contour macro.
\contourlength{1.5pt}

% Define a few layers for TikZ to allow easier layering
\pgfdeclarelayer{bg}
\pgfdeclarelayer{fg}
\pgfsetlayers{bg,main,fg}

%%%%%%%%%%%%%%%%%%%%%%%%%%%%%%%%%%%%%%%%%%%%%%%%%%%%%%%%%%%%%%%%%%%%%%%%%%%%%%%%
% Definitions
%%%%%%%%%%%%%%%%%%%%%%%%%%%%%%%%%%%%%%%%%%%%%%%%%%%%%%%%%%%%%%%%%%%%%%%%%%%%%%%%

% Used in place of \chapter for preface sections. Prevents numbering but
% includes the chapter in the ToC
\newcommand{\prefacesection}[1]{
	\chapter*{#1}
	\addcontentsline{toc}{chapter}{#1}
}

% Adds 'discard if' and  'discard if not' options for \addplot to enable
% filtering of data. Taken from
% http://tex.stackexchange.com/questions/58548/is-it-possible-to-change-the-color-of-a-single-bar-when-the-bar-plot-is-based-on
\pgfplotsset{
    discard if/.style 2 args={
        x filter/.code={
            \edef\tempa{\thisrow{#1}}
            \edef\tempb{#2}
            \ifx\tempa\tempb
                \def\pgfmathresult{inf}
            \fi
        }
    },
    discard if not/.style 2 args={
        x filter/.code={
            \edef\tempa{\thisrow{#1}}
            \edef\tempb{#2}
            \ifx\tempa\tempb
            \else
                \def\pgfmathresult{inf}
            \fi
        }
    }
}

% Make PGFplots Treat "NA" (regardless of letter case) as "nan". From:
% http://tex.stackexchange.com/questions/110441/skip-specific-string-in-a-numeric-column-while-using-pgfplots
\makeatletter
\expandafter\def\csname pgffltA@N\endcsname{\pgfflt@readundef}
\expandafter\def\csname pgffltA@n\endcsname{\pgfflt@readundef}
\def\pgfflt@readundef #1{%
    \def\pgfflt@readnan@ok{1}%
    \if#1a\else\if#1A\else\def\pgfflt@readnan@ok{0}\fi\fi
    \if\pgfflt@readnan@ok1%
        \pgfmathfloat@a@S=3\relax%
        \pgfmathfloat@a@Mtok={0.0}%
        \pgfmathfloat@a@E=0%
        \expandafter\pgfflt@finish
    \else
        \def\pgfflt@readnan@{\pgfflt@error #1}%
        \expandafter\pgfflt@readnan@
    \fi
}
\makeatother

%%%%%%%%%%%%%%%%%%%%%%%%%%%%%%%%%%%%%%%%%%%%%%%%%%%%%%%%%%%%%%%%%%%%%%%%%%%%%%%%
% Document body
%%%%%%%%%%%%%%%%%%%%%%%%%%%%%%%%%%%%%%%%%%%%%%%%%%%%%%%%%%%%%%%%%%%%%%%%%%%%%%%%
\begin{document}
	
	% The title page
	\begin{titlepage}
	
	
	\begin{center}
		
		\vspace*{1.0in}
		
		{\LARGE\textbf{\thesistitle}}
		
		\vfill
		
		\textsc{A thesis submitted to the University of Manchester\\for the degree of Doctor
		of Philosophy\\in the Faculty of Science and Engineering.}
		
		\vfill
		
		\thesisyear
		
		\vfill
		
		\thesisauthor
		
		\vfill
		
		School of Computer Science
		
		\vfill
		
		\color{gray}{
			{\tiny{}Revision \texttt{\documentclass[12pt,twoside]{report}

\newcommand{\thesistitle}{Building and operating large-scale SpiNNaker machines}
\newcommand{\thesisauthor}{Jonathan Heathcote}
\newcommand{\thesisyear}{2016}

%%%%%%%%%%%%%%%%%%%%%%%%%%%%%%%%%%%%%%%%%%%%%%%%%%%%%%%%%%%%%%%%%%%%%%%%%%%%%%%%
% Used packages
%%%%%%%%%%%%%%%%%%%%%%%%%%%%%%%%%%%%%%%%%%%%%%%%%%%%%%%%%%%%%%%%%%%%%%%%%%%%%%%%

% Nice printing of URLs
\usepackage{url}

% Actually not tear-your-eyes-out-ugly tables
\usepackage{booktabs}

% Adjust linespacing for localised parts of the paper (e.g. abstract)
\usepackage{setspace}

% For the \ifthenelse macro
\usepackage{ifthen}

% For the \degree macro
\usepackage{gensymb}

% For subfigure support
\usepackage{caption}
\usepackage{subcaption}

% SI unit and number formatting
\usepackage{siunitx}

% Used to draw labels with a white outline to make them stand-out in diagrams
\usepackage[outline]{contour}

% TikZ + PGF Plots for diagram/plot drawing
\usepackage{tikz}
\usepackage{tikz3d}
\usepackage{pgfplots}
\usetikzlibrary{ hexagon
               , calc
               , backgrounds
               , positioning
               , decorations.pathreplacing
               , decorations.markings
               , arrows
               , positioning
               , automata
               , shadows
               , fit
               , shapes
               , arrows
               , patterns
               , spy
               }
\usepgfplotslibrary{statistics}

%%%%%%%%%%%%%%%%%%%%%%%%%%%%%%%%%%%%%%%%%%%%%%%%%%%%%%%%%%%%%%%%%%%%%%%%%%%%%%%%
% Environment settings
%%%%%%%%%%%%%%%%%%%%%%%%%%%%%%%%%%%%%%%%%%%%%%%%%%%%%%%%%%%%%%%%%%%%%%%%%%%%%%%%

% 1.5 linespacing (as required by university)
\renewcommand{\baselinestretch}{1.5}


% Specifies the thickness of the contour added by the \contour macro.
\contourlength{1.5pt}

% Define a few layers for TikZ to allow easier layering
\pgfdeclarelayer{bg}
\pgfdeclarelayer{fg}
\pgfsetlayers{bg,main,fg}

%%%%%%%%%%%%%%%%%%%%%%%%%%%%%%%%%%%%%%%%%%%%%%%%%%%%%%%%%%%%%%%%%%%%%%%%%%%%%%%%
% Definitions
%%%%%%%%%%%%%%%%%%%%%%%%%%%%%%%%%%%%%%%%%%%%%%%%%%%%%%%%%%%%%%%%%%%%%%%%%%%%%%%%

% Used in place of \chapter for preface sections. Prevents numbering but
% includes the chapter in the ToC
\newcommand{\prefacesection}[1]{
	\chapter*{#1}
	\addcontentsline{toc}{chapter}{#1}
}

% Adds 'discard if' and  'discard if not' options for \addplot to enable
% filtering of data. Taken from
% http://tex.stackexchange.com/questions/58548/is-it-possible-to-change-the-color-of-a-single-bar-when-the-bar-plot-is-based-on
\pgfplotsset{
    discard if/.style 2 args={
        x filter/.code={
            \edef\tempa{\thisrow{#1}}
            \edef\tempb{#2}
            \ifx\tempa\tempb
                \def\pgfmathresult{inf}
            \fi
        }
    },
    discard if not/.style 2 args={
        x filter/.code={
            \edef\tempa{\thisrow{#1}}
            \edef\tempb{#2}
            \ifx\tempa\tempb
            \else
                \def\pgfmathresult{inf}
            \fi
        }
    }
}

% Make PGFplots Treat "NA" (regardless of letter case) as "nan". From:
% http://tex.stackexchange.com/questions/110441/skip-specific-string-in-a-numeric-column-while-using-pgfplots
\makeatletter
\expandafter\def\csname pgffltA@N\endcsname{\pgfflt@readundef}
\expandafter\def\csname pgffltA@n\endcsname{\pgfflt@readundef}
\def\pgfflt@readundef #1{%
    \def\pgfflt@readnan@ok{1}%
    \if#1a\else\if#1A\else\def\pgfflt@readnan@ok{0}\fi\fi
    \if\pgfflt@readnan@ok1%
        \pgfmathfloat@a@S=3\relax%
        \pgfmathfloat@a@Mtok={0.0}%
        \pgfmathfloat@a@E=0%
        \expandafter\pgfflt@finish
    \else
        \def\pgfflt@readnan@{\pgfflt@error #1}%
        \expandafter\pgfflt@readnan@
    \fi
}
\makeatother

%%%%%%%%%%%%%%%%%%%%%%%%%%%%%%%%%%%%%%%%%%%%%%%%%%%%%%%%%%%%%%%%%%%%%%%%%%%%%%%%
% Document body
%%%%%%%%%%%%%%%%%%%%%%%%%%%%%%%%%%%%%%%%%%%%%%%%%%%%%%%%%%%%%%%%%%%%%%%%%%%%%%%%
\begin{document}
	
	% The title page
	\input{titlepage}
	
	% The table of contents which, per university regulations, is followed by a
	% total wordcount.
	\tableofcontents
	\vfill
	\noindent This thesis contains
		\immediate\write18{texcount -1 -sum -inc thesis.tex > thesis.wordcount}%
		\input{thesis.wordcount}words.
	
	\clearpage
	\listoffigures
	
	\clearpage
	\listoftables
	
	% Abstract
	\input{abstract}
	
	% Declaration of non-submission elsewhere
	\input{declaration}
	
	% University-prescribed copyright statement...
	\input{copyright}
	
	% Acknowledgements
	\input{acknowledgements}
	
	% Main body
	\input{introduction.tex}
	\input{background.tex}
	\input{building.tex}
	\input{shortestPaths.tex}
	\input{routing.tex}
	\input{placement.tex}
	\input{discussion.tex}
	\input{future.tex}
	\input{conclusions.tex}
	
	% Bibliography
	\bibliography{references}
	\bibliographystyle{alpha}
	
\end{document}
}\documentclass[12pt,twoside]{report}

\newcommand{\thesistitle}{Building and operating large-scale SpiNNaker machines}
\newcommand{\thesisauthor}{Jonathan Heathcote}
\newcommand{\thesisyear}{2016}

%%%%%%%%%%%%%%%%%%%%%%%%%%%%%%%%%%%%%%%%%%%%%%%%%%%%%%%%%%%%%%%%%%%%%%%%%%%%%%%%
% Used packages
%%%%%%%%%%%%%%%%%%%%%%%%%%%%%%%%%%%%%%%%%%%%%%%%%%%%%%%%%%%%%%%%%%%%%%%%%%%%%%%%

% Nice printing of URLs
\usepackage{url}

% Actually not tear-your-eyes-out-ugly tables
\usepackage{booktabs}

% Adjust linespacing for localised parts of the paper (e.g. abstract)
\usepackage{setspace}

% For the \ifthenelse macro
\usepackage{ifthen}

% For the \degree macro
\usepackage{gensymb}

% For subfigure support
\usepackage{caption}
\usepackage{subcaption}

% SI unit and number formatting
\usepackage{siunitx}

% Used to draw labels with a white outline to make them stand-out in diagrams
\usepackage[outline]{contour}

% TikZ + PGF Plots for diagram/plot drawing
\usepackage{tikz}
\usepackage{tikz3d}
\usepackage{pgfplots}
\usetikzlibrary{ hexagon
               , calc
               , backgrounds
               , positioning
               , decorations.pathreplacing
               , decorations.markings
               , arrows
               , positioning
               , automata
               , shadows
               , fit
               , shapes
               , arrows
               , patterns
               , spy
               }
\usepgfplotslibrary{statistics}

%%%%%%%%%%%%%%%%%%%%%%%%%%%%%%%%%%%%%%%%%%%%%%%%%%%%%%%%%%%%%%%%%%%%%%%%%%%%%%%%
% Environment settings
%%%%%%%%%%%%%%%%%%%%%%%%%%%%%%%%%%%%%%%%%%%%%%%%%%%%%%%%%%%%%%%%%%%%%%%%%%%%%%%%

% 1.5 linespacing (as required by university)
\renewcommand{\baselinestretch}{1.5}


% Specifies the thickness of the contour added by the \contour macro.
\contourlength{1.5pt}

% Define a few layers for TikZ to allow easier layering
\pgfdeclarelayer{bg}
\pgfdeclarelayer{fg}
\pgfsetlayers{bg,main,fg}

%%%%%%%%%%%%%%%%%%%%%%%%%%%%%%%%%%%%%%%%%%%%%%%%%%%%%%%%%%%%%%%%%%%%%%%%%%%%%%%%
% Definitions
%%%%%%%%%%%%%%%%%%%%%%%%%%%%%%%%%%%%%%%%%%%%%%%%%%%%%%%%%%%%%%%%%%%%%%%%%%%%%%%%

% Used in place of \chapter for preface sections. Prevents numbering but
% includes the chapter in the ToC
\newcommand{\prefacesection}[1]{
	\chapter*{#1}
	\addcontentsline{toc}{chapter}{#1}
}

% Adds 'discard if' and  'discard if not' options for \addplot to enable
% filtering of data. Taken from
% http://tex.stackexchange.com/questions/58548/is-it-possible-to-change-the-color-of-a-single-bar-when-the-bar-plot-is-based-on
\pgfplotsset{
    discard if/.style 2 args={
        x filter/.code={
            \edef\tempa{\thisrow{#1}}
            \edef\tempb{#2}
            \ifx\tempa\tempb
                \def\pgfmathresult{inf}
            \fi
        }
    },
    discard if not/.style 2 args={
        x filter/.code={
            \edef\tempa{\thisrow{#1}}
            \edef\tempb{#2}
            \ifx\tempa\tempb
            \else
                \def\pgfmathresult{inf}
            \fi
        }
    }
}

% Make PGFplots Treat "NA" (regardless of letter case) as "nan". From:
% http://tex.stackexchange.com/questions/110441/skip-specific-string-in-a-numeric-column-while-using-pgfplots
\makeatletter
\expandafter\def\csname pgffltA@N\endcsname{\pgfflt@readundef}
\expandafter\def\csname pgffltA@n\endcsname{\pgfflt@readundef}
\def\pgfflt@readundef #1{%
    \def\pgfflt@readnan@ok{1}%
    \if#1a\else\if#1A\else\def\pgfflt@readnan@ok{0}\fi\fi
    \if\pgfflt@readnan@ok1%
        \pgfmathfloat@a@S=3\relax%
        \pgfmathfloat@a@Mtok={0.0}%
        \pgfmathfloat@a@E=0%
        \expandafter\pgfflt@finish
    \else
        \def\pgfflt@readnan@{\pgfflt@error #1}%
        \expandafter\pgfflt@readnan@
    \fi
}
\makeatother

%%%%%%%%%%%%%%%%%%%%%%%%%%%%%%%%%%%%%%%%%%%%%%%%%%%%%%%%%%%%%%%%%%%%%%%%%%%%%%%%
% Document body
%%%%%%%%%%%%%%%%%%%%%%%%%%%%%%%%%%%%%%%%%%%%%%%%%%%%%%%%%%%%%%%%%%%%%%%%%%%%%%%%
\begin{document}
	
	% The title page
	\input{titlepage}
	
	% The table of contents which, per university regulations, is followed by a
	% total wordcount.
	\tableofcontents
	\vfill
	\noindent This thesis contains
		\immediate\write18{texcount -1 -sum -inc thesis.tex > thesis.wordcount}%
		\input{thesis.wordcount}words.
	
	\clearpage
	\listoffigures
	
	\clearpage
	\listoftables
	
	% Abstract
	\input{abstract}
	
	% Declaration of non-submission elsewhere
	\input{declaration}
	
	% University-prescribed copyright statement...
	\input{copyright}
	
	% Acknowledgements
	\input{acknowledgements}
	
	% Main body
	\input{introduction.tex}
	\input{background.tex}
	\input{building.tex}
	\input{shortestPaths.tex}
	\input{routing.tex}
	\input{placement.tex}
	\input{discussion.tex}
	\input{future.tex}
	\input{conclusions.tex}
	
	% Bibliography
	\bibliography{references}
	\bibliographystyle{alpha}
	
\end{document}
}
		}
		
	\end{center}
	
\end{titlepage}

	
	% The table of contents which, per university regulations, is followed by a
	% total wordcount.
	\tableofcontents
	\vfill
	\noindent This thesis contains
		\immediate\write18{texcount -1 -sum -inc thesis.tex > thesis.wordcount}%
		\documentclass[12pt,twoside]{report}

\newcommand{\thesistitle}{Building and operating large-scale SpiNNaker machines}
\newcommand{\thesisauthor}{Jonathan Heathcote}
\newcommand{\thesisyear}{2016}

%%%%%%%%%%%%%%%%%%%%%%%%%%%%%%%%%%%%%%%%%%%%%%%%%%%%%%%%%%%%%%%%%%%%%%%%%%%%%%%%
% Used packages
%%%%%%%%%%%%%%%%%%%%%%%%%%%%%%%%%%%%%%%%%%%%%%%%%%%%%%%%%%%%%%%%%%%%%%%%%%%%%%%%

% Nice printing of URLs
\usepackage{url}

% Actually not tear-your-eyes-out-ugly tables
\usepackage{booktabs}

% Adjust linespacing for localised parts of the paper (e.g. abstract)
\usepackage{setspace}

% For the \ifthenelse macro
\usepackage{ifthen}

% For the \degree macro
\usepackage{gensymb}

% For subfigure support
\usepackage{caption}
\usepackage{subcaption}

% SI unit and number formatting
\usepackage{siunitx}

% Used to draw labels with a white outline to make them stand-out in diagrams
\usepackage[outline]{contour}

% TikZ + PGF Plots for diagram/plot drawing
\usepackage{tikz}
\usepackage{tikz3d}
\usepackage{pgfplots}
\usetikzlibrary{ hexagon
               , calc
               , backgrounds
               , positioning
               , decorations.pathreplacing
               , decorations.markings
               , arrows
               , positioning
               , automata
               , shadows
               , fit
               , shapes
               , arrows
               , patterns
               , spy
               }
\usepgfplotslibrary{statistics}

%%%%%%%%%%%%%%%%%%%%%%%%%%%%%%%%%%%%%%%%%%%%%%%%%%%%%%%%%%%%%%%%%%%%%%%%%%%%%%%%
% Environment settings
%%%%%%%%%%%%%%%%%%%%%%%%%%%%%%%%%%%%%%%%%%%%%%%%%%%%%%%%%%%%%%%%%%%%%%%%%%%%%%%%

% 1.5 linespacing (as required by university)
\renewcommand{\baselinestretch}{1.5}


% Specifies the thickness of the contour added by the \contour macro.
\contourlength{1.5pt}

% Define a few layers for TikZ to allow easier layering
\pgfdeclarelayer{bg}
\pgfdeclarelayer{fg}
\pgfsetlayers{bg,main,fg}

%%%%%%%%%%%%%%%%%%%%%%%%%%%%%%%%%%%%%%%%%%%%%%%%%%%%%%%%%%%%%%%%%%%%%%%%%%%%%%%%
% Definitions
%%%%%%%%%%%%%%%%%%%%%%%%%%%%%%%%%%%%%%%%%%%%%%%%%%%%%%%%%%%%%%%%%%%%%%%%%%%%%%%%

% Used in place of \chapter for preface sections. Prevents numbering but
% includes the chapter in the ToC
\newcommand{\prefacesection}[1]{
	\chapter*{#1}
	\addcontentsline{toc}{chapter}{#1}
}

% Adds 'discard if' and  'discard if not' options for \addplot to enable
% filtering of data. Taken from
% http://tex.stackexchange.com/questions/58548/is-it-possible-to-change-the-color-of-a-single-bar-when-the-bar-plot-is-based-on
\pgfplotsset{
    discard if/.style 2 args={
        x filter/.code={
            \edef\tempa{\thisrow{#1}}
            \edef\tempb{#2}
            \ifx\tempa\tempb
                \def\pgfmathresult{inf}
            \fi
        }
    },
    discard if not/.style 2 args={
        x filter/.code={
            \edef\tempa{\thisrow{#1}}
            \edef\tempb{#2}
            \ifx\tempa\tempb
            \else
                \def\pgfmathresult{inf}
            \fi
        }
    }
}

% Make PGFplots Treat "NA" (regardless of letter case) as "nan". From:
% http://tex.stackexchange.com/questions/110441/skip-specific-string-in-a-numeric-column-while-using-pgfplots
\makeatletter
\expandafter\def\csname pgffltA@N\endcsname{\pgfflt@readundef}
\expandafter\def\csname pgffltA@n\endcsname{\pgfflt@readundef}
\def\pgfflt@readundef #1{%
    \def\pgfflt@readnan@ok{1}%
    \if#1a\else\if#1A\else\def\pgfflt@readnan@ok{0}\fi\fi
    \if\pgfflt@readnan@ok1%
        \pgfmathfloat@a@S=3\relax%
        \pgfmathfloat@a@Mtok={0.0}%
        \pgfmathfloat@a@E=0%
        \expandafter\pgfflt@finish
    \else
        \def\pgfflt@readnan@{\pgfflt@error #1}%
        \expandafter\pgfflt@readnan@
    \fi
}
\makeatother

%%%%%%%%%%%%%%%%%%%%%%%%%%%%%%%%%%%%%%%%%%%%%%%%%%%%%%%%%%%%%%%%%%%%%%%%%%%%%%%%
% Document body
%%%%%%%%%%%%%%%%%%%%%%%%%%%%%%%%%%%%%%%%%%%%%%%%%%%%%%%%%%%%%%%%%%%%%%%%%%%%%%%%
\begin{document}
	
	% The title page
	\begin{titlepage}
	
	
	\begin{center}
		
		\vspace*{1.0in}
		
		{\LARGE\textbf{\thesistitle}}
		
		\vfill
		
		\textsc{A thesis submitted to the University of Manchester\\for the degree of Doctor
		of Philosophy\\in the Faculty of Science and Engineering.}
		
		\vfill
		
		\thesisyear
		
		\vfill
		
		\thesisauthor
		
		\vfill
		
		School of Computer Science
		
		\vfill
		
		\color{gray}{
			{\tiny{}Revision \texttt{\input{thesis.fullhash}}\input{thesis.date}}
		}
		
	\end{center}
	
\end{titlepage}

	
	% The table of contents which, per university regulations, is followed by a
	% total wordcount.
	\tableofcontents
	\vfill
	\noindent This thesis contains
		\immediate\write18{texcount -1 -sum -inc thesis.tex > thesis.wordcount}%
		\documentclass[12pt,twoside]{report}

\newcommand{\thesistitle}{Building and operating large-scale SpiNNaker machines}
\newcommand{\thesisauthor}{Jonathan Heathcote}
\newcommand{\thesisyear}{2016}

%%%%%%%%%%%%%%%%%%%%%%%%%%%%%%%%%%%%%%%%%%%%%%%%%%%%%%%%%%%%%%%%%%%%%%%%%%%%%%%%
% Used packages
%%%%%%%%%%%%%%%%%%%%%%%%%%%%%%%%%%%%%%%%%%%%%%%%%%%%%%%%%%%%%%%%%%%%%%%%%%%%%%%%

% Nice printing of URLs
\usepackage{url}

% Actually not tear-your-eyes-out-ugly tables
\usepackage{booktabs}

% Adjust linespacing for localised parts of the paper (e.g. abstract)
\usepackage{setspace}

% For the \ifthenelse macro
\usepackage{ifthen}

% For the \degree macro
\usepackage{gensymb}

% For subfigure support
\usepackage{caption}
\usepackage{subcaption}

% SI unit and number formatting
\usepackage{siunitx}

% Used to draw labels with a white outline to make them stand-out in diagrams
\usepackage[outline]{contour}

% TikZ + PGF Plots for diagram/plot drawing
\usepackage{tikz}
\usepackage{tikz3d}
\usepackage{pgfplots}
\usetikzlibrary{ hexagon
               , calc
               , backgrounds
               , positioning
               , decorations.pathreplacing
               , decorations.markings
               , arrows
               , positioning
               , automata
               , shadows
               , fit
               , shapes
               , arrows
               , patterns
               , spy
               }
\usepgfplotslibrary{statistics}

%%%%%%%%%%%%%%%%%%%%%%%%%%%%%%%%%%%%%%%%%%%%%%%%%%%%%%%%%%%%%%%%%%%%%%%%%%%%%%%%
% Environment settings
%%%%%%%%%%%%%%%%%%%%%%%%%%%%%%%%%%%%%%%%%%%%%%%%%%%%%%%%%%%%%%%%%%%%%%%%%%%%%%%%

% 1.5 linespacing (as required by university)
\renewcommand{\baselinestretch}{1.5}


% Specifies the thickness of the contour added by the \contour macro.
\contourlength{1.5pt}

% Define a few layers for TikZ to allow easier layering
\pgfdeclarelayer{bg}
\pgfdeclarelayer{fg}
\pgfsetlayers{bg,main,fg}

%%%%%%%%%%%%%%%%%%%%%%%%%%%%%%%%%%%%%%%%%%%%%%%%%%%%%%%%%%%%%%%%%%%%%%%%%%%%%%%%
% Definitions
%%%%%%%%%%%%%%%%%%%%%%%%%%%%%%%%%%%%%%%%%%%%%%%%%%%%%%%%%%%%%%%%%%%%%%%%%%%%%%%%

% Used in place of \chapter for preface sections. Prevents numbering but
% includes the chapter in the ToC
\newcommand{\prefacesection}[1]{
	\chapter*{#1}
	\addcontentsline{toc}{chapter}{#1}
}

% Adds 'discard if' and  'discard if not' options for \addplot to enable
% filtering of data. Taken from
% http://tex.stackexchange.com/questions/58548/is-it-possible-to-change-the-color-of-a-single-bar-when-the-bar-plot-is-based-on
\pgfplotsset{
    discard if/.style 2 args={
        x filter/.code={
            \edef\tempa{\thisrow{#1}}
            \edef\tempb{#2}
            \ifx\tempa\tempb
                \def\pgfmathresult{inf}
            \fi
        }
    },
    discard if not/.style 2 args={
        x filter/.code={
            \edef\tempa{\thisrow{#1}}
            \edef\tempb{#2}
            \ifx\tempa\tempb
            \else
                \def\pgfmathresult{inf}
            \fi
        }
    }
}

% Make PGFplots Treat "NA" (regardless of letter case) as "nan". From:
% http://tex.stackexchange.com/questions/110441/skip-specific-string-in-a-numeric-column-while-using-pgfplots
\makeatletter
\expandafter\def\csname pgffltA@N\endcsname{\pgfflt@readundef}
\expandafter\def\csname pgffltA@n\endcsname{\pgfflt@readundef}
\def\pgfflt@readundef #1{%
    \def\pgfflt@readnan@ok{1}%
    \if#1a\else\if#1A\else\def\pgfflt@readnan@ok{0}\fi\fi
    \if\pgfflt@readnan@ok1%
        \pgfmathfloat@a@S=3\relax%
        \pgfmathfloat@a@Mtok={0.0}%
        \pgfmathfloat@a@E=0%
        \expandafter\pgfflt@finish
    \else
        \def\pgfflt@readnan@{\pgfflt@error #1}%
        \expandafter\pgfflt@readnan@
    \fi
}
\makeatother

%%%%%%%%%%%%%%%%%%%%%%%%%%%%%%%%%%%%%%%%%%%%%%%%%%%%%%%%%%%%%%%%%%%%%%%%%%%%%%%%
% Document body
%%%%%%%%%%%%%%%%%%%%%%%%%%%%%%%%%%%%%%%%%%%%%%%%%%%%%%%%%%%%%%%%%%%%%%%%%%%%%%%%
\begin{document}
	
	% The title page
	\input{titlepage}
	
	% The table of contents which, per university regulations, is followed by a
	% total wordcount.
	\tableofcontents
	\vfill
	\noindent This thesis contains
		\immediate\write18{texcount -1 -sum -inc thesis.tex > thesis.wordcount}%
		\input{thesis.wordcount}words.
	
	\clearpage
	\listoffigures
	
	\clearpage
	\listoftables
	
	% Abstract
	\input{abstract}
	
	% Declaration of non-submission elsewhere
	\input{declaration}
	
	% University-prescribed copyright statement...
	\input{copyright}
	
	% Acknowledgements
	\input{acknowledgements}
	
	% Main body
	\input{introduction.tex}
	\input{background.tex}
	\input{building.tex}
	\input{shortestPaths.tex}
	\input{routing.tex}
	\input{placement.tex}
	\input{discussion.tex}
	\input{future.tex}
	\input{conclusions.tex}
	
	% Bibliography
	\bibliography{references}
	\bibliographystyle{alpha}
	
\end{document}
words.
	
	\clearpage
	\listoffigures
	
	\clearpage
	\listoftables
	
	% Abstract
	{
	\prefacesection{Abstract}
	
	% Single line spacing for the abstract page
	\setstretch{1.0}
	
	
	\vfill
	
	% Standard thesis information
	\begin{center}
		\textsc{\large\thesistitle}
		
		\vspace{0.5em}
		
		\thesisauthor
		
		\vspace{0.5em}
		
		A thesis submitted to the University of Manchester\\
		for the degree of Doctor of Philosophy, 2016
	\end{center}
	
	\vfill
	
	% The abstract
	
	SpiNNaker is an unconventional super computer architecture designed to
	simulate up to one billion biologically realistic neurons in real-time. To
	achieve this goal, SpiNNaker employs a novel network architecture which poses
	a number of practical problems in scaling up from desktop prototypes to
	machine room filling installations.
	
	SpiNNaker's hexagonal torus network topology has received mostly theoretical
	treatment in the literature. This thesis tackles some of the challenges
	encountered when building `real-world' systems.  Firstly, a scheme is devised
	for physically laying out hexagonal torus topologies in machine rooms which
	avoids long cables; this is demonstrated on a half-million core SpiNNaker
	prototype.  Secondly, to improve the performance of existing routing
	algorithms, a more efficient process is proposed for finding (logically)
	short paths through hexagonal torus topologies. This is complemented by a
	formula which provides routing algorithms greater flexibility when finding
	paths, potentially resulting in a more balanced network utilisation.
	
	The scale of SpiNNaker's network and the models intended for it also present
	their own challenges. Placement and routing algorithms are developed which
	assign processes to nodes and generate paths through SpiNNaker's network.
	These algorithms reduce congestion and tolerate network faults. The proposed
	placement algorithm is inspired by techniques used in chip design and is
	shown to enable larger applications to run on SpiNNaker -- with good
	performance -- than the previous state-of-the-art. Likewise the routing
	algorithm developed is able to tolerate network faults, inevitably present in
	large scale systems, with little performance overhead.
	
	
	% Required to ensure single line spacing is used for this whole block
	\par%
}

	
	% Declaration of non-submission elsewhere
	\prefacesection{Declaration}

% Single line spacing for the declaration
{
	\setstretch{1.0}
	No portion of the work referred to in this thesis has been submitted in support
	of an application for another degree or qualification of this or any other
	university or other institute of learning.
	
	\par%
}


	
	% University-prescribed copyright statement...
	\input{copyright}
	
	% Acknowledgements
	{
	\prefacesection{Acknowledgements}
	
	% Single line spacing
	\setstretch{1.0}
	
	It is often said that it is not \emph{what} you know but \emph{who} you know.
	Throughout the course of my PhD I've been exceptionally lucky to have been
	helped along by a great number of people.
	
	Both my supervisor, Jim Garside, and co-supervisor, Steve Furber, have each
	spent countless hours patiently discussing and describing all manner of
	things with me while giving me great freedom in my project. Jim's office door
	has always been open to my unexpected interruptions be it about work, writing
	or walking.  Likewise, Steve has always managed to find time for both
	technical and frivolous endeavours alike. I'm also hugely grateful to Luis
	Plana who has been a rich source of sage advice, insightful questions
	patiently suffered many a foolish question.
	
	Various parts of the work in this thesis (and numerous side projects) would
	not have been possible if not for the multitude of discussions,
	collaborations and even sheer physical hard work of Steve Temple, Javier
	Navaridas, Simon Davidson and Dave Clark. I'm also indebted to Andrew Mundy
	and Jamie Knight, both of whom have donated so much time and effort towards
	verifying and using software implementations of the ideas in this thesis.
	
	The injection of lunchtime silliness by Andrew and Jamie, along with Amanieu
	d'Antras and Andrew Webb and the other CDT members has always brightened my
	day. So to has the friendly and stimulating environment of the School of
	Computer Science and its many staff and students. Of course, I am also very
	grateful for the funding the school has provided for my research.
	
	I cannot thank my wonderful wife, Ann-Marie, enough for being by my side. She
	has given me so much kindness, love and patience and endured a lifetime's
	quota of conversations about hexagons. Finally, thanks too to rest of my
	family, especially my parents, who are to blame for starting me down this
	path and co-suffering drafts and endless rants about this document.
	
	% Required to ensure single line spacing is used for this whole block
	\par%
}

	
	% Main body
	\chapter{Introduction}

\label{sec:introduction}

%Problem area
%
%* Network construction and exploitation
%  * Cabling: Build it cheaply in terms of tech cost and install cost
%  * Routing: Get around it cheaply and reliably
%  * Placement: Use it efficiently

The Spiking Neural Network Architecture (SpiNNaker) is a novel super computer
architecture designed to simulate biologically realistic models of brains in
real time \cite{furber07}. Though neurons, the building blocks of the brain,
are relatively well understood, their complex interactions remain mysterious.
Just as understanding the workings of a transistor is insufficient to
understand a modern microprocessor, neuroscientists believe that understanding
the neurons in isolation cannot explain the brain and that understanding their
connectivity is key \cite{eliasmith13,eliasmith14}. Experiments on real brains,
however, are fraught with difficulty. Variations between individuals can be
significant and it is only possible to record tens or hundreds of the trillions
of signals in the brain, and even then only with limited control over which
signals are recorded. Computer simulations of models of large neural networks,
however, enable researchers to develop repeatable experiments and gain complete
visibility of any signal and any neuron. Models such as SPAUN
\cite{eliasmith12}, built from millions of simulated neurons, have shown great
promise in expanding our understanding of higher level brain functions such as
memory and simple problem solving.  Unfortunately these neural models are
expensive to simulate, requiring hours of compute time to simulate each second
of neural activity. As well as being inconvenient, this precludes the use of
robotics to immerse these models in real world environments and also limits
studies of long-term behaviours such as learning.

SpiNNaker is designed to enable the real time simulation of models containing
up to one billion neurons -- approximately \SI{1}{\percent} of a human brain or
ten mouse brains \cite{furber06}. To achieve this goal, the largest planned
SpiNNaker machine will contain over one million low-powered computer processors
interconnected by a bespoke network architecture.

SpiNNaker's large processor count matches the current trend in super computers
where processor counts are growing exponentially \cite{meuer16j}, mimicking the
growth of the number of components in the processors themselves predicted by
Gordon Moore's famous `law' \cite{moore75}. As a result of this growth, the
interconnection networks which enable these processors to work together have
grown in importance \cite{dally04}.  Network designers must carefully balance
performance against practicality and financial cost.  SpiNNaker's network is no
exception to this rule and, as the systems scale up from desktop prototypes to
machine-room scale installations, the reality of building and exploiting these
machines presents an array of challenges.

As in all super computers, SpiNNaker's network interconnects its processors in
a particular network topology which defines how different processors may
communicate with each other. Unlike the tree and $N$-dimensional torus
topologies found in contemporary super computers \cite{dally04}, SpiNNaker
employs a `hexagonal torus topology'. In this topology, nodes in SpiNNaker's
network fit together in a honeycomb-like pattern where messages may `hop' from
node to node to reach their destination. As we will see in
chapter~\ref{sec:background}, the hexagonal torus topology, in theory, sits at
a `sweet spot' in terms of network performance and practicality. As the first
known large-scale installation of the hexagonal torus topology, however, there
remain a number of practical challenges for large spinnaker machines arising
from this choice.

As super computer networks have grown in scale to millions of processors the
task of dealing with previously rare faults has grown.  Though fault rates in
networks remain consistently low, architectures such as SpiNNaker may have
hundreds of thousands of links meaning even fault rates of a fraction of a
percent will impact tens or hundreds of links. To enable reliable operation,
networks must be able to adapt the routes taken by messages through the network
to avoid faulty links and nodes. The techniques employed are often closely tied
to a particular network architecture and consequently SpiNNaker's novel network
architecture demands its own approach.

Another challenge introduced by the growing scale of super computers is making
\emph{efficient} use of network resources. Communicating processes should be
located on logically `nearby' nodes to reduce network load. The neural models
for which SpiNNaker is designed are often described abstractly, rather than
geometrically, using modelling languages such as PyNN~\cite{davison08} and
Nengo~\cite{eliasmith04}.  Because of this, the communication requirements of
simulations can be highly irregular making an efficient placement of processes
onto processors in the machine non-trivial.

%Contributions
%
%* Cabling scheme for hexagonal toruses without long cables
%* Efficient installation technique for dense systems
%* Exhaustive and efficient route calculation in hex toruses
%* Fault tolerant routing scheme exploiting SpiNNaker's odd router
%* Placement based on SA a: works very well and b: suggests circuit placement is
%  a good source of inspiration.

This thesis addresses the practical challenges of scaling up the SpiNNaker
architecture in a real-world setting summarised by these research questions:

\begin{enumerate}
	
	\item Can the hexagonal torus topology be deployed and used in real, large
	scale systems?
	
	\item Does SpiNNaker's router architecture help, or hinder fault tolerance?
	
	\item How can the parts of a neural simulation be placed onto a large
	hexagonal torus topology to reduce network load?
	
\end{enumerate}

%Structure
%
%* Chapter 2: Background: detailed dive into what's in SpiNNaker, why its
%  really so unusual. Also looks at what applications run on SpiNNaker and how
%  they work.
%* Chapter 3: How to build a really big SpiNNaker machine.
%* Chapter 4: How to find your way around that machine.
%* Chapter 5: How to find your way around that machine even when its broken.
%* Chapter 6: Now you can walk, time to run.
%* Chapter 7: Wrapping up.
%* Appendices: Hard-to-come-by theoretical and practical details useful if
%  you're about to continue where this research left off but be useful but
%  otherwise hard to come by, especially in one place.

Chapter~\ref{sec:background} introduces the SpiNNaker architecture and, in
particular, describes its hexagonal torus topology and network architecture.

In chapter~\ref{sec:building}, I develop a cabling scheme for large hexagonal
torus topologies which enables arbitrarily large networks to be constructed
using only short, inexpensive cables. This theoretical work is then evaluated
through the construction of a range of prototype SpiNNaker systems. The largest
of these prototypes contains over half a million processor cores and spans
several machine room cabinets. In addition, I propose the use of built-in
diagnostic facilities to assist technicians performing network installation and
maintenance. This technique is found to greatly reduce the effort required and
the number of mistakes made.

In chapters~\ref{sec:shortestPaths}~and~\ref{sec:routing} I develop new routing
techniques for SpiNNaker's network. Chapter~\ref{sec:shortestPaths} develops a
new approach to finding the shortest paths through hexagonal torus topologies,
an integral part of many routing algorithms. This newly proposed approach is
cheaper to compute than the state of the art and, unlike previous efforts, is
able to discover all valid short paths through the topology. This theoretical
advance brings hexagonal torus topologies in line with conventional topologies
by providing routing algorithms with complete information about the paths
available to them. In chapter \ref{sec:routing} I propose a fault tolerant
routing algorithm for SpiNNaker which is able to avoid arbitrary static fault
patterns with minimal performance overhead. A key finding of this chapter is
that the flexibility afforded to fault tolerant routing algorithms by
SpiNNaker's unconventional router architecture is what facilities the low
overheads reported in this chapter.

Finally, in chapter~\ref{sec:placement}, I explore the problem of application
placement in SpiNNaker's network. As in other networks and applications, neural
simulations should be arranged such that communication occurs primarily between
processors close together in the network to control network load. Due to the
irregular connectivity and large scale of the neural models expected to run on
SpiNNaker, an automated approach is necessary. I develop a novel placement
algorithm based on algorithms used for circuit layout in computer chips. My
algorithm is found to allow some larger neural models to run on SpiNNaker for
the first time while enabling other applications to run at greater speeds. In
addition, synthetic benchmarks containing over one million processes indicate
that this algorithm should handle the anticipated demands of the neural models
expected to run on large-scale SpiNNaker installations.

	\chapter{The SpiNNaker Architecture}
	
	\label{sec:background}
	
	SpiNNaker is a massively parallel computer architecture designed to simulate
	biologically realistic neural models \cite{furber07}. In this chapter we will
	explore this unconventional architecture in detail, starting with its purpose
	before focusing on its most unconventional feature: its network.
	
	% * Purpose
	%   * Spiking neural simulations
	%     * Neural modelling: PyNN, Nengo...
	%     * Parallelisation + communication
	
	\section{Neural simulation}
		
		Human brains contain billions of neurons connected together by trillions of
		`synapses'. Neurons communicate by transmitting and receiving `spikes'
		through their synapses. Each spike is `valueless' in that a spike's only
		significant features are when it arrives and where it has come from.
		
		\begin{figure}
			\center
			\buildfig{figures/lif-neuron.tex}
			
			\caption{A Leaky Integrate-and-Fire (LIF) neuron.}
			\label{fig:lif-neuron}
		\end{figure}
		
		Though some detailed models of the electrochemical processes occurring
		inside neurons are computationally intensive, simplified models such as the
		Leaky Integrate-and-Fire (LIF) model can be implemented in just a handful
		of CPU instructions \cite{vainbrand11}. Figure~\ref{fig:lif-neuron}
		illustrates a simple LIF neuron in which incoming spikes cause charge to
		build up (integrated) which over time, leaks away. If an incoming spike
		causes the charge to rise above a certain threshold, the neuron `fires'
		producing an outgoing spike. Despite the simplicity of this model, large
		neural networks such as Spaun \cite{eliasmith12} -- built entirely from LIF
		neurons -- exhibit complex behaviours such as fine motor control and
		problem solving.
		
		The computational expense of large scale neural simulations does not arise
		from the cost of modelling neurons but instead from distributing spikes. In
		biology, neurons produce spikes at an average rate of \SI{10}{\hertz} and
		synapses connect each neuron's output to (order) \num{1000}~neurons
		\cite{navaridas09}. Consider an example neural model with $7\times10^7$
		neurons, approximately the number in a house mouse and
		$\nicefrac{1}{10}^\textrm{th}$ of the design target of SpiNNaker. This
		network might produce $7\times10^8$~spikes per second. Because each neuron
		connects to many others, this equates to $7\times10^{11}$ spikes being
		received per second. If each spike were transmitted as a UDP datagram
		containing a single \SI{32}{\bit} payload, the total network throughput
		required for this simulation would be \SI{179.2}{\tera\bit\per\second}. At
		the time of writing, this is more than double the bisection bandwidth (the
		theoretical worst-case throughput) of the world's most powerful super
		computer \cite{dongarra16}.
	
	\section{Network architecture}
		
		Architectures such as IBM's Blue Gene \cite{chiu11} and Cray's XK7
		\cite{ornl16} employ powerful compute nodes connected together using
		networks designed to transfer multi-kilobyte blocks of data between nodes.
		Since neural models have relatively light computational requirements and
		communications are based on small pieces of data (spikes), this type of
		architecture is poorly suited to the task.
		
		SpiNNaker's architectural target is to support realtime simulations of up
		to one billion neurons. Since neural models such as LIF are inexpensive to
		model and many neurons can be simulated independently in parallel,
		SpiNNaker employs many small, energy efficient ARM processors
		\cite{furber07}. To support the unusual communication requirements of
		neural simulations, a bespoke interconnection network is used which is the
		background to this thesis.
		
	%   * SpiNNaker chip
	%     * Cores
	%     * SDRAM
	%     * NoC
	%     * Router
		
		\begin{figure}
			\center
			%\includegraphics[width=19mm]{figures/spinnakerChip.jpg}
			\buildfig{figures/hex-chips.tex}
			
			\caption[SpiNNaker chips connected to their six neighbours.]%
			{SpiNNaker chips (actual size) connected to their six neighbours.}
			\label{fig:spinnakerChip}
		\end{figure}
		
		The fundamental building block of the SpiNNaker architecture is the
		SpiNNaker chip (figure \ref{fig:spinnakerChip}) \cite{furber13}. Each chip
		contains eighteen low power ARM 968 processor cores each capable of
		simulating between \num{200} and \num{2000} LIF neurons in real time
		\cite{mundy15}.  Each core has a total of \SI{96}{\kilo\byte} of private
		Tightly-Coupled Memory (TCM) and shares access to \SI{128}{\mega\byte} of
		on-chip SDRAM with other cores on the same chip. Finally, each chip
		contains a programmable router which routes network packets to and from the
		local cores and six neighbouring SpiNNaker chips. SpiNNaker machines are
		constructed by combining many SpiNNaker chips.
		
		\begin{figure}
			\center
			\buildfig{figures/spinnaker-packet.tex}
			
			\caption{SpiNNaker's \SI{40}{\bit} and \SI{72}{\bit} multicast packet
			format.}
			\label{fig:spinnaker-packet}
		\end{figure}
		
		Processor cores can communicate by sending and receiving network packets
		forwarded by routers through the network. Since SpiNNaker's network is
		designed to transmit neural spike events efficiently, individual network
		packets are small, either \SI{40}{\bit} or \SI{72}{\bit} compared with tens
		or hundreds of byte packets in typical network architectures.
		
		In a real-time simulation, the time at which a spike is produced is
		implicitly indicated by the time it is received -- since at biological
		timescales a computer network delivers packets `instantaneously'.
		Consequently, the only information which must be explicitly encoded is the
		identity of the neuron which produced the spike. In SpiNNaker, a spike may
		be encoded by using a single \SI{40}{\bit} `multicast packet' whose format
		is illustrated in figure~\ref{fig:spinnaker-packet}.  The \SI{8}{\bit}
		header is used by SpiNNaker's routers to determine the type of packet and
		the \SI{32}{\bit} `routing key' is used to identify the neuron which
		produced the packet. The routing key is also used by SpiNNaker's routers to
		determine how the packet should be directed through the network.
		
		The optional \SI{32}{\bit} payload is not used by conventional spiking
		neural simulations \cite{galluppi10} but has been exploited to enable more
		efficient simulation of a particular class of neural models \cite{mundy15}.
	
	\section{The SpiNNaker router}
		
		The SpiNNaker router employs an unconventional design which, despite its
		compact size and small energy requirements, implements a flexible multicast
		routing scheme. Unlike conventional routers which often employ hard-coded
		routing rules \cite[chapter~8]{dally04}, the SpiNNaker router uses a
		programmable `routing table' to determine how packets should be forwarded.
		In addition, to avoid deadlocks, SpiNNaker's router employs a simple,
		timeout-based mechanism which exploits the ability of neural networks to
		tolerate occasional missing packets. As we will see in chapter
		\ref{sec:routing}, this mechanism greatly simplifies the task of routing in
		SpiNNaker's network. In this section we'll look at these features in
		greater detail.
		
		\subsection{Routing tables}
		
			When a multicast packet arrives at a SpiNNaker router (either from a
			local core or a neighbouring chip), the router looks up the routing key
			in its routing table. This table consists of \num{1024} programmable
			table entries, each specifying a routing key bit pattern and mask to
			match and a set of routes.  When a multicast packet's key is matched by a
			routing entry the packet is forwarded along every route specified by that
			entry, potentially duplicating the packet. This `multicast' technique
			allows packets to be transmitted once but received in a number of places
			while making efficient use of the network \cite{navaridas12}.
			
			Though routing table entries are in finite supply (\num{1024} entries per
			router), it is still possible for many thousands of traffic flows to be
			routed through a single router. The bit pattern and mask in each routing
			entry matches against the 32~bits of a routing key as either
			`\texttt{1}', `\texttt{0}' or `\texttt{X}' (don't care).  This means that
			a single routing entry may, for example, be used to match all routing
			keys with a certain prefix. If a routing key is not matched by any entry
			in the routing table then the packet is `default routed' in a straight
			line. For example if a packet with an unmatched key is received from the
			chip to the left, the packet will be default routed to the chip on the
			right. By assigning routing keys such that neurons whose spikes are sent
			to similar destinations share a similar prefix, the number of routing
			entries required by a simulation is greatly reduced \cite{davies12}.
			
			\begin{figure}
				\center
				\buildfig{figures/routing-example.tex}
				
				\caption[Multicast routing example.]%
				{Multicast routing example with \SI{4}{\bit} routing keys. Each
				box represents a SpiNNaker chip whose router has been programmed with
				the routing entries shown. Grey lines mark connections between chips.}
				\label{fig:routing-example}
			\end{figure}
			
			Consider the simplified example in figure~\ref{fig:routing-example} in
			which a number of (\SI{4}{\bit}) routing table entries have been
			configured in the routers of a small SpiNNaker network. If a packet with
			the routing key \texttt{1011} is transmitted by a core in the chip
			labelled $(0, 0, 0)$, this will match the first routing table entry on
			that chip and will be routed to chip $(1, 0, 0)$. On chip $(1, 0, 0)$,
			the packet once again matches the first routing entry and is routed to
			chip $(1, 0, -1)$. On $(1, 0, -1)$, no match is made so the packet is
			default routed to $(1, 0, -2)$. On this chip, the packet matches a
			routing entry which routes the packet to core~7. In this example, default
			routing allows only three routing table entries to direct a packet
			through four chips.
			
			As a second example, if a packet with the routing key \texttt{0010} is
			transmitted by a core on chip $(0, 0, 0)$, this key will be matched by
			the second routing entry since \texttt{X}s in the table entry will match
			both \texttt{1}s and \texttt{0}s in the corresponding bits of the routing
			key. When the packet arrives at chip $(0, 0, -1)$ the matching routing
			entry forwards the packet to both $(0, 1, -1)$ and $(1, 0, -1)$
			simultaneously. The copy of the packet arriving at $(0, 1, -1)$ is routed
			to core~5 on that chip.  Meanwhile, the copy forwarded to $(1, 0, -1)$ is
			duplicated again with one copy being routed to core~11 and another being
			routed to chip $(1, 0, -2)$. Here the packet is finally delivered to
			core~6. In this example, the ability of the router to multicast
			(duplicate) packets as they pass through the network meant that sending
			one copy of the packet was sufficient to reach three destination cores.
			In addition, by using \texttt{X}s in the routing table entry, the same
			routing entries are sufficient to route packets with the keys
			\texttt{0000}, \texttt{0001}, \texttt{0010} and \texttt{0011}.
			
			In spite of these mechanisms, it is still possible for an application to
			run out of routing table entries. As we will see in
			chapter~\ref{sec:placement} by arranging applications appropriately
			within SpiNNaker's network, routing table usage can be reduced. In
			addition, other behaviours of SpiNNaker's router may be exploited to
			compress an applications routing tables further, however the techniques
			employed are beyond the scope of this thesis \cite{mundy16}.
		
		\subsection{Timeouts}
			
			SpiNNaker's router is built on a pipeline architecture. As shown in
			figure~\ref{fig:router-architecture}, the router is fed packets by an
			arbiter which serialises packets arriving from other chips and local
			cores. Every (\SI{100}{\mega\hertz}) clock cycle, the router pipeline
			accepts one packet from the arbiter and routes a packet to one or several
			output links. If any of the required output ports are busy then the
			packet is not forwarded to any output link and the pipeline stalls. Once
			a packet has been blocked for a programmable timeout, it is dropped
			(discarded) and routing continues as usual for next packet in the
			pipeline. Links become blocked while transmitting packets or waiting for
			the remote receiver to become ready. For example, a receiving processor
			core may be busy performing some computation or a receiving router may be
			blocked waiting for some of its outputs to become ready.
			
			\begin{figure}
				\center
				\buildfig{figures/router-architecture.tex}
				
				\caption{SpiNNaker router architecture}
				\label{fig:router-architecture}
			\end{figure}
			
			The timeout-based packet dropping mechanism is designed to defuse
			deadlocks in the network. For example, if two routers are trying to send
			each other a packet at the same time they may become deadlocked, each
			waiting for the other router to accept a packet before continuing.
			SpiNNaker's timeout mechanism breaks deadlocks by dropping packets which
			have been blocked for some time and therefore may be in a deadlock.  Once
			a packet has been dropped it is left to software to either tolerate the
			missing packet or trigger a retransmission. In neural simulations, as in
			biology, the loss of a single spike is unlikely to have a significant
			impact on the behaviour of a neural model and therefore these simulations
			are inherently tolerant of occasional dropped packets. During application
			loading and other system tasks, a higher level, software driven protocol
			based on acknowledgements and retransmissions is used to ensure
			guaranteed delivery.
			
			% TODO: MENTION TIMEOUT VALUE USED?
			% Router timeouts must be configured to be long enough that delays in
			% packet transmission, for example due to the time taken for packets to
			% traverse a link, do not trigger packet dropping. Conversely, the timeout
			% should be as short as possible to reduce the time the router is
			% blocked and maximise network throughput.
	
	\section{The hexagonal torus topology}
		
		Each SpiNNaker chip is a node in a `hexagonal torus topology' as
		illustrated in figure~\ref{fig:hexagonalTorusTopology}. Network packets
		sent by SpiNNaker's processor cores may `hop' through several nodes in the
		network to reach their intended destination. In each hop, a packet may
		advance one node along one of the three axes of the topology. For example,
		a packet sent by the node labelled $\alpha$ (in the bottom-left corner) to
		the node labelled $\beta$, might take the following sequence of hops:
		X$^+$, X$^+$, Z$^-$. Packets sent from $\alpha$ to $\gamma$ might take the
		route: X$^-$, X$^-$, Y$^+$, Y$^+$. The first hop of this route `wraps
		around' from the bottom-left node to the bottom-right node in a single hop.
		
		\begin{figure}
			\center
			\buildfig{figures/hexagonalTorusTopology.tex}
			
			\caption[A hexagonal torus topology.]%
			{A hexagonal torus topology. Each hexagon represents a node (i.e.
			a SpiNNaker chip). Touching nodes are directly connected. Nodes on edges
			$a$, $b$ and $c$ are also directly connected to the corresponding nodes
			on edges $a'$, $b'$ and $c'$, respectively. The three axes of the
			hexagonal torus topology, `X', `Y' and `Z' are also shown.}
			\label{fig:hexagonalTorusTopology}
		\end{figure}
		
		\begin{figure}
			\center
			\begin{subfigure}{0.39\linewidth}
				\center
				\includegraphics[width=\linewidth]{figures/torus-3d-flat.pdf}
				\caption{}
				\label{fig:torus-3d-flat}
			\end{subfigure}
			~~
			\begin{subfigure}{0.26\linewidth}
				\center
				\includegraphics[width=\linewidth]{figures/torus-3d-tube.pdf}
				\caption{}
				\label{fig:torus-3d-tube}
			\end{subfigure}
			~~
			\begin{subfigure}{0.23\linewidth}
				\center
				\includegraphics[width=\linewidth]{figures/torus-3d-torus.pdf}
				\caption{}
				\label{fig:torus-3d-torus}
			\end{subfigure}
			
			\caption{Visualisation of a hexagonal torus topology as a torus.}
			\label{fig:torus-3d}
		\end{figure}
		
		The wrap around connections in the topology are what give it the `torus'
		part of its name. Figure~\ref{fig:torus-3d-flat} shows a hexagonal torus
		topology drawn flat as in the previous figure. If the topology is rolled up
		into a tube such that the top and bottom nodes become directly adjacent, a
		tube is formed as in figure~\ref{fig:torus-3d-tube}. This tube can then be
		bent to bring together the nodes at the ends of the tube to form a torus as
		shown in figure~\ref{fig:torus-3d-torus}.
		
		A hexagonal torus topology is typically defined in terms of its width and
		height along the X and Y axes respectively. For example,
		figure~\ref{fig:hexagonalTorusTopology} shows a $10\times10$ hexagonal
		torus.  The nodes in a hexagonal torus topology are addressed using
		hexagonal coordinates of the form $(x, y, z)$ \cite{patel15}. The bottom
		left node (labelled $\alpha$ in the figure) has the coordinate $(0, 0, 0)$
		and other nodes are assigned coordinates according to the number of hops
		along each dimension from $(0, 0, 0)$, for example node $\beta$ has the
		coordinate $(2, 0, -1)$. Because the hexagonal torus topology's axes are
		non-orthogonal, it is possible to define several coordinates for the same
		location. For example $(3, 1, 0)$ and $(1, -1, -2)$ are also valid
		coordinates for node $\beta$. These dual coordinates emerge from the fact
		that adding $(1, 1, 1)$ to a coordinate produces an equivalent, but
		different, coordinate. This phenomenon is explained in detail in
		appendix~\ref{app:minimal-hex-coordinates} and related phenomena will be
		discussed in chapter~\ref{sec:shortestPaths}.
		
		The hexagonal torus topology was chosen over a more conventional network
		topology -- such as a 2D or 3D torus (sometimes known as a 2-ary $N$-cube
		or 3-ary $N$-cube respectively) \cite[chapters~3~and~5]{dally04} -- due to
		its balance of theoretical performance and practicality. The bisection
		bandwidth of a topology indicates the theoretical worst-case total
		throughput the network is able to sustain \cite[chapter~1]{dally04}.  In
		networks with homogeneous link throughput, bisection bandwidth is
		determined by the number of links cut by a balanced bisection of the
		network.  Figure~\ref{fig:bisection-bandwidth} illustrates the bisections
		of several torus topologies.
		
		\begin{figure}
			\center
			\begin{subfigure}[b]{0.3\linewidth}
				\center
				\buildfig{figures/bisection-bandwidth-2d.tex}
				
				\caption{2D Torus}
				\label{fig:bisection-bandwidth-2d}
			\end{subfigure}
			\begin{subfigure}[b]{0.3\linewidth}
				\center
				\buildfig{figures/bisection-bandwidth-hex.tex}
				
				\caption{Hexagonal Torus}
				\label{fig:bisection-bandwidth-hex}
			\end{subfigure}
			\begin{subfigure}[b]{0.3\linewidth}
				\center
				\buildfig{figures/bisection-bandwidth-3d.tex}
				
				\caption{3D Torus}
				\label{fig:bisection-bandwidth-3d}
			\end{subfigure}
			
			\caption[Bisections of torus topologies.]%
			{Bisections of torus topologies. Connections cut by the bisection
			are drawn as lines.}
			\label{fig:bisection-bandwidth}
		\end{figure}
		
		In a $N \times N$ 2D torus topology, the bisection bandwidth is $2N$~links
		and each node requires four links. The hexagonal torus topology requires
		six links per node but provides double bisection bandwidth ($4N$~links).
		The 3D torus topology also requires six links per node but by connecting
		the nodes differently achieves a bisection bandwidth of $8N$~links.  The 3D
		torus topology, however, comes at a price -- when immersed into the
		(approximately) 2D space provided by a large machine room or row of server
		cabinets, some connections require long cables. By contrast, the 2D and
		hexagonal torus topologies are both inherently two dimensional and
		consequently do not suffer from this effect. The hexagonal torus topology,
		therefore, shares the practicality of construction of a 2D torus while
		still gaining some of the performance of a 3D torus topology. In addition,
		because nodes in a hexagonal torus topology have a greater number of links,
		greater redundancy is available in the network to tolerate faults.
		
		Most torus topologies, including hexagonal, 2D and 3D toruses, have a
		related `mesh' topology. These mesh topologies maintain the same general
		connectivity structure as their torus topologies but omit wrap-around
		links. In practice, this saves a small number of links at the expense of
		halving the network's bisection bandwidth.  Because of their poorer
		performance, mesh networks are rarely used as the basis of a network
		architecture. Mesh networks, however, are occasionally formed when a
		network is partitioned into several smaller sub-networks to allow multiple
		users to share a system \cite{spalloc16}.
		
		\begin{figure}
			\center
			\begin{subfigure}[b]{0.45\linewidth}
				\center
				\buildfig{figures/hexagonal-torus.tex}
				\caption{Hexagonal torus}
				\label{fig:topo-compare-hexagonal-torus}
			\end{subfigure}
			\begin{subfigure}[b]{0.45\linewidth}
				\center
				\buildfig{figures/h-torus.tex}
				\caption{H-torus}
				\label{fig:topo-compare-h-torus}
			\end{subfigure}
			
			\caption[Hexagonal torus vs. H-torus topology.]%
			{Hexagonal torus vs. H-torus topology. Each numbered hexagon
			represents a node. The thick outline indicates the bounds of the
			topology after which the network repeats. In each topology, the path
			taken by advancing in the Y$^+$ direction from the node labelled `0' is
			shown.}
			\label{fig:topo-compare}
		\end{figure}
		
		\label{sec:hex-vs-h-torus}
		
		The hexagonal torus topology is not to be confused with the `H-torus'
		topology. This topology also uses a hexagonal tiling of nodes and even
		wraps this tiling into a torus-like topology \cite{zhao08}. However,
		H-torus topologies have very different characteristics to the hexagonal
		torus topology and are related to `twisted torus' topologies
		\cite{camara10}. For example, figure~\ref{fig:topo-compare} illustrates one
		major difference in the way paths wrap around the peripheries of both
		topologies.
	
	\section{Scaling-up SpiNNaker machines}
		
		To build large SpiNNaker systems comprising of tens of thousands of
		SpiNNaker chips, groups of forty-eight chips are mounted onto circuit
		boards as illustrated in figure~\ref{fig:spinnakerBoard}. These boards may
		be connected together to form larger systems.  Figure~\ref{fig:threeboard}
		shows a prototype three board system. Though the chips are
		\emph{physically} arranged in a (nearly) $7\times7$ grid on each SpiNNaker
		board, they logically form a hexagonal `wrapped triple'
		\cite{davidsonWiring} (see appendix~\ref{sec:partitioning}) which logically
		fit together as illustrated in figure~\ref{fig:threeboard-separate}. The
		labelled exposed corners of the three forty-eight chip boards connect
		together to form a $12\times12$ hexagonal torus topology as illustrated in
		figure~\ref{fig:threeboard-wrapped}. Larger SpiNNaker machines are
		assembled by combining more boards.
		
		\begin{figure}
			\center
			\begin{subfigure}[b]{0.45\linewidth}
				\center
				\includegraphics[width=\linewidth]{figures/spinnakerBoard.jpg}
				
				\caption{A SpiNNaker board}
				\label{fig:spinnakerBoard}
			\end{subfigure}
			~~~
			\begin{subfigure}[b]{0.45\linewidth}
				\center
				\includegraphics[width=\linewidth]{figures/threeboard.jpg}
				
				\caption{Three board prototype}
				\label{fig:threeboard}
			\end{subfigure}
			
			\vspace*{1em}
			
			\begin{subfigure}[b]{0.45\linewidth}
				\center
				\buildfig{figures/threeboard-separate.tex}
				
				\caption{Three board topology}
				\label{fig:threeboard-separate}
			\end{subfigure}
			~~~
			\begin{subfigure}[b]{0.45\linewidth}
				\center
				\buildfig{figures/threeboard-wrapped.tex}
				
				\caption{\ldots{}as a parallelogram}
				\label{fig:threeboard-wrapped}
			\end{subfigure}
			
			\caption{SpiNNaker boards and their topology.}
			\label{fig:spinnaker-boards}
		\end{figure}
		
		
		SpiNNaker chips on the same circuit board connect using low power links
		requiring sixteen wires each.  If this link technology were used to connect
		chips on neighbouring boards, each pair of boards would need to be
		connected with a 128~wire cable.  Cables and connectors supporting this
		many signals are expensive, unreliable and physically large. Instead,
		chip-to-chip connections between boards are multiplexed and demultiplexed
		onto a single High-Speed Serial (HSS) link \cite{athavale05} carried via
		commodity S-ATA cables which are often used to connect hard disks in
		desktop computers and servers \cite{sata3spec}. The six high-speed links
		are implemented by three onboard FPGAs (the three large chips at the top of
		the SpiNNaker board) and are logically transparent to the underlying
		network. The underlying technology and the choice of S-ATA cables limits
		each board-to-board connection to spanning at most one metre gaps. In
		chapter~\ref{sec:building} I present a cabling scheme for hexagonal torus
		topologies which enable large SpiNNaker systems to be assembled using only
		short cables between boards.
		
	\section{Conclusions}
		
		The SpiNNaker architecture has been designed to enable the simulation of
		large biologically realistic neural models in real time. To support this,
		its network architecture takes on an unconventional design based on a
		custom router and hexagonal torus topology. In the remainder of this
		thesis, I will tackle a number of the challenges in scaling up the
		SpiNNaker architecture outlined in this chapter.

	\chapter{Building large SpiNNaker machines}
	
	Like any super computer, physically putting together a large SpiNNaker
	machine poses many challenges in terms of organisation, assembly and
	maintainance. One of the key tasks in this process is the installation of
	network cables such that a desired overall network topology is constructed.
	The largest planned SpiNNaker machine will use \num{3600} S-ATA
	\cite{sata3spec} cables to interconnect its \num{1200} circuit boards,
	creating a hexagonal torus topology. Since the machine will be installed
	within standard server room cabinets (which are not available in a
	giant-doughnut form-factor) a mapping from a board's logical location in the
	network topology to its physical location must be constructed. In addition,
	the interconnect technology employed by SpiNNaker restricts the length of
	S-ATA cables used to $\le$~\SI{1}{\meter}, constraining the possible mappings
	used. In addition the practical issues of installation complexity and
	maintainance must be considered since all \num{3600} cables must ultimately
	be installed and maintained by human operators.
	
	In this chapter I describe a novel technique for physically laying out
	machines configured in hexagonal torus topologies, such as SpiNNaker, in
	commercial machine rooms, building on the techniques used in more
	conventional torus topologies. In addition, I also propose a new methodology
	for installing and maintaining super computer cabling which which exploits
	existing diagnostic features of the SpiNNaker hardware to interactively guide
	and validate cable installation. Finally, I demonstrate how these new
	techniques have been used successfully to interconnect a prototype
	\num{518400} core SpiNNaker machine in substantially less time than the
	industry norm.
	
	In this chapter, the term \emph{unit} refers to the smallest physical
	ecomponent between which connections connections are to be made. For example,
	in a SpiNNaker machine a unit is a 48-chip board while in data center, a unit
	might be a server blade.
	
	\section{Related work}
		
		In this section I describe the techniques conventionally employed when
		laying out and interconnecting the units within super computers. Due to
		SpiNNaker's hexagonal torus topology and dense physical packing of units,
		these existing techniques are found to be insufficient. In the remainder of
		the chapter we will explore solutions to the limitations exposed below.
		
		\subsection{Avoiding long cables}
			
			Na\"ive arrangements of torus topologies, including hexagonal torus
			topologies, feature long `wrap-around' connections which connect units at
			the peripheries of the system. These connections can be problematic for
			numerous reasons:
			
			\begin{description}
				
				\item[Performance] Signal quality diminishes as cables get longer,
				requiring the use of slower signalling speeds, increased error
				correction overhead or more complex hardware.
				
				\item[Energy] Longer cables require higher drive strengths and/or
				buffering to maintain signal integrity.
				
				\item[Cost] Cost Shorter cables are cheaper than long ones.  Longer
				cables imply more wire in a given space making the tasks of routing or
				cable installation more difficult increasing labour costs by as much as
				$5\times$ \cite{curtis12}.
				
			\end{description}
			
			In conventional torus topologies the need for long cables is eliminated
			by folding and interleaving units of the network \cite{dally04}. For
			example, for a 1D torus topology (a ring network), one long connection
			exists to connect the two opposite sides of the system. To remove these
			long connections, half the units are `folded' on top of the others and
			then this arrangement of units is interleaved as illustrated in figure
			\ref{fig:ring-folding}.
			
			\begin{figure}
				\center
				\begin{subfigure}[b]{0.39\linewidth}
					\center
					\buildfig{figures/ring-folding-row.tex}
					\caption{A ring network}
					\label{fig:ring-folding-row}
				\end{subfigure}
				\begin{subfigure}[b]{0.24\linewidth}
					\center
					\buildfig{figures/ring-folding-folded.tex}
					\caption{Folded}
					\label{fig:ring-folding-folded}
				\end{subfigure}
				\begin{subfigure}[b]{0.35\linewidth}
					\center
					\buildfig{figures/ring-folding-interleaved.tex}
					\caption{Folded and interleaved}
					\label{fig:ring-folding-interleaved}
				\end{subfigure}
				
				\caption{Folding and interleaving a ring network to reduce maximum wire
				length.}
				\label{fig:ring-folding}
			\end{figure}
			
			Folding and interleaving has the effect of approximately doubling the
			average cable length but also eliminates the need for a cable spanning
			the entire system. Since the mean cable length is typically already
			short, doubling it in exchange for a substantially reduced maximum cable
			length is often preferable.
			
			The folding and interleaving process may be extended to $N$-dimensional
			torus topologies by folding each dimension in turn. Since all dimensions
			are orthogonal, the folding process only moves units in the dimension
			being folded. In the hexagonal torus topology, however, the three
			dimensions are non-orthogonal and thus folding in one dimension also
			moves units in other dimensions, preventing the edges of the torus
			meeting as illustrated in figure \ref{fig:failing-to-fold-hex-toruses}.
			
			\begin{figure}
				\center
				\begin{subfigure}[b]{0.24\linewidth}
					\center
					\buildfig{figures/failing-to-fold-hex-toruses-none.tex}
					\caption{Not folded}
					\label{fig:failing-to-fold-hex-toruses-none}
				\end{subfigure}
				\begin{subfigure}[b]{0.24\linewidth}
					\center
					\buildfig{figures/failing-to-fold-hex-toruses-x.tex}
					\caption{X}
					\label{fig:failing-to-fold-hex-toruses-x}
				\end{subfigure}
				\begin{subfigure}[b]{0.24\linewidth}
					\center
					\buildfig{figures/failing-to-fold-hex-toruses-y.tex}
					\caption{Y}
					\label{fig:failing-to-fold-hex-toruses-y}
				\end{subfigure}
				\begin{subfigure}[b]{0.24\linewidth}
					\center
					\buildfig{figures/failing-to-fold-hex-toruses-z.tex}
					\caption{Z}
					\label{fig:failing-to-fold-hex-toruses-z}
				\end{subfigure}
				
				\caption{Schematics showing hexagonal torus topologies folded along
				each of their non-orthogonal dimensions. Note that folding along
				the Z axis brings the \emph{wrong} edges closer together.}
				\label{fig:failing-to-fold-hex-toruses}
			\end{figure}
		
		\subsection{Cabling installation}
			
			Existing machine room installations feature very repetitive cabling
			patterns which can easily be memorised by a human technician. For example
			in BlueGene super computers the connectivity between units is highly
			regular \cite{lakner07} while in data centre networks cabling often
			centres around a small number of high-port-count switches
			\cite{cisco07,csernai15}. Cable installation is usually only aided by
			the labelling of connectors and sockets in a standardised manner
			\cite{tia2006} such as in figure \ref{fig:bgWiring}.
			
			\begin{figure}
				\center
				\begin{subfigure}[t]{0.5\textwidth}
					\begin{tikzpicture}
						\node (cables) [inner sep=0]
						      {\includegraphics[width=\textwidth]{figures/bgCables.png}};
						\node (sockets) [inner sep=0, below=1.0em of cables]
						      {\includegraphics[width=\textwidth]{figures/bgSockets.png}};
						
						% Point at label on cable
						\draw [white, <-, line width=0.4em]
						      ([shift={(0.7cm, -0.3cm)}]cables.center)
						      -- ++(45:1cm);
						
						% Point at label on socket
						\draw [white, <-, line width=0.4em]
						      ([shift={(-1.0cm, 1.1cm)}]sockets.center)
						      -- ++(-45:1cm);
					\end{tikzpicture}
					
					\caption{Pre-labelled cables and sockets}
					\label{fig:bgWiringLabels}
				\end{subfigure}
				~
				\begin{subfigure}[t]{0.30\textwidth}
					\includegraphics[height=6.15cm]{figures/bgWiring.jpg}
					
					\caption{Installation of cables}
					\label{fig:bgWiringInstallation}
				\end{subfigure}
				
				\caption{BlueGene/Q cable installation \cite{cscs13}}
				\label{fig:bgWiring}
			\end{figure}
			
			Despite the regularity and careful labelling of cables, the cost of
			installation and maintenance alone can be significant with costs in the
			range of \$45-95 per \SI{1}{\meter} cable run and \$100-400 for runs of
			\SI{10}{\meter} reported in the literature \cite{mudigonda11}. Much of
			this cost is due to the care required during installation to avoid
			miswiring and ensure that cooling airflow is not hampered by cable runs
			\cite{cisco07}.
			
			Many researchers have attempted to control cable installation costs by
			trying to reduce the number or length of cables required by developing
			alternative network topologies \cite{curtis12, popa10, mudigonda11}.
			Unfortunately, these techniques do not apply to SpiNNaker since its
			network topology is fixed.
			
			Some super computers make use of large custom `midplane` PCBs in place of
			cables to interconnect units within a cabinet and thus simplify the task
			of cable installation \cite{prickett10}. This scheme can greatly reduce
			wiring complexity since only coarser-grain cabinet-to-cabinet
			connectivity is provided by cables. Unfortunately this technique is
			expensive and also constrains the dimensions of the network topology
			supported by the machine. Since the SpiNNaker platform is designed to
			scale from desktop machines to machine-room installations, this scheme is
			not practical.
	
	\section{Folding \& interleaving hexagonal toruses}
		
		The first step towards a practical machine-room installation of a large
		machine using a hexagonal torus topology is to find an arrangement of
		boards between which cable lengths are minimised. In this section I
		describe a sequence of transformations which map the positions of units in
		a hexagonal torus topology onto a regular rectangular grid which may be
		folded and interleaved to eliminate long wires. It is worth emphasising
		that this transformation only affects the \emph{physical} positions of
		units and \emph{not} their connectivity.
		
		As described earlier in \S\ref{sec:parititioning} (page
		\pageref{sec:parititioning}), hexagonal torus topologies may be partitioned
		into units containing wrapped-triples of nodes. For example, in SpiNNaker,
		chips (nodes) are partitioned into circuit boards (units) containing 48
		chips. For completeness, this section describes the process of folding both
		systems whose units are individual nodes and those whose units are
		wrapped-triples.
		
		The transformation process is divided into two parts, each described
		separately in this section.
		
		\begin{description}
			
			\item[Parallelogram to rectangle] Units of the system are transformed
			from a parallelogram shape to a rectangular shape.
			
			\item[Uncrinkle] Units within the rectangle are moved such that they all
			lie on a regular (and fully packed) 2D grid.
			
		\end{description}
		
		\subsection{Parallelogram to rectangle}
			
			The hexagonal torus topology is most naturally drawn as a parallelogram
			as illustrated in figures \ref{fig:hex-to-plane-node-native} and
			\ref{fig:hex-to-plane-native}. Two transformations are presented which
			transform these arangements of units into a rectangular form: shearing
			and slicing.
			
			A \SI{30}{\degree} shear transformation distorts networks such that the X
			and Y axes become orthogonal leading to a rectangular arrangement of
			units as illustrated in figures \ref{fig:hex-to-plane-node-shear} and
			\ref{fig:hex-to-plane-shear}.
			
			The slice transformation slices the units protruding from the
			left-hand-side of the parallelogram and moves them into the matching gap
			on the opposite side of the parallelogram as illustrated in figures
			\ref{fig:hex-to-plane-node-slice} and \ref{fig:hex-to-plane-slice}.
			 
			While the shear transformation introduces some distortion causing cables
			in the Z dimension to become $\sqrt{2}\times$ longer it leaves the
			pattern of wrap-around connections remains unchanged. By contrast, the
			slice transformation does not elongate any cables but changes the pattern
			of wrap-around connections. The exact pattern wrap-around connections
			produced when slicing depends on the aspect ratio of the network as
			illustrated in \ref{fig:slicing-examples} and influences the choice of
			folding technique applied as described later.
			
			\begin{figure}
				\center
				\begin{subfigure}[b]{0.32\linewidth}
					\center
					\buildfig{figures/hex-to-plane-node-native.tex}
					
					\caption{$7 \times 7$ node torus}
					\label{fig:hex-to-plane-node-native}
				\end{subfigure}
				\begin{subfigure}[b]{0.32\linewidth}
					\center
					\buildfig{figures/hex-to-plane-node-shear.tex}
					
					\caption{Sheared}
					\label{fig:hex-to-plane-node-shear}
				\end{subfigure}
				\begin{subfigure}[b]{0.32\linewidth}
					\center
					\buildfig{figures/hex-to-plane-node-slice.tex}
					
					\caption{Sliced}
					\label{fig:hex-to-plane-node-slice}
				\end{subfigure}
				
				\caption{Transformations of hexagonal toruses of nodes into a
				rectangular form. Thin lines show wrap-around links. Pointy-topped
				hexagons represent individual nodes.}
				\label{fig:hex-to-plane-node}
			\end{figure}
			
			\begin{figure}
				
				\begin{subfigure}[b]{0.32\linewidth}
					\center
					\buildfig{figures/hex-to-plane-native.tex}
					
					\caption{$4 \times 4$ triad torus}
					\label{fig:hex-to-plane-native}
				\end{subfigure}
				\begin{subfigure}[b]{0.32\linewidth}
					\center
					\buildfig{figures/hex-to-plane-shear.tex}
					
					\caption{Sheared}
					\label{fig:hex-to-plane-shear}
				\end{subfigure}
				\begin{subfigure}[b]{0.32\linewidth}
					\center
					\buildfig{figures/hex-to-plane-slice.tex}
					
					\caption{Sliced}
					\label{fig:hex-to-plane-slice}
				\end{subfigure}
				
				\caption{Transformations of hexagonal toruses of wrapped triples into a
				rectangular form.  Thin lines show wrap-around links. Flat-topped
				hexagons represent a wrapped triple of nodes.}
				\label{fig:hex-to-plane}
			\end{figure}
			
			\begin{figure}
				\center
				\buildfig{figures/slicing-examples.tex}
				\caption{Patterns of wiring in sliced systems of various sizes.}
				\label{fig:slicing-examples}
			\end{figure}
			
		\subsection{Uncrinkling}
			
			Though the transformmation step yields rectangular arrangements of units,
			these arrangements do not fall onto a regular 2D grid, with the exception
			of the shear transform on individual nodes. Figure \ref{fig:uncrinkling}
			illustrates how the various arrangements of hexagons may be moved to
			`uncrinkle' the units into a regular grid.
			
			\begin{figure}
				\center
				\begin{subfigure}[b]{0.44\linewidth}
					\center
					\buildfig{figures/uncrinkling-node-sheared.tex}
					
					\caption{$7 \times 7$ nodes, sheared}
					\label{fig:uncrinkling-node-sheared}
				\end{subfigure}
				\begin{subfigure}[b]{0.44\linewidth}
					\center
					\buildfig{figures/uncrinkling-node-sliced.tex}
					
					\caption{$7 \times 7$ nodes, sliced}
					\label{fig:uncrinkling-node-sliced}
				\end{subfigure}
				
				\vspace{1cm}
				
				\begin{subfigure}[b]{0.44\linewidth}
					\center
					\buildfig{figures/uncrinkling-sheared.tex}
					
					\caption{$4 \times 4$ triples, sheared}
					\label{fig:uncrinkling-sheared}
				\end{subfigure}
				\begin{subfigure}[b]{0.44\linewidth}
					\center
					\buildfig{figures/uncrinkling-sliced.tex}
					
					\caption{$4 \times 4$ triples, sliced}
					\label{fig:uncrinkling-sliced}
				\end{subfigure}
				
				\vspace{1em}
				
				\caption{Mapping rectangular arrangements of units into a square grid.
				Thick lines show how layers of units are uncrinkled.  Annotations show
				how the relative positions of nodes and wrapped triples change after
				uncrinkling.}
				\label{fig:uncrinkling}
			\end{figure}
			
			In the figure, the numbered units enumerate the different positions on
			the crinkle and those labelled alphabetically are those that immediately
			surround them. From this we can observe that uncrinkling largely
			preserves spatial locality but some distortion is introduced, separating
			previously neighbouring units. For example, in figure
			\ref{fig:uncrinkling-sheared}, the units labelled `1' and `i' are
			neighbours before uncrinkling but are separated by a (Euclidean) distance
			of $\sqrt{5}$ afterwards. Note that the distortion introduced depends on
			what part of the crinkle is considered, for example `2' and `a' have
			distance 2 but are logically connected in the same way.
		
		\subsection{Folding and Interleaving}
			
			Once a regular grid of units has been formed, this may be folded in the
			conventional way, eliminating long cables crossing from left-to-right and
			top-to-bottom as illustrated in \ref{fig:folding-sheared}.
			
			Unfortunately, for sliced systems whose dimensions are not of the ratio
			$1:2$, the pattern of wrap-around cables may also include some cables
			which do not cross directly to the opposite side of the system (refer
			back to figure \ref{fig:slicing-examples}). As a result of these
			connections, folding does not successfully eliminate all long
			connections. An exception to this rule is sliced systems whose dimensions
			are in the ratio $1:1$ where folding twice along the Y axis may
			successfully eliminate all wrap-around connections as illustrated in
			\ref{fig:folding-sliced}.
			
			\begin{figure}
				\begin{subfigure}{\linewidth}
					\center
					\buildfig{figures/folding-sheared.tex}
					\caption{$N \times M$ sheared systems and $N \times 2N$ sliced systems}
					\label{fig:folding-sheared}
				\end{subfigure}
				
				\vspace{1em}
				
				\begin{subfigure}{\linewidth}
					\center
					\buildfig{figures/folding-sliced.tex}
					\caption{$N \times N$ sliced systems}
					\label{fig:folding-sliced}
				\end{subfigure}
				
				\caption{Schematic illustrating elimination of long wrap-around links
				during folding. In each example a single link has been highlighted to
				aid in following the process.}
				\label{fig:folding}
			\end{figure}
			
			Once folded, the 2D grid is straight-forwardly interleaved as illustrated
			previously in figure \ref{fig:ring-folding}. The interleaving process
			introduces some additional distortion to the layout of units and causes
			most connections to become twice as long. For sliced $1:1$ systems, the
			additional fold results in additional overhead during interleaving since
			four layers of the system are interleaved.
		
		\subsection{Mapping to Cabinets}
			
			In the final step of the process is to map the 2D grid of units into
			positions in machine room cabinets as illustrated in figure
			\ref{fig:million-core-machine}. As illustrated in figure
			\ref{fig:cabinetisation}, first the grid of units is partitioned into
			groups of columns, one per cabinet, then groups of rows one per frame per
			cabinet. The units in each group are then allocated to slots within a
			frame, interleaving the rows of the groups as shown.
			
			\begin{figure}
				\center
				\buildfig{figures/cabinet-units.tex}
				
				\caption{An illustration of the physical construction of a
				multi-cabinet SpiNNaker system. (Note: network cables \emph{not}
				installed.)}
				\label{fig:cabinet-units}
			\end{figure}
			
			\begin{figure}
				\center
				\buildfig{figures/cabinetisation.tex}
				
				\caption{Mapping from 2D space to cabinets, frames and boards.}
				\label{fig:cabinetisation}
			\end{figure}
		
	\section{Cable installation}
		
		Cable installation is performed by a team of (human) technicians who must
		ensure that all network cables are correctly installed. As illustrated in
		previously in figure \ref{fig:cabinet-units}, the density of SpiNNaker's
		units, combined with the nature of the hexagonal torus topology, poses a
		challenge. To address this challenge I propose a semi-automated approach to
		cable installation which exploits diagnostic facilities available in the
		majority of super computers in order to guide technicians through the
		cabling process, interactively guiding installation and maintenance.
		
		\subsection{Interactive technician guidance and validation}
			
			While automated systems for validating cabling correctness are
			commonplace, these systems are typically used only after cabling has been
			completed \cite{lakner07}. As with other large-scale machines, SpiNNaker
			includes a low-bandwidth system management bus which may be used to
			interrogate network hardware and control diagnostic LEDs prior to the
			installation of the main SpiNNaker network interconnect.  Using these
			facilities I have constructed a tool called SpiNNer which interactively
			guides a technician, or team of technicians, through the cable
			installation process, validating each connection in real-time.
			
			Diagnostic LEDs mounted on each SpiNNaker board (figure
			\ref{fig:interactive-wiring-guide-leds}) are used to indicate the
			endpoints of the cable currently being installed. Simultaneously a
			Text-To-Speech (TTS) system gives an audible indication of which cable
			type is to be used and location of each connection.  Additionally, a GUI
			via a computer display (figure \ref{fig:interactive-wiring-guide-gui}).
			The centre of the display shows a `big-picture' perspective of the
			locations of the boards to be connected. The detailed views on the left
			and right indicate which of the six sockets on each board the cables
			should connect.
			
			\begin{figure}
				\center
				\begin{subfigure}[b]{0.40\textwidth}
					\begin{tikzpicture}
						\node (leds) [inner sep=0]
						      {\includegraphics[width=\textwidth]{figures/leds.jpg}};
						% Point at left LED
						\draw [white, <-, line width=0.4em]
						      ([shift={(-0.0cm, -0.6cm)}]leds.center)
						      -- ++(225:1cm);
						% Point at right LED
						\draw [white, <-, line width=0.4em]
						      ([shift={(1.1cm, -1.1cm)}]leds.center)
						      -- ++(225:1cm);
					\end{tikzpicture}
					
					\caption{Diagnostic LEDs}
					\label{fig:interactive-wiring-guide-leds}
				\end{subfigure}
				~
				\begin{subfigure}[b]{0.546\textwidth}
					\begin{tikzpicture}[thin, black!20!white]
						\node (screen) [inner sep=0]
						      {\includegraphics[width=\textwidth]{figures/wiring_guide_screenshot.png}};
						\draw (screen.south west) rectangle (screen.north east);
					\end{tikzpicture}
					
					\caption{Interactive wiring guide GUI}
					\label{fig:interactive-wiring-guide-gui}
				\end{subfigure}
				
				\caption{The SpiNNer interactive wiring guide uses a GUI,
				text-to-speech and diagnostic LEDs to assist during cable
				installation.}
				\label{fig:interactive-wiring-guide}
			\end{figure}
			
			SpiNNer also validates the connectivity of the system in real-time by
			polling the diagnostic interfaces of the network hardware at the
			endpoints of the cable being installed to determine if they are correctly
			connected. If a miswiring occurs, this is immediately detected and
			announced via TTS enabling the technician to immediately correct the
			error. Once a cable has been installed correctly, the software
			automatically advances to the next cable meaning direct interaction with
			the software by the technician is minimal. In practice, it is rarely
			necessary to refer to the GUI.
		
			SpiNNer presents the cables in an order intended to maximise ease of
			installation. Cables are installed in three groups with intra-frame
			cables being installed first, followed by intra-cabinet cables and
			inter-cabinet cables. Within each group, the tightest cables are
			installed first resulting in slacker cables naturally being installed
			over the top of already installed cables. By grouping cables in this
			manner, multiple technicians may work independently on the wiring within
			individual frames and cabinets.
			
			SpiNNer may also be used to repair or replace cables in the system.
			During maintenance, obstructing cables may be blindly removed alongside
			any cable being replaced. At the conclusion of the process, the wiring
			guide may be used to interactively guide re-installation of all removed
			cables.
		
		\subsection{Cable selection}
			
			Controlling slack is critical to ensuring reliable and maintainable
			cabling installations. If cables are too tight, cables and connectors can
			become easily damaged and when too slack, the excess cable obstructs
			other cables and can easily become tangled and damaged \cite{cisco07}. It
			has been observed that when ready-made cables are employed technicians
			frequently over-estimate the cable lengths required preferring to use
			overly long cables for all connections \cite{mazaris97}. To solve this
			problem, the SpiNNer wiring guide software dictates the cable lengths to
			be used by an installer based the rule of (three-)thumbs according to
			Mazaris \cite{mazaris97}. This rule suggests that an ideal amount of
			slack is approximately that which can be wrapped around three fingers.
			Specifically, the shortest available cable is selected which ensures at
			least \SI{5}{\centi\meter} of slack.
			
			The SpiNNer tool allocates cables assuming all cables take a Euclidean
			straight-line path between the endpoints of the connection. The result is
			that wiring is not routed through dedicated cable management structures
			but is simply suspended by its connectors in front of the machine. As
			demonstrated later, this unconventional approach leads neither to cooling
			problems nor increased maintenance effort.
	
	\section{Results and Evaluation}
		
		This stuff has been used and works. In this section I'll go over the
		overheads introduced by the various transformations and
		folding/interleaving steps and show a wiring scheme for a large machine
		which uses only short cables. I'll then show how SpiNNer was used to
		install this wiring plan into a very large machine without foobaring the
		cooling and in very little time. I'll also report on difficulty of
		maintenance.
		
		\subsection{Cable length}
			
			The transformation from regular hexagonal torus to a folded and
			interleaved form introduces some overhead to the cable lengths required.
			Using figure \ref{fig:uncrinkling} (page \pageref{fig:uncrinkling}), it
			is possible to compute the exact overhead introduced when each type of
			transformation proposed.
			
			For example, to compute the mean overhead introduced by the slicing
			technique when applied to units composed of wrapped triples, consider
			figure \ref{fig:uncrinkling-sliced}. The uncrinkling pattern used to
			transform this topology is a repeating pattern of two units, a pair of
			which have been labelled $1$ and $2$ respectively. Unit $1$ is
			immediately surrounded by six units labelled $a$, $b$, $c$, $2$, $g$ and
			$h$. Similarly, unit $2$ is surrounded by units $1$, $c$, $d$, $e$, $f$
			and $g$. Before the transformation, the distances, $D$, to each of these
			units is $1$ but after the transformation is applied, this is not always
			the case. Additionally, folding and interleaving introduce additional
			overhead. In this example, if the system is folded into $f_x$ columns and
			$f_y$ rows, the distances between previously neighbouring units become:
			
			\begin{equation*}
				\begin{aligned}[c]
					D_{1\,\leftrightarrow{}\,a} &= \sqrt{f_x^2 + f_y^2} \\
					D_{1\,\leftrightarrow{}\,b} &= f_y \\
					D_{1\,\leftrightarrow{}\,c} &= \sqrt{f_x^2 + f_y^2} \\
					D_{1\,\leftrightarrow{}\,2} &= f_x \\
					D_{1\,\leftrightarrow{}\,g} &= f_y \\
					D_{1\,\leftrightarrow{}\,h} &= f_x
				\end{aligned}
				\hspace{2cm}
				\begin{aligned}[c]
					D_{2\,\leftrightarrow{}\,1} &= f_x \\
					D_{2\,\leftrightarrow{}\,c} &= f_y \\
					D_{2\,\leftrightarrow{}\,d} &= f_x \\
					D_{2\,\leftrightarrow{}\,e} &= \sqrt{f_x^2 + f_y^2} \\
					D_{2\,\leftrightarrow{}\,f} &= f_y \\
					D_{2\,\leftrightarrow{}\,g} &= \sqrt{f_x^2 + f_y^2}
				\end{aligned}
			\end{equation*}
			
			From these values, the mean and maximum connection distances after
			folding and interleaving may be computed. Table
			\ref{tab:transform-overhead} gives the mean and maximum connection
			distances for each of the four transformations described in this chapter.
			
			\begin{table}
				\begin{subtable}[b]{\linewidth}
					\center
					\begin{tabular}{l c c}
						\toprule
						& Shear & Slice \\
						\addlinespace
						Nodes &
							$\frac{f_x + f_y + \sqrt{f_x^2 + f_y^2}}{3}$ &
							$\frac{f_x + f_y + \sqrt{f_x^2 + f_y^2}}{3}$ \\
						\addlinespace
						Triples &
							$\frac{7f_x + 3\sqrt{f_x^2 + f_y^2} + \sqrt{(2f_x)^2 + f_y^2}}{9}$ &
							$\frac{f_x + f_y + \sqrt{f_x^2 + f_y^2}}{3}$ \\
						\bottomrule
					\end{tabular}
					
					\caption{Mean}
					\label{tab:transform-overhead-mean}
				\end{subtable}
				
				\vspace{1em}
				
				\begin{subtable}[b]{\linewidth}
					\center
					\begin{tabular}{l c c}
						\toprule
						& Shear & Slice \\
						\addlinespace
						Nodes &
							$\sqrt{f_x^2 + f_y^2}$ &
							$\sqrt{f_x^2 + f_y^2}$ \\
						\addlinespace
						Triples &
							$\sqrt{(2f_x)^2 + f_y^2}$ &
							$\sqrt{f_x^2 + f_y^2}$ \\
						\bottomrule
					\end{tabular}
					
					\caption{Maximum}
					\label{tab:transform-overhead-max}
				\end{subtable}
				
				\caption{Overheads introduced when transforming unit positions onto a
				regular grid.}
				\label{tab:transform-overhead}
			\end{table}
			
			From these results it is evident that shearing and slicing networks
			whose units are nodes result in identical mean and maximum overhead in
			cable length when folded similarly. Since sliced networks may require
			folding more than once along each axis the shearing approach is
			preferable in general.
			
			For networks constructed from units of wrapped triples, the slicing
			approach suffers the same mean and maximum overhead has networks of
			nodes, and less overhead than shearing for the same number of folds. For
			systems with an aspect ratio of $1:2$ (where both slicing and shearing
			require $f_x = f_y = 2$), the slicing transformation yields lower mean
			and maximum overhead than shearing. For all other aspect ratios (where
			slicing requires a greater degree of folding) the shearing technique
			produces lower overhead. The recommended transformations for a given
			machine are thus given in table \ref{tab:transform-recommended}.
			
			\begin{table}
				\center
				\begin{tabular}{lcc}
					\toprule
					                         & $1:2$  & Other \\
					\addlinespace
					\multirow{2}{*}{Nodes}   & Either & Shear\\
					                         & \footnotesize $\mu\approx2.28 \quad \vee\approx2.83$
					                         & \footnotesize $\mu\approx2.28 \quad \vee\approx2.83$\\
					\addlinespace
					\multirow{2}{*}{Triples} & Slice  & Shear\\
					                         & \footnotesize $\mu\approx2.28 \quad \vee\approx2.83$
					                         & \footnotesize $\mu\approx3.00 \quad \vee\approx4.47$\\
					\bottomrule
				\end{tabular}
				
				\caption{Recommended transformation and folding scheme for different
				system types. $\mu$ and $\vee$ give the mean and maximum wire
				distortion introduced, respectively.}
				\label{tab:transform-recommended}
			\end{table}
			
			\begin{figure}
				\center
				\buildfig{figures/million-core-machine.tex}
				
				\caption{Cabling plan for a \num{1036800} core SpiNNaker
				machine's \num{3600} cables.}
				\label{fig:million-core-machine}
			\end{figure}
			
			Following folding and mapping to physical locations, the cabling plans
			for large machines require no large gaps to be spanned.  The largest
			planned SpiNNaker machine, illustrated in figure
			\ref{fig:million-core-machine}, will be \SI{6}{\meter} wide but the
			largest gap any cable must span is \SI{66}{\centi\meter}. This distance
			is well within the \SI{1}{\meter} allowed by the hardware and cables.
			
		\subsection{Installation practicality}
			
			\begin{table}
				\center
				\begin{tabular}{lrr@{$\,$}l}
					\toprule
						System & Number of Cables & \multicolumn{2}{r}{Installation time} \\
					\midrule
						24 boards  & \num{72}   & \num{10} & \si{\minute}         \\
						1 cabinet  & \num{360}  & \num{4}  & \si{\hour}$^\dagger$ \\
						2 cabinets & \num{720}  & \num{2}  & \si{\hour}           \\
						5 cabinets & \num{1800} & ?        &                      \\
					\bottomrule
				\end{tabular}
				
				\caption{Installation times for various sizes of machine.
				$\dagger$~This machine was installed without real-time validation of
				connectivity.}
				\label{tab:install-time}
			\end{table}
			
			A number of SpiNNaker machines of various scales have been assembled
			using the techniques described in this chapter ranging from single frames
			of 24 boards to a half-scale 5 cabinet machine. Table
			\ref{tab:install-time} gives the reported installation times of each of
			these machines.
			
			The single cabinet machine's installation time is notably
			disproportionate to its size. When this system was assembled, real-time
			connection validation was not yet available. As a result, though cable
			installation was rapid correcting errors was extremely costly, requiring
			careful retracing of many installation steps.
			
			TODO: TALK ABOUT MULTI-PERSON-WIRING IN PRACTICE ON FIVE CABINET MACHINE.
			
			\begin{figure}
				
				\center
				\buildfig{figures/wire-length-histogram.tex}
				
				\caption{Histogram of connection distances in a ten-cabinet,
				one-million core SpiNNaker machine annotated with the suggested cable
				length.}
				\label{fig:wire-length-histogram}
				
			\end{figure}
			
			FIGURE \ref{fig:wire-length-histogram} SHOWS THE DISTRIBUTION OF CABLE
			LENGTHS REQUIRED. IN PRACTICE THE SLACK ALLOCATED PROVED ADEQUATE. AS
			SHOWN IN FIGURE \ref{fig:install-histogram}, THE MOST IMPORTANT FACTOR IS
			WHETHER LEAVING THE FRAME OR NOT. LEAVING THE FRAME TAKES THE LONGEST.
			
			\begin{figure}
				\builddata{data/build_connection_log.tex}
				\buildfig{figures/install-histogram.tex}
				
				\caption{Histogram of cable installation times}
				\label{fig:install-histogram}
			\end{figure}
			
			TODO: COMPARE DIRECTLY WITH INSTALL TIMES REPORTED IN LITERATURE.
		
		\subsection{Thermal Impact}
			
			TODO: SHOW HOW TEMPERATURE IS CHANGED
			
		\subsection{Maintenance}
			
			TOOD: QUANTIFY CABLE REMOVALS REQUIRED. EXPERIMENT: REMOVE/REPLACE RANDOM
			BOARDS AND MEASURE TIME TAKEN, CABLES REMOVED. COMPARE WITH STANDARD DATA
			CENTRE WIRING

	\chapter{Finding shortest path vectors in SpiNNaker's network}
	
	Once a SpiNNaker machine has been constructed as described in the previous
	chapter, its network forms a large hexagonal torus topology. To exploit this
	network routing algorithms must be used to generate routes for packets to
	follow between nodes. As well as ensuring that packets arrive at the correct
	destination, routing algorithms often attempt to produce routes which make
	efficient use of the network. This often involves attempting to reduce
	congestion by ensuring packets do not travel further through the network than
	absolutely necessary.
	
	Many popular routing algorithms for torus topologies, including all published
	algorithms designed for SpiNNaker's hexagonal torus topology
	\cite{davies12,navaridas14}, internally function by computing shortest path
	vectors and generating routes from them. Existing methods of calculating
	shortest path vectors in hexagonal torus topologies are unable to generate
	all possible shortest path vectors and, as a result, reduces the diversity of
	routes produced by routing algorithms, potentially worsening network
	contention.
	
	In this chapter I describe a novel technique for computing shortest path
	vectors in hexagonal torus topologies which yields \emph{all} possible
	shortest path vectors for any pair of nodes. Further, implementations of this
	new technique execute an order of magnitude faster than the existing
	alternatives.
	
	\section{Related work}
		
		TODO: INTRODUCE SECTION
		
		\begin{figure}
			\center
			
			\begin{subfigure}{\linewidth}
				\center
				\buildfig{figures/distance-map-mesh.tex}
				\caption{2D mesh topology}
				\label{fig:distance-map-mesh}
			\end{subfigure}
			
			\vspace{1em}
			
			\begin{subfigure}{\linewidth}
				\center
				\buildfig{figures/distance-map-torus.tex}
				\caption{2D torus topology}
				\label{fig:distance-map-torus}
			\end{subfigure}
			
			\vspace{1em}
			
			\begin{subfigure}{\linewidth}
				\center
				\buildfig{figures/distance-map-hex-mesh.tex}
				\caption{Hexagonal mesh topology}
				\label{fig:distance-map-hex-mesh}
			\end{subfigure}
			
			\vspace{1em}
			
			\begin{subfigure}{\linewidth}
				\center
				\buildfig{figures/distance-map-hex-torus.tex}
				\caption{Hexagonal torus topology}
				\label{fig:distance-map-hex-torus}
			\end{subfigure}
			
			\caption{Plots showing distance from various locations marked
			         {\color{red}$\times$}. Darker areas are further away. Contour
			         lines show equidistant points.}
			\label{fig:distance-map}
		\end{figure}
		
		\subsection{Mesh Networks}
			
			In a (non-hexagonal) mesh network topology, shortest path vectors are
			computed by taking the element-wise difference between the source and
			destination nodes' coordinates.
			
			\begin{figure}
				\center
				\buildfig{figures/mesh-topology-coordinates.tex}
				\caption{An example 2D mesh network with example shortest-path routes
				from `A' to `B' and `B' to `C'.}
				\label{fig:mesh-topology-coordinates}
			\end{figure}
			
			For example, figure \ref{fig:mesh-topology-coordinates} illustrates a 2D
			mesh topology. In this topology, the nodes labelled `A', `B' and `C' have
			position vectors $(1, 2)$, $(4, 5)$ and $(6, 1)$ respectively. The
			shortest path vector from node `A' to `B' is thus simply $(4, 5) - (1, 2)
			= (3, 3)$ and from `B' to `C' is $(6, 1) - (4, 5) = (2, -4)$.
			
			A route may be produced from a shortest path vector by advancing the
			number of hops specified for each dimension in the vector. For example
			any permutation of the hops X$^+\,$X$^+\,$X$^+\,$Y$^+\,$Y$^+\,$Y$^+$, an
			example of which is included in the figure. Likewise a route from `B' to
			`C' may be constructed from any permutation of
			X$^+\,$X$^+\,$Y$^-\,$Y$^-\,$Y$^-\,$Y$^-$.
			
			Many popular routing algorithms such as Dimension Order Routing (DOR),
			Right-Turn Only Routing (RTOR) and Longest Dimension First Routing (LDFR)
			\cite{dally04,davies12} directly follow the above procedure and just
			prescribe a specific permutation of hop order. For example, DOR produces
			routes with X hops first, Y hops second and so on.
			
			The length of routes produced from a shortest path vector have a number
			of hops proportional to the magnitude of the vector, thus a shortest path
			vector yields a route with the minimum number of hops. For a two
			dimensional vector $(a, b)$ the magnitude is given as:
			%
			\begin{equation}
				\| (a, b) \| = \lvert a \rvert + \lvert b \rvert
			\end{equation}
		
		\subsection{Torus Networks}
			
			Computing shortest path vectors in (non-hexagonal) torus topologies is
			also straight forward. As an example, lets find the shortest path vector
			from node `A' to `B' in the 2D torus topology shown in figure
			\ref{fig:torus-shortest-path-example}. First, both nodes are translated
			such that the source node, `A', is at the centre of the network (figure
			\ref{fig:torus-shortest-path-translate}). Note that this translation may
			result in the destination node `wrapping around' the network. After
			translation, the shortest path vector is computed as in a mesh topology.
			As illustrated in \ref{fig:torus-shortest-path-routed}, the computed
			shortest path vector may be used to produce routes between the two nodes
			in their original positions.
			
			\begin{figure}
				\center
				\begin{subfigure}{0.3\linewidth}
					\center
					\buildfig{figures/torus-shortest-path-example.tex}
					\caption{Original}
					\label{fig:torus-shortest-path-example}
				\end{subfigure}
				\begin{subfigure}{0.3\linewidth}
					\center
					\buildfig{figures/torus-shortest-path-translate.tex}
					\caption{Translated}
					\label{fig:torus-shortest-path-translate}
				\end{subfigure}
				\begin{subfigure}{0.3\linewidth}
					\center
					\buildfig{figures/torus-shortest-path-routed.tex}
					\caption{Routed}
					\label{fig:torus-shortest-path-routed}
				\end{subfigure}
				
				\caption{Finding shortest paths in a 2D torus topology.}
				\label{fig:torus-shortest-path}
			\end{figure}
			
			This process works because vectors from the centre (though not other
			locations) of a torus topology are identical to those in mesh topologies
			(see figures \ref{fig:distance-map-mesh} and
			\ref{fig:distance-map-torus}).
		
		\subsection{Hexagonal Mesh Networks}
			
			In hexagonal mesh topologies it is conventional to define three `axes' X,
			Y and Z as shown in figure \ref{fig:hex-mesh-topology-coordinates}
			\cite{patel15}. In this example, the three labelled nodes `A', `B' and
			`C' may be given position vectors such as $(1, 1, 0)$, $(3, 2, 0)$ and
			$(0, 0, -7)$ respectively. As in other mesh networks, a vector between
			two nodes is found by subtracting the nodes' vectors. For example, a
			vector from `A' to `B' is $(3, 2, 0) - (1, 1, 0) = (2, 1, 0)$. This
			vector can then be converted into a route in the same way as a mesh
			network by taking any permutation of the three hops  X$^+\,$X$^+\,$Y$^+$.
			
			\begin{figure}
				\center
				\buildfig{figures/hex-mesh-topology-coordinates.tex}
				\caption{An example hexagonal mesh network topology.}
				\label{fig:hex-mesh-topology-coordinates}
			\end{figure}
			
			As explained in detail in appendix \ref{app:minimal-hex-coordinates},
			there are an infinite number of vectors between any two points. For
			example, the vectors $(1, 0, -1)$ and $(3, 2, 1)$ also reach node `B'
			from `A' in the example. However, for a given pair of nodes, there is
			always a single, unique vector whose magnitude is minimal which is
			given by the function:
			%
			\begin{equation}
				\operatorname{minimiseVector}(x,y,z)
					= (x,y,z) - \operatorname{median}(x,y,z) \cdot (1,1,1)
			\end{equation}
			%
			An important side-effect of this function is that a minimised vector will
			always contain at least one zero element meaning that shortest path
			routes will use at most two of the three available dimensions.
			
			To aid the reader's intuition, figure \ref{fig:distance-map-hex-mesh}
			illustrates how distances grow in a hexagonal mesh topology.
		
		\subsection{Hexagonal Torus Networks}
			
			Unfortunately, unlike non-hexagonal torus topologies, the translation
			technique cannot be used to compute shortest path vectors. As illustrated
			in figures \ref{fig:distance-map-hex-mesh} and
			\ref{fig:distance-map-hex-torus}, shortest path vectors from the center
			of a hexagonal mesh network are not equivalent to those of a hexagonal
			torus network.
			
			Prior research into routing in SpiNNaker's network has been based on the
			INSEE \cite{navaridas09,ghasempour15} interconnect simulator. Internally
			INSEE tries a set of twelve candidate vectors and picks the shortest as
			the shortest path vector to use for routing.
			
			\begin{figure}
				\center
				\begin{subfigure}{0.45\linewidth}
					\center
					\buildfig{figures/insee-vector-candidates-no-wrap.tex}
					\caption{$(\Delta_\textrm{X}, \Delta_\textrm{Y}) = (5,3)$}
					\label{fig:insee-vector-candidates-no-wrap}
				\end{subfigure}
				\begin{subfigure}{0.45\linewidth}
					\center
					\buildfig{figures/insee-vector-candidates-wrap-x.tex}
					\caption{$(\Delta'_\textrm{X}, \Delta_\textrm{Y}) = (-3,3)$}
					\label{fig:insee-vector-candidates-wrap-x}
				\end{subfigure}
				
				\vspace{1em}
				
				\begin{subfigure}{0.45\linewidth}
					\center
					\buildfig{figures/insee-vector-candidates-wrap-y.tex}
					\caption{$(\Delta_\textrm{X}, \Delta'_\textrm{Y}) = (5,-5)$}
					\label{fig:insee-vector-candidates-wrap-y}
				\end{subfigure}
				\begin{subfigure}{0.45\linewidth}
					\center
					\buildfig{figures/insee-vector-candidates-wrap.tex}
					\caption{$(\Delta'_\textrm{X}, \Delta'_\textrm{Y}) = (-3,-5)$}
					\label{fig:insee-vector-candidates-wrap}
				\end{subfigure}
				
				\vspace{1em}
				
				% Key
				\begin{tikzpicture}[thick]
					\coordinate (last);
					
					% #1 colour
					% #2 label
					\newcommand{\colourkeyentry}[2]{
						\node [#1] [right=of last, fill, rectangle, minimum size=1em] (last) {};
						\node [right=0 of last] (last) {#2};
					}
					
					\colourkeyentry{cb3class0}{$(\textrm{X}, \textrm{Y}, 0)$}
					\colourkeyentry{cb3class1}{$(\textrm{X} - \textrm{Y}, 0, - \textrm{Y})$}
					\colourkeyentry{cb3class2}{$(0, \textrm{Y} - \textrm{X}, - \textrm{X})$}
					
				\end{tikzpicture}
				
				\caption{The twelve candidate shortest-path vectors considered by INSEE
				represented as dimension-order routes. $W=H=8$,
				$(\Delta_\textrm{X},\Delta_\textrm{Y}) = (5, 3)$ and
				$(\Delta'_\textrm{X},\Delta'_\textrm{Y}) = (-3, -5)$.}
				\label{fig:insee-vector-candidates}
			\end{figure}
			
			The twelve vectors considered are constructed as follows.
			
			First a shortest path vector from the source to target node are
			constructed as if the network was a 2D mesh yielding a vector
			$(\Delta_\textrm{X},\Delta_\textrm{Y})$. From this, another vector
			$(\Delta'_\textrm{X},\Delta'_\textrm{Y})$, is defined:
			%
			\begin{align}
				\Delta'_\textrm{X} &= \Delta_\textrm{X} - \operatorname{sign}(\Delta_\textrm{X})W
				\\
				\Delta'_\textrm{Y} &= \Delta_\textrm{Y} - \operatorname{sign}(\Delta_\textrm{Y})H
			\end{align}
			%
			Where $W$ and $H$ are the width and height of the network respectively
			(in nodes). This new vector yields routes from the source to destination
			node that in a torus topology that \emph{always} wrap around the `X' and
			`Y' dimensions.
			
			From the pair of vectors defined, four possible 2D vectors can be
			produced: $(\Delta_\textrm{X},\Delta_\textrm{Y})$,
			$(\Delta'_\textrm{X},\Delta_\textrm{Y})$,
			$(\Delta_\textrm{X},\Delta'_\textrm{Y})$ and
			$(\Delta'_\textrm{X},\Delta'_\textrm{Y})$. Further, each 2D vector may be
			converted into one of three 3D vectors where either X, Y or Z are zero
			for a total of twelve candidate vectors.  Figure
			\ref{fig:insee-vector-candidates} illustrates all twelve candidate
			vectors for an example pair of nodes.
			
			\begin{figure}
				\center
				\buildfig{figures/xyz-protocol-regions.tex}
				
				\caption{The four regions defined by the XYZ-protocol.}
				\label{fig:xyz-protocol-regions}
			\end{figure}
			
			A more efficient technique is proposed by Hoffmann and D\'es\'erable
			called the XYZ-Protocol \cite{hoffmann15,hoffmann11}. If the source and
			destination nodes are translated such that the source node lies at the
			center of the topolgoy, the destination will lie in one of four regions
			illustrated in figure \ref{fig:xyz-protocol-regions}.
			
			If the destination lies in regions 1 or 4, a route may be constructed as
			if in a hexagonal mesh topology.
			
			Alternatively, if the destination lies in regions 2 or 3, the algorithm
			tests whether taking a mesh-like route within the region or
			wrapping-around either the X or Y dimension yields the shorter vector.
			The shortest of these vectors is then chosen.
			
			TODO DESCRIBE SPIRAL ROUTES.
			
			TODO DESCRIBE RTOR AND LDFR.
		
	\section{Dimension order routing in hexagonal torus topologies}
		
		So, existing solutions have two problems: trying 12 options and picking one
		is a bit kludgey and there are actually more options than that.
		
		\subsection{Simpler minimal hexagonal torus vectors}
			
			If you redraw your route such that it is sourced from bottom left corner
			(which we'll now call (0, 0)), there are four possible ways this route
			could wrap.
			
			TODO: DESCRIBE WAYS OF WRAPPING
			
			For each of these wrappings, all the possible routes we can take are
			strictly limited in terms of the dimensions used since we're stuck in a
			corner.
			
			In each case, the function computing the minimal hex vector function
			simplifies to a much simpler operation.
			
			TODO: DESCRIBE MINIMUM VECTOR LENGTH FUNCTIONS FOR EACH CASE
			
			This gives us a cheap way to compute which of the four possible wrappings
			are shortest. Based on this we can pick one of at most two (is this
			easily provable?) vectors in some fair manner to reduce load. This vector
			can then be minimised in the usual way.
			
			This also leads to confirming a theoretical result giving the length of a
			shortest path in a hexagonal torus topology.
			
			TODO: DESCRIBE HOW TO GET LENGTH FUNCTION AND COMPARE WITH \cite{xiao04}
		
		\subsection{Generating spiralling routes}
			
			In non-hexagonal torus topologies the previous technique would reveal all
			possible shortest vectors (e.g. in cases where you can wrap more than one
			way). Unfortunately, due to the addition of a non-orthogonal axes,
			hexagonal toruses also have other cases when the width and height do not
			match.
			
			TODO: TORUS SPIRALLING EXAMPLE
			
			It is possible to calculate the maximum number of spirals thus:
			
			TODO: DESCRIBE HOW MAX NUMBER OF SPIRALS IS COMPUTED
			
			Given a number of spirals, the vector can be updated this (note that the
			change does not add a multiple of (1, 1, 1) but also does not result in
			the vector changing length and thus becoming non-minimal!).
			
			TODO: DESCRIBE TRANSFORMATION
			
			TODO: PROVE THAT MINIMALITY IS MAINTAINED
		
		\subsection{Proof of completeness}
		
			TODO: PROOF OF COMPLETENESS BY EXHAUSTIVE SEARCH
	
		\subsection{Conclusions}
			
			This approach is simpler, smaller, has sounder theoretical basis, and
			finds more routes than alternatives. This is good for load balancing and
			fault avoidance and also good for completeness.


	\chapter{Routing packets in large SpiNNaker machines}
	
	\label{sec:routing}
	
	So far, this thesis has focused on tackling the practical challenges
	resulting from SpiNNaker's hexagonal torus network topology. In this chapter,
	I adjust my focus towards the practical challenges resulting from SpiNNaker's
	large scale. Faults in large systems are inevitable and in the half-million
	core, \num{28800} chip SpiNNaker machine recently completed at the University
	of Manchester, around \SI{1}{\percent} of chips exhibited faults\footnote{Of
	the faulty chips discovered, the vast majority of faults are attributed to a
	currently unknown SDRAM failure}. These faults result in gaps and broken
	links in the network topology which routing algorithms must avoid in order to
	ensure correct system operation.
	
	In this chapter I tackle the problem of extending existing routing algorithms
	for SpiNNaker's network to enable them to route-around known, static faults.
	Though dynamic or transient faults may also occur, in this work such faults
	are ignored and other techniques, such as protocol-level fault tolerance, are
	relied on instead.
	
	Numerous heuristic-based fault-tolerant routing algorithms exist which target
	different network topologies and router architectures. Unfortunately as I
	will show, these algorithms are not portable and rely on or attempt to work
	around specific features of their target network architecture. In particular,
	existing work is dominated by the challenge of developing routing schemes
	which avoid or defuse network deadlocks. Due to SpiNNaker's unconventional
	use of timeout-based flow-control, it is not subject to the routing
	restrictions present in other architectures intended to cope with deadlocks.
	
	In this chapter I introduce a graph-search based post-processing step for
	non-fault-tolerant routing algorithms which guarantees routability in
	SpiNNaker systems without disconnected subregions. I also demonstrate that
	this technique introduces both negligible computational overhead to the
	routing algorithm runtime and resulting network performance.
	
	TODO: NOTE THE FAULT RATES ENCOUNTERED IN PRACTICE...
	
	\section{Related work}
		
		Existing work on routing in SpiNNaker's network has ignored the challenge
		of avoiding faults and instead focused on producing efficient multicast
		routes. As a result this section is broken into two halves. In the first
		half I survey the existing non-fault-tolerant approaches to routing used in
		SpiNNaker to-date. In the second I discuss the approaches to fault tolerant
		routing taken in other systems.
		
		\subsection{Multicast routing in SpiNNaker}
			
			Various fault-intolerant multicast routing algorithms exist for many
			networks and a number have been proposed and evaluated specifically in the
			context of SpiNNaker.
			
			In 2012, Davies \emph{et al.} evaluated the use of three common torus
			routing algorithms in SpiNNaker and found that simple oblivious routing is
			suitable for typical neural applications \cite{davies12}. The three
			routing techniques are:
			
			\begin{description}
				
				\item[Dimension Order Routing (DOR)] Packets are routed along each
				dimension (e.g. $X$, $Y$ and $Z$) in turn until no further hops are
				available in that direction.  The order in which the dimensions are
				traversed is fixed.
				
				\item[Right Turn Only Routing (RTOR)] As in DOR except the dimension
				order is chosen such that routes only contain right-turns.
				
				\item[Longest Dimension First Routing (LDFR)] As in DOR except the
				dimension order is chosen in descending order of number of hops in each
				dimension.
				
			\end{description}
			
			These unicast routing techniques are converted into a multicast routing
			algorithm by merging together the routes produced between the source node
			and each destination node as illustrated in figure
			\ref{fig:simple-routers}.
			
			\begin{figure}
				\center
				\begin{subfigure}{0.3\linewidth}
					\center
					\buildfig{figures/simple-routers-dor.tex}
					
					\caption{DOR}
					\label{fig:simple-routers-dor}
				\end{subfigure}
				\begin{subfigure}{0.3\linewidth}
					\center
					\buildfig{figures/simple-routers-rtor.tex}
					
					\caption{RTOR}
					\label{fig:simple-routers-dor}
				\end{subfigure}
				\begin{subfigure}{0.3\linewidth}
					\center
					\buildfig{figures/simple-routers-ldfr.tex}
					
					\caption{LDFR}
					\label{fig:simple-routers-dor}
				\end{subfigure}
				
				\caption{Example multicast routes produced by merging together unicast
				routes from a central source node to each destination node.}
				\label{fig:simple-routers}
			\end{figure}
			
			In 2014, Navaridas \emph{et al.} introduced two new algorithms, `Enhanced
			Shortest Path Routing' (ESPR) and `Neighbourhood Exploring Routing' (NER)
			which produce multicast routing trees with fewer total hops
			\cite{navaridas14}. In both algorithms, routes are generated sequentially
			for each of the destinations of a route using LDFR. Unlike LDFR, however,
			these algorithms search a limited area of the network for other,
			already-connected destination nodes to which LDFR routes may be
			constructed. If no suitable destination is found, a LDFR route is
			constructed to the source node. Figure \ref{fig:search-regions} illustrates
			the shape of the searched regions of each algorithm. ESPR searches the
			trapezoidal region between the source and destination nodes while NER
			searches a fixed radius out from the destination node.
			
			\begin{figure}
				\center
				\begin{subfigure}{0.45\linewidth}
					\center
					\buildfig{figures/search-regions-espr.tex}
					
					\caption{ESPR}
					\label{fig:search-regions-espr}
				\end{subfigure}
				\begin{subfigure}{0.45\linewidth}
					\center
					\buildfig{figures/search-regions-ner.tex}
					
					\caption{NER}
					\label{fig:search-regions-espr}
				\end{subfigure}
				
				\caption{The ESPR and NER algorithms attempt to connect the node marked
				`D' to the closest node in the shaded region which is connected to the
				source node, `S'. If no connected node is found in the shaded region, the
				LDFR route is taken to `S'. The dotted line indicates the route chosen
				from `D'.}
				\label{fig:search-regions}
			\end{figure}
			
			Unfortunately none of these routing algorithms make any allowance for the
			avoidance of network faults. As a result their utility in real-world
			systems is limited.
		
		\subsection{Fault-tolerant routing}
			
			Numerous fault-tolerant routing algorithms have been proposed for
			super-computer networks. These algorithms are largely constrained by the
			need to maintain deadlock freedom. Since SpiNNaker's routers employ a
			timeout based deadlock-breaking strategy, much of this effort is
			unnecessary in SpiNNaker. As described below, this frequently renders
			existing fault-tolerant routing algorithms unnecessarily complex and
			inflexible.
			
			Deadlocks occur in a network if a cyclic dependency exists on any limited
			resource in the network. For example, as illustrated in figure
			\ref{fig:ring-deadlock}, in a ring network a deadlock may form when every
			node is waiting on the next node to accept a packet before accepting new
			packets from the previous node.
			
			\begin{figure}
				\center
				\buildfig{figures/ring-deadlock.tex}
				
				\caption{A deadlock in a ring network where each node is waiting for
				the next to accept a packet before accepting any further packets.}
				\label{fig:ring-deadlock}
			\end{figure}
			
			To prevent deadlocks, combinations of router microarchitectural features
			and routing restrictions may be employed. For example, a simple
			deadlock-free routing algorithm for mesh and torus networks mandates the
			use of DOR \cite{dally93}. Packets travelling in a -ve direction along
			each axis take priority over those travelling in a +ve direction. Packets
			travelling along the Y axis take priority over those travelling along the
			X dimension. Given these rules it is possible to define a total ordering
			on all hops in the network. Figure \ref{fig:deadlock-free-dor}
			illustrates a $3\times3$ mesh network whose hops have been numbered
			according to the total ordering defined above.  Any `X-then-Y' DOR route
			through this network results in the use of hops labelled with strictly
			increasing numbers. As a result, no cyclic dependencies (and thus no
			deadlocks) may occur.
			
			\begin{figure}
				\center
				\buildfig{figures/deadlock-free-dor.tex}
			
				\caption{Deadlock-free routing of two example routes using DOR in a 2D
				mesh topology. The numbers of the hops taken by each route are given on
				the right.}
				\label{fig:deadlock-free-dor}
			\end{figure}
			
			Unfortunately, the routing restrictions imposed to ensure deadlock
			freedom can result in fault-intolerant routing. In the example above, if
			the node at the bottom-right corner of the figure was faulty, the dotted
			example route would be blocked as no alternative routes are allowed.
			
			In practice, the routing rules used may be more relaxed, for example
			requiring that any route whose length is equal to a DOR must exist to
			guarantee routability \cite{rodrigo09}.
			
			Alternative routing strategies take a hybrid approach whereby an
			efficient but fault-intollerant routing algorithm is used where possible
			and in the presence of faults a less efficient but more robust strategy
			is employed. For example, the Immucube network architecture employs three
			virtual networks which operate independently over the same physical links
			\cite{puente07}. Initially messages are routed using a high-performance
			but potentially-deadlockable routing scheme in the first virtual network.
			If a deadlock is occurs, the deadlocked packet is dropped into the second
			virtual network in which packets are routed using a less efficient but
			deadlock-free but fault-intolerant routing algorithm. Finally, upon
			encountering a fault, packets are dropped onto the third virtual network
			which forms a ring network routing packets to every node in the network.
			
			Releated approaches \cite{mejia06,boppana95} divide the network into
			regions in which different routing rules are enforced to ensure deadlock
			freedom and, when required, fault tolerance.
			
			TODO FIGURE?
			
			The BlueGene/L supercomputer \cite{adiga02} uses DOR for its torus
			network and implements fault-tolerance by sacrificing otherwise
			functioning `lamb' nodes to ensure no route passes through a known dead
			link \cite{ho04}. In figure \ref{fig:lamb-nodes} an example scenario is
			shown where a single dead node is present and all nodes in the same row
			or column as the dead node have been made into lamb nodes. The lamb nodes
			may not be used in an application except as a through-route for other
			traffic. This pattern of lamb nodes guarantees that all dimension-order
			routes between all pairs of non-lamb nodes are not obstructed by the
			faulty node. This approach trades use of higher performance routing
			logic for wasted resources. This type of approach is most appropriate
			when algorithmic routing is used and routing rules are inflexible.
			
			\begin{figure}
				\center
				\buildfig{figures/lamb-nodes.tex}
				
				\caption{`Lamb' nodes may be disabled to ensure DOR will never
				encounter a fault.}
				\label{fig:lamb-nodes}
			\end{figure}
			
			Other algorithms proposed for the BlueGene architecture attempt to avoid
			the need for lamb nodes by generating routes which reach their destination
			via a `proxy' node \cite{gomez04}. By appropriately selecting the location
			of such a proxy, the existing routing algorithm used by the system can be
			guaranteed to select a route free of faults.
			
			TODO: EXAMPLE OF PROXY ROUTING TO AVOID FAULT
			
			Finally, many algorithms in in the field are distributed and use only local
			information along with limited information from their peers to generate
			routes \cite{fick09b}. In SpiNNaker, route generation is conventionally
			carried out centrally since no special on-chip hardware facilities exist
			for route generation. Centralised route generation also enables the routing
			algorithm to consider all available routes. As a result, there is little
			incentive for the use of distributed routing algorithms on SpiNNaker since
			global system information could be compactly shared for one-off routing
			passes.
			
			Algorithms for other architectures such as IP networks tend to be poor fits
			for static, regular network topologies since they use expensive graph-based
			algorithms for route discovery which aren't necessary here. They also tend
			to heavily feature graph topology discovery etc. which aren't needed here.
			
			Work on fault-tolerance in data centre networks does exploit the regularity
			of the network topology in routing algorithms \cite{guo08,liao12}.
			Unfortunately, the approaches used are not general enough to be applied to
			mesh-like topologies such as the one in SpiNNaker.
			
			Outside the field of computer networks, routing algorithms used to route
			wires across the surfaces of chips are required to solve similar problems
			to fault-tolerant network routing problems in mesh networks. Like mesh
			networks, the routes are defined within a regular Manhattan geometry and
			congested areas, rather than faults must be avoided by the algorithms
			\cite{kahng11}.  Unfortunately, these algorithms are designed for
			occasional batch operation prior to the multi-month process of chip
			manufacturing and so runtimes of hours or days are commonplace
			\cite{nam08}. As such these algorithms would be inappropriate for use
			with applications such as SpiNNaker where users' applications tend to be
			short-lived and thus routing should not be allowed to dominate runtime.
	
	\section{Partial graph search repair}
		
		In this section I introduce a novel post-processing algorithm, Partial
		Graph Search (PGS) repair, for routes produced by non-fault-tolerant
		routing algorithms.
		
		PGS repair guarantees routability for networks with no disconnected
		subregions by using a graph search algorithm to route around faults in the
		original route.  General-purpose graph search algorithms such as Breadth
		First Search (BFS), Dijkstra's Algorithm and A* are guaranteed to find
		shortest-path routes between pairs of points in arbitrary graphs. Such
		algorithms are generally a poor choice in highly regular network topologies
		such as meshes and toruses due to their high computational cost. In PGS
		repair, graph searching is only used for \emph{part} of the routing
		problem: to repair gaps in routes generated by more efficient routing
		algorithms.
		
		Real world super computer architectures are designed to ensure that faults
		are isolated \cite{gara05,alverson12} and thus tend to only impact a
		localised region of the network. Since PGS repair is only needed to route
		around these isolated faults, the space searched by the graph search
		algorithm should be very small in practice resulting in only short
		runtimes. In addition since faults are rare in real-world systems, the
		graph search process will only rarely be invoked.
		
		The PGS repair post-processing technique starts with a route produced by a
		non-fault-tolerant routing algorithm such as ESPR or NER. If this route is
		not obstructed by a fault, the algorithm terminates immediately without
		modifying the route. If the route attempts to use a faulty link, the
		algorithm proceeds as follows.
		
		The routing tree produced by the underlying routing algorithm is broken
		into subtrees wherever it attempts to route through a broken link and
		each subtree is assigned a unique colour, as illustrated in figure
		\ref{fig:pgs-repair-colouring}. From each disconnected subtree's root
		node in turn, a graph search is performed to find a short, fault-free
		route to a subtree node of a different colour. The subtree is then
		attached to the tree discovered by the graph search and re-coloured to
		match the tree it is connected to.
		
		\begin{figure}
			\center
			\begin{subfigure}{0.32\linewidth}
				\hspace*{-1.5em}
				\buildfig{figures/pgs-repair-colouring.tex}
				
				\caption{}
				\label{fig:pgs-repair-colouring}
			\end{subfigure}
			\begin{subfigure}{0.32\linewidth}
				\hspace*{-1.5em}
				\buildfig{figures/pgs-repair-colouring-fix1.tex}
				
				\caption{}
				\label{fig:pgs-repair-colouring-fix1}
			\end{subfigure}
			\begin{subfigure}{0.32\linewidth}
				\hspace*{-1.5em}
				\buildfig{figures/pgs-repair-colouring-fix2.tex}
				
				\caption{}
				\label{fig:pgs-repair-colouring-fix2}
			\end{subfigure}
			
			\caption{PGS repair process example showing a disconnected multicast
			route from A to B, C, D, E and F. $\times$ indicates a broken link.}
			\label{fig:pgs-repair-colouring-steps}
		\end{figure}
		
		For example in figure \ref{fig:pgs-repair-colouring-fix1} a path from the
		root of the subtree containing nodes E and F is found which connects it to
		the subtree rooted at A. Similarly in figure
		\ref{fig:pgs-repair-colouring-fix2} a path is also found connecting the
		subtree containing nodes C and D back to the subtree rooted at node A.
		
		If the routing tree was broken into $N+1$ subtrees by faults there will be
		$N$ subtrees disconnected from the root node. Each of the $N$ iterations of
		the algorithm connect a disconnected subtree to another subtree reducing
		the number of subtrees by $1$ each time. After $N$ iterations, therefore,
		exactly $1$ subtree remains which connects every node in the original
		routing tree without traversing faulty links.
		
		TODO: EXPLAIN THE FIDDLINESS HERE TO ENSURE WE DON'T CREATE LOOPS.
		
	\section{Evaluation \& Results}
		
		The PGS repair technique, by design, is able to work around all possible
		fault patterns which don't completely disconnect parts of the network. This
		result this evaluation focuses on the impact on performance PGS repair
		imposes. The metrics of interest in this evaluation are:
		
		\begin{itemize}
			\item Algorithm runtime
			\item Network congestion
			\item Routing table utilisation
		\end{itemize}
		
		\subsection{Traffic Patterns}
			
			In this evaluation, two standard benchmark multicast traffic patterns are
			used which have been used in previous research into SpiNNaker's network:
			
			\begin{figure}
				\center
				\buildfig{figures/traffic-distribution-centroids.tex}
				
				\caption{An example 4-centroid distribution with four centroids. The
				$\times$ marks the location of the origin node. Lighter colours
				indicate greater likelihood of a connection.}
				\label{fig:traffic-distribution-centroids}
			\end{figure}
			
			\begin{description}
				
				\item[Uniform] Destinations are chosen with uniform probability
				anywhere in the machine.
				
				\item[$N$-Centroids] Destinations are clustered around one of $N$
				randomly chosen `centroids' as illustrated in figure
				\ref{fig:traffic-distribution-centroids}.
				
			\end{description}
			
			The uniform traffic pattern is widely used in networks research
			\cite{dally04,davies12} while the centroids model was developed
			specifically to reproduce the traffic patterns found in the neural
			applications SpiNNaker is designed for \cite{navaridas14}. In this work
			we consider 3 centroids.
		
		\subsection{Fault model}
			
			In addition two different fault models are used which are representative of
			the faults found in real SpiNNaker systems:
			
			\begin{figure}
				\center
				\begin{subfigure}{0.48\linewidth}
					\hspace*{-1.5cm}
					\buildfig{figures/fault-example-uniform.tex}
					
					\caption{Uniform}
					\label{fig:fault-example-uniform}
				\end{subfigure}
				\begin{subfigure}{0.48\linewidth}
					\hspace*{-1.5cm}
					\buildfig{figures/fault-example-hss.tex}
					
					\caption{HSS Link}
					\label{fig:fault-example-hss}
				\end{subfigure}
				
				\caption{The two link fault models considered.}
				\label{fig:fault-example}
			\end{figure}
			
			\begin{description}
				
				\item[Uniform] Links are selected and disabled at random (figure
				\ref{fig:fault-example-uniform}).
				
				\item[HSS Link] Groups of links corresponding with randomly selected
				single High-Speed Serial (HSS) link between SpiNNaker boards are disabled
				together (figure \ref{fig:fault-example-uniform}).
				
			\end{description}
			
			The uniform link failure model models isolated failures resulting from
			isolated manufacturing defects in individual links. The HSS Link failure
			model models faults arising from failing or disconnected board-to-board
			links which carry several chip-to-chip traffic flows via a single cable in
			SpiNNaker systems. Though SpiNNaker-specific, the later fault model is
			analogous to failure modes arising in other architectures where a single
			fault may render several links impassable in a single area.
			
			A range of failure rates are explored in this section. My measurements of
			current large-scale SpiNNaker installations the link failure rate is about
			\SI{0.03}{\percent} with failures due to both individual chip-to-chip links
			and board-to-board HSS links. Exact link failure statistics for commercial
			super computer installations are not widely available, however, published
			Mean-Time-Between-Failure (MTBF) statistics place an upper bound on link
			failure rates at a similar \SI{0.03}{\percent} in one-year-old BlueGene/Q
			systems \cite{chiu11}.
			
			Unfortunately presently undiagnosed problem with the SDRAM packaged with
			approximately \SI{1}{\percent} of SpiNNaker chips has rendered these chips
			unusable for most applications. The gaps in the network resulting from the
			loss of these chips currently dominate true link failures leaving just over
			\SI{1}{\percent} of links inoperable.
			
			Surprisingly, research into fault tolerant routing in super computers
			appears to focus on benchmarks with even higher fault rates ranging from
			\SI{3}{\percent} to as high as \SI{7}{\percent}
			\cite{ho04,gomez04,mejia06}.
			
			In this evaluation, fault rates ranging from \SI{0.01}{\percent} to
			\SI{5}{\percent} are considered to cover both realistic fault levels
			along with the more extreme cases considered in related work.
		
		\subsection{Base routing algorithm}
			
			Since the PGS repair process is routing algorithm agnostic all
			experiments use the NER algorithm which has been found to be appropriate
			for SpiNNaker applications \cite{navaridas14}.
		
		\subsection{Algorithm runtime}
			
			To assess the impact of the PGS repair process on routing algorithm
			runtime, the algorithm was used to process a large number of randomly
			generated routing problems and the runtime recorded.
			
			\num{10000} one-to-sixteen multicast routing problems were generated in a
			$256\times256$ hexagonal torus topology, the largest size possible for a
			SpiNNaker system. Other quantities of multicast destinations were also
			evaluated but are omitted for brevity since the pattern of results are
			similar to those outlined here.
			
			TODO: APPENDIX WITH OTHER RUNS?
			
			The NER and PGS repair algorithms were written in C and compiled with GCC
			4.8.3 with \verb|-O2| level optimisations and executed on a cluster of
			idle workstations with 3.10 GHz Intel Core-i5-2400 CPUs.
			
			\begin{figure}
				\center
				\buildrplot{figures/routing-runtimes.R}
				
				\caption{Mean runtime of routing and PGS repair overhead. PGS repair
				overhead is stacked above the routing runtime (i.e. bars do not
				overlap). Error bars indicate 95\% confidence interval. Note different
				Y-scale for HSS link and uniform fault models.}
				\label{fig:routing-runtimes}
			\end{figure}
			
			Figure \ref{fig:routing-runtimes} shows the average runtimes recorded for
			both the NER and PGS repair algorithms. In fault-free networks the
			PGS-repair post-processing step is not required and incurs no penalty
			while the runtime of the algorithm grows with the fault rate for both
			fault and traffic models.
			
			Notably the HSS fault model results in longer runtimes for the PGS repair
			process compared with an equivalent fault-density of uniform faults.
			Because the HSS fault model produces contiguous lines of faults the PGS
			repair algorithm must construct a longer path to avoid the fault.  Since
			the space explored by a graph algorithm typically grows with $O(H^2)$
			with respect to the hops in the discovered route, $H$, this increase in
			search distance has a large impact on the runtime of the PGS repair
			process.
			
			The runtime of the PGS repair algorithm remains roughly in proportion to
			the runtime of the underlying routing algorithm with respect to different
			traffic models. The centroid traffic pattern tends to result in routes
			with fewer hops than a uniform traffic pattern with the same number of
			destination nodes as segments of routes are often shared between
			destination nodes. Since the NER algorithm's runtime is strongly related
			to the number of hops in the output route the runtime of the algorithm is
			greater for uniform traffic. Likewise the probability of PGS repair being
			required increases with the number of hops in route and hence the runtime
			of the PGS repair algorithm increases roughly in proportion.
		
		\subsection{Routing table usage}
			
			In order to gain a realistic measure of routing table usage it is
			necessary to determine the effect of many routes being generated for a
			single set of faults. To enable a sufficiently large number of sample to
			be collected the experimental setup considered previously is reduced to a
			network containing $48\times48$ nodes.
			
			\num{1000} $48\times48$ node network models are produced according to the
			HSS link and uniform fault models. For each of these models
			$48\times48\times16=$~\num{36864} one-to-sixteen routes are generated using
			the centroid and uniform traffic models. This corresponds to one
			multicast route per application core. As is convention in SpiNNaker,
			routing table entries are inserted for each route at the source of the
			route, at each destination and at each corner or fork. The number of
			routing table entries at each node in the model is counted and the
			maximum number of entries in a single node is reported for each network
			model.  The \emph{maximum} number of routing entries of any router was
			chosen since the number of entries available per SpiNNaker router is
			bounded by hardware.
			
			\begin{figure}
				\center
				\buildrplot{figures/routing-entries.R}
				
				\caption{Violin plot showing the distribution of maximum table sizes
				for \num{1000} random networks. The red line at \num{1024} entries
				indicates the size of SpiNNaker's routing tables.}
				\label{fig:routing-entries}
			\end{figure}
			
			
			Figure \ref{fig:routing-entries} shows the distributions of the largest
			routing table sizes for each fault and traffic model.
			
			\begin{figure}
				\center
				\begin{subfigure}{0.48\linewidth}
					\center
					\buildfig{figures/hss-link-routing-table-usage.tex}
					
					\caption{Routing table entries}
					\label{fig:hss-link-routing-table-usage}
				\end{subfigure}
				\begin{subfigure}{0.48\linewidth}
					\center
					\buildfig{figures/hss-link-resource-usage.tex}
					
					\caption{Routes passing through chip}
					\label{fig:hss-link-resource-usage}
				\end{subfigure}
				
				\caption{The impact of a HSS link fault on routing table usage and
				congestion. Each hexagon represents a single chip, the red line
				indicates the chip-to-chip connections broken by the HSS link fault.}
				\label{fig:hss-link-usage}
			\end{figure}
			
			The HSS link failure model has a much greater impact on peak routing
			table resource usage than uniform link failures for a given fault rate.
			This is because HSS link faults result in a large concentration of routes
			being disrupted and then re-routed around the same obstacle in a single
			location. Figure \ref{fig:hss-link-routing-table-usage} shows how routing
			table usage varies around a HSS link fault in one instance of the
			experiment. There are clear peaks in routing table usage around the ends
			of the line of faults which result from routes produced by PGS repair
			finding shortest paths around the edge of the faults.
		
		\subsection{Network congestion}
			
			To measure the impact of PGS repair on network congestion, two
			experiments were performed, one using the same model used to measure
			routing table usage and one based on tests run on SpiNNaker hardware.
			
			For each of the network fault and traffic pattern described previously,
			the paths taken for the \num{36864} one-to-sixteen multicast routes
			generated are used to compute the number of times each link in the
			network is used. The number of routes passing through the most-used link
			is then recorded, giving an indication of the level of congestion in the
			network.
			
			\begin{figure}
				\center
				\buildrplot{figures/routing-resource.R}
				
				\caption{Violin plot showing the distribution of maximum
				routes-per-chip for \num{1000} random networks.}
				\label{fig:routing-resource}
			\end{figure}
			
			The results are presented in figure \ref{fig:routing-resource} and follow
			the same trends as the results previously shown for routing table usage.
			Again, HSS link faults result in routes with the greatest congestion due
			to the concentration of routes finding shortest paths around an obstacle
			(see \ref{fig:hss-link-resource-usage}).
			
			To verify that the results above, an additional experiment has been
			carried out which attempts to mimic the model used previously in actual
			SpiNNaker hardware. In these experiments a large SpiNNaker machine is
			divided into independent 48-board (2304-chip) sections. Because the
			48-board systems used in these experiments are cut out of a larger
			machine, they lack wrap-around links and thus form hexagonal mesh
			topologies, rather than hexagonal toruses.
			
			Due to the SDRAM issue described above, fault rates below
			\SI{1}{\percent} cannot be modelled.  To simulate higher fault rates,
			additional links are disabled in software according to the fault models
			described used previously. Since some faults are due to genuine hardware
			faults, these faults cannot be placed randomly in each experiment. To
			reduce, bias each combination of fault rate, fault model and traffic
			pattern is repeated XXX times across randomly chosen physical machines.
			
			XXX 1-to-XXX routes are generated in both uniform and XXX-centroid
			distributions as used throughout this evaluation. Synthetic network
			traffic is generated at the source of each route following a Bernoulli
			distribution. Traffic consumers running on all destination nodes accept
			packets as quickly as possible from the network and log their arrival.
			The Bernoulli probability is set the same for every route's traffic
			generator and increased in steps of XXX and the number of packets dropped
			in an XXX second period logged. The network is considered saturated once
			less than \SI{99}{\percent} of packets successfully arrive at their
			destination.
			
			Figure \ref{XXX} shows the distributions of the saturation points for
			each experimental configuration.
			
			TODO: ANALYSIS
		
	\section{Conclusions}
		
		In this chapter I described how SpiNNaker's unconventional network and
		router architecture render existing fault tolerant routing algorithms
		unsuitable. I introduced PGS repair, a post-processing technique for
		existing non-fault tolerant routing algorithms designed for SpiNNaker such
		as NER.
		
		Unlike some other fault tolerant routing algorithms for other
		architectures, PGS repair is able to work-around arbitrary fault patterns
		by exploiting SpiNNaker's inbuilt deadlock avoidance mechanisms. In the
		presence of realistic failure rates of up to \SI{1}{\percent}, only small
		overheads of up to XXX, XXX and XXX for in algorithm runtime, routing table
		usage and network performance are incurred respectively. This low
		performance overhead makes PGS repair appropriate for use in real
		applications. At the time of writing the algorithm has been successfully
		used in a number of neural and non-neural SpiNNaker applications.
		
		At more extreme fault rates not expected in real-world systems, the
		algorithm still functions correctly but the results incur much greater
		routing table and congestion overheads, particularly when faults are
		concentrated. Future extensions to this algorithm might aim to reduce this
		overhead by producing longer and more varied routes around faults to even
		out the load.

	\chapter{Placing applications in large SpiNNaker machines}
	
	In the previous chapter I tackled the problem of scale in generating routes
	for very large networks such as SpiNNaker. In this work the centroid traffic
	pattern was used as an approximation of the expected network traffic
	generated by `well behaved' neural network simulation software running on
	SpiNNaker. The traffic produced largely exhibits strong locality, that is
	most communication occurs between either nearby nodes or clusters of nodes.
	In reality, neural simulation applications are not specified geometrically
	but rather as abstract graphs of communicating neurons
	\cite{davison08,eliasmith13}. Applications must then \emph{place} these
	neurons onto nodes in a SpiNNaker system, attempting maximise communication
	locality.
	
	In this chapter I re-evaluate the suitability of simulated annealing as a
	technique for finding high quality placements for large parallel
	applications. Though this technique had fallen out of fashion in the field of
	application placement by the early 1990s, it has found wide use for placing
	components in computer chip and FPGA designs. In the intervening years,
	placement problems in super computers have grown in size from tens or
	hundreds of nodes to millions, a scale at which chip placement techniques
	were operating in the mid 1990s. I adapt the simulated annealing algorithm
	used by the VPR academic circuit placement software to produce placements for
	applications running on SpiNNaker. In that in a range of real and synthetic
	benchmarks simulated annealing produces high quality placements enabling
	efficient use of SpiNNaker's network resources.
	
	
	%In the field of chip design, Moore's `Law' \cite{moore65,moore75} observes a
	%similar exponential growth in the number of components within a single chip.
	%Today modern processors contain billions of components and an analagous
	%placement problem exists in attempting to place interconnected components
	%near to eachother. In this chapter I explore the techniques used for circuit
	%placement and adapt one such technique, Simulated Annealing (SA)
	%\cite{kirkpatrick83}, for use in application placement. Despite some early
	%interest in SA for application placement in the 1980s and early 1990s, the
	%technique has since fallen out of favour. I find that at the scales of modern
	%placement problems SA-based placement is able to produce solutions of
	%superiour quality to contemporary methods.
	%
	%TODO: SUMMARISE RESULTS...
	
	\section{Related work}
		
		The placement problem has been tackled independently in the literature by
		researchers in both the application and chip placement communities. In this
		survey I cover application and chip placement separately as these two
		communities have remained largely isolated from one another. First I
		explore the techniques applied to application placement before moving on to
		contrast this with the techniques used in circuit placement.
		
		In the application placement literature, the placement problem is often
		referred under the umbrella term `mapping'. Unfortunately term is often
		used more broadly to include other tasks such as routing and application
		partitioning. To avoid ambiguity I use the term `placement', as preferred
		by the chip and FPGA design communities, to refer specifically to the
		problem of assigning nodes in an application's communication graph to nodes
		in a machine's connectivity graph.
		
		\subsection{Application placement algorithms}
			
			TODO: GENERAL INTRO
			
			\subsubsection{Application-specific approaches (manual placement)}
				
				In the case of some applications such as finite element modelling
				\cite{bermejo13}, the structure of the problem itself leads to a
				natural placement of the computation on nodes in a machine. For example
				when simulating a 3D volume in an node super computer with a $3 \times
				4 \times 2$ 3D torus or mesh topology network, the modelled volume
				might be divided into as in figure \ref{fig:fem-partitioning}. Each
				cuboid in the model is then assigned to the corresponding node in the
				network topology.
				
				\begin{figure}
					\center
					\buildfig{figures/fem-partitioning.tex}
					
					\caption{Example partitioning of a 3D space to fit into a super
					computer with a $3\times4\times2$ torus or mesh topology.}
					\label{fig:fem-partitioning}
				\end{figure}
				
				When the number of dimensions in a problem do not match that of the
				underlying network architecture, the common solution is to either
				divide only along a subset of the axes or to divide into additional
				pieces on the existing axes \cite{gilge14}.
			
			\subsubsection{Sequential placement}
				
				In the case where a placement solution is non-obvious one of the
				simplest and most popular strategies is to apply a simple sequential
				placement algorithm. Sequential placement algorithms function by
				iterating over the vertices in the application's communication graph
				and assigning them to a free node in the target machine. Sequential
				placement algorithms are differentiated by the order in which they
				iterate over vertices in the communication graph and fill nodes in the
				target machine. A number of widely used orderings are described below.
				
				\begin{figure}
					\center
					\begin{subfigure}{0.32\linewidth}
						\center
						\buildfig{figures/sequential-row-order.tex}
						\caption{Row-order}
						\label{fig:sequential-row-order}
					\end{subfigure}
					\begin{subfigure}{0.32\linewidth}
						\center
						\buildfig{figures/sequential-alternating.tex}
						\caption{Alternating}
						\label{fig:sequential-alternating}
					\end{subfigure}
					\begin{subfigure}{0.32\linewidth}
						\center
						\buildfig{figures/sequential-hilbert.tex}
						\caption{Hilbert curve}
						\label{fig:sequential-hilbert}
					\end{subfigure}
					
					\caption{Space-filling curves in 2D mesh and torus topologies.}
					\label{fig:sequential}
				\end{figure}
				
				Super computer management software such as SLURM \cite{yoo03} and Blue
				Gene's system software \cite{gilge14} by default na\"ively iterate over
				vertices in an application communication graph in the order they are
				provided. The nodes in the target machine are then iterated over in a
				simple space-filling curve through the network topology. Figure
				\ref{fig:hilbert-placement} illustrates the default patterns available
				in these software packages. The row-order (figure
				\ref{fig:sequential-row-order}) and alternating (figure
				\ref{fig:sequential-alternating}) curves correspond with 2D versions of
				the default node assignment orders used in SLURM and BlueGene systems.
				
				\begin{figure}
					\center
					\buildfig{figures/hilbert-placement.tex}
					
					\caption{A Hilbert curve, coloured from blue to red.}
					\label{fig:hilbert-placement}
				\end{figure}
				
				The Cray extensions to SLURM software provide a Hilbert curve
				\cite{hilbert91} (figure \ref{fig:sequential-hilbert}) node assignment
				order. Unlike the row-order and alternating space filling curves the
				Hilbert curve ensures that pairs of vertices close together in the node
				iteration order are also close together in the target machine's network
				\cite{moon01, zumbusch99}. Figure \ref{fig:hilbert-placement} shows a
				5$^\textrm{th}$-order Hilbert curve where each point in the curve is
				coloured according to its position along the curve. In this figure it
				is possible to see that nearby positions in the curve (which share
				similar colours) are also close in 2D space.
				
				When the proximity of vertices in the vertex-ordering supplied by an
				application is a good estimator of those vertices communication
				requirements, the sequential assignment schemes discussed above can be
				very effective. These techniques have also proven adequate in
				small-scale and densely connected applications such as early neural
				simulations running on prototype SpiNNaker machines with tens of nodes
				\cite{galluppi10} but growing beyond this scale has proven problematic.
				
				\begin{figure}
					\center
					\begin{subfigure}{0.45\linewidth}
						\center
						\buildfig{figures/rcm-initial.tex}
						
						\caption{Original permutation}
						\label{fig:rcm-initial}
					\end{subfigure}
					\begin{subfigure}{0.45\linewidth}
						\center
						\buildfig{figures/rcm-sorted.tex}
						
						\caption{RCM permutation}
						\label{fig:rcm-sorted}
					\end{subfigure}
					
					\caption{Adjacency matrix representation of a graph before and after
					permutation by the RCM algorithm.}
					\label{fig:rcm}
				\end{figure}
				
				A number of algorithms have been proposed for automatically selecting
				good vertex iteration orders, typically using a graph-traversal based
				heuristic. A typical method, described by Hoefler \emph{et al.}
				\cite{hoefler11} exploits the Reverse-Cuthill-McKee (RCM) algorithm
				\cite{cuthill69}. An application's communication matrix is represented
				as an adjacency matrix, $M$, where $M_{i,j}$ is 1 if node $i$ is
				connected by an edge to node $j$ and 0 otherwise. An example matrix is
				illustrated in figure \ref{fig:rcm-initial}. The RCM algorithm uses a
				simple heuristic to permute the matrix (i.e. renumber the nodes in the
				graph) in order to reduce the bandwidth of the matrix. Figure
				\ref{fig:rcm-sorted} shows the RCM-permuted version of the example
				adjacency matrix. When a graph's vertices are ordered as in a
				bandwidth-reduced sparse matrix, vertices close together in the
				ordering are likely to communicate while those further apart tend not
				to communicate.
				
			\subsubsection{Optimisation-based Placement}
				
				% Citations from short report about optimisation in placement...
				% \cite{chen06,jeannot14} and \cite{jeannot10} ("subsets of apps")
				
				In the academic community, a number of attempts have been made to use
				more sophisticated optimisation algorithms for the placement of
				applications. In 1985, Steele \cite{steele85} proposed the use of
				simulated annealing for placing applications in the 6D torus topology
				of the 64 node `Caltech Cosmic Cube' machine. Simulated annealing,
				originally developed by Kirkpatrick \emph{et al.} \cite{kirkpatrick83},
				is a general-purpose optimisation algorithm which works by analogy to
				the physical process of annealing. In brief simulated annealing
				functions by randomly swapping vertices in a candidate placement
				solution, accepting swaps which move connected vertices closer together
				and rejecting some proportion of swaps which move connected vertices
				further apart. The simulated annealing algorithm is described in detail
				later in this chapter.
				
				Towards the end of the 1980s, application placement appeared to be
				becoming less important as super computer network architectures
				improved:
				%
				\begin{displayquote}
					``Careful placement was necessary because of the slow communication
					and non-uniform addressing of early concurrent computers. However,
					the development of message passing machines with fast communications
					and a uniform global address space  has made placement less of an
					issue. In such machines a random placement performs nearly as well as
					an optimum placement.''
					
					\noindent --- W. Dally, 1987 \cite{dally87}
				\end{displayquote}
				%
				In addition, network and problem sizes remained small, so small in fact
				that linear-programming based optimal placement still appeared in
				benchmarks comparing placement algorithms \cite{xu91}. In this
				environment, simpler sequential placement algorithms gained favour over
				more computationally expensive algorithms such as simulated annealing.
				
				As problem and machine sizes have grown and network utilisation has
				once again become an important factor in application performance
				\cite{navaridas09b} more complex optimisation algorithms have
				reappeared in the literature. One popular approach employs graph
				partitioning algorithms such as METIS \cite{karypis98} to perform
				recursive bipartitioning based placement
				\cite{phillips14,hoefler11,pellegrini96}.  This placement process is
				illustrated in figure \ref{fig:partitioning}.
				
				In the first step, the application communication graph and machine
				connectivity graph are bipartitioned such that the number of edges
				between partitions is minimised. Each half of the communication graph
				is associated with one of the halves of the machine connectivity graph.
				The partitioning process is then repeated recursively on each of the
				two communication and connectivity graph pairs. The process halts when
				the graphs can no longer be partitioned at which point the vertices in
				the communication graph are placed on their associated node.
				
				\begin{figure}
					\center
					\buildfig{figures/partitioning.tex}
					
					\caption{Illustration of application placement by recursive
					partitioning.}
					\label{fig:partitioning}
				\end{figure}
				
				TODO: PARTITIONING IS GREAT AND ALL BUT QUALITY ISN'T ALWAYS GREAT AND
				IT DOESN'T DEAL WELL WITH MULTI-CONSTRAINT SCENARIOS E.G. PROCESSOR AND
				MEMORY RESTRICTIONS.
				
				Unfortunately, many of these simply aren't suited to the scale of
				neural applications running on SpiNNaker (e.g. only cope with tens of
				nodes while SpiNNaker may contain hundreds of thousands).
				
				Additionally, a number of algorithms have been developed which make
				assumptions about the topologies of the problem or network. Tree match
				for example attempts to map tree-shaped problems to tree-shaped
				networks. Such algorithms can be highly effective but again do not
				apply to SpiNNaker or its neural applications.
		
		\subsection{Chip placement algorithms}
			
			The chip-design industry has, for many years, dealt with problems
			analogous to the task of placing super computer jobs in a way suited to
			SpiNNaker. Modern CPUs have millions or billions of components with
			strictly fixed connectivity. CPU designers must place each of these onto
			a chip such that the connection lengths are controlled to reduce
			congestion and increase performance. As such, these algorithms are
			ideally suited to future super computer placement work since they already
			operate at the scales required \cite{nam07}.
			
			\subsubsection{Cost functions}
				
				HPWL is popular but a bit crap for high fan-outs. It is, however, quite
				simple.
				
				TODO: SELECT A BETTER COST FUNCTION...
			
			\subsubsection{Simulated annealing}
				
				One of the oldest techniques used for circuit placement is simulated
				annealing and this remains popular today thanks to its sheer
				versatility (see VPR, other open FPGA tools).
				
				SA works by analogy with the physical process of annealing.
				The simulated annealing algorithm works by selecting random pairs of
				components on a chip, swapping them and evaluating some cost function.
				If the swap reduces the cost function, it is kept, if not, depending on
				a function of the current temperature and the cost introduced by the
				swap.
				
				TODO: ILLUSTRATION OF SIMULATED ANNEALING SWAP OPERATION
				
				By occasionally allowing costly swaps, the annealing algorithm avoids
				becoming trapped in local minima. As the algorithm proceeds, the
				temperature is slowly reduced and with it the proportion of costly
				swaps which are retained. This causes the placement to move from
				exploration early on towards refinement later on.
				
				The temperature schedule of an annealing algorithm is critical to its
				success. In general these schedules are computed based on the
				performance of the algorithm as it runs. In VPR the following schedule
				is used.
				
				TODO: DESCRIBE VPR'S SCHEDULE
				
				TODO: FIND AND DESCRIBE ALTERNATIVE SCHEDULE?
				
				Unfortunately, SA is very difficult to parallelise, especially in the
				case of placement. As a result, its scalability has been limited and
				resulted in significantly reduced usage in recent work.
			
			\subsubsection{Partitioning placement}
				
				Partitioning based placement solves the placement problem using
				graph-partitioning recursively on the problem graph to assign each part
				of the circuit to some area in the super chip. Though a number of
				algorithms have proven successful in academic placement contests over
				the years, they are not popular in industrial settings.
			
			\subsubsection{Analytical placement}
				
				In analytical placement, cost function for the circuit graph is
				approximated in a form which is amenable to solutions with standard
				numerical or symbolic algebraic techniques. Using these techniques,
				exact minimum cost (in terms of the approximation) configurations can
				be obtained.
				
				Quadratic placement is a popular analytical placement technique which
				approximates the cost of a placement as the sum of the squares of the
				distances between connected circuit elements.
				
				TODO: FIGURE EXAMPLE QUADRATIC PLACEMENT PROBLEM AND SOLUTION
				
				As such this gives a quadratic cost function like so which we must
				minimise.
				
				TODO: QUADRATIC COST EQN
				
				To minimise the function we differentiate and solve using simple
				symbolic manipulation.
				
				TODO: QUADRATIC COST SOLUTION
				
				Unfortunately, quadratic placement doesn't contain any congestion
				relief by default so various schemes exist. For example, extra anchor
				nodes are inserted which gently pull the circuit components apart from
				each other. As a result, the algorithm generally proceeds by iterating,
				regenerating anchors each time.
				
				Other non-quadratic analytical methods exist too with numerical
				solutions. The approaches are often similar.
			
			\subsubsection{Hierarchical clustering}
				
				Many placement algorithms scale super-linearly with problem size and so
				larger problems become increasingly problematic to handle. To solve
				this problem clustering techniques are first applied to first simplify
				the placement problem. A solution is then found at the coarse level and
				then hierarchically fleshed out.
				
				Various clustering algorithms are in use.
				
				TODO: TALK ABOUT CLUSTERING IN PLACEMENT...
				
				TODO: DESCRIBE THE ALGORITHM I IMPLEMENTED.
	
	\section{Application placement by simulated annealing}
		
		\label{sec:placement-by-annealing}	
		
		I have implemented a simplified SA based application placement algorithm
		based on the approach used in the popular VPR place and route tool chain.
		The algorithm is written in C and is optimised for experimentation rather
		than performance but is production-ready. It has been integrated into the
		`Rig' SpiNNaker software tools and has been used to place very large
		simulations. More on that later.
		
		\subsection{Representation}
			
			Model each chip as having a quantity of various resources (e.g. Cores,
			SDRAM) available. The application graph consists of vertices which each
			consume some quantity of these resources. Vertices must be placed on a
			single chip such that the resources required on a given chip do not
			exceed those available. Vertices are then interconnected by 1:N nets with
			weights which act as hints. The nets are treated as a soft constraint:
			vertices connected via a net will, ideally, be placed near to each other,
			with priority being given to nets with higher weights. Additionally there
			will be a list of placement constraints (see later).
			
			A key observation is that while vertices in an application may frequently
			have a 1:1 correspondence with application cores, this need-not be the
			case. For example, a vertex may represent a block of SDRAM which is
			shared. A vertex may also represent some other resource, for example,
			external IO availability. By making these resource types user-defined,
			applications programmers can express flexible hard-constraints on their
			application.
			
			Another observation is that generic soft constraints can be expressed may
			be expressed using a net with an appropriate weight.
			
			As a result of these facilities, application programmers can easily
			express their own application-specific hard and soft placement
			constraints without having to modify the algorithm. This representation
			has become a de-facto standard for placement problem interchange for
			SpiNNaker applications.
		
		\subsection{Cost function}
			
			At present I've used HPWL despite this being really bad for high-fan-out
			multicast and totally ignorant to the hexagonal nature of SpiNNaker...
			
			To compute bounding boxes for tori I use the following approach. For each
			dimension, sort the points on that dimension and find the largest gap
			between them on a ring. The bounding box goes the other way.
			
			TODO: FIGURE ILLUSTRATING BOUNDING BOX COMPUTATION FOR TORI.
		
		\subsection{Annealing schedule}
			
			The annealing schedule is that used by VPR. Despite being for circuit
			placement, it seems to work jolly well.
			
			TODO: DESCRIBE AND RATIONALISE THE SCHEDULE
		
		\subsection{Constraint handling}
			
			Various hard and soft constraints may be expressed by software
			approaches. For each we explain how they may be handled by the placement
			algorithm:
			
			\subsubsection{Location Constraint}
				
				The vertex is placed on a chip and removed from the set of movement
				candidates.
			
			\subsubsection{Same-chip constraint}
				
				When two vertices must always be placed on the same chip they are
				simply combined into one vertex which consumes the sum of their
				resources. Placement then treats them as one chip and thus is forced to
				atomically place the vertices.
			
			\subsubsection{Reserve resource constraint}
				
				Simply reduce resource availability on that chip.
			
			\subsubsection{Keep near Ethernet}
				
				Simply add a net.
	
	\section{Evaluation}
		
		\label{sec:placement-results}
		
		Though benchmarks exist for super computer loads and chip placement tasks,
		such things don't exist for neural applications. As a result I use a
		selection of real applications for SpiNNaker along with some synthetic
		benchmarks based on biological data.
		
		\subsection{Benchmark networks}
			
			First some real networks.
			
			Some nengo networks: SPAUN: `The world's largest functional brain model'.
			Word-net network from Jamie: Example of some learning.
			
			TODO: DESCRIBE SHAPE OF NENGO NETWORKS
			
			Some PyNN networks: Microcortical column model from PyNN. Note almost
			broadcast connectivity but varying weights. Try and extract a vision
			netlist from Anna. Maybe try and get a netlist for Tom's barrel cortex.
			
			Now for some artificial networks. Pipeline, noisy pipeline, mesh,
			Gaussian 2D.
		
		\subsection{Experiments}
			
			We compare random, linear, greedy and annealing based placement
			approaches to placement. We compare static metrics (such as mean/max
			congestion, table usage) along with experiments based on simulated
			network traffic in real hardware. Network Tester generates artificial
			traffic in proportion with the weights given for each model. We compare
			the relative level of traffic sustainable. We also consider use of
			machines of various sizes.
		
		\subsection{Results}
			
			SA placement is slow but rather effective, especially for some networks.
			Generally worth doing. Will need to be sped up for very large machines...
			
			TODO: EXPERIMENTS!
	

	\chapter{Discussion}

\section{Suitability of the hexagonal torus topology}
	\subsection{Physical scalability}
	\subsection{Routability}
	\subsection{Placeability}

\section{Suitability of the SpiNNaker router}
	\subsection{Deadlock avoidance}
	\subsection{Routing table size}

\section{Suitability of circuit placers for application placement}
	\subsection{Quality}
	\subsection{Runtime}
	\subsection{Routing resources}
	\subsection{Flexibility}
	\subsection{Scalability}


	\chapter{Future research}
	
	In this thesis I have presented a number of new techniques which have made it
	possible to assemble and operate the SpiNNaker super computer. This work
	opens up a range of possibie lines of research to extend this work to future
	architectures and applications. In this chapter I focus on two anticipated
	challenges of future systems: growing scale and greater dynamicism in
	applications.
	
	\section{Scaling up}
		
		TODO: INTRO
		
		\subsection{Grid machine room layouts}
			
			In chapter XXX, I developed a machine room layout for hexagonal torus
			topologies which allowed machines occupying a row of standard
			machine-room cabinets to scale up without the need for long
			interconnecting cables. For larger installations, however, it will be
			necessary to employ multiple rows of cabinets in a 2D arrangement.
		
		\subsection{Routing congestion control}
		
		\subsection{Parallel place and route}
	
	\section{Structural plasticity and dynamic fault-tolerance}
		\subsection{Plasticity models}
		\subsection{Incremental placement}
		\subsection{Incremental routing}
		\subsection{Hot-spare routes}

	\chapter{Conclusions and future research}
	
	The SpiNNaker architecture was designed to tackle the challenges presented by
	the simulation of biologically realistic neural networks. One of its
	distinguishing features is its network architecture which employs both an
	unconventional network topology and multicast router architecture. The
	hexagonal torus topology used by SpiNNaker was chosen to enable greater
	performance while maintaining ease of construction and scalability compared
	with conventional network topologies. SpiNNaker's router design centres
	around packets mimicking the neural `spike' signals they are designed to
	convey by being small, multicast and not guaranteed to arrive at their
	destination.  This novel design, though largely complete before this work
	began, left a number of open problems which this thesis has attempted to
	address.
	
	In this concluding chapter I begin by summarising the answers to the research
	questions raised in chapter~\ref{sec:introduction}. This is followed by a
	discussion of new research topics which have been uncovered by this work.
	
	\section{Answers to research questions}
		
		Each of the three research questions are answered below.
		
		\subsubsection{1. Can the hexagonal torus topology be deployed and used in
		real, large-scale systems?}
		
		In chapter~\ref{sec:building}, I introduced a cabling scheme and assembly
		technique which has been used successfully to build a prototype SpiNNaker
		system with over half a million processor cores using the hexagonal torus
		topology. The techniques shown are expected to enable a final SpiNNaker
		machine of double this size to be built, filling a six metre long row of
		machine-room cabinets.
		
		Though SpiNNaker's processor-count places it amongst some of the world's
		largest supercomputers (see figure \ref{fig:top500-num-processors} on page
		\pageref{fig:top500-num-processors}), it is comparatively compact, filling
		one row of cabinets compared with the warehouse-scale installations found
		in commercial systems. In spite of this, the folding and interleaving
		techniques described allow hexagonal torus topologies to scale to
		arbitrarily large installations without cables which span the machine.
		
		Chapter~\ref{sec:shortestPaths} described an efficient and general
		technique for finding, and enumerating shortest path vectors in hexagonal
		torus topologies. These developments bring the hexagonal torus topology in
		line with other topologies by enabling routing algorithms to exploit all
		possible paths in a network. Further, chapter~\ref{sec:placement}
		demonstrated that placement algorithms are also adaptable to hexagonal
		torus topologies thanks to their similarity to 2D toruses.
		
		Though, as this thesis highlights, hexagonal toruses lack many of the
		intuitive properties enjoyed by other topologies, it is still possible to
		reason about them with only limited computational effort.  Now that the
		practicality and scalability of the topology has also been demonstrated in
		practice, it represents a credible option for future network architectures.
		
		\subsubsection{2. Does SpiNNaker's router architecture help, or hinder
		fault tolerance?}
		
		SpiNNaker's unconventional use of packet dropping to avoid deadlocks
		greatly simplifies the router architecture, part of the motivation for this
		design. In chapter~\ref{sec:routing} this feature is used to the advantage
		of PGS repair to add fault tolerance to existing routing algorithms.
		Compared with the often complex and wasteful methods used to tolerate
		faults in other networks, PGS repair incurs very little performance
		overhead in the presence of static faults.
		
		Routing table usage does increase in the presence of faults, however, which
		may be a concern for applications which already require many routing table
		entries. Routing table usage, as well as other overheads, were most
		significantly increased in the presence of contiguous groups of network
		faults. This is because the PGS repair algorithm produces routes which pass
		tightly around the corners of faults, resulting in concentrations of
		routing table entries in those areas.  Though the symptoms of this problem
		can be attributed to the design of SpiNNaker's multicast routing mechanism,
		the responsibility lies with the behaviour of the PGS repair algorithm.
		Potential improvements to the PGS repair algorithm are discussed later in
		\S\ref{sec:pgs-repair-improvements}.
		
		The overall answer to this research question, therefore, is that the
		flexibility provided to routing algorithms in SpiNNaker's architecture is
		of great benefit, enabling arbitrary fault patterns to be inexpensively
		avoided.
		
		\subsubsection{3. How can the parts of a neural simulation be placed onto a
		large hexagonal torus topology to reduce network load?}
		
		In chapter~\ref{sec:placement}, I explored a number of contemporary
		approaches to the problem of placing applications with irregular
		communication patterns into network topologies. I observed that researchers
		working on circuit placement for chips and FPGAs are tackling similar
		problems and working at scales as large, or larger than, those faced in
		application placement. Based on this I developed a
		simulated annealing based placement algorithm inspired by the techniques
		used in circuit placement, with specific adaptations for use in application
		placement and SpiNNaker's network topology.
		
		The simulated annealing based placement algorithm consistently outperforms
		pre-existing placement algorithms included in benchmarks in terms of
		placement quality.  In the case of one benchmark, simulated annealing based
		placement made it possible to run that neural simulation in real-time for
		the first time.  At larger scales, simulated annealing was also found to be
		able to produce good quality placements in benchmarks containing over one
		million processes -- the largest size supported by the SpiNNaker
		architecture.
		
		The major shortcoming of simulated annealing based placement is its
		execution speed. Though its execution time grows in proportion to the size
		of the problem, the implementation used took over 12~hours to place a
		synthetic problem for the largest planned SpiNNaker machine. Though
		tractable -- particularly given the relative output quality compared with
		the prior state-of-the-art -- the algorithm is unlikely to function
		comfortably as-is on larger problems.
		
		The conclusion to be drawn from this result, however, is not just that
		simulated annealing is a good solution for today's placement problems but
		that circuit placement techniques in general could be successfully adapted
		to fulfil this role. The placement problems faced by chip designers are
		growing at roughly the same exponential rate as the size of super computers
		but circuit designs hold the lead in terms of problem size. Consequently,
		as approaches are retired by chip placement researchers, they may find new
		life in the field of application placement.
		
	\section{Future research}
		
		Though the goals of this study have largely been met, there also remain
		some important limitations which future work may hope to address.
		Furthermore, this work has uncovered a number of new research areas
		warranting future enquiry. This section outlines a number of future lines
		of research.
		
		\subsection{Warehouse-scale cabling}
			
			In chapter~\ref{sec:building} I developed and implemented a number of
			cabling schemes for the SpiNNaker architecture spanning up to a six metre
			row of machine-room cabinets -- a relatively small installation by
			current standards. In SpiNNaker, the cabling exists in a 2D plane (i.e.
			across the faces of the cabinets) but as the system is scaled up, a
			single row of cabinets will tend towards a 1D line. Since embedding a 2D
			structure in a 1D space necessarily results in long connections, this
			cannot scale indefinitely.
			
			\begin{figure}
				\center
				\buildfig{figures/multi-row-cabling.tex}
				
				\caption{Multiple rows of interconnected cabinets.}
				\label{fig:multi-row-cabling}
			\end{figure}
			
			In conventional large-scale super computer installations, nodes are
			installed in rows of cabinets as illustrated in
			figure~\ref{fig:multi-row-cabling}.  From a `bird's-eye' view, the system
			approximates a 2D space, spread across the floor of a machine-room.
			Therefore, in principle, the folding and interleaving techniques
			described in chapter~\ref{sec:building} still apply. Unfortunately for
			SpiNNaker, cables connecting between rows of cabinets would be longer
			than the one metre limit imposed by its hardware because of the spacing
			between rows of cabinets.  Future SpiNNaker systems will need to consider
			alternative link technologies.  For example, a hybrid system could be
			used in which intra-cabinet connections continue to use the current HSS
			link technology while inter-cabinet links might use optical connections.
			This type of architecture could be supported by the use of pluggable
			`SFP+' transceiver modules~\cite{sff01}.
		
		\subsection{Cabling assistance for other architectures}
			
			A secondary result of the construction of prototype SpiNNaker systems in
			chapter~\ref{sec:building} was the use of real-time guidance and feedback
			to assist cable installation. I am not aware of this technique's use by
			existing architectures and, following the success experienced in this
			project, it is possible that the technique may also be useful in
			conventional systems.
			
			During the construction of prototype SpiNNaker machines, each cable took
			seconds to install compared with the minutes reported for existing
			systems~\cite{mudigonda11}. Part of this increase in efficiency appears
			to result from the immediate identification of mistakes made during
			cabling, saving time-consuming backtracking later on.
			
			In many real-world network installations, units are less densely packed
			than in SpiNNaker and so longer cables are often required. As a
			consequence, cabling errors may become more likely as cabling patterns
			are spread over a larger area making them more difficult to visually
			verify. Like SpiNNaker, conventional networking hardware is often
			equipped with a generous range of indicator LEDs and diagnostic
			facilities which might be used to implement real-time installation
			guidance. Future work could explore the use of this technique in the
			construction of other large-scale networks, such as data centres.
		
		\subsection{Congestion mitigation}
			
			\label{sec:wiggly-board-allocations}
			
			In chapter~\ref{sec:routing} I found that contiguous network faults cause
			hot-spots of congestion and routing table depletion where the PGS repair
			algorithm routed many paths around the edges of faults.  However, it is
			not just faults which can cause contiguous blockages in the network
			topology. In reality, researchers do not always require a full-sized
			SpiNNaker system to perform their experiments so large SpiNNaker systems
			are soft-partitioned on demand into many smaller
			machines~\cite{spalloc16}. To ensure isolation between partitioned
			sub-machines, HSS links between boards in different partitions are
			disabled. Because of SpiNNaker's `wrapped triple' partitioning scheme,
			the resulting sub-machines have hexagonal \emph{mesh} topologies (i.e.
			without wrap-around links) with irregular boundaries as in
			figure~\ref{fig:spalloc-mesh}.
			
			\begin{figure}
				\center
				\buildfig{figures/spalloc-mesh.tex}
				
				\caption[Irregular edges of a partitioned SpiNNaker system.]%
				{Irregular edges in a SpiNNaker system comprised of 24~boards
				partitioned from a larger machine.  Each hexagon represents a SpiNNaker
				chip. No wrap-around connections are present.}
				\label{fig:spalloc-mesh}
			\end{figure}
			
			In partitioned systems, the `tooth'-like gaps on the periphery of the
			network result in similar congestion to the HSS link failures considered
			in chapter~\ref{sec:routing}. When a route is generated between nodes on
			opposite sides of a gap, the PGS repair process will produce a
			shortest-path route around it. Since many routes may be blocked by a
			single gap, a hot-spot may develop around the corners of the gap.
			
			In chapter~\ref{sec:placement}, the `CConv' benchmark application was
			found to run correctly the majority of the time when placed by the
			simulated annealing algorithm but would occasionally fail by a
			significant margin. Preliminary experiments suggest these occasional
			failures are caused by placement solutions which place heavily
			communicating parts of the application on opposite sides of gaps along
			the perimeter of the network. Two possible approaches which future work
			may consider are presented below.
			
			\subsubsection{Avoiding hotspots with PGS repair}
				
				\label{sec:pgs-repair-improvements}	
				
				Network congestion around faults and network irregularities could be
				reduced by forcing the PGS repair process to take more varied routes
				around faults. For example, in circuit routing algorithms, routers
				avoid congestion by increasing the cost of routes which pass through
				congested areas~\cite{kahng11}. A similar technique could be used in
				PGS repair to spread the routes it produces.
				
				An alternative approach would be to adapt the base routing algorithms
				used prior to PGS repair to, for example, attempt alternative dimension
				order routes which may avoid blockages due to faulty links.
			
			\subsubsection{Fault and irregularity aware placement}
				
				One of the shortcomings of the simulated annealing based placer
				developed in chapter~\ref{sec:placement} is that it does not account
				for network faults, or irregularities, when estimating the cost of
				placement solutions.  Future work may exploit techniques used in
				congestion-aware circuit placement which could be adapted for
				application placement~\cite{viswanathan07}.
		
		\subsection{Reducing placement execution time}
			
			The simulated annealing based placer presented in
			chapter~\ref{sec:placement} produced good quality placements but its
			execution time limits its use beyond one million vertex placement
			problems. Future work should explore possibilities for improving the
			performance and scalability of this technique.
			
			In addition to considering alternative placement algorithms based on
			other methods, one possible approach is to attempt to reduce the execution
			time of simulated annealing based placement by shrinking the application
			graph being placed.
			
			For example, graph clustering~\cite{schaeffer07} may be used to group
			together strongly connected vertices which would then be placed as a
			single unit.  Unfortunately, clustering can suffer from the same problems
			as graph-partitioning-based placement: vertices may be grouped together
			in ways which, in practice, cannot be packed together into a given portion
			of a machine.  A possible solution to this problem is to use a two-phase
			placement approach~\cite{kahng11}. In a `global' placement phase,
			solutions are permitted which can slightly over-allocate resources but
			overall achieve good placement quality. In the `detailed' placement phase
			which follows, the solution is `legalised' by making small changes to the
			global placement to eliminate over allocation.
			
			An alternative approach suited to SpiNNaker could be to limit the
			clustering process to clusters which fit on a single SpiNNaker chip. In
			typical SpiNNaker application graphs, clustering to this level may reduce
			placement problem sizes by an order of magnitude and, consequently,
			reduce execution times by the same ratio. Preliminary experiments suggest
			that this approach might result in little placement quality loss for
			large placement problems whilst substantially reducing overall execution
			time.
		
		\subsection{Benchmarking}
			
			One of the most significant limitations of this study has been the
			unavailability of large-scale SpiNNaker applications for use as
			benchmarks. As a consequence, much of the scalability experimentation
			performed has relied on simple synthetic benchmarks based on projections
			of future application behaviour.
			
			In the short term, more sophisticated synthetic benchmark generation
			techniques used by the circuit placement community~\cite{nam07} may offer
			alternative benchmarks for future work. In the longer term, however, it
			is hoped that the availability of large SpiNNaker systems -- and
			placement and routing algorithms better suited to exploit them -- will
			lead to larger scale applications being developed. Hopefully these
			applications will lead to more interesting and representative benchmarks
			for use in future work.
	
	\section{Closing remarks}
		
		One of the primary outcomes of this work is that a number of the practical
		challenges faced in scaling up the SpiNNaker architecture have been
		addressed leading to the construction of large-scale SpiNNaker machines.
		The development of an effective placement algorithm for SpiNNaker
		applications has been shown to enable some neural simulations to exploit
		SpiNNaker's architecture for the first time. The availability of larger
		SpiNNaker machines paves the way for future large-scale neural modelling
		work built on much larger models such as Spaun, `the world's largest
		functional brain model'~\cite{eliasmith12}.
		
		Beyond the SpiNNaker project, the hexagonal torus topology has also been
		validated as a scalable and practical candidate for future network
		architectures. As super computers become ever larger, the physical
		scalability afforded by the 2D nature of the hexagonal torus topology may
		make it a compelling option. In addition, the finding that circuit
		placement techniques can be adapted to support placement of SpiNNaker
		software indicates that these algorithms may also be applicable to other
		applications. Indeed, if this is the case, circuit placement may offer a
		long-term source of placement algorithms able to handle the demands of
		future applications.
		
		% This thesis has explored and tackled a number of the challenges posed in
		% scaling up the unconventional SpiNNaker architecture. Along the way I have
		% demonstrated that the hexagonal torus topology may be a practical choice in
		% future applications which can scale up to the physical dimensions expected
		% of future super computers. I have also developed new efficient and
		% effective methods of placing and routing neural simulation software on
		% SpiNNaker which -- it is hoped -- will enable a new generation of large
		% scale neural simulations on spinnaker.
		
		Although this work stops short of demonstrating truly large-scale
		neuroscientific simulations running at the scale of newly completed
		SpiNNaker machines (largely because such simulations do not yet exist) a
		number of smaller-scale neural simulations have been made possible for the
		first time. The algorithms and techniques devised in this work have
		subsequently been incorporated into various software libraries and tools
		now being used by researchers building SpiNNaker applications, vindicating
		the efforts of this thesis (see appendix~\ref{sec:software}). A successor
		to the SpiNNaker architecture is also in the early stages of design and is
		building on experience of the existing architecture. The current intention
		is to retain the hexagonal torus topology used by SpiNNaker, a decision
		supported by the findings of this thesis.
		
		With SpiNNaker's hardware architecture now operating at scales close to its
		architectural limits, it is hoped that the contributions of this work will
		enable researchers to develop larger and more detailed neural models for
		this unique architecture.

	
	% Bibliography
	\bibliography{references}
	\bibliographystyle{alpha}
	
\end{document}
words.
	
	\clearpage
	\listoffigures
	
	\clearpage
	\listoftables
	
	% Abstract
	{
	\prefacesection{Abstract}
	
	% Single line spacing for the abstract page
	\setstretch{1.0}
	
	
	\vfill
	
	% Standard thesis information
	\begin{center}
		\textsc{\large\thesistitle}
		
		\vspace{0.5em}
		
		\thesisauthor
		
		\vspace{0.5em}
		
		A thesis submitted to the University of Manchester\\
		for the degree of Doctor of Philosophy, 2016
	\end{center}
	
	\vfill
	
	% The abstract
	
	SpiNNaker is an unconventional super computer architecture designed to
	simulate up to one billion biologically realistic neurons in real-time. To
	achieve this goal, SpiNNaker employs a novel network architecture which poses
	a number of practical problems in scaling up from desktop prototypes to
	machine room filling installations.
	
	SpiNNaker's hexagonal torus network topology has received mostly theoretical
	treatment in the literature. This thesis tackles some of the challenges
	encountered when building `real-world' systems.  Firstly, a scheme is devised
	for physically laying out hexagonal torus topologies in machine rooms which
	avoids long cables; this is demonstrated on a half-million core SpiNNaker
	prototype.  Secondly, to improve the performance of existing routing
	algorithms, a more efficient process is proposed for finding (logically)
	short paths through hexagonal torus topologies. This is complemented by a
	formula which provides routing algorithms greater flexibility when finding
	paths, potentially resulting in a more balanced network utilisation.
	
	The scale of SpiNNaker's network and the models intended for it also present
	their own challenges. Placement and routing algorithms are developed which
	assign processes to nodes and generate paths through SpiNNaker's network.
	These algorithms reduce congestion and tolerate network faults. The proposed
	placement algorithm is inspired by techniques used in chip design and is
	shown to enable larger applications to run on SpiNNaker -- with good
	performance -- than the previous state-of-the-art. Likewise the routing
	algorithm developed is able to tolerate network faults, inevitably present in
	large scale systems, with little performance overhead.
	
	
	% Required to ensure single line spacing is used for this whole block
	\par%
}

	
	% Declaration of non-submission elsewhere
	\prefacesection{Declaration}

% Single line spacing for the declaration
{
	\setstretch{1.0}
	No portion of the work referred to in this thesis has been submitted in support
	of an application for another degree or qualification of this or any other
	university or other institute of learning.
	
	\par%
}


	
	% University-prescribed copyright statement...
	\input{copyright}
	
	% Acknowledgements
	{
	\prefacesection{Acknowledgements}
	
	% Single line spacing
	\setstretch{1.0}
	
	It is often said that it is not \emph{what} you know but \emph{who} you know.
	Throughout the course of my PhD I've been exceptionally lucky to have been
	helped along by a great number of people.
	
	Both my supervisor, Jim Garside, and co-supervisor, Steve Furber, have each
	spent countless hours patiently discussing and describing all manner of
	things with me while giving me great freedom in my project. Jim's office door
	has always been open to my unexpected interruptions be it about work, writing
	or walking.  Likewise, Steve has always managed to find time for both
	technical and frivolous endeavours alike. I'm also hugely grateful to Luis
	Plana who has been a rich source of sage advice, insightful questions
	patiently suffered many a foolish question.
	
	Various parts of the work in this thesis (and numerous side projects) would
	not have been possible if not for the multitude of discussions,
	collaborations and even sheer physical hard work of Steve Temple, Javier
	Navaridas, Simon Davidson and Dave Clark. I'm also indebted to Andrew Mundy
	and Jamie Knight, both of whom have donated so much time and effort towards
	verifying and using software implementations of the ideas in this thesis.
	
	The injection of lunchtime silliness by Andrew and Jamie, along with Amanieu
	d'Antras and Andrew Webb and the other CDT members has always brightened my
	day. So to has the friendly and stimulating environment of the School of
	Computer Science and its many staff and students. Of course, I am also very
	grateful for the funding the school has provided for my research.
	
	I cannot thank my wonderful wife, Ann-Marie, enough for being by my side. She
	has given me so much kindness, love and patience and endured a lifetime's
	quota of conversations about hexagons. Finally, thanks too to rest of my
	family, especially my parents, who are to blame for starting me down this
	path and co-suffering drafts and endless rants about this document.
	
	% Required to ensure single line spacing is used for this whole block
	\par%
}

	
	% Main body
	\chapter{Introduction}

\label{sec:introduction}

%Problem area
%
%* Network construction and exploitation
%  * Cabling: Build it cheaply in terms of tech cost and install cost
%  * Routing: Get around it cheaply and reliably
%  * Placement: Use it efficiently

The Spiking Neural Network Architecture (SpiNNaker) is a novel super computer
architecture designed to simulate biologically realistic models of brains in
real time \cite{furber07}. Though neurons, the building blocks of the brain,
are relatively well understood, their complex interactions remain mysterious.
Just as understanding the workings of a transistor is insufficient to
understand a modern microprocessor, neuroscientists believe that understanding
the neurons in isolation cannot explain the brain and that understanding their
connectivity is key \cite{eliasmith13,eliasmith14}. Experiments on real brains,
however, are fraught with difficulty. Variations between individuals can be
significant and it is only possible to record tens or hundreds of the trillions
of signals in the brain, and even then only with limited control over which
signals are recorded. Computer simulations of models of large neural networks,
however, enable researchers to develop repeatable experiments and gain complete
visibility of any signal and any neuron. Models such as SPAUN
\cite{eliasmith12}, built from millions of simulated neurons, have shown great
promise in expanding our understanding of higher level brain functions such as
memory and simple problem solving.  Unfortunately these neural models are
expensive to simulate, requiring hours of compute time to simulate each second
of neural activity. As well as being inconvenient, this precludes the use of
robotics to immerse these models in real world environments and also limits
studies of long-term behaviours such as learning.

SpiNNaker is designed to enable the real time simulation of models containing
up to one billion neurons -- approximately \SI{1}{\percent} of a human brain or
ten mouse brains \cite{furber06}. To achieve this goal, the largest planned
SpiNNaker machine will contain over one million low-powered computer processors
interconnected by a bespoke network architecture.

SpiNNaker's large processor count matches the current trend in super computers
where processor counts are growing exponentially \cite{meuer16j}, mimicking the
growth of the number of components in the processors themselves predicted by
Gordon Moore's famous `law' \cite{moore75}. As a result of this growth, the
interconnection networks which enable these processors to work together have
grown in importance \cite{dally04}.  Network designers must carefully balance
performance against practicality and financial cost.  SpiNNaker's network is no
exception to this rule and, as the systems scale up from desktop prototypes to
machine-room scale installations, the reality of building and exploiting these
machines presents an array of challenges.

As in all super computers, SpiNNaker's network interconnects its processors in
a particular network topology which defines how different processors may
communicate with each other. Unlike the tree and $N$-dimensional torus
topologies found in contemporary super computers \cite{dally04}, SpiNNaker
employs a `hexagonal torus topology'. In this topology, nodes in SpiNNaker's
network fit together in a honeycomb-like pattern where messages may `hop' from
node to node to reach their destination. As we will see in
chapter~\ref{sec:background}, the hexagonal torus topology, in theory, sits at
a `sweet spot' in terms of network performance and practicality. As the first
known large-scale installation of the hexagonal torus topology, however, there
remain a number of practical challenges for large spinnaker machines arising
from this choice.

As super computer networks have grown in scale to millions of processors the
task of dealing with previously rare faults has grown.  Though fault rates in
networks remain consistently low, architectures such as SpiNNaker may have
hundreds of thousands of links meaning even fault rates of a fraction of a
percent will impact tens or hundreds of links. To enable reliable operation,
networks must be able to adapt the routes taken by messages through the network
to avoid faulty links and nodes. The techniques employed are often closely tied
to a particular network architecture and consequently SpiNNaker's novel network
architecture demands its own approach.

Another challenge introduced by the growing scale of super computers is making
\emph{efficient} use of network resources. Communicating processes should be
located on logically `nearby' nodes to reduce network load. The neural models
for which SpiNNaker is designed are often described abstractly, rather than
geometrically, using modelling languages such as PyNN~\cite{davison08} and
Nengo~\cite{eliasmith04}.  Because of this, the communication requirements of
simulations can be highly irregular making an efficient placement of processes
onto processors in the machine non-trivial.

%Contributions
%
%* Cabling scheme for hexagonal toruses without long cables
%* Efficient installation technique for dense systems
%* Exhaustive and efficient route calculation in hex toruses
%* Fault tolerant routing scheme exploiting SpiNNaker's odd router
%* Placement based on SA a: works very well and b: suggests circuit placement is
%  a good source of inspiration.

This thesis addresses the practical challenges of scaling up the SpiNNaker
architecture in a real-world setting summarised by these research questions:

\begin{enumerate}
	
	\item Can the hexagonal torus topology be deployed and used in real, large
	scale systems?
	
	\item Does SpiNNaker's router architecture help, or hinder fault tolerance?
	
	\item How can the parts of a neural simulation be placed onto a large
	hexagonal torus topology to reduce network load?
	
\end{enumerate}

%Structure
%
%* Chapter 2: Background: detailed dive into what's in SpiNNaker, why its
%  really so unusual. Also looks at what applications run on SpiNNaker and how
%  they work.
%* Chapter 3: How to build a really big SpiNNaker machine.
%* Chapter 4: How to find your way around that machine.
%* Chapter 5: How to find your way around that machine even when its broken.
%* Chapter 6: Now you can walk, time to run.
%* Chapter 7: Wrapping up.
%* Appendices: Hard-to-come-by theoretical and practical details useful if
%  you're about to continue where this research left off but be useful but
%  otherwise hard to come by, especially in one place.

Chapter~\ref{sec:background} introduces the SpiNNaker architecture and, in
particular, describes its hexagonal torus topology and network architecture.

In chapter~\ref{sec:building}, I develop a cabling scheme for large hexagonal
torus topologies which enables arbitrarily large networks to be constructed
using only short, inexpensive cables. This theoretical work is then evaluated
through the construction of a range of prototype SpiNNaker systems. The largest
of these prototypes contains over half a million processor cores and spans
several machine room cabinets. In addition, I propose the use of built-in
diagnostic facilities to assist technicians performing network installation and
maintenance. This technique is found to greatly reduce the effort required and
the number of mistakes made.

In chapters~\ref{sec:shortestPaths}~and~\ref{sec:routing} I develop new routing
techniques for SpiNNaker's network. Chapter~\ref{sec:shortestPaths} develops a
new approach to finding the shortest paths through hexagonal torus topologies,
an integral part of many routing algorithms. This newly proposed approach is
cheaper to compute than the state of the art and, unlike previous efforts, is
able to discover all valid short paths through the topology. This theoretical
advance brings hexagonal torus topologies in line with conventional topologies
by providing routing algorithms with complete information about the paths
available to them. In chapter \ref{sec:routing} I propose a fault tolerant
routing algorithm for SpiNNaker which is able to avoid arbitrary static fault
patterns with minimal performance overhead. A key finding of this chapter is
that the flexibility afforded to fault tolerant routing algorithms by
SpiNNaker's unconventional router architecture is what facilities the low
overheads reported in this chapter.

Finally, in chapter~\ref{sec:placement}, I explore the problem of application
placement in SpiNNaker's network. As in other networks and applications, neural
simulations should be arranged such that communication occurs primarily between
processors close together in the network to control network load. Due to the
irregular connectivity and large scale of the neural models expected to run on
SpiNNaker, an automated approach is necessary. I develop a novel placement
algorithm based on algorithms used for circuit layout in computer chips. My
algorithm is found to allow some larger neural models to run on SpiNNaker for
the first time while enabling other applications to run at greater speeds. In
addition, synthetic benchmarks containing over one million processes indicate
that this algorithm should handle the anticipated demands of the neural models
expected to run on large-scale SpiNNaker installations.

	\chapter{The SpiNNaker Architecture}
	
	\label{sec:background}
	
	SpiNNaker is a massively parallel computer architecture designed to simulate
	biologically realistic neural models \cite{furber07}. In this chapter we will
	explore this unconventional architecture in detail, starting with its purpose
	before focusing on its most unconventional feature: its network.
	
	% * Purpose
	%   * Spiking neural simulations
	%     * Neural modelling: PyNN, Nengo...
	%     * Parallelisation + communication
	
	\section{Neural simulation}
		
		Human brains contain billions of neurons connected together by trillions of
		`synapses'. Neurons communicate by transmitting and receiving `spikes'
		through their synapses. Each spike is `valueless' in that a spike's only
		significant features are when it arrives and where it has come from.
		
		\begin{figure}
			\center
			\buildfig{figures/lif-neuron.tex}
			
			\caption{A Leaky Integrate-and-Fire (LIF) neuron.}
			\label{fig:lif-neuron}
		\end{figure}
		
		Though some detailed models of the electrochemical processes occurring
		inside neurons are computationally intensive, simplified models such as the
		Leaky Integrate-and-Fire (LIF) model can be implemented in just a handful
		of CPU instructions \cite{vainbrand11}. Figure~\ref{fig:lif-neuron}
		illustrates a simple LIF neuron in which incoming spikes cause charge to
		build up (integrated) which over time, leaks away. If an incoming spike
		causes the charge to rise above a certain threshold, the neuron `fires'
		producing an outgoing spike. Despite the simplicity of this model, large
		neural networks such as Spaun \cite{eliasmith12} -- built entirely from LIF
		neurons -- exhibit complex behaviours such as fine motor control and
		problem solving.
		
		The computational expense of large scale neural simulations does not arise
		from the cost of modelling neurons but instead from distributing spikes. In
		biology, neurons produce spikes at an average rate of \SI{10}{\hertz} and
		synapses connect each neuron's output to (order) \num{1000}~neurons
		\cite{navaridas09}. Consider an example neural model with $7\times10^7$
		neurons, approximately the number in a house mouse and
		$\nicefrac{1}{10}^\textrm{th}$ of the design target of SpiNNaker. This
		network might produce $7\times10^8$~spikes per second. Because each neuron
		connects to many others, this equates to $7\times10^{11}$ spikes being
		received per second. If each spike were transmitted as a UDP datagram
		containing a single \SI{32}{\bit} payload, the total network throughput
		required for this simulation would be \SI{179.2}{\tera\bit\per\second}. At
		the time of writing, this is more than double the bisection bandwidth (the
		theoretical worst-case throughput) of the world's most powerful super
		computer \cite{dongarra16}.
	
	\section{Network architecture}
		
		Architectures such as IBM's Blue Gene \cite{chiu11} and Cray's XK7
		\cite{ornl16} employ powerful compute nodes connected together using
		networks designed to transfer multi-kilobyte blocks of data between nodes.
		Since neural models have relatively light computational requirements and
		communications are based on small pieces of data (spikes), this type of
		architecture is poorly suited to the task.
		
		SpiNNaker's architectural target is to support realtime simulations of up
		to one billion neurons. Since neural models such as LIF are inexpensive to
		model and many neurons can be simulated independently in parallel,
		SpiNNaker employs many small, energy efficient ARM processors
		\cite{furber07}. To support the unusual communication requirements of
		neural simulations, a bespoke interconnection network is used which is the
		background to this thesis.
		
	%   * SpiNNaker chip
	%     * Cores
	%     * SDRAM
	%     * NoC
	%     * Router
		
		\begin{figure}
			\center
			%\includegraphics[width=19mm]{figures/spinnakerChip.jpg}
			\buildfig{figures/hex-chips.tex}
			
			\caption[SpiNNaker chips connected to their six neighbours.]%
			{SpiNNaker chips (actual size) connected to their six neighbours.}
			\label{fig:spinnakerChip}
		\end{figure}
		
		The fundamental building block of the SpiNNaker architecture is the
		SpiNNaker chip (figure \ref{fig:spinnakerChip}) \cite{furber13}. Each chip
		contains eighteen low power ARM 968 processor cores each capable of
		simulating between \num{200} and \num{2000} LIF neurons in real time
		\cite{mundy15}.  Each core has a total of \SI{96}{\kilo\byte} of private
		Tightly-Coupled Memory (TCM) and shares access to \SI{128}{\mega\byte} of
		on-chip SDRAM with other cores on the same chip. Finally, each chip
		contains a programmable router which routes network packets to and from the
		local cores and six neighbouring SpiNNaker chips. SpiNNaker machines are
		constructed by combining many SpiNNaker chips.
		
		\begin{figure}
			\center
			\buildfig{figures/spinnaker-packet.tex}
			
			\caption{SpiNNaker's \SI{40}{\bit} and \SI{72}{\bit} multicast packet
			format.}
			\label{fig:spinnaker-packet}
		\end{figure}
		
		Processor cores can communicate by sending and receiving network packets
		forwarded by routers through the network. Since SpiNNaker's network is
		designed to transmit neural spike events efficiently, individual network
		packets are small, either \SI{40}{\bit} or \SI{72}{\bit} compared with tens
		or hundreds of byte packets in typical network architectures.
		
		In a real-time simulation, the time at which a spike is produced is
		implicitly indicated by the time it is received -- since at biological
		timescales a computer network delivers packets `instantaneously'.
		Consequently, the only information which must be explicitly encoded is the
		identity of the neuron which produced the spike. In SpiNNaker, a spike may
		be encoded by using a single \SI{40}{\bit} `multicast packet' whose format
		is illustrated in figure~\ref{fig:spinnaker-packet}.  The \SI{8}{\bit}
		header is used by SpiNNaker's routers to determine the type of packet and
		the \SI{32}{\bit} `routing key' is used to identify the neuron which
		produced the packet. The routing key is also used by SpiNNaker's routers to
		determine how the packet should be directed through the network.
		
		The optional \SI{32}{\bit} payload is not used by conventional spiking
		neural simulations \cite{galluppi10} but has been exploited to enable more
		efficient simulation of a particular class of neural models \cite{mundy15}.
	
	\section{The SpiNNaker router}
		
		The SpiNNaker router employs an unconventional design which, despite its
		compact size and small energy requirements, implements a flexible multicast
		routing scheme. Unlike conventional routers which often employ hard-coded
		routing rules \cite[chapter~8]{dally04}, the SpiNNaker router uses a
		programmable `routing table' to determine how packets should be forwarded.
		In addition, to avoid deadlocks, SpiNNaker's router employs a simple,
		timeout-based mechanism which exploits the ability of neural networks to
		tolerate occasional missing packets. As we will see in chapter
		\ref{sec:routing}, this mechanism greatly simplifies the task of routing in
		SpiNNaker's network. In this section we'll look at these features in
		greater detail.
		
		\subsection{Routing tables}
		
			When a multicast packet arrives at a SpiNNaker router (either from a
			local core or a neighbouring chip), the router looks up the routing key
			in its routing table. This table consists of \num{1024} programmable
			table entries, each specifying a routing key bit pattern and mask to
			match and a set of routes.  When a multicast packet's key is matched by a
			routing entry the packet is forwarded along every route specified by that
			entry, potentially duplicating the packet. This `multicast' technique
			allows packets to be transmitted once but received in a number of places
			while making efficient use of the network \cite{navaridas12}.
			
			Though routing table entries are in finite supply (\num{1024} entries per
			router), it is still possible for many thousands of traffic flows to be
			routed through a single router. The bit pattern and mask in each routing
			entry matches against the 32~bits of a routing key as either
			`\texttt{1}', `\texttt{0}' or `\texttt{X}' (don't care).  This means that
			a single routing entry may, for example, be used to match all routing
			keys with a certain prefix. If a routing key is not matched by any entry
			in the routing table then the packet is `default routed' in a straight
			line. For example if a packet with an unmatched key is received from the
			chip to the left, the packet will be default routed to the chip on the
			right. By assigning routing keys such that neurons whose spikes are sent
			to similar destinations share a similar prefix, the number of routing
			entries required by a simulation is greatly reduced \cite{davies12}.
			
			\begin{figure}
				\center
				\buildfig{figures/routing-example.tex}
				
				\caption[Multicast routing example.]%
				{Multicast routing example with \SI{4}{\bit} routing keys. Each
				box represents a SpiNNaker chip whose router has been programmed with
				the routing entries shown. Grey lines mark connections between chips.}
				\label{fig:routing-example}
			\end{figure}
			
			Consider the simplified example in figure~\ref{fig:routing-example} in
			which a number of (\SI{4}{\bit}) routing table entries have been
			configured in the routers of a small SpiNNaker network. If a packet with
			the routing key \texttt{1011} is transmitted by a core in the chip
			labelled $(0, 0, 0)$, this will match the first routing table entry on
			that chip and will be routed to chip $(1, 0, 0)$. On chip $(1, 0, 0)$,
			the packet once again matches the first routing entry and is routed to
			chip $(1, 0, -1)$. On $(1, 0, -1)$, no match is made so the packet is
			default routed to $(1, 0, -2)$. On this chip, the packet matches a
			routing entry which routes the packet to core~7. In this example, default
			routing allows only three routing table entries to direct a packet
			through four chips.
			
			As a second example, if a packet with the routing key \texttt{0010} is
			transmitted by a core on chip $(0, 0, 0)$, this key will be matched by
			the second routing entry since \texttt{X}s in the table entry will match
			both \texttt{1}s and \texttt{0}s in the corresponding bits of the routing
			key. When the packet arrives at chip $(0, 0, -1)$ the matching routing
			entry forwards the packet to both $(0, 1, -1)$ and $(1, 0, -1)$
			simultaneously. The copy of the packet arriving at $(0, 1, -1)$ is routed
			to core~5 on that chip.  Meanwhile, the copy forwarded to $(1, 0, -1)$ is
			duplicated again with one copy being routed to core~11 and another being
			routed to chip $(1, 0, -2)$. Here the packet is finally delivered to
			core~6. In this example, the ability of the router to multicast
			(duplicate) packets as they pass through the network meant that sending
			one copy of the packet was sufficient to reach three destination cores.
			In addition, by using \texttt{X}s in the routing table entry, the same
			routing entries are sufficient to route packets with the keys
			\texttt{0000}, \texttt{0001}, \texttt{0010} and \texttt{0011}.
			
			In spite of these mechanisms, it is still possible for an application to
			run out of routing table entries. As we will see in
			chapter~\ref{sec:placement} by arranging applications appropriately
			within SpiNNaker's network, routing table usage can be reduced. In
			addition, other behaviours of SpiNNaker's router may be exploited to
			compress an applications routing tables further, however the techniques
			employed are beyond the scope of this thesis \cite{mundy16}.
		
		\subsection{Timeouts}
			
			SpiNNaker's router is built on a pipeline architecture. As shown in
			figure~\ref{fig:router-architecture}, the router is fed packets by an
			arbiter which serialises packets arriving from other chips and local
			cores. Every (\SI{100}{\mega\hertz}) clock cycle, the router pipeline
			accepts one packet from the arbiter and routes a packet to one or several
			output links. If any of the required output ports are busy then the
			packet is not forwarded to any output link and the pipeline stalls. Once
			a packet has been blocked for a programmable timeout, it is dropped
			(discarded) and routing continues as usual for next packet in the
			pipeline. Links become blocked while transmitting packets or waiting for
			the remote receiver to become ready. For example, a receiving processor
			core may be busy performing some computation or a receiving router may be
			blocked waiting for some of its outputs to become ready.
			
			\begin{figure}
				\center
				\buildfig{figures/router-architecture.tex}
				
				\caption{SpiNNaker router architecture}
				\label{fig:router-architecture}
			\end{figure}
			
			The timeout-based packet dropping mechanism is designed to defuse
			deadlocks in the network. For example, if two routers are trying to send
			each other a packet at the same time they may become deadlocked, each
			waiting for the other router to accept a packet before continuing.
			SpiNNaker's timeout mechanism breaks deadlocks by dropping packets which
			have been blocked for some time and therefore may be in a deadlock.  Once
			a packet has been dropped it is left to software to either tolerate the
			missing packet or trigger a retransmission. In neural simulations, as in
			biology, the loss of a single spike is unlikely to have a significant
			impact on the behaviour of a neural model and therefore these simulations
			are inherently tolerant of occasional dropped packets. During application
			loading and other system tasks, a higher level, software driven protocol
			based on acknowledgements and retransmissions is used to ensure
			guaranteed delivery.
			
			% TODO: MENTION TIMEOUT VALUE USED?
			% Router timeouts must be configured to be long enough that delays in
			% packet transmission, for example due to the time taken for packets to
			% traverse a link, do not trigger packet dropping. Conversely, the timeout
			% should be as short as possible to reduce the time the router is
			% blocked and maximise network throughput.
	
	\section{The hexagonal torus topology}
		
		Each SpiNNaker chip is a node in a `hexagonal torus topology' as
		illustrated in figure~\ref{fig:hexagonalTorusTopology}. Network packets
		sent by SpiNNaker's processor cores may `hop' through several nodes in the
		network to reach their intended destination. In each hop, a packet may
		advance one node along one of the three axes of the topology. For example,
		a packet sent by the node labelled $\alpha$ (in the bottom-left corner) to
		the node labelled $\beta$, might take the following sequence of hops:
		X$^+$, X$^+$, Z$^-$. Packets sent from $\alpha$ to $\gamma$ might take the
		route: X$^-$, X$^-$, Y$^+$, Y$^+$. The first hop of this route `wraps
		around' from the bottom-left node to the bottom-right node in a single hop.
		
		\begin{figure}
			\center
			\buildfig{figures/hexagonalTorusTopology.tex}
			
			\caption[A hexagonal torus topology.]%
			{A hexagonal torus topology. Each hexagon represents a node (i.e.
			a SpiNNaker chip). Touching nodes are directly connected. Nodes on edges
			$a$, $b$ and $c$ are also directly connected to the corresponding nodes
			on edges $a'$, $b'$ and $c'$, respectively. The three axes of the
			hexagonal torus topology, `X', `Y' and `Z' are also shown.}
			\label{fig:hexagonalTorusTopology}
		\end{figure}
		
		\begin{figure}
			\center
			\begin{subfigure}{0.39\linewidth}
				\center
				\includegraphics[width=\linewidth]{figures/torus-3d-flat.pdf}
				\caption{}
				\label{fig:torus-3d-flat}
			\end{subfigure}
			~~
			\begin{subfigure}{0.26\linewidth}
				\center
				\includegraphics[width=\linewidth]{figures/torus-3d-tube.pdf}
				\caption{}
				\label{fig:torus-3d-tube}
			\end{subfigure}
			~~
			\begin{subfigure}{0.23\linewidth}
				\center
				\includegraphics[width=\linewidth]{figures/torus-3d-torus.pdf}
				\caption{}
				\label{fig:torus-3d-torus}
			\end{subfigure}
			
			\caption{Visualisation of a hexagonal torus topology as a torus.}
			\label{fig:torus-3d}
		\end{figure}
		
		The wrap around connections in the topology are what give it the `torus'
		part of its name. Figure~\ref{fig:torus-3d-flat} shows a hexagonal torus
		topology drawn flat as in the previous figure. If the topology is rolled up
		into a tube such that the top and bottom nodes become directly adjacent, a
		tube is formed as in figure~\ref{fig:torus-3d-tube}. This tube can then be
		bent to bring together the nodes at the ends of the tube to form a torus as
		shown in figure~\ref{fig:torus-3d-torus}.
		
		A hexagonal torus topology is typically defined in terms of its width and
		height along the X and Y axes respectively. For example,
		figure~\ref{fig:hexagonalTorusTopology} shows a $10\times10$ hexagonal
		torus.  The nodes in a hexagonal torus topology are addressed using
		hexagonal coordinates of the form $(x, y, z)$ \cite{patel15}. The bottom
		left node (labelled $\alpha$ in the figure) has the coordinate $(0, 0, 0)$
		and other nodes are assigned coordinates according to the number of hops
		along each dimension from $(0, 0, 0)$, for example node $\beta$ has the
		coordinate $(2, 0, -1)$. Because the hexagonal torus topology's axes are
		non-orthogonal, it is possible to define several coordinates for the same
		location. For example $(3, 1, 0)$ and $(1, -1, -2)$ are also valid
		coordinates for node $\beta$. These dual coordinates emerge from the fact
		that adding $(1, 1, 1)$ to a coordinate produces an equivalent, but
		different, coordinate. This phenomenon is explained in detail in
		appendix~\ref{app:minimal-hex-coordinates} and related phenomena will be
		discussed in chapter~\ref{sec:shortestPaths}.
		
		The hexagonal torus topology was chosen over a more conventional network
		topology -- such as a 2D or 3D torus (sometimes known as a 2-ary $N$-cube
		or 3-ary $N$-cube respectively) \cite[chapters~3~and~5]{dally04} -- due to
		its balance of theoretical performance and practicality. The bisection
		bandwidth of a topology indicates the theoretical worst-case total
		throughput the network is able to sustain \cite[chapter~1]{dally04}.  In
		networks with homogeneous link throughput, bisection bandwidth is
		determined by the number of links cut by a balanced bisection of the
		network.  Figure~\ref{fig:bisection-bandwidth} illustrates the bisections
		of several torus topologies.
		
		\begin{figure}
			\center
			\begin{subfigure}[b]{0.3\linewidth}
				\center
				\buildfig{figures/bisection-bandwidth-2d.tex}
				
				\caption{2D Torus}
				\label{fig:bisection-bandwidth-2d}
			\end{subfigure}
			\begin{subfigure}[b]{0.3\linewidth}
				\center
				\buildfig{figures/bisection-bandwidth-hex.tex}
				
				\caption{Hexagonal Torus}
				\label{fig:bisection-bandwidth-hex}
			\end{subfigure}
			\begin{subfigure}[b]{0.3\linewidth}
				\center
				\buildfig{figures/bisection-bandwidth-3d.tex}
				
				\caption{3D Torus}
				\label{fig:bisection-bandwidth-3d}
			\end{subfigure}
			
			\caption[Bisections of torus topologies.]%
			{Bisections of torus topologies. Connections cut by the bisection
			are drawn as lines.}
			\label{fig:bisection-bandwidth}
		\end{figure}
		
		In a $N \times N$ 2D torus topology, the bisection bandwidth is $2N$~links
		and each node requires four links. The hexagonal torus topology requires
		six links per node but provides double bisection bandwidth ($4N$~links).
		The 3D torus topology also requires six links per node but by connecting
		the nodes differently achieves a bisection bandwidth of $8N$~links.  The 3D
		torus topology, however, comes at a price -- when immersed into the
		(approximately) 2D space provided by a large machine room or row of server
		cabinets, some connections require long cables. By contrast, the 2D and
		hexagonal torus topologies are both inherently two dimensional and
		consequently do not suffer from this effect. The hexagonal torus topology,
		therefore, shares the practicality of construction of a 2D torus while
		still gaining some of the performance of a 3D torus topology. In addition,
		because nodes in a hexagonal torus topology have a greater number of links,
		greater redundancy is available in the network to tolerate faults.
		
		Most torus topologies, including hexagonal, 2D and 3D toruses, have a
		related `mesh' topology. These mesh topologies maintain the same general
		connectivity structure as their torus topologies but omit wrap-around
		links. In practice, this saves a small number of links at the expense of
		halving the network's bisection bandwidth.  Because of their poorer
		performance, mesh networks are rarely used as the basis of a network
		architecture. Mesh networks, however, are occasionally formed when a
		network is partitioned into several smaller sub-networks to allow multiple
		users to share a system \cite{spalloc16}.
		
		\begin{figure}
			\center
			\begin{subfigure}[b]{0.45\linewidth}
				\center
				\buildfig{figures/hexagonal-torus.tex}
				\caption{Hexagonal torus}
				\label{fig:topo-compare-hexagonal-torus}
			\end{subfigure}
			\begin{subfigure}[b]{0.45\linewidth}
				\center
				\buildfig{figures/h-torus.tex}
				\caption{H-torus}
				\label{fig:topo-compare-h-torus}
			\end{subfigure}
			
			\caption[Hexagonal torus vs. H-torus topology.]%
			{Hexagonal torus vs. H-torus topology. Each numbered hexagon
			represents a node. The thick outline indicates the bounds of the
			topology after which the network repeats. In each topology, the path
			taken by advancing in the Y$^+$ direction from the node labelled `0' is
			shown.}
			\label{fig:topo-compare}
		\end{figure}
		
		\label{sec:hex-vs-h-torus}
		
		The hexagonal torus topology is not to be confused with the `H-torus'
		topology. This topology also uses a hexagonal tiling of nodes and even
		wraps this tiling into a torus-like topology \cite{zhao08}. However,
		H-torus topologies have very different characteristics to the hexagonal
		torus topology and are related to `twisted torus' topologies
		\cite{camara10}. For example, figure~\ref{fig:topo-compare} illustrates one
		major difference in the way paths wrap around the peripheries of both
		topologies.
	
	\section{Scaling-up SpiNNaker machines}
		
		To build large SpiNNaker systems comprising of tens of thousands of
		SpiNNaker chips, groups of forty-eight chips are mounted onto circuit
		boards as illustrated in figure~\ref{fig:spinnakerBoard}. These boards may
		be connected together to form larger systems.  Figure~\ref{fig:threeboard}
		shows a prototype three board system. Though the chips are
		\emph{physically} arranged in a (nearly) $7\times7$ grid on each SpiNNaker
		board, they logically form a hexagonal `wrapped triple'
		\cite{davidsonWiring} (see appendix~\ref{sec:partitioning}) which logically
		fit together as illustrated in figure~\ref{fig:threeboard-separate}. The
		labelled exposed corners of the three forty-eight chip boards connect
		together to form a $12\times12$ hexagonal torus topology as illustrated in
		figure~\ref{fig:threeboard-wrapped}. Larger SpiNNaker machines are
		assembled by combining more boards.
		
		\begin{figure}
			\center
			\begin{subfigure}[b]{0.45\linewidth}
				\center
				\includegraphics[width=\linewidth]{figures/spinnakerBoard.jpg}
				
				\caption{A SpiNNaker board}
				\label{fig:spinnakerBoard}
			\end{subfigure}
			~~~
			\begin{subfigure}[b]{0.45\linewidth}
				\center
				\includegraphics[width=\linewidth]{figures/threeboard.jpg}
				
				\caption{Three board prototype}
				\label{fig:threeboard}
			\end{subfigure}
			
			\vspace*{1em}
			
			\begin{subfigure}[b]{0.45\linewidth}
				\center
				\buildfig{figures/threeboard-separate.tex}
				
				\caption{Three board topology}
				\label{fig:threeboard-separate}
			\end{subfigure}
			~~~
			\begin{subfigure}[b]{0.45\linewidth}
				\center
				\buildfig{figures/threeboard-wrapped.tex}
				
				\caption{\ldots{}as a parallelogram}
				\label{fig:threeboard-wrapped}
			\end{subfigure}
			
			\caption{SpiNNaker boards and their topology.}
			\label{fig:spinnaker-boards}
		\end{figure}
		
		
		SpiNNaker chips on the same circuit board connect using low power links
		requiring sixteen wires each.  If this link technology were used to connect
		chips on neighbouring boards, each pair of boards would need to be
		connected with a 128~wire cable.  Cables and connectors supporting this
		many signals are expensive, unreliable and physically large. Instead,
		chip-to-chip connections between boards are multiplexed and demultiplexed
		onto a single High-Speed Serial (HSS) link \cite{athavale05} carried via
		commodity S-ATA cables which are often used to connect hard disks in
		desktop computers and servers \cite{sata3spec}. The six high-speed links
		are implemented by three onboard FPGAs (the three large chips at the top of
		the SpiNNaker board) and are logically transparent to the underlying
		network. The underlying technology and the choice of S-ATA cables limits
		each board-to-board connection to spanning at most one metre gaps. In
		chapter~\ref{sec:building} I present a cabling scheme for hexagonal torus
		topologies which enable large SpiNNaker systems to be assembled using only
		short cables between boards.
		
	\section{Conclusions}
		
		The SpiNNaker architecture has been designed to enable the simulation of
		large biologically realistic neural models in real time. To support this,
		its network architecture takes on an unconventional design based on a
		custom router and hexagonal torus topology. In the remainder of this
		thesis, I will tackle a number of the challenges in scaling up the
		SpiNNaker architecture outlined in this chapter.

	\chapter{Building large SpiNNaker machines}
	
	Like any super computer, physically putting together a large SpiNNaker
	machine poses many challenges in terms of organisation, assembly and
	maintainance. One of the key tasks in this process is the installation of
	network cables such that a desired overall network topology is constructed.
	The largest planned SpiNNaker machine will use \num{3600} S-ATA
	\cite{sata3spec} cables to interconnect its \num{1200} circuit boards,
	creating a hexagonal torus topology. Since the machine will be installed
	within standard server room cabinets (which are not available in a
	giant-doughnut form-factor) a mapping from a board's logical location in the
	network topology to its physical location must be constructed. In addition,
	the interconnect technology employed by SpiNNaker restricts the length of
	S-ATA cables used to $\le$~\SI{1}{\meter}, constraining the possible mappings
	used. In addition the practical issues of installation complexity and
	maintainance must be considered since all \num{3600} cables must ultimately
	be installed and maintained by human operators.
	
	In this chapter I describe a novel technique for physically laying out
	machines configured in hexagonal torus topologies, such as SpiNNaker, in
	commercial machine rooms, building on the techniques used in more
	conventional torus topologies. In addition, I also propose a new methodology
	for installing and maintaining super computer cabling which which exploits
	existing diagnostic features of the SpiNNaker hardware to interactively guide
	and validate cable installation. Finally, I demonstrate how these new
	techniques have been used successfully to interconnect a prototype
	\num{518400} core SpiNNaker machine in substantially less time than the
	industry norm.
	
	In this chapter, the term \emph{unit} refers to the smallest physical
	ecomponent between which connections connections are to be made. For example,
	in a SpiNNaker machine a unit is a 48-chip board while in data center, a unit
	might be a server blade.
	
	\section{Related work}
		
		In this section I describe the techniques conventionally employed when
		laying out and interconnecting the units within super computers. Due to
		SpiNNaker's hexagonal torus topology and dense physical packing of units,
		these existing techniques are found to be insufficient. In the remainder of
		the chapter we will explore solutions to the limitations exposed below.
		
		\subsection{Avoiding long cables}
			
			Na\"ive arrangements of torus topologies, including hexagonal torus
			topologies, feature long `wrap-around' connections which connect units at
			the peripheries of the system. These connections can be problematic for
			numerous reasons:
			
			\begin{description}
				
				\item[Performance] Signal quality diminishes as cables get longer,
				requiring the use of slower signalling speeds, increased error
				correction overhead or more complex hardware.
				
				\item[Energy] Longer cables require higher drive strengths and/or
				buffering to maintain signal integrity.
				
				\item[Cost] Cost Shorter cables are cheaper than long ones.  Longer
				cables imply more wire in a given space making the tasks of routing or
				cable installation more difficult increasing labour costs by as much as
				$5\times$ \cite{curtis12}.
				
			\end{description}
			
			In conventional torus topologies the need for long cables is eliminated
			by folding and interleaving units of the network \cite{dally04}. For
			example, for a 1D torus topology (a ring network), one long connection
			exists to connect the two opposite sides of the system. To remove these
			long connections, half the units are `folded' on top of the others and
			then this arrangement of units is interleaved as illustrated in figure
			\ref{fig:ring-folding}.
			
			\begin{figure}
				\center
				\begin{subfigure}[b]{0.39\linewidth}
					\center
					\buildfig{figures/ring-folding-row.tex}
					\caption{A ring network}
					\label{fig:ring-folding-row}
				\end{subfigure}
				\begin{subfigure}[b]{0.24\linewidth}
					\center
					\buildfig{figures/ring-folding-folded.tex}
					\caption{Folded}
					\label{fig:ring-folding-folded}
				\end{subfigure}
				\begin{subfigure}[b]{0.35\linewidth}
					\center
					\buildfig{figures/ring-folding-interleaved.tex}
					\caption{Folded and interleaved}
					\label{fig:ring-folding-interleaved}
				\end{subfigure}
				
				\caption{Folding and interleaving a ring network to reduce maximum wire
				length.}
				\label{fig:ring-folding}
			\end{figure}
			
			Folding and interleaving has the effect of approximately doubling the
			average cable length but also eliminates the need for a cable spanning
			the entire system. Since the mean cable length is typically already
			short, doubling it in exchange for a substantially reduced maximum cable
			length is often preferable.
			
			The folding and interleaving process may be extended to $N$-dimensional
			torus topologies by folding each dimension in turn. Since all dimensions
			are orthogonal, the folding process only moves units in the dimension
			being folded. In the hexagonal torus topology, however, the three
			dimensions are non-orthogonal and thus folding in one dimension also
			moves units in other dimensions, preventing the edges of the torus
			meeting as illustrated in figure \ref{fig:failing-to-fold-hex-toruses}.
			
			\begin{figure}
				\center
				\begin{subfigure}[b]{0.24\linewidth}
					\center
					\buildfig{figures/failing-to-fold-hex-toruses-none.tex}
					\caption{Not folded}
					\label{fig:failing-to-fold-hex-toruses-none}
				\end{subfigure}
				\begin{subfigure}[b]{0.24\linewidth}
					\center
					\buildfig{figures/failing-to-fold-hex-toruses-x.tex}
					\caption{X}
					\label{fig:failing-to-fold-hex-toruses-x}
				\end{subfigure}
				\begin{subfigure}[b]{0.24\linewidth}
					\center
					\buildfig{figures/failing-to-fold-hex-toruses-y.tex}
					\caption{Y}
					\label{fig:failing-to-fold-hex-toruses-y}
				\end{subfigure}
				\begin{subfigure}[b]{0.24\linewidth}
					\center
					\buildfig{figures/failing-to-fold-hex-toruses-z.tex}
					\caption{Z}
					\label{fig:failing-to-fold-hex-toruses-z}
				\end{subfigure}
				
				\caption{Schematics showing hexagonal torus topologies folded along
				each of their non-orthogonal dimensions. Note that folding along
				the Z axis brings the \emph{wrong} edges closer together.}
				\label{fig:failing-to-fold-hex-toruses}
			\end{figure}
		
		\subsection{Cabling installation}
			
			Existing machine room installations feature very repetitive cabling
			patterns which can easily be memorised by a human technician. For example
			in BlueGene super computers the connectivity between units is highly
			regular \cite{lakner07} while in data centre networks cabling often
			centres around a small number of high-port-count switches
			\cite{cisco07,csernai15}. Cable installation is usually only aided by
			the labelling of connectors and sockets in a standardised manner
			\cite{tia2006} such as in figure \ref{fig:bgWiring}.
			
			\begin{figure}
				\center
				\begin{subfigure}[t]{0.5\textwidth}
					\begin{tikzpicture}
						\node (cables) [inner sep=0]
						      {\includegraphics[width=\textwidth]{figures/bgCables.png}};
						\node (sockets) [inner sep=0, below=1.0em of cables]
						      {\includegraphics[width=\textwidth]{figures/bgSockets.png}};
						
						% Point at label on cable
						\draw [white, <-, line width=0.4em]
						      ([shift={(0.7cm, -0.3cm)}]cables.center)
						      -- ++(45:1cm);
						
						% Point at label on socket
						\draw [white, <-, line width=0.4em]
						      ([shift={(-1.0cm, 1.1cm)}]sockets.center)
						      -- ++(-45:1cm);
					\end{tikzpicture}
					
					\caption{Pre-labelled cables and sockets}
					\label{fig:bgWiringLabels}
				\end{subfigure}
				~
				\begin{subfigure}[t]{0.30\textwidth}
					\includegraphics[height=6.15cm]{figures/bgWiring.jpg}
					
					\caption{Installation of cables}
					\label{fig:bgWiringInstallation}
				\end{subfigure}
				
				\caption{BlueGene/Q cable installation \cite{cscs13}}
				\label{fig:bgWiring}
			\end{figure}
			
			Despite the regularity and careful labelling of cables, the cost of
			installation and maintenance alone can be significant with costs in the
			range of \$45-95 per \SI{1}{\meter} cable run and \$100-400 for runs of
			\SI{10}{\meter} reported in the literature \cite{mudigonda11}. Much of
			this cost is due to the care required during installation to avoid
			miswiring and ensure that cooling airflow is not hampered by cable runs
			\cite{cisco07}.
			
			Many researchers have attempted to control cable installation costs by
			trying to reduce the number or length of cables required by developing
			alternative network topologies \cite{curtis12, popa10, mudigonda11}.
			Unfortunately, these techniques do not apply to SpiNNaker since its
			network topology is fixed.
			
			Some super computers make use of large custom `midplane` PCBs in place of
			cables to interconnect units within a cabinet and thus simplify the task
			of cable installation \cite{prickett10}. This scheme can greatly reduce
			wiring complexity since only coarser-grain cabinet-to-cabinet
			connectivity is provided by cables. Unfortunately this technique is
			expensive and also constrains the dimensions of the network topology
			supported by the machine. Since the SpiNNaker platform is designed to
			scale from desktop machines to machine-room installations, this scheme is
			not practical.
	
	\section{Folding \& interleaving hexagonal toruses}
		
		The first step towards a practical machine-room installation of a large
		machine using a hexagonal torus topology is to find an arrangement of
		boards between which cable lengths are minimised. In this section I
		describe a sequence of transformations which map the positions of units in
		a hexagonal torus topology onto a regular rectangular grid which may be
		folded and interleaved to eliminate long wires. It is worth emphasising
		that this transformation only affects the \emph{physical} positions of
		units and \emph{not} their connectivity.
		
		As described earlier in \S\ref{sec:parititioning} (page
		\pageref{sec:parititioning}), hexagonal torus topologies may be partitioned
		into units containing wrapped-triples of nodes. For example, in SpiNNaker,
		chips (nodes) are partitioned into circuit boards (units) containing 48
		chips. For completeness, this section describes the process of folding both
		systems whose units are individual nodes and those whose units are
		wrapped-triples.
		
		The transformation process is divided into two parts, each described
		separately in this section.
		
		\begin{description}
			
			\item[Parallelogram to rectangle] Units of the system are transformed
			from a parallelogram shape to a rectangular shape.
			
			\item[Uncrinkle] Units within the rectangle are moved such that they all
			lie on a regular (and fully packed) 2D grid.
			
		\end{description}
		
		\subsection{Parallelogram to rectangle}
			
			The hexagonal torus topology is most naturally drawn as a parallelogram
			as illustrated in figures \ref{fig:hex-to-plane-node-native} and
			\ref{fig:hex-to-plane-native}. Two transformations are presented which
			transform these arangements of units into a rectangular form: shearing
			and slicing.
			
			A \SI{30}{\degree} shear transformation distorts networks such that the X
			and Y axes become orthogonal leading to a rectangular arrangement of
			units as illustrated in figures \ref{fig:hex-to-plane-node-shear} and
			\ref{fig:hex-to-plane-shear}.
			
			The slice transformation slices the units protruding from the
			left-hand-side of the parallelogram and moves them into the matching gap
			on the opposite side of the parallelogram as illustrated in figures
			\ref{fig:hex-to-plane-node-slice} and \ref{fig:hex-to-plane-slice}.
			 
			While the shear transformation introduces some distortion causing cables
			in the Z dimension to become $\sqrt{2}\times$ longer it leaves the
			pattern of wrap-around connections remains unchanged. By contrast, the
			slice transformation does not elongate any cables but changes the pattern
			of wrap-around connections. The exact pattern wrap-around connections
			produced when slicing depends on the aspect ratio of the network as
			illustrated in \ref{fig:slicing-examples} and influences the choice of
			folding technique applied as described later.
			
			\begin{figure}
				\center
				\begin{subfigure}[b]{0.32\linewidth}
					\center
					\buildfig{figures/hex-to-plane-node-native.tex}
					
					\caption{$7 \times 7$ node torus}
					\label{fig:hex-to-plane-node-native}
				\end{subfigure}
				\begin{subfigure}[b]{0.32\linewidth}
					\center
					\buildfig{figures/hex-to-plane-node-shear.tex}
					
					\caption{Sheared}
					\label{fig:hex-to-plane-node-shear}
				\end{subfigure}
				\begin{subfigure}[b]{0.32\linewidth}
					\center
					\buildfig{figures/hex-to-plane-node-slice.tex}
					
					\caption{Sliced}
					\label{fig:hex-to-plane-node-slice}
				\end{subfigure}
				
				\caption{Transformations of hexagonal toruses of nodes into a
				rectangular form. Thin lines show wrap-around links. Pointy-topped
				hexagons represent individual nodes.}
				\label{fig:hex-to-plane-node}
			\end{figure}
			
			\begin{figure}
				
				\begin{subfigure}[b]{0.32\linewidth}
					\center
					\buildfig{figures/hex-to-plane-native.tex}
					
					\caption{$4 \times 4$ triad torus}
					\label{fig:hex-to-plane-native}
				\end{subfigure}
				\begin{subfigure}[b]{0.32\linewidth}
					\center
					\buildfig{figures/hex-to-plane-shear.tex}
					
					\caption{Sheared}
					\label{fig:hex-to-plane-shear}
				\end{subfigure}
				\begin{subfigure}[b]{0.32\linewidth}
					\center
					\buildfig{figures/hex-to-plane-slice.tex}
					
					\caption{Sliced}
					\label{fig:hex-to-plane-slice}
				\end{subfigure}
				
				\caption{Transformations of hexagonal toruses of wrapped triples into a
				rectangular form.  Thin lines show wrap-around links. Flat-topped
				hexagons represent a wrapped triple of nodes.}
				\label{fig:hex-to-plane}
			\end{figure}
			
			\begin{figure}
				\center
				\buildfig{figures/slicing-examples.tex}
				\caption{Patterns of wiring in sliced systems of various sizes.}
				\label{fig:slicing-examples}
			\end{figure}
			
		\subsection{Uncrinkling}
			
			Though the transformmation step yields rectangular arrangements of units,
			these arrangements do not fall onto a regular 2D grid, with the exception
			of the shear transform on individual nodes. Figure \ref{fig:uncrinkling}
			illustrates how the various arrangements of hexagons may be moved to
			`uncrinkle' the units into a regular grid.
			
			\begin{figure}
				\center
				\begin{subfigure}[b]{0.44\linewidth}
					\center
					\buildfig{figures/uncrinkling-node-sheared.tex}
					
					\caption{$7 \times 7$ nodes, sheared}
					\label{fig:uncrinkling-node-sheared}
				\end{subfigure}
				\begin{subfigure}[b]{0.44\linewidth}
					\center
					\buildfig{figures/uncrinkling-node-sliced.tex}
					
					\caption{$7 \times 7$ nodes, sliced}
					\label{fig:uncrinkling-node-sliced}
				\end{subfigure}
				
				\vspace{1cm}
				
				\begin{subfigure}[b]{0.44\linewidth}
					\center
					\buildfig{figures/uncrinkling-sheared.tex}
					
					\caption{$4 \times 4$ triples, sheared}
					\label{fig:uncrinkling-sheared}
				\end{subfigure}
				\begin{subfigure}[b]{0.44\linewidth}
					\center
					\buildfig{figures/uncrinkling-sliced.tex}
					
					\caption{$4 \times 4$ triples, sliced}
					\label{fig:uncrinkling-sliced}
				\end{subfigure}
				
				\vspace{1em}
				
				\caption{Mapping rectangular arrangements of units into a square grid.
				Thick lines show how layers of units are uncrinkled.  Annotations show
				how the relative positions of nodes and wrapped triples change after
				uncrinkling.}
				\label{fig:uncrinkling}
			\end{figure}
			
			In the figure, the numbered units enumerate the different positions on
			the crinkle and those labelled alphabetically are those that immediately
			surround them. From this we can observe that uncrinkling largely
			preserves spatial locality but some distortion is introduced, separating
			previously neighbouring units. For example, in figure
			\ref{fig:uncrinkling-sheared}, the units labelled `1' and `i' are
			neighbours before uncrinkling but are separated by a (Euclidean) distance
			of $\sqrt{5}$ afterwards. Note that the distortion introduced depends on
			what part of the crinkle is considered, for example `2' and `a' have
			distance 2 but are logically connected in the same way.
		
		\subsection{Folding and Interleaving}
			
			Once a regular grid of units has been formed, this may be folded in the
			conventional way, eliminating long cables crossing from left-to-right and
			top-to-bottom as illustrated in \ref{fig:folding-sheared}.
			
			Unfortunately, for sliced systems whose dimensions are not of the ratio
			$1:2$, the pattern of wrap-around cables may also include some cables
			which do not cross directly to the opposite side of the system (refer
			back to figure \ref{fig:slicing-examples}). As a result of these
			connections, folding does not successfully eliminate all long
			connections. An exception to this rule is sliced systems whose dimensions
			are in the ratio $1:1$ where folding twice along the Y axis may
			successfully eliminate all wrap-around connections as illustrated in
			\ref{fig:folding-sliced}.
			
			\begin{figure}
				\begin{subfigure}{\linewidth}
					\center
					\buildfig{figures/folding-sheared.tex}
					\caption{$N \times M$ sheared systems and $N \times 2N$ sliced systems}
					\label{fig:folding-sheared}
				\end{subfigure}
				
				\vspace{1em}
				
				\begin{subfigure}{\linewidth}
					\center
					\buildfig{figures/folding-sliced.tex}
					\caption{$N \times N$ sliced systems}
					\label{fig:folding-sliced}
				\end{subfigure}
				
				\caption{Schematic illustrating elimination of long wrap-around links
				during folding. In each example a single link has been highlighted to
				aid in following the process.}
				\label{fig:folding}
			\end{figure}
			
			Once folded, the 2D grid is straight-forwardly interleaved as illustrated
			previously in figure \ref{fig:ring-folding}. The interleaving process
			introduces some additional distortion to the layout of units and causes
			most connections to become twice as long. For sliced $1:1$ systems, the
			additional fold results in additional overhead during interleaving since
			four layers of the system are interleaved.
		
		\subsection{Mapping to Cabinets}
			
			In the final step of the process is to map the 2D grid of units into
			positions in machine room cabinets as illustrated in figure
			\ref{fig:million-core-machine}. As illustrated in figure
			\ref{fig:cabinetisation}, first the grid of units is partitioned into
			groups of columns, one per cabinet, then groups of rows one per frame per
			cabinet. The units in each group are then allocated to slots within a
			frame, interleaving the rows of the groups as shown.
			
			\begin{figure}
				\center
				\buildfig{figures/cabinet-units.tex}
				
				\caption{An illustration of the physical construction of a
				multi-cabinet SpiNNaker system. (Note: network cables \emph{not}
				installed.)}
				\label{fig:cabinet-units}
			\end{figure}
			
			\begin{figure}
				\center
				\buildfig{figures/cabinetisation.tex}
				
				\caption{Mapping from 2D space to cabinets, frames and boards.}
				\label{fig:cabinetisation}
			\end{figure}
		
	\section{Cable installation}
		
		Cable installation is performed by a team of (human) technicians who must
		ensure that all network cables are correctly installed. As illustrated in
		previously in figure \ref{fig:cabinet-units}, the density of SpiNNaker's
		units, combined with the nature of the hexagonal torus topology, poses a
		challenge. To address this challenge I propose a semi-automated approach to
		cable installation which exploits diagnostic facilities available in the
		majority of super computers in order to guide technicians through the
		cabling process, interactively guiding installation and maintenance.
		
		\subsection{Interactive technician guidance and validation}
			
			While automated systems for validating cabling correctness are
			commonplace, these systems are typically used only after cabling has been
			completed \cite{lakner07}. As with other large-scale machines, SpiNNaker
			includes a low-bandwidth system management bus which may be used to
			interrogate network hardware and control diagnostic LEDs prior to the
			installation of the main SpiNNaker network interconnect.  Using these
			facilities I have constructed a tool called SpiNNer which interactively
			guides a technician, or team of technicians, through the cable
			installation process, validating each connection in real-time.
			
			Diagnostic LEDs mounted on each SpiNNaker board (figure
			\ref{fig:interactive-wiring-guide-leds}) are used to indicate the
			endpoints of the cable currently being installed. Simultaneously a
			Text-To-Speech (TTS) system gives an audible indication of which cable
			type is to be used and location of each connection.  Additionally, a GUI
			via a computer display (figure \ref{fig:interactive-wiring-guide-gui}).
			The centre of the display shows a `big-picture' perspective of the
			locations of the boards to be connected. The detailed views on the left
			and right indicate which of the six sockets on each board the cables
			should connect.
			
			\begin{figure}
				\center
				\begin{subfigure}[b]{0.40\textwidth}
					\begin{tikzpicture}
						\node (leds) [inner sep=0]
						      {\includegraphics[width=\textwidth]{figures/leds.jpg}};
						% Point at left LED
						\draw [white, <-, line width=0.4em]
						      ([shift={(-0.0cm, -0.6cm)}]leds.center)
						      -- ++(225:1cm);
						% Point at right LED
						\draw [white, <-, line width=0.4em]
						      ([shift={(1.1cm, -1.1cm)}]leds.center)
						      -- ++(225:1cm);
					\end{tikzpicture}
					
					\caption{Diagnostic LEDs}
					\label{fig:interactive-wiring-guide-leds}
				\end{subfigure}
				~
				\begin{subfigure}[b]{0.546\textwidth}
					\begin{tikzpicture}[thin, black!20!white]
						\node (screen) [inner sep=0]
						      {\includegraphics[width=\textwidth]{figures/wiring_guide_screenshot.png}};
						\draw (screen.south west) rectangle (screen.north east);
					\end{tikzpicture}
					
					\caption{Interactive wiring guide GUI}
					\label{fig:interactive-wiring-guide-gui}
				\end{subfigure}
				
				\caption{The SpiNNer interactive wiring guide uses a GUI,
				text-to-speech and diagnostic LEDs to assist during cable
				installation.}
				\label{fig:interactive-wiring-guide}
			\end{figure}
			
			SpiNNer also validates the connectivity of the system in real-time by
			polling the diagnostic interfaces of the network hardware at the
			endpoints of the cable being installed to determine if they are correctly
			connected. If a miswiring occurs, this is immediately detected and
			announced via TTS enabling the technician to immediately correct the
			error. Once a cable has been installed correctly, the software
			automatically advances to the next cable meaning direct interaction with
			the software by the technician is minimal. In practice, it is rarely
			necessary to refer to the GUI.
		
			SpiNNer presents the cables in an order intended to maximise ease of
			installation. Cables are installed in three groups with intra-frame
			cables being installed first, followed by intra-cabinet cables and
			inter-cabinet cables. Within each group, the tightest cables are
			installed first resulting in slacker cables naturally being installed
			over the top of already installed cables. By grouping cables in this
			manner, multiple technicians may work independently on the wiring within
			individual frames and cabinets.
			
			SpiNNer may also be used to repair or replace cables in the system.
			During maintenance, obstructing cables may be blindly removed alongside
			any cable being replaced. At the conclusion of the process, the wiring
			guide may be used to interactively guide re-installation of all removed
			cables.
		
		\subsection{Cable selection}
			
			Controlling slack is critical to ensuring reliable and maintainable
			cabling installations. If cables are too tight, cables and connectors can
			become easily damaged and when too slack, the excess cable obstructs
			other cables and can easily become tangled and damaged \cite{cisco07}. It
			has been observed that when ready-made cables are employed technicians
			frequently over-estimate the cable lengths required preferring to use
			overly long cables for all connections \cite{mazaris97}. To solve this
			problem, the SpiNNer wiring guide software dictates the cable lengths to
			be used by an installer based the rule of (three-)thumbs according to
			Mazaris \cite{mazaris97}. This rule suggests that an ideal amount of
			slack is approximately that which can be wrapped around three fingers.
			Specifically, the shortest available cable is selected which ensures at
			least \SI{5}{\centi\meter} of slack.
			
			The SpiNNer tool allocates cables assuming all cables take a Euclidean
			straight-line path between the endpoints of the connection. The result is
			that wiring is not routed through dedicated cable management structures
			but is simply suspended by its connectors in front of the machine. As
			demonstrated later, this unconventional approach leads neither to cooling
			problems nor increased maintenance effort.
	
	\section{Results and Evaluation}
		
		This stuff has been used and works. In this section I'll go over the
		overheads introduced by the various transformations and
		folding/interleaving steps and show a wiring scheme for a large machine
		which uses only short cables. I'll then show how SpiNNer was used to
		install this wiring plan into a very large machine without foobaring the
		cooling and in very little time. I'll also report on difficulty of
		maintenance.
		
		\subsection{Cable length}
			
			The transformation from regular hexagonal torus to a folded and
			interleaved form introduces some overhead to the cable lengths required.
			Using figure \ref{fig:uncrinkling} (page \pageref{fig:uncrinkling}), it
			is possible to compute the exact overhead introduced when each type of
			transformation proposed.
			
			For example, to compute the mean overhead introduced by the slicing
			technique when applied to units composed of wrapped triples, consider
			figure \ref{fig:uncrinkling-sliced}. The uncrinkling pattern used to
			transform this topology is a repeating pattern of two units, a pair of
			which have been labelled $1$ and $2$ respectively. Unit $1$ is
			immediately surrounded by six units labelled $a$, $b$, $c$, $2$, $g$ and
			$h$. Similarly, unit $2$ is surrounded by units $1$, $c$, $d$, $e$, $f$
			and $g$. Before the transformation, the distances, $D$, to each of these
			units is $1$ but after the transformation is applied, this is not always
			the case. Additionally, folding and interleaving introduce additional
			overhead. In this example, if the system is folded into $f_x$ columns and
			$f_y$ rows, the distances between previously neighbouring units become:
			
			\begin{equation*}
				\begin{aligned}[c]
					D_{1\,\leftrightarrow{}\,a} &= \sqrt{f_x^2 + f_y^2} \\
					D_{1\,\leftrightarrow{}\,b} &= f_y \\
					D_{1\,\leftrightarrow{}\,c} &= \sqrt{f_x^2 + f_y^2} \\
					D_{1\,\leftrightarrow{}\,2} &= f_x \\
					D_{1\,\leftrightarrow{}\,g} &= f_y \\
					D_{1\,\leftrightarrow{}\,h} &= f_x
				\end{aligned}
				\hspace{2cm}
				\begin{aligned}[c]
					D_{2\,\leftrightarrow{}\,1} &= f_x \\
					D_{2\,\leftrightarrow{}\,c} &= f_y \\
					D_{2\,\leftrightarrow{}\,d} &= f_x \\
					D_{2\,\leftrightarrow{}\,e} &= \sqrt{f_x^2 + f_y^2} \\
					D_{2\,\leftrightarrow{}\,f} &= f_y \\
					D_{2\,\leftrightarrow{}\,g} &= \sqrt{f_x^2 + f_y^2}
				\end{aligned}
			\end{equation*}
			
			From these values, the mean and maximum connection distances after
			folding and interleaving may be computed. Table
			\ref{tab:transform-overhead} gives the mean and maximum connection
			distances for each of the four transformations described in this chapter.
			
			\begin{table}
				\begin{subtable}[b]{\linewidth}
					\center
					\begin{tabular}{l c c}
						\toprule
						& Shear & Slice \\
						\addlinespace
						Nodes &
							$\frac{f_x + f_y + \sqrt{f_x^2 + f_y^2}}{3}$ &
							$\frac{f_x + f_y + \sqrt{f_x^2 + f_y^2}}{3}$ \\
						\addlinespace
						Triples &
							$\frac{7f_x + 3\sqrt{f_x^2 + f_y^2} + \sqrt{(2f_x)^2 + f_y^2}}{9}$ &
							$\frac{f_x + f_y + \sqrt{f_x^2 + f_y^2}}{3}$ \\
						\bottomrule
					\end{tabular}
					
					\caption{Mean}
					\label{tab:transform-overhead-mean}
				\end{subtable}
				
				\vspace{1em}
				
				\begin{subtable}[b]{\linewidth}
					\center
					\begin{tabular}{l c c}
						\toprule
						& Shear & Slice \\
						\addlinespace
						Nodes &
							$\sqrt{f_x^2 + f_y^2}$ &
							$\sqrt{f_x^2 + f_y^2}$ \\
						\addlinespace
						Triples &
							$\sqrt{(2f_x)^2 + f_y^2}$ &
							$\sqrt{f_x^2 + f_y^2}$ \\
						\bottomrule
					\end{tabular}
					
					\caption{Maximum}
					\label{tab:transform-overhead-max}
				\end{subtable}
				
				\caption{Overheads introduced when transforming unit positions onto a
				regular grid.}
				\label{tab:transform-overhead}
			\end{table}
			
			From these results it is evident that shearing and slicing networks
			whose units are nodes result in identical mean and maximum overhead in
			cable length when folded similarly. Since sliced networks may require
			folding more than once along each axis the shearing approach is
			preferable in general.
			
			For networks constructed from units of wrapped triples, the slicing
			approach suffers the same mean and maximum overhead has networks of
			nodes, and less overhead than shearing for the same number of folds. For
			systems with an aspect ratio of $1:2$ (where both slicing and shearing
			require $f_x = f_y = 2$), the slicing transformation yields lower mean
			and maximum overhead than shearing. For all other aspect ratios (where
			slicing requires a greater degree of folding) the shearing technique
			produces lower overhead. The recommended transformations for a given
			machine are thus given in table \ref{tab:transform-recommended}.
			
			\begin{table}
				\center
				\begin{tabular}{lcc}
					\toprule
					                         & $1:2$  & Other \\
					\addlinespace
					\multirow{2}{*}{Nodes}   & Either & Shear\\
					                         & \footnotesize $\mu\approx2.28 \quad \vee\approx2.83$
					                         & \footnotesize $\mu\approx2.28 \quad \vee\approx2.83$\\
					\addlinespace
					\multirow{2}{*}{Triples} & Slice  & Shear\\
					                         & \footnotesize $\mu\approx2.28 \quad \vee\approx2.83$
					                         & \footnotesize $\mu\approx3.00 \quad \vee\approx4.47$\\
					\bottomrule
				\end{tabular}
				
				\caption{Recommended transformation and folding scheme for different
				system types. $\mu$ and $\vee$ give the mean and maximum wire
				distortion introduced, respectively.}
				\label{tab:transform-recommended}
			\end{table}
			
			\begin{figure}
				\center
				\buildfig{figures/million-core-machine.tex}
				
				\caption{Cabling plan for a \num{1036800} core SpiNNaker
				machine's \num{3600} cables.}
				\label{fig:million-core-machine}
			\end{figure}
			
			Following folding and mapping to physical locations, the cabling plans
			for large machines require no large gaps to be spanned.  The largest
			planned SpiNNaker machine, illustrated in figure
			\ref{fig:million-core-machine}, will be \SI{6}{\meter} wide but the
			largest gap any cable must span is \SI{66}{\centi\meter}. This distance
			is well within the \SI{1}{\meter} allowed by the hardware and cables.
			
		\subsection{Installation practicality}
			
			\begin{table}
				\center
				\begin{tabular}{lrr@{$\,$}l}
					\toprule
						System & Number of Cables & \multicolumn{2}{r}{Installation time} \\
					\midrule
						24 boards  & \num{72}   & \num{10} & \si{\minute}         \\
						1 cabinet  & \num{360}  & \num{4}  & \si{\hour}$^\dagger$ \\
						2 cabinets & \num{720}  & \num{2}  & \si{\hour}           \\
						5 cabinets & \num{1800} & ?        &                      \\
					\bottomrule
				\end{tabular}
				
				\caption{Installation times for various sizes of machine.
				$\dagger$~This machine was installed without real-time validation of
				connectivity.}
				\label{tab:install-time}
			\end{table}
			
			A number of SpiNNaker machines of various scales have been assembled
			using the techniques described in this chapter ranging from single frames
			of 24 boards to a half-scale 5 cabinet machine. Table
			\ref{tab:install-time} gives the reported installation times of each of
			these machines.
			
			The single cabinet machine's installation time is notably
			disproportionate to its size. When this system was assembled, real-time
			connection validation was not yet available. As a result, though cable
			installation was rapid correcting errors was extremely costly, requiring
			careful retracing of many installation steps.
			
			TODO: TALK ABOUT MULTI-PERSON-WIRING IN PRACTICE ON FIVE CABINET MACHINE.
			
			\begin{figure}
				
				\center
				\buildfig{figures/wire-length-histogram.tex}
				
				\caption{Histogram of connection distances in a ten-cabinet,
				one-million core SpiNNaker machine annotated with the suggested cable
				length.}
				\label{fig:wire-length-histogram}
				
			\end{figure}
			
			FIGURE \ref{fig:wire-length-histogram} SHOWS THE DISTRIBUTION OF CABLE
			LENGTHS REQUIRED. IN PRACTICE THE SLACK ALLOCATED PROVED ADEQUATE. AS
			SHOWN IN FIGURE \ref{fig:install-histogram}, THE MOST IMPORTANT FACTOR IS
			WHETHER LEAVING THE FRAME OR NOT. LEAVING THE FRAME TAKES THE LONGEST.
			
			\begin{figure}
				\builddata{data/build_connection_log.tex}
				\buildfig{figures/install-histogram.tex}
				
				\caption{Histogram of cable installation times}
				\label{fig:install-histogram}
			\end{figure}
			
			TODO: COMPARE DIRECTLY WITH INSTALL TIMES REPORTED IN LITERATURE.
		
		\subsection{Thermal Impact}
			
			TODO: SHOW HOW TEMPERATURE IS CHANGED
			
		\subsection{Maintenance}
			
			TOOD: QUANTIFY CABLE REMOVALS REQUIRED. EXPERIMENT: REMOVE/REPLACE RANDOM
			BOARDS AND MEASURE TIME TAKEN, CABLES REMOVED. COMPARE WITH STANDARD DATA
			CENTRE WIRING

	\chapter{Finding shortest path vectors in SpiNNaker's network}
	
	Once a SpiNNaker machine has been constructed as described in the previous
	chapter, its network forms a large hexagonal torus topology. To exploit this
	network routing algorithms must be used to generate routes for packets to
	follow between nodes. As well as ensuring that packets arrive at the correct
	destination, routing algorithms often attempt to produce routes which make
	efficient use of the network. This often involves attempting to reduce
	congestion by ensuring packets do not travel further through the network than
	absolutely necessary.
	
	Many popular routing algorithms for torus topologies, including all published
	algorithms designed for SpiNNaker's hexagonal torus topology
	\cite{davies12,navaridas14}, internally function by computing shortest path
	vectors and generating routes from them. Existing methods of calculating
	shortest path vectors in hexagonal torus topologies are unable to generate
	all possible shortest path vectors and, as a result, reduces the diversity of
	routes produced by routing algorithms, potentially worsening network
	contention.
	
	In this chapter I describe a novel technique for computing shortest path
	vectors in hexagonal torus topologies which yields \emph{all} possible
	shortest path vectors for any pair of nodes. Further, implementations of this
	new technique execute an order of magnitude faster than the existing
	alternatives.
	
	\section{Related work}
		
		TODO: INTRODUCE SECTION
		
		\begin{figure}
			\center
			
			\begin{subfigure}{\linewidth}
				\center
				\buildfig{figures/distance-map-mesh.tex}
				\caption{2D mesh topology}
				\label{fig:distance-map-mesh}
			\end{subfigure}
			
			\vspace{1em}
			
			\begin{subfigure}{\linewidth}
				\center
				\buildfig{figures/distance-map-torus.tex}
				\caption{2D torus topology}
				\label{fig:distance-map-torus}
			\end{subfigure}
			
			\vspace{1em}
			
			\begin{subfigure}{\linewidth}
				\center
				\buildfig{figures/distance-map-hex-mesh.tex}
				\caption{Hexagonal mesh topology}
				\label{fig:distance-map-hex-mesh}
			\end{subfigure}
			
			\vspace{1em}
			
			\begin{subfigure}{\linewidth}
				\center
				\buildfig{figures/distance-map-hex-torus.tex}
				\caption{Hexagonal torus topology}
				\label{fig:distance-map-hex-torus}
			\end{subfigure}
			
			\caption{Plots showing distance from various locations marked
			         {\color{red}$\times$}. Darker areas are further away. Contour
			         lines show equidistant points.}
			\label{fig:distance-map}
		\end{figure}
		
		\subsection{Mesh Networks}
			
			In a (non-hexagonal) mesh network topology, shortest path vectors are
			computed by taking the element-wise difference between the source and
			destination nodes' coordinates.
			
			\begin{figure}
				\center
				\buildfig{figures/mesh-topology-coordinates.tex}
				\caption{An example 2D mesh network with example shortest-path routes
				from `A' to `B' and `B' to `C'.}
				\label{fig:mesh-topology-coordinates}
			\end{figure}
			
			For example, figure \ref{fig:mesh-topology-coordinates} illustrates a 2D
			mesh topology. In this topology, the nodes labelled `A', `B' and `C' have
			position vectors $(1, 2)$, $(4, 5)$ and $(6, 1)$ respectively. The
			shortest path vector from node `A' to `B' is thus simply $(4, 5) - (1, 2)
			= (3, 3)$ and from `B' to `C' is $(6, 1) - (4, 5) = (2, -4)$.
			
			A route may be produced from a shortest path vector by advancing the
			number of hops specified for each dimension in the vector. For example
			any permutation of the hops X$^+\,$X$^+\,$X$^+\,$Y$^+\,$Y$^+\,$Y$^+$, an
			example of which is included in the figure. Likewise a route from `B' to
			`C' may be constructed from any permutation of
			X$^+\,$X$^+\,$Y$^-\,$Y$^-\,$Y$^-\,$Y$^-$.
			
			Many popular routing algorithms such as Dimension Order Routing (DOR),
			Right-Turn Only Routing (RTOR) and Longest Dimension First Routing (LDFR)
			\cite{dally04,davies12} directly follow the above procedure and just
			prescribe a specific permutation of hop order. For example, DOR produces
			routes with X hops first, Y hops second and so on.
			
			The length of routes produced from a shortest path vector have a number
			of hops proportional to the magnitude of the vector, thus a shortest path
			vector yields a route with the minimum number of hops. For a two
			dimensional vector $(a, b)$ the magnitude is given as:
			%
			\begin{equation}
				\| (a, b) \| = \lvert a \rvert + \lvert b \rvert
			\end{equation}
		
		\subsection{Torus Networks}
			
			Computing shortest path vectors in (non-hexagonal) torus topologies is
			also straight forward. As an example, lets find the shortest path vector
			from node `A' to `B' in the 2D torus topology shown in figure
			\ref{fig:torus-shortest-path-example}. First, both nodes are translated
			such that the source node, `A', is at the centre of the network (figure
			\ref{fig:torus-shortest-path-translate}). Note that this translation may
			result in the destination node `wrapping around' the network. After
			translation, the shortest path vector is computed as in a mesh topology.
			As illustrated in \ref{fig:torus-shortest-path-routed}, the computed
			shortest path vector may be used to produce routes between the two nodes
			in their original positions.
			
			\begin{figure}
				\center
				\begin{subfigure}{0.3\linewidth}
					\center
					\buildfig{figures/torus-shortest-path-example.tex}
					\caption{Original}
					\label{fig:torus-shortest-path-example}
				\end{subfigure}
				\begin{subfigure}{0.3\linewidth}
					\center
					\buildfig{figures/torus-shortest-path-translate.tex}
					\caption{Translated}
					\label{fig:torus-shortest-path-translate}
				\end{subfigure}
				\begin{subfigure}{0.3\linewidth}
					\center
					\buildfig{figures/torus-shortest-path-routed.tex}
					\caption{Routed}
					\label{fig:torus-shortest-path-routed}
				\end{subfigure}
				
				\caption{Finding shortest paths in a 2D torus topology.}
				\label{fig:torus-shortest-path}
			\end{figure}
			
			This process works because vectors from the centre (though not other
			locations) of a torus topology are identical to those in mesh topologies
			(see figures \ref{fig:distance-map-mesh} and
			\ref{fig:distance-map-torus}).
		
		\subsection{Hexagonal Mesh Networks}
			
			In hexagonal mesh topologies it is conventional to define three `axes' X,
			Y and Z as shown in figure \ref{fig:hex-mesh-topology-coordinates}
			\cite{patel15}. In this example, the three labelled nodes `A', `B' and
			`C' may be given position vectors such as $(1, 1, 0)$, $(3, 2, 0)$ and
			$(0, 0, -7)$ respectively. As in other mesh networks, a vector between
			two nodes is found by subtracting the nodes' vectors. For example, a
			vector from `A' to `B' is $(3, 2, 0) - (1, 1, 0) = (2, 1, 0)$. This
			vector can then be converted into a route in the same way as a mesh
			network by taking any permutation of the three hops  X$^+\,$X$^+\,$Y$^+$.
			
			\begin{figure}
				\center
				\buildfig{figures/hex-mesh-topology-coordinates.tex}
				\caption{An example hexagonal mesh network topology.}
				\label{fig:hex-mesh-topology-coordinates}
			\end{figure}
			
			As explained in detail in appendix \ref{app:minimal-hex-coordinates},
			there are an infinite number of vectors between any two points. For
			example, the vectors $(1, 0, -1)$ and $(3, 2, 1)$ also reach node `B'
			from `A' in the example. However, for a given pair of nodes, there is
			always a single, unique vector whose magnitude is minimal which is
			given by the function:
			%
			\begin{equation}
				\operatorname{minimiseVector}(x,y,z)
					= (x,y,z) - \operatorname{median}(x,y,z) \cdot (1,1,1)
			\end{equation}
			%
			An important side-effect of this function is that a minimised vector will
			always contain at least one zero element meaning that shortest path
			routes will use at most two of the three available dimensions.
			
			To aid the reader's intuition, figure \ref{fig:distance-map-hex-mesh}
			illustrates how distances grow in a hexagonal mesh topology.
		
		\subsection{Hexagonal Torus Networks}
			
			Unfortunately, unlike non-hexagonal torus topologies, the translation
			technique cannot be used to compute shortest path vectors. As illustrated
			in figures \ref{fig:distance-map-hex-mesh} and
			\ref{fig:distance-map-hex-torus}, shortest path vectors from the center
			of a hexagonal mesh network are not equivalent to those of a hexagonal
			torus network.
			
			Prior research into routing in SpiNNaker's network has been based on the
			INSEE \cite{navaridas09,ghasempour15} interconnect simulator. Internally
			INSEE tries a set of twelve candidate vectors and picks the shortest as
			the shortest path vector to use for routing.
			
			\begin{figure}
				\center
				\begin{subfigure}{0.45\linewidth}
					\center
					\buildfig{figures/insee-vector-candidates-no-wrap.tex}
					\caption{$(\Delta_\textrm{X}, \Delta_\textrm{Y}) = (5,3)$}
					\label{fig:insee-vector-candidates-no-wrap}
				\end{subfigure}
				\begin{subfigure}{0.45\linewidth}
					\center
					\buildfig{figures/insee-vector-candidates-wrap-x.tex}
					\caption{$(\Delta'_\textrm{X}, \Delta_\textrm{Y}) = (-3,3)$}
					\label{fig:insee-vector-candidates-wrap-x}
				\end{subfigure}
				
				\vspace{1em}
				
				\begin{subfigure}{0.45\linewidth}
					\center
					\buildfig{figures/insee-vector-candidates-wrap-y.tex}
					\caption{$(\Delta_\textrm{X}, \Delta'_\textrm{Y}) = (5,-5)$}
					\label{fig:insee-vector-candidates-wrap-y}
				\end{subfigure}
				\begin{subfigure}{0.45\linewidth}
					\center
					\buildfig{figures/insee-vector-candidates-wrap.tex}
					\caption{$(\Delta'_\textrm{X}, \Delta'_\textrm{Y}) = (-3,-5)$}
					\label{fig:insee-vector-candidates-wrap}
				\end{subfigure}
				
				\vspace{1em}
				
				% Key
				\begin{tikzpicture}[thick]
					\coordinate (last);
					
					% #1 colour
					% #2 label
					\newcommand{\colourkeyentry}[2]{
						\node [#1] [right=of last, fill, rectangle, minimum size=1em] (last) {};
						\node [right=0 of last] (last) {#2};
					}
					
					\colourkeyentry{cb3class0}{$(\textrm{X}, \textrm{Y}, 0)$}
					\colourkeyentry{cb3class1}{$(\textrm{X} - \textrm{Y}, 0, - \textrm{Y})$}
					\colourkeyentry{cb3class2}{$(0, \textrm{Y} - \textrm{X}, - \textrm{X})$}
					
				\end{tikzpicture}
				
				\caption{The twelve candidate shortest-path vectors considered by INSEE
				represented as dimension-order routes. $W=H=8$,
				$(\Delta_\textrm{X},\Delta_\textrm{Y}) = (5, 3)$ and
				$(\Delta'_\textrm{X},\Delta'_\textrm{Y}) = (-3, -5)$.}
				\label{fig:insee-vector-candidates}
			\end{figure}
			
			The twelve vectors considered are constructed as follows.
			
			First a shortest path vector from the source to target node are
			constructed as if the network was a 2D mesh yielding a vector
			$(\Delta_\textrm{X},\Delta_\textrm{Y})$. From this, another vector
			$(\Delta'_\textrm{X},\Delta'_\textrm{Y})$, is defined:
			%
			\begin{align}
				\Delta'_\textrm{X} &= \Delta_\textrm{X} - \operatorname{sign}(\Delta_\textrm{X})W
				\\
				\Delta'_\textrm{Y} &= \Delta_\textrm{Y} - \operatorname{sign}(\Delta_\textrm{Y})H
			\end{align}
			%
			Where $W$ and $H$ are the width and height of the network respectively
			(in nodes). This new vector yields routes from the source to destination
			node that in a torus topology that \emph{always} wrap around the `X' and
			`Y' dimensions.
			
			From the pair of vectors defined, four possible 2D vectors can be
			produced: $(\Delta_\textrm{X},\Delta_\textrm{Y})$,
			$(\Delta'_\textrm{X},\Delta_\textrm{Y})$,
			$(\Delta_\textrm{X},\Delta'_\textrm{Y})$ and
			$(\Delta'_\textrm{X},\Delta'_\textrm{Y})$. Further, each 2D vector may be
			converted into one of three 3D vectors where either X, Y or Z are zero
			for a total of twelve candidate vectors.  Figure
			\ref{fig:insee-vector-candidates} illustrates all twelve candidate
			vectors for an example pair of nodes.
			
			\begin{figure}
				\center
				\buildfig{figures/xyz-protocol-regions.tex}
				
				\caption{The four regions defined by the XYZ-protocol.}
				\label{fig:xyz-protocol-regions}
			\end{figure}
			
			A more efficient technique is proposed by Hoffmann and D\'es\'erable
			called the XYZ-Protocol \cite{hoffmann15,hoffmann11}. If the source and
			destination nodes are translated such that the source node lies at the
			center of the topolgoy, the destination will lie in one of four regions
			illustrated in figure \ref{fig:xyz-protocol-regions}.
			
			If the destination lies in regions 1 or 4, a route may be constructed as
			if in a hexagonal mesh topology.
			
			Alternatively, if the destination lies in regions 2 or 3, the algorithm
			tests whether taking a mesh-like route within the region or
			wrapping-around either the X or Y dimension yields the shorter vector.
			The shortest of these vectors is then chosen.
			
			TODO DESCRIBE SPIRAL ROUTES.
			
			TODO DESCRIBE RTOR AND LDFR.
		
	\section{Dimension order routing in hexagonal torus topologies}
		
		So, existing solutions have two problems: trying 12 options and picking one
		is a bit kludgey and there are actually more options than that.
		
		\subsection{Simpler minimal hexagonal torus vectors}
			
			If you redraw your route such that it is sourced from bottom left corner
			(which we'll now call (0, 0)), there are four possible ways this route
			could wrap.
			
			TODO: DESCRIBE WAYS OF WRAPPING
			
			For each of these wrappings, all the possible routes we can take are
			strictly limited in terms of the dimensions used since we're stuck in a
			corner.
			
			In each case, the function computing the minimal hex vector function
			simplifies to a much simpler operation.
			
			TODO: DESCRIBE MINIMUM VECTOR LENGTH FUNCTIONS FOR EACH CASE
			
			This gives us a cheap way to compute which of the four possible wrappings
			are shortest. Based on this we can pick one of at most two (is this
			easily provable?) vectors in some fair manner to reduce load. This vector
			can then be minimised in the usual way.
			
			This also leads to confirming a theoretical result giving the length of a
			shortest path in a hexagonal torus topology.
			
			TODO: DESCRIBE HOW TO GET LENGTH FUNCTION AND COMPARE WITH \cite{xiao04}
		
		\subsection{Generating spiralling routes}
			
			In non-hexagonal torus topologies the previous technique would reveal all
			possible shortest vectors (e.g. in cases where you can wrap more than one
			way). Unfortunately, due to the addition of a non-orthogonal axes,
			hexagonal toruses also have other cases when the width and height do not
			match.
			
			TODO: TORUS SPIRALLING EXAMPLE
			
			It is possible to calculate the maximum number of spirals thus:
			
			TODO: DESCRIBE HOW MAX NUMBER OF SPIRALS IS COMPUTED
			
			Given a number of spirals, the vector can be updated this (note that the
			change does not add a multiple of (1, 1, 1) but also does not result in
			the vector changing length and thus becoming non-minimal!).
			
			TODO: DESCRIBE TRANSFORMATION
			
			TODO: PROVE THAT MINIMALITY IS MAINTAINED
		
		\subsection{Proof of completeness}
		
			TODO: PROOF OF COMPLETENESS BY EXHAUSTIVE SEARCH
	
		\subsection{Conclusions}
			
			This approach is simpler, smaller, has sounder theoretical basis, and
			finds more routes than alternatives. This is good for load balancing and
			fault avoidance and also good for completeness.


	\chapter{Routing packets in large SpiNNaker machines}
	
	\label{sec:routing}
	
	So far, this thesis has focused on tackling the practical challenges
	resulting from SpiNNaker's hexagonal torus network topology. In this chapter,
	I adjust my focus towards the practical challenges resulting from SpiNNaker's
	large scale. Faults in large systems are inevitable and in the half-million
	core, \num{28800} chip SpiNNaker machine recently completed at the University
	of Manchester, around \SI{1}{\percent} of chips exhibited faults\footnote{Of
	the faulty chips discovered, the vast majority of faults are attributed to a
	currently unknown SDRAM failure}. These faults result in gaps and broken
	links in the network topology which routing algorithms must avoid in order to
	ensure correct system operation.
	
	In this chapter I tackle the problem of extending existing routing algorithms
	for SpiNNaker's network to enable them to route-around known, static faults.
	Though dynamic or transient faults may also occur, in this work such faults
	are ignored and other techniques, such as protocol-level fault tolerance, are
	relied on instead.
	
	Numerous heuristic-based fault-tolerant routing algorithms exist which target
	different network topologies and router architectures. Unfortunately as I
	will show, these algorithms are not portable and rely on or attempt to work
	around specific features of their target network architecture. In particular,
	existing work is dominated by the challenge of developing routing schemes
	which avoid or defuse network deadlocks. Due to SpiNNaker's unconventional
	use of timeout-based flow-control, it is not subject to the routing
	restrictions present in other architectures intended to cope with deadlocks.
	
	In this chapter I introduce a graph-search based post-processing step for
	non-fault-tolerant routing algorithms which guarantees routability in
	SpiNNaker systems without disconnected subregions. I also demonstrate that
	this technique introduces both negligible computational overhead to the
	routing algorithm runtime and resulting network performance.
	
	TODO: NOTE THE FAULT RATES ENCOUNTERED IN PRACTICE...
	
	\section{Related work}
		
		Existing work on routing in SpiNNaker's network has ignored the challenge
		of avoiding faults and instead focused on producing efficient multicast
		routes. As a result this section is broken into two halves. In the first
		half I survey the existing non-fault-tolerant approaches to routing used in
		SpiNNaker to-date. In the second I discuss the approaches to fault tolerant
		routing taken in other systems.
		
		\subsection{Multicast routing in SpiNNaker}
			
			Various fault-intolerant multicast routing algorithms exist for many
			networks and a number have been proposed and evaluated specifically in the
			context of SpiNNaker.
			
			In 2012, Davies \emph{et al.} evaluated the use of three common torus
			routing algorithms in SpiNNaker and found that simple oblivious routing is
			suitable for typical neural applications \cite{davies12}. The three
			routing techniques are:
			
			\begin{description}
				
				\item[Dimension Order Routing (DOR)] Packets are routed along each
				dimension (e.g. $X$, $Y$ and $Z$) in turn until no further hops are
				available in that direction.  The order in which the dimensions are
				traversed is fixed.
				
				\item[Right Turn Only Routing (RTOR)] As in DOR except the dimension
				order is chosen such that routes only contain right-turns.
				
				\item[Longest Dimension First Routing (LDFR)] As in DOR except the
				dimension order is chosen in descending order of number of hops in each
				dimension.
				
			\end{description}
			
			These unicast routing techniques are converted into a multicast routing
			algorithm by merging together the routes produced between the source node
			and each destination node as illustrated in figure
			\ref{fig:simple-routers}.
			
			\begin{figure}
				\center
				\begin{subfigure}{0.3\linewidth}
					\center
					\buildfig{figures/simple-routers-dor.tex}
					
					\caption{DOR}
					\label{fig:simple-routers-dor}
				\end{subfigure}
				\begin{subfigure}{0.3\linewidth}
					\center
					\buildfig{figures/simple-routers-rtor.tex}
					
					\caption{RTOR}
					\label{fig:simple-routers-dor}
				\end{subfigure}
				\begin{subfigure}{0.3\linewidth}
					\center
					\buildfig{figures/simple-routers-ldfr.tex}
					
					\caption{LDFR}
					\label{fig:simple-routers-dor}
				\end{subfigure}
				
				\caption{Example multicast routes produced by merging together unicast
				routes from a central source node to each destination node.}
				\label{fig:simple-routers}
			\end{figure}
			
			In 2014, Navaridas \emph{et al.} introduced two new algorithms, `Enhanced
			Shortest Path Routing' (ESPR) and `Neighbourhood Exploring Routing' (NER)
			which produce multicast routing trees with fewer total hops
			\cite{navaridas14}. In both algorithms, routes are generated sequentially
			for each of the destinations of a route using LDFR. Unlike LDFR, however,
			these algorithms search a limited area of the network for other,
			already-connected destination nodes to which LDFR routes may be
			constructed. If no suitable destination is found, a LDFR route is
			constructed to the source node. Figure \ref{fig:search-regions} illustrates
			the shape of the searched regions of each algorithm. ESPR searches the
			trapezoidal region between the source and destination nodes while NER
			searches a fixed radius out from the destination node.
			
			\begin{figure}
				\center
				\begin{subfigure}{0.45\linewidth}
					\center
					\buildfig{figures/search-regions-espr.tex}
					
					\caption{ESPR}
					\label{fig:search-regions-espr}
				\end{subfigure}
				\begin{subfigure}{0.45\linewidth}
					\center
					\buildfig{figures/search-regions-ner.tex}
					
					\caption{NER}
					\label{fig:search-regions-espr}
				\end{subfigure}
				
				\caption{The ESPR and NER algorithms attempt to connect the node marked
				`D' to the closest node in the shaded region which is connected to the
				source node, `S'. If no connected node is found in the shaded region, the
				LDFR route is taken to `S'. The dotted line indicates the route chosen
				from `D'.}
				\label{fig:search-regions}
			\end{figure}
			
			Unfortunately none of these routing algorithms make any allowance for the
			avoidance of network faults. As a result their utility in real-world
			systems is limited.
		
		\subsection{Fault-tolerant routing}
			
			Numerous fault-tolerant routing algorithms have been proposed for
			super-computer networks. These algorithms are largely constrained by the
			need to maintain deadlock freedom. Since SpiNNaker's routers employ a
			timeout based deadlock-breaking strategy, much of this effort is
			unnecessary in SpiNNaker. As described below, this frequently renders
			existing fault-tolerant routing algorithms unnecessarily complex and
			inflexible.
			
			Deadlocks occur in a network if a cyclic dependency exists on any limited
			resource in the network. For example, as illustrated in figure
			\ref{fig:ring-deadlock}, in a ring network a deadlock may form when every
			node is waiting on the next node to accept a packet before accepting new
			packets from the previous node.
			
			\begin{figure}
				\center
				\buildfig{figures/ring-deadlock.tex}
				
				\caption{A deadlock in a ring network where each node is waiting for
				the next to accept a packet before accepting any further packets.}
				\label{fig:ring-deadlock}
			\end{figure}
			
			To prevent deadlocks, combinations of router microarchitectural features
			and routing restrictions may be employed. For example, a simple
			deadlock-free routing algorithm for mesh and torus networks mandates the
			use of DOR \cite{dally93}. Packets travelling in a -ve direction along
			each axis take priority over those travelling in a +ve direction. Packets
			travelling along the Y axis take priority over those travelling along the
			X dimension. Given these rules it is possible to define a total ordering
			on all hops in the network. Figure \ref{fig:deadlock-free-dor}
			illustrates a $3\times3$ mesh network whose hops have been numbered
			according to the total ordering defined above.  Any `X-then-Y' DOR route
			through this network results in the use of hops labelled with strictly
			increasing numbers. As a result, no cyclic dependencies (and thus no
			deadlocks) may occur.
			
			\begin{figure}
				\center
				\buildfig{figures/deadlock-free-dor.tex}
			
				\caption{Deadlock-free routing of two example routes using DOR in a 2D
				mesh topology. The numbers of the hops taken by each route are given on
				the right.}
				\label{fig:deadlock-free-dor}
			\end{figure}
			
			Unfortunately, the routing restrictions imposed to ensure deadlock
			freedom can result in fault-intolerant routing. In the example above, if
			the node at the bottom-right corner of the figure was faulty, the dotted
			example route would be blocked as no alternative routes are allowed.
			
			In practice, the routing rules used may be more relaxed, for example
			requiring that any route whose length is equal to a DOR must exist to
			guarantee routability \cite{rodrigo09}.
			
			Alternative routing strategies take a hybrid approach whereby an
			efficient but fault-intollerant routing algorithm is used where possible
			and in the presence of faults a less efficient but more robust strategy
			is employed. For example, the Immucube network architecture employs three
			virtual networks which operate independently over the same physical links
			\cite{puente07}. Initially messages are routed using a high-performance
			but potentially-deadlockable routing scheme in the first virtual network.
			If a deadlock is occurs, the deadlocked packet is dropped into the second
			virtual network in which packets are routed using a less efficient but
			deadlock-free but fault-intolerant routing algorithm. Finally, upon
			encountering a fault, packets are dropped onto the third virtual network
			which forms a ring network routing packets to every node in the network.
			
			Releated approaches \cite{mejia06,boppana95} divide the network into
			regions in which different routing rules are enforced to ensure deadlock
			freedom and, when required, fault tolerance.
			
			TODO FIGURE?
			
			The BlueGene/L supercomputer \cite{adiga02} uses DOR for its torus
			network and implements fault-tolerance by sacrificing otherwise
			functioning `lamb' nodes to ensure no route passes through a known dead
			link \cite{ho04}. In figure \ref{fig:lamb-nodes} an example scenario is
			shown where a single dead node is present and all nodes in the same row
			or column as the dead node have been made into lamb nodes. The lamb nodes
			may not be used in an application except as a through-route for other
			traffic. This pattern of lamb nodes guarantees that all dimension-order
			routes between all pairs of non-lamb nodes are not obstructed by the
			faulty node. This approach trades use of higher performance routing
			logic for wasted resources. This type of approach is most appropriate
			when algorithmic routing is used and routing rules are inflexible.
			
			\begin{figure}
				\center
				\buildfig{figures/lamb-nodes.tex}
				
				\caption{`Lamb' nodes may be disabled to ensure DOR will never
				encounter a fault.}
				\label{fig:lamb-nodes}
			\end{figure}
			
			Other algorithms proposed for the BlueGene architecture attempt to avoid
			the need for lamb nodes by generating routes which reach their destination
			via a `proxy' node \cite{gomez04}. By appropriately selecting the location
			of such a proxy, the existing routing algorithm used by the system can be
			guaranteed to select a route free of faults.
			
			TODO: EXAMPLE OF PROXY ROUTING TO AVOID FAULT
			
			Finally, many algorithms in in the field are distributed and use only local
			information along with limited information from their peers to generate
			routes \cite{fick09b}. In SpiNNaker, route generation is conventionally
			carried out centrally since no special on-chip hardware facilities exist
			for route generation. Centralised route generation also enables the routing
			algorithm to consider all available routes. As a result, there is little
			incentive for the use of distributed routing algorithms on SpiNNaker since
			global system information could be compactly shared for one-off routing
			passes.
			
			Algorithms for other architectures such as IP networks tend to be poor fits
			for static, regular network topologies since they use expensive graph-based
			algorithms for route discovery which aren't necessary here. They also tend
			to heavily feature graph topology discovery etc. which aren't needed here.
			
			Work on fault-tolerance in data centre networks does exploit the regularity
			of the network topology in routing algorithms \cite{guo08,liao12}.
			Unfortunately, the approaches used are not general enough to be applied to
			mesh-like topologies such as the one in SpiNNaker.
			
			Outside the field of computer networks, routing algorithms used to route
			wires across the surfaces of chips are required to solve similar problems
			to fault-tolerant network routing problems in mesh networks. Like mesh
			networks, the routes are defined within a regular Manhattan geometry and
			congested areas, rather than faults must be avoided by the algorithms
			\cite{kahng11}.  Unfortunately, these algorithms are designed for
			occasional batch operation prior to the multi-month process of chip
			manufacturing and so runtimes of hours or days are commonplace
			\cite{nam08}. As such these algorithms would be inappropriate for use
			with applications such as SpiNNaker where users' applications tend to be
			short-lived and thus routing should not be allowed to dominate runtime.
	
	\section{Partial graph search repair}
		
		In this section I introduce a novel post-processing algorithm, Partial
		Graph Search (PGS) repair, for routes produced by non-fault-tolerant
		routing algorithms.
		
		PGS repair guarantees routability for networks with no disconnected
		subregions by using a graph search algorithm to route around faults in the
		original route.  General-purpose graph search algorithms such as Breadth
		First Search (BFS), Dijkstra's Algorithm and A* are guaranteed to find
		shortest-path routes between pairs of points in arbitrary graphs. Such
		algorithms are generally a poor choice in highly regular network topologies
		such as meshes and toruses due to their high computational cost. In PGS
		repair, graph searching is only used for \emph{part} of the routing
		problem: to repair gaps in routes generated by more efficient routing
		algorithms.
		
		Real world super computer architectures are designed to ensure that faults
		are isolated \cite{gara05,alverson12} and thus tend to only impact a
		localised region of the network. Since PGS repair is only needed to route
		around these isolated faults, the space searched by the graph search
		algorithm should be very small in practice resulting in only short
		runtimes. In addition since faults are rare in real-world systems, the
		graph search process will only rarely be invoked.
		
		The PGS repair post-processing technique starts with a route produced by a
		non-fault-tolerant routing algorithm such as ESPR or NER. If this route is
		not obstructed by a fault, the algorithm terminates immediately without
		modifying the route. If the route attempts to use a faulty link, the
		algorithm proceeds as follows.
		
		The routing tree produced by the underlying routing algorithm is broken
		into subtrees wherever it attempts to route through a broken link and
		each subtree is assigned a unique colour, as illustrated in figure
		\ref{fig:pgs-repair-colouring}. From each disconnected subtree's root
		node in turn, a graph search is performed to find a short, fault-free
		route to a subtree node of a different colour. The subtree is then
		attached to the tree discovered by the graph search and re-coloured to
		match the tree it is connected to.
		
		\begin{figure}
			\center
			\begin{subfigure}{0.32\linewidth}
				\hspace*{-1.5em}
				\buildfig{figures/pgs-repair-colouring.tex}
				
				\caption{}
				\label{fig:pgs-repair-colouring}
			\end{subfigure}
			\begin{subfigure}{0.32\linewidth}
				\hspace*{-1.5em}
				\buildfig{figures/pgs-repair-colouring-fix1.tex}
				
				\caption{}
				\label{fig:pgs-repair-colouring-fix1}
			\end{subfigure}
			\begin{subfigure}{0.32\linewidth}
				\hspace*{-1.5em}
				\buildfig{figures/pgs-repair-colouring-fix2.tex}
				
				\caption{}
				\label{fig:pgs-repair-colouring-fix2}
			\end{subfigure}
			
			\caption{PGS repair process example showing a disconnected multicast
			route from A to B, C, D, E and F. $\times$ indicates a broken link.}
			\label{fig:pgs-repair-colouring-steps}
		\end{figure}
		
		For example in figure \ref{fig:pgs-repair-colouring-fix1} a path from the
		root of the subtree containing nodes E and F is found which connects it to
		the subtree rooted at A. Similarly in figure
		\ref{fig:pgs-repair-colouring-fix2} a path is also found connecting the
		subtree containing nodes C and D back to the subtree rooted at node A.
		
		If the routing tree was broken into $N+1$ subtrees by faults there will be
		$N$ subtrees disconnected from the root node. Each of the $N$ iterations of
		the algorithm connect a disconnected subtree to another subtree reducing
		the number of subtrees by $1$ each time. After $N$ iterations, therefore,
		exactly $1$ subtree remains which connects every node in the original
		routing tree without traversing faulty links.
		
		TODO: EXPLAIN THE FIDDLINESS HERE TO ENSURE WE DON'T CREATE LOOPS.
		
	\section{Evaluation \& Results}
		
		The PGS repair technique, by design, is able to work around all possible
		fault patterns which don't completely disconnect parts of the network. This
		result this evaluation focuses on the impact on performance PGS repair
		imposes. The metrics of interest in this evaluation are:
		
		\begin{itemize}
			\item Algorithm runtime
			\item Network congestion
			\item Routing table utilisation
		\end{itemize}
		
		\subsection{Traffic Patterns}
			
			In this evaluation, two standard benchmark multicast traffic patterns are
			used which have been used in previous research into SpiNNaker's network:
			
			\begin{figure}
				\center
				\buildfig{figures/traffic-distribution-centroids.tex}
				
				\caption{An example 4-centroid distribution with four centroids. The
				$\times$ marks the location of the origin node. Lighter colours
				indicate greater likelihood of a connection.}
				\label{fig:traffic-distribution-centroids}
			\end{figure}
			
			\begin{description}
				
				\item[Uniform] Destinations are chosen with uniform probability
				anywhere in the machine.
				
				\item[$N$-Centroids] Destinations are clustered around one of $N$
				randomly chosen `centroids' as illustrated in figure
				\ref{fig:traffic-distribution-centroids}.
				
			\end{description}
			
			The uniform traffic pattern is widely used in networks research
			\cite{dally04,davies12} while the centroids model was developed
			specifically to reproduce the traffic patterns found in the neural
			applications SpiNNaker is designed for \cite{navaridas14}. In this work
			we consider 3 centroids.
		
		\subsection{Fault model}
			
			In addition two different fault models are used which are representative of
			the faults found in real SpiNNaker systems:
			
			\begin{figure}
				\center
				\begin{subfigure}{0.48\linewidth}
					\hspace*{-1.5cm}
					\buildfig{figures/fault-example-uniform.tex}
					
					\caption{Uniform}
					\label{fig:fault-example-uniform}
				\end{subfigure}
				\begin{subfigure}{0.48\linewidth}
					\hspace*{-1.5cm}
					\buildfig{figures/fault-example-hss.tex}
					
					\caption{HSS Link}
					\label{fig:fault-example-hss}
				\end{subfigure}
				
				\caption{The two link fault models considered.}
				\label{fig:fault-example}
			\end{figure}
			
			\begin{description}
				
				\item[Uniform] Links are selected and disabled at random (figure
				\ref{fig:fault-example-uniform}).
				
				\item[HSS Link] Groups of links corresponding with randomly selected
				single High-Speed Serial (HSS) link between SpiNNaker boards are disabled
				together (figure \ref{fig:fault-example-uniform}).
				
			\end{description}
			
			The uniform link failure model models isolated failures resulting from
			isolated manufacturing defects in individual links. The HSS Link failure
			model models faults arising from failing or disconnected board-to-board
			links which carry several chip-to-chip traffic flows via a single cable in
			SpiNNaker systems. Though SpiNNaker-specific, the later fault model is
			analogous to failure modes arising in other architectures where a single
			fault may render several links impassable in a single area.
			
			A range of failure rates are explored in this section. My measurements of
			current large-scale SpiNNaker installations the link failure rate is about
			\SI{0.03}{\percent} with failures due to both individual chip-to-chip links
			and board-to-board HSS links. Exact link failure statistics for commercial
			super computer installations are not widely available, however, published
			Mean-Time-Between-Failure (MTBF) statistics place an upper bound on link
			failure rates at a similar \SI{0.03}{\percent} in one-year-old BlueGene/Q
			systems \cite{chiu11}.
			
			Unfortunately presently undiagnosed problem with the SDRAM packaged with
			approximately \SI{1}{\percent} of SpiNNaker chips has rendered these chips
			unusable for most applications. The gaps in the network resulting from the
			loss of these chips currently dominate true link failures leaving just over
			\SI{1}{\percent} of links inoperable.
			
			Surprisingly, research into fault tolerant routing in super computers
			appears to focus on benchmarks with even higher fault rates ranging from
			\SI{3}{\percent} to as high as \SI{7}{\percent}
			\cite{ho04,gomez04,mejia06}.
			
			In this evaluation, fault rates ranging from \SI{0.01}{\percent} to
			\SI{5}{\percent} are considered to cover both realistic fault levels
			along with the more extreme cases considered in related work.
		
		\subsection{Base routing algorithm}
			
			Since the PGS repair process is routing algorithm agnostic all
			experiments use the NER algorithm which has been found to be appropriate
			for SpiNNaker applications \cite{navaridas14}.
		
		\subsection{Algorithm runtime}
			
			To assess the impact of the PGS repair process on routing algorithm
			runtime, the algorithm was used to process a large number of randomly
			generated routing problems and the runtime recorded.
			
			\num{10000} one-to-sixteen multicast routing problems were generated in a
			$256\times256$ hexagonal torus topology, the largest size possible for a
			SpiNNaker system. Other quantities of multicast destinations were also
			evaluated but are omitted for brevity since the pattern of results are
			similar to those outlined here.
			
			TODO: APPENDIX WITH OTHER RUNS?
			
			The NER and PGS repair algorithms were written in C and compiled with GCC
			4.8.3 with \verb|-O2| level optimisations and executed on a cluster of
			idle workstations with 3.10 GHz Intel Core-i5-2400 CPUs.
			
			\begin{figure}
				\center
				\buildrplot{figures/routing-runtimes.R}
				
				\caption{Mean runtime of routing and PGS repair overhead. PGS repair
				overhead is stacked above the routing runtime (i.e. bars do not
				overlap). Error bars indicate 95\% confidence interval. Note different
				Y-scale for HSS link and uniform fault models.}
				\label{fig:routing-runtimes}
			\end{figure}
			
			Figure \ref{fig:routing-runtimes} shows the average runtimes recorded for
			both the NER and PGS repair algorithms. In fault-free networks the
			PGS-repair post-processing step is not required and incurs no penalty
			while the runtime of the algorithm grows with the fault rate for both
			fault and traffic models.
			
			Notably the HSS fault model results in longer runtimes for the PGS repair
			process compared with an equivalent fault-density of uniform faults.
			Because the HSS fault model produces contiguous lines of faults the PGS
			repair algorithm must construct a longer path to avoid the fault.  Since
			the space explored by a graph algorithm typically grows with $O(H^2)$
			with respect to the hops in the discovered route, $H$, this increase in
			search distance has a large impact on the runtime of the PGS repair
			process.
			
			The runtime of the PGS repair algorithm remains roughly in proportion to
			the runtime of the underlying routing algorithm with respect to different
			traffic models. The centroid traffic pattern tends to result in routes
			with fewer hops than a uniform traffic pattern with the same number of
			destination nodes as segments of routes are often shared between
			destination nodes. Since the NER algorithm's runtime is strongly related
			to the number of hops in the output route the runtime of the algorithm is
			greater for uniform traffic. Likewise the probability of PGS repair being
			required increases with the number of hops in route and hence the runtime
			of the PGS repair algorithm increases roughly in proportion.
		
		\subsection{Routing table usage}
			
			In order to gain a realistic measure of routing table usage it is
			necessary to determine the effect of many routes being generated for a
			single set of faults. To enable a sufficiently large number of sample to
			be collected the experimental setup considered previously is reduced to a
			network containing $48\times48$ nodes.
			
			\num{1000} $48\times48$ node network models are produced according to the
			HSS link and uniform fault models. For each of these models
			$48\times48\times16=$~\num{36864} one-to-sixteen routes are generated using
			the centroid and uniform traffic models. This corresponds to one
			multicast route per application core. As is convention in SpiNNaker,
			routing table entries are inserted for each route at the source of the
			route, at each destination and at each corner or fork. The number of
			routing table entries at each node in the model is counted and the
			maximum number of entries in a single node is reported for each network
			model.  The \emph{maximum} number of routing entries of any router was
			chosen since the number of entries available per SpiNNaker router is
			bounded by hardware.
			
			\begin{figure}
				\center
				\buildrplot{figures/routing-entries.R}
				
				\caption{Violin plot showing the distribution of maximum table sizes
				for \num{1000} random networks. The red line at \num{1024} entries
				indicates the size of SpiNNaker's routing tables.}
				\label{fig:routing-entries}
			\end{figure}
			
			
			Figure \ref{fig:routing-entries} shows the distributions of the largest
			routing table sizes for each fault and traffic model.
			
			\begin{figure}
				\center
				\begin{subfigure}{0.48\linewidth}
					\center
					\buildfig{figures/hss-link-routing-table-usage.tex}
					
					\caption{Routing table entries}
					\label{fig:hss-link-routing-table-usage}
				\end{subfigure}
				\begin{subfigure}{0.48\linewidth}
					\center
					\buildfig{figures/hss-link-resource-usage.tex}
					
					\caption{Routes passing through chip}
					\label{fig:hss-link-resource-usage}
				\end{subfigure}
				
				\caption{The impact of a HSS link fault on routing table usage and
				congestion. Each hexagon represents a single chip, the red line
				indicates the chip-to-chip connections broken by the HSS link fault.}
				\label{fig:hss-link-usage}
			\end{figure}
			
			The HSS link failure model has a much greater impact on peak routing
			table resource usage than uniform link failures for a given fault rate.
			This is because HSS link faults result in a large concentration of routes
			being disrupted and then re-routed around the same obstacle in a single
			location. Figure \ref{fig:hss-link-routing-table-usage} shows how routing
			table usage varies around a HSS link fault in one instance of the
			experiment. There are clear peaks in routing table usage around the ends
			of the line of faults which result from routes produced by PGS repair
			finding shortest paths around the edge of the faults.
		
		\subsection{Network congestion}
			
			To measure the impact of PGS repair on network congestion, two
			experiments were performed, one using the same model used to measure
			routing table usage and one based on tests run on SpiNNaker hardware.
			
			For each of the network fault and traffic pattern described previously,
			the paths taken for the \num{36864} one-to-sixteen multicast routes
			generated are used to compute the number of times each link in the
			network is used. The number of routes passing through the most-used link
			is then recorded, giving an indication of the level of congestion in the
			network.
			
			\begin{figure}
				\center
				\buildrplot{figures/routing-resource.R}
				
				\caption{Violin plot showing the distribution of maximum
				routes-per-chip for \num{1000} random networks.}
				\label{fig:routing-resource}
			\end{figure}
			
			The results are presented in figure \ref{fig:routing-resource} and follow
			the same trends as the results previously shown for routing table usage.
			Again, HSS link faults result in routes with the greatest congestion due
			to the concentration of routes finding shortest paths around an obstacle
			(see \ref{fig:hss-link-resource-usage}).
			
			To verify that the results above, an additional experiment has been
			carried out which attempts to mimic the model used previously in actual
			SpiNNaker hardware. In these experiments a large SpiNNaker machine is
			divided into independent 48-board (2304-chip) sections. Because the
			48-board systems used in these experiments are cut out of a larger
			machine, they lack wrap-around links and thus form hexagonal mesh
			topologies, rather than hexagonal toruses.
			
			Due to the SDRAM issue described above, fault rates below
			\SI{1}{\percent} cannot be modelled.  To simulate higher fault rates,
			additional links are disabled in software according to the fault models
			described used previously. Since some faults are due to genuine hardware
			faults, these faults cannot be placed randomly in each experiment. To
			reduce, bias each combination of fault rate, fault model and traffic
			pattern is repeated XXX times across randomly chosen physical machines.
			
			XXX 1-to-XXX routes are generated in both uniform and XXX-centroid
			distributions as used throughout this evaluation. Synthetic network
			traffic is generated at the source of each route following a Bernoulli
			distribution. Traffic consumers running on all destination nodes accept
			packets as quickly as possible from the network and log their arrival.
			The Bernoulli probability is set the same for every route's traffic
			generator and increased in steps of XXX and the number of packets dropped
			in an XXX second period logged. The network is considered saturated once
			less than \SI{99}{\percent} of packets successfully arrive at their
			destination.
			
			Figure \ref{XXX} shows the distributions of the saturation points for
			each experimental configuration.
			
			TODO: ANALYSIS
		
	\section{Conclusions}
		
		In this chapter I described how SpiNNaker's unconventional network and
		router architecture render existing fault tolerant routing algorithms
		unsuitable. I introduced PGS repair, a post-processing technique for
		existing non-fault tolerant routing algorithms designed for SpiNNaker such
		as NER.
		
		Unlike some other fault tolerant routing algorithms for other
		architectures, PGS repair is able to work-around arbitrary fault patterns
		by exploiting SpiNNaker's inbuilt deadlock avoidance mechanisms. In the
		presence of realistic failure rates of up to \SI{1}{\percent}, only small
		overheads of up to XXX, XXX and XXX for in algorithm runtime, routing table
		usage and network performance are incurred respectively. This low
		performance overhead makes PGS repair appropriate for use in real
		applications. At the time of writing the algorithm has been successfully
		used in a number of neural and non-neural SpiNNaker applications.
		
		At more extreme fault rates not expected in real-world systems, the
		algorithm still functions correctly but the results incur much greater
		routing table and congestion overheads, particularly when faults are
		concentrated. Future extensions to this algorithm might aim to reduce this
		overhead by producing longer and more varied routes around faults to even
		out the load.

	\chapter{Placing applications in large SpiNNaker machines}
	
	In the previous chapter I tackled the problem of scale in generating routes
	for very large networks such as SpiNNaker. In this work the centroid traffic
	pattern was used as an approximation of the expected network traffic
	generated by `well behaved' neural network simulation software running on
	SpiNNaker. The traffic produced largely exhibits strong locality, that is
	most communication occurs between either nearby nodes or clusters of nodes.
	In reality, neural simulation applications are not specified geometrically
	but rather as abstract graphs of communicating neurons
	\cite{davison08,eliasmith13}. Applications must then \emph{place} these
	neurons onto nodes in a SpiNNaker system, attempting maximise communication
	locality.
	
	In this chapter I re-evaluate the suitability of simulated annealing as a
	technique for finding high quality placements for large parallel
	applications. Though this technique had fallen out of fashion in the field of
	application placement by the early 1990s, it has found wide use for placing
	components in computer chip and FPGA designs. In the intervening years,
	placement problems in super computers have grown in size from tens or
	hundreds of nodes to millions, a scale at which chip placement techniques
	were operating in the mid 1990s. I adapt the simulated annealing algorithm
	used by the VPR academic circuit placement software to produce placements for
	applications running on SpiNNaker. In that in a range of real and synthetic
	benchmarks simulated annealing produces high quality placements enabling
	efficient use of SpiNNaker's network resources.
	
	
	%In the field of chip design, Moore's `Law' \cite{moore65,moore75} observes a
	%similar exponential growth in the number of components within a single chip.
	%Today modern processors contain billions of components and an analagous
	%placement problem exists in attempting to place interconnected components
	%near to eachother. In this chapter I explore the techniques used for circuit
	%placement and adapt one such technique, Simulated Annealing (SA)
	%\cite{kirkpatrick83}, for use in application placement. Despite some early
	%interest in SA for application placement in the 1980s and early 1990s, the
	%technique has since fallen out of favour. I find that at the scales of modern
	%placement problems SA-based placement is able to produce solutions of
	%superiour quality to contemporary methods.
	%
	%TODO: SUMMARISE RESULTS...
	
	\section{Related work}
		
		The placement problem has been tackled independently in the literature by
		researchers in both the application and chip placement communities. In this
		survey I cover application and chip placement separately as these two
		communities have remained largely isolated from one another. First I
		explore the techniques applied to application placement before moving on to
		contrast this with the techniques used in circuit placement.
		
		In the application placement literature, the placement problem is often
		referred under the umbrella term `mapping'. Unfortunately term is often
		used more broadly to include other tasks such as routing and application
		partitioning. To avoid ambiguity I use the term `placement', as preferred
		by the chip and FPGA design communities, to refer specifically to the
		problem of assigning nodes in an application's communication graph to nodes
		in a machine's connectivity graph.
		
		\subsection{Application placement algorithms}
			
			TODO: GENERAL INTRO
			
			\subsubsection{Application-specific approaches (manual placement)}
				
				In the case of some applications such as finite element modelling
				\cite{bermejo13}, the structure of the problem itself leads to a
				natural placement of the computation on nodes in a machine. For example
				when simulating a 3D volume in an node super computer with a $3 \times
				4 \times 2$ 3D torus or mesh topology network, the modelled volume
				might be divided into as in figure \ref{fig:fem-partitioning}. Each
				cuboid in the model is then assigned to the corresponding node in the
				network topology.
				
				\begin{figure}
					\center
					\buildfig{figures/fem-partitioning.tex}
					
					\caption{Example partitioning of a 3D space to fit into a super
					computer with a $3\times4\times2$ torus or mesh topology.}
					\label{fig:fem-partitioning}
				\end{figure}
				
				When the number of dimensions in a problem do not match that of the
				underlying network architecture, the common solution is to either
				divide only along a subset of the axes or to divide into additional
				pieces on the existing axes \cite{gilge14}.
			
			\subsubsection{Sequential placement}
				
				In the case where a placement solution is non-obvious one of the
				simplest and most popular strategies is to apply a simple sequential
				placement algorithm. Sequential placement algorithms function by
				iterating over the vertices in the application's communication graph
				and assigning them to a free node in the target machine. Sequential
				placement algorithms are differentiated by the order in which they
				iterate over vertices in the communication graph and fill nodes in the
				target machine. A number of widely used orderings are described below.
				
				\begin{figure}
					\center
					\begin{subfigure}{0.32\linewidth}
						\center
						\buildfig{figures/sequential-row-order.tex}
						\caption{Row-order}
						\label{fig:sequential-row-order}
					\end{subfigure}
					\begin{subfigure}{0.32\linewidth}
						\center
						\buildfig{figures/sequential-alternating.tex}
						\caption{Alternating}
						\label{fig:sequential-alternating}
					\end{subfigure}
					\begin{subfigure}{0.32\linewidth}
						\center
						\buildfig{figures/sequential-hilbert.tex}
						\caption{Hilbert curve}
						\label{fig:sequential-hilbert}
					\end{subfigure}
					
					\caption{Space-filling curves in 2D mesh and torus topologies.}
					\label{fig:sequential}
				\end{figure}
				
				Super computer management software such as SLURM \cite{yoo03} and Blue
				Gene's system software \cite{gilge14} by default na\"ively iterate over
				vertices in an application communication graph in the order they are
				provided. The nodes in the target machine are then iterated over in a
				simple space-filling curve through the network topology. Figure
				\ref{fig:hilbert-placement} illustrates the default patterns available
				in these software packages. The row-order (figure
				\ref{fig:sequential-row-order}) and alternating (figure
				\ref{fig:sequential-alternating}) curves correspond with 2D versions of
				the default node assignment orders used in SLURM and BlueGene systems.
				
				\begin{figure}
					\center
					\buildfig{figures/hilbert-placement.tex}
					
					\caption{A Hilbert curve, coloured from blue to red.}
					\label{fig:hilbert-placement}
				\end{figure}
				
				The Cray extensions to SLURM software provide a Hilbert curve
				\cite{hilbert91} (figure \ref{fig:sequential-hilbert}) node assignment
				order. Unlike the row-order and alternating space filling curves the
				Hilbert curve ensures that pairs of vertices close together in the node
				iteration order are also close together in the target machine's network
				\cite{moon01, zumbusch99}. Figure \ref{fig:hilbert-placement} shows a
				5$^\textrm{th}$-order Hilbert curve where each point in the curve is
				coloured according to its position along the curve. In this figure it
				is possible to see that nearby positions in the curve (which share
				similar colours) are also close in 2D space.
				
				When the proximity of vertices in the vertex-ordering supplied by an
				application is a good estimator of those vertices communication
				requirements, the sequential assignment schemes discussed above can be
				very effective. These techniques have also proven adequate in
				small-scale and densely connected applications such as early neural
				simulations running on prototype SpiNNaker machines with tens of nodes
				\cite{galluppi10} but growing beyond this scale has proven problematic.
				
				\begin{figure}
					\center
					\begin{subfigure}{0.45\linewidth}
						\center
						\buildfig{figures/rcm-initial.tex}
						
						\caption{Original permutation}
						\label{fig:rcm-initial}
					\end{subfigure}
					\begin{subfigure}{0.45\linewidth}
						\center
						\buildfig{figures/rcm-sorted.tex}
						
						\caption{RCM permutation}
						\label{fig:rcm-sorted}
					\end{subfigure}
					
					\caption{Adjacency matrix representation of a graph before and after
					permutation by the RCM algorithm.}
					\label{fig:rcm}
				\end{figure}
				
				A number of algorithms have been proposed for automatically selecting
				good vertex iteration orders, typically using a graph-traversal based
				heuristic. A typical method, described by Hoefler \emph{et al.}
				\cite{hoefler11} exploits the Reverse-Cuthill-McKee (RCM) algorithm
				\cite{cuthill69}. An application's communication matrix is represented
				as an adjacency matrix, $M$, where $M_{i,j}$ is 1 if node $i$ is
				connected by an edge to node $j$ and 0 otherwise. An example matrix is
				illustrated in figure \ref{fig:rcm-initial}. The RCM algorithm uses a
				simple heuristic to permute the matrix (i.e. renumber the nodes in the
				graph) in order to reduce the bandwidth of the matrix. Figure
				\ref{fig:rcm-sorted} shows the RCM-permuted version of the example
				adjacency matrix. When a graph's vertices are ordered as in a
				bandwidth-reduced sparse matrix, vertices close together in the
				ordering are likely to communicate while those further apart tend not
				to communicate.
				
			\subsubsection{Optimisation-based Placement}
				
				% Citations from short report about optimisation in placement...
				% \cite{chen06,jeannot14} and \cite{jeannot10} ("subsets of apps")
				
				In the academic community, a number of attempts have been made to use
				more sophisticated optimisation algorithms for the placement of
				applications. In 1985, Steele \cite{steele85} proposed the use of
				simulated annealing for placing applications in the 6D torus topology
				of the 64 node `Caltech Cosmic Cube' machine. Simulated annealing,
				originally developed by Kirkpatrick \emph{et al.} \cite{kirkpatrick83},
				is a general-purpose optimisation algorithm which works by analogy to
				the physical process of annealing. In brief simulated annealing
				functions by randomly swapping vertices in a candidate placement
				solution, accepting swaps which move connected vertices closer together
				and rejecting some proportion of swaps which move connected vertices
				further apart. The simulated annealing algorithm is described in detail
				later in this chapter.
				
				Towards the end of the 1980s, application placement appeared to be
				becoming less important as super computer network architectures
				improved:
				%
				\begin{displayquote}
					``Careful placement was necessary because of the slow communication
					and non-uniform addressing of early concurrent computers. However,
					the development of message passing machines with fast communications
					and a uniform global address space  has made placement less of an
					issue. In such machines a random placement performs nearly as well as
					an optimum placement.''
					
					\noindent --- W. Dally, 1987 \cite{dally87}
				\end{displayquote}
				%
				In addition, network and problem sizes remained small, so small in fact
				that linear-programming based optimal placement still appeared in
				benchmarks comparing placement algorithms \cite{xu91}. In this
				environment, simpler sequential placement algorithms gained favour over
				more computationally expensive algorithms such as simulated annealing.
				
				As problem and machine sizes have grown and network utilisation has
				once again become an important factor in application performance
				\cite{navaridas09b} more complex optimisation algorithms have
				reappeared in the literature. One popular approach employs graph
				partitioning algorithms such as METIS \cite{karypis98} to perform
				recursive bipartitioning based placement
				\cite{phillips14,hoefler11,pellegrini96}.  This placement process is
				illustrated in figure \ref{fig:partitioning}.
				
				In the first step, the application communication graph and machine
				connectivity graph are bipartitioned such that the number of edges
				between partitions is minimised. Each half of the communication graph
				is associated with one of the halves of the machine connectivity graph.
				The partitioning process is then repeated recursively on each of the
				two communication and connectivity graph pairs. The process halts when
				the graphs can no longer be partitioned at which point the vertices in
				the communication graph are placed on their associated node.
				
				\begin{figure}
					\center
					\buildfig{figures/partitioning.tex}
					
					\caption{Illustration of application placement by recursive
					partitioning.}
					\label{fig:partitioning}
				\end{figure}
				
				TODO: PARTITIONING IS GREAT AND ALL BUT QUALITY ISN'T ALWAYS GREAT AND
				IT DOESN'T DEAL WELL WITH MULTI-CONSTRAINT SCENARIOS E.G. PROCESSOR AND
				MEMORY RESTRICTIONS.
				
				Unfortunately, many of these simply aren't suited to the scale of
				neural applications running on SpiNNaker (e.g. only cope with tens of
				nodes while SpiNNaker may contain hundreds of thousands).
				
				Additionally, a number of algorithms have been developed which make
				assumptions about the topologies of the problem or network. Tree match
				for example attempts to map tree-shaped problems to tree-shaped
				networks. Such algorithms can be highly effective but again do not
				apply to SpiNNaker or its neural applications.
		
		\subsection{Chip placement algorithms}
			
			The chip-design industry has, for many years, dealt with problems
			analogous to the task of placing super computer jobs in a way suited to
			SpiNNaker. Modern CPUs have millions or billions of components with
			strictly fixed connectivity. CPU designers must place each of these onto
			a chip such that the connection lengths are controlled to reduce
			congestion and increase performance. As such, these algorithms are
			ideally suited to future super computer placement work since they already
			operate at the scales required \cite{nam07}.
			
			\subsubsection{Cost functions}
				
				HPWL is popular but a bit crap for high fan-outs. It is, however, quite
				simple.
				
				TODO: SELECT A BETTER COST FUNCTION...
			
			\subsubsection{Simulated annealing}
				
				One of the oldest techniques used for circuit placement is simulated
				annealing and this remains popular today thanks to its sheer
				versatility (see VPR, other open FPGA tools).
				
				SA works by analogy with the physical process of annealing.
				The simulated annealing algorithm works by selecting random pairs of
				components on a chip, swapping them and evaluating some cost function.
				If the swap reduces the cost function, it is kept, if not, depending on
				a function of the current temperature and the cost introduced by the
				swap.
				
				TODO: ILLUSTRATION OF SIMULATED ANNEALING SWAP OPERATION
				
				By occasionally allowing costly swaps, the annealing algorithm avoids
				becoming trapped in local minima. As the algorithm proceeds, the
				temperature is slowly reduced and with it the proportion of costly
				swaps which are retained. This causes the placement to move from
				exploration early on towards refinement later on.
				
				The temperature schedule of an annealing algorithm is critical to its
				success. In general these schedules are computed based on the
				performance of the algorithm as it runs. In VPR the following schedule
				is used.
				
				TODO: DESCRIBE VPR'S SCHEDULE
				
				TODO: FIND AND DESCRIBE ALTERNATIVE SCHEDULE?
				
				Unfortunately, SA is very difficult to parallelise, especially in the
				case of placement. As a result, its scalability has been limited and
				resulted in significantly reduced usage in recent work.
			
			\subsubsection{Partitioning placement}
				
				Partitioning based placement solves the placement problem using
				graph-partitioning recursively on the problem graph to assign each part
				of the circuit to some area in the super chip. Though a number of
				algorithms have proven successful in academic placement contests over
				the years, they are not popular in industrial settings.
			
			\subsubsection{Analytical placement}
				
				In analytical placement, cost function for the circuit graph is
				approximated in a form which is amenable to solutions with standard
				numerical or symbolic algebraic techniques. Using these techniques,
				exact minimum cost (in terms of the approximation) configurations can
				be obtained.
				
				Quadratic placement is a popular analytical placement technique which
				approximates the cost of a placement as the sum of the squares of the
				distances between connected circuit elements.
				
				TODO: FIGURE EXAMPLE QUADRATIC PLACEMENT PROBLEM AND SOLUTION
				
				As such this gives a quadratic cost function like so which we must
				minimise.
				
				TODO: QUADRATIC COST EQN
				
				To minimise the function we differentiate and solve using simple
				symbolic manipulation.
				
				TODO: QUADRATIC COST SOLUTION
				
				Unfortunately, quadratic placement doesn't contain any congestion
				relief by default so various schemes exist. For example, extra anchor
				nodes are inserted which gently pull the circuit components apart from
				each other. As a result, the algorithm generally proceeds by iterating,
				regenerating anchors each time.
				
				Other non-quadratic analytical methods exist too with numerical
				solutions. The approaches are often similar.
			
			\subsubsection{Hierarchical clustering}
				
				Many placement algorithms scale super-linearly with problem size and so
				larger problems become increasingly problematic to handle. To solve
				this problem clustering techniques are first applied to first simplify
				the placement problem. A solution is then found at the coarse level and
				then hierarchically fleshed out.
				
				Various clustering algorithms are in use.
				
				TODO: TALK ABOUT CLUSTERING IN PLACEMENT...
				
				TODO: DESCRIBE THE ALGORITHM I IMPLEMENTED.
	
	\section{Application placement by simulated annealing}
		
		\label{sec:placement-by-annealing}	
		
		I have implemented a simplified SA based application placement algorithm
		based on the approach used in the popular VPR place and route tool chain.
		The algorithm is written in C and is optimised for experimentation rather
		than performance but is production-ready. It has been integrated into the
		`Rig' SpiNNaker software tools and has been used to place very large
		simulations. More on that later.
		
		\subsection{Representation}
			
			Model each chip as having a quantity of various resources (e.g. Cores,
			SDRAM) available. The application graph consists of vertices which each
			consume some quantity of these resources. Vertices must be placed on a
			single chip such that the resources required on a given chip do not
			exceed those available. Vertices are then interconnected by 1:N nets with
			weights which act as hints. The nets are treated as a soft constraint:
			vertices connected via a net will, ideally, be placed near to each other,
			with priority being given to nets with higher weights. Additionally there
			will be a list of placement constraints (see later).
			
			A key observation is that while vertices in an application may frequently
			have a 1:1 correspondence with application cores, this need-not be the
			case. For example, a vertex may represent a block of SDRAM which is
			shared. A vertex may also represent some other resource, for example,
			external IO availability. By making these resource types user-defined,
			applications programmers can express flexible hard-constraints on their
			application.
			
			Another observation is that generic soft constraints can be expressed may
			be expressed using a net with an appropriate weight.
			
			As a result of these facilities, application programmers can easily
			express their own application-specific hard and soft placement
			constraints without having to modify the algorithm. This representation
			has become a de-facto standard for placement problem interchange for
			SpiNNaker applications.
		
		\subsection{Cost function}
			
			At present I've used HPWL despite this being really bad for high-fan-out
			multicast and totally ignorant to the hexagonal nature of SpiNNaker...
			
			To compute bounding boxes for tori I use the following approach. For each
			dimension, sort the points on that dimension and find the largest gap
			between them on a ring. The bounding box goes the other way.
			
			TODO: FIGURE ILLUSTRATING BOUNDING BOX COMPUTATION FOR TORI.
		
		\subsection{Annealing schedule}
			
			The annealing schedule is that used by VPR. Despite being for circuit
			placement, it seems to work jolly well.
			
			TODO: DESCRIBE AND RATIONALISE THE SCHEDULE
		
		\subsection{Constraint handling}
			
			Various hard and soft constraints may be expressed by software
			approaches. For each we explain how they may be handled by the placement
			algorithm:
			
			\subsubsection{Location Constraint}
				
				The vertex is placed on a chip and removed from the set of movement
				candidates.
			
			\subsubsection{Same-chip constraint}
				
				When two vertices must always be placed on the same chip they are
				simply combined into one vertex which consumes the sum of their
				resources. Placement then treats them as one chip and thus is forced to
				atomically place the vertices.
			
			\subsubsection{Reserve resource constraint}
				
				Simply reduce resource availability on that chip.
			
			\subsubsection{Keep near Ethernet}
				
				Simply add a net.
	
	\section{Evaluation}
		
		\label{sec:placement-results}
		
		Though benchmarks exist for super computer loads and chip placement tasks,
		such things don't exist for neural applications. As a result I use a
		selection of real applications for SpiNNaker along with some synthetic
		benchmarks based on biological data.
		
		\subsection{Benchmark networks}
			
			First some real networks.
			
			Some nengo networks: SPAUN: `The world's largest functional brain model'.
			Word-net network from Jamie: Example of some learning.
			
			TODO: DESCRIBE SHAPE OF NENGO NETWORKS
			
			Some PyNN networks: Microcortical column model from PyNN. Note almost
			broadcast connectivity but varying weights. Try and extract a vision
			netlist from Anna. Maybe try and get a netlist for Tom's barrel cortex.
			
			Now for some artificial networks. Pipeline, noisy pipeline, mesh,
			Gaussian 2D.
		
		\subsection{Experiments}
			
			We compare random, linear, greedy and annealing based placement
			approaches to placement. We compare static metrics (such as mean/max
			congestion, table usage) along with experiments based on simulated
			network traffic in real hardware. Network Tester generates artificial
			traffic in proportion with the weights given for each model. We compare
			the relative level of traffic sustainable. We also consider use of
			machines of various sizes.
		
		\subsection{Results}
			
			SA placement is slow but rather effective, especially for some networks.
			Generally worth doing. Will need to be sped up for very large machines...
			
			TODO: EXPERIMENTS!
	

	\chapter{Discussion}

\section{Suitability of the hexagonal torus topology}
	\subsection{Physical scalability}
	\subsection{Routability}
	\subsection{Placeability}

\section{Suitability of the SpiNNaker router}
	\subsection{Deadlock avoidance}
	\subsection{Routing table size}

\section{Suitability of circuit placers for application placement}
	\subsection{Quality}
	\subsection{Runtime}
	\subsection{Routing resources}
	\subsection{Flexibility}
	\subsection{Scalability}


	\chapter{Future research}
	
	In this thesis I have presented a number of new techniques which have made it
	possible to assemble and operate the SpiNNaker super computer. This work
	opens up a range of possibie lines of research to extend this work to future
	architectures and applications. In this chapter I focus on two anticipated
	challenges of future systems: growing scale and greater dynamicism in
	applications.
	
	\section{Scaling up}
		
		TODO: INTRO
		
		\subsection{Grid machine room layouts}
			
			In chapter XXX, I developed a machine room layout for hexagonal torus
			topologies which allowed machines occupying a row of standard
			machine-room cabinets to scale up without the need for long
			interconnecting cables. For larger installations, however, it will be
			necessary to employ multiple rows of cabinets in a 2D arrangement.
		
		\subsection{Routing congestion control}
		
		\subsection{Parallel place and route}
	
	\section{Structural plasticity and dynamic fault-tolerance}
		\subsection{Plasticity models}
		\subsection{Incremental placement}
		\subsection{Incremental routing}
		\subsection{Hot-spare routes}

	\chapter{Conclusions and future research}
	
	The SpiNNaker architecture was designed to tackle the challenges presented by
	the simulation of biologically realistic neural networks. One of its
	distinguishing features is its network architecture which employs both an
	unconventional network topology and multicast router architecture. The
	hexagonal torus topology used by SpiNNaker was chosen to enable greater
	performance while maintaining ease of construction and scalability compared
	with conventional network topologies. SpiNNaker's router design centres
	around packets mimicking the neural `spike' signals they are designed to
	convey by being small, multicast and not guaranteed to arrive at their
	destination.  This novel design, though largely complete before this work
	began, left a number of open problems which this thesis has attempted to
	address.
	
	In this concluding chapter I begin by summarising the answers to the research
	questions raised in chapter~\ref{sec:introduction}. This is followed by a
	discussion of new research topics which have been uncovered by this work.
	
	\section{Answers to research questions}
		
		Each of the three research questions are answered below.
		
		\subsubsection{1. Can the hexagonal torus topology be deployed and used in
		real, large-scale systems?}
		
		In chapter~\ref{sec:building}, I introduced a cabling scheme and assembly
		technique which has been used successfully to build a prototype SpiNNaker
		system with over half a million processor cores using the hexagonal torus
		topology. The techniques shown are expected to enable a final SpiNNaker
		machine of double this size to be built, filling a six metre long row of
		machine-room cabinets.
		
		Though SpiNNaker's processor-count places it amongst some of the world's
		largest supercomputers (see figure \ref{fig:top500-num-processors} on page
		\pageref{fig:top500-num-processors}), it is comparatively compact, filling
		one row of cabinets compared with the warehouse-scale installations found
		in commercial systems. In spite of this, the folding and interleaving
		techniques described allow hexagonal torus topologies to scale to
		arbitrarily large installations without cables which span the machine.
		
		Chapter~\ref{sec:shortestPaths} described an efficient and general
		technique for finding, and enumerating shortest path vectors in hexagonal
		torus topologies. These developments bring the hexagonal torus topology in
		line with other topologies by enabling routing algorithms to exploit all
		possible paths in a network. Further, chapter~\ref{sec:placement}
		demonstrated that placement algorithms are also adaptable to hexagonal
		torus topologies thanks to their similarity to 2D toruses.
		
		Though, as this thesis highlights, hexagonal toruses lack many of the
		intuitive properties enjoyed by other topologies, it is still possible to
		reason about them with only limited computational effort.  Now that the
		practicality and scalability of the topology has also been demonstrated in
		practice, it represents a credible option for future network architectures.
		
		\subsubsection{2. Does SpiNNaker's router architecture help, or hinder
		fault tolerance?}
		
		SpiNNaker's unconventional use of packet dropping to avoid deadlocks
		greatly simplifies the router architecture, part of the motivation for this
		design. In chapter~\ref{sec:routing} this feature is used to the advantage
		of PGS repair to add fault tolerance to existing routing algorithms.
		Compared with the often complex and wasteful methods used to tolerate
		faults in other networks, PGS repair incurs very little performance
		overhead in the presence of static faults.
		
		Routing table usage does increase in the presence of faults, however, which
		may be a concern for applications which already require many routing table
		entries. Routing table usage, as well as other overheads, were most
		significantly increased in the presence of contiguous groups of network
		faults. This is because the PGS repair algorithm produces routes which pass
		tightly around the corners of faults, resulting in concentrations of
		routing table entries in those areas.  Though the symptoms of this problem
		can be attributed to the design of SpiNNaker's multicast routing mechanism,
		the responsibility lies with the behaviour of the PGS repair algorithm.
		Potential improvements to the PGS repair algorithm are discussed later in
		\S\ref{sec:pgs-repair-improvements}.
		
		The overall answer to this research question, therefore, is that the
		flexibility provided to routing algorithms in SpiNNaker's architecture is
		of great benefit, enabling arbitrary fault patterns to be inexpensively
		avoided.
		
		\subsubsection{3. How can the parts of a neural simulation be placed onto a
		large hexagonal torus topology to reduce network load?}
		
		In chapter~\ref{sec:placement}, I explored a number of contemporary
		approaches to the problem of placing applications with irregular
		communication patterns into network topologies. I observed that researchers
		working on circuit placement for chips and FPGAs are tackling similar
		problems and working at scales as large, or larger than, those faced in
		application placement. Based on this I developed a
		simulated annealing based placement algorithm inspired by the techniques
		used in circuit placement, with specific adaptations for use in application
		placement and SpiNNaker's network topology.
		
		The simulated annealing based placement algorithm consistently outperforms
		pre-existing placement algorithms included in benchmarks in terms of
		placement quality.  In the case of one benchmark, simulated annealing based
		placement made it possible to run that neural simulation in real-time for
		the first time.  At larger scales, simulated annealing was also found to be
		able to produce good quality placements in benchmarks containing over one
		million processes -- the largest size supported by the SpiNNaker
		architecture.
		
		The major shortcoming of simulated annealing based placement is its
		execution speed. Though its execution time grows in proportion to the size
		of the problem, the implementation used took over 12~hours to place a
		synthetic problem for the largest planned SpiNNaker machine. Though
		tractable -- particularly given the relative output quality compared with
		the prior state-of-the-art -- the algorithm is unlikely to function
		comfortably as-is on larger problems.
		
		The conclusion to be drawn from this result, however, is not just that
		simulated annealing is a good solution for today's placement problems but
		that circuit placement techniques in general could be successfully adapted
		to fulfil this role. The placement problems faced by chip designers are
		growing at roughly the same exponential rate as the size of super computers
		but circuit designs hold the lead in terms of problem size. Consequently,
		as approaches are retired by chip placement researchers, they may find new
		life in the field of application placement.
		
	\section{Future research}
		
		Though the goals of this study have largely been met, there also remain
		some important limitations which future work may hope to address.
		Furthermore, this work has uncovered a number of new research areas
		warranting future enquiry. This section outlines a number of future lines
		of research.
		
		\subsection{Warehouse-scale cabling}
			
			In chapter~\ref{sec:building} I developed and implemented a number of
			cabling schemes for the SpiNNaker architecture spanning up to a six metre
			row of machine-room cabinets -- a relatively small installation by
			current standards. In SpiNNaker, the cabling exists in a 2D plane (i.e.
			across the faces of the cabinets) but as the system is scaled up, a
			single row of cabinets will tend towards a 1D line. Since embedding a 2D
			structure in a 1D space necessarily results in long connections, this
			cannot scale indefinitely.
			
			\begin{figure}
				\center
				\buildfig{figures/multi-row-cabling.tex}
				
				\caption{Multiple rows of interconnected cabinets.}
				\label{fig:multi-row-cabling}
			\end{figure}
			
			In conventional large-scale super computer installations, nodes are
			installed in rows of cabinets as illustrated in
			figure~\ref{fig:multi-row-cabling}.  From a `bird's-eye' view, the system
			approximates a 2D space, spread across the floor of a machine-room.
			Therefore, in principle, the folding and interleaving techniques
			described in chapter~\ref{sec:building} still apply. Unfortunately for
			SpiNNaker, cables connecting between rows of cabinets would be longer
			than the one metre limit imposed by its hardware because of the spacing
			between rows of cabinets.  Future SpiNNaker systems will need to consider
			alternative link technologies.  For example, a hybrid system could be
			used in which intra-cabinet connections continue to use the current HSS
			link technology while inter-cabinet links might use optical connections.
			This type of architecture could be supported by the use of pluggable
			`SFP+' transceiver modules~\cite{sff01}.
		
		\subsection{Cabling assistance for other architectures}
			
			A secondary result of the construction of prototype SpiNNaker systems in
			chapter~\ref{sec:building} was the use of real-time guidance and feedback
			to assist cable installation. I am not aware of this technique's use by
			existing architectures and, following the success experienced in this
			project, it is possible that the technique may also be useful in
			conventional systems.
			
			During the construction of prototype SpiNNaker machines, each cable took
			seconds to install compared with the minutes reported for existing
			systems~\cite{mudigonda11}. Part of this increase in efficiency appears
			to result from the immediate identification of mistakes made during
			cabling, saving time-consuming backtracking later on.
			
			In many real-world network installations, units are less densely packed
			than in SpiNNaker and so longer cables are often required. As a
			consequence, cabling errors may become more likely as cabling patterns
			are spread over a larger area making them more difficult to visually
			verify. Like SpiNNaker, conventional networking hardware is often
			equipped with a generous range of indicator LEDs and diagnostic
			facilities which might be used to implement real-time installation
			guidance. Future work could explore the use of this technique in the
			construction of other large-scale networks, such as data centres.
		
		\subsection{Congestion mitigation}
			
			\label{sec:wiggly-board-allocations}
			
			In chapter~\ref{sec:routing} I found that contiguous network faults cause
			hot-spots of congestion and routing table depletion where the PGS repair
			algorithm routed many paths around the edges of faults.  However, it is
			not just faults which can cause contiguous blockages in the network
			topology. In reality, researchers do not always require a full-sized
			SpiNNaker system to perform their experiments so large SpiNNaker systems
			are soft-partitioned on demand into many smaller
			machines~\cite{spalloc16}. To ensure isolation between partitioned
			sub-machines, HSS links between boards in different partitions are
			disabled. Because of SpiNNaker's `wrapped triple' partitioning scheme,
			the resulting sub-machines have hexagonal \emph{mesh} topologies (i.e.
			without wrap-around links) with irregular boundaries as in
			figure~\ref{fig:spalloc-mesh}.
			
			\begin{figure}
				\center
				\buildfig{figures/spalloc-mesh.tex}
				
				\caption[Irregular edges of a partitioned SpiNNaker system.]%
				{Irregular edges in a SpiNNaker system comprised of 24~boards
				partitioned from a larger machine.  Each hexagon represents a SpiNNaker
				chip. No wrap-around connections are present.}
				\label{fig:spalloc-mesh}
			\end{figure}
			
			In partitioned systems, the `tooth'-like gaps on the periphery of the
			network result in similar congestion to the HSS link failures considered
			in chapter~\ref{sec:routing}. When a route is generated between nodes on
			opposite sides of a gap, the PGS repair process will produce a
			shortest-path route around it. Since many routes may be blocked by a
			single gap, a hot-spot may develop around the corners of the gap.
			
			In chapter~\ref{sec:placement}, the `CConv' benchmark application was
			found to run correctly the majority of the time when placed by the
			simulated annealing algorithm but would occasionally fail by a
			significant margin. Preliminary experiments suggest these occasional
			failures are caused by placement solutions which place heavily
			communicating parts of the application on opposite sides of gaps along
			the perimeter of the network. Two possible approaches which future work
			may consider are presented below.
			
			\subsubsection{Avoiding hotspots with PGS repair}
				
				\label{sec:pgs-repair-improvements}	
				
				Network congestion around faults and network irregularities could be
				reduced by forcing the PGS repair process to take more varied routes
				around faults. For example, in circuit routing algorithms, routers
				avoid congestion by increasing the cost of routes which pass through
				congested areas~\cite{kahng11}. A similar technique could be used in
				PGS repair to spread the routes it produces.
				
				An alternative approach would be to adapt the base routing algorithms
				used prior to PGS repair to, for example, attempt alternative dimension
				order routes which may avoid blockages due to faulty links.
			
			\subsubsection{Fault and irregularity aware placement}
				
				One of the shortcomings of the simulated annealing based placer
				developed in chapter~\ref{sec:placement} is that it does not account
				for network faults, or irregularities, when estimating the cost of
				placement solutions.  Future work may exploit techniques used in
				congestion-aware circuit placement which could be adapted for
				application placement~\cite{viswanathan07}.
		
		\subsection{Reducing placement execution time}
			
			The simulated annealing based placer presented in
			chapter~\ref{sec:placement} produced good quality placements but its
			execution time limits its use beyond one million vertex placement
			problems. Future work should explore possibilities for improving the
			performance and scalability of this technique.
			
			In addition to considering alternative placement algorithms based on
			other methods, one possible approach is to attempt to reduce the execution
			time of simulated annealing based placement by shrinking the application
			graph being placed.
			
			For example, graph clustering~\cite{schaeffer07} may be used to group
			together strongly connected vertices which would then be placed as a
			single unit.  Unfortunately, clustering can suffer from the same problems
			as graph-partitioning-based placement: vertices may be grouped together
			in ways which, in practice, cannot be packed together into a given portion
			of a machine.  A possible solution to this problem is to use a two-phase
			placement approach~\cite{kahng11}. In a `global' placement phase,
			solutions are permitted which can slightly over-allocate resources but
			overall achieve good placement quality. In the `detailed' placement phase
			which follows, the solution is `legalised' by making small changes to the
			global placement to eliminate over allocation.
			
			An alternative approach suited to SpiNNaker could be to limit the
			clustering process to clusters which fit on a single SpiNNaker chip. In
			typical SpiNNaker application graphs, clustering to this level may reduce
			placement problem sizes by an order of magnitude and, consequently,
			reduce execution times by the same ratio. Preliminary experiments suggest
			that this approach might result in little placement quality loss for
			large placement problems whilst substantially reducing overall execution
			time.
		
		\subsection{Benchmarking}
			
			One of the most significant limitations of this study has been the
			unavailability of large-scale SpiNNaker applications for use as
			benchmarks. As a consequence, much of the scalability experimentation
			performed has relied on simple synthetic benchmarks based on projections
			of future application behaviour.
			
			In the short term, more sophisticated synthetic benchmark generation
			techniques used by the circuit placement community~\cite{nam07} may offer
			alternative benchmarks for future work. In the longer term, however, it
			is hoped that the availability of large SpiNNaker systems -- and
			placement and routing algorithms better suited to exploit them -- will
			lead to larger scale applications being developed. Hopefully these
			applications will lead to more interesting and representative benchmarks
			for use in future work.
	
	\section{Closing remarks}
		
		One of the primary outcomes of this work is that a number of the practical
		challenges faced in scaling up the SpiNNaker architecture have been
		addressed leading to the construction of large-scale SpiNNaker machines.
		The development of an effective placement algorithm for SpiNNaker
		applications has been shown to enable some neural simulations to exploit
		SpiNNaker's architecture for the first time. The availability of larger
		SpiNNaker machines paves the way for future large-scale neural modelling
		work built on much larger models such as Spaun, `the world's largest
		functional brain model'~\cite{eliasmith12}.
		
		Beyond the SpiNNaker project, the hexagonal torus topology has also been
		validated as a scalable and practical candidate for future network
		architectures. As super computers become ever larger, the physical
		scalability afforded by the 2D nature of the hexagonal torus topology may
		make it a compelling option. In addition, the finding that circuit
		placement techniques can be adapted to support placement of SpiNNaker
		software indicates that these algorithms may also be applicable to other
		applications. Indeed, if this is the case, circuit placement may offer a
		long-term source of placement algorithms able to handle the demands of
		future applications.
		
		% This thesis has explored and tackled a number of the challenges posed in
		% scaling up the unconventional SpiNNaker architecture. Along the way I have
		% demonstrated that the hexagonal torus topology may be a practical choice in
		% future applications which can scale up to the physical dimensions expected
		% of future super computers. I have also developed new efficient and
		% effective methods of placing and routing neural simulation software on
		% SpiNNaker which -- it is hoped -- will enable a new generation of large
		% scale neural simulations on spinnaker.
		
		Although this work stops short of demonstrating truly large-scale
		neuroscientific simulations running at the scale of newly completed
		SpiNNaker machines (largely because such simulations do not yet exist) a
		number of smaller-scale neural simulations have been made possible for the
		first time. The algorithms and techniques devised in this work have
		subsequently been incorporated into various software libraries and tools
		now being used by researchers building SpiNNaker applications, vindicating
		the efforts of this thesis (see appendix~\ref{sec:software}). A successor
		to the SpiNNaker architecture is also in the early stages of design and is
		building on experience of the existing architecture. The current intention
		is to retain the hexagonal torus topology used by SpiNNaker, a decision
		supported by the findings of this thesis.
		
		With SpiNNaker's hardware architecture now operating at scales close to its
		architectural limits, it is hoped that the contributions of this work will
		enable researchers to develop larger and more detailed neural models for
		this unique architecture.

	
	% Bibliography
	\bibliography{references}
	\bibliographystyle{alpha}
	
\end{document}
};
	\end{tikzpicture}
}}

%%%%%%%%%%%%%%%%%%%%%%%%%%%%%%%%%%%%%%%%%%%%%%%%%%%%%%%%%%%%%%%%%%%%%%%%%%%%%%%%
% Document body
%%%%%%%%%%%%%%%%%%%%%%%%%%%%%%%%%%%%%%%%%%%%%%%%%%%%%%%%%%%%%%%%%%%%%%%%%%%%%%%%
\begin{document}
	
	% The title page
	\begin{titlepage}
	
	
	\begin{center}
		
		\vspace*{1.0in}
		
		{\LARGE\textbf{\thesistitle}}
		
		\vfill
		
		\textsc{A thesis submitted to the University of Manchester\\for the degree of Doctor
		of Philosophy\\in the Faculty of Science and Engineering.}
		
		\vfill
		
		\thesisyear
		
		\vfill
		
		\thesisauthor
		
		\vfill
		
		School of Computer Science
		
		\vfill
		
		\color{gray}{
			{\tiny{}Revision \texttt{\documentclass[12pt,twoside]{report}

\newcommand{\thesistitle}{Building and operating large-scale SpiNNaker machines}
\newcommand{\thesisauthor}{Jonathan Heathcote}
\newcommand{\thesisyear}{2016}

%%%%%%%%%%%%%%%%%%%%%%%%%%%%%%%%%%%%%%%%%%%%%%%%%%%%%%%%%%%%%%%%%%%%%%%%%%%%%%%%
% Used packages
%%%%%%%%%%%%%%%%%%%%%%%%%%%%%%%%%%%%%%%%%%%%%%%%%%%%%%%%%%%%%%%%%%%%%%%%%%%%%%%%

% Nice printing of URLs
\usepackage{url}

% Actually not tear-your-eyes-out-ugly tables
\usepackage{booktabs}

% Adjust linespacing for localised parts of the paper (e.g. abstract)
\usepackage{setspace}

% For the \ifthenelse macro
\usepackage{ifthen}

% For the \degree macro
\usepackage{gensymb}

% For subfigure support
\usepackage{caption}
\usepackage{subcaption}

% SI unit and number formatting
\usepackage{siunitx}

% Used to draw labels with a white outline to make them stand-out in diagrams
\usepackage[outline]{contour}

% TikZ + PGF Plots for diagram/plot drawing
\usepackage{tikz}
\usepackage{tikz3d}
\usepackage{pgfplots}
\usetikzlibrary{ hexagon
               , calc
               , backgrounds
               , positioning
               , decorations.pathreplacing
               , decorations.markings
               , arrows
               , positioning
               , automata
               , shadows
               , fit
               , shapes
               , arrows
               , patterns
               , spy
               }
\usepgfplotslibrary{statistics}

%%%%%%%%%%%%%%%%%%%%%%%%%%%%%%%%%%%%%%%%%%%%%%%%%%%%%%%%%%%%%%%%%%%%%%%%%%%%%%%%
% Environment settings
%%%%%%%%%%%%%%%%%%%%%%%%%%%%%%%%%%%%%%%%%%%%%%%%%%%%%%%%%%%%%%%%%%%%%%%%%%%%%%%%

% 1.5 linespacing (as required by university)
\renewcommand{\baselinestretch}{1.5}


% Specifies the thickness of the contour added by the \contour macro.
\contourlength{1.5pt}

% Define a few layers for TikZ to allow easier layering
\pgfdeclarelayer{bg}
\pgfdeclarelayer{fg}
\pgfsetlayers{bg,main,fg}

%%%%%%%%%%%%%%%%%%%%%%%%%%%%%%%%%%%%%%%%%%%%%%%%%%%%%%%%%%%%%%%%%%%%%%%%%%%%%%%%
% Definitions
%%%%%%%%%%%%%%%%%%%%%%%%%%%%%%%%%%%%%%%%%%%%%%%%%%%%%%%%%%%%%%%%%%%%%%%%%%%%%%%%

% Used in place of \chapter for preface sections. Prevents numbering but
% includes the chapter in the ToC
\newcommand{\prefacesection}[1]{
	\chapter*{#1}
	\addcontentsline{toc}{chapter}{#1}
}

% Adds 'discard if' and  'discard if not' options for \addplot to enable
% filtering of data. Taken from
% http://tex.stackexchange.com/questions/58548/is-it-possible-to-change-the-color-of-a-single-bar-when-the-bar-plot-is-based-on
\pgfplotsset{
    discard if/.style 2 args={
        x filter/.code={
            \edef\tempa{\thisrow{#1}}
            \edef\tempb{#2}
            \ifx\tempa\tempb
                \def\pgfmathresult{inf}
            \fi
        }
    },
    discard if not/.style 2 args={
        x filter/.code={
            \edef\tempa{\thisrow{#1}}
            \edef\tempb{#2}
            \ifx\tempa\tempb
            \else
                \def\pgfmathresult{inf}
            \fi
        }
    }
}

% Make PGFplots Treat "NA" (regardless of letter case) as "nan". From:
% http://tex.stackexchange.com/questions/110441/skip-specific-string-in-a-numeric-column-while-using-pgfplots
\makeatletter
\expandafter\def\csname pgffltA@N\endcsname{\pgfflt@readundef}
\expandafter\def\csname pgffltA@n\endcsname{\pgfflt@readundef}
\def\pgfflt@readundef #1{%
    \def\pgfflt@readnan@ok{1}%
    \if#1a\else\if#1A\else\def\pgfflt@readnan@ok{0}\fi\fi
    \if\pgfflt@readnan@ok1%
        \pgfmathfloat@a@S=3\relax%
        \pgfmathfloat@a@Mtok={0.0}%
        \pgfmathfloat@a@E=0%
        \expandafter\pgfflt@finish
    \else
        \def\pgfflt@readnan@{\pgfflt@error #1}%
        \expandafter\pgfflt@readnan@
    \fi
}
\makeatother

%%%%%%%%%%%%%%%%%%%%%%%%%%%%%%%%%%%%%%%%%%%%%%%%%%%%%%%%%%%%%%%%%%%%%%%%%%%%%%%%
% Document body
%%%%%%%%%%%%%%%%%%%%%%%%%%%%%%%%%%%%%%%%%%%%%%%%%%%%%%%%%%%%%%%%%%%%%%%%%%%%%%%%
\begin{document}
	
	% The title page
	\begin{titlepage}
	
	
	\begin{center}
		
		\vspace*{1.0in}
		
		{\LARGE\textbf{\thesistitle}}
		
		\vfill
		
		\textsc{A thesis submitted to the University of Manchester\\for the degree of Doctor
		of Philosophy\\in the Faculty of Science and Engineering.}
		
		\vfill
		
		\thesisyear
		
		\vfill
		
		\thesisauthor
		
		\vfill
		
		School of Computer Science
		
		\vfill
		
		\color{gray}{
			{\tiny{}Revision \texttt{\input{thesis.fullhash}}\input{thesis.date}}
		}
		
	\end{center}
	
\end{titlepage}

	
	% The table of contents which, per university regulations, is followed by a
	% total wordcount.
	\tableofcontents
	\vfill
	\noindent This thesis contains
		\immediate\write18{texcount -1 -sum -inc thesis.tex > thesis.wordcount}%
		\documentclass[12pt,twoside]{report}

\newcommand{\thesistitle}{Building and operating large-scale SpiNNaker machines}
\newcommand{\thesisauthor}{Jonathan Heathcote}
\newcommand{\thesisyear}{2016}

%%%%%%%%%%%%%%%%%%%%%%%%%%%%%%%%%%%%%%%%%%%%%%%%%%%%%%%%%%%%%%%%%%%%%%%%%%%%%%%%
% Used packages
%%%%%%%%%%%%%%%%%%%%%%%%%%%%%%%%%%%%%%%%%%%%%%%%%%%%%%%%%%%%%%%%%%%%%%%%%%%%%%%%

% Nice printing of URLs
\usepackage{url}

% Actually not tear-your-eyes-out-ugly tables
\usepackage{booktabs}

% Adjust linespacing for localised parts of the paper (e.g. abstract)
\usepackage{setspace}

% For the \ifthenelse macro
\usepackage{ifthen}

% For the \degree macro
\usepackage{gensymb}

% For subfigure support
\usepackage{caption}
\usepackage{subcaption}

% SI unit and number formatting
\usepackage{siunitx}

% Used to draw labels with a white outline to make them stand-out in diagrams
\usepackage[outline]{contour}

% TikZ + PGF Plots for diagram/plot drawing
\usepackage{tikz}
\usepackage{tikz3d}
\usepackage{pgfplots}
\usetikzlibrary{ hexagon
               , calc
               , backgrounds
               , positioning
               , decorations.pathreplacing
               , decorations.markings
               , arrows
               , positioning
               , automata
               , shadows
               , fit
               , shapes
               , arrows
               , patterns
               , spy
               }
\usepgfplotslibrary{statistics}

%%%%%%%%%%%%%%%%%%%%%%%%%%%%%%%%%%%%%%%%%%%%%%%%%%%%%%%%%%%%%%%%%%%%%%%%%%%%%%%%
% Environment settings
%%%%%%%%%%%%%%%%%%%%%%%%%%%%%%%%%%%%%%%%%%%%%%%%%%%%%%%%%%%%%%%%%%%%%%%%%%%%%%%%

% 1.5 linespacing (as required by university)
\renewcommand{\baselinestretch}{1.5}


% Specifies the thickness of the contour added by the \contour macro.
\contourlength{1.5pt}

% Define a few layers for TikZ to allow easier layering
\pgfdeclarelayer{bg}
\pgfdeclarelayer{fg}
\pgfsetlayers{bg,main,fg}

%%%%%%%%%%%%%%%%%%%%%%%%%%%%%%%%%%%%%%%%%%%%%%%%%%%%%%%%%%%%%%%%%%%%%%%%%%%%%%%%
% Definitions
%%%%%%%%%%%%%%%%%%%%%%%%%%%%%%%%%%%%%%%%%%%%%%%%%%%%%%%%%%%%%%%%%%%%%%%%%%%%%%%%

% Used in place of \chapter for preface sections. Prevents numbering but
% includes the chapter in the ToC
\newcommand{\prefacesection}[1]{
	\chapter*{#1}
	\addcontentsline{toc}{chapter}{#1}
}

% Adds 'discard if' and  'discard if not' options for \addplot to enable
% filtering of data. Taken from
% http://tex.stackexchange.com/questions/58548/is-it-possible-to-change-the-color-of-a-single-bar-when-the-bar-plot-is-based-on
\pgfplotsset{
    discard if/.style 2 args={
        x filter/.code={
            \edef\tempa{\thisrow{#1}}
            \edef\tempb{#2}
            \ifx\tempa\tempb
                \def\pgfmathresult{inf}
            \fi
        }
    },
    discard if not/.style 2 args={
        x filter/.code={
            \edef\tempa{\thisrow{#1}}
            \edef\tempb{#2}
            \ifx\tempa\tempb
            \else
                \def\pgfmathresult{inf}
            \fi
        }
    }
}

% Make PGFplots Treat "NA" (regardless of letter case) as "nan". From:
% http://tex.stackexchange.com/questions/110441/skip-specific-string-in-a-numeric-column-while-using-pgfplots
\makeatletter
\expandafter\def\csname pgffltA@N\endcsname{\pgfflt@readundef}
\expandafter\def\csname pgffltA@n\endcsname{\pgfflt@readundef}
\def\pgfflt@readundef #1{%
    \def\pgfflt@readnan@ok{1}%
    \if#1a\else\if#1A\else\def\pgfflt@readnan@ok{0}\fi\fi
    \if\pgfflt@readnan@ok1%
        \pgfmathfloat@a@S=3\relax%
        \pgfmathfloat@a@Mtok={0.0}%
        \pgfmathfloat@a@E=0%
        \expandafter\pgfflt@finish
    \else
        \def\pgfflt@readnan@{\pgfflt@error #1}%
        \expandafter\pgfflt@readnan@
    \fi
}
\makeatother

%%%%%%%%%%%%%%%%%%%%%%%%%%%%%%%%%%%%%%%%%%%%%%%%%%%%%%%%%%%%%%%%%%%%%%%%%%%%%%%%
% Document body
%%%%%%%%%%%%%%%%%%%%%%%%%%%%%%%%%%%%%%%%%%%%%%%%%%%%%%%%%%%%%%%%%%%%%%%%%%%%%%%%
\begin{document}
	
	% The title page
	\input{titlepage}
	
	% The table of contents which, per university regulations, is followed by a
	% total wordcount.
	\tableofcontents
	\vfill
	\noindent This thesis contains
		\immediate\write18{texcount -1 -sum -inc thesis.tex > thesis.wordcount}%
		\input{thesis.wordcount}words.
	
	\clearpage
	\listoffigures
	
	\clearpage
	\listoftables
	
	% Abstract
	\input{abstract}
	
	% Declaration of non-submission elsewhere
	\input{declaration}
	
	% University-prescribed copyright statement...
	\input{copyright}
	
	% Acknowledgements
	\input{acknowledgements}
	
	% Main body
	\input{introduction.tex}
	\input{background.tex}
	\input{building.tex}
	\input{shortestPaths.tex}
	\input{routing.tex}
	\input{placement.tex}
	\input{discussion.tex}
	\input{future.tex}
	\input{conclusions.tex}
	
	% Bibliography
	\bibliography{references}
	\bibliographystyle{alpha}
	
\end{document}
words.
	
	\clearpage
	\listoffigures
	
	\clearpage
	\listoftables
	
	% Abstract
	{
	\prefacesection{Abstract}
	
	% Single line spacing for the abstract page
	\setstretch{1.0}
	
	
	\vfill
	
	% Standard thesis information
	\begin{center}
		\textsc{\large\thesistitle}
		
		\vspace{0.5em}
		
		\thesisauthor
		
		\vspace{0.5em}
		
		A thesis submitted to the University of Manchester\\
		for the degree of Doctor of Philosophy, 2016
	\end{center}
	
	\vfill
	
	% The abstract
	
	SpiNNaker is an unconventional super computer architecture designed to
	simulate up to one billion biologically realistic neurons in real-time. To
	achieve this goal, SpiNNaker employs a novel network architecture which poses
	a number of practical problems in scaling up from desktop prototypes to
	machine room filling installations.
	
	SpiNNaker's hexagonal torus network topology has received mostly theoretical
	treatment in the literature. This thesis tackles some of the challenges
	encountered when building `real-world' systems.  Firstly, a scheme is devised
	for physically laying out hexagonal torus topologies in machine rooms which
	avoids long cables; this is demonstrated on a half-million core SpiNNaker
	prototype.  Secondly, to improve the performance of existing routing
	algorithms, a more efficient process is proposed for finding (logically)
	short paths through hexagonal torus topologies. This is complemented by a
	formula which provides routing algorithms greater flexibility when finding
	paths, potentially resulting in a more balanced network utilisation.
	
	The scale of SpiNNaker's network and the models intended for it also present
	their own challenges. Placement and routing algorithms are developed which
	assign processes to nodes and generate paths through SpiNNaker's network.
	These algorithms reduce congestion and tolerate network faults. The proposed
	placement algorithm is inspired by techniques used in chip design and is
	shown to enable larger applications to run on SpiNNaker -- with good
	performance -- than the previous state-of-the-art. Likewise the routing
	algorithm developed is able to tolerate network faults, inevitably present in
	large scale systems, with little performance overhead.
	
	
	% Required to ensure single line spacing is used for this whole block
	\par%
}

	
	% Declaration of non-submission elsewhere
	\prefacesection{Declaration}

% Single line spacing for the declaration
{
	\setstretch{1.0}
	No portion of the work referred to in this thesis has been submitted in support
	of an application for another degree or qualification of this or any other
	university or other institute of learning.
	
	\par%
}


	
	% University-prescribed copyright statement...
	\input{copyright}
	
	% Acknowledgements
	{
	\prefacesection{Acknowledgements}
	
	% Single line spacing
	\setstretch{1.0}
	
	It is often said that it is not \emph{what} you know but \emph{who} you know.
	Throughout the course of my PhD I've been exceptionally lucky to have been
	helped along by a great number of people.
	
	Both my supervisor, Jim Garside, and co-supervisor, Steve Furber, have each
	spent countless hours patiently discussing and describing all manner of
	things with me while giving me great freedom in my project. Jim's office door
	has always been open to my unexpected interruptions be it about work, writing
	or walking.  Likewise, Steve has always managed to find time for both
	technical and frivolous endeavours alike. I'm also hugely grateful to Luis
	Plana who has been a rich source of sage advice, insightful questions
	patiently suffered many a foolish question.
	
	Various parts of the work in this thesis (and numerous side projects) would
	not have been possible if not for the multitude of discussions,
	collaborations and even sheer physical hard work of Steve Temple, Javier
	Navaridas, Simon Davidson and Dave Clark. I'm also indebted to Andrew Mundy
	and Jamie Knight, both of whom have donated so much time and effort towards
	verifying and using software implementations of the ideas in this thesis.
	
	The injection of lunchtime silliness by Andrew and Jamie, along with Amanieu
	d'Antras and Andrew Webb and the other CDT members has always brightened my
	day. So to has the friendly and stimulating environment of the School of
	Computer Science and its many staff and students. Of course, I am also very
	grateful for the funding the school has provided for my research.
	
	I cannot thank my wonderful wife, Ann-Marie, enough for being by my side. She
	has given me so much kindness, love and patience and endured a lifetime's
	quota of conversations about hexagons. Finally, thanks too to rest of my
	family, especially my parents, who are to blame for starting me down this
	path and co-suffering drafts and endless rants about this document.
	
	% Required to ensure single line spacing is used for this whole block
	\par%
}

	
	% Main body
	\chapter{Introduction}

\label{sec:introduction}

%Problem area
%
%* Network construction and exploitation
%  * Cabling: Build it cheaply in terms of tech cost and install cost
%  * Routing: Get around it cheaply and reliably
%  * Placement: Use it efficiently

The Spiking Neural Network Architecture (SpiNNaker) is a novel super computer
architecture designed to simulate biologically realistic models of brains in
real time \cite{furber07}. Though neurons, the building blocks of the brain,
are relatively well understood, their complex interactions remain mysterious.
Just as understanding the workings of a transistor is insufficient to
understand a modern microprocessor, neuroscientists believe that understanding
the neurons in isolation cannot explain the brain and that understanding their
connectivity is key \cite{eliasmith13,eliasmith14}. Experiments on real brains,
however, are fraught with difficulty. Variations between individuals can be
significant and it is only possible to record tens or hundreds of the trillions
of signals in the brain, and even then only with limited control over which
signals are recorded. Computer simulations of models of large neural networks,
however, enable researchers to develop repeatable experiments and gain complete
visibility of any signal and any neuron. Models such as SPAUN
\cite{eliasmith12}, built from millions of simulated neurons, have shown great
promise in expanding our understanding of higher level brain functions such as
memory and simple problem solving.  Unfortunately these neural models are
expensive to simulate, requiring hours of compute time to simulate each second
of neural activity. As well as being inconvenient, this precludes the use of
robotics to immerse these models in real world environments and also limits
studies of long-term behaviours such as learning.

SpiNNaker is designed to enable the real time simulation of models containing
up to one billion neurons -- approximately \SI{1}{\percent} of a human brain or
ten mouse brains \cite{furber06}. To achieve this goal, the largest planned
SpiNNaker machine will contain over one million low-powered computer processors
interconnected by a bespoke network architecture.

SpiNNaker's large processor count matches the current trend in super computers
where processor counts are growing exponentially \cite{meuer16j}, mimicking the
growth of the number of components in the processors themselves predicted by
Gordon Moore's famous `law' \cite{moore75}. As a result of this growth, the
interconnection networks which enable these processors to work together have
grown in importance \cite{dally04}.  Network designers must carefully balance
performance against practicality and financial cost.  SpiNNaker's network is no
exception to this rule and, as the systems scale up from desktop prototypes to
machine-room scale installations, the reality of building and exploiting these
machines presents an array of challenges.

As in all super computers, SpiNNaker's network interconnects its processors in
a particular network topology which defines how different processors may
communicate with each other. Unlike the tree and $N$-dimensional torus
topologies found in contemporary super computers \cite{dally04}, SpiNNaker
employs a `hexagonal torus topology'. In this topology, nodes in SpiNNaker's
network fit together in a honeycomb-like pattern where messages may `hop' from
node to node to reach their destination. As we will see in
chapter~\ref{sec:background}, the hexagonal torus topology, in theory, sits at
a `sweet spot' in terms of network performance and practicality. As the first
known large-scale installation of the hexagonal torus topology, however, there
remain a number of practical challenges for large spinnaker machines arising
from this choice.

As super computer networks have grown in scale to millions of processors the
task of dealing with previously rare faults has grown.  Though fault rates in
networks remain consistently low, architectures such as SpiNNaker may have
hundreds of thousands of links meaning even fault rates of a fraction of a
percent will impact tens or hundreds of links. To enable reliable operation,
networks must be able to adapt the routes taken by messages through the network
to avoid faulty links and nodes. The techniques employed are often closely tied
to a particular network architecture and consequently SpiNNaker's novel network
architecture demands its own approach.

Another challenge introduced by the growing scale of super computers is making
\emph{efficient} use of network resources. Communicating processes should be
located on logically `nearby' nodes to reduce network load. The neural models
for which SpiNNaker is designed are often described abstractly, rather than
geometrically, using modelling languages such as PyNN~\cite{davison08} and
Nengo~\cite{eliasmith04}.  Because of this, the communication requirements of
simulations can be highly irregular making an efficient placement of processes
onto processors in the machine non-trivial.

%Contributions
%
%* Cabling scheme for hexagonal toruses without long cables
%* Efficient installation technique for dense systems
%* Exhaustive and efficient route calculation in hex toruses
%* Fault tolerant routing scheme exploiting SpiNNaker's odd router
%* Placement based on SA a: works very well and b: suggests circuit placement is
%  a good source of inspiration.

This thesis addresses the practical challenges of scaling up the SpiNNaker
architecture in a real-world setting summarised by these research questions:

\begin{enumerate}
	
	\item Can the hexagonal torus topology be deployed and used in real, large
	scale systems?
	
	\item Does SpiNNaker's router architecture help, or hinder fault tolerance?
	
	\item How can the parts of a neural simulation be placed onto a large
	hexagonal torus topology to reduce network load?
	
\end{enumerate}

%Structure
%
%* Chapter 2: Background: detailed dive into what's in SpiNNaker, why its
%  really so unusual. Also looks at what applications run on SpiNNaker and how
%  they work.
%* Chapter 3: How to build a really big SpiNNaker machine.
%* Chapter 4: How to find your way around that machine.
%* Chapter 5: How to find your way around that machine even when its broken.
%* Chapter 6: Now you can walk, time to run.
%* Chapter 7: Wrapping up.
%* Appendices: Hard-to-come-by theoretical and practical details useful if
%  you're about to continue where this research left off but be useful but
%  otherwise hard to come by, especially in one place.

Chapter~\ref{sec:background} introduces the SpiNNaker architecture and, in
particular, describes its hexagonal torus topology and network architecture.

In chapter~\ref{sec:building}, I develop a cabling scheme for large hexagonal
torus topologies which enables arbitrarily large networks to be constructed
using only short, inexpensive cables. This theoretical work is then evaluated
through the construction of a range of prototype SpiNNaker systems. The largest
of these prototypes contains over half a million processor cores and spans
several machine room cabinets. In addition, I propose the use of built-in
diagnostic facilities to assist technicians performing network installation and
maintenance. This technique is found to greatly reduce the effort required and
the number of mistakes made.

In chapters~\ref{sec:shortestPaths}~and~\ref{sec:routing} I develop new routing
techniques for SpiNNaker's network. Chapter~\ref{sec:shortestPaths} develops a
new approach to finding the shortest paths through hexagonal torus topologies,
an integral part of many routing algorithms. This newly proposed approach is
cheaper to compute than the state of the art and, unlike previous efforts, is
able to discover all valid short paths through the topology. This theoretical
advance brings hexagonal torus topologies in line with conventional topologies
by providing routing algorithms with complete information about the paths
available to them. In chapter \ref{sec:routing} I propose a fault tolerant
routing algorithm for SpiNNaker which is able to avoid arbitrary static fault
patterns with minimal performance overhead. A key finding of this chapter is
that the flexibility afforded to fault tolerant routing algorithms by
SpiNNaker's unconventional router architecture is what facilities the low
overheads reported in this chapter.

Finally, in chapter~\ref{sec:placement}, I explore the problem of application
placement in SpiNNaker's network. As in other networks and applications, neural
simulations should be arranged such that communication occurs primarily between
processors close together in the network to control network load. Due to the
irregular connectivity and large scale of the neural models expected to run on
SpiNNaker, an automated approach is necessary. I develop a novel placement
algorithm based on algorithms used for circuit layout in computer chips. My
algorithm is found to allow some larger neural models to run on SpiNNaker for
the first time while enabling other applications to run at greater speeds. In
addition, synthetic benchmarks containing over one million processes indicate
that this algorithm should handle the anticipated demands of the neural models
expected to run on large-scale SpiNNaker installations.

	\chapter{The SpiNNaker Architecture}
	
	\label{sec:background}
	
	SpiNNaker is a massively parallel computer architecture designed to simulate
	biologically realistic neural models \cite{furber07}. In this chapter we will
	explore this unconventional architecture in detail, starting with its purpose
	before focusing on its most unconventional feature: its network.
	
	% * Purpose
	%   * Spiking neural simulations
	%     * Neural modelling: PyNN, Nengo...
	%     * Parallelisation + communication
	
	\section{Neural simulation}
		
		Human brains contain billions of neurons connected together by trillions of
		`synapses'. Neurons communicate by transmitting and receiving `spikes'
		through their synapses. Each spike is `valueless' in that a spike's only
		significant features are when it arrives and where it has come from.
		
		\begin{figure}
			\center
			\buildfig{figures/lif-neuron.tex}
			
			\caption{A Leaky Integrate-and-Fire (LIF) neuron.}
			\label{fig:lif-neuron}
		\end{figure}
		
		Though some detailed models of the electrochemical processes occurring
		inside neurons are computationally intensive, simplified models such as the
		Leaky Integrate-and-Fire (LIF) model can be implemented in just a handful
		of CPU instructions \cite{vainbrand11}. Figure~\ref{fig:lif-neuron}
		illustrates a simple LIF neuron in which incoming spikes cause charge to
		build up (integrated) which over time, leaks away. If an incoming spike
		causes the charge to rise above a certain threshold, the neuron `fires'
		producing an outgoing spike. Despite the simplicity of this model, large
		neural networks such as Spaun \cite{eliasmith12} -- built entirely from LIF
		neurons -- exhibit complex behaviours such as fine motor control and
		problem solving.
		
		The computational expense of large scale neural simulations does not arise
		from the cost of modelling neurons but instead from distributing spikes. In
		biology, neurons produce spikes at an average rate of \SI{10}{\hertz} and
		synapses connect each neuron's output to (order) \num{1000}~neurons
		\cite{navaridas09}. Consider an example neural model with $7\times10^7$
		neurons, approximately the number in a house mouse and
		$\nicefrac{1}{10}^\textrm{th}$ of the design target of SpiNNaker. This
		network might produce $7\times10^8$~spikes per second. Because each neuron
		connects to many others, this equates to $7\times10^{11}$ spikes being
		received per second. If each spike were transmitted as a UDP datagram
		containing a single \SI{32}{\bit} payload, the total network throughput
		required for this simulation would be \SI{179.2}{\tera\bit\per\second}. At
		the time of writing, this is more than double the bisection bandwidth (the
		theoretical worst-case throughput) of the world's most powerful super
		computer \cite{dongarra16}.
	
	\section{Network architecture}
		
		Architectures such as IBM's Blue Gene \cite{chiu11} and Cray's XK7
		\cite{ornl16} employ powerful compute nodes connected together using
		networks designed to transfer multi-kilobyte blocks of data between nodes.
		Since neural models have relatively light computational requirements and
		communications are based on small pieces of data (spikes), this type of
		architecture is poorly suited to the task.
		
		SpiNNaker's architectural target is to support realtime simulations of up
		to one billion neurons. Since neural models such as LIF are inexpensive to
		model and many neurons can be simulated independently in parallel,
		SpiNNaker employs many small, energy efficient ARM processors
		\cite{furber07}. To support the unusual communication requirements of
		neural simulations, a bespoke interconnection network is used which is the
		background to this thesis.
		
	%   * SpiNNaker chip
	%     * Cores
	%     * SDRAM
	%     * NoC
	%     * Router
		
		\begin{figure}
			\center
			%\includegraphics[width=19mm]{figures/spinnakerChip.jpg}
			\buildfig{figures/hex-chips.tex}
			
			\caption[SpiNNaker chips connected to their six neighbours.]%
			{SpiNNaker chips (actual size) connected to their six neighbours.}
			\label{fig:spinnakerChip}
		\end{figure}
		
		The fundamental building block of the SpiNNaker architecture is the
		SpiNNaker chip (figure \ref{fig:spinnakerChip}) \cite{furber13}. Each chip
		contains eighteen low power ARM 968 processor cores each capable of
		simulating between \num{200} and \num{2000} LIF neurons in real time
		\cite{mundy15}.  Each core has a total of \SI{96}{\kilo\byte} of private
		Tightly-Coupled Memory (TCM) and shares access to \SI{128}{\mega\byte} of
		on-chip SDRAM with other cores on the same chip. Finally, each chip
		contains a programmable router which routes network packets to and from the
		local cores and six neighbouring SpiNNaker chips. SpiNNaker machines are
		constructed by combining many SpiNNaker chips.
		
		\begin{figure}
			\center
			\buildfig{figures/spinnaker-packet.tex}
			
			\caption{SpiNNaker's \SI{40}{\bit} and \SI{72}{\bit} multicast packet
			format.}
			\label{fig:spinnaker-packet}
		\end{figure}
		
		Processor cores can communicate by sending and receiving network packets
		forwarded by routers through the network. Since SpiNNaker's network is
		designed to transmit neural spike events efficiently, individual network
		packets are small, either \SI{40}{\bit} or \SI{72}{\bit} compared with tens
		or hundreds of byte packets in typical network architectures.
		
		In a real-time simulation, the time at which a spike is produced is
		implicitly indicated by the time it is received -- since at biological
		timescales a computer network delivers packets `instantaneously'.
		Consequently, the only information which must be explicitly encoded is the
		identity of the neuron which produced the spike. In SpiNNaker, a spike may
		be encoded by using a single \SI{40}{\bit} `multicast packet' whose format
		is illustrated in figure~\ref{fig:spinnaker-packet}.  The \SI{8}{\bit}
		header is used by SpiNNaker's routers to determine the type of packet and
		the \SI{32}{\bit} `routing key' is used to identify the neuron which
		produced the packet. The routing key is also used by SpiNNaker's routers to
		determine how the packet should be directed through the network.
		
		The optional \SI{32}{\bit} payload is not used by conventional spiking
		neural simulations \cite{galluppi10} but has been exploited to enable more
		efficient simulation of a particular class of neural models \cite{mundy15}.
	
	\section{The SpiNNaker router}
		
		The SpiNNaker router employs an unconventional design which, despite its
		compact size and small energy requirements, implements a flexible multicast
		routing scheme. Unlike conventional routers which often employ hard-coded
		routing rules \cite[chapter~8]{dally04}, the SpiNNaker router uses a
		programmable `routing table' to determine how packets should be forwarded.
		In addition, to avoid deadlocks, SpiNNaker's router employs a simple,
		timeout-based mechanism which exploits the ability of neural networks to
		tolerate occasional missing packets. As we will see in chapter
		\ref{sec:routing}, this mechanism greatly simplifies the task of routing in
		SpiNNaker's network. In this section we'll look at these features in
		greater detail.
		
		\subsection{Routing tables}
		
			When a multicast packet arrives at a SpiNNaker router (either from a
			local core or a neighbouring chip), the router looks up the routing key
			in its routing table. This table consists of \num{1024} programmable
			table entries, each specifying a routing key bit pattern and mask to
			match and a set of routes.  When a multicast packet's key is matched by a
			routing entry the packet is forwarded along every route specified by that
			entry, potentially duplicating the packet. This `multicast' technique
			allows packets to be transmitted once but received in a number of places
			while making efficient use of the network \cite{navaridas12}.
			
			Though routing table entries are in finite supply (\num{1024} entries per
			router), it is still possible for many thousands of traffic flows to be
			routed through a single router. The bit pattern and mask in each routing
			entry matches against the 32~bits of a routing key as either
			`\texttt{1}', `\texttt{0}' or `\texttt{X}' (don't care).  This means that
			a single routing entry may, for example, be used to match all routing
			keys with a certain prefix. If a routing key is not matched by any entry
			in the routing table then the packet is `default routed' in a straight
			line. For example if a packet with an unmatched key is received from the
			chip to the left, the packet will be default routed to the chip on the
			right. By assigning routing keys such that neurons whose spikes are sent
			to similar destinations share a similar prefix, the number of routing
			entries required by a simulation is greatly reduced \cite{davies12}.
			
			\begin{figure}
				\center
				\buildfig{figures/routing-example.tex}
				
				\caption[Multicast routing example.]%
				{Multicast routing example with \SI{4}{\bit} routing keys. Each
				box represents a SpiNNaker chip whose router has been programmed with
				the routing entries shown. Grey lines mark connections between chips.}
				\label{fig:routing-example}
			\end{figure}
			
			Consider the simplified example in figure~\ref{fig:routing-example} in
			which a number of (\SI{4}{\bit}) routing table entries have been
			configured in the routers of a small SpiNNaker network. If a packet with
			the routing key \texttt{1011} is transmitted by a core in the chip
			labelled $(0, 0, 0)$, this will match the first routing table entry on
			that chip and will be routed to chip $(1, 0, 0)$. On chip $(1, 0, 0)$,
			the packet once again matches the first routing entry and is routed to
			chip $(1, 0, -1)$. On $(1, 0, -1)$, no match is made so the packet is
			default routed to $(1, 0, -2)$. On this chip, the packet matches a
			routing entry which routes the packet to core~7. In this example, default
			routing allows only three routing table entries to direct a packet
			through four chips.
			
			As a second example, if a packet with the routing key \texttt{0010} is
			transmitted by a core on chip $(0, 0, 0)$, this key will be matched by
			the second routing entry since \texttt{X}s in the table entry will match
			both \texttt{1}s and \texttt{0}s in the corresponding bits of the routing
			key. When the packet arrives at chip $(0, 0, -1)$ the matching routing
			entry forwards the packet to both $(0, 1, -1)$ and $(1, 0, -1)$
			simultaneously. The copy of the packet arriving at $(0, 1, -1)$ is routed
			to core~5 on that chip.  Meanwhile, the copy forwarded to $(1, 0, -1)$ is
			duplicated again with one copy being routed to core~11 and another being
			routed to chip $(1, 0, -2)$. Here the packet is finally delivered to
			core~6. In this example, the ability of the router to multicast
			(duplicate) packets as they pass through the network meant that sending
			one copy of the packet was sufficient to reach three destination cores.
			In addition, by using \texttt{X}s in the routing table entry, the same
			routing entries are sufficient to route packets with the keys
			\texttt{0000}, \texttt{0001}, \texttt{0010} and \texttt{0011}.
			
			In spite of these mechanisms, it is still possible for an application to
			run out of routing table entries. As we will see in
			chapter~\ref{sec:placement} by arranging applications appropriately
			within SpiNNaker's network, routing table usage can be reduced. In
			addition, other behaviours of SpiNNaker's router may be exploited to
			compress an applications routing tables further, however the techniques
			employed are beyond the scope of this thesis \cite{mundy16}.
		
		\subsection{Timeouts}
			
			SpiNNaker's router is built on a pipeline architecture. As shown in
			figure~\ref{fig:router-architecture}, the router is fed packets by an
			arbiter which serialises packets arriving from other chips and local
			cores. Every (\SI{100}{\mega\hertz}) clock cycle, the router pipeline
			accepts one packet from the arbiter and routes a packet to one or several
			output links. If any of the required output ports are busy then the
			packet is not forwarded to any output link and the pipeline stalls. Once
			a packet has been blocked for a programmable timeout, it is dropped
			(discarded) and routing continues as usual for next packet in the
			pipeline. Links become blocked while transmitting packets or waiting for
			the remote receiver to become ready. For example, a receiving processor
			core may be busy performing some computation or a receiving router may be
			blocked waiting for some of its outputs to become ready.
			
			\begin{figure}
				\center
				\buildfig{figures/router-architecture.tex}
				
				\caption{SpiNNaker router architecture}
				\label{fig:router-architecture}
			\end{figure}
			
			The timeout-based packet dropping mechanism is designed to defuse
			deadlocks in the network. For example, if two routers are trying to send
			each other a packet at the same time they may become deadlocked, each
			waiting for the other router to accept a packet before continuing.
			SpiNNaker's timeout mechanism breaks deadlocks by dropping packets which
			have been blocked for some time and therefore may be in a deadlock.  Once
			a packet has been dropped it is left to software to either tolerate the
			missing packet or trigger a retransmission. In neural simulations, as in
			biology, the loss of a single spike is unlikely to have a significant
			impact on the behaviour of a neural model and therefore these simulations
			are inherently tolerant of occasional dropped packets. During application
			loading and other system tasks, a higher level, software driven protocol
			based on acknowledgements and retransmissions is used to ensure
			guaranteed delivery.
			
			% TODO: MENTION TIMEOUT VALUE USED?
			% Router timeouts must be configured to be long enough that delays in
			% packet transmission, for example due to the time taken for packets to
			% traverse a link, do not trigger packet dropping. Conversely, the timeout
			% should be as short as possible to reduce the time the router is
			% blocked and maximise network throughput.
	
	\section{The hexagonal torus topology}
		
		Each SpiNNaker chip is a node in a `hexagonal torus topology' as
		illustrated in figure~\ref{fig:hexagonalTorusTopology}. Network packets
		sent by SpiNNaker's processor cores may `hop' through several nodes in the
		network to reach their intended destination. In each hop, a packet may
		advance one node along one of the three axes of the topology. For example,
		a packet sent by the node labelled $\alpha$ (in the bottom-left corner) to
		the node labelled $\beta$, might take the following sequence of hops:
		X$^+$, X$^+$, Z$^-$. Packets sent from $\alpha$ to $\gamma$ might take the
		route: X$^-$, X$^-$, Y$^+$, Y$^+$. The first hop of this route `wraps
		around' from the bottom-left node to the bottom-right node in a single hop.
		
		\begin{figure}
			\center
			\buildfig{figures/hexagonalTorusTopology.tex}
			
			\caption[A hexagonal torus topology.]%
			{A hexagonal torus topology. Each hexagon represents a node (i.e.
			a SpiNNaker chip). Touching nodes are directly connected. Nodes on edges
			$a$, $b$ and $c$ are also directly connected to the corresponding nodes
			on edges $a'$, $b'$ and $c'$, respectively. The three axes of the
			hexagonal torus topology, `X', `Y' and `Z' are also shown.}
			\label{fig:hexagonalTorusTopology}
		\end{figure}
		
		\begin{figure}
			\center
			\begin{subfigure}{0.39\linewidth}
				\center
				\includegraphics[width=\linewidth]{figures/torus-3d-flat.pdf}
				\caption{}
				\label{fig:torus-3d-flat}
			\end{subfigure}
			~~
			\begin{subfigure}{0.26\linewidth}
				\center
				\includegraphics[width=\linewidth]{figures/torus-3d-tube.pdf}
				\caption{}
				\label{fig:torus-3d-tube}
			\end{subfigure}
			~~
			\begin{subfigure}{0.23\linewidth}
				\center
				\includegraphics[width=\linewidth]{figures/torus-3d-torus.pdf}
				\caption{}
				\label{fig:torus-3d-torus}
			\end{subfigure}
			
			\caption{Visualisation of a hexagonal torus topology as a torus.}
			\label{fig:torus-3d}
		\end{figure}
		
		The wrap around connections in the topology are what give it the `torus'
		part of its name. Figure~\ref{fig:torus-3d-flat} shows a hexagonal torus
		topology drawn flat as in the previous figure. If the topology is rolled up
		into a tube such that the top and bottom nodes become directly adjacent, a
		tube is formed as in figure~\ref{fig:torus-3d-tube}. This tube can then be
		bent to bring together the nodes at the ends of the tube to form a torus as
		shown in figure~\ref{fig:torus-3d-torus}.
		
		A hexagonal torus topology is typically defined in terms of its width and
		height along the X and Y axes respectively. For example,
		figure~\ref{fig:hexagonalTorusTopology} shows a $10\times10$ hexagonal
		torus.  The nodes in a hexagonal torus topology are addressed using
		hexagonal coordinates of the form $(x, y, z)$ \cite{patel15}. The bottom
		left node (labelled $\alpha$ in the figure) has the coordinate $(0, 0, 0)$
		and other nodes are assigned coordinates according to the number of hops
		along each dimension from $(0, 0, 0)$, for example node $\beta$ has the
		coordinate $(2, 0, -1)$. Because the hexagonal torus topology's axes are
		non-orthogonal, it is possible to define several coordinates for the same
		location. For example $(3, 1, 0)$ and $(1, -1, -2)$ are also valid
		coordinates for node $\beta$. These dual coordinates emerge from the fact
		that adding $(1, 1, 1)$ to a coordinate produces an equivalent, but
		different, coordinate. This phenomenon is explained in detail in
		appendix~\ref{app:minimal-hex-coordinates} and related phenomena will be
		discussed in chapter~\ref{sec:shortestPaths}.
		
		The hexagonal torus topology was chosen over a more conventional network
		topology -- such as a 2D or 3D torus (sometimes known as a 2-ary $N$-cube
		or 3-ary $N$-cube respectively) \cite[chapters~3~and~5]{dally04} -- due to
		its balance of theoretical performance and practicality. The bisection
		bandwidth of a topology indicates the theoretical worst-case total
		throughput the network is able to sustain \cite[chapter~1]{dally04}.  In
		networks with homogeneous link throughput, bisection bandwidth is
		determined by the number of links cut by a balanced bisection of the
		network.  Figure~\ref{fig:bisection-bandwidth} illustrates the bisections
		of several torus topologies.
		
		\begin{figure}
			\center
			\begin{subfigure}[b]{0.3\linewidth}
				\center
				\buildfig{figures/bisection-bandwidth-2d.tex}
				
				\caption{2D Torus}
				\label{fig:bisection-bandwidth-2d}
			\end{subfigure}
			\begin{subfigure}[b]{0.3\linewidth}
				\center
				\buildfig{figures/bisection-bandwidth-hex.tex}
				
				\caption{Hexagonal Torus}
				\label{fig:bisection-bandwidth-hex}
			\end{subfigure}
			\begin{subfigure}[b]{0.3\linewidth}
				\center
				\buildfig{figures/bisection-bandwidth-3d.tex}
				
				\caption{3D Torus}
				\label{fig:bisection-bandwidth-3d}
			\end{subfigure}
			
			\caption[Bisections of torus topologies.]%
			{Bisections of torus topologies. Connections cut by the bisection
			are drawn as lines.}
			\label{fig:bisection-bandwidth}
		\end{figure}
		
		In a $N \times N$ 2D torus topology, the bisection bandwidth is $2N$~links
		and each node requires four links. The hexagonal torus topology requires
		six links per node but provides double bisection bandwidth ($4N$~links).
		The 3D torus topology also requires six links per node but by connecting
		the nodes differently achieves a bisection bandwidth of $8N$~links.  The 3D
		torus topology, however, comes at a price -- when immersed into the
		(approximately) 2D space provided by a large machine room or row of server
		cabinets, some connections require long cables. By contrast, the 2D and
		hexagonal torus topologies are both inherently two dimensional and
		consequently do not suffer from this effect. The hexagonal torus topology,
		therefore, shares the practicality of construction of a 2D torus while
		still gaining some of the performance of a 3D torus topology. In addition,
		because nodes in a hexagonal torus topology have a greater number of links,
		greater redundancy is available in the network to tolerate faults.
		
		Most torus topologies, including hexagonal, 2D and 3D toruses, have a
		related `mesh' topology. These mesh topologies maintain the same general
		connectivity structure as their torus topologies but omit wrap-around
		links. In practice, this saves a small number of links at the expense of
		halving the network's bisection bandwidth.  Because of their poorer
		performance, mesh networks are rarely used as the basis of a network
		architecture. Mesh networks, however, are occasionally formed when a
		network is partitioned into several smaller sub-networks to allow multiple
		users to share a system \cite{spalloc16}.
		
		\begin{figure}
			\center
			\begin{subfigure}[b]{0.45\linewidth}
				\center
				\buildfig{figures/hexagonal-torus.tex}
				\caption{Hexagonal torus}
				\label{fig:topo-compare-hexagonal-torus}
			\end{subfigure}
			\begin{subfigure}[b]{0.45\linewidth}
				\center
				\buildfig{figures/h-torus.tex}
				\caption{H-torus}
				\label{fig:topo-compare-h-torus}
			\end{subfigure}
			
			\caption[Hexagonal torus vs. H-torus topology.]%
			{Hexagonal torus vs. H-torus topology. Each numbered hexagon
			represents a node. The thick outline indicates the bounds of the
			topology after which the network repeats. In each topology, the path
			taken by advancing in the Y$^+$ direction from the node labelled `0' is
			shown.}
			\label{fig:topo-compare}
		\end{figure}
		
		\label{sec:hex-vs-h-torus}
		
		The hexagonal torus topology is not to be confused with the `H-torus'
		topology. This topology also uses a hexagonal tiling of nodes and even
		wraps this tiling into a torus-like topology \cite{zhao08}. However,
		H-torus topologies have very different characteristics to the hexagonal
		torus topology and are related to `twisted torus' topologies
		\cite{camara10}. For example, figure~\ref{fig:topo-compare} illustrates one
		major difference in the way paths wrap around the peripheries of both
		topologies.
	
	\section{Scaling-up SpiNNaker machines}
		
		To build large SpiNNaker systems comprising of tens of thousands of
		SpiNNaker chips, groups of forty-eight chips are mounted onto circuit
		boards as illustrated in figure~\ref{fig:spinnakerBoard}. These boards may
		be connected together to form larger systems.  Figure~\ref{fig:threeboard}
		shows a prototype three board system. Though the chips are
		\emph{physically} arranged in a (nearly) $7\times7$ grid on each SpiNNaker
		board, they logically form a hexagonal `wrapped triple'
		\cite{davidsonWiring} (see appendix~\ref{sec:partitioning}) which logically
		fit together as illustrated in figure~\ref{fig:threeboard-separate}. The
		labelled exposed corners of the three forty-eight chip boards connect
		together to form a $12\times12$ hexagonal torus topology as illustrated in
		figure~\ref{fig:threeboard-wrapped}. Larger SpiNNaker machines are
		assembled by combining more boards.
		
		\begin{figure}
			\center
			\begin{subfigure}[b]{0.45\linewidth}
				\center
				\includegraphics[width=\linewidth]{figures/spinnakerBoard.jpg}
				
				\caption{A SpiNNaker board}
				\label{fig:spinnakerBoard}
			\end{subfigure}
			~~~
			\begin{subfigure}[b]{0.45\linewidth}
				\center
				\includegraphics[width=\linewidth]{figures/threeboard.jpg}
				
				\caption{Three board prototype}
				\label{fig:threeboard}
			\end{subfigure}
			
			\vspace*{1em}
			
			\begin{subfigure}[b]{0.45\linewidth}
				\center
				\buildfig{figures/threeboard-separate.tex}
				
				\caption{Three board topology}
				\label{fig:threeboard-separate}
			\end{subfigure}
			~~~
			\begin{subfigure}[b]{0.45\linewidth}
				\center
				\buildfig{figures/threeboard-wrapped.tex}
				
				\caption{\ldots{}as a parallelogram}
				\label{fig:threeboard-wrapped}
			\end{subfigure}
			
			\caption{SpiNNaker boards and their topology.}
			\label{fig:spinnaker-boards}
		\end{figure}
		
		
		SpiNNaker chips on the same circuit board connect using low power links
		requiring sixteen wires each.  If this link technology were used to connect
		chips on neighbouring boards, each pair of boards would need to be
		connected with a 128~wire cable.  Cables and connectors supporting this
		many signals are expensive, unreliable and physically large. Instead,
		chip-to-chip connections between boards are multiplexed and demultiplexed
		onto a single High-Speed Serial (HSS) link \cite{athavale05} carried via
		commodity S-ATA cables which are often used to connect hard disks in
		desktop computers and servers \cite{sata3spec}. The six high-speed links
		are implemented by three onboard FPGAs (the three large chips at the top of
		the SpiNNaker board) and are logically transparent to the underlying
		network. The underlying technology and the choice of S-ATA cables limits
		each board-to-board connection to spanning at most one metre gaps. In
		chapter~\ref{sec:building} I present a cabling scheme for hexagonal torus
		topologies which enable large SpiNNaker systems to be assembled using only
		short cables between boards.
		
	\section{Conclusions}
		
		The SpiNNaker architecture has been designed to enable the simulation of
		large biologically realistic neural models in real time. To support this,
		its network architecture takes on an unconventional design based on a
		custom router and hexagonal torus topology. In the remainder of this
		thesis, I will tackle a number of the challenges in scaling up the
		SpiNNaker architecture outlined in this chapter.

	\chapter{Building large SpiNNaker machines}
	
	Like any super computer, physically putting together a large SpiNNaker
	machine poses many challenges in terms of organisation, assembly and
	maintainance. One of the key tasks in this process is the installation of
	network cables such that a desired overall network topology is constructed.
	The largest planned SpiNNaker machine will use \num{3600} S-ATA
	\cite{sata3spec} cables to interconnect its \num{1200} circuit boards,
	creating a hexagonal torus topology. Since the machine will be installed
	within standard server room cabinets (which are not available in a
	giant-doughnut form-factor) a mapping from a board's logical location in the
	network topology to its physical location must be constructed. In addition,
	the interconnect technology employed by SpiNNaker restricts the length of
	S-ATA cables used to $\le$~\SI{1}{\meter}, constraining the possible mappings
	used. In addition the practical issues of installation complexity and
	maintainance must be considered since all \num{3600} cables must ultimately
	be installed and maintained by human operators.
	
	In this chapter I describe a novel technique for physically laying out
	machines configured in hexagonal torus topologies, such as SpiNNaker, in
	commercial machine rooms, building on the techniques used in more
	conventional torus topologies. In addition, I also propose a new methodology
	for installing and maintaining super computer cabling which which exploits
	existing diagnostic features of the SpiNNaker hardware to interactively guide
	and validate cable installation. Finally, I demonstrate how these new
	techniques have been used successfully to interconnect a prototype
	\num{518400} core SpiNNaker machine in substantially less time than the
	industry norm.
	
	In this chapter, the term \emph{unit} refers to the smallest physical
	ecomponent between which connections connections are to be made. For example,
	in a SpiNNaker machine a unit is a 48-chip board while in data center, a unit
	might be a server blade.
	
	\section{Related work}
		
		In this section I describe the techniques conventionally employed when
		laying out and interconnecting the units within super computers. Due to
		SpiNNaker's hexagonal torus topology and dense physical packing of units,
		these existing techniques are found to be insufficient. In the remainder of
		the chapter we will explore solutions to the limitations exposed below.
		
		\subsection{Avoiding long cables}
			
			Na\"ive arrangements of torus topologies, including hexagonal torus
			topologies, feature long `wrap-around' connections which connect units at
			the peripheries of the system. These connections can be problematic for
			numerous reasons:
			
			\begin{description}
				
				\item[Performance] Signal quality diminishes as cables get longer,
				requiring the use of slower signalling speeds, increased error
				correction overhead or more complex hardware.
				
				\item[Energy] Longer cables require higher drive strengths and/or
				buffering to maintain signal integrity.
				
				\item[Cost] Cost Shorter cables are cheaper than long ones.  Longer
				cables imply more wire in a given space making the tasks of routing or
				cable installation more difficult increasing labour costs by as much as
				$5\times$ \cite{curtis12}.
				
			\end{description}
			
			In conventional torus topologies the need for long cables is eliminated
			by folding and interleaving units of the network \cite{dally04}. For
			example, for a 1D torus topology (a ring network), one long connection
			exists to connect the two opposite sides of the system. To remove these
			long connections, half the units are `folded' on top of the others and
			then this arrangement of units is interleaved as illustrated in figure
			\ref{fig:ring-folding}.
			
			\begin{figure}
				\center
				\begin{subfigure}[b]{0.39\linewidth}
					\center
					\buildfig{figures/ring-folding-row.tex}
					\caption{A ring network}
					\label{fig:ring-folding-row}
				\end{subfigure}
				\begin{subfigure}[b]{0.24\linewidth}
					\center
					\buildfig{figures/ring-folding-folded.tex}
					\caption{Folded}
					\label{fig:ring-folding-folded}
				\end{subfigure}
				\begin{subfigure}[b]{0.35\linewidth}
					\center
					\buildfig{figures/ring-folding-interleaved.tex}
					\caption{Folded and interleaved}
					\label{fig:ring-folding-interleaved}
				\end{subfigure}
				
				\caption{Folding and interleaving a ring network to reduce maximum wire
				length.}
				\label{fig:ring-folding}
			\end{figure}
			
			Folding and interleaving has the effect of approximately doubling the
			average cable length but also eliminates the need for a cable spanning
			the entire system. Since the mean cable length is typically already
			short, doubling it in exchange for a substantially reduced maximum cable
			length is often preferable.
			
			The folding and interleaving process may be extended to $N$-dimensional
			torus topologies by folding each dimension in turn. Since all dimensions
			are orthogonal, the folding process only moves units in the dimension
			being folded. In the hexagonal torus topology, however, the three
			dimensions are non-orthogonal and thus folding in one dimension also
			moves units in other dimensions, preventing the edges of the torus
			meeting as illustrated in figure \ref{fig:failing-to-fold-hex-toruses}.
			
			\begin{figure}
				\center
				\begin{subfigure}[b]{0.24\linewidth}
					\center
					\buildfig{figures/failing-to-fold-hex-toruses-none.tex}
					\caption{Not folded}
					\label{fig:failing-to-fold-hex-toruses-none}
				\end{subfigure}
				\begin{subfigure}[b]{0.24\linewidth}
					\center
					\buildfig{figures/failing-to-fold-hex-toruses-x.tex}
					\caption{X}
					\label{fig:failing-to-fold-hex-toruses-x}
				\end{subfigure}
				\begin{subfigure}[b]{0.24\linewidth}
					\center
					\buildfig{figures/failing-to-fold-hex-toruses-y.tex}
					\caption{Y}
					\label{fig:failing-to-fold-hex-toruses-y}
				\end{subfigure}
				\begin{subfigure}[b]{0.24\linewidth}
					\center
					\buildfig{figures/failing-to-fold-hex-toruses-z.tex}
					\caption{Z}
					\label{fig:failing-to-fold-hex-toruses-z}
				\end{subfigure}
				
				\caption{Schematics showing hexagonal torus topologies folded along
				each of their non-orthogonal dimensions. Note that folding along
				the Z axis brings the \emph{wrong} edges closer together.}
				\label{fig:failing-to-fold-hex-toruses}
			\end{figure}
		
		\subsection{Cabling installation}
			
			Existing machine room installations feature very repetitive cabling
			patterns which can easily be memorised by a human technician. For example
			in BlueGene super computers the connectivity between units is highly
			regular \cite{lakner07} while in data centre networks cabling often
			centres around a small number of high-port-count switches
			\cite{cisco07,csernai15}. Cable installation is usually only aided by
			the labelling of connectors and sockets in a standardised manner
			\cite{tia2006} such as in figure \ref{fig:bgWiring}.
			
			\begin{figure}
				\center
				\begin{subfigure}[t]{0.5\textwidth}
					\begin{tikzpicture}
						\node (cables) [inner sep=0]
						      {\includegraphics[width=\textwidth]{figures/bgCables.png}};
						\node (sockets) [inner sep=0, below=1.0em of cables]
						      {\includegraphics[width=\textwidth]{figures/bgSockets.png}};
						
						% Point at label on cable
						\draw [white, <-, line width=0.4em]
						      ([shift={(0.7cm, -0.3cm)}]cables.center)
						      -- ++(45:1cm);
						
						% Point at label on socket
						\draw [white, <-, line width=0.4em]
						      ([shift={(-1.0cm, 1.1cm)}]sockets.center)
						      -- ++(-45:1cm);
					\end{tikzpicture}
					
					\caption{Pre-labelled cables and sockets}
					\label{fig:bgWiringLabels}
				\end{subfigure}
				~
				\begin{subfigure}[t]{0.30\textwidth}
					\includegraphics[height=6.15cm]{figures/bgWiring.jpg}
					
					\caption{Installation of cables}
					\label{fig:bgWiringInstallation}
				\end{subfigure}
				
				\caption{BlueGene/Q cable installation \cite{cscs13}}
				\label{fig:bgWiring}
			\end{figure}
			
			Despite the regularity and careful labelling of cables, the cost of
			installation and maintenance alone can be significant with costs in the
			range of \$45-95 per \SI{1}{\meter} cable run and \$100-400 for runs of
			\SI{10}{\meter} reported in the literature \cite{mudigonda11}. Much of
			this cost is due to the care required during installation to avoid
			miswiring and ensure that cooling airflow is not hampered by cable runs
			\cite{cisco07}.
			
			Many researchers have attempted to control cable installation costs by
			trying to reduce the number or length of cables required by developing
			alternative network topologies \cite{curtis12, popa10, mudigonda11}.
			Unfortunately, these techniques do not apply to SpiNNaker since its
			network topology is fixed.
			
			Some super computers make use of large custom `midplane` PCBs in place of
			cables to interconnect units within a cabinet and thus simplify the task
			of cable installation \cite{prickett10}. This scheme can greatly reduce
			wiring complexity since only coarser-grain cabinet-to-cabinet
			connectivity is provided by cables. Unfortunately this technique is
			expensive and also constrains the dimensions of the network topology
			supported by the machine. Since the SpiNNaker platform is designed to
			scale from desktop machines to machine-room installations, this scheme is
			not practical.
	
	\section{Folding \& interleaving hexagonal toruses}
		
		The first step towards a practical machine-room installation of a large
		machine using a hexagonal torus topology is to find an arrangement of
		boards between which cable lengths are minimised. In this section I
		describe a sequence of transformations which map the positions of units in
		a hexagonal torus topology onto a regular rectangular grid which may be
		folded and interleaved to eliminate long wires. It is worth emphasising
		that this transformation only affects the \emph{physical} positions of
		units and \emph{not} their connectivity.
		
		As described earlier in \S\ref{sec:parititioning} (page
		\pageref{sec:parititioning}), hexagonal torus topologies may be partitioned
		into units containing wrapped-triples of nodes. For example, in SpiNNaker,
		chips (nodes) are partitioned into circuit boards (units) containing 48
		chips. For completeness, this section describes the process of folding both
		systems whose units are individual nodes and those whose units are
		wrapped-triples.
		
		The transformation process is divided into two parts, each described
		separately in this section.
		
		\begin{description}
			
			\item[Parallelogram to rectangle] Units of the system are transformed
			from a parallelogram shape to a rectangular shape.
			
			\item[Uncrinkle] Units within the rectangle are moved such that they all
			lie on a regular (and fully packed) 2D grid.
			
		\end{description}
		
		\subsection{Parallelogram to rectangle}
			
			The hexagonal torus topology is most naturally drawn as a parallelogram
			as illustrated in figures \ref{fig:hex-to-plane-node-native} and
			\ref{fig:hex-to-plane-native}. Two transformations are presented which
			transform these arangements of units into a rectangular form: shearing
			and slicing.
			
			A \SI{30}{\degree} shear transformation distorts networks such that the X
			and Y axes become orthogonal leading to a rectangular arrangement of
			units as illustrated in figures \ref{fig:hex-to-plane-node-shear} and
			\ref{fig:hex-to-plane-shear}.
			
			The slice transformation slices the units protruding from the
			left-hand-side of the parallelogram and moves them into the matching gap
			on the opposite side of the parallelogram as illustrated in figures
			\ref{fig:hex-to-plane-node-slice} and \ref{fig:hex-to-plane-slice}.
			 
			While the shear transformation introduces some distortion causing cables
			in the Z dimension to become $\sqrt{2}\times$ longer it leaves the
			pattern of wrap-around connections remains unchanged. By contrast, the
			slice transformation does not elongate any cables but changes the pattern
			of wrap-around connections. The exact pattern wrap-around connections
			produced when slicing depends on the aspect ratio of the network as
			illustrated in \ref{fig:slicing-examples} and influences the choice of
			folding technique applied as described later.
			
			\begin{figure}
				\center
				\begin{subfigure}[b]{0.32\linewidth}
					\center
					\buildfig{figures/hex-to-plane-node-native.tex}
					
					\caption{$7 \times 7$ node torus}
					\label{fig:hex-to-plane-node-native}
				\end{subfigure}
				\begin{subfigure}[b]{0.32\linewidth}
					\center
					\buildfig{figures/hex-to-plane-node-shear.tex}
					
					\caption{Sheared}
					\label{fig:hex-to-plane-node-shear}
				\end{subfigure}
				\begin{subfigure}[b]{0.32\linewidth}
					\center
					\buildfig{figures/hex-to-plane-node-slice.tex}
					
					\caption{Sliced}
					\label{fig:hex-to-plane-node-slice}
				\end{subfigure}
				
				\caption{Transformations of hexagonal toruses of nodes into a
				rectangular form. Thin lines show wrap-around links. Pointy-topped
				hexagons represent individual nodes.}
				\label{fig:hex-to-plane-node}
			\end{figure}
			
			\begin{figure}
				
				\begin{subfigure}[b]{0.32\linewidth}
					\center
					\buildfig{figures/hex-to-plane-native.tex}
					
					\caption{$4 \times 4$ triad torus}
					\label{fig:hex-to-plane-native}
				\end{subfigure}
				\begin{subfigure}[b]{0.32\linewidth}
					\center
					\buildfig{figures/hex-to-plane-shear.tex}
					
					\caption{Sheared}
					\label{fig:hex-to-plane-shear}
				\end{subfigure}
				\begin{subfigure}[b]{0.32\linewidth}
					\center
					\buildfig{figures/hex-to-plane-slice.tex}
					
					\caption{Sliced}
					\label{fig:hex-to-plane-slice}
				\end{subfigure}
				
				\caption{Transformations of hexagonal toruses of wrapped triples into a
				rectangular form.  Thin lines show wrap-around links. Flat-topped
				hexagons represent a wrapped triple of nodes.}
				\label{fig:hex-to-plane}
			\end{figure}
			
			\begin{figure}
				\center
				\buildfig{figures/slicing-examples.tex}
				\caption{Patterns of wiring in sliced systems of various sizes.}
				\label{fig:slicing-examples}
			\end{figure}
			
		\subsection{Uncrinkling}
			
			Though the transformmation step yields rectangular arrangements of units,
			these arrangements do not fall onto a regular 2D grid, with the exception
			of the shear transform on individual nodes. Figure \ref{fig:uncrinkling}
			illustrates how the various arrangements of hexagons may be moved to
			`uncrinkle' the units into a regular grid.
			
			\begin{figure}
				\center
				\begin{subfigure}[b]{0.44\linewidth}
					\center
					\buildfig{figures/uncrinkling-node-sheared.tex}
					
					\caption{$7 \times 7$ nodes, sheared}
					\label{fig:uncrinkling-node-sheared}
				\end{subfigure}
				\begin{subfigure}[b]{0.44\linewidth}
					\center
					\buildfig{figures/uncrinkling-node-sliced.tex}
					
					\caption{$7 \times 7$ nodes, sliced}
					\label{fig:uncrinkling-node-sliced}
				\end{subfigure}
				
				\vspace{1cm}
				
				\begin{subfigure}[b]{0.44\linewidth}
					\center
					\buildfig{figures/uncrinkling-sheared.tex}
					
					\caption{$4 \times 4$ triples, sheared}
					\label{fig:uncrinkling-sheared}
				\end{subfigure}
				\begin{subfigure}[b]{0.44\linewidth}
					\center
					\buildfig{figures/uncrinkling-sliced.tex}
					
					\caption{$4 \times 4$ triples, sliced}
					\label{fig:uncrinkling-sliced}
				\end{subfigure}
				
				\vspace{1em}
				
				\caption{Mapping rectangular arrangements of units into a square grid.
				Thick lines show how layers of units are uncrinkled.  Annotations show
				how the relative positions of nodes and wrapped triples change after
				uncrinkling.}
				\label{fig:uncrinkling}
			\end{figure}
			
			In the figure, the numbered units enumerate the different positions on
			the crinkle and those labelled alphabetically are those that immediately
			surround them. From this we can observe that uncrinkling largely
			preserves spatial locality but some distortion is introduced, separating
			previously neighbouring units. For example, in figure
			\ref{fig:uncrinkling-sheared}, the units labelled `1' and `i' are
			neighbours before uncrinkling but are separated by a (Euclidean) distance
			of $\sqrt{5}$ afterwards. Note that the distortion introduced depends on
			what part of the crinkle is considered, for example `2' and `a' have
			distance 2 but are logically connected in the same way.
		
		\subsection{Folding and Interleaving}
			
			Once a regular grid of units has been formed, this may be folded in the
			conventional way, eliminating long cables crossing from left-to-right and
			top-to-bottom as illustrated in \ref{fig:folding-sheared}.
			
			Unfortunately, for sliced systems whose dimensions are not of the ratio
			$1:2$, the pattern of wrap-around cables may also include some cables
			which do not cross directly to the opposite side of the system (refer
			back to figure \ref{fig:slicing-examples}). As a result of these
			connections, folding does not successfully eliminate all long
			connections. An exception to this rule is sliced systems whose dimensions
			are in the ratio $1:1$ where folding twice along the Y axis may
			successfully eliminate all wrap-around connections as illustrated in
			\ref{fig:folding-sliced}.
			
			\begin{figure}
				\begin{subfigure}{\linewidth}
					\center
					\buildfig{figures/folding-sheared.tex}
					\caption{$N \times M$ sheared systems and $N \times 2N$ sliced systems}
					\label{fig:folding-sheared}
				\end{subfigure}
				
				\vspace{1em}
				
				\begin{subfigure}{\linewidth}
					\center
					\buildfig{figures/folding-sliced.tex}
					\caption{$N \times N$ sliced systems}
					\label{fig:folding-sliced}
				\end{subfigure}
				
				\caption{Schematic illustrating elimination of long wrap-around links
				during folding. In each example a single link has been highlighted to
				aid in following the process.}
				\label{fig:folding}
			\end{figure}
			
			Once folded, the 2D grid is straight-forwardly interleaved as illustrated
			previously in figure \ref{fig:ring-folding}. The interleaving process
			introduces some additional distortion to the layout of units and causes
			most connections to become twice as long. For sliced $1:1$ systems, the
			additional fold results in additional overhead during interleaving since
			four layers of the system are interleaved.
		
		\subsection{Mapping to Cabinets}
			
			In the final step of the process is to map the 2D grid of units into
			positions in machine room cabinets as illustrated in figure
			\ref{fig:million-core-machine}. As illustrated in figure
			\ref{fig:cabinetisation}, first the grid of units is partitioned into
			groups of columns, one per cabinet, then groups of rows one per frame per
			cabinet. The units in each group are then allocated to slots within a
			frame, interleaving the rows of the groups as shown.
			
			\begin{figure}
				\center
				\buildfig{figures/cabinet-units.tex}
				
				\caption{An illustration of the physical construction of a
				multi-cabinet SpiNNaker system. (Note: network cables \emph{not}
				installed.)}
				\label{fig:cabinet-units}
			\end{figure}
			
			\begin{figure}
				\center
				\buildfig{figures/cabinetisation.tex}
				
				\caption{Mapping from 2D space to cabinets, frames and boards.}
				\label{fig:cabinetisation}
			\end{figure}
		
	\section{Cable installation}
		
		Cable installation is performed by a team of (human) technicians who must
		ensure that all network cables are correctly installed. As illustrated in
		previously in figure \ref{fig:cabinet-units}, the density of SpiNNaker's
		units, combined with the nature of the hexagonal torus topology, poses a
		challenge. To address this challenge I propose a semi-automated approach to
		cable installation which exploits diagnostic facilities available in the
		majority of super computers in order to guide technicians through the
		cabling process, interactively guiding installation and maintenance.
		
		\subsection{Interactive technician guidance and validation}
			
			While automated systems for validating cabling correctness are
			commonplace, these systems are typically used only after cabling has been
			completed \cite{lakner07}. As with other large-scale machines, SpiNNaker
			includes a low-bandwidth system management bus which may be used to
			interrogate network hardware and control diagnostic LEDs prior to the
			installation of the main SpiNNaker network interconnect.  Using these
			facilities I have constructed a tool called SpiNNer which interactively
			guides a technician, or team of technicians, through the cable
			installation process, validating each connection in real-time.
			
			Diagnostic LEDs mounted on each SpiNNaker board (figure
			\ref{fig:interactive-wiring-guide-leds}) are used to indicate the
			endpoints of the cable currently being installed. Simultaneously a
			Text-To-Speech (TTS) system gives an audible indication of which cable
			type is to be used and location of each connection.  Additionally, a GUI
			via a computer display (figure \ref{fig:interactive-wiring-guide-gui}).
			The centre of the display shows a `big-picture' perspective of the
			locations of the boards to be connected. The detailed views on the left
			and right indicate which of the six sockets on each board the cables
			should connect.
			
			\begin{figure}
				\center
				\begin{subfigure}[b]{0.40\textwidth}
					\begin{tikzpicture}
						\node (leds) [inner sep=0]
						      {\includegraphics[width=\textwidth]{figures/leds.jpg}};
						% Point at left LED
						\draw [white, <-, line width=0.4em]
						      ([shift={(-0.0cm, -0.6cm)}]leds.center)
						      -- ++(225:1cm);
						% Point at right LED
						\draw [white, <-, line width=0.4em]
						      ([shift={(1.1cm, -1.1cm)}]leds.center)
						      -- ++(225:1cm);
					\end{tikzpicture}
					
					\caption{Diagnostic LEDs}
					\label{fig:interactive-wiring-guide-leds}
				\end{subfigure}
				~
				\begin{subfigure}[b]{0.546\textwidth}
					\begin{tikzpicture}[thin, black!20!white]
						\node (screen) [inner sep=0]
						      {\includegraphics[width=\textwidth]{figures/wiring_guide_screenshot.png}};
						\draw (screen.south west) rectangle (screen.north east);
					\end{tikzpicture}
					
					\caption{Interactive wiring guide GUI}
					\label{fig:interactive-wiring-guide-gui}
				\end{subfigure}
				
				\caption{The SpiNNer interactive wiring guide uses a GUI,
				text-to-speech and diagnostic LEDs to assist during cable
				installation.}
				\label{fig:interactive-wiring-guide}
			\end{figure}
			
			SpiNNer also validates the connectivity of the system in real-time by
			polling the diagnostic interfaces of the network hardware at the
			endpoints of the cable being installed to determine if they are correctly
			connected. If a miswiring occurs, this is immediately detected and
			announced via TTS enabling the technician to immediately correct the
			error. Once a cable has been installed correctly, the software
			automatically advances to the next cable meaning direct interaction with
			the software by the technician is minimal. In practice, it is rarely
			necessary to refer to the GUI.
		
			SpiNNer presents the cables in an order intended to maximise ease of
			installation. Cables are installed in three groups with intra-frame
			cables being installed first, followed by intra-cabinet cables and
			inter-cabinet cables. Within each group, the tightest cables are
			installed first resulting in slacker cables naturally being installed
			over the top of already installed cables. By grouping cables in this
			manner, multiple technicians may work independently on the wiring within
			individual frames and cabinets.
			
			SpiNNer may also be used to repair or replace cables in the system.
			During maintenance, obstructing cables may be blindly removed alongside
			any cable being replaced. At the conclusion of the process, the wiring
			guide may be used to interactively guide re-installation of all removed
			cables.
		
		\subsection{Cable selection}
			
			Controlling slack is critical to ensuring reliable and maintainable
			cabling installations. If cables are too tight, cables and connectors can
			become easily damaged and when too slack, the excess cable obstructs
			other cables and can easily become tangled and damaged \cite{cisco07}. It
			has been observed that when ready-made cables are employed technicians
			frequently over-estimate the cable lengths required preferring to use
			overly long cables for all connections \cite{mazaris97}. To solve this
			problem, the SpiNNer wiring guide software dictates the cable lengths to
			be used by an installer based the rule of (three-)thumbs according to
			Mazaris \cite{mazaris97}. This rule suggests that an ideal amount of
			slack is approximately that which can be wrapped around three fingers.
			Specifically, the shortest available cable is selected which ensures at
			least \SI{5}{\centi\meter} of slack.
			
			The SpiNNer tool allocates cables assuming all cables take a Euclidean
			straight-line path between the endpoints of the connection. The result is
			that wiring is not routed through dedicated cable management structures
			but is simply suspended by its connectors in front of the machine. As
			demonstrated later, this unconventional approach leads neither to cooling
			problems nor increased maintenance effort.
	
	\section{Results and Evaluation}
		
		This stuff has been used and works. In this section I'll go over the
		overheads introduced by the various transformations and
		folding/interleaving steps and show a wiring scheme for a large machine
		which uses only short cables. I'll then show how SpiNNer was used to
		install this wiring plan into a very large machine without foobaring the
		cooling and in very little time. I'll also report on difficulty of
		maintenance.
		
		\subsection{Cable length}
			
			The transformation from regular hexagonal torus to a folded and
			interleaved form introduces some overhead to the cable lengths required.
			Using figure \ref{fig:uncrinkling} (page \pageref{fig:uncrinkling}), it
			is possible to compute the exact overhead introduced when each type of
			transformation proposed.
			
			For example, to compute the mean overhead introduced by the slicing
			technique when applied to units composed of wrapped triples, consider
			figure \ref{fig:uncrinkling-sliced}. The uncrinkling pattern used to
			transform this topology is a repeating pattern of two units, a pair of
			which have been labelled $1$ and $2$ respectively. Unit $1$ is
			immediately surrounded by six units labelled $a$, $b$, $c$, $2$, $g$ and
			$h$. Similarly, unit $2$ is surrounded by units $1$, $c$, $d$, $e$, $f$
			and $g$. Before the transformation, the distances, $D$, to each of these
			units is $1$ but after the transformation is applied, this is not always
			the case. Additionally, folding and interleaving introduce additional
			overhead. In this example, if the system is folded into $f_x$ columns and
			$f_y$ rows, the distances between previously neighbouring units become:
			
			\begin{equation*}
				\begin{aligned}[c]
					D_{1\,\leftrightarrow{}\,a} &= \sqrt{f_x^2 + f_y^2} \\
					D_{1\,\leftrightarrow{}\,b} &= f_y \\
					D_{1\,\leftrightarrow{}\,c} &= \sqrt{f_x^2 + f_y^2} \\
					D_{1\,\leftrightarrow{}\,2} &= f_x \\
					D_{1\,\leftrightarrow{}\,g} &= f_y \\
					D_{1\,\leftrightarrow{}\,h} &= f_x
				\end{aligned}
				\hspace{2cm}
				\begin{aligned}[c]
					D_{2\,\leftrightarrow{}\,1} &= f_x \\
					D_{2\,\leftrightarrow{}\,c} &= f_y \\
					D_{2\,\leftrightarrow{}\,d} &= f_x \\
					D_{2\,\leftrightarrow{}\,e} &= \sqrt{f_x^2 + f_y^2} \\
					D_{2\,\leftrightarrow{}\,f} &= f_y \\
					D_{2\,\leftrightarrow{}\,g} &= \sqrt{f_x^2 + f_y^2}
				\end{aligned}
			\end{equation*}
			
			From these values, the mean and maximum connection distances after
			folding and interleaving may be computed. Table
			\ref{tab:transform-overhead} gives the mean and maximum connection
			distances for each of the four transformations described in this chapter.
			
			\begin{table}
				\begin{subtable}[b]{\linewidth}
					\center
					\begin{tabular}{l c c}
						\toprule
						& Shear & Slice \\
						\addlinespace
						Nodes &
							$\frac{f_x + f_y + \sqrt{f_x^2 + f_y^2}}{3}$ &
							$\frac{f_x + f_y + \sqrt{f_x^2 + f_y^2}}{3}$ \\
						\addlinespace
						Triples &
							$\frac{7f_x + 3\sqrt{f_x^2 + f_y^2} + \sqrt{(2f_x)^2 + f_y^2}}{9}$ &
							$\frac{f_x + f_y + \sqrt{f_x^2 + f_y^2}}{3}$ \\
						\bottomrule
					\end{tabular}
					
					\caption{Mean}
					\label{tab:transform-overhead-mean}
				\end{subtable}
				
				\vspace{1em}
				
				\begin{subtable}[b]{\linewidth}
					\center
					\begin{tabular}{l c c}
						\toprule
						& Shear & Slice \\
						\addlinespace
						Nodes &
							$\sqrt{f_x^2 + f_y^2}$ &
							$\sqrt{f_x^2 + f_y^2}$ \\
						\addlinespace
						Triples &
							$\sqrt{(2f_x)^2 + f_y^2}$ &
							$\sqrt{f_x^2 + f_y^2}$ \\
						\bottomrule
					\end{tabular}
					
					\caption{Maximum}
					\label{tab:transform-overhead-max}
				\end{subtable}
				
				\caption{Overheads introduced when transforming unit positions onto a
				regular grid.}
				\label{tab:transform-overhead}
			\end{table}
			
			From these results it is evident that shearing and slicing networks
			whose units are nodes result in identical mean and maximum overhead in
			cable length when folded similarly. Since sliced networks may require
			folding more than once along each axis the shearing approach is
			preferable in general.
			
			For networks constructed from units of wrapped triples, the slicing
			approach suffers the same mean and maximum overhead has networks of
			nodes, and less overhead than shearing for the same number of folds. For
			systems with an aspect ratio of $1:2$ (where both slicing and shearing
			require $f_x = f_y = 2$), the slicing transformation yields lower mean
			and maximum overhead than shearing. For all other aspect ratios (where
			slicing requires a greater degree of folding) the shearing technique
			produces lower overhead. The recommended transformations for a given
			machine are thus given in table \ref{tab:transform-recommended}.
			
			\begin{table}
				\center
				\begin{tabular}{lcc}
					\toprule
					                         & $1:2$  & Other \\
					\addlinespace
					\multirow{2}{*}{Nodes}   & Either & Shear\\
					                         & \footnotesize $\mu\approx2.28 \quad \vee\approx2.83$
					                         & \footnotesize $\mu\approx2.28 \quad \vee\approx2.83$\\
					\addlinespace
					\multirow{2}{*}{Triples} & Slice  & Shear\\
					                         & \footnotesize $\mu\approx2.28 \quad \vee\approx2.83$
					                         & \footnotesize $\mu\approx3.00 \quad \vee\approx4.47$\\
					\bottomrule
				\end{tabular}
				
				\caption{Recommended transformation and folding scheme for different
				system types. $\mu$ and $\vee$ give the mean and maximum wire
				distortion introduced, respectively.}
				\label{tab:transform-recommended}
			\end{table}
			
			\begin{figure}
				\center
				\buildfig{figures/million-core-machine.tex}
				
				\caption{Cabling plan for a \num{1036800} core SpiNNaker
				machine's \num{3600} cables.}
				\label{fig:million-core-machine}
			\end{figure}
			
			Following folding and mapping to physical locations, the cabling plans
			for large machines require no large gaps to be spanned.  The largest
			planned SpiNNaker machine, illustrated in figure
			\ref{fig:million-core-machine}, will be \SI{6}{\meter} wide but the
			largest gap any cable must span is \SI{66}{\centi\meter}. This distance
			is well within the \SI{1}{\meter} allowed by the hardware and cables.
			
		\subsection{Installation practicality}
			
			\begin{table}
				\center
				\begin{tabular}{lrr@{$\,$}l}
					\toprule
						System & Number of Cables & \multicolumn{2}{r}{Installation time} \\
					\midrule
						24 boards  & \num{72}   & \num{10} & \si{\minute}         \\
						1 cabinet  & \num{360}  & \num{4}  & \si{\hour}$^\dagger$ \\
						2 cabinets & \num{720}  & \num{2}  & \si{\hour}           \\
						5 cabinets & \num{1800} & ?        &                      \\
					\bottomrule
				\end{tabular}
				
				\caption{Installation times for various sizes of machine.
				$\dagger$~This machine was installed without real-time validation of
				connectivity.}
				\label{tab:install-time}
			\end{table}
			
			A number of SpiNNaker machines of various scales have been assembled
			using the techniques described in this chapter ranging from single frames
			of 24 boards to a half-scale 5 cabinet machine. Table
			\ref{tab:install-time} gives the reported installation times of each of
			these machines.
			
			The single cabinet machine's installation time is notably
			disproportionate to its size. When this system was assembled, real-time
			connection validation was not yet available. As a result, though cable
			installation was rapid correcting errors was extremely costly, requiring
			careful retracing of many installation steps.
			
			TODO: TALK ABOUT MULTI-PERSON-WIRING IN PRACTICE ON FIVE CABINET MACHINE.
			
			\begin{figure}
				
				\center
				\buildfig{figures/wire-length-histogram.tex}
				
				\caption{Histogram of connection distances in a ten-cabinet,
				one-million core SpiNNaker machine annotated with the suggested cable
				length.}
				\label{fig:wire-length-histogram}
				
			\end{figure}
			
			FIGURE \ref{fig:wire-length-histogram} SHOWS THE DISTRIBUTION OF CABLE
			LENGTHS REQUIRED. IN PRACTICE THE SLACK ALLOCATED PROVED ADEQUATE. AS
			SHOWN IN FIGURE \ref{fig:install-histogram}, THE MOST IMPORTANT FACTOR IS
			WHETHER LEAVING THE FRAME OR NOT. LEAVING THE FRAME TAKES THE LONGEST.
			
			\begin{figure}
				\builddata{data/build_connection_log.tex}
				\buildfig{figures/install-histogram.tex}
				
				\caption{Histogram of cable installation times}
				\label{fig:install-histogram}
			\end{figure}
			
			TODO: COMPARE DIRECTLY WITH INSTALL TIMES REPORTED IN LITERATURE.
		
		\subsection{Thermal Impact}
			
			TODO: SHOW HOW TEMPERATURE IS CHANGED
			
		\subsection{Maintenance}
			
			TOOD: QUANTIFY CABLE REMOVALS REQUIRED. EXPERIMENT: REMOVE/REPLACE RANDOM
			BOARDS AND MEASURE TIME TAKEN, CABLES REMOVED. COMPARE WITH STANDARD DATA
			CENTRE WIRING

	\chapter{Finding shortest path vectors in SpiNNaker's network}
	
	Once a SpiNNaker machine has been constructed as described in the previous
	chapter, its network forms a large hexagonal torus topology. To exploit this
	network routing algorithms must be used to generate routes for packets to
	follow between nodes. As well as ensuring that packets arrive at the correct
	destination, routing algorithms often attempt to produce routes which make
	efficient use of the network. This often involves attempting to reduce
	congestion by ensuring packets do not travel further through the network than
	absolutely necessary.
	
	Many popular routing algorithms for torus topologies, including all published
	algorithms designed for SpiNNaker's hexagonal torus topology
	\cite{davies12,navaridas14}, internally function by computing shortest path
	vectors and generating routes from them. Existing methods of calculating
	shortest path vectors in hexagonal torus topologies are unable to generate
	all possible shortest path vectors and, as a result, reduces the diversity of
	routes produced by routing algorithms, potentially worsening network
	contention.
	
	In this chapter I describe a novel technique for computing shortest path
	vectors in hexagonal torus topologies which yields \emph{all} possible
	shortest path vectors for any pair of nodes. Further, implementations of this
	new technique execute an order of magnitude faster than the existing
	alternatives.
	
	\section{Related work}
		
		TODO: INTRODUCE SECTION
		
		\begin{figure}
			\center
			
			\begin{subfigure}{\linewidth}
				\center
				\buildfig{figures/distance-map-mesh.tex}
				\caption{2D mesh topology}
				\label{fig:distance-map-mesh}
			\end{subfigure}
			
			\vspace{1em}
			
			\begin{subfigure}{\linewidth}
				\center
				\buildfig{figures/distance-map-torus.tex}
				\caption{2D torus topology}
				\label{fig:distance-map-torus}
			\end{subfigure}
			
			\vspace{1em}
			
			\begin{subfigure}{\linewidth}
				\center
				\buildfig{figures/distance-map-hex-mesh.tex}
				\caption{Hexagonal mesh topology}
				\label{fig:distance-map-hex-mesh}
			\end{subfigure}
			
			\vspace{1em}
			
			\begin{subfigure}{\linewidth}
				\center
				\buildfig{figures/distance-map-hex-torus.tex}
				\caption{Hexagonal torus topology}
				\label{fig:distance-map-hex-torus}
			\end{subfigure}
			
			\caption{Plots showing distance from various locations marked
			         {\color{red}$\times$}. Darker areas are further away. Contour
			         lines show equidistant points.}
			\label{fig:distance-map}
		\end{figure}
		
		\subsection{Mesh Networks}
			
			In a (non-hexagonal) mesh network topology, shortest path vectors are
			computed by taking the element-wise difference between the source and
			destination nodes' coordinates.
			
			\begin{figure}
				\center
				\buildfig{figures/mesh-topology-coordinates.tex}
				\caption{An example 2D mesh network with example shortest-path routes
				from `A' to `B' and `B' to `C'.}
				\label{fig:mesh-topology-coordinates}
			\end{figure}
			
			For example, figure \ref{fig:mesh-topology-coordinates} illustrates a 2D
			mesh topology. In this topology, the nodes labelled `A', `B' and `C' have
			position vectors $(1, 2)$, $(4, 5)$ and $(6, 1)$ respectively. The
			shortest path vector from node `A' to `B' is thus simply $(4, 5) - (1, 2)
			= (3, 3)$ and from `B' to `C' is $(6, 1) - (4, 5) = (2, -4)$.
			
			A route may be produced from a shortest path vector by advancing the
			number of hops specified for each dimension in the vector. For example
			any permutation of the hops X$^+\,$X$^+\,$X$^+\,$Y$^+\,$Y$^+\,$Y$^+$, an
			example of which is included in the figure. Likewise a route from `B' to
			`C' may be constructed from any permutation of
			X$^+\,$X$^+\,$Y$^-\,$Y$^-\,$Y$^-\,$Y$^-$.
			
			Many popular routing algorithms such as Dimension Order Routing (DOR),
			Right-Turn Only Routing (RTOR) and Longest Dimension First Routing (LDFR)
			\cite{dally04,davies12} directly follow the above procedure and just
			prescribe a specific permutation of hop order. For example, DOR produces
			routes with X hops first, Y hops second and so on.
			
			The length of routes produced from a shortest path vector have a number
			of hops proportional to the magnitude of the vector, thus a shortest path
			vector yields a route with the minimum number of hops. For a two
			dimensional vector $(a, b)$ the magnitude is given as:
			%
			\begin{equation}
				\| (a, b) \| = \lvert a \rvert + \lvert b \rvert
			\end{equation}
		
		\subsection{Torus Networks}
			
			Computing shortest path vectors in (non-hexagonal) torus topologies is
			also straight forward. As an example, lets find the shortest path vector
			from node `A' to `B' in the 2D torus topology shown in figure
			\ref{fig:torus-shortest-path-example}. First, both nodes are translated
			such that the source node, `A', is at the centre of the network (figure
			\ref{fig:torus-shortest-path-translate}). Note that this translation may
			result in the destination node `wrapping around' the network. After
			translation, the shortest path vector is computed as in a mesh topology.
			As illustrated in \ref{fig:torus-shortest-path-routed}, the computed
			shortest path vector may be used to produce routes between the two nodes
			in their original positions.
			
			\begin{figure}
				\center
				\begin{subfigure}{0.3\linewidth}
					\center
					\buildfig{figures/torus-shortest-path-example.tex}
					\caption{Original}
					\label{fig:torus-shortest-path-example}
				\end{subfigure}
				\begin{subfigure}{0.3\linewidth}
					\center
					\buildfig{figures/torus-shortest-path-translate.tex}
					\caption{Translated}
					\label{fig:torus-shortest-path-translate}
				\end{subfigure}
				\begin{subfigure}{0.3\linewidth}
					\center
					\buildfig{figures/torus-shortest-path-routed.tex}
					\caption{Routed}
					\label{fig:torus-shortest-path-routed}
				\end{subfigure}
				
				\caption{Finding shortest paths in a 2D torus topology.}
				\label{fig:torus-shortest-path}
			\end{figure}
			
			This process works because vectors from the centre (though not other
			locations) of a torus topology are identical to those in mesh topologies
			(see figures \ref{fig:distance-map-mesh} and
			\ref{fig:distance-map-torus}).
		
		\subsection{Hexagonal Mesh Networks}
			
			In hexagonal mesh topologies it is conventional to define three `axes' X,
			Y and Z as shown in figure \ref{fig:hex-mesh-topology-coordinates}
			\cite{patel15}. In this example, the three labelled nodes `A', `B' and
			`C' may be given position vectors such as $(1, 1, 0)$, $(3, 2, 0)$ and
			$(0, 0, -7)$ respectively. As in other mesh networks, a vector between
			two nodes is found by subtracting the nodes' vectors. For example, a
			vector from `A' to `B' is $(3, 2, 0) - (1, 1, 0) = (2, 1, 0)$. This
			vector can then be converted into a route in the same way as a mesh
			network by taking any permutation of the three hops  X$^+\,$X$^+\,$Y$^+$.
			
			\begin{figure}
				\center
				\buildfig{figures/hex-mesh-topology-coordinates.tex}
				\caption{An example hexagonal mesh network topology.}
				\label{fig:hex-mesh-topology-coordinates}
			\end{figure}
			
			As explained in detail in appendix \ref{app:minimal-hex-coordinates},
			there are an infinite number of vectors between any two points. For
			example, the vectors $(1, 0, -1)$ and $(3, 2, 1)$ also reach node `B'
			from `A' in the example. However, for a given pair of nodes, there is
			always a single, unique vector whose magnitude is minimal which is
			given by the function:
			%
			\begin{equation}
				\operatorname{minimiseVector}(x,y,z)
					= (x,y,z) - \operatorname{median}(x,y,z) \cdot (1,1,1)
			\end{equation}
			%
			An important side-effect of this function is that a minimised vector will
			always contain at least one zero element meaning that shortest path
			routes will use at most two of the three available dimensions.
			
			To aid the reader's intuition, figure \ref{fig:distance-map-hex-mesh}
			illustrates how distances grow in a hexagonal mesh topology.
		
		\subsection{Hexagonal Torus Networks}
			
			Unfortunately, unlike non-hexagonal torus topologies, the translation
			technique cannot be used to compute shortest path vectors. As illustrated
			in figures \ref{fig:distance-map-hex-mesh} and
			\ref{fig:distance-map-hex-torus}, shortest path vectors from the center
			of a hexagonal mesh network are not equivalent to those of a hexagonal
			torus network.
			
			Prior research into routing in SpiNNaker's network has been based on the
			INSEE \cite{navaridas09,ghasempour15} interconnect simulator. Internally
			INSEE tries a set of twelve candidate vectors and picks the shortest as
			the shortest path vector to use for routing.
			
			\begin{figure}
				\center
				\begin{subfigure}{0.45\linewidth}
					\center
					\buildfig{figures/insee-vector-candidates-no-wrap.tex}
					\caption{$(\Delta_\textrm{X}, \Delta_\textrm{Y}) = (5,3)$}
					\label{fig:insee-vector-candidates-no-wrap}
				\end{subfigure}
				\begin{subfigure}{0.45\linewidth}
					\center
					\buildfig{figures/insee-vector-candidates-wrap-x.tex}
					\caption{$(\Delta'_\textrm{X}, \Delta_\textrm{Y}) = (-3,3)$}
					\label{fig:insee-vector-candidates-wrap-x}
				\end{subfigure}
				
				\vspace{1em}
				
				\begin{subfigure}{0.45\linewidth}
					\center
					\buildfig{figures/insee-vector-candidates-wrap-y.tex}
					\caption{$(\Delta_\textrm{X}, \Delta'_\textrm{Y}) = (5,-5)$}
					\label{fig:insee-vector-candidates-wrap-y}
				\end{subfigure}
				\begin{subfigure}{0.45\linewidth}
					\center
					\buildfig{figures/insee-vector-candidates-wrap.tex}
					\caption{$(\Delta'_\textrm{X}, \Delta'_\textrm{Y}) = (-3,-5)$}
					\label{fig:insee-vector-candidates-wrap}
				\end{subfigure}
				
				\vspace{1em}
				
				% Key
				\begin{tikzpicture}[thick]
					\coordinate (last);
					
					% #1 colour
					% #2 label
					\newcommand{\colourkeyentry}[2]{
						\node [#1] [right=of last, fill, rectangle, minimum size=1em] (last) {};
						\node [right=0 of last] (last) {#2};
					}
					
					\colourkeyentry{cb3class0}{$(\textrm{X}, \textrm{Y}, 0)$}
					\colourkeyentry{cb3class1}{$(\textrm{X} - \textrm{Y}, 0, - \textrm{Y})$}
					\colourkeyentry{cb3class2}{$(0, \textrm{Y} - \textrm{X}, - \textrm{X})$}
					
				\end{tikzpicture}
				
				\caption{The twelve candidate shortest-path vectors considered by INSEE
				represented as dimension-order routes. $W=H=8$,
				$(\Delta_\textrm{X},\Delta_\textrm{Y}) = (5, 3)$ and
				$(\Delta'_\textrm{X},\Delta'_\textrm{Y}) = (-3, -5)$.}
				\label{fig:insee-vector-candidates}
			\end{figure}
			
			The twelve vectors considered are constructed as follows.
			
			First a shortest path vector from the source to target node are
			constructed as if the network was a 2D mesh yielding a vector
			$(\Delta_\textrm{X},\Delta_\textrm{Y})$. From this, another vector
			$(\Delta'_\textrm{X},\Delta'_\textrm{Y})$, is defined:
			%
			\begin{align}
				\Delta'_\textrm{X} &= \Delta_\textrm{X} - \operatorname{sign}(\Delta_\textrm{X})W
				\\
				\Delta'_\textrm{Y} &= \Delta_\textrm{Y} - \operatorname{sign}(\Delta_\textrm{Y})H
			\end{align}
			%
			Where $W$ and $H$ are the width and height of the network respectively
			(in nodes). This new vector yields routes from the source to destination
			node that in a torus topology that \emph{always} wrap around the `X' and
			`Y' dimensions.
			
			From the pair of vectors defined, four possible 2D vectors can be
			produced: $(\Delta_\textrm{X},\Delta_\textrm{Y})$,
			$(\Delta'_\textrm{X},\Delta_\textrm{Y})$,
			$(\Delta_\textrm{X},\Delta'_\textrm{Y})$ and
			$(\Delta'_\textrm{X},\Delta'_\textrm{Y})$. Further, each 2D vector may be
			converted into one of three 3D vectors where either X, Y or Z are zero
			for a total of twelve candidate vectors.  Figure
			\ref{fig:insee-vector-candidates} illustrates all twelve candidate
			vectors for an example pair of nodes.
			
			\begin{figure}
				\center
				\buildfig{figures/xyz-protocol-regions.tex}
				
				\caption{The four regions defined by the XYZ-protocol.}
				\label{fig:xyz-protocol-regions}
			\end{figure}
			
			A more efficient technique is proposed by Hoffmann and D\'es\'erable
			called the XYZ-Protocol \cite{hoffmann15,hoffmann11}. If the source and
			destination nodes are translated such that the source node lies at the
			center of the topolgoy, the destination will lie in one of four regions
			illustrated in figure \ref{fig:xyz-protocol-regions}.
			
			If the destination lies in regions 1 or 4, a route may be constructed as
			if in a hexagonal mesh topology.
			
			Alternatively, if the destination lies in regions 2 or 3, the algorithm
			tests whether taking a mesh-like route within the region or
			wrapping-around either the X or Y dimension yields the shorter vector.
			The shortest of these vectors is then chosen.
			
			TODO DESCRIBE SPIRAL ROUTES.
			
			TODO DESCRIBE RTOR AND LDFR.
		
	\section{Dimension order routing in hexagonal torus topologies}
		
		So, existing solutions have two problems: trying 12 options and picking one
		is a bit kludgey and there are actually more options than that.
		
		\subsection{Simpler minimal hexagonal torus vectors}
			
			If you redraw your route such that it is sourced from bottom left corner
			(which we'll now call (0, 0)), there are four possible ways this route
			could wrap.
			
			TODO: DESCRIBE WAYS OF WRAPPING
			
			For each of these wrappings, all the possible routes we can take are
			strictly limited in terms of the dimensions used since we're stuck in a
			corner.
			
			In each case, the function computing the minimal hex vector function
			simplifies to a much simpler operation.
			
			TODO: DESCRIBE MINIMUM VECTOR LENGTH FUNCTIONS FOR EACH CASE
			
			This gives us a cheap way to compute which of the four possible wrappings
			are shortest. Based on this we can pick one of at most two (is this
			easily provable?) vectors in some fair manner to reduce load. This vector
			can then be minimised in the usual way.
			
			This also leads to confirming a theoretical result giving the length of a
			shortest path in a hexagonal torus topology.
			
			TODO: DESCRIBE HOW TO GET LENGTH FUNCTION AND COMPARE WITH \cite{xiao04}
		
		\subsection{Generating spiralling routes}
			
			In non-hexagonal torus topologies the previous technique would reveal all
			possible shortest vectors (e.g. in cases where you can wrap more than one
			way). Unfortunately, due to the addition of a non-orthogonal axes,
			hexagonal toruses also have other cases when the width and height do not
			match.
			
			TODO: TORUS SPIRALLING EXAMPLE
			
			It is possible to calculate the maximum number of spirals thus:
			
			TODO: DESCRIBE HOW MAX NUMBER OF SPIRALS IS COMPUTED
			
			Given a number of spirals, the vector can be updated this (note that the
			change does not add a multiple of (1, 1, 1) but also does not result in
			the vector changing length and thus becoming non-minimal!).
			
			TODO: DESCRIBE TRANSFORMATION
			
			TODO: PROVE THAT MINIMALITY IS MAINTAINED
		
		\subsection{Proof of completeness}
		
			TODO: PROOF OF COMPLETENESS BY EXHAUSTIVE SEARCH
	
		\subsection{Conclusions}
			
			This approach is simpler, smaller, has sounder theoretical basis, and
			finds more routes than alternatives. This is good for load balancing and
			fault avoidance and also good for completeness.


	\chapter{Routing packets in large SpiNNaker machines}
	
	\label{sec:routing}
	
	So far, this thesis has focused on tackling the practical challenges
	resulting from SpiNNaker's hexagonal torus network topology. In this chapter,
	I adjust my focus towards the practical challenges resulting from SpiNNaker's
	large scale. Faults in large systems are inevitable and in the half-million
	core, \num{28800} chip SpiNNaker machine recently completed at the University
	of Manchester, around \SI{1}{\percent} of chips exhibited faults\footnote{Of
	the faulty chips discovered, the vast majority of faults are attributed to a
	currently unknown SDRAM failure}. These faults result in gaps and broken
	links in the network topology which routing algorithms must avoid in order to
	ensure correct system operation.
	
	In this chapter I tackle the problem of extending existing routing algorithms
	for SpiNNaker's network to enable them to route-around known, static faults.
	Though dynamic or transient faults may also occur, in this work such faults
	are ignored and other techniques, such as protocol-level fault tolerance, are
	relied on instead.
	
	Numerous heuristic-based fault-tolerant routing algorithms exist which target
	different network topologies and router architectures. Unfortunately as I
	will show, these algorithms are not portable and rely on or attempt to work
	around specific features of their target network architecture. In particular,
	existing work is dominated by the challenge of developing routing schemes
	which avoid or defuse network deadlocks. Due to SpiNNaker's unconventional
	use of timeout-based flow-control, it is not subject to the routing
	restrictions present in other architectures intended to cope with deadlocks.
	
	In this chapter I introduce a graph-search based post-processing step for
	non-fault-tolerant routing algorithms which guarantees routability in
	SpiNNaker systems without disconnected subregions. I also demonstrate that
	this technique introduces both negligible computational overhead to the
	routing algorithm runtime and resulting network performance.
	
	TODO: NOTE THE FAULT RATES ENCOUNTERED IN PRACTICE...
	
	\section{Related work}
		
		Existing work on routing in SpiNNaker's network has ignored the challenge
		of avoiding faults and instead focused on producing efficient multicast
		routes. As a result this section is broken into two halves. In the first
		half I survey the existing non-fault-tolerant approaches to routing used in
		SpiNNaker to-date. In the second I discuss the approaches to fault tolerant
		routing taken in other systems.
		
		\subsection{Multicast routing in SpiNNaker}
			
			Various fault-intolerant multicast routing algorithms exist for many
			networks and a number have been proposed and evaluated specifically in the
			context of SpiNNaker.
			
			In 2012, Davies \emph{et al.} evaluated the use of three common torus
			routing algorithms in SpiNNaker and found that simple oblivious routing is
			suitable for typical neural applications \cite{davies12}. The three
			routing techniques are:
			
			\begin{description}
				
				\item[Dimension Order Routing (DOR)] Packets are routed along each
				dimension (e.g. $X$, $Y$ and $Z$) in turn until no further hops are
				available in that direction.  The order in which the dimensions are
				traversed is fixed.
				
				\item[Right Turn Only Routing (RTOR)] As in DOR except the dimension
				order is chosen such that routes only contain right-turns.
				
				\item[Longest Dimension First Routing (LDFR)] As in DOR except the
				dimension order is chosen in descending order of number of hops in each
				dimension.
				
			\end{description}
			
			These unicast routing techniques are converted into a multicast routing
			algorithm by merging together the routes produced between the source node
			and each destination node as illustrated in figure
			\ref{fig:simple-routers}.
			
			\begin{figure}
				\center
				\begin{subfigure}{0.3\linewidth}
					\center
					\buildfig{figures/simple-routers-dor.tex}
					
					\caption{DOR}
					\label{fig:simple-routers-dor}
				\end{subfigure}
				\begin{subfigure}{0.3\linewidth}
					\center
					\buildfig{figures/simple-routers-rtor.tex}
					
					\caption{RTOR}
					\label{fig:simple-routers-dor}
				\end{subfigure}
				\begin{subfigure}{0.3\linewidth}
					\center
					\buildfig{figures/simple-routers-ldfr.tex}
					
					\caption{LDFR}
					\label{fig:simple-routers-dor}
				\end{subfigure}
				
				\caption{Example multicast routes produced by merging together unicast
				routes from a central source node to each destination node.}
				\label{fig:simple-routers}
			\end{figure}
			
			In 2014, Navaridas \emph{et al.} introduced two new algorithms, `Enhanced
			Shortest Path Routing' (ESPR) and `Neighbourhood Exploring Routing' (NER)
			which produce multicast routing trees with fewer total hops
			\cite{navaridas14}. In both algorithms, routes are generated sequentially
			for each of the destinations of a route using LDFR. Unlike LDFR, however,
			these algorithms search a limited area of the network for other,
			already-connected destination nodes to which LDFR routes may be
			constructed. If no suitable destination is found, a LDFR route is
			constructed to the source node. Figure \ref{fig:search-regions} illustrates
			the shape of the searched regions of each algorithm. ESPR searches the
			trapezoidal region between the source and destination nodes while NER
			searches a fixed radius out from the destination node.
			
			\begin{figure}
				\center
				\begin{subfigure}{0.45\linewidth}
					\center
					\buildfig{figures/search-regions-espr.tex}
					
					\caption{ESPR}
					\label{fig:search-regions-espr}
				\end{subfigure}
				\begin{subfigure}{0.45\linewidth}
					\center
					\buildfig{figures/search-regions-ner.tex}
					
					\caption{NER}
					\label{fig:search-regions-espr}
				\end{subfigure}
				
				\caption{The ESPR and NER algorithms attempt to connect the node marked
				`D' to the closest node in the shaded region which is connected to the
				source node, `S'. If no connected node is found in the shaded region, the
				LDFR route is taken to `S'. The dotted line indicates the route chosen
				from `D'.}
				\label{fig:search-regions}
			\end{figure}
			
			Unfortunately none of these routing algorithms make any allowance for the
			avoidance of network faults. As a result their utility in real-world
			systems is limited.
		
		\subsection{Fault-tolerant routing}
			
			Numerous fault-tolerant routing algorithms have been proposed for
			super-computer networks. These algorithms are largely constrained by the
			need to maintain deadlock freedom. Since SpiNNaker's routers employ a
			timeout based deadlock-breaking strategy, much of this effort is
			unnecessary in SpiNNaker. As described below, this frequently renders
			existing fault-tolerant routing algorithms unnecessarily complex and
			inflexible.
			
			Deadlocks occur in a network if a cyclic dependency exists on any limited
			resource in the network. For example, as illustrated in figure
			\ref{fig:ring-deadlock}, in a ring network a deadlock may form when every
			node is waiting on the next node to accept a packet before accepting new
			packets from the previous node.
			
			\begin{figure}
				\center
				\buildfig{figures/ring-deadlock.tex}
				
				\caption{A deadlock in a ring network where each node is waiting for
				the next to accept a packet before accepting any further packets.}
				\label{fig:ring-deadlock}
			\end{figure}
			
			To prevent deadlocks, combinations of router microarchitectural features
			and routing restrictions may be employed. For example, a simple
			deadlock-free routing algorithm for mesh and torus networks mandates the
			use of DOR \cite{dally93}. Packets travelling in a -ve direction along
			each axis take priority over those travelling in a +ve direction. Packets
			travelling along the Y axis take priority over those travelling along the
			X dimension. Given these rules it is possible to define a total ordering
			on all hops in the network. Figure \ref{fig:deadlock-free-dor}
			illustrates a $3\times3$ mesh network whose hops have been numbered
			according to the total ordering defined above.  Any `X-then-Y' DOR route
			through this network results in the use of hops labelled with strictly
			increasing numbers. As a result, no cyclic dependencies (and thus no
			deadlocks) may occur.
			
			\begin{figure}
				\center
				\buildfig{figures/deadlock-free-dor.tex}
			
				\caption{Deadlock-free routing of two example routes using DOR in a 2D
				mesh topology. The numbers of the hops taken by each route are given on
				the right.}
				\label{fig:deadlock-free-dor}
			\end{figure}
			
			Unfortunately, the routing restrictions imposed to ensure deadlock
			freedom can result in fault-intolerant routing. In the example above, if
			the node at the bottom-right corner of the figure was faulty, the dotted
			example route would be blocked as no alternative routes are allowed.
			
			In practice, the routing rules used may be more relaxed, for example
			requiring that any route whose length is equal to a DOR must exist to
			guarantee routability \cite{rodrigo09}.
			
			Alternative routing strategies take a hybrid approach whereby an
			efficient but fault-intollerant routing algorithm is used where possible
			and in the presence of faults a less efficient but more robust strategy
			is employed. For example, the Immucube network architecture employs three
			virtual networks which operate independently over the same physical links
			\cite{puente07}. Initially messages are routed using a high-performance
			but potentially-deadlockable routing scheme in the first virtual network.
			If a deadlock is occurs, the deadlocked packet is dropped into the second
			virtual network in which packets are routed using a less efficient but
			deadlock-free but fault-intolerant routing algorithm. Finally, upon
			encountering a fault, packets are dropped onto the third virtual network
			which forms a ring network routing packets to every node in the network.
			
			Releated approaches \cite{mejia06,boppana95} divide the network into
			regions in which different routing rules are enforced to ensure deadlock
			freedom and, when required, fault tolerance.
			
			TODO FIGURE?
			
			The BlueGene/L supercomputer \cite{adiga02} uses DOR for its torus
			network and implements fault-tolerance by sacrificing otherwise
			functioning `lamb' nodes to ensure no route passes through a known dead
			link \cite{ho04}. In figure \ref{fig:lamb-nodes} an example scenario is
			shown where a single dead node is present and all nodes in the same row
			or column as the dead node have been made into lamb nodes. The lamb nodes
			may not be used in an application except as a through-route for other
			traffic. This pattern of lamb nodes guarantees that all dimension-order
			routes between all pairs of non-lamb nodes are not obstructed by the
			faulty node. This approach trades use of higher performance routing
			logic for wasted resources. This type of approach is most appropriate
			when algorithmic routing is used and routing rules are inflexible.
			
			\begin{figure}
				\center
				\buildfig{figures/lamb-nodes.tex}
				
				\caption{`Lamb' nodes may be disabled to ensure DOR will never
				encounter a fault.}
				\label{fig:lamb-nodes}
			\end{figure}
			
			Other algorithms proposed for the BlueGene architecture attempt to avoid
			the need for lamb nodes by generating routes which reach their destination
			via a `proxy' node \cite{gomez04}. By appropriately selecting the location
			of such a proxy, the existing routing algorithm used by the system can be
			guaranteed to select a route free of faults.
			
			TODO: EXAMPLE OF PROXY ROUTING TO AVOID FAULT
			
			Finally, many algorithms in in the field are distributed and use only local
			information along with limited information from their peers to generate
			routes \cite{fick09b}. In SpiNNaker, route generation is conventionally
			carried out centrally since no special on-chip hardware facilities exist
			for route generation. Centralised route generation also enables the routing
			algorithm to consider all available routes. As a result, there is little
			incentive for the use of distributed routing algorithms on SpiNNaker since
			global system information could be compactly shared for one-off routing
			passes.
			
			Algorithms for other architectures such as IP networks tend to be poor fits
			for static, regular network topologies since they use expensive graph-based
			algorithms for route discovery which aren't necessary here. They also tend
			to heavily feature graph topology discovery etc. which aren't needed here.
			
			Work on fault-tolerance in data centre networks does exploit the regularity
			of the network topology in routing algorithms \cite{guo08,liao12}.
			Unfortunately, the approaches used are not general enough to be applied to
			mesh-like topologies such as the one in SpiNNaker.
			
			Outside the field of computer networks, routing algorithms used to route
			wires across the surfaces of chips are required to solve similar problems
			to fault-tolerant network routing problems in mesh networks. Like mesh
			networks, the routes are defined within a regular Manhattan geometry and
			congested areas, rather than faults must be avoided by the algorithms
			\cite{kahng11}.  Unfortunately, these algorithms are designed for
			occasional batch operation prior to the multi-month process of chip
			manufacturing and so runtimes of hours or days are commonplace
			\cite{nam08}. As such these algorithms would be inappropriate for use
			with applications such as SpiNNaker where users' applications tend to be
			short-lived and thus routing should not be allowed to dominate runtime.
	
	\section{Partial graph search repair}
		
		In this section I introduce a novel post-processing algorithm, Partial
		Graph Search (PGS) repair, for routes produced by non-fault-tolerant
		routing algorithms.
		
		PGS repair guarantees routability for networks with no disconnected
		subregions by using a graph search algorithm to route around faults in the
		original route.  General-purpose graph search algorithms such as Breadth
		First Search (BFS), Dijkstra's Algorithm and A* are guaranteed to find
		shortest-path routes between pairs of points in arbitrary graphs. Such
		algorithms are generally a poor choice in highly regular network topologies
		such as meshes and toruses due to their high computational cost. In PGS
		repair, graph searching is only used for \emph{part} of the routing
		problem: to repair gaps in routes generated by more efficient routing
		algorithms.
		
		Real world super computer architectures are designed to ensure that faults
		are isolated \cite{gara05,alverson12} and thus tend to only impact a
		localised region of the network. Since PGS repair is only needed to route
		around these isolated faults, the space searched by the graph search
		algorithm should be very small in practice resulting in only short
		runtimes. In addition since faults are rare in real-world systems, the
		graph search process will only rarely be invoked.
		
		The PGS repair post-processing technique starts with a route produced by a
		non-fault-tolerant routing algorithm such as ESPR or NER. If this route is
		not obstructed by a fault, the algorithm terminates immediately without
		modifying the route. If the route attempts to use a faulty link, the
		algorithm proceeds as follows.
		
		The routing tree produced by the underlying routing algorithm is broken
		into subtrees wherever it attempts to route through a broken link and
		each subtree is assigned a unique colour, as illustrated in figure
		\ref{fig:pgs-repair-colouring}. From each disconnected subtree's root
		node in turn, a graph search is performed to find a short, fault-free
		route to a subtree node of a different colour. The subtree is then
		attached to the tree discovered by the graph search and re-coloured to
		match the tree it is connected to.
		
		\begin{figure}
			\center
			\begin{subfigure}{0.32\linewidth}
				\hspace*{-1.5em}
				\buildfig{figures/pgs-repair-colouring.tex}
				
				\caption{}
				\label{fig:pgs-repair-colouring}
			\end{subfigure}
			\begin{subfigure}{0.32\linewidth}
				\hspace*{-1.5em}
				\buildfig{figures/pgs-repair-colouring-fix1.tex}
				
				\caption{}
				\label{fig:pgs-repair-colouring-fix1}
			\end{subfigure}
			\begin{subfigure}{0.32\linewidth}
				\hspace*{-1.5em}
				\buildfig{figures/pgs-repair-colouring-fix2.tex}
				
				\caption{}
				\label{fig:pgs-repair-colouring-fix2}
			\end{subfigure}
			
			\caption{PGS repair process example showing a disconnected multicast
			route from A to B, C, D, E and F. $\times$ indicates a broken link.}
			\label{fig:pgs-repair-colouring-steps}
		\end{figure}
		
		For example in figure \ref{fig:pgs-repair-colouring-fix1} a path from the
		root of the subtree containing nodes E and F is found which connects it to
		the subtree rooted at A. Similarly in figure
		\ref{fig:pgs-repair-colouring-fix2} a path is also found connecting the
		subtree containing nodes C and D back to the subtree rooted at node A.
		
		If the routing tree was broken into $N+1$ subtrees by faults there will be
		$N$ subtrees disconnected from the root node. Each of the $N$ iterations of
		the algorithm connect a disconnected subtree to another subtree reducing
		the number of subtrees by $1$ each time. After $N$ iterations, therefore,
		exactly $1$ subtree remains which connects every node in the original
		routing tree without traversing faulty links.
		
		TODO: EXPLAIN THE FIDDLINESS HERE TO ENSURE WE DON'T CREATE LOOPS.
		
	\section{Evaluation \& Results}
		
		The PGS repair technique, by design, is able to work around all possible
		fault patterns which don't completely disconnect parts of the network. This
		result this evaluation focuses on the impact on performance PGS repair
		imposes. The metrics of interest in this evaluation are:
		
		\begin{itemize}
			\item Algorithm runtime
			\item Network congestion
			\item Routing table utilisation
		\end{itemize}
		
		\subsection{Traffic Patterns}
			
			In this evaluation, two standard benchmark multicast traffic patterns are
			used which have been used in previous research into SpiNNaker's network:
			
			\begin{figure}
				\center
				\buildfig{figures/traffic-distribution-centroids.tex}
				
				\caption{An example 4-centroid distribution with four centroids. The
				$\times$ marks the location of the origin node. Lighter colours
				indicate greater likelihood of a connection.}
				\label{fig:traffic-distribution-centroids}
			\end{figure}
			
			\begin{description}
				
				\item[Uniform] Destinations are chosen with uniform probability
				anywhere in the machine.
				
				\item[$N$-Centroids] Destinations are clustered around one of $N$
				randomly chosen `centroids' as illustrated in figure
				\ref{fig:traffic-distribution-centroids}.
				
			\end{description}
			
			The uniform traffic pattern is widely used in networks research
			\cite{dally04,davies12} while the centroids model was developed
			specifically to reproduce the traffic patterns found in the neural
			applications SpiNNaker is designed for \cite{navaridas14}. In this work
			we consider 3 centroids.
		
		\subsection{Fault model}
			
			In addition two different fault models are used which are representative of
			the faults found in real SpiNNaker systems:
			
			\begin{figure}
				\center
				\begin{subfigure}{0.48\linewidth}
					\hspace*{-1.5cm}
					\buildfig{figures/fault-example-uniform.tex}
					
					\caption{Uniform}
					\label{fig:fault-example-uniform}
				\end{subfigure}
				\begin{subfigure}{0.48\linewidth}
					\hspace*{-1.5cm}
					\buildfig{figures/fault-example-hss.tex}
					
					\caption{HSS Link}
					\label{fig:fault-example-hss}
				\end{subfigure}
				
				\caption{The two link fault models considered.}
				\label{fig:fault-example}
			\end{figure}
			
			\begin{description}
				
				\item[Uniform] Links are selected and disabled at random (figure
				\ref{fig:fault-example-uniform}).
				
				\item[HSS Link] Groups of links corresponding with randomly selected
				single High-Speed Serial (HSS) link between SpiNNaker boards are disabled
				together (figure \ref{fig:fault-example-uniform}).
				
			\end{description}
			
			The uniform link failure model models isolated failures resulting from
			isolated manufacturing defects in individual links. The HSS Link failure
			model models faults arising from failing or disconnected board-to-board
			links which carry several chip-to-chip traffic flows via a single cable in
			SpiNNaker systems. Though SpiNNaker-specific, the later fault model is
			analogous to failure modes arising in other architectures where a single
			fault may render several links impassable in a single area.
			
			A range of failure rates are explored in this section. My measurements of
			current large-scale SpiNNaker installations the link failure rate is about
			\SI{0.03}{\percent} with failures due to both individual chip-to-chip links
			and board-to-board HSS links. Exact link failure statistics for commercial
			super computer installations are not widely available, however, published
			Mean-Time-Between-Failure (MTBF) statistics place an upper bound on link
			failure rates at a similar \SI{0.03}{\percent} in one-year-old BlueGene/Q
			systems \cite{chiu11}.
			
			Unfortunately presently undiagnosed problem with the SDRAM packaged with
			approximately \SI{1}{\percent} of SpiNNaker chips has rendered these chips
			unusable for most applications. The gaps in the network resulting from the
			loss of these chips currently dominate true link failures leaving just over
			\SI{1}{\percent} of links inoperable.
			
			Surprisingly, research into fault tolerant routing in super computers
			appears to focus on benchmarks with even higher fault rates ranging from
			\SI{3}{\percent} to as high as \SI{7}{\percent}
			\cite{ho04,gomez04,mejia06}.
			
			In this evaluation, fault rates ranging from \SI{0.01}{\percent} to
			\SI{5}{\percent} are considered to cover both realistic fault levels
			along with the more extreme cases considered in related work.
		
		\subsection{Base routing algorithm}
			
			Since the PGS repair process is routing algorithm agnostic all
			experiments use the NER algorithm which has been found to be appropriate
			for SpiNNaker applications \cite{navaridas14}.
		
		\subsection{Algorithm runtime}
			
			To assess the impact of the PGS repair process on routing algorithm
			runtime, the algorithm was used to process a large number of randomly
			generated routing problems and the runtime recorded.
			
			\num{10000} one-to-sixteen multicast routing problems were generated in a
			$256\times256$ hexagonal torus topology, the largest size possible for a
			SpiNNaker system. Other quantities of multicast destinations were also
			evaluated but are omitted for brevity since the pattern of results are
			similar to those outlined here.
			
			TODO: APPENDIX WITH OTHER RUNS?
			
			The NER and PGS repair algorithms were written in C and compiled with GCC
			4.8.3 with \verb|-O2| level optimisations and executed on a cluster of
			idle workstations with 3.10 GHz Intel Core-i5-2400 CPUs.
			
			\begin{figure}
				\center
				\buildrplot{figures/routing-runtimes.R}
				
				\caption{Mean runtime of routing and PGS repair overhead. PGS repair
				overhead is stacked above the routing runtime (i.e. bars do not
				overlap). Error bars indicate 95\% confidence interval. Note different
				Y-scale for HSS link and uniform fault models.}
				\label{fig:routing-runtimes}
			\end{figure}
			
			Figure \ref{fig:routing-runtimes} shows the average runtimes recorded for
			both the NER and PGS repair algorithms. In fault-free networks the
			PGS-repair post-processing step is not required and incurs no penalty
			while the runtime of the algorithm grows with the fault rate for both
			fault and traffic models.
			
			Notably the HSS fault model results in longer runtimes for the PGS repair
			process compared with an equivalent fault-density of uniform faults.
			Because the HSS fault model produces contiguous lines of faults the PGS
			repair algorithm must construct a longer path to avoid the fault.  Since
			the space explored by a graph algorithm typically grows with $O(H^2)$
			with respect to the hops in the discovered route, $H$, this increase in
			search distance has a large impact on the runtime of the PGS repair
			process.
			
			The runtime of the PGS repair algorithm remains roughly in proportion to
			the runtime of the underlying routing algorithm with respect to different
			traffic models. The centroid traffic pattern tends to result in routes
			with fewer hops than a uniform traffic pattern with the same number of
			destination nodes as segments of routes are often shared between
			destination nodes. Since the NER algorithm's runtime is strongly related
			to the number of hops in the output route the runtime of the algorithm is
			greater for uniform traffic. Likewise the probability of PGS repair being
			required increases with the number of hops in route and hence the runtime
			of the PGS repair algorithm increases roughly in proportion.
		
		\subsection{Routing table usage}
			
			In order to gain a realistic measure of routing table usage it is
			necessary to determine the effect of many routes being generated for a
			single set of faults. To enable a sufficiently large number of sample to
			be collected the experimental setup considered previously is reduced to a
			network containing $48\times48$ nodes.
			
			\num{1000} $48\times48$ node network models are produced according to the
			HSS link and uniform fault models. For each of these models
			$48\times48\times16=$~\num{36864} one-to-sixteen routes are generated using
			the centroid and uniform traffic models. This corresponds to one
			multicast route per application core. As is convention in SpiNNaker,
			routing table entries are inserted for each route at the source of the
			route, at each destination and at each corner or fork. The number of
			routing table entries at each node in the model is counted and the
			maximum number of entries in a single node is reported for each network
			model.  The \emph{maximum} number of routing entries of any router was
			chosen since the number of entries available per SpiNNaker router is
			bounded by hardware.
			
			\begin{figure}
				\center
				\buildrplot{figures/routing-entries.R}
				
				\caption{Violin plot showing the distribution of maximum table sizes
				for \num{1000} random networks. The red line at \num{1024} entries
				indicates the size of SpiNNaker's routing tables.}
				\label{fig:routing-entries}
			\end{figure}
			
			
			Figure \ref{fig:routing-entries} shows the distributions of the largest
			routing table sizes for each fault and traffic model.
			
			\begin{figure}
				\center
				\begin{subfigure}{0.48\linewidth}
					\center
					\buildfig{figures/hss-link-routing-table-usage.tex}
					
					\caption{Routing table entries}
					\label{fig:hss-link-routing-table-usage}
				\end{subfigure}
				\begin{subfigure}{0.48\linewidth}
					\center
					\buildfig{figures/hss-link-resource-usage.tex}
					
					\caption{Routes passing through chip}
					\label{fig:hss-link-resource-usage}
				\end{subfigure}
				
				\caption{The impact of a HSS link fault on routing table usage and
				congestion. Each hexagon represents a single chip, the red line
				indicates the chip-to-chip connections broken by the HSS link fault.}
				\label{fig:hss-link-usage}
			\end{figure}
			
			The HSS link failure model has a much greater impact on peak routing
			table resource usage than uniform link failures for a given fault rate.
			This is because HSS link faults result in a large concentration of routes
			being disrupted and then re-routed around the same obstacle in a single
			location. Figure \ref{fig:hss-link-routing-table-usage} shows how routing
			table usage varies around a HSS link fault in one instance of the
			experiment. There are clear peaks in routing table usage around the ends
			of the line of faults which result from routes produced by PGS repair
			finding shortest paths around the edge of the faults.
		
		\subsection{Network congestion}
			
			To measure the impact of PGS repair on network congestion, two
			experiments were performed, one using the same model used to measure
			routing table usage and one based on tests run on SpiNNaker hardware.
			
			For each of the network fault and traffic pattern described previously,
			the paths taken for the \num{36864} one-to-sixteen multicast routes
			generated are used to compute the number of times each link in the
			network is used. The number of routes passing through the most-used link
			is then recorded, giving an indication of the level of congestion in the
			network.
			
			\begin{figure}
				\center
				\buildrplot{figures/routing-resource.R}
				
				\caption{Violin plot showing the distribution of maximum
				routes-per-chip for \num{1000} random networks.}
				\label{fig:routing-resource}
			\end{figure}
			
			The results are presented in figure \ref{fig:routing-resource} and follow
			the same trends as the results previously shown for routing table usage.
			Again, HSS link faults result in routes with the greatest congestion due
			to the concentration of routes finding shortest paths around an obstacle
			(see \ref{fig:hss-link-resource-usage}).
			
			To verify that the results above, an additional experiment has been
			carried out which attempts to mimic the model used previously in actual
			SpiNNaker hardware. In these experiments a large SpiNNaker machine is
			divided into independent 48-board (2304-chip) sections. Because the
			48-board systems used in these experiments are cut out of a larger
			machine, they lack wrap-around links and thus form hexagonal mesh
			topologies, rather than hexagonal toruses.
			
			Due to the SDRAM issue described above, fault rates below
			\SI{1}{\percent} cannot be modelled.  To simulate higher fault rates,
			additional links are disabled in software according to the fault models
			described used previously. Since some faults are due to genuine hardware
			faults, these faults cannot be placed randomly in each experiment. To
			reduce, bias each combination of fault rate, fault model and traffic
			pattern is repeated XXX times across randomly chosen physical machines.
			
			XXX 1-to-XXX routes are generated in both uniform and XXX-centroid
			distributions as used throughout this evaluation. Synthetic network
			traffic is generated at the source of each route following a Bernoulli
			distribution. Traffic consumers running on all destination nodes accept
			packets as quickly as possible from the network and log their arrival.
			The Bernoulli probability is set the same for every route's traffic
			generator and increased in steps of XXX and the number of packets dropped
			in an XXX second period logged. The network is considered saturated once
			less than \SI{99}{\percent} of packets successfully arrive at their
			destination.
			
			Figure \ref{XXX} shows the distributions of the saturation points for
			each experimental configuration.
			
			TODO: ANALYSIS
		
	\section{Conclusions}
		
		In this chapter I described how SpiNNaker's unconventional network and
		router architecture render existing fault tolerant routing algorithms
		unsuitable. I introduced PGS repair, a post-processing technique for
		existing non-fault tolerant routing algorithms designed for SpiNNaker such
		as NER.
		
		Unlike some other fault tolerant routing algorithms for other
		architectures, PGS repair is able to work-around arbitrary fault patterns
		by exploiting SpiNNaker's inbuilt deadlock avoidance mechanisms. In the
		presence of realistic failure rates of up to \SI{1}{\percent}, only small
		overheads of up to XXX, XXX and XXX for in algorithm runtime, routing table
		usage and network performance are incurred respectively. This low
		performance overhead makes PGS repair appropriate for use in real
		applications. At the time of writing the algorithm has been successfully
		used in a number of neural and non-neural SpiNNaker applications.
		
		At more extreme fault rates not expected in real-world systems, the
		algorithm still functions correctly but the results incur much greater
		routing table and congestion overheads, particularly when faults are
		concentrated. Future extensions to this algorithm might aim to reduce this
		overhead by producing longer and more varied routes around faults to even
		out the load.

	\chapter{Placing applications in large SpiNNaker machines}
	
	In the previous chapter I tackled the problem of scale in generating routes
	for very large networks such as SpiNNaker. In this work the centroid traffic
	pattern was used as an approximation of the expected network traffic
	generated by `well behaved' neural network simulation software running on
	SpiNNaker. The traffic produced largely exhibits strong locality, that is
	most communication occurs between either nearby nodes or clusters of nodes.
	In reality, neural simulation applications are not specified geometrically
	but rather as abstract graphs of communicating neurons
	\cite{davison08,eliasmith13}. Applications must then \emph{place} these
	neurons onto nodes in a SpiNNaker system, attempting maximise communication
	locality.
	
	In this chapter I re-evaluate the suitability of simulated annealing as a
	technique for finding high quality placements for large parallel
	applications. Though this technique had fallen out of fashion in the field of
	application placement by the early 1990s, it has found wide use for placing
	components in computer chip and FPGA designs. In the intervening years,
	placement problems in super computers have grown in size from tens or
	hundreds of nodes to millions, a scale at which chip placement techniques
	were operating in the mid 1990s. I adapt the simulated annealing algorithm
	used by the VPR academic circuit placement software to produce placements for
	applications running on SpiNNaker. In that in a range of real and synthetic
	benchmarks simulated annealing produces high quality placements enabling
	efficient use of SpiNNaker's network resources.
	
	
	%In the field of chip design, Moore's `Law' \cite{moore65,moore75} observes a
	%similar exponential growth in the number of components within a single chip.
	%Today modern processors contain billions of components and an analagous
	%placement problem exists in attempting to place interconnected components
	%near to eachother. In this chapter I explore the techniques used for circuit
	%placement and adapt one such technique, Simulated Annealing (SA)
	%\cite{kirkpatrick83}, for use in application placement. Despite some early
	%interest in SA for application placement in the 1980s and early 1990s, the
	%technique has since fallen out of favour. I find that at the scales of modern
	%placement problems SA-based placement is able to produce solutions of
	%superiour quality to contemporary methods.
	%
	%TODO: SUMMARISE RESULTS...
	
	\section{Related work}
		
		The placement problem has been tackled independently in the literature by
		researchers in both the application and chip placement communities. In this
		survey I cover application and chip placement separately as these two
		communities have remained largely isolated from one another. First I
		explore the techniques applied to application placement before moving on to
		contrast this with the techniques used in circuit placement.
		
		In the application placement literature, the placement problem is often
		referred under the umbrella term `mapping'. Unfortunately term is often
		used more broadly to include other tasks such as routing and application
		partitioning. To avoid ambiguity I use the term `placement', as preferred
		by the chip and FPGA design communities, to refer specifically to the
		problem of assigning nodes in an application's communication graph to nodes
		in a machine's connectivity graph.
		
		\subsection{Application placement algorithms}
			
			TODO: GENERAL INTRO
			
			\subsubsection{Application-specific approaches (manual placement)}
				
				In the case of some applications such as finite element modelling
				\cite{bermejo13}, the structure of the problem itself leads to a
				natural placement of the computation on nodes in a machine. For example
				when simulating a 3D volume in an node super computer with a $3 \times
				4 \times 2$ 3D torus or mesh topology network, the modelled volume
				might be divided into as in figure \ref{fig:fem-partitioning}. Each
				cuboid in the model is then assigned to the corresponding node in the
				network topology.
				
				\begin{figure}
					\center
					\buildfig{figures/fem-partitioning.tex}
					
					\caption{Example partitioning of a 3D space to fit into a super
					computer with a $3\times4\times2$ torus or mesh topology.}
					\label{fig:fem-partitioning}
				\end{figure}
				
				When the number of dimensions in a problem do not match that of the
				underlying network architecture, the common solution is to either
				divide only along a subset of the axes or to divide into additional
				pieces on the existing axes \cite{gilge14}.
			
			\subsubsection{Sequential placement}
				
				In the case where a placement solution is non-obvious one of the
				simplest and most popular strategies is to apply a simple sequential
				placement algorithm. Sequential placement algorithms function by
				iterating over the vertices in the application's communication graph
				and assigning them to a free node in the target machine. Sequential
				placement algorithms are differentiated by the order in which they
				iterate over vertices in the communication graph and fill nodes in the
				target machine. A number of widely used orderings are described below.
				
				\begin{figure}
					\center
					\begin{subfigure}{0.32\linewidth}
						\center
						\buildfig{figures/sequential-row-order.tex}
						\caption{Row-order}
						\label{fig:sequential-row-order}
					\end{subfigure}
					\begin{subfigure}{0.32\linewidth}
						\center
						\buildfig{figures/sequential-alternating.tex}
						\caption{Alternating}
						\label{fig:sequential-alternating}
					\end{subfigure}
					\begin{subfigure}{0.32\linewidth}
						\center
						\buildfig{figures/sequential-hilbert.tex}
						\caption{Hilbert curve}
						\label{fig:sequential-hilbert}
					\end{subfigure}
					
					\caption{Space-filling curves in 2D mesh and torus topologies.}
					\label{fig:sequential}
				\end{figure}
				
				Super computer management software such as SLURM \cite{yoo03} and Blue
				Gene's system software \cite{gilge14} by default na\"ively iterate over
				vertices in an application communication graph in the order they are
				provided. The nodes in the target machine are then iterated over in a
				simple space-filling curve through the network topology. Figure
				\ref{fig:hilbert-placement} illustrates the default patterns available
				in these software packages. The row-order (figure
				\ref{fig:sequential-row-order}) and alternating (figure
				\ref{fig:sequential-alternating}) curves correspond with 2D versions of
				the default node assignment orders used in SLURM and BlueGene systems.
				
				\begin{figure}
					\center
					\buildfig{figures/hilbert-placement.tex}
					
					\caption{A Hilbert curve, coloured from blue to red.}
					\label{fig:hilbert-placement}
				\end{figure}
				
				The Cray extensions to SLURM software provide a Hilbert curve
				\cite{hilbert91} (figure \ref{fig:sequential-hilbert}) node assignment
				order. Unlike the row-order and alternating space filling curves the
				Hilbert curve ensures that pairs of vertices close together in the node
				iteration order are also close together in the target machine's network
				\cite{moon01, zumbusch99}. Figure \ref{fig:hilbert-placement} shows a
				5$^\textrm{th}$-order Hilbert curve where each point in the curve is
				coloured according to its position along the curve. In this figure it
				is possible to see that nearby positions in the curve (which share
				similar colours) are also close in 2D space.
				
				When the proximity of vertices in the vertex-ordering supplied by an
				application is a good estimator of those vertices communication
				requirements, the sequential assignment schemes discussed above can be
				very effective. These techniques have also proven adequate in
				small-scale and densely connected applications such as early neural
				simulations running on prototype SpiNNaker machines with tens of nodes
				\cite{galluppi10} but growing beyond this scale has proven problematic.
				
				\begin{figure}
					\center
					\begin{subfigure}{0.45\linewidth}
						\center
						\buildfig{figures/rcm-initial.tex}
						
						\caption{Original permutation}
						\label{fig:rcm-initial}
					\end{subfigure}
					\begin{subfigure}{0.45\linewidth}
						\center
						\buildfig{figures/rcm-sorted.tex}
						
						\caption{RCM permutation}
						\label{fig:rcm-sorted}
					\end{subfigure}
					
					\caption{Adjacency matrix representation of a graph before and after
					permutation by the RCM algorithm.}
					\label{fig:rcm}
				\end{figure}
				
				A number of algorithms have been proposed for automatically selecting
				good vertex iteration orders, typically using a graph-traversal based
				heuristic. A typical method, described by Hoefler \emph{et al.}
				\cite{hoefler11} exploits the Reverse-Cuthill-McKee (RCM) algorithm
				\cite{cuthill69}. An application's communication matrix is represented
				as an adjacency matrix, $M$, where $M_{i,j}$ is 1 if node $i$ is
				connected by an edge to node $j$ and 0 otherwise. An example matrix is
				illustrated in figure \ref{fig:rcm-initial}. The RCM algorithm uses a
				simple heuristic to permute the matrix (i.e. renumber the nodes in the
				graph) in order to reduce the bandwidth of the matrix. Figure
				\ref{fig:rcm-sorted} shows the RCM-permuted version of the example
				adjacency matrix. When a graph's vertices are ordered as in a
				bandwidth-reduced sparse matrix, vertices close together in the
				ordering are likely to communicate while those further apart tend not
				to communicate.
				
			\subsubsection{Optimisation-based Placement}
				
				% Citations from short report about optimisation in placement...
				% \cite{chen06,jeannot14} and \cite{jeannot10} ("subsets of apps")
				
				In the academic community, a number of attempts have been made to use
				more sophisticated optimisation algorithms for the placement of
				applications. In 1985, Steele \cite{steele85} proposed the use of
				simulated annealing for placing applications in the 6D torus topology
				of the 64 node `Caltech Cosmic Cube' machine. Simulated annealing,
				originally developed by Kirkpatrick \emph{et al.} \cite{kirkpatrick83},
				is a general-purpose optimisation algorithm which works by analogy to
				the physical process of annealing. In brief simulated annealing
				functions by randomly swapping vertices in a candidate placement
				solution, accepting swaps which move connected vertices closer together
				and rejecting some proportion of swaps which move connected vertices
				further apart. The simulated annealing algorithm is described in detail
				later in this chapter.
				
				Towards the end of the 1980s, application placement appeared to be
				becoming less important as super computer network architectures
				improved:
				%
				\begin{displayquote}
					``Careful placement was necessary because of the slow communication
					and non-uniform addressing of early concurrent computers. However,
					the development of message passing machines with fast communications
					and a uniform global address space  has made placement less of an
					issue. In such machines a random placement performs nearly as well as
					an optimum placement.''
					
					\noindent --- W. Dally, 1987 \cite{dally87}
				\end{displayquote}
				%
				In addition, network and problem sizes remained small, so small in fact
				that linear-programming based optimal placement still appeared in
				benchmarks comparing placement algorithms \cite{xu91}. In this
				environment, simpler sequential placement algorithms gained favour over
				more computationally expensive algorithms such as simulated annealing.
				
				As problem and machine sizes have grown and network utilisation has
				once again become an important factor in application performance
				\cite{navaridas09b} more complex optimisation algorithms have
				reappeared in the literature. One popular approach employs graph
				partitioning algorithms such as METIS \cite{karypis98} to perform
				recursive bipartitioning based placement
				\cite{phillips14,hoefler11,pellegrini96}.  This placement process is
				illustrated in figure \ref{fig:partitioning}.
				
				In the first step, the application communication graph and machine
				connectivity graph are bipartitioned such that the number of edges
				between partitions is minimised. Each half of the communication graph
				is associated with one of the halves of the machine connectivity graph.
				The partitioning process is then repeated recursively on each of the
				two communication and connectivity graph pairs. The process halts when
				the graphs can no longer be partitioned at which point the vertices in
				the communication graph are placed on their associated node.
				
				\begin{figure}
					\center
					\buildfig{figures/partitioning.tex}
					
					\caption{Illustration of application placement by recursive
					partitioning.}
					\label{fig:partitioning}
				\end{figure}
				
				TODO: PARTITIONING IS GREAT AND ALL BUT QUALITY ISN'T ALWAYS GREAT AND
				IT DOESN'T DEAL WELL WITH MULTI-CONSTRAINT SCENARIOS E.G. PROCESSOR AND
				MEMORY RESTRICTIONS.
				
				Unfortunately, many of these simply aren't suited to the scale of
				neural applications running on SpiNNaker (e.g. only cope with tens of
				nodes while SpiNNaker may contain hundreds of thousands).
				
				Additionally, a number of algorithms have been developed which make
				assumptions about the topologies of the problem or network. Tree match
				for example attempts to map tree-shaped problems to tree-shaped
				networks. Such algorithms can be highly effective but again do not
				apply to SpiNNaker or its neural applications.
		
		\subsection{Chip placement algorithms}
			
			The chip-design industry has, for many years, dealt with problems
			analogous to the task of placing super computer jobs in a way suited to
			SpiNNaker. Modern CPUs have millions or billions of components with
			strictly fixed connectivity. CPU designers must place each of these onto
			a chip such that the connection lengths are controlled to reduce
			congestion and increase performance. As such, these algorithms are
			ideally suited to future super computer placement work since they already
			operate at the scales required \cite{nam07}.
			
			\subsubsection{Cost functions}
				
				HPWL is popular but a bit crap for high fan-outs. It is, however, quite
				simple.
				
				TODO: SELECT A BETTER COST FUNCTION...
			
			\subsubsection{Simulated annealing}
				
				One of the oldest techniques used for circuit placement is simulated
				annealing and this remains popular today thanks to its sheer
				versatility (see VPR, other open FPGA tools).
				
				SA works by analogy with the physical process of annealing.
				The simulated annealing algorithm works by selecting random pairs of
				components on a chip, swapping them and evaluating some cost function.
				If the swap reduces the cost function, it is kept, if not, depending on
				a function of the current temperature and the cost introduced by the
				swap.
				
				TODO: ILLUSTRATION OF SIMULATED ANNEALING SWAP OPERATION
				
				By occasionally allowing costly swaps, the annealing algorithm avoids
				becoming trapped in local minima. As the algorithm proceeds, the
				temperature is slowly reduced and with it the proportion of costly
				swaps which are retained. This causes the placement to move from
				exploration early on towards refinement later on.
				
				The temperature schedule of an annealing algorithm is critical to its
				success. In general these schedules are computed based on the
				performance of the algorithm as it runs. In VPR the following schedule
				is used.
				
				TODO: DESCRIBE VPR'S SCHEDULE
				
				TODO: FIND AND DESCRIBE ALTERNATIVE SCHEDULE?
				
				Unfortunately, SA is very difficult to parallelise, especially in the
				case of placement. As a result, its scalability has been limited and
				resulted in significantly reduced usage in recent work.
			
			\subsubsection{Partitioning placement}
				
				Partitioning based placement solves the placement problem using
				graph-partitioning recursively on the problem graph to assign each part
				of the circuit to some area in the super chip. Though a number of
				algorithms have proven successful in academic placement contests over
				the years, they are not popular in industrial settings.
			
			\subsubsection{Analytical placement}
				
				In analytical placement, cost function for the circuit graph is
				approximated in a form which is amenable to solutions with standard
				numerical or symbolic algebraic techniques. Using these techniques,
				exact minimum cost (in terms of the approximation) configurations can
				be obtained.
				
				Quadratic placement is a popular analytical placement technique which
				approximates the cost of a placement as the sum of the squares of the
				distances between connected circuit elements.
				
				TODO: FIGURE EXAMPLE QUADRATIC PLACEMENT PROBLEM AND SOLUTION
				
				As such this gives a quadratic cost function like so which we must
				minimise.
				
				TODO: QUADRATIC COST EQN
				
				To minimise the function we differentiate and solve using simple
				symbolic manipulation.
				
				TODO: QUADRATIC COST SOLUTION
				
				Unfortunately, quadratic placement doesn't contain any congestion
				relief by default so various schemes exist. For example, extra anchor
				nodes are inserted which gently pull the circuit components apart from
				each other. As a result, the algorithm generally proceeds by iterating,
				regenerating anchors each time.
				
				Other non-quadratic analytical methods exist too with numerical
				solutions. The approaches are often similar.
			
			\subsubsection{Hierarchical clustering}
				
				Many placement algorithms scale super-linearly with problem size and so
				larger problems become increasingly problematic to handle. To solve
				this problem clustering techniques are first applied to first simplify
				the placement problem. A solution is then found at the coarse level and
				then hierarchically fleshed out.
				
				Various clustering algorithms are in use.
				
				TODO: TALK ABOUT CLUSTERING IN PLACEMENT...
				
				TODO: DESCRIBE THE ALGORITHM I IMPLEMENTED.
	
	\section{Application placement by simulated annealing}
		
		\label{sec:placement-by-annealing}	
		
		I have implemented a simplified SA based application placement algorithm
		based on the approach used in the popular VPR place and route tool chain.
		The algorithm is written in C and is optimised for experimentation rather
		than performance but is production-ready. It has been integrated into the
		`Rig' SpiNNaker software tools and has been used to place very large
		simulations. More on that later.
		
		\subsection{Representation}
			
			Model each chip as having a quantity of various resources (e.g. Cores,
			SDRAM) available. The application graph consists of vertices which each
			consume some quantity of these resources. Vertices must be placed on a
			single chip such that the resources required on a given chip do not
			exceed those available. Vertices are then interconnected by 1:N nets with
			weights which act as hints. The nets are treated as a soft constraint:
			vertices connected via a net will, ideally, be placed near to each other,
			with priority being given to nets with higher weights. Additionally there
			will be a list of placement constraints (see later).
			
			A key observation is that while vertices in an application may frequently
			have a 1:1 correspondence with application cores, this need-not be the
			case. For example, a vertex may represent a block of SDRAM which is
			shared. A vertex may also represent some other resource, for example,
			external IO availability. By making these resource types user-defined,
			applications programmers can express flexible hard-constraints on their
			application.
			
			Another observation is that generic soft constraints can be expressed may
			be expressed using a net with an appropriate weight.
			
			As a result of these facilities, application programmers can easily
			express their own application-specific hard and soft placement
			constraints without having to modify the algorithm. This representation
			has become a de-facto standard for placement problem interchange for
			SpiNNaker applications.
		
		\subsection{Cost function}
			
			At present I've used HPWL despite this being really bad for high-fan-out
			multicast and totally ignorant to the hexagonal nature of SpiNNaker...
			
			To compute bounding boxes for tori I use the following approach. For each
			dimension, sort the points on that dimension and find the largest gap
			between them on a ring. The bounding box goes the other way.
			
			TODO: FIGURE ILLUSTRATING BOUNDING BOX COMPUTATION FOR TORI.
		
		\subsection{Annealing schedule}
			
			The annealing schedule is that used by VPR. Despite being for circuit
			placement, it seems to work jolly well.
			
			TODO: DESCRIBE AND RATIONALISE THE SCHEDULE
		
		\subsection{Constraint handling}
			
			Various hard and soft constraints may be expressed by software
			approaches. For each we explain how they may be handled by the placement
			algorithm:
			
			\subsubsection{Location Constraint}
				
				The vertex is placed on a chip and removed from the set of movement
				candidates.
			
			\subsubsection{Same-chip constraint}
				
				When two vertices must always be placed on the same chip they are
				simply combined into one vertex which consumes the sum of their
				resources. Placement then treats them as one chip and thus is forced to
				atomically place the vertices.
			
			\subsubsection{Reserve resource constraint}
				
				Simply reduce resource availability on that chip.
			
			\subsubsection{Keep near Ethernet}
				
				Simply add a net.
	
	\section{Evaluation}
		
		\label{sec:placement-results}
		
		Though benchmarks exist for super computer loads and chip placement tasks,
		such things don't exist for neural applications. As a result I use a
		selection of real applications for SpiNNaker along with some synthetic
		benchmarks based on biological data.
		
		\subsection{Benchmark networks}
			
			First some real networks.
			
			Some nengo networks: SPAUN: `The world's largest functional brain model'.
			Word-net network from Jamie: Example of some learning.
			
			TODO: DESCRIBE SHAPE OF NENGO NETWORKS
			
			Some PyNN networks: Microcortical column model from PyNN. Note almost
			broadcast connectivity but varying weights. Try and extract a vision
			netlist from Anna. Maybe try and get a netlist for Tom's barrel cortex.
			
			Now for some artificial networks. Pipeline, noisy pipeline, mesh,
			Gaussian 2D.
		
		\subsection{Experiments}
			
			We compare random, linear, greedy and annealing based placement
			approaches to placement. We compare static metrics (such as mean/max
			congestion, table usage) along with experiments based on simulated
			network traffic in real hardware. Network Tester generates artificial
			traffic in proportion with the weights given for each model. We compare
			the relative level of traffic sustainable. We also consider use of
			machines of various sizes.
		
		\subsection{Results}
			
			SA placement is slow but rather effective, especially for some networks.
			Generally worth doing. Will need to be sped up for very large machines...
			
			TODO: EXPERIMENTS!
	

	\chapter{Discussion}

\section{Suitability of the hexagonal torus topology}
	\subsection{Physical scalability}
	\subsection{Routability}
	\subsection{Placeability}

\section{Suitability of the SpiNNaker router}
	\subsection{Deadlock avoidance}
	\subsection{Routing table size}

\section{Suitability of circuit placers for application placement}
	\subsection{Quality}
	\subsection{Runtime}
	\subsection{Routing resources}
	\subsection{Flexibility}
	\subsection{Scalability}


	\chapter{Future research}
	
	In this thesis I have presented a number of new techniques which have made it
	possible to assemble and operate the SpiNNaker super computer. This work
	opens up a range of possibie lines of research to extend this work to future
	architectures and applications. In this chapter I focus on two anticipated
	challenges of future systems: growing scale and greater dynamicism in
	applications.
	
	\section{Scaling up}
		
		TODO: INTRO
		
		\subsection{Grid machine room layouts}
			
			In chapter XXX, I developed a machine room layout for hexagonal torus
			topologies which allowed machines occupying a row of standard
			machine-room cabinets to scale up without the need for long
			interconnecting cables. For larger installations, however, it will be
			necessary to employ multiple rows of cabinets in a 2D arrangement.
		
		\subsection{Routing congestion control}
		
		\subsection{Parallel place and route}
	
	\section{Structural plasticity and dynamic fault-tolerance}
		\subsection{Plasticity models}
		\subsection{Incremental placement}
		\subsection{Incremental routing}
		\subsection{Hot-spare routes}

	\chapter{Conclusions and future research}
	
	The SpiNNaker architecture was designed to tackle the challenges presented by
	the simulation of biologically realistic neural networks. One of its
	distinguishing features is its network architecture which employs both an
	unconventional network topology and multicast router architecture. The
	hexagonal torus topology used by SpiNNaker was chosen to enable greater
	performance while maintaining ease of construction and scalability compared
	with conventional network topologies. SpiNNaker's router design centres
	around packets mimicking the neural `spike' signals they are designed to
	convey by being small, multicast and not guaranteed to arrive at their
	destination.  This novel design, though largely complete before this work
	began, left a number of open problems which this thesis has attempted to
	address.
	
	In this concluding chapter I begin by summarising the answers to the research
	questions raised in chapter~\ref{sec:introduction}. This is followed by a
	discussion of new research topics which have been uncovered by this work.
	
	\section{Answers to research questions}
		
		Each of the three research questions are answered below.
		
		\subsubsection{1. Can the hexagonal torus topology be deployed and used in
		real, large-scale systems?}
		
		In chapter~\ref{sec:building}, I introduced a cabling scheme and assembly
		technique which has been used successfully to build a prototype SpiNNaker
		system with over half a million processor cores using the hexagonal torus
		topology. The techniques shown are expected to enable a final SpiNNaker
		machine of double this size to be built, filling a six metre long row of
		machine-room cabinets.
		
		Though SpiNNaker's processor-count places it amongst some of the world's
		largest supercomputers (see figure \ref{fig:top500-num-processors} on page
		\pageref{fig:top500-num-processors}), it is comparatively compact, filling
		one row of cabinets compared with the warehouse-scale installations found
		in commercial systems. In spite of this, the folding and interleaving
		techniques described allow hexagonal torus topologies to scale to
		arbitrarily large installations without cables which span the machine.
		
		Chapter~\ref{sec:shortestPaths} described an efficient and general
		technique for finding, and enumerating shortest path vectors in hexagonal
		torus topologies. These developments bring the hexagonal torus topology in
		line with other topologies by enabling routing algorithms to exploit all
		possible paths in a network. Further, chapter~\ref{sec:placement}
		demonstrated that placement algorithms are also adaptable to hexagonal
		torus topologies thanks to their similarity to 2D toruses.
		
		Though, as this thesis highlights, hexagonal toruses lack many of the
		intuitive properties enjoyed by other topologies, it is still possible to
		reason about them with only limited computational effort.  Now that the
		practicality and scalability of the topology has also been demonstrated in
		practice, it represents a credible option for future network architectures.
		
		\subsubsection{2. Does SpiNNaker's router architecture help, or hinder
		fault tolerance?}
		
		SpiNNaker's unconventional use of packet dropping to avoid deadlocks
		greatly simplifies the router architecture, part of the motivation for this
		design. In chapter~\ref{sec:routing} this feature is used to the advantage
		of PGS repair to add fault tolerance to existing routing algorithms.
		Compared with the often complex and wasteful methods used to tolerate
		faults in other networks, PGS repair incurs very little performance
		overhead in the presence of static faults.
		
		Routing table usage does increase in the presence of faults, however, which
		may be a concern for applications which already require many routing table
		entries. Routing table usage, as well as other overheads, were most
		significantly increased in the presence of contiguous groups of network
		faults. This is because the PGS repair algorithm produces routes which pass
		tightly around the corners of faults, resulting in concentrations of
		routing table entries in those areas.  Though the symptoms of this problem
		can be attributed to the design of SpiNNaker's multicast routing mechanism,
		the responsibility lies with the behaviour of the PGS repair algorithm.
		Potential improvements to the PGS repair algorithm are discussed later in
		\S\ref{sec:pgs-repair-improvements}.
		
		The overall answer to this research question, therefore, is that the
		flexibility provided to routing algorithms in SpiNNaker's architecture is
		of great benefit, enabling arbitrary fault patterns to be inexpensively
		avoided.
		
		\subsubsection{3. How can the parts of a neural simulation be placed onto a
		large hexagonal torus topology to reduce network load?}
		
		In chapter~\ref{sec:placement}, I explored a number of contemporary
		approaches to the problem of placing applications with irregular
		communication patterns into network topologies. I observed that researchers
		working on circuit placement for chips and FPGAs are tackling similar
		problems and working at scales as large, or larger than, those faced in
		application placement. Based on this I developed a
		simulated annealing based placement algorithm inspired by the techniques
		used in circuit placement, with specific adaptations for use in application
		placement and SpiNNaker's network topology.
		
		The simulated annealing based placement algorithm consistently outperforms
		pre-existing placement algorithms included in benchmarks in terms of
		placement quality.  In the case of one benchmark, simulated annealing based
		placement made it possible to run that neural simulation in real-time for
		the first time.  At larger scales, simulated annealing was also found to be
		able to produce good quality placements in benchmarks containing over one
		million processes -- the largest size supported by the SpiNNaker
		architecture.
		
		The major shortcoming of simulated annealing based placement is its
		execution speed. Though its execution time grows in proportion to the size
		of the problem, the implementation used took over 12~hours to place a
		synthetic problem for the largest planned SpiNNaker machine. Though
		tractable -- particularly given the relative output quality compared with
		the prior state-of-the-art -- the algorithm is unlikely to function
		comfortably as-is on larger problems.
		
		The conclusion to be drawn from this result, however, is not just that
		simulated annealing is a good solution for today's placement problems but
		that circuit placement techniques in general could be successfully adapted
		to fulfil this role. The placement problems faced by chip designers are
		growing at roughly the same exponential rate as the size of super computers
		but circuit designs hold the lead in terms of problem size. Consequently,
		as approaches are retired by chip placement researchers, they may find new
		life in the field of application placement.
		
	\section{Future research}
		
		Though the goals of this study have largely been met, there also remain
		some important limitations which future work may hope to address.
		Furthermore, this work has uncovered a number of new research areas
		warranting future enquiry. This section outlines a number of future lines
		of research.
		
		\subsection{Warehouse-scale cabling}
			
			In chapter~\ref{sec:building} I developed and implemented a number of
			cabling schemes for the SpiNNaker architecture spanning up to a six metre
			row of machine-room cabinets -- a relatively small installation by
			current standards. In SpiNNaker, the cabling exists in a 2D plane (i.e.
			across the faces of the cabinets) but as the system is scaled up, a
			single row of cabinets will tend towards a 1D line. Since embedding a 2D
			structure in a 1D space necessarily results in long connections, this
			cannot scale indefinitely.
			
			\begin{figure}
				\center
				\buildfig{figures/multi-row-cabling.tex}
				
				\caption{Multiple rows of interconnected cabinets.}
				\label{fig:multi-row-cabling}
			\end{figure}
			
			In conventional large-scale super computer installations, nodes are
			installed in rows of cabinets as illustrated in
			figure~\ref{fig:multi-row-cabling}.  From a `bird's-eye' view, the system
			approximates a 2D space, spread across the floor of a machine-room.
			Therefore, in principle, the folding and interleaving techniques
			described in chapter~\ref{sec:building} still apply. Unfortunately for
			SpiNNaker, cables connecting between rows of cabinets would be longer
			than the one metre limit imposed by its hardware because of the spacing
			between rows of cabinets.  Future SpiNNaker systems will need to consider
			alternative link technologies.  For example, a hybrid system could be
			used in which intra-cabinet connections continue to use the current HSS
			link technology while inter-cabinet links might use optical connections.
			This type of architecture could be supported by the use of pluggable
			`SFP+' transceiver modules~\cite{sff01}.
		
		\subsection{Cabling assistance for other architectures}
			
			A secondary result of the construction of prototype SpiNNaker systems in
			chapter~\ref{sec:building} was the use of real-time guidance and feedback
			to assist cable installation. I am not aware of this technique's use by
			existing architectures and, following the success experienced in this
			project, it is possible that the technique may also be useful in
			conventional systems.
			
			During the construction of prototype SpiNNaker machines, each cable took
			seconds to install compared with the minutes reported for existing
			systems~\cite{mudigonda11}. Part of this increase in efficiency appears
			to result from the immediate identification of mistakes made during
			cabling, saving time-consuming backtracking later on.
			
			In many real-world network installations, units are less densely packed
			than in SpiNNaker and so longer cables are often required. As a
			consequence, cabling errors may become more likely as cabling patterns
			are spread over a larger area making them more difficult to visually
			verify. Like SpiNNaker, conventional networking hardware is often
			equipped with a generous range of indicator LEDs and diagnostic
			facilities which might be used to implement real-time installation
			guidance. Future work could explore the use of this technique in the
			construction of other large-scale networks, such as data centres.
		
		\subsection{Congestion mitigation}
			
			\label{sec:wiggly-board-allocations}
			
			In chapter~\ref{sec:routing} I found that contiguous network faults cause
			hot-spots of congestion and routing table depletion where the PGS repair
			algorithm routed many paths around the edges of faults.  However, it is
			not just faults which can cause contiguous blockages in the network
			topology. In reality, researchers do not always require a full-sized
			SpiNNaker system to perform their experiments so large SpiNNaker systems
			are soft-partitioned on demand into many smaller
			machines~\cite{spalloc16}. To ensure isolation between partitioned
			sub-machines, HSS links between boards in different partitions are
			disabled. Because of SpiNNaker's `wrapped triple' partitioning scheme,
			the resulting sub-machines have hexagonal \emph{mesh} topologies (i.e.
			without wrap-around links) with irregular boundaries as in
			figure~\ref{fig:spalloc-mesh}.
			
			\begin{figure}
				\center
				\buildfig{figures/spalloc-mesh.tex}
				
				\caption[Irregular edges of a partitioned SpiNNaker system.]%
				{Irregular edges in a SpiNNaker system comprised of 24~boards
				partitioned from a larger machine.  Each hexagon represents a SpiNNaker
				chip. No wrap-around connections are present.}
				\label{fig:spalloc-mesh}
			\end{figure}
			
			In partitioned systems, the `tooth'-like gaps on the periphery of the
			network result in similar congestion to the HSS link failures considered
			in chapter~\ref{sec:routing}. When a route is generated between nodes on
			opposite sides of a gap, the PGS repair process will produce a
			shortest-path route around it. Since many routes may be blocked by a
			single gap, a hot-spot may develop around the corners of the gap.
			
			In chapter~\ref{sec:placement}, the `CConv' benchmark application was
			found to run correctly the majority of the time when placed by the
			simulated annealing algorithm but would occasionally fail by a
			significant margin. Preliminary experiments suggest these occasional
			failures are caused by placement solutions which place heavily
			communicating parts of the application on opposite sides of gaps along
			the perimeter of the network. Two possible approaches which future work
			may consider are presented below.
			
			\subsubsection{Avoiding hotspots with PGS repair}
				
				\label{sec:pgs-repair-improvements}	
				
				Network congestion around faults and network irregularities could be
				reduced by forcing the PGS repair process to take more varied routes
				around faults. For example, in circuit routing algorithms, routers
				avoid congestion by increasing the cost of routes which pass through
				congested areas~\cite{kahng11}. A similar technique could be used in
				PGS repair to spread the routes it produces.
				
				An alternative approach would be to adapt the base routing algorithms
				used prior to PGS repair to, for example, attempt alternative dimension
				order routes which may avoid blockages due to faulty links.
			
			\subsubsection{Fault and irregularity aware placement}
				
				One of the shortcomings of the simulated annealing based placer
				developed in chapter~\ref{sec:placement} is that it does not account
				for network faults, or irregularities, when estimating the cost of
				placement solutions.  Future work may exploit techniques used in
				congestion-aware circuit placement which could be adapted for
				application placement~\cite{viswanathan07}.
		
		\subsection{Reducing placement execution time}
			
			The simulated annealing based placer presented in
			chapter~\ref{sec:placement} produced good quality placements but its
			execution time limits its use beyond one million vertex placement
			problems. Future work should explore possibilities for improving the
			performance and scalability of this technique.
			
			In addition to considering alternative placement algorithms based on
			other methods, one possible approach is to attempt to reduce the execution
			time of simulated annealing based placement by shrinking the application
			graph being placed.
			
			For example, graph clustering~\cite{schaeffer07} may be used to group
			together strongly connected vertices which would then be placed as a
			single unit.  Unfortunately, clustering can suffer from the same problems
			as graph-partitioning-based placement: vertices may be grouped together
			in ways which, in practice, cannot be packed together into a given portion
			of a machine.  A possible solution to this problem is to use a two-phase
			placement approach~\cite{kahng11}. In a `global' placement phase,
			solutions are permitted which can slightly over-allocate resources but
			overall achieve good placement quality. In the `detailed' placement phase
			which follows, the solution is `legalised' by making small changes to the
			global placement to eliminate over allocation.
			
			An alternative approach suited to SpiNNaker could be to limit the
			clustering process to clusters which fit on a single SpiNNaker chip. In
			typical SpiNNaker application graphs, clustering to this level may reduce
			placement problem sizes by an order of magnitude and, consequently,
			reduce execution times by the same ratio. Preliminary experiments suggest
			that this approach might result in little placement quality loss for
			large placement problems whilst substantially reducing overall execution
			time.
		
		\subsection{Benchmarking}
			
			One of the most significant limitations of this study has been the
			unavailability of large-scale SpiNNaker applications for use as
			benchmarks. As a consequence, much of the scalability experimentation
			performed has relied on simple synthetic benchmarks based on projections
			of future application behaviour.
			
			In the short term, more sophisticated synthetic benchmark generation
			techniques used by the circuit placement community~\cite{nam07} may offer
			alternative benchmarks for future work. In the longer term, however, it
			is hoped that the availability of large SpiNNaker systems -- and
			placement and routing algorithms better suited to exploit them -- will
			lead to larger scale applications being developed. Hopefully these
			applications will lead to more interesting and representative benchmarks
			for use in future work.
	
	\section{Closing remarks}
		
		One of the primary outcomes of this work is that a number of the practical
		challenges faced in scaling up the SpiNNaker architecture have been
		addressed leading to the construction of large-scale SpiNNaker machines.
		The development of an effective placement algorithm for SpiNNaker
		applications has been shown to enable some neural simulations to exploit
		SpiNNaker's architecture for the first time. The availability of larger
		SpiNNaker machines paves the way for future large-scale neural modelling
		work built on much larger models such as Spaun, `the world's largest
		functional brain model'~\cite{eliasmith12}.
		
		Beyond the SpiNNaker project, the hexagonal torus topology has also been
		validated as a scalable and practical candidate for future network
		architectures. As super computers become ever larger, the physical
		scalability afforded by the 2D nature of the hexagonal torus topology may
		make it a compelling option. In addition, the finding that circuit
		placement techniques can be adapted to support placement of SpiNNaker
		software indicates that these algorithms may also be applicable to other
		applications. Indeed, if this is the case, circuit placement may offer a
		long-term source of placement algorithms able to handle the demands of
		future applications.
		
		% This thesis has explored and tackled a number of the challenges posed in
		% scaling up the unconventional SpiNNaker architecture. Along the way I have
		% demonstrated that the hexagonal torus topology may be a practical choice in
		% future applications which can scale up to the physical dimensions expected
		% of future super computers. I have also developed new efficient and
		% effective methods of placing and routing neural simulation software on
		% SpiNNaker which -- it is hoped -- will enable a new generation of large
		% scale neural simulations on spinnaker.
		
		Although this work stops short of demonstrating truly large-scale
		neuroscientific simulations running at the scale of newly completed
		SpiNNaker machines (largely because such simulations do not yet exist) a
		number of smaller-scale neural simulations have been made possible for the
		first time. The algorithms and techniques devised in this work have
		subsequently been incorporated into various software libraries and tools
		now being used by researchers building SpiNNaker applications, vindicating
		the efforts of this thesis (see appendix~\ref{sec:software}). A successor
		to the SpiNNaker architecture is also in the early stages of design and is
		building on experience of the existing architecture. The current intention
		is to retain the hexagonal torus topology used by SpiNNaker, a decision
		supported by the findings of this thesis.
		
		With SpiNNaker's hardware architecture now operating at scales close to its
		architectural limits, it is hoped that the contributions of this work will
		enable researchers to develop larger and more detailed neural models for
		this unique architecture.

	
	% Bibliography
	\bibliography{references}
	\bibliographystyle{alpha}
	
\end{document}
}\documentclass[12pt,twoside]{report}

\newcommand{\thesistitle}{Building and operating large-scale SpiNNaker machines}
\newcommand{\thesisauthor}{Jonathan Heathcote}
\newcommand{\thesisyear}{2016}

%%%%%%%%%%%%%%%%%%%%%%%%%%%%%%%%%%%%%%%%%%%%%%%%%%%%%%%%%%%%%%%%%%%%%%%%%%%%%%%%
% Used packages
%%%%%%%%%%%%%%%%%%%%%%%%%%%%%%%%%%%%%%%%%%%%%%%%%%%%%%%%%%%%%%%%%%%%%%%%%%%%%%%%

% Nice printing of URLs
\usepackage{url}

% Actually not tear-your-eyes-out-ugly tables
\usepackage{booktabs}

% Adjust linespacing for localised parts of the paper (e.g. abstract)
\usepackage{setspace}

% For the \ifthenelse macro
\usepackage{ifthen}

% For the \degree macro
\usepackage{gensymb}

% For subfigure support
\usepackage{caption}
\usepackage{subcaption}

% SI unit and number formatting
\usepackage{siunitx}

% Used to draw labels with a white outline to make them stand-out in diagrams
\usepackage[outline]{contour}

% TikZ + PGF Plots for diagram/plot drawing
\usepackage{tikz}
\usepackage{tikz3d}
\usepackage{pgfplots}
\usetikzlibrary{ hexagon
               , calc
               , backgrounds
               , positioning
               , decorations.pathreplacing
               , decorations.markings
               , arrows
               , positioning
               , automata
               , shadows
               , fit
               , shapes
               , arrows
               , patterns
               , spy
               }
\usepgfplotslibrary{statistics}

%%%%%%%%%%%%%%%%%%%%%%%%%%%%%%%%%%%%%%%%%%%%%%%%%%%%%%%%%%%%%%%%%%%%%%%%%%%%%%%%
% Environment settings
%%%%%%%%%%%%%%%%%%%%%%%%%%%%%%%%%%%%%%%%%%%%%%%%%%%%%%%%%%%%%%%%%%%%%%%%%%%%%%%%

% 1.5 linespacing (as required by university)
\renewcommand{\baselinestretch}{1.5}


% Specifies the thickness of the contour added by the \contour macro.
\contourlength{1.5pt}

% Define a few layers for TikZ to allow easier layering
\pgfdeclarelayer{bg}
\pgfdeclarelayer{fg}
\pgfsetlayers{bg,main,fg}

%%%%%%%%%%%%%%%%%%%%%%%%%%%%%%%%%%%%%%%%%%%%%%%%%%%%%%%%%%%%%%%%%%%%%%%%%%%%%%%%
% Definitions
%%%%%%%%%%%%%%%%%%%%%%%%%%%%%%%%%%%%%%%%%%%%%%%%%%%%%%%%%%%%%%%%%%%%%%%%%%%%%%%%

% Used in place of \chapter for preface sections. Prevents numbering but
% includes the chapter in the ToC
\newcommand{\prefacesection}[1]{
	\chapter*{#1}
	\addcontentsline{toc}{chapter}{#1}
}

% Adds 'discard if' and  'discard if not' options for \addplot to enable
% filtering of data. Taken from
% http://tex.stackexchange.com/questions/58548/is-it-possible-to-change-the-color-of-a-single-bar-when-the-bar-plot-is-based-on
\pgfplotsset{
    discard if/.style 2 args={
        x filter/.code={
            \edef\tempa{\thisrow{#1}}
            \edef\tempb{#2}
            \ifx\tempa\tempb
                \def\pgfmathresult{inf}
            \fi
        }
    },
    discard if not/.style 2 args={
        x filter/.code={
            \edef\tempa{\thisrow{#1}}
            \edef\tempb{#2}
            \ifx\tempa\tempb
            \else
                \def\pgfmathresult{inf}
            \fi
        }
    }
}

% Make PGFplots Treat "NA" (regardless of letter case) as "nan". From:
% http://tex.stackexchange.com/questions/110441/skip-specific-string-in-a-numeric-column-while-using-pgfplots
\makeatletter
\expandafter\def\csname pgffltA@N\endcsname{\pgfflt@readundef}
\expandafter\def\csname pgffltA@n\endcsname{\pgfflt@readundef}
\def\pgfflt@readundef #1{%
    \def\pgfflt@readnan@ok{1}%
    \if#1a\else\if#1A\else\def\pgfflt@readnan@ok{0}\fi\fi
    \if\pgfflt@readnan@ok1%
        \pgfmathfloat@a@S=3\relax%
        \pgfmathfloat@a@Mtok={0.0}%
        \pgfmathfloat@a@E=0%
        \expandafter\pgfflt@finish
    \else
        \def\pgfflt@readnan@{\pgfflt@error #1}%
        \expandafter\pgfflt@readnan@
    \fi
}
\makeatother

%%%%%%%%%%%%%%%%%%%%%%%%%%%%%%%%%%%%%%%%%%%%%%%%%%%%%%%%%%%%%%%%%%%%%%%%%%%%%%%%
% Document body
%%%%%%%%%%%%%%%%%%%%%%%%%%%%%%%%%%%%%%%%%%%%%%%%%%%%%%%%%%%%%%%%%%%%%%%%%%%%%%%%
\begin{document}
	
	% The title page
	\begin{titlepage}
	
	
	\begin{center}
		
		\vspace*{1.0in}
		
		{\LARGE\textbf{\thesistitle}}
		
		\vfill
		
		\textsc{A thesis submitted to the University of Manchester\\for the degree of Doctor
		of Philosophy\\in the Faculty of Science and Engineering.}
		
		\vfill
		
		\thesisyear
		
		\vfill
		
		\thesisauthor
		
		\vfill
		
		School of Computer Science
		
		\vfill
		
		\color{gray}{
			{\tiny{}Revision \texttt{\input{thesis.fullhash}}\input{thesis.date}}
		}
		
	\end{center}
	
\end{titlepage}

	
	% The table of contents which, per university regulations, is followed by a
	% total wordcount.
	\tableofcontents
	\vfill
	\noindent This thesis contains
		\immediate\write18{texcount -1 -sum -inc thesis.tex > thesis.wordcount}%
		\documentclass[12pt,twoside]{report}

\newcommand{\thesistitle}{Building and operating large-scale SpiNNaker machines}
\newcommand{\thesisauthor}{Jonathan Heathcote}
\newcommand{\thesisyear}{2016}

%%%%%%%%%%%%%%%%%%%%%%%%%%%%%%%%%%%%%%%%%%%%%%%%%%%%%%%%%%%%%%%%%%%%%%%%%%%%%%%%
% Used packages
%%%%%%%%%%%%%%%%%%%%%%%%%%%%%%%%%%%%%%%%%%%%%%%%%%%%%%%%%%%%%%%%%%%%%%%%%%%%%%%%

% Nice printing of URLs
\usepackage{url}

% Actually not tear-your-eyes-out-ugly tables
\usepackage{booktabs}

% Adjust linespacing for localised parts of the paper (e.g. abstract)
\usepackage{setspace}

% For the \ifthenelse macro
\usepackage{ifthen}

% For the \degree macro
\usepackage{gensymb}

% For subfigure support
\usepackage{caption}
\usepackage{subcaption}

% SI unit and number formatting
\usepackage{siunitx}

% Used to draw labels with a white outline to make them stand-out in diagrams
\usepackage[outline]{contour}

% TikZ + PGF Plots for diagram/plot drawing
\usepackage{tikz}
\usepackage{tikz3d}
\usepackage{pgfplots}
\usetikzlibrary{ hexagon
               , calc
               , backgrounds
               , positioning
               , decorations.pathreplacing
               , decorations.markings
               , arrows
               , positioning
               , automata
               , shadows
               , fit
               , shapes
               , arrows
               , patterns
               , spy
               }
\usepgfplotslibrary{statistics}

%%%%%%%%%%%%%%%%%%%%%%%%%%%%%%%%%%%%%%%%%%%%%%%%%%%%%%%%%%%%%%%%%%%%%%%%%%%%%%%%
% Environment settings
%%%%%%%%%%%%%%%%%%%%%%%%%%%%%%%%%%%%%%%%%%%%%%%%%%%%%%%%%%%%%%%%%%%%%%%%%%%%%%%%

% 1.5 linespacing (as required by university)
\renewcommand{\baselinestretch}{1.5}


% Specifies the thickness of the contour added by the \contour macro.
\contourlength{1.5pt}

% Define a few layers for TikZ to allow easier layering
\pgfdeclarelayer{bg}
\pgfdeclarelayer{fg}
\pgfsetlayers{bg,main,fg}

%%%%%%%%%%%%%%%%%%%%%%%%%%%%%%%%%%%%%%%%%%%%%%%%%%%%%%%%%%%%%%%%%%%%%%%%%%%%%%%%
% Definitions
%%%%%%%%%%%%%%%%%%%%%%%%%%%%%%%%%%%%%%%%%%%%%%%%%%%%%%%%%%%%%%%%%%%%%%%%%%%%%%%%

% Used in place of \chapter for preface sections. Prevents numbering but
% includes the chapter in the ToC
\newcommand{\prefacesection}[1]{
	\chapter*{#1}
	\addcontentsline{toc}{chapter}{#1}
}

% Adds 'discard if' and  'discard if not' options for \addplot to enable
% filtering of data. Taken from
% http://tex.stackexchange.com/questions/58548/is-it-possible-to-change-the-color-of-a-single-bar-when-the-bar-plot-is-based-on
\pgfplotsset{
    discard if/.style 2 args={
        x filter/.code={
            \edef\tempa{\thisrow{#1}}
            \edef\tempb{#2}
            \ifx\tempa\tempb
                \def\pgfmathresult{inf}
            \fi
        }
    },
    discard if not/.style 2 args={
        x filter/.code={
            \edef\tempa{\thisrow{#1}}
            \edef\tempb{#2}
            \ifx\tempa\tempb
            \else
                \def\pgfmathresult{inf}
            \fi
        }
    }
}

% Make PGFplots Treat "NA" (regardless of letter case) as "nan". From:
% http://tex.stackexchange.com/questions/110441/skip-specific-string-in-a-numeric-column-while-using-pgfplots
\makeatletter
\expandafter\def\csname pgffltA@N\endcsname{\pgfflt@readundef}
\expandafter\def\csname pgffltA@n\endcsname{\pgfflt@readundef}
\def\pgfflt@readundef #1{%
    \def\pgfflt@readnan@ok{1}%
    \if#1a\else\if#1A\else\def\pgfflt@readnan@ok{0}\fi\fi
    \if\pgfflt@readnan@ok1%
        \pgfmathfloat@a@S=3\relax%
        \pgfmathfloat@a@Mtok={0.0}%
        \pgfmathfloat@a@E=0%
        \expandafter\pgfflt@finish
    \else
        \def\pgfflt@readnan@{\pgfflt@error #1}%
        \expandafter\pgfflt@readnan@
    \fi
}
\makeatother

%%%%%%%%%%%%%%%%%%%%%%%%%%%%%%%%%%%%%%%%%%%%%%%%%%%%%%%%%%%%%%%%%%%%%%%%%%%%%%%%
% Document body
%%%%%%%%%%%%%%%%%%%%%%%%%%%%%%%%%%%%%%%%%%%%%%%%%%%%%%%%%%%%%%%%%%%%%%%%%%%%%%%%
\begin{document}
	
	% The title page
	\input{titlepage}
	
	% The table of contents which, per university regulations, is followed by a
	% total wordcount.
	\tableofcontents
	\vfill
	\noindent This thesis contains
		\immediate\write18{texcount -1 -sum -inc thesis.tex > thesis.wordcount}%
		\input{thesis.wordcount}words.
	
	\clearpage
	\listoffigures
	
	\clearpage
	\listoftables
	
	% Abstract
	\input{abstract}
	
	% Declaration of non-submission elsewhere
	\input{declaration}
	
	% University-prescribed copyright statement...
	\input{copyright}
	
	% Acknowledgements
	\input{acknowledgements}
	
	% Main body
	\input{introduction.tex}
	\input{background.tex}
	\input{building.tex}
	\input{shortestPaths.tex}
	\input{routing.tex}
	\input{placement.tex}
	\input{discussion.tex}
	\input{future.tex}
	\input{conclusions.tex}
	
	% Bibliography
	\bibliography{references}
	\bibliographystyle{alpha}
	
\end{document}
words.
	
	\clearpage
	\listoffigures
	
	\clearpage
	\listoftables
	
	% Abstract
	{
	\prefacesection{Abstract}
	
	% Single line spacing for the abstract page
	\setstretch{1.0}
	
	
	\vfill
	
	% Standard thesis information
	\begin{center}
		\textsc{\large\thesistitle}
		
		\vspace{0.5em}
		
		\thesisauthor
		
		\vspace{0.5em}
		
		A thesis submitted to the University of Manchester\\
		for the degree of Doctor of Philosophy, 2016
	\end{center}
	
	\vfill
	
	% The abstract
	
	SpiNNaker is an unconventional super computer architecture designed to
	simulate up to one billion biologically realistic neurons in real-time. To
	achieve this goal, SpiNNaker employs a novel network architecture which poses
	a number of practical problems in scaling up from desktop prototypes to
	machine room filling installations.
	
	SpiNNaker's hexagonal torus network topology has received mostly theoretical
	treatment in the literature. This thesis tackles some of the challenges
	encountered when building `real-world' systems.  Firstly, a scheme is devised
	for physically laying out hexagonal torus topologies in machine rooms which
	avoids long cables; this is demonstrated on a half-million core SpiNNaker
	prototype.  Secondly, to improve the performance of existing routing
	algorithms, a more efficient process is proposed for finding (logically)
	short paths through hexagonal torus topologies. This is complemented by a
	formula which provides routing algorithms greater flexibility when finding
	paths, potentially resulting in a more balanced network utilisation.
	
	The scale of SpiNNaker's network and the models intended for it also present
	their own challenges. Placement and routing algorithms are developed which
	assign processes to nodes and generate paths through SpiNNaker's network.
	These algorithms reduce congestion and tolerate network faults. The proposed
	placement algorithm is inspired by techniques used in chip design and is
	shown to enable larger applications to run on SpiNNaker -- with good
	performance -- than the previous state-of-the-art. Likewise the routing
	algorithm developed is able to tolerate network faults, inevitably present in
	large scale systems, with little performance overhead.
	
	
	% Required to ensure single line spacing is used for this whole block
	\par%
}

	
	% Declaration of non-submission elsewhere
	\prefacesection{Declaration}

% Single line spacing for the declaration
{
	\setstretch{1.0}
	No portion of the work referred to in this thesis has been submitted in support
	of an application for another degree or qualification of this or any other
	university or other institute of learning.
	
	\par%
}


	
	% University-prescribed copyright statement...
	\input{copyright}
	
	% Acknowledgements
	{
	\prefacesection{Acknowledgements}
	
	% Single line spacing
	\setstretch{1.0}
	
	It is often said that it is not \emph{what} you know but \emph{who} you know.
	Throughout the course of my PhD I've been exceptionally lucky to have been
	helped along by a great number of people.
	
	Both my supervisor, Jim Garside, and co-supervisor, Steve Furber, have each
	spent countless hours patiently discussing and describing all manner of
	things with me while giving me great freedom in my project. Jim's office door
	has always been open to my unexpected interruptions be it about work, writing
	or walking.  Likewise, Steve has always managed to find time for both
	technical and frivolous endeavours alike. I'm also hugely grateful to Luis
	Plana who has been a rich source of sage advice, insightful questions
	patiently suffered many a foolish question.
	
	Various parts of the work in this thesis (and numerous side projects) would
	not have been possible if not for the multitude of discussions,
	collaborations and even sheer physical hard work of Steve Temple, Javier
	Navaridas, Simon Davidson and Dave Clark. I'm also indebted to Andrew Mundy
	and Jamie Knight, both of whom have donated so much time and effort towards
	verifying and using software implementations of the ideas in this thesis.
	
	The injection of lunchtime silliness by Andrew and Jamie, along with Amanieu
	d'Antras and Andrew Webb and the other CDT members has always brightened my
	day. So to has the friendly and stimulating environment of the School of
	Computer Science and its many staff and students. Of course, I am also very
	grateful for the funding the school has provided for my research.
	
	I cannot thank my wonderful wife, Ann-Marie, enough for being by my side. She
	has given me so much kindness, love and patience and endured a lifetime's
	quota of conversations about hexagons. Finally, thanks too to rest of my
	family, especially my parents, who are to blame for starting me down this
	path and co-suffering drafts and endless rants about this document.
	
	% Required to ensure single line spacing is used for this whole block
	\par%
}

	
	% Main body
	\chapter{Introduction}

\label{sec:introduction}

%Problem area
%
%* Network construction and exploitation
%  * Cabling: Build it cheaply in terms of tech cost and install cost
%  * Routing: Get around it cheaply and reliably
%  * Placement: Use it efficiently

The Spiking Neural Network Architecture (SpiNNaker) is a novel super computer
architecture designed to simulate biologically realistic models of brains in
real time \cite{furber07}. Though neurons, the building blocks of the brain,
are relatively well understood, their complex interactions remain mysterious.
Just as understanding the workings of a transistor is insufficient to
understand a modern microprocessor, neuroscientists believe that understanding
the neurons in isolation cannot explain the brain and that understanding their
connectivity is key \cite{eliasmith13,eliasmith14}. Experiments on real brains,
however, are fraught with difficulty. Variations between individuals can be
significant and it is only possible to record tens or hundreds of the trillions
of signals in the brain, and even then only with limited control over which
signals are recorded. Computer simulations of models of large neural networks,
however, enable researchers to develop repeatable experiments and gain complete
visibility of any signal and any neuron. Models such as SPAUN
\cite{eliasmith12}, built from millions of simulated neurons, have shown great
promise in expanding our understanding of higher level brain functions such as
memory and simple problem solving.  Unfortunately these neural models are
expensive to simulate, requiring hours of compute time to simulate each second
of neural activity. As well as being inconvenient, this precludes the use of
robotics to immerse these models in real world environments and also limits
studies of long-term behaviours such as learning.

SpiNNaker is designed to enable the real time simulation of models containing
up to one billion neurons -- approximately \SI{1}{\percent} of a human brain or
ten mouse brains \cite{furber06}. To achieve this goal, the largest planned
SpiNNaker machine will contain over one million low-powered computer processors
interconnected by a bespoke network architecture.

SpiNNaker's large processor count matches the current trend in super computers
where processor counts are growing exponentially \cite{meuer16j}, mimicking the
growth of the number of components in the processors themselves predicted by
Gordon Moore's famous `law' \cite{moore75}. As a result of this growth, the
interconnection networks which enable these processors to work together have
grown in importance \cite{dally04}.  Network designers must carefully balance
performance against practicality and financial cost.  SpiNNaker's network is no
exception to this rule and, as the systems scale up from desktop prototypes to
machine-room scale installations, the reality of building and exploiting these
machines presents an array of challenges.

As in all super computers, SpiNNaker's network interconnects its processors in
a particular network topology which defines how different processors may
communicate with each other. Unlike the tree and $N$-dimensional torus
topologies found in contemporary super computers \cite{dally04}, SpiNNaker
employs a `hexagonal torus topology'. In this topology, nodes in SpiNNaker's
network fit together in a honeycomb-like pattern where messages may `hop' from
node to node to reach their destination. As we will see in
chapter~\ref{sec:background}, the hexagonal torus topology, in theory, sits at
a `sweet spot' in terms of network performance and practicality. As the first
known large-scale installation of the hexagonal torus topology, however, there
remain a number of practical challenges for large spinnaker machines arising
from this choice.

As super computer networks have grown in scale to millions of processors the
task of dealing with previously rare faults has grown.  Though fault rates in
networks remain consistently low, architectures such as SpiNNaker may have
hundreds of thousands of links meaning even fault rates of a fraction of a
percent will impact tens or hundreds of links. To enable reliable operation,
networks must be able to adapt the routes taken by messages through the network
to avoid faulty links and nodes. The techniques employed are often closely tied
to a particular network architecture and consequently SpiNNaker's novel network
architecture demands its own approach.

Another challenge introduced by the growing scale of super computers is making
\emph{efficient} use of network resources. Communicating processes should be
located on logically `nearby' nodes to reduce network load. The neural models
for which SpiNNaker is designed are often described abstractly, rather than
geometrically, using modelling languages such as PyNN~\cite{davison08} and
Nengo~\cite{eliasmith04}.  Because of this, the communication requirements of
simulations can be highly irregular making an efficient placement of processes
onto processors in the machine non-trivial.

%Contributions
%
%* Cabling scheme for hexagonal toruses without long cables
%* Efficient installation technique for dense systems
%* Exhaustive and efficient route calculation in hex toruses
%* Fault tolerant routing scheme exploiting SpiNNaker's odd router
%* Placement based on SA a: works very well and b: suggests circuit placement is
%  a good source of inspiration.

This thesis addresses the practical challenges of scaling up the SpiNNaker
architecture in a real-world setting summarised by these research questions:

\begin{enumerate}
	
	\item Can the hexagonal torus topology be deployed and used in real, large
	scale systems?
	
	\item Does SpiNNaker's router architecture help, or hinder fault tolerance?
	
	\item How can the parts of a neural simulation be placed onto a large
	hexagonal torus topology to reduce network load?
	
\end{enumerate}

%Structure
%
%* Chapter 2: Background: detailed dive into what's in SpiNNaker, why its
%  really so unusual. Also looks at what applications run on SpiNNaker and how
%  they work.
%* Chapter 3: How to build a really big SpiNNaker machine.
%* Chapter 4: How to find your way around that machine.
%* Chapter 5: How to find your way around that machine even when its broken.
%* Chapter 6: Now you can walk, time to run.
%* Chapter 7: Wrapping up.
%* Appendices: Hard-to-come-by theoretical and practical details useful if
%  you're about to continue where this research left off but be useful but
%  otherwise hard to come by, especially in one place.

Chapter~\ref{sec:background} introduces the SpiNNaker architecture and, in
particular, describes its hexagonal torus topology and network architecture.

In chapter~\ref{sec:building}, I develop a cabling scheme for large hexagonal
torus topologies which enables arbitrarily large networks to be constructed
using only short, inexpensive cables. This theoretical work is then evaluated
through the construction of a range of prototype SpiNNaker systems. The largest
of these prototypes contains over half a million processor cores and spans
several machine room cabinets. In addition, I propose the use of built-in
diagnostic facilities to assist technicians performing network installation and
maintenance. This technique is found to greatly reduce the effort required and
the number of mistakes made.

In chapters~\ref{sec:shortestPaths}~and~\ref{sec:routing} I develop new routing
techniques for SpiNNaker's network. Chapter~\ref{sec:shortestPaths} develops a
new approach to finding the shortest paths through hexagonal torus topologies,
an integral part of many routing algorithms. This newly proposed approach is
cheaper to compute than the state of the art and, unlike previous efforts, is
able to discover all valid short paths through the topology. This theoretical
advance brings hexagonal torus topologies in line with conventional topologies
by providing routing algorithms with complete information about the paths
available to them. In chapter \ref{sec:routing} I propose a fault tolerant
routing algorithm for SpiNNaker which is able to avoid arbitrary static fault
patterns with minimal performance overhead. A key finding of this chapter is
that the flexibility afforded to fault tolerant routing algorithms by
SpiNNaker's unconventional router architecture is what facilities the low
overheads reported in this chapter.

Finally, in chapter~\ref{sec:placement}, I explore the problem of application
placement in SpiNNaker's network. As in other networks and applications, neural
simulations should be arranged such that communication occurs primarily between
processors close together in the network to control network load. Due to the
irregular connectivity and large scale of the neural models expected to run on
SpiNNaker, an automated approach is necessary. I develop a novel placement
algorithm based on algorithms used for circuit layout in computer chips. My
algorithm is found to allow some larger neural models to run on SpiNNaker for
the first time while enabling other applications to run at greater speeds. In
addition, synthetic benchmarks containing over one million processes indicate
that this algorithm should handle the anticipated demands of the neural models
expected to run on large-scale SpiNNaker installations.

	\chapter{The SpiNNaker Architecture}
	
	\label{sec:background}
	
	SpiNNaker is a massively parallel computer architecture designed to simulate
	biologically realistic neural models \cite{furber07}. In this chapter we will
	explore this unconventional architecture in detail, starting with its purpose
	before focusing on its most unconventional feature: its network.
	
	% * Purpose
	%   * Spiking neural simulations
	%     * Neural modelling: PyNN, Nengo...
	%     * Parallelisation + communication
	
	\section{Neural simulation}
		
		Human brains contain billions of neurons connected together by trillions of
		`synapses'. Neurons communicate by transmitting and receiving `spikes'
		through their synapses. Each spike is `valueless' in that a spike's only
		significant features are when it arrives and where it has come from.
		
		\begin{figure}
			\center
			\buildfig{figures/lif-neuron.tex}
			
			\caption{A Leaky Integrate-and-Fire (LIF) neuron.}
			\label{fig:lif-neuron}
		\end{figure}
		
		Though some detailed models of the electrochemical processes occurring
		inside neurons are computationally intensive, simplified models such as the
		Leaky Integrate-and-Fire (LIF) model can be implemented in just a handful
		of CPU instructions \cite{vainbrand11}. Figure~\ref{fig:lif-neuron}
		illustrates a simple LIF neuron in which incoming spikes cause charge to
		build up (integrated) which over time, leaks away. If an incoming spike
		causes the charge to rise above a certain threshold, the neuron `fires'
		producing an outgoing spike. Despite the simplicity of this model, large
		neural networks such as Spaun \cite{eliasmith12} -- built entirely from LIF
		neurons -- exhibit complex behaviours such as fine motor control and
		problem solving.
		
		The computational expense of large scale neural simulations does not arise
		from the cost of modelling neurons but instead from distributing spikes. In
		biology, neurons produce spikes at an average rate of \SI{10}{\hertz} and
		synapses connect each neuron's output to (order) \num{1000}~neurons
		\cite{navaridas09}. Consider an example neural model with $7\times10^7$
		neurons, approximately the number in a house mouse and
		$\nicefrac{1}{10}^\textrm{th}$ of the design target of SpiNNaker. This
		network might produce $7\times10^8$~spikes per second. Because each neuron
		connects to many others, this equates to $7\times10^{11}$ spikes being
		received per second. If each spike were transmitted as a UDP datagram
		containing a single \SI{32}{\bit} payload, the total network throughput
		required for this simulation would be \SI{179.2}{\tera\bit\per\second}. At
		the time of writing, this is more than double the bisection bandwidth (the
		theoretical worst-case throughput) of the world's most powerful super
		computer \cite{dongarra16}.
	
	\section{Network architecture}
		
		Architectures such as IBM's Blue Gene \cite{chiu11} and Cray's XK7
		\cite{ornl16} employ powerful compute nodes connected together using
		networks designed to transfer multi-kilobyte blocks of data between nodes.
		Since neural models have relatively light computational requirements and
		communications are based on small pieces of data (spikes), this type of
		architecture is poorly suited to the task.
		
		SpiNNaker's architectural target is to support realtime simulations of up
		to one billion neurons. Since neural models such as LIF are inexpensive to
		model and many neurons can be simulated independently in parallel,
		SpiNNaker employs many small, energy efficient ARM processors
		\cite{furber07}. To support the unusual communication requirements of
		neural simulations, a bespoke interconnection network is used which is the
		background to this thesis.
		
	%   * SpiNNaker chip
	%     * Cores
	%     * SDRAM
	%     * NoC
	%     * Router
		
		\begin{figure}
			\center
			%\includegraphics[width=19mm]{figures/spinnakerChip.jpg}
			\buildfig{figures/hex-chips.tex}
			
			\caption[SpiNNaker chips connected to their six neighbours.]%
			{SpiNNaker chips (actual size) connected to their six neighbours.}
			\label{fig:spinnakerChip}
		\end{figure}
		
		The fundamental building block of the SpiNNaker architecture is the
		SpiNNaker chip (figure \ref{fig:spinnakerChip}) \cite{furber13}. Each chip
		contains eighteen low power ARM 968 processor cores each capable of
		simulating between \num{200} and \num{2000} LIF neurons in real time
		\cite{mundy15}.  Each core has a total of \SI{96}{\kilo\byte} of private
		Tightly-Coupled Memory (TCM) and shares access to \SI{128}{\mega\byte} of
		on-chip SDRAM with other cores on the same chip. Finally, each chip
		contains a programmable router which routes network packets to and from the
		local cores and six neighbouring SpiNNaker chips. SpiNNaker machines are
		constructed by combining many SpiNNaker chips.
		
		\begin{figure}
			\center
			\buildfig{figures/spinnaker-packet.tex}
			
			\caption{SpiNNaker's \SI{40}{\bit} and \SI{72}{\bit} multicast packet
			format.}
			\label{fig:spinnaker-packet}
		\end{figure}
		
		Processor cores can communicate by sending and receiving network packets
		forwarded by routers through the network. Since SpiNNaker's network is
		designed to transmit neural spike events efficiently, individual network
		packets are small, either \SI{40}{\bit} or \SI{72}{\bit} compared with tens
		or hundreds of byte packets in typical network architectures.
		
		In a real-time simulation, the time at which a spike is produced is
		implicitly indicated by the time it is received -- since at biological
		timescales a computer network delivers packets `instantaneously'.
		Consequently, the only information which must be explicitly encoded is the
		identity of the neuron which produced the spike. In SpiNNaker, a spike may
		be encoded by using a single \SI{40}{\bit} `multicast packet' whose format
		is illustrated in figure~\ref{fig:spinnaker-packet}.  The \SI{8}{\bit}
		header is used by SpiNNaker's routers to determine the type of packet and
		the \SI{32}{\bit} `routing key' is used to identify the neuron which
		produced the packet. The routing key is also used by SpiNNaker's routers to
		determine how the packet should be directed through the network.
		
		The optional \SI{32}{\bit} payload is not used by conventional spiking
		neural simulations \cite{galluppi10} but has been exploited to enable more
		efficient simulation of a particular class of neural models \cite{mundy15}.
	
	\section{The SpiNNaker router}
		
		The SpiNNaker router employs an unconventional design which, despite its
		compact size and small energy requirements, implements a flexible multicast
		routing scheme. Unlike conventional routers which often employ hard-coded
		routing rules \cite[chapter~8]{dally04}, the SpiNNaker router uses a
		programmable `routing table' to determine how packets should be forwarded.
		In addition, to avoid deadlocks, SpiNNaker's router employs a simple,
		timeout-based mechanism which exploits the ability of neural networks to
		tolerate occasional missing packets. As we will see in chapter
		\ref{sec:routing}, this mechanism greatly simplifies the task of routing in
		SpiNNaker's network. In this section we'll look at these features in
		greater detail.
		
		\subsection{Routing tables}
		
			When a multicast packet arrives at a SpiNNaker router (either from a
			local core or a neighbouring chip), the router looks up the routing key
			in its routing table. This table consists of \num{1024} programmable
			table entries, each specifying a routing key bit pattern and mask to
			match and a set of routes.  When a multicast packet's key is matched by a
			routing entry the packet is forwarded along every route specified by that
			entry, potentially duplicating the packet. This `multicast' technique
			allows packets to be transmitted once but received in a number of places
			while making efficient use of the network \cite{navaridas12}.
			
			Though routing table entries are in finite supply (\num{1024} entries per
			router), it is still possible for many thousands of traffic flows to be
			routed through a single router. The bit pattern and mask in each routing
			entry matches against the 32~bits of a routing key as either
			`\texttt{1}', `\texttt{0}' or `\texttt{X}' (don't care).  This means that
			a single routing entry may, for example, be used to match all routing
			keys with a certain prefix. If a routing key is not matched by any entry
			in the routing table then the packet is `default routed' in a straight
			line. For example if a packet with an unmatched key is received from the
			chip to the left, the packet will be default routed to the chip on the
			right. By assigning routing keys such that neurons whose spikes are sent
			to similar destinations share a similar prefix, the number of routing
			entries required by a simulation is greatly reduced \cite{davies12}.
			
			\begin{figure}
				\center
				\buildfig{figures/routing-example.tex}
				
				\caption[Multicast routing example.]%
				{Multicast routing example with \SI{4}{\bit} routing keys. Each
				box represents a SpiNNaker chip whose router has been programmed with
				the routing entries shown. Grey lines mark connections between chips.}
				\label{fig:routing-example}
			\end{figure}
			
			Consider the simplified example in figure~\ref{fig:routing-example} in
			which a number of (\SI{4}{\bit}) routing table entries have been
			configured in the routers of a small SpiNNaker network. If a packet with
			the routing key \texttt{1011} is transmitted by a core in the chip
			labelled $(0, 0, 0)$, this will match the first routing table entry on
			that chip and will be routed to chip $(1, 0, 0)$. On chip $(1, 0, 0)$,
			the packet once again matches the first routing entry and is routed to
			chip $(1, 0, -1)$. On $(1, 0, -1)$, no match is made so the packet is
			default routed to $(1, 0, -2)$. On this chip, the packet matches a
			routing entry which routes the packet to core~7. In this example, default
			routing allows only three routing table entries to direct a packet
			through four chips.
			
			As a second example, if a packet with the routing key \texttt{0010} is
			transmitted by a core on chip $(0, 0, 0)$, this key will be matched by
			the second routing entry since \texttt{X}s in the table entry will match
			both \texttt{1}s and \texttt{0}s in the corresponding bits of the routing
			key. When the packet arrives at chip $(0, 0, -1)$ the matching routing
			entry forwards the packet to both $(0, 1, -1)$ and $(1, 0, -1)$
			simultaneously. The copy of the packet arriving at $(0, 1, -1)$ is routed
			to core~5 on that chip.  Meanwhile, the copy forwarded to $(1, 0, -1)$ is
			duplicated again with one copy being routed to core~11 and another being
			routed to chip $(1, 0, -2)$. Here the packet is finally delivered to
			core~6. In this example, the ability of the router to multicast
			(duplicate) packets as they pass through the network meant that sending
			one copy of the packet was sufficient to reach three destination cores.
			In addition, by using \texttt{X}s in the routing table entry, the same
			routing entries are sufficient to route packets with the keys
			\texttt{0000}, \texttt{0001}, \texttt{0010} and \texttt{0011}.
			
			In spite of these mechanisms, it is still possible for an application to
			run out of routing table entries. As we will see in
			chapter~\ref{sec:placement} by arranging applications appropriately
			within SpiNNaker's network, routing table usage can be reduced. In
			addition, other behaviours of SpiNNaker's router may be exploited to
			compress an applications routing tables further, however the techniques
			employed are beyond the scope of this thesis \cite{mundy16}.
		
		\subsection{Timeouts}
			
			SpiNNaker's router is built on a pipeline architecture. As shown in
			figure~\ref{fig:router-architecture}, the router is fed packets by an
			arbiter which serialises packets arriving from other chips and local
			cores. Every (\SI{100}{\mega\hertz}) clock cycle, the router pipeline
			accepts one packet from the arbiter and routes a packet to one or several
			output links. If any of the required output ports are busy then the
			packet is not forwarded to any output link and the pipeline stalls. Once
			a packet has been blocked for a programmable timeout, it is dropped
			(discarded) and routing continues as usual for next packet in the
			pipeline. Links become blocked while transmitting packets or waiting for
			the remote receiver to become ready. For example, a receiving processor
			core may be busy performing some computation or a receiving router may be
			blocked waiting for some of its outputs to become ready.
			
			\begin{figure}
				\center
				\buildfig{figures/router-architecture.tex}
				
				\caption{SpiNNaker router architecture}
				\label{fig:router-architecture}
			\end{figure}
			
			The timeout-based packet dropping mechanism is designed to defuse
			deadlocks in the network. For example, if two routers are trying to send
			each other a packet at the same time they may become deadlocked, each
			waiting for the other router to accept a packet before continuing.
			SpiNNaker's timeout mechanism breaks deadlocks by dropping packets which
			have been blocked for some time and therefore may be in a deadlock.  Once
			a packet has been dropped it is left to software to either tolerate the
			missing packet or trigger a retransmission. In neural simulations, as in
			biology, the loss of a single spike is unlikely to have a significant
			impact on the behaviour of a neural model and therefore these simulations
			are inherently tolerant of occasional dropped packets. During application
			loading and other system tasks, a higher level, software driven protocol
			based on acknowledgements and retransmissions is used to ensure
			guaranteed delivery.
			
			% TODO: MENTION TIMEOUT VALUE USED?
			% Router timeouts must be configured to be long enough that delays in
			% packet transmission, for example due to the time taken for packets to
			% traverse a link, do not trigger packet dropping. Conversely, the timeout
			% should be as short as possible to reduce the time the router is
			% blocked and maximise network throughput.
	
	\section{The hexagonal torus topology}
		
		Each SpiNNaker chip is a node in a `hexagonal torus topology' as
		illustrated in figure~\ref{fig:hexagonalTorusTopology}. Network packets
		sent by SpiNNaker's processor cores may `hop' through several nodes in the
		network to reach their intended destination. In each hop, a packet may
		advance one node along one of the three axes of the topology. For example,
		a packet sent by the node labelled $\alpha$ (in the bottom-left corner) to
		the node labelled $\beta$, might take the following sequence of hops:
		X$^+$, X$^+$, Z$^-$. Packets sent from $\alpha$ to $\gamma$ might take the
		route: X$^-$, X$^-$, Y$^+$, Y$^+$. The first hop of this route `wraps
		around' from the bottom-left node to the bottom-right node in a single hop.
		
		\begin{figure}
			\center
			\buildfig{figures/hexagonalTorusTopology.tex}
			
			\caption[A hexagonal torus topology.]%
			{A hexagonal torus topology. Each hexagon represents a node (i.e.
			a SpiNNaker chip). Touching nodes are directly connected. Nodes on edges
			$a$, $b$ and $c$ are also directly connected to the corresponding nodes
			on edges $a'$, $b'$ and $c'$, respectively. The three axes of the
			hexagonal torus topology, `X', `Y' and `Z' are also shown.}
			\label{fig:hexagonalTorusTopology}
		\end{figure}
		
		\begin{figure}
			\center
			\begin{subfigure}{0.39\linewidth}
				\center
				\includegraphics[width=\linewidth]{figures/torus-3d-flat.pdf}
				\caption{}
				\label{fig:torus-3d-flat}
			\end{subfigure}
			~~
			\begin{subfigure}{0.26\linewidth}
				\center
				\includegraphics[width=\linewidth]{figures/torus-3d-tube.pdf}
				\caption{}
				\label{fig:torus-3d-tube}
			\end{subfigure}
			~~
			\begin{subfigure}{0.23\linewidth}
				\center
				\includegraphics[width=\linewidth]{figures/torus-3d-torus.pdf}
				\caption{}
				\label{fig:torus-3d-torus}
			\end{subfigure}
			
			\caption{Visualisation of a hexagonal torus topology as a torus.}
			\label{fig:torus-3d}
		\end{figure}
		
		The wrap around connections in the topology are what give it the `torus'
		part of its name. Figure~\ref{fig:torus-3d-flat} shows a hexagonal torus
		topology drawn flat as in the previous figure. If the topology is rolled up
		into a tube such that the top and bottom nodes become directly adjacent, a
		tube is formed as in figure~\ref{fig:torus-3d-tube}. This tube can then be
		bent to bring together the nodes at the ends of the tube to form a torus as
		shown in figure~\ref{fig:torus-3d-torus}.
		
		A hexagonal torus topology is typically defined in terms of its width and
		height along the X and Y axes respectively. For example,
		figure~\ref{fig:hexagonalTorusTopology} shows a $10\times10$ hexagonal
		torus.  The nodes in a hexagonal torus topology are addressed using
		hexagonal coordinates of the form $(x, y, z)$ \cite{patel15}. The bottom
		left node (labelled $\alpha$ in the figure) has the coordinate $(0, 0, 0)$
		and other nodes are assigned coordinates according to the number of hops
		along each dimension from $(0, 0, 0)$, for example node $\beta$ has the
		coordinate $(2, 0, -1)$. Because the hexagonal torus topology's axes are
		non-orthogonal, it is possible to define several coordinates for the same
		location. For example $(3, 1, 0)$ and $(1, -1, -2)$ are also valid
		coordinates for node $\beta$. These dual coordinates emerge from the fact
		that adding $(1, 1, 1)$ to a coordinate produces an equivalent, but
		different, coordinate. This phenomenon is explained in detail in
		appendix~\ref{app:minimal-hex-coordinates} and related phenomena will be
		discussed in chapter~\ref{sec:shortestPaths}.
		
		The hexagonal torus topology was chosen over a more conventional network
		topology -- such as a 2D or 3D torus (sometimes known as a 2-ary $N$-cube
		or 3-ary $N$-cube respectively) \cite[chapters~3~and~5]{dally04} -- due to
		its balance of theoretical performance and practicality. The bisection
		bandwidth of a topology indicates the theoretical worst-case total
		throughput the network is able to sustain \cite[chapter~1]{dally04}.  In
		networks with homogeneous link throughput, bisection bandwidth is
		determined by the number of links cut by a balanced bisection of the
		network.  Figure~\ref{fig:bisection-bandwidth} illustrates the bisections
		of several torus topologies.
		
		\begin{figure}
			\center
			\begin{subfigure}[b]{0.3\linewidth}
				\center
				\buildfig{figures/bisection-bandwidth-2d.tex}
				
				\caption{2D Torus}
				\label{fig:bisection-bandwidth-2d}
			\end{subfigure}
			\begin{subfigure}[b]{0.3\linewidth}
				\center
				\buildfig{figures/bisection-bandwidth-hex.tex}
				
				\caption{Hexagonal Torus}
				\label{fig:bisection-bandwidth-hex}
			\end{subfigure}
			\begin{subfigure}[b]{0.3\linewidth}
				\center
				\buildfig{figures/bisection-bandwidth-3d.tex}
				
				\caption{3D Torus}
				\label{fig:bisection-bandwidth-3d}
			\end{subfigure}
			
			\caption[Bisections of torus topologies.]%
			{Bisections of torus topologies. Connections cut by the bisection
			are drawn as lines.}
			\label{fig:bisection-bandwidth}
		\end{figure}
		
		In a $N \times N$ 2D torus topology, the bisection bandwidth is $2N$~links
		and each node requires four links. The hexagonal torus topology requires
		six links per node but provides double bisection bandwidth ($4N$~links).
		The 3D torus topology also requires six links per node but by connecting
		the nodes differently achieves a bisection bandwidth of $8N$~links.  The 3D
		torus topology, however, comes at a price -- when immersed into the
		(approximately) 2D space provided by a large machine room or row of server
		cabinets, some connections require long cables. By contrast, the 2D and
		hexagonal torus topologies are both inherently two dimensional and
		consequently do not suffer from this effect. The hexagonal torus topology,
		therefore, shares the practicality of construction of a 2D torus while
		still gaining some of the performance of a 3D torus topology. In addition,
		because nodes in a hexagonal torus topology have a greater number of links,
		greater redundancy is available in the network to tolerate faults.
		
		Most torus topologies, including hexagonal, 2D and 3D toruses, have a
		related `mesh' topology. These mesh topologies maintain the same general
		connectivity structure as their torus topologies but omit wrap-around
		links. In practice, this saves a small number of links at the expense of
		halving the network's bisection bandwidth.  Because of their poorer
		performance, mesh networks are rarely used as the basis of a network
		architecture. Mesh networks, however, are occasionally formed when a
		network is partitioned into several smaller sub-networks to allow multiple
		users to share a system \cite{spalloc16}.
		
		\begin{figure}
			\center
			\begin{subfigure}[b]{0.45\linewidth}
				\center
				\buildfig{figures/hexagonal-torus.tex}
				\caption{Hexagonal torus}
				\label{fig:topo-compare-hexagonal-torus}
			\end{subfigure}
			\begin{subfigure}[b]{0.45\linewidth}
				\center
				\buildfig{figures/h-torus.tex}
				\caption{H-torus}
				\label{fig:topo-compare-h-torus}
			\end{subfigure}
			
			\caption[Hexagonal torus vs. H-torus topology.]%
			{Hexagonal torus vs. H-torus topology. Each numbered hexagon
			represents a node. The thick outline indicates the bounds of the
			topology after which the network repeats. In each topology, the path
			taken by advancing in the Y$^+$ direction from the node labelled `0' is
			shown.}
			\label{fig:topo-compare}
		\end{figure}
		
		\label{sec:hex-vs-h-torus}
		
		The hexagonal torus topology is not to be confused with the `H-torus'
		topology. This topology also uses a hexagonal tiling of nodes and even
		wraps this tiling into a torus-like topology \cite{zhao08}. However,
		H-torus topologies have very different characteristics to the hexagonal
		torus topology and are related to `twisted torus' topologies
		\cite{camara10}. For example, figure~\ref{fig:topo-compare} illustrates one
		major difference in the way paths wrap around the peripheries of both
		topologies.
	
	\section{Scaling-up SpiNNaker machines}
		
		To build large SpiNNaker systems comprising of tens of thousands of
		SpiNNaker chips, groups of forty-eight chips are mounted onto circuit
		boards as illustrated in figure~\ref{fig:spinnakerBoard}. These boards may
		be connected together to form larger systems.  Figure~\ref{fig:threeboard}
		shows a prototype three board system. Though the chips are
		\emph{physically} arranged in a (nearly) $7\times7$ grid on each SpiNNaker
		board, they logically form a hexagonal `wrapped triple'
		\cite{davidsonWiring} (see appendix~\ref{sec:partitioning}) which logically
		fit together as illustrated in figure~\ref{fig:threeboard-separate}. The
		labelled exposed corners of the three forty-eight chip boards connect
		together to form a $12\times12$ hexagonal torus topology as illustrated in
		figure~\ref{fig:threeboard-wrapped}. Larger SpiNNaker machines are
		assembled by combining more boards.
		
		\begin{figure}
			\center
			\begin{subfigure}[b]{0.45\linewidth}
				\center
				\includegraphics[width=\linewidth]{figures/spinnakerBoard.jpg}
				
				\caption{A SpiNNaker board}
				\label{fig:spinnakerBoard}
			\end{subfigure}
			~~~
			\begin{subfigure}[b]{0.45\linewidth}
				\center
				\includegraphics[width=\linewidth]{figures/threeboard.jpg}
				
				\caption{Three board prototype}
				\label{fig:threeboard}
			\end{subfigure}
			
			\vspace*{1em}
			
			\begin{subfigure}[b]{0.45\linewidth}
				\center
				\buildfig{figures/threeboard-separate.tex}
				
				\caption{Three board topology}
				\label{fig:threeboard-separate}
			\end{subfigure}
			~~~
			\begin{subfigure}[b]{0.45\linewidth}
				\center
				\buildfig{figures/threeboard-wrapped.tex}
				
				\caption{\ldots{}as a parallelogram}
				\label{fig:threeboard-wrapped}
			\end{subfigure}
			
			\caption{SpiNNaker boards and their topology.}
			\label{fig:spinnaker-boards}
		\end{figure}
		
		
		SpiNNaker chips on the same circuit board connect using low power links
		requiring sixteen wires each.  If this link technology were used to connect
		chips on neighbouring boards, each pair of boards would need to be
		connected with a 128~wire cable.  Cables and connectors supporting this
		many signals are expensive, unreliable and physically large. Instead,
		chip-to-chip connections between boards are multiplexed and demultiplexed
		onto a single High-Speed Serial (HSS) link \cite{athavale05} carried via
		commodity S-ATA cables which are often used to connect hard disks in
		desktop computers and servers \cite{sata3spec}. The six high-speed links
		are implemented by three onboard FPGAs (the three large chips at the top of
		the SpiNNaker board) and are logically transparent to the underlying
		network. The underlying technology and the choice of S-ATA cables limits
		each board-to-board connection to spanning at most one metre gaps. In
		chapter~\ref{sec:building} I present a cabling scheme for hexagonal torus
		topologies which enable large SpiNNaker systems to be assembled using only
		short cables between boards.
		
	\section{Conclusions}
		
		The SpiNNaker architecture has been designed to enable the simulation of
		large biologically realistic neural models in real time. To support this,
		its network architecture takes on an unconventional design based on a
		custom router and hexagonal torus topology. In the remainder of this
		thesis, I will tackle a number of the challenges in scaling up the
		SpiNNaker architecture outlined in this chapter.

	\chapter{Building large SpiNNaker machines}
	
	Like any super computer, physically putting together a large SpiNNaker
	machine poses many challenges in terms of organisation, assembly and
	maintainance. One of the key tasks in this process is the installation of
	network cables such that a desired overall network topology is constructed.
	The largest planned SpiNNaker machine will use \num{3600} S-ATA
	\cite{sata3spec} cables to interconnect its \num{1200} circuit boards,
	creating a hexagonal torus topology. Since the machine will be installed
	within standard server room cabinets (which are not available in a
	giant-doughnut form-factor) a mapping from a board's logical location in the
	network topology to its physical location must be constructed. In addition,
	the interconnect technology employed by SpiNNaker restricts the length of
	S-ATA cables used to $\le$~\SI{1}{\meter}, constraining the possible mappings
	used. In addition the practical issues of installation complexity and
	maintainance must be considered since all \num{3600} cables must ultimately
	be installed and maintained by human operators.
	
	In this chapter I describe a novel technique for physically laying out
	machines configured in hexagonal torus topologies, such as SpiNNaker, in
	commercial machine rooms, building on the techniques used in more
	conventional torus topologies. In addition, I also propose a new methodology
	for installing and maintaining super computer cabling which which exploits
	existing diagnostic features of the SpiNNaker hardware to interactively guide
	and validate cable installation. Finally, I demonstrate how these new
	techniques have been used successfully to interconnect a prototype
	\num{518400} core SpiNNaker machine in substantially less time than the
	industry norm.
	
	In this chapter, the term \emph{unit} refers to the smallest physical
	ecomponent between which connections connections are to be made. For example,
	in a SpiNNaker machine a unit is a 48-chip board while in data center, a unit
	might be a server blade.
	
	\section{Related work}
		
		In this section I describe the techniques conventionally employed when
		laying out and interconnecting the units within super computers. Due to
		SpiNNaker's hexagonal torus topology and dense physical packing of units,
		these existing techniques are found to be insufficient. In the remainder of
		the chapter we will explore solutions to the limitations exposed below.
		
		\subsection{Avoiding long cables}
			
			Na\"ive arrangements of torus topologies, including hexagonal torus
			topologies, feature long `wrap-around' connections which connect units at
			the peripheries of the system. These connections can be problematic for
			numerous reasons:
			
			\begin{description}
				
				\item[Performance] Signal quality diminishes as cables get longer,
				requiring the use of slower signalling speeds, increased error
				correction overhead or more complex hardware.
				
				\item[Energy] Longer cables require higher drive strengths and/or
				buffering to maintain signal integrity.
				
				\item[Cost] Cost Shorter cables are cheaper than long ones.  Longer
				cables imply more wire in a given space making the tasks of routing or
				cable installation more difficult increasing labour costs by as much as
				$5\times$ \cite{curtis12}.
				
			\end{description}
			
			In conventional torus topologies the need for long cables is eliminated
			by folding and interleaving units of the network \cite{dally04}. For
			example, for a 1D torus topology (a ring network), one long connection
			exists to connect the two opposite sides of the system. To remove these
			long connections, half the units are `folded' on top of the others and
			then this arrangement of units is interleaved as illustrated in figure
			\ref{fig:ring-folding}.
			
			\begin{figure}
				\center
				\begin{subfigure}[b]{0.39\linewidth}
					\center
					\buildfig{figures/ring-folding-row.tex}
					\caption{A ring network}
					\label{fig:ring-folding-row}
				\end{subfigure}
				\begin{subfigure}[b]{0.24\linewidth}
					\center
					\buildfig{figures/ring-folding-folded.tex}
					\caption{Folded}
					\label{fig:ring-folding-folded}
				\end{subfigure}
				\begin{subfigure}[b]{0.35\linewidth}
					\center
					\buildfig{figures/ring-folding-interleaved.tex}
					\caption{Folded and interleaved}
					\label{fig:ring-folding-interleaved}
				\end{subfigure}
				
				\caption{Folding and interleaving a ring network to reduce maximum wire
				length.}
				\label{fig:ring-folding}
			\end{figure}
			
			Folding and interleaving has the effect of approximately doubling the
			average cable length but also eliminates the need for a cable spanning
			the entire system. Since the mean cable length is typically already
			short, doubling it in exchange for a substantially reduced maximum cable
			length is often preferable.
			
			The folding and interleaving process may be extended to $N$-dimensional
			torus topologies by folding each dimension in turn. Since all dimensions
			are orthogonal, the folding process only moves units in the dimension
			being folded. In the hexagonal torus topology, however, the three
			dimensions are non-orthogonal and thus folding in one dimension also
			moves units in other dimensions, preventing the edges of the torus
			meeting as illustrated in figure \ref{fig:failing-to-fold-hex-toruses}.
			
			\begin{figure}
				\center
				\begin{subfigure}[b]{0.24\linewidth}
					\center
					\buildfig{figures/failing-to-fold-hex-toruses-none.tex}
					\caption{Not folded}
					\label{fig:failing-to-fold-hex-toruses-none}
				\end{subfigure}
				\begin{subfigure}[b]{0.24\linewidth}
					\center
					\buildfig{figures/failing-to-fold-hex-toruses-x.tex}
					\caption{X}
					\label{fig:failing-to-fold-hex-toruses-x}
				\end{subfigure}
				\begin{subfigure}[b]{0.24\linewidth}
					\center
					\buildfig{figures/failing-to-fold-hex-toruses-y.tex}
					\caption{Y}
					\label{fig:failing-to-fold-hex-toruses-y}
				\end{subfigure}
				\begin{subfigure}[b]{0.24\linewidth}
					\center
					\buildfig{figures/failing-to-fold-hex-toruses-z.tex}
					\caption{Z}
					\label{fig:failing-to-fold-hex-toruses-z}
				\end{subfigure}
				
				\caption{Schematics showing hexagonal torus topologies folded along
				each of their non-orthogonal dimensions. Note that folding along
				the Z axis brings the \emph{wrong} edges closer together.}
				\label{fig:failing-to-fold-hex-toruses}
			\end{figure}
		
		\subsection{Cabling installation}
			
			Existing machine room installations feature very repetitive cabling
			patterns which can easily be memorised by a human technician. For example
			in BlueGene super computers the connectivity between units is highly
			regular \cite{lakner07} while in data centre networks cabling often
			centres around a small number of high-port-count switches
			\cite{cisco07,csernai15}. Cable installation is usually only aided by
			the labelling of connectors and sockets in a standardised manner
			\cite{tia2006} such as in figure \ref{fig:bgWiring}.
			
			\begin{figure}
				\center
				\begin{subfigure}[t]{0.5\textwidth}
					\begin{tikzpicture}
						\node (cables) [inner sep=0]
						      {\includegraphics[width=\textwidth]{figures/bgCables.png}};
						\node (sockets) [inner sep=0, below=1.0em of cables]
						      {\includegraphics[width=\textwidth]{figures/bgSockets.png}};
						
						% Point at label on cable
						\draw [white, <-, line width=0.4em]
						      ([shift={(0.7cm, -0.3cm)}]cables.center)
						      -- ++(45:1cm);
						
						% Point at label on socket
						\draw [white, <-, line width=0.4em]
						      ([shift={(-1.0cm, 1.1cm)}]sockets.center)
						      -- ++(-45:1cm);
					\end{tikzpicture}
					
					\caption{Pre-labelled cables and sockets}
					\label{fig:bgWiringLabels}
				\end{subfigure}
				~
				\begin{subfigure}[t]{0.30\textwidth}
					\includegraphics[height=6.15cm]{figures/bgWiring.jpg}
					
					\caption{Installation of cables}
					\label{fig:bgWiringInstallation}
				\end{subfigure}
				
				\caption{BlueGene/Q cable installation \cite{cscs13}}
				\label{fig:bgWiring}
			\end{figure}
			
			Despite the regularity and careful labelling of cables, the cost of
			installation and maintenance alone can be significant with costs in the
			range of \$45-95 per \SI{1}{\meter} cable run and \$100-400 for runs of
			\SI{10}{\meter} reported in the literature \cite{mudigonda11}. Much of
			this cost is due to the care required during installation to avoid
			miswiring and ensure that cooling airflow is not hampered by cable runs
			\cite{cisco07}.
			
			Many researchers have attempted to control cable installation costs by
			trying to reduce the number or length of cables required by developing
			alternative network topologies \cite{curtis12, popa10, mudigonda11}.
			Unfortunately, these techniques do not apply to SpiNNaker since its
			network topology is fixed.
			
			Some super computers make use of large custom `midplane` PCBs in place of
			cables to interconnect units within a cabinet and thus simplify the task
			of cable installation \cite{prickett10}. This scheme can greatly reduce
			wiring complexity since only coarser-grain cabinet-to-cabinet
			connectivity is provided by cables. Unfortunately this technique is
			expensive and also constrains the dimensions of the network topology
			supported by the machine. Since the SpiNNaker platform is designed to
			scale from desktop machines to machine-room installations, this scheme is
			not practical.
	
	\section{Folding \& interleaving hexagonal toruses}
		
		The first step towards a practical machine-room installation of a large
		machine using a hexagonal torus topology is to find an arrangement of
		boards between which cable lengths are minimised. In this section I
		describe a sequence of transformations which map the positions of units in
		a hexagonal torus topology onto a regular rectangular grid which may be
		folded and interleaved to eliminate long wires. It is worth emphasising
		that this transformation only affects the \emph{physical} positions of
		units and \emph{not} their connectivity.
		
		As described earlier in \S\ref{sec:parititioning} (page
		\pageref{sec:parititioning}), hexagonal torus topologies may be partitioned
		into units containing wrapped-triples of nodes. For example, in SpiNNaker,
		chips (nodes) are partitioned into circuit boards (units) containing 48
		chips. For completeness, this section describes the process of folding both
		systems whose units are individual nodes and those whose units are
		wrapped-triples.
		
		The transformation process is divided into two parts, each described
		separately in this section.
		
		\begin{description}
			
			\item[Parallelogram to rectangle] Units of the system are transformed
			from a parallelogram shape to a rectangular shape.
			
			\item[Uncrinkle] Units within the rectangle are moved such that they all
			lie on a regular (and fully packed) 2D grid.
			
		\end{description}
		
		\subsection{Parallelogram to rectangle}
			
			The hexagonal torus topology is most naturally drawn as a parallelogram
			as illustrated in figures \ref{fig:hex-to-plane-node-native} and
			\ref{fig:hex-to-plane-native}. Two transformations are presented which
			transform these arangements of units into a rectangular form: shearing
			and slicing.
			
			A \SI{30}{\degree} shear transformation distorts networks such that the X
			and Y axes become orthogonal leading to a rectangular arrangement of
			units as illustrated in figures \ref{fig:hex-to-plane-node-shear} and
			\ref{fig:hex-to-plane-shear}.
			
			The slice transformation slices the units protruding from the
			left-hand-side of the parallelogram and moves them into the matching gap
			on the opposite side of the parallelogram as illustrated in figures
			\ref{fig:hex-to-plane-node-slice} and \ref{fig:hex-to-plane-slice}.
			 
			While the shear transformation introduces some distortion causing cables
			in the Z dimension to become $\sqrt{2}\times$ longer it leaves the
			pattern of wrap-around connections remains unchanged. By contrast, the
			slice transformation does not elongate any cables but changes the pattern
			of wrap-around connections. The exact pattern wrap-around connections
			produced when slicing depends on the aspect ratio of the network as
			illustrated in \ref{fig:slicing-examples} and influences the choice of
			folding technique applied as described later.
			
			\begin{figure}
				\center
				\begin{subfigure}[b]{0.32\linewidth}
					\center
					\buildfig{figures/hex-to-plane-node-native.tex}
					
					\caption{$7 \times 7$ node torus}
					\label{fig:hex-to-plane-node-native}
				\end{subfigure}
				\begin{subfigure}[b]{0.32\linewidth}
					\center
					\buildfig{figures/hex-to-plane-node-shear.tex}
					
					\caption{Sheared}
					\label{fig:hex-to-plane-node-shear}
				\end{subfigure}
				\begin{subfigure}[b]{0.32\linewidth}
					\center
					\buildfig{figures/hex-to-plane-node-slice.tex}
					
					\caption{Sliced}
					\label{fig:hex-to-plane-node-slice}
				\end{subfigure}
				
				\caption{Transformations of hexagonal toruses of nodes into a
				rectangular form. Thin lines show wrap-around links. Pointy-topped
				hexagons represent individual nodes.}
				\label{fig:hex-to-plane-node}
			\end{figure}
			
			\begin{figure}
				
				\begin{subfigure}[b]{0.32\linewidth}
					\center
					\buildfig{figures/hex-to-plane-native.tex}
					
					\caption{$4 \times 4$ triad torus}
					\label{fig:hex-to-plane-native}
				\end{subfigure}
				\begin{subfigure}[b]{0.32\linewidth}
					\center
					\buildfig{figures/hex-to-plane-shear.tex}
					
					\caption{Sheared}
					\label{fig:hex-to-plane-shear}
				\end{subfigure}
				\begin{subfigure}[b]{0.32\linewidth}
					\center
					\buildfig{figures/hex-to-plane-slice.tex}
					
					\caption{Sliced}
					\label{fig:hex-to-plane-slice}
				\end{subfigure}
				
				\caption{Transformations of hexagonal toruses of wrapped triples into a
				rectangular form.  Thin lines show wrap-around links. Flat-topped
				hexagons represent a wrapped triple of nodes.}
				\label{fig:hex-to-plane}
			\end{figure}
			
			\begin{figure}
				\center
				\buildfig{figures/slicing-examples.tex}
				\caption{Patterns of wiring in sliced systems of various sizes.}
				\label{fig:slicing-examples}
			\end{figure}
			
		\subsection{Uncrinkling}
			
			Though the transformmation step yields rectangular arrangements of units,
			these arrangements do not fall onto a regular 2D grid, with the exception
			of the shear transform on individual nodes. Figure \ref{fig:uncrinkling}
			illustrates how the various arrangements of hexagons may be moved to
			`uncrinkle' the units into a regular grid.
			
			\begin{figure}
				\center
				\begin{subfigure}[b]{0.44\linewidth}
					\center
					\buildfig{figures/uncrinkling-node-sheared.tex}
					
					\caption{$7 \times 7$ nodes, sheared}
					\label{fig:uncrinkling-node-sheared}
				\end{subfigure}
				\begin{subfigure}[b]{0.44\linewidth}
					\center
					\buildfig{figures/uncrinkling-node-sliced.tex}
					
					\caption{$7 \times 7$ nodes, sliced}
					\label{fig:uncrinkling-node-sliced}
				\end{subfigure}
				
				\vspace{1cm}
				
				\begin{subfigure}[b]{0.44\linewidth}
					\center
					\buildfig{figures/uncrinkling-sheared.tex}
					
					\caption{$4 \times 4$ triples, sheared}
					\label{fig:uncrinkling-sheared}
				\end{subfigure}
				\begin{subfigure}[b]{0.44\linewidth}
					\center
					\buildfig{figures/uncrinkling-sliced.tex}
					
					\caption{$4 \times 4$ triples, sliced}
					\label{fig:uncrinkling-sliced}
				\end{subfigure}
				
				\vspace{1em}
				
				\caption{Mapping rectangular arrangements of units into a square grid.
				Thick lines show how layers of units are uncrinkled.  Annotations show
				how the relative positions of nodes and wrapped triples change after
				uncrinkling.}
				\label{fig:uncrinkling}
			\end{figure}
			
			In the figure, the numbered units enumerate the different positions on
			the crinkle and those labelled alphabetically are those that immediately
			surround them. From this we can observe that uncrinkling largely
			preserves spatial locality but some distortion is introduced, separating
			previously neighbouring units. For example, in figure
			\ref{fig:uncrinkling-sheared}, the units labelled `1' and `i' are
			neighbours before uncrinkling but are separated by a (Euclidean) distance
			of $\sqrt{5}$ afterwards. Note that the distortion introduced depends on
			what part of the crinkle is considered, for example `2' and `a' have
			distance 2 but are logically connected in the same way.
		
		\subsection{Folding and Interleaving}
			
			Once a regular grid of units has been formed, this may be folded in the
			conventional way, eliminating long cables crossing from left-to-right and
			top-to-bottom as illustrated in \ref{fig:folding-sheared}.
			
			Unfortunately, for sliced systems whose dimensions are not of the ratio
			$1:2$, the pattern of wrap-around cables may also include some cables
			which do not cross directly to the opposite side of the system (refer
			back to figure \ref{fig:slicing-examples}). As a result of these
			connections, folding does not successfully eliminate all long
			connections. An exception to this rule is sliced systems whose dimensions
			are in the ratio $1:1$ where folding twice along the Y axis may
			successfully eliminate all wrap-around connections as illustrated in
			\ref{fig:folding-sliced}.
			
			\begin{figure}
				\begin{subfigure}{\linewidth}
					\center
					\buildfig{figures/folding-sheared.tex}
					\caption{$N \times M$ sheared systems and $N \times 2N$ sliced systems}
					\label{fig:folding-sheared}
				\end{subfigure}
				
				\vspace{1em}
				
				\begin{subfigure}{\linewidth}
					\center
					\buildfig{figures/folding-sliced.tex}
					\caption{$N \times N$ sliced systems}
					\label{fig:folding-sliced}
				\end{subfigure}
				
				\caption{Schematic illustrating elimination of long wrap-around links
				during folding. In each example a single link has been highlighted to
				aid in following the process.}
				\label{fig:folding}
			\end{figure}
			
			Once folded, the 2D grid is straight-forwardly interleaved as illustrated
			previously in figure \ref{fig:ring-folding}. The interleaving process
			introduces some additional distortion to the layout of units and causes
			most connections to become twice as long. For sliced $1:1$ systems, the
			additional fold results in additional overhead during interleaving since
			four layers of the system are interleaved.
		
		\subsection{Mapping to Cabinets}
			
			In the final step of the process is to map the 2D grid of units into
			positions in machine room cabinets as illustrated in figure
			\ref{fig:million-core-machine}. As illustrated in figure
			\ref{fig:cabinetisation}, first the grid of units is partitioned into
			groups of columns, one per cabinet, then groups of rows one per frame per
			cabinet. The units in each group are then allocated to slots within a
			frame, interleaving the rows of the groups as shown.
			
			\begin{figure}
				\center
				\buildfig{figures/cabinet-units.tex}
				
				\caption{An illustration of the physical construction of a
				multi-cabinet SpiNNaker system. (Note: network cables \emph{not}
				installed.)}
				\label{fig:cabinet-units}
			\end{figure}
			
			\begin{figure}
				\center
				\buildfig{figures/cabinetisation.tex}
				
				\caption{Mapping from 2D space to cabinets, frames and boards.}
				\label{fig:cabinetisation}
			\end{figure}
		
	\section{Cable installation}
		
		Cable installation is performed by a team of (human) technicians who must
		ensure that all network cables are correctly installed. As illustrated in
		previously in figure \ref{fig:cabinet-units}, the density of SpiNNaker's
		units, combined with the nature of the hexagonal torus topology, poses a
		challenge. To address this challenge I propose a semi-automated approach to
		cable installation which exploits diagnostic facilities available in the
		majority of super computers in order to guide technicians through the
		cabling process, interactively guiding installation and maintenance.
		
		\subsection{Interactive technician guidance and validation}
			
			While automated systems for validating cabling correctness are
			commonplace, these systems are typically used only after cabling has been
			completed \cite{lakner07}. As with other large-scale machines, SpiNNaker
			includes a low-bandwidth system management bus which may be used to
			interrogate network hardware and control diagnostic LEDs prior to the
			installation of the main SpiNNaker network interconnect.  Using these
			facilities I have constructed a tool called SpiNNer which interactively
			guides a technician, or team of technicians, through the cable
			installation process, validating each connection in real-time.
			
			Diagnostic LEDs mounted on each SpiNNaker board (figure
			\ref{fig:interactive-wiring-guide-leds}) are used to indicate the
			endpoints of the cable currently being installed. Simultaneously a
			Text-To-Speech (TTS) system gives an audible indication of which cable
			type is to be used and location of each connection.  Additionally, a GUI
			via a computer display (figure \ref{fig:interactive-wiring-guide-gui}).
			The centre of the display shows a `big-picture' perspective of the
			locations of the boards to be connected. The detailed views on the left
			and right indicate which of the six sockets on each board the cables
			should connect.
			
			\begin{figure}
				\center
				\begin{subfigure}[b]{0.40\textwidth}
					\begin{tikzpicture}
						\node (leds) [inner sep=0]
						      {\includegraphics[width=\textwidth]{figures/leds.jpg}};
						% Point at left LED
						\draw [white, <-, line width=0.4em]
						      ([shift={(-0.0cm, -0.6cm)}]leds.center)
						      -- ++(225:1cm);
						% Point at right LED
						\draw [white, <-, line width=0.4em]
						      ([shift={(1.1cm, -1.1cm)}]leds.center)
						      -- ++(225:1cm);
					\end{tikzpicture}
					
					\caption{Diagnostic LEDs}
					\label{fig:interactive-wiring-guide-leds}
				\end{subfigure}
				~
				\begin{subfigure}[b]{0.546\textwidth}
					\begin{tikzpicture}[thin, black!20!white]
						\node (screen) [inner sep=0]
						      {\includegraphics[width=\textwidth]{figures/wiring_guide_screenshot.png}};
						\draw (screen.south west) rectangle (screen.north east);
					\end{tikzpicture}
					
					\caption{Interactive wiring guide GUI}
					\label{fig:interactive-wiring-guide-gui}
				\end{subfigure}
				
				\caption{The SpiNNer interactive wiring guide uses a GUI,
				text-to-speech and diagnostic LEDs to assist during cable
				installation.}
				\label{fig:interactive-wiring-guide}
			\end{figure}
			
			SpiNNer also validates the connectivity of the system in real-time by
			polling the diagnostic interfaces of the network hardware at the
			endpoints of the cable being installed to determine if they are correctly
			connected. If a miswiring occurs, this is immediately detected and
			announced via TTS enabling the technician to immediately correct the
			error. Once a cable has been installed correctly, the software
			automatically advances to the next cable meaning direct interaction with
			the software by the technician is minimal. In practice, it is rarely
			necessary to refer to the GUI.
		
			SpiNNer presents the cables in an order intended to maximise ease of
			installation. Cables are installed in three groups with intra-frame
			cables being installed first, followed by intra-cabinet cables and
			inter-cabinet cables. Within each group, the tightest cables are
			installed first resulting in slacker cables naturally being installed
			over the top of already installed cables. By grouping cables in this
			manner, multiple technicians may work independently on the wiring within
			individual frames and cabinets.
			
			SpiNNer may also be used to repair or replace cables in the system.
			During maintenance, obstructing cables may be blindly removed alongside
			any cable being replaced. At the conclusion of the process, the wiring
			guide may be used to interactively guide re-installation of all removed
			cables.
		
		\subsection{Cable selection}
			
			Controlling slack is critical to ensuring reliable and maintainable
			cabling installations. If cables are too tight, cables and connectors can
			become easily damaged and when too slack, the excess cable obstructs
			other cables and can easily become tangled and damaged \cite{cisco07}. It
			has been observed that when ready-made cables are employed technicians
			frequently over-estimate the cable lengths required preferring to use
			overly long cables for all connections \cite{mazaris97}. To solve this
			problem, the SpiNNer wiring guide software dictates the cable lengths to
			be used by an installer based the rule of (three-)thumbs according to
			Mazaris \cite{mazaris97}. This rule suggests that an ideal amount of
			slack is approximately that which can be wrapped around three fingers.
			Specifically, the shortest available cable is selected which ensures at
			least \SI{5}{\centi\meter} of slack.
			
			The SpiNNer tool allocates cables assuming all cables take a Euclidean
			straight-line path between the endpoints of the connection. The result is
			that wiring is not routed through dedicated cable management structures
			but is simply suspended by its connectors in front of the machine. As
			demonstrated later, this unconventional approach leads neither to cooling
			problems nor increased maintenance effort.
	
	\section{Results and Evaluation}
		
		This stuff has been used and works. In this section I'll go over the
		overheads introduced by the various transformations and
		folding/interleaving steps and show a wiring scheme for a large machine
		which uses only short cables. I'll then show how SpiNNer was used to
		install this wiring plan into a very large machine without foobaring the
		cooling and in very little time. I'll also report on difficulty of
		maintenance.
		
		\subsection{Cable length}
			
			The transformation from regular hexagonal torus to a folded and
			interleaved form introduces some overhead to the cable lengths required.
			Using figure \ref{fig:uncrinkling} (page \pageref{fig:uncrinkling}), it
			is possible to compute the exact overhead introduced when each type of
			transformation proposed.
			
			For example, to compute the mean overhead introduced by the slicing
			technique when applied to units composed of wrapped triples, consider
			figure \ref{fig:uncrinkling-sliced}. The uncrinkling pattern used to
			transform this topology is a repeating pattern of two units, a pair of
			which have been labelled $1$ and $2$ respectively. Unit $1$ is
			immediately surrounded by six units labelled $a$, $b$, $c$, $2$, $g$ and
			$h$. Similarly, unit $2$ is surrounded by units $1$, $c$, $d$, $e$, $f$
			and $g$. Before the transformation, the distances, $D$, to each of these
			units is $1$ but after the transformation is applied, this is not always
			the case. Additionally, folding and interleaving introduce additional
			overhead. In this example, if the system is folded into $f_x$ columns and
			$f_y$ rows, the distances between previously neighbouring units become:
			
			\begin{equation*}
				\begin{aligned}[c]
					D_{1\,\leftrightarrow{}\,a} &= \sqrt{f_x^2 + f_y^2} \\
					D_{1\,\leftrightarrow{}\,b} &= f_y \\
					D_{1\,\leftrightarrow{}\,c} &= \sqrt{f_x^2 + f_y^2} \\
					D_{1\,\leftrightarrow{}\,2} &= f_x \\
					D_{1\,\leftrightarrow{}\,g} &= f_y \\
					D_{1\,\leftrightarrow{}\,h} &= f_x
				\end{aligned}
				\hspace{2cm}
				\begin{aligned}[c]
					D_{2\,\leftrightarrow{}\,1} &= f_x \\
					D_{2\,\leftrightarrow{}\,c} &= f_y \\
					D_{2\,\leftrightarrow{}\,d} &= f_x \\
					D_{2\,\leftrightarrow{}\,e} &= \sqrt{f_x^2 + f_y^2} \\
					D_{2\,\leftrightarrow{}\,f} &= f_y \\
					D_{2\,\leftrightarrow{}\,g} &= \sqrt{f_x^2 + f_y^2}
				\end{aligned}
			\end{equation*}
			
			From these values, the mean and maximum connection distances after
			folding and interleaving may be computed. Table
			\ref{tab:transform-overhead} gives the mean and maximum connection
			distances for each of the four transformations described in this chapter.
			
			\begin{table}
				\begin{subtable}[b]{\linewidth}
					\center
					\begin{tabular}{l c c}
						\toprule
						& Shear & Slice \\
						\addlinespace
						Nodes &
							$\frac{f_x + f_y + \sqrt{f_x^2 + f_y^2}}{3}$ &
							$\frac{f_x + f_y + \sqrt{f_x^2 + f_y^2}}{3}$ \\
						\addlinespace
						Triples &
							$\frac{7f_x + 3\sqrt{f_x^2 + f_y^2} + \sqrt{(2f_x)^2 + f_y^2}}{9}$ &
							$\frac{f_x + f_y + \sqrt{f_x^2 + f_y^2}}{3}$ \\
						\bottomrule
					\end{tabular}
					
					\caption{Mean}
					\label{tab:transform-overhead-mean}
				\end{subtable}
				
				\vspace{1em}
				
				\begin{subtable}[b]{\linewidth}
					\center
					\begin{tabular}{l c c}
						\toprule
						& Shear & Slice \\
						\addlinespace
						Nodes &
							$\sqrt{f_x^2 + f_y^2}$ &
							$\sqrt{f_x^2 + f_y^2}$ \\
						\addlinespace
						Triples &
							$\sqrt{(2f_x)^2 + f_y^2}$ &
							$\sqrt{f_x^2 + f_y^2}$ \\
						\bottomrule
					\end{tabular}
					
					\caption{Maximum}
					\label{tab:transform-overhead-max}
				\end{subtable}
				
				\caption{Overheads introduced when transforming unit positions onto a
				regular grid.}
				\label{tab:transform-overhead}
			\end{table}
			
			From these results it is evident that shearing and slicing networks
			whose units are nodes result in identical mean and maximum overhead in
			cable length when folded similarly. Since sliced networks may require
			folding more than once along each axis the shearing approach is
			preferable in general.
			
			For networks constructed from units of wrapped triples, the slicing
			approach suffers the same mean and maximum overhead has networks of
			nodes, and less overhead than shearing for the same number of folds. For
			systems with an aspect ratio of $1:2$ (where both slicing and shearing
			require $f_x = f_y = 2$), the slicing transformation yields lower mean
			and maximum overhead than shearing. For all other aspect ratios (where
			slicing requires a greater degree of folding) the shearing technique
			produces lower overhead. The recommended transformations for a given
			machine are thus given in table \ref{tab:transform-recommended}.
			
			\begin{table}
				\center
				\begin{tabular}{lcc}
					\toprule
					                         & $1:2$  & Other \\
					\addlinespace
					\multirow{2}{*}{Nodes}   & Either & Shear\\
					                         & \footnotesize $\mu\approx2.28 \quad \vee\approx2.83$
					                         & \footnotesize $\mu\approx2.28 \quad \vee\approx2.83$\\
					\addlinespace
					\multirow{2}{*}{Triples} & Slice  & Shear\\
					                         & \footnotesize $\mu\approx2.28 \quad \vee\approx2.83$
					                         & \footnotesize $\mu\approx3.00 \quad \vee\approx4.47$\\
					\bottomrule
				\end{tabular}
				
				\caption{Recommended transformation and folding scheme for different
				system types. $\mu$ and $\vee$ give the mean and maximum wire
				distortion introduced, respectively.}
				\label{tab:transform-recommended}
			\end{table}
			
			\begin{figure}
				\center
				\buildfig{figures/million-core-machine.tex}
				
				\caption{Cabling plan for a \num{1036800} core SpiNNaker
				machine's \num{3600} cables.}
				\label{fig:million-core-machine}
			\end{figure}
			
			Following folding and mapping to physical locations, the cabling plans
			for large machines require no large gaps to be spanned.  The largest
			planned SpiNNaker machine, illustrated in figure
			\ref{fig:million-core-machine}, will be \SI{6}{\meter} wide but the
			largest gap any cable must span is \SI{66}{\centi\meter}. This distance
			is well within the \SI{1}{\meter} allowed by the hardware and cables.
			
		\subsection{Installation practicality}
			
			\begin{table}
				\center
				\begin{tabular}{lrr@{$\,$}l}
					\toprule
						System & Number of Cables & \multicolumn{2}{r}{Installation time} \\
					\midrule
						24 boards  & \num{72}   & \num{10} & \si{\minute}         \\
						1 cabinet  & \num{360}  & \num{4}  & \si{\hour}$^\dagger$ \\
						2 cabinets & \num{720}  & \num{2}  & \si{\hour}           \\
						5 cabinets & \num{1800} & ?        &                      \\
					\bottomrule
				\end{tabular}
				
				\caption{Installation times for various sizes of machine.
				$\dagger$~This machine was installed without real-time validation of
				connectivity.}
				\label{tab:install-time}
			\end{table}
			
			A number of SpiNNaker machines of various scales have been assembled
			using the techniques described in this chapter ranging from single frames
			of 24 boards to a half-scale 5 cabinet machine. Table
			\ref{tab:install-time} gives the reported installation times of each of
			these machines.
			
			The single cabinet machine's installation time is notably
			disproportionate to its size. When this system was assembled, real-time
			connection validation was not yet available. As a result, though cable
			installation was rapid correcting errors was extremely costly, requiring
			careful retracing of many installation steps.
			
			TODO: TALK ABOUT MULTI-PERSON-WIRING IN PRACTICE ON FIVE CABINET MACHINE.
			
			\begin{figure}
				
				\center
				\buildfig{figures/wire-length-histogram.tex}
				
				\caption{Histogram of connection distances in a ten-cabinet,
				one-million core SpiNNaker machine annotated with the suggested cable
				length.}
				\label{fig:wire-length-histogram}
				
			\end{figure}
			
			FIGURE \ref{fig:wire-length-histogram} SHOWS THE DISTRIBUTION OF CABLE
			LENGTHS REQUIRED. IN PRACTICE THE SLACK ALLOCATED PROVED ADEQUATE. AS
			SHOWN IN FIGURE \ref{fig:install-histogram}, THE MOST IMPORTANT FACTOR IS
			WHETHER LEAVING THE FRAME OR NOT. LEAVING THE FRAME TAKES THE LONGEST.
			
			\begin{figure}
				\builddata{data/build_connection_log.tex}
				\buildfig{figures/install-histogram.tex}
				
				\caption{Histogram of cable installation times}
				\label{fig:install-histogram}
			\end{figure}
			
			TODO: COMPARE DIRECTLY WITH INSTALL TIMES REPORTED IN LITERATURE.
		
		\subsection{Thermal Impact}
			
			TODO: SHOW HOW TEMPERATURE IS CHANGED
			
		\subsection{Maintenance}
			
			TOOD: QUANTIFY CABLE REMOVALS REQUIRED. EXPERIMENT: REMOVE/REPLACE RANDOM
			BOARDS AND MEASURE TIME TAKEN, CABLES REMOVED. COMPARE WITH STANDARD DATA
			CENTRE WIRING

	\chapter{Finding shortest path vectors in SpiNNaker's network}
	
	Once a SpiNNaker machine has been constructed as described in the previous
	chapter, its network forms a large hexagonal torus topology. To exploit this
	network routing algorithms must be used to generate routes for packets to
	follow between nodes. As well as ensuring that packets arrive at the correct
	destination, routing algorithms often attempt to produce routes which make
	efficient use of the network. This often involves attempting to reduce
	congestion by ensuring packets do not travel further through the network than
	absolutely necessary.
	
	Many popular routing algorithms for torus topologies, including all published
	algorithms designed for SpiNNaker's hexagonal torus topology
	\cite{davies12,navaridas14}, internally function by computing shortest path
	vectors and generating routes from them. Existing methods of calculating
	shortest path vectors in hexagonal torus topologies are unable to generate
	all possible shortest path vectors and, as a result, reduces the diversity of
	routes produced by routing algorithms, potentially worsening network
	contention.
	
	In this chapter I describe a novel technique for computing shortest path
	vectors in hexagonal torus topologies which yields \emph{all} possible
	shortest path vectors for any pair of nodes. Further, implementations of this
	new technique execute an order of magnitude faster than the existing
	alternatives.
	
	\section{Related work}
		
		TODO: INTRODUCE SECTION
		
		\begin{figure}
			\center
			
			\begin{subfigure}{\linewidth}
				\center
				\buildfig{figures/distance-map-mesh.tex}
				\caption{2D mesh topology}
				\label{fig:distance-map-mesh}
			\end{subfigure}
			
			\vspace{1em}
			
			\begin{subfigure}{\linewidth}
				\center
				\buildfig{figures/distance-map-torus.tex}
				\caption{2D torus topology}
				\label{fig:distance-map-torus}
			\end{subfigure}
			
			\vspace{1em}
			
			\begin{subfigure}{\linewidth}
				\center
				\buildfig{figures/distance-map-hex-mesh.tex}
				\caption{Hexagonal mesh topology}
				\label{fig:distance-map-hex-mesh}
			\end{subfigure}
			
			\vspace{1em}
			
			\begin{subfigure}{\linewidth}
				\center
				\buildfig{figures/distance-map-hex-torus.tex}
				\caption{Hexagonal torus topology}
				\label{fig:distance-map-hex-torus}
			\end{subfigure}
			
			\caption{Plots showing distance from various locations marked
			         {\color{red}$\times$}. Darker areas are further away. Contour
			         lines show equidistant points.}
			\label{fig:distance-map}
		\end{figure}
		
		\subsection{Mesh Networks}
			
			In a (non-hexagonal) mesh network topology, shortest path vectors are
			computed by taking the element-wise difference between the source and
			destination nodes' coordinates.
			
			\begin{figure}
				\center
				\buildfig{figures/mesh-topology-coordinates.tex}
				\caption{An example 2D mesh network with example shortest-path routes
				from `A' to `B' and `B' to `C'.}
				\label{fig:mesh-topology-coordinates}
			\end{figure}
			
			For example, figure \ref{fig:mesh-topology-coordinates} illustrates a 2D
			mesh topology. In this topology, the nodes labelled `A', `B' and `C' have
			position vectors $(1, 2)$, $(4, 5)$ and $(6, 1)$ respectively. The
			shortest path vector from node `A' to `B' is thus simply $(4, 5) - (1, 2)
			= (3, 3)$ and from `B' to `C' is $(6, 1) - (4, 5) = (2, -4)$.
			
			A route may be produced from a shortest path vector by advancing the
			number of hops specified for each dimension in the vector. For example
			any permutation of the hops X$^+\,$X$^+\,$X$^+\,$Y$^+\,$Y$^+\,$Y$^+$, an
			example of which is included in the figure. Likewise a route from `B' to
			`C' may be constructed from any permutation of
			X$^+\,$X$^+\,$Y$^-\,$Y$^-\,$Y$^-\,$Y$^-$.
			
			Many popular routing algorithms such as Dimension Order Routing (DOR),
			Right-Turn Only Routing (RTOR) and Longest Dimension First Routing (LDFR)
			\cite{dally04,davies12} directly follow the above procedure and just
			prescribe a specific permutation of hop order. For example, DOR produces
			routes with X hops first, Y hops second and so on.
			
			The length of routes produced from a shortest path vector have a number
			of hops proportional to the magnitude of the vector, thus a shortest path
			vector yields a route with the minimum number of hops. For a two
			dimensional vector $(a, b)$ the magnitude is given as:
			%
			\begin{equation}
				\| (a, b) \| = \lvert a \rvert + \lvert b \rvert
			\end{equation}
		
		\subsection{Torus Networks}
			
			Computing shortest path vectors in (non-hexagonal) torus topologies is
			also straight forward. As an example, lets find the shortest path vector
			from node `A' to `B' in the 2D torus topology shown in figure
			\ref{fig:torus-shortest-path-example}. First, both nodes are translated
			such that the source node, `A', is at the centre of the network (figure
			\ref{fig:torus-shortest-path-translate}). Note that this translation may
			result in the destination node `wrapping around' the network. After
			translation, the shortest path vector is computed as in a mesh topology.
			As illustrated in \ref{fig:torus-shortest-path-routed}, the computed
			shortest path vector may be used to produce routes between the two nodes
			in their original positions.
			
			\begin{figure}
				\center
				\begin{subfigure}{0.3\linewidth}
					\center
					\buildfig{figures/torus-shortest-path-example.tex}
					\caption{Original}
					\label{fig:torus-shortest-path-example}
				\end{subfigure}
				\begin{subfigure}{0.3\linewidth}
					\center
					\buildfig{figures/torus-shortest-path-translate.tex}
					\caption{Translated}
					\label{fig:torus-shortest-path-translate}
				\end{subfigure}
				\begin{subfigure}{0.3\linewidth}
					\center
					\buildfig{figures/torus-shortest-path-routed.tex}
					\caption{Routed}
					\label{fig:torus-shortest-path-routed}
				\end{subfigure}
				
				\caption{Finding shortest paths in a 2D torus topology.}
				\label{fig:torus-shortest-path}
			\end{figure}
			
			This process works because vectors from the centre (though not other
			locations) of a torus topology are identical to those in mesh topologies
			(see figures \ref{fig:distance-map-mesh} and
			\ref{fig:distance-map-torus}).
		
		\subsection{Hexagonal Mesh Networks}
			
			In hexagonal mesh topologies it is conventional to define three `axes' X,
			Y and Z as shown in figure \ref{fig:hex-mesh-topology-coordinates}
			\cite{patel15}. In this example, the three labelled nodes `A', `B' and
			`C' may be given position vectors such as $(1, 1, 0)$, $(3, 2, 0)$ and
			$(0, 0, -7)$ respectively. As in other mesh networks, a vector between
			two nodes is found by subtracting the nodes' vectors. For example, a
			vector from `A' to `B' is $(3, 2, 0) - (1, 1, 0) = (2, 1, 0)$. This
			vector can then be converted into a route in the same way as a mesh
			network by taking any permutation of the three hops  X$^+\,$X$^+\,$Y$^+$.
			
			\begin{figure}
				\center
				\buildfig{figures/hex-mesh-topology-coordinates.tex}
				\caption{An example hexagonal mesh network topology.}
				\label{fig:hex-mesh-topology-coordinates}
			\end{figure}
			
			As explained in detail in appendix \ref{app:minimal-hex-coordinates},
			there are an infinite number of vectors between any two points. For
			example, the vectors $(1, 0, -1)$ and $(3, 2, 1)$ also reach node `B'
			from `A' in the example. However, for a given pair of nodes, there is
			always a single, unique vector whose magnitude is minimal which is
			given by the function:
			%
			\begin{equation}
				\operatorname{minimiseVector}(x,y,z)
					= (x,y,z) - \operatorname{median}(x,y,z) \cdot (1,1,1)
			\end{equation}
			%
			An important side-effect of this function is that a minimised vector will
			always contain at least one zero element meaning that shortest path
			routes will use at most two of the three available dimensions.
			
			To aid the reader's intuition, figure \ref{fig:distance-map-hex-mesh}
			illustrates how distances grow in a hexagonal mesh topology.
		
		\subsection{Hexagonal Torus Networks}
			
			Unfortunately, unlike non-hexagonal torus topologies, the translation
			technique cannot be used to compute shortest path vectors. As illustrated
			in figures \ref{fig:distance-map-hex-mesh} and
			\ref{fig:distance-map-hex-torus}, shortest path vectors from the center
			of a hexagonal mesh network are not equivalent to those of a hexagonal
			torus network.
			
			Prior research into routing in SpiNNaker's network has been based on the
			INSEE \cite{navaridas09,ghasempour15} interconnect simulator. Internally
			INSEE tries a set of twelve candidate vectors and picks the shortest as
			the shortest path vector to use for routing.
			
			\begin{figure}
				\center
				\begin{subfigure}{0.45\linewidth}
					\center
					\buildfig{figures/insee-vector-candidates-no-wrap.tex}
					\caption{$(\Delta_\textrm{X}, \Delta_\textrm{Y}) = (5,3)$}
					\label{fig:insee-vector-candidates-no-wrap}
				\end{subfigure}
				\begin{subfigure}{0.45\linewidth}
					\center
					\buildfig{figures/insee-vector-candidates-wrap-x.tex}
					\caption{$(\Delta'_\textrm{X}, \Delta_\textrm{Y}) = (-3,3)$}
					\label{fig:insee-vector-candidates-wrap-x}
				\end{subfigure}
				
				\vspace{1em}
				
				\begin{subfigure}{0.45\linewidth}
					\center
					\buildfig{figures/insee-vector-candidates-wrap-y.tex}
					\caption{$(\Delta_\textrm{X}, \Delta'_\textrm{Y}) = (5,-5)$}
					\label{fig:insee-vector-candidates-wrap-y}
				\end{subfigure}
				\begin{subfigure}{0.45\linewidth}
					\center
					\buildfig{figures/insee-vector-candidates-wrap.tex}
					\caption{$(\Delta'_\textrm{X}, \Delta'_\textrm{Y}) = (-3,-5)$}
					\label{fig:insee-vector-candidates-wrap}
				\end{subfigure}
				
				\vspace{1em}
				
				% Key
				\begin{tikzpicture}[thick]
					\coordinate (last);
					
					% #1 colour
					% #2 label
					\newcommand{\colourkeyentry}[2]{
						\node [#1] [right=of last, fill, rectangle, minimum size=1em] (last) {};
						\node [right=0 of last] (last) {#2};
					}
					
					\colourkeyentry{cb3class0}{$(\textrm{X}, \textrm{Y}, 0)$}
					\colourkeyentry{cb3class1}{$(\textrm{X} - \textrm{Y}, 0, - \textrm{Y})$}
					\colourkeyentry{cb3class2}{$(0, \textrm{Y} - \textrm{X}, - \textrm{X})$}
					
				\end{tikzpicture}
				
				\caption{The twelve candidate shortest-path vectors considered by INSEE
				represented as dimension-order routes. $W=H=8$,
				$(\Delta_\textrm{X},\Delta_\textrm{Y}) = (5, 3)$ and
				$(\Delta'_\textrm{X},\Delta'_\textrm{Y}) = (-3, -5)$.}
				\label{fig:insee-vector-candidates}
			\end{figure}
			
			The twelve vectors considered are constructed as follows.
			
			First a shortest path vector from the source to target node are
			constructed as if the network was a 2D mesh yielding a vector
			$(\Delta_\textrm{X},\Delta_\textrm{Y})$. From this, another vector
			$(\Delta'_\textrm{X},\Delta'_\textrm{Y})$, is defined:
			%
			\begin{align}
				\Delta'_\textrm{X} &= \Delta_\textrm{X} - \operatorname{sign}(\Delta_\textrm{X})W
				\\
				\Delta'_\textrm{Y} &= \Delta_\textrm{Y} - \operatorname{sign}(\Delta_\textrm{Y})H
			\end{align}
			%
			Where $W$ and $H$ are the width and height of the network respectively
			(in nodes). This new vector yields routes from the source to destination
			node that in a torus topology that \emph{always} wrap around the `X' and
			`Y' dimensions.
			
			From the pair of vectors defined, four possible 2D vectors can be
			produced: $(\Delta_\textrm{X},\Delta_\textrm{Y})$,
			$(\Delta'_\textrm{X},\Delta_\textrm{Y})$,
			$(\Delta_\textrm{X},\Delta'_\textrm{Y})$ and
			$(\Delta'_\textrm{X},\Delta'_\textrm{Y})$. Further, each 2D vector may be
			converted into one of three 3D vectors where either X, Y or Z are zero
			for a total of twelve candidate vectors.  Figure
			\ref{fig:insee-vector-candidates} illustrates all twelve candidate
			vectors for an example pair of nodes.
			
			\begin{figure}
				\center
				\buildfig{figures/xyz-protocol-regions.tex}
				
				\caption{The four regions defined by the XYZ-protocol.}
				\label{fig:xyz-protocol-regions}
			\end{figure}
			
			A more efficient technique is proposed by Hoffmann and D\'es\'erable
			called the XYZ-Protocol \cite{hoffmann15,hoffmann11}. If the source and
			destination nodes are translated such that the source node lies at the
			center of the topolgoy, the destination will lie in one of four regions
			illustrated in figure \ref{fig:xyz-protocol-regions}.
			
			If the destination lies in regions 1 or 4, a route may be constructed as
			if in a hexagonal mesh topology.
			
			Alternatively, if the destination lies in regions 2 or 3, the algorithm
			tests whether taking a mesh-like route within the region or
			wrapping-around either the X or Y dimension yields the shorter vector.
			The shortest of these vectors is then chosen.
			
			TODO DESCRIBE SPIRAL ROUTES.
			
			TODO DESCRIBE RTOR AND LDFR.
		
	\section{Dimension order routing in hexagonal torus topologies}
		
		So, existing solutions have two problems: trying 12 options and picking one
		is a bit kludgey and there are actually more options than that.
		
		\subsection{Simpler minimal hexagonal torus vectors}
			
			If you redraw your route such that it is sourced from bottom left corner
			(which we'll now call (0, 0)), there are four possible ways this route
			could wrap.
			
			TODO: DESCRIBE WAYS OF WRAPPING
			
			For each of these wrappings, all the possible routes we can take are
			strictly limited in terms of the dimensions used since we're stuck in a
			corner.
			
			In each case, the function computing the minimal hex vector function
			simplifies to a much simpler operation.
			
			TODO: DESCRIBE MINIMUM VECTOR LENGTH FUNCTIONS FOR EACH CASE
			
			This gives us a cheap way to compute which of the four possible wrappings
			are shortest. Based on this we can pick one of at most two (is this
			easily provable?) vectors in some fair manner to reduce load. This vector
			can then be minimised in the usual way.
			
			This also leads to confirming a theoretical result giving the length of a
			shortest path in a hexagonal torus topology.
			
			TODO: DESCRIBE HOW TO GET LENGTH FUNCTION AND COMPARE WITH \cite{xiao04}
		
		\subsection{Generating spiralling routes}
			
			In non-hexagonal torus topologies the previous technique would reveal all
			possible shortest vectors (e.g. in cases where you can wrap more than one
			way). Unfortunately, due to the addition of a non-orthogonal axes,
			hexagonal toruses also have other cases when the width and height do not
			match.
			
			TODO: TORUS SPIRALLING EXAMPLE
			
			It is possible to calculate the maximum number of spirals thus:
			
			TODO: DESCRIBE HOW MAX NUMBER OF SPIRALS IS COMPUTED
			
			Given a number of spirals, the vector can be updated this (note that the
			change does not add a multiple of (1, 1, 1) but also does not result in
			the vector changing length and thus becoming non-minimal!).
			
			TODO: DESCRIBE TRANSFORMATION
			
			TODO: PROVE THAT MINIMALITY IS MAINTAINED
		
		\subsection{Proof of completeness}
		
			TODO: PROOF OF COMPLETENESS BY EXHAUSTIVE SEARCH
	
		\subsection{Conclusions}
			
			This approach is simpler, smaller, has sounder theoretical basis, and
			finds more routes than alternatives. This is good for load balancing and
			fault avoidance and also good for completeness.


	\chapter{Routing packets in large SpiNNaker machines}
	
	\label{sec:routing}
	
	So far, this thesis has focused on tackling the practical challenges
	resulting from SpiNNaker's hexagonal torus network topology. In this chapter,
	I adjust my focus towards the practical challenges resulting from SpiNNaker's
	large scale. Faults in large systems are inevitable and in the half-million
	core, \num{28800} chip SpiNNaker machine recently completed at the University
	of Manchester, around \SI{1}{\percent} of chips exhibited faults\footnote{Of
	the faulty chips discovered, the vast majority of faults are attributed to a
	currently unknown SDRAM failure}. These faults result in gaps and broken
	links in the network topology which routing algorithms must avoid in order to
	ensure correct system operation.
	
	In this chapter I tackle the problem of extending existing routing algorithms
	for SpiNNaker's network to enable them to route-around known, static faults.
	Though dynamic or transient faults may also occur, in this work such faults
	are ignored and other techniques, such as protocol-level fault tolerance, are
	relied on instead.
	
	Numerous heuristic-based fault-tolerant routing algorithms exist which target
	different network topologies and router architectures. Unfortunately as I
	will show, these algorithms are not portable and rely on or attempt to work
	around specific features of their target network architecture. In particular,
	existing work is dominated by the challenge of developing routing schemes
	which avoid or defuse network deadlocks. Due to SpiNNaker's unconventional
	use of timeout-based flow-control, it is not subject to the routing
	restrictions present in other architectures intended to cope with deadlocks.
	
	In this chapter I introduce a graph-search based post-processing step for
	non-fault-tolerant routing algorithms which guarantees routability in
	SpiNNaker systems without disconnected subregions. I also demonstrate that
	this technique introduces both negligible computational overhead to the
	routing algorithm runtime and resulting network performance.
	
	TODO: NOTE THE FAULT RATES ENCOUNTERED IN PRACTICE...
	
	\section{Related work}
		
		Existing work on routing in SpiNNaker's network has ignored the challenge
		of avoiding faults and instead focused on producing efficient multicast
		routes. As a result this section is broken into two halves. In the first
		half I survey the existing non-fault-tolerant approaches to routing used in
		SpiNNaker to-date. In the second I discuss the approaches to fault tolerant
		routing taken in other systems.
		
		\subsection{Multicast routing in SpiNNaker}
			
			Various fault-intolerant multicast routing algorithms exist for many
			networks and a number have been proposed and evaluated specifically in the
			context of SpiNNaker.
			
			In 2012, Davies \emph{et al.} evaluated the use of three common torus
			routing algorithms in SpiNNaker and found that simple oblivious routing is
			suitable for typical neural applications \cite{davies12}. The three
			routing techniques are:
			
			\begin{description}
				
				\item[Dimension Order Routing (DOR)] Packets are routed along each
				dimension (e.g. $X$, $Y$ and $Z$) in turn until no further hops are
				available in that direction.  The order in which the dimensions are
				traversed is fixed.
				
				\item[Right Turn Only Routing (RTOR)] As in DOR except the dimension
				order is chosen such that routes only contain right-turns.
				
				\item[Longest Dimension First Routing (LDFR)] As in DOR except the
				dimension order is chosen in descending order of number of hops in each
				dimension.
				
			\end{description}
			
			These unicast routing techniques are converted into a multicast routing
			algorithm by merging together the routes produced between the source node
			and each destination node as illustrated in figure
			\ref{fig:simple-routers}.
			
			\begin{figure}
				\center
				\begin{subfigure}{0.3\linewidth}
					\center
					\buildfig{figures/simple-routers-dor.tex}
					
					\caption{DOR}
					\label{fig:simple-routers-dor}
				\end{subfigure}
				\begin{subfigure}{0.3\linewidth}
					\center
					\buildfig{figures/simple-routers-rtor.tex}
					
					\caption{RTOR}
					\label{fig:simple-routers-dor}
				\end{subfigure}
				\begin{subfigure}{0.3\linewidth}
					\center
					\buildfig{figures/simple-routers-ldfr.tex}
					
					\caption{LDFR}
					\label{fig:simple-routers-dor}
				\end{subfigure}
				
				\caption{Example multicast routes produced by merging together unicast
				routes from a central source node to each destination node.}
				\label{fig:simple-routers}
			\end{figure}
			
			In 2014, Navaridas \emph{et al.} introduced two new algorithms, `Enhanced
			Shortest Path Routing' (ESPR) and `Neighbourhood Exploring Routing' (NER)
			which produce multicast routing trees with fewer total hops
			\cite{navaridas14}. In both algorithms, routes are generated sequentially
			for each of the destinations of a route using LDFR. Unlike LDFR, however,
			these algorithms search a limited area of the network for other,
			already-connected destination nodes to which LDFR routes may be
			constructed. If no suitable destination is found, a LDFR route is
			constructed to the source node. Figure \ref{fig:search-regions} illustrates
			the shape of the searched regions of each algorithm. ESPR searches the
			trapezoidal region between the source and destination nodes while NER
			searches a fixed radius out from the destination node.
			
			\begin{figure}
				\center
				\begin{subfigure}{0.45\linewidth}
					\center
					\buildfig{figures/search-regions-espr.tex}
					
					\caption{ESPR}
					\label{fig:search-regions-espr}
				\end{subfigure}
				\begin{subfigure}{0.45\linewidth}
					\center
					\buildfig{figures/search-regions-ner.tex}
					
					\caption{NER}
					\label{fig:search-regions-espr}
				\end{subfigure}
				
				\caption{The ESPR and NER algorithms attempt to connect the node marked
				`D' to the closest node in the shaded region which is connected to the
				source node, `S'. If no connected node is found in the shaded region, the
				LDFR route is taken to `S'. The dotted line indicates the route chosen
				from `D'.}
				\label{fig:search-regions}
			\end{figure}
			
			Unfortunately none of these routing algorithms make any allowance for the
			avoidance of network faults. As a result their utility in real-world
			systems is limited.
		
		\subsection{Fault-tolerant routing}
			
			Numerous fault-tolerant routing algorithms have been proposed for
			super-computer networks. These algorithms are largely constrained by the
			need to maintain deadlock freedom. Since SpiNNaker's routers employ a
			timeout based deadlock-breaking strategy, much of this effort is
			unnecessary in SpiNNaker. As described below, this frequently renders
			existing fault-tolerant routing algorithms unnecessarily complex and
			inflexible.
			
			Deadlocks occur in a network if a cyclic dependency exists on any limited
			resource in the network. For example, as illustrated in figure
			\ref{fig:ring-deadlock}, in a ring network a deadlock may form when every
			node is waiting on the next node to accept a packet before accepting new
			packets from the previous node.
			
			\begin{figure}
				\center
				\buildfig{figures/ring-deadlock.tex}
				
				\caption{A deadlock in a ring network where each node is waiting for
				the next to accept a packet before accepting any further packets.}
				\label{fig:ring-deadlock}
			\end{figure}
			
			To prevent deadlocks, combinations of router microarchitectural features
			and routing restrictions may be employed. For example, a simple
			deadlock-free routing algorithm for mesh and torus networks mandates the
			use of DOR \cite{dally93}. Packets travelling in a -ve direction along
			each axis take priority over those travelling in a +ve direction. Packets
			travelling along the Y axis take priority over those travelling along the
			X dimension. Given these rules it is possible to define a total ordering
			on all hops in the network. Figure \ref{fig:deadlock-free-dor}
			illustrates a $3\times3$ mesh network whose hops have been numbered
			according to the total ordering defined above.  Any `X-then-Y' DOR route
			through this network results in the use of hops labelled with strictly
			increasing numbers. As a result, no cyclic dependencies (and thus no
			deadlocks) may occur.
			
			\begin{figure}
				\center
				\buildfig{figures/deadlock-free-dor.tex}
			
				\caption{Deadlock-free routing of two example routes using DOR in a 2D
				mesh topology. The numbers of the hops taken by each route are given on
				the right.}
				\label{fig:deadlock-free-dor}
			\end{figure}
			
			Unfortunately, the routing restrictions imposed to ensure deadlock
			freedom can result in fault-intolerant routing. In the example above, if
			the node at the bottom-right corner of the figure was faulty, the dotted
			example route would be blocked as no alternative routes are allowed.
			
			In practice, the routing rules used may be more relaxed, for example
			requiring that any route whose length is equal to a DOR must exist to
			guarantee routability \cite{rodrigo09}.
			
			Alternative routing strategies take a hybrid approach whereby an
			efficient but fault-intollerant routing algorithm is used where possible
			and in the presence of faults a less efficient but more robust strategy
			is employed. For example, the Immucube network architecture employs three
			virtual networks which operate independently over the same physical links
			\cite{puente07}. Initially messages are routed using a high-performance
			but potentially-deadlockable routing scheme in the first virtual network.
			If a deadlock is occurs, the deadlocked packet is dropped into the second
			virtual network in which packets are routed using a less efficient but
			deadlock-free but fault-intolerant routing algorithm. Finally, upon
			encountering a fault, packets are dropped onto the third virtual network
			which forms a ring network routing packets to every node in the network.
			
			Releated approaches \cite{mejia06,boppana95} divide the network into
			regions in which different routing rules are enforced to ensure deadlock
			freedom and, when required, fault tolerance.
			
			TODO FIGURE?
			
			The BlueGene/L supercomputer \cite{adiga02} uses DOR for its torus
			network and implements fault-tolerance by sacrificing otherwise
			functioning `lamb' nodes to ensure no route passes through a known dead
			link \cite{ho04}. In figure \ref{fig:lamb-nodes} an example scenario is
			shown where a single dead node is present and all nodes in the same row
			or column as the dead node have been made into lamb nodes. The lamb nodes
			may not be used in an application except as a through-route for other
			traffic. This pattern of lamb nodes guarantees that all dimension-order
			routes between all pairs of non-lamb nodes are not obstructed by the
			faulty node. This approach trades use of higher performance routing
			logic for wasted resources. This type of approach is most appropriate
			when algorithmic routing is used and routing rules are inflexible.
			
			\begin{figure}
				\center
				\buildfig{figures/lamb-nodes.tex}
				
				\caption{`Lamb' nodes may be disabled to ensure DOR will never
				encounter a fault.}
				\label{fig:lamb-nodes}
			\end{figure}
			
			Other algorithms proposed for the BlueGene architecture attempt to avoid
			the need for lamb nodes by generating routes which reach their destination
			via a `proxy' node \cite{gomez04}. By appropriately selecting the location
			of such a proxy, the existing routing algorithm used by the system can be
			guaranteed to select a route free of faults.
			
			TODO: EXAMPLE OF PROXY ROUTING TO AVOID FAULT
			
			Finally, many algorithms in in the field are distributed and use only local
			information along with limited information from their peers to generate
			routes \cite{fick09b}. In SpiNNaker, route generation is conventionally
			carried out centrally since no special on-chip hardware facilities exist
			for route generation. Centralised route generation also enables the routing
			algorithm to consider all available routes. As a result, there is little
			incentive for the use of distributed routing algorithms on SpiNNaker since
			global system information could be compactly shared for one-off routing
			passes.
			
			Algorithms for other architectures such as IP networks tend to be poor fits
			for static, regular network topologies since they use expensive graph-based
			algorithms for route discovery which aren't necessary here. They also tend
			to heavily feature graph topology discovery etc. which aren't needed here.
			
			Work on fault-tolerance in data centre networks does exploit the regularity
			of the network topology in routing algorithms \cite{guo08,liao12}.
			Unfortunately, the approaches used are not general enough to be applied to
			mesh-like topologies such as the one in SpiNNaker.
			
			Outside the field of computer networks, routing algorithms used to route
			wires across the surfaces of chips are required to solve similar problems
			to fault-tolerant network routing problems in mesh networks. Like mesh
			networks, the routes are defined within a regular Manhattan geometry and
			congested areas, rather than faults must be avoided by the algorithms
			\cite{kahng11}.  Unfortunately, these algorithms are designed for
			occasional batch operation prior to the multi-month process of chip
			manufacturing and so runtimes of hours or days are commonplace
			\cite{nam08}. As such these algorithms would be inappropriate for use
			with applications such as SpiNNaker where users' applications tend to be
			short-lived and thus routing should not be allowed to dominate runtime.
	
	\section{Partial graph search repair}
		
		In this section I introduce a novel post-processing algorithm, Partial
		Graph Search (PGS) repair, for routes produced by non-fault-tolerant
		routing algorithms.
		
		PGS repair guarantees routability for networks with no disconnected
		subregions by using a graph search algorithm to route around faults in the
		original route.  General-purpose graph search algorithms such as Breadth
		First Search (BFS), Dijkstra's Algorithm and A* are guaranteed to find
		shortest-path routes between pairs of points in arbitrary graphs. Such
		algorithms are generally a poor choice in highly regular network topologies
		such as meshes and toruses due to their high computational cost. In PGS
		repair, graph searching is only used for \emph{part} of the routing
		problem: to repair gaps in routes generated by more efficient routing
		algorithms.
		
		Real world super computer architectures are designed to ensure that faults
		are isolated \cite{gara05,alverson12} and thus tend to only impact a
		localised region of the network. Since PGS repair is only needed to route
		around these isolated faults, the space searched by the graph search
		algorithm should be very small in practice resulting in only short
		runtimes. In addition since faults are rare in real-world systems, the
		graph search process will only rarely be invoked.
		
		The PGS repair post-processing technique starts with a route produced by a
		non-fault-tolerant routing algorithm such as ESPR or NER. If this route is
		not obstructed by a fault, the algorithm terminates immediately without
		modifying the route. If the route attempts to use a faulty link, the
		algorithm proceeds as follows.
		
		The routing tree produced by the underlying routing algorithm is broken
		into subtrees wherever it attempts to route through a broken link and
		each subtree is assigned a unique colour, as illustrated in figure
		\ref{fig:pgs-repair-colouring}. From each disconnected subtree's root
		node in turn, a graph search is performed to find a short, fault-free
		route to a subtree node of a different colour. The subtree is then
		attached to the tree discovered by the graph search and re-coloured to
		match the tree it is connected to.
		
		\begin{figure}
			\center
			\begin{subfigure}{0.32\linewidth}
				\hspace*{-1.5em}
				\buildfig{figures/pgs-repair-colouring.tex}
				
				\caption{}
				\label{fig:pgs-repair-colouring}
			\end{subfigure}
			\begin{subfigure}{0.32\linewidth}
				\hspace*{-1.5em}
				\buildfig{figures/pgs-repair-colouring-fix1.tex}
				
				\caption{}
				\label{fig:pgs-repair-colouring-fix1}
			\end{subfigure}
			\begin{subfigure}{0.32\linewidth}
				\hspace*{-1.5em}
				\buildfig{figures/pgs-repair-colouring-fix2.tex}
				
				\caption{}
				\label{fig:pgs-repair-colouring-fix2}
			\end{subfigure}
			
			\caption{PGS repair process example showing a disconnected multicast
			route from A to B, C, D, E and F. $\times$ indicates a broken link.}
			\label{fig:pgs-repair-colouring-steps}
		\end{figure}
		
		For example in figure \ref{fig:pgs-repair-colouring-fix1} a path from the
		root of the subtree containing nodes E and F is found which connects it to
		the subtree rooted at A. Similarly in figure
		\ref{fig:pgs-repair-colouring-fix2} a path is also found connecting the
		subtree containing nodes C and D back to the subtree rooted at node A.
		
		If the routing tree was broken into $N+1$ subtrees by faults there will be
		$N$ subtrees disconnected from the root node. Each of the $N$ iterations of
		the algorithm connect a disconnected subtree to another subtree reducing
		the number of subtrees by $1$ each time. After $N$ iterations, therefore,
		exactly $1$ subtree remains which connects every node in the original
		routing tree without traversing faulty links.
		
		TODO: EXPLAIN THE FIDDLINESS HERE TO ENSURE WE DON'T CREATE LOOPS.
		
	\section{Evaluation \& Results}
		
		The PGS repair technique, by design, is able to work around all possible
		fault patterns which don't completely disconnect parts of the network. This
		result this evaluation focuses on the impact on performance PGS repair
		imposes. The metrics of interest in this evaluation are:
		
		\begin{itemize}
			\item Algorithm runtime
			\item Network congestion
			\item Routing table utilisation
		\end{itemize}
		
		\subsection{Traffic Patterns}
			
			In this evaluation, two standard benchmark multicast traffic patterns are
			used which have been used in previous research into SpiNNaker's network:
			
			\begin{figure}
				\center
				\buildfig{figures/traffic-distribution-centroids.tex}
				
				\caption{An example 4-centroid distribution with four centroids. The
				$\times$ marks the location of the origin node. Lighter colours
				indicate greater likelihood of a connection.}
				\label{fig:traffic-distribution-centroids}
			\end{figure}
			
			\begin{description}
				
				\item[Uniform] Destinations are chosen with uniform probability
				anywhere in the machine.
				
				\item[$N$-Centroids] Destinations are clustered around one of $N$
				randomly chosen `centroids' as illustrated in figure
				\ref{fig:traffic-distribution-centroids}.
				
			\end{description}
			
			The uniform traffic pattern is widely used in networks research
			\cite{dally04,davies12} while the centroids model was developed
			specifically to reproduce the traffic patterns found in the neural
			applications SpiNNaker is designed for \cite{navaridas14}. In this work
			we consider 3 centroids.
		
		\subsection{Fault model}
			
			In addition two different fault models are used which are representative of
			the faults found in real SpiNNaker systems:
			
			\begin{figure}
				\center
				\begin{subfigure}{0.48\linewidth}
					\hspace*{-1.5cm}
					\buildfig{figures/fault-example-uniform.tex}
					
					\caption{Uniform}
					\label{fig:fault-example-uniform}
				\end{subfigure}
				\begin{subfigure}{0.48\linewidth}
					\hspace*{-1.5cm}
					\buildfig{figures/fault-example-hss.tex}
					
					\caption{HSS Link}
					\label{fig:fault-example-hss}
				\end{subfigure}
				
				\caption{The two link fault models considered.}
				\label{fig:fault-example}
			\end{figure}
			
			\begin{description}
				
				\item[Uniform] Links are selected and disabled at random (figure
				\ref{fig:fault-example-uniform}).
				
				\item[HSS Link] Groups of links corresponding with randomly selected
				single High-Speed Serial (HSS) link between SpiNNaker boards are disabled
				together (figure \ref{fig:fault-example-uniform}).
				
			\end{description}
			
			The uniform link failure model models isolated failures resulting from
			isolated manufacturing defects in individual links. The HSS Link failure
			model models faults arising from failing or disconnected board-to-board
			links which carry several chip-to-chip traffic flows via a single cable in
			SpiNNaker systems. Though SpiNNaker-specific, the later fault model is
			analogous to failure modes arising in other architectures where a single
			fault may render several links impassable in a single area.
			
			A range of failure rates are explored in this section. My measurements of
			current large-scale SpiNNaker installations the link failure rate is about
			\SI{0.03}{\percent} with failures due to both individual chip-to-chip links
			and board-to-board HSS links. Exact link failure statistics for commercial
			super computer installations are not widely available, however, published
			Mean-Time-Between-Failure (MTBF) statistics place an upper bound on link
			failure rates at a similar \SI{0.03}{\percent} in one-year-old BlueGene/Q
			systems \cite{chiu11}.
			
			Unfortunately presently undiagnosed problem with the SDRAM packaged with
			approximately \SI{1}{\percent} of SpiNNaker chips has rendered these chips
			unusable for most applications. The gaps in the network resulting from the
			loss of these chips currently dominate true link failures leaving just over
			\SI{1}{\percent} of links inoperable.
			
			Surprisingly, research into fault tolerant routing in super computers
			appears to focus on benchmarks with even higher fault rates ranging from
			\SI{3}{\percent} to as high as \SI{7}{\percent}
			\cite{ho04,gomez04,mejia06}.
			
			In this evaluation, fault rates ranging from \SI{0.01}{\percent} to
			\SI{5}{\percent} are considered to cover both realistic fault levels
			along with the more extreme cases considered in related work.
		
		\subsection{Base routing algorithm}
			
			Since the PGS repair process is routing algorithm agnostic all
			experiments use the NER algorithm which has been found to be appropriate
			for SpiNNaker applications \cite{navaridas14}.
		
		\subsection{Algorithm runtime}
			
			To assess the impact of the PGS repair process on routing algorithm
			runtime, the algorithm was used to process a large number of randomly
			generated routing problems and the runtime recorded.
			
			\num{10000} one-to-sixteen multicast routing problems were generated in a
			$256\times256$ hexagonal torus topology, the largest size possible for a
			SpiNNaker system. Other quantities of multicast destinations were also
			evaluated but are omitted for brevity since the pattern of results are
			similar to those outlined here.
			
			TODO: APPENDIX WITH OTHER RUNS?
			
			The NER and PGS repair algorithms were written in C and compiled with GCC
			4.8.3 with \verb|-O2| level optimisations and executed on a cluster of
			idle workstations with 3.10 GHz Intel Core-i5-2400 CPUs.
			
			\begin{figure}
				\center
				\buildrplot{figures/routing-runtimes.R}
				
				\caption{Mean runtime of routing and PGS repair overhead. PGS repair
				overhead is stacked above the routing runtime (i.e. bars do not
				overlap). Error bars indicate 95\% confidence interval. Note different
				Y-scale for HSS link and uniform fault models.}
				\label{fig:routing-runtimes}
			\end{figure}
			
			Figure \ref{fig:routing-runtimes} shows the average runtimes recorded for
			both the NER and PGS repair algorithms. In fault-free networks the
			PGS-repair post-processing step is not required and incurs no penalty
			while the runtime of the algorithm grows with the fault rate for both
			fault and traffic models.
			
			Notably the HSS fault model results in longer runtimes for the PGS repair
			process compared with an equivalent fault-density of uniform faults.
			Because the HSS fault model produces contiguous lines of faults the PGS
			repair algorithm must construct a longer path to avoid the fault.  Since
			the space explored by a graph algorithm typically grows with $O(H^2)$
			with respect to the hops in the discovered route, $H$, this increase in
			search distance has a large impact on the runtime of the PGS repair
			process.
			
			The runtime of the PGS repair algorithm remains roughly in proportion to
			the runtime of the underlying routing algorithm with respect to different
			traffic models. The centroid traffic pattern tends to result in routes
			with fewer hops than a uniform traffic pattern with the same number of
			destination nodes as segments of routes are often shared between
			destination nodes. Since the NER algorithm's runtime is strongly related
			to the number of hops in the output route the runtime of the algorithm is
			greater for uniform traffic. Likewise the probability of PGS repair being
			required increases with the number of hops in route and hence the runtime
			of the PGS repair algorithm increases roughly in proportion.
		
		\subsection{Routing table usage}
			
			In order to gain a realistic measure of routing table usage it is
			necessary to determine the effect of many routes being generated for a
			single set of faults. To enable a sufficiently large number of sample to
			be collected the experimental setup considered previously is reduced to a
			network containing $48\times48$ nodes.
			
			\num{1000} $48\times48$ node network models are produced according to the
			HSS link and uniform fault models. For each of these models
			$48\times48\times16=$~\num{36864} one-to-sixteen routes are generated using
			the centroid and uniform traffic models. This corresponds to one
			multicast route per application core. As is convention in SpiNNaker,
			routing table entries are inserted for each route at the source of the
			route, at each destination and at each corner or fork. The number of
			routing table entries at each node in the model is counted and the
			maximum number of entries in a single node is reported for each network
			model.  The \emph{maximum} number of routing entries of any router was
			chosen since the number of entries available per SpiNNaker router is
			bounded by hardware.
			
			\begin{figure}
				\center
				\buildrplot{figures/routing-entries.R}
				
				\caption{Violin plot showing the distribution of maximum table sizes
				for \num{1000} random networks. The red line at \num{1024} entries
				indicates the size of SpiNNaker's routing tables.}
				\label{fig:routing-entries}
			\end{figure}
			
			
			Figure \ref{fig:routing-entries} shows the distributions of the largest
			routing table sizes for each fault and traffic model.
			
			\begin{figure}
				\center
				\begin{subfigure}{0.48\linewidth}
					\center
					\buildfig{figures/hss-link-routing-table-usage.tex}
					
					\caption{Routing table entries}
					\label{fig:hss-link-routing-table-usage}
				\end{subfigure}
				\begin{subfigure}{0.48\linewidth}
					\center
					\buildfig{figures/hss-link-resource-usage.tex}
					
					\caption{Routes passing through chip}
					\label{fig:hss-link-resource-usage}
				\end{subfigure}
				
				\caption{The impact of a HSS link fault on routing table usage and
				congestion. Each hexagon represents a single chip, the red line
				indicates the chip-to-chip connections broken by the HSS link fault.}
				\label{fig:hss-link-usage}
			\end{figure}
			
			The HSS link failure model has a much greater impact on peak routing
			table resource usage than uniform link failures for a given fault rate.
			This is because HSS link faults result in a large concentration of routes
			being disrupted and then re-routed around the same obstacle in a single
			location. Figure \ref{fig:hss-link-routing-table-usage} shows how routing
			table usage varies around a HSS link fault in one instance of the
			experiment. There are clear peaks in routing table usage around the ends
			of the line of faults which result from routes produced by PGS repair
			finding shortest paths around the edge of the faults.
		
		\subsection{Network congestion}
			
			To measure the impact of PGS repair on network congestion, two
			experiments were performed, one using the same model used to measure
			routing table usage and one based on tests run on SpiNNaker hardware.
			
			For each of the network fault and traffic pattern described previously,
			the paths taken for the \num{36864} one-to-sixteen multicast routes
			generated are used to compute the number of times each link in the
			network is used. The number of routes passing through the most-used link
			is then recorded, giving an indication of the level of congestion in the
			network.
			
			\begin{figure}
				\center
				\buildrplot{figures/routing-resource.R}
				
				\caption{Violin plot showing the distribution of maximum
				routes-per-chip for \num{1000} random networks.}
				\label{fig:routing-resource}
			\end{figure}
			
			The results are presented in figure \ref{fig:routing-resource} and follow
			the same trends as the results previously shown for routing table usage.
			Again, HSS link faults result in routes with the greatest congestion due
			to the concentration of routes finding shortest paths around an obstacle
			(see \ref{fig:hss-link-resource-usage}).
			
			To verify that the results above, an additional experiment has been
			carried out which attempts to mimic the model used previously in actual
			SpiNNaker hardware. In these experiments a large SpiNNaker machine is
			divided into independent 48-board (2304-chip) sections. Because the
			48-board systems used in these experiments are cut out of a larger
			machine, they lack wrap-around links and thus form hexagonal mesh
			topologies, rather than hexagonal toruses.
			
			Due to the SDRAM issue described above, fault rates below
			\SI{1}{\percent} cannot be modelled.  To simulate higher fault rates,
			additional links are disabled in software according to the fault models
			described used previously. Since some faults are due to genuine hardware
			faults, these faults cannot be placed randomly in each experiment. To
			reduce, bias each combination of fault rate, fault model and traffic
			pattern is repeated XXX times across randomly chosen physical machines.
			
			XXX 1-to-XXX routes are generated in both uniform and XXX-centroid
			distributions as used throughout this evaluation. Synthetic network
			traffic is generated at the source of each route following a Bernoulli
			distribution. Traffic consumers running on all destination nodes accept
			packets as quickly as possible from the network and log their arrival.
			The Bernoulli probability is set the same for every route's traffic
			generator and increased in steps of XXX and the number of packets dropped
			in an XXX second period logged. The network is considered saturated once
			less than \SI{99}{\percent} of packets successfully arrive at their
			destination.
			
			Figure \ref{XXX} shows the distributions of the saturation points for
			each experimental configuration.
			
			TODO: ANALYSIS
		
	\section{Conclusions}
		
		In this chapter I described how SpiNNaker's unconventional network and
		router architecture render existing fault tolerant routing algorithms
		unsuitable. I introduced PGS repair, a post-processing technique for
		existing non-fault tolerant routing algorithms designed for SpiNNaker such
		as NER.
		
		Unlike some other fault tolerant routing algorithms for other
		architectures, PGS repair is able to work-around arbitrary fault patterns
		by exploiting SpiNNaker's inbuilt deadlock avoidance mechanisms. In the
		presence of realistic failure rates of up to \SI{1}{\percent}, only small
		overheads of up to XXX, XXX and XXX for in algorithm runtime, routing table
		usage and network performance are incurred respectively. This low
		performance overhead makes PGS repair appropriate for use in real
		applications. At the time of writing the algorithm has been successfully
		used in a number of neural and non-neural SpiNNaker applications.
		
		At more extreme fault rates not expected in real-world systems, the
		algorithm still functions correctly but the results incur much greater
		routing table and congestion overheads, particularly when faults are
		concentrated. Future extensions to this algorithm might aim to reduce this
		overhead by producing longer and more varied routes around faults to even
		out the load.

	\chapter{Placing applications in large SpiNNaker machines}
	
	In the previous chapter I tackled the problem of scale in generating routes
	for very large networks such as SpiNNaker. In this work the centroid traffic
	pattern was used as an approximation of the expected network traffic
	generated by `well behaved' neural network simulation software running on
	SpiNNaker. The traffic produced largely exhibits strong locality, that is
	most communication occurs between either nearby nodes or clusters of nodes.
	In reality, neural simulation applications are not specified geometrically
	but rather as abstract graphs of communicating neurons
	\cite{davison08,eliasmith13}. Applications must then \emph{place} these
	neurons onto nodes in a SpiNNaker system, attempting maximise communication
	locality.
	
	In this chapter I re-evaluate the suitability of simulated annealing as a
	technique for finding high quality placements for large parallel
	applications. Though this technique had fallen out of fashion in the field of
	application placement by the early 1990s, it has found wide use for placing
	components in computer chip and FPGA designs. In the intervening years,
	placement problems in super computers have grown in size from tens or
	hundreds of nodes to millions, a scale at which chip placement techniques
	were operating in the mid 1990s. I adapt the simulated annealing algorithm
	used by the VPR academic circuit placement software to produce placements for
	applications running on SpiNNaker. In that in a range of real and synthetic
	benchmarks simulated annealing produces high quality placements enabling
	efficient use of SpiNNaker's network resources.
	
	
	%In the field of chip design, Moore's `Law' \cite{moore65,moore75} observes a
	%similar exponential growth in the number of components within a single chip.
	%Today modern processors contain billions of components and an analagous
	%placement problem exists in attempting to place interconnected components
	%near to eachother. In this chapter I explore the techniques used for circuit
	%placement and adapt one such technique, Simulated Annealing (SA)
	%\cite{kirkpatrick83}, for use in application placement. Despite some early
	%interest in SA for application placement in the 1980s and early 1990s, the
	%technique has since fallen out of favour. I find that at the scales of modern
	%placement problems SA-based placement is able to produce solutions of
	%superiour quality to contemporary methods.
	%
	%TODO: SUMMARISE RESULTS...
	
	\section{Related work}
		
		The placement problem has been tackled independently in the literature by
		researchers in both the application and chip placement communities. In this
		survey I cover application and chip placement separately as these two
		communities have remained largely isolated from one another. First I
		explore the techniques applied to application placement before moving on to
		contrast this with the techniques used in circuit placement.
		
		In the application placement literature, the placement problem is often
		referred under the umbrella term `mapping'. Unfortunately term is often
		used more broadly to include other tasks such as routing and application
		partitioning. To avoid ambiguity I use the term `placement', as preferred
		by the chip and FPGA design communities, to refer specifically to the
		problem of assigning nodes in an application's communication graph to nodes
		in a machine's connectivity graph.
		
		\subsection{Application placement algorithms}
			
			TODO: GENERAL INTRO
			
			\subsubsection{Application-specific approaches (manual placement)}
				
				In the case of some applications such as finite element modelling
				\cite{bermejo13}, the structure of the problem itself leads to a
				natural placement of the computation on nodes in a machine. For example
				when simulating a 3D volume in an node super computer with a $3 \times
				4 \times 2$ 3D torus or mesh topology network, the modelled volume
				might be divided into as in figure \ref{fig:fem-partitioning}. Each
				cuboid in the model is then assigned to the corresponding node in the
				network topology.
				
				\begin{figure}
					\center
					\buildfig{figures/fem-partitioning.tex}
					
					\caption{Example partitioning of a 3D space to fit into a super
					computer with a $3\times4\times2$ torus or mesh topology.}
					\label{fig:fem-partitioning}
				\end{figure}
				
				When the number of dimensions in a problem do not match that of the
				underlying network architecture, the common solution is to either
				divide only along a subset of the axes or to divide into additional
				pieces on the existing axes \cite{gilge14}.
			
			\subsubsection{Sequential placement}
				
				In the case where a placement solution is non-obvious one of the
				simplest and most popular strategies is to apply a simple sequential
				placement algorithm. Sequential placement algorithms function by
				iterating over the vertices in the application's communication graph
				and assigning them to a free node in the target machine. Sequential
				placement algorithms are differentiated by the order in which they
				iterate over vertices in the communication graph and fill nodes in the
				target machine. A number of widely used orderings are described below.
				
				\begin{figure}
					\center
					\begin{subfigure}{0.32\linewidth}
						\center
						\buildfig{figures/sequential-row-order.tex}
						\caption{Row-order}
						\label{fig:sequential-row-order}
					\end{subfigure}
					\begin{subfigure}{0.32\linewidth}
						\center
						\buildfig{figures/sequential-alternating.tex}
						\caption{Alternating}
						\label{fig:sequential-alternating}
					\end{subfigure}
					\begin{subfigure}{0.32\linewidth}
						\center
						\buildfig{figures/sequential-hilbert.tex}
						\caption{Hilbert curve}
						\label{fig:sequential-hilbert}
					\end{subfigure}
					
					\caption{Space-filling curves in 2D mesh and torus topologies.}
					\label{fig:sequential}
				\end{figure}
				
				Super computer management software such as SLURM \cite{yoo03} and Blue
				Gene's system software \cite{gilge14} by default na\"ively iterate over
				vertices in an application communication graph in the order they are
				provided. The nodes in the target machine are then iterated over in a
				simple space-filling curve through the network topology. Figure
				\ref{fig:hilbert-placement} illustrates the default patterns available
				in these software packages. The row-order (figure
				\ref{fig:sequential-row-order}) and alternating (figure
				\ref{fig:sequential-alternating}) curves correspond with 2D versions of
				the default node assignment orders used in SLURM and BlueGene systems.
				
				\begin{figure}
					\center
					\buildfig{figures/hilbert-placement.tex}
					
					\caption{A Hilbert curve, coloured from blue to red.}
					\label{fig:hilbert-placement}
				\end{figure}
				
				The Cray extensions to SLURM software provide a Hilbert curve
				\cite{hilbert91} (figure \ref{fig:sequential-hilbert}) node assignment
				order. Unlike the row-order and alternating space filling curves the
				Hilbert curve ensures that pairs of vertices close together in the node
				iteration order are also close together in the target machine's network
				\cite{moon01, zumbusch99}. Figure \ref{fig:hilbert-placement} shows a
				5$^\textrm{th}$-order Hilbert curve where each point in the curve is
				coloured according to its position along the curve. In this figure it
				is possible to see that nearby positions in the curve (which share
				similar colours) are also close in 2D space.
				
				When the proximity of vertices in the vertex-ordering supplied by an
				application is a good estimator of those vertices communication
				requirements, the sequential assignment schemes discussed above can be
				very effective. These techniques have also proven adequate in
				small-scale and densely connected applications such as early neural
				simulations running on prototype SpiNNaker machines with tens of nodes
				\cite{galluppi10} but growing beyond this scale has proven problematic.
				
				\begin{figure}
					\center
					\begin{subfigure}{0.45\linewidth}
						\center
						\buildfig{figures/rcm-initial.tex}
						
						\caption{Original permutation}
						\label{fig:rcm-initial}
					\end{subfigure}
					\begin{subfigure}{0.45\linewidth}
						\center
						\buildfig{figures/rcm-sorted.tex}
						
						\caption{RCM permutation}
						\label{fig:rcm-sorted}
					\end{subfigure}
					
					\caption{Adjacency matrix representation of a graph before and after
					permutation by the RCM algorithm.}
					\label{fig:rcm}
				\end{figure}
				
				A number of algorithms have been proposed for automatically selecting
				good vertex iteration orders, typically using a graph-traversal based
				heuristic. A typical method, described by Hoefler \emph{et al.}
				\cite{hoefler11} exploits the Reverse-Cuthill-McKee (RCM) algorithm
				\cite{cuthill69}. An application's communication matrix is represented
				as an adjacency matrix, $M$, where $M_{i,j}$ is 1 if node $i$ is
				connected by an edge to node $j$ and 0 otherwise. An example matrix is
				illustrated in figure \ref{fig:rcm-initial}. The RCM algorithm uses a
				simple heuristic to permute the matrix (i.e. renumber the nodes in the
				graph) in order to reduce the bandwidth of the matrix. Figure
				\ref{fig:rcm-sorted} shows the RCM-permuted version of the example
				adjacency matrix. When a graph's vertices are ordered as in a
				bandwidth-reduced sparse matrix, vertices close together in the
				ordering are likely to communicate while those further apart tend not
				to communicate.
				
			\subsubsection{Optimisation-based Placement}
				
				% Citations from short report about optimisation in placement...
				% \cite{chen06,jeannot14} and \cite{jeannot10} ("subsets of apps")
				
				In the academic community, a number of attempts have been made to use
				more sophisticated optimisation algorithms for the placement of
				applications. In 1985, Steele \cite{steele85} proposed the use of
				simulated annealing for placing applications in the 6D torus topology
				of the 64 node `Caltech Cosmic Cube' machine. Simulated annealing,
				originally developed by Kirkpatrick \emph{et al.} \cite{kirkpatrick83},
				is a general-purpose optimisation algorithm which works by analogy to
				the physical process of annealing. In brief simulated annealing
				functions by randomly swapping vertices in a candidate placement
				solution, accepting swaps which move connected vertices closer together
				and rejecting some proportion of swaps which move connected vertices
				further apart. The simulated annealing algorithm is described in detail
				later in this chapter.
				
				Towards the end of the 1980s, application placement appeared to be
				becoming less important as super computer network architectures
				improved:
				%
				\begin{displayquote}
					``Careful placement was necessary because of the slow communication
					and non-uniform addressing of early concurrent computers. However,
					the development of message passing machines with fast communications
					and a uniform global address space  has made placement less of an
					issue. In such machines a random placement performs nearly as well as
					an optimum placement.''
					
					\noindent --- W. Dally, 1987 \cite{dally87}
				\end{displayquote}
				%
				In addition, network and problem sizes remained small, so small in fact
				that linear-programming based optimal placement still appeared in
				benchmarks comparing placement algorithms \cite{xu91}. In this
				environment, simpler sequential placement algorithms gained favour over
				more computationally expensive algorithms such as simulated annealing.
				
				As problem and machine sizes have grown and network utilisation has
				once again become an important factor in application performance
				\cite{navaridas09b} more complex optimisation algorithms have
				reappeared in the literature. One popular approach employs graph
				partitioning algorithms such as METIS \cite{karypis98} to perform
				recursive bipartitioning based placement
				\cite{phillips14,hoefler11,pellegrini96}.  This placement process is
				illustrated in figure \ref{fig:partitioning}.
				
				In the first step, the application communication graph and machine
				connectivity graph are bipartitioned such that the number of edges
				between partitions is minimised. Each half of the communication graph
				is associated with one of the halves of the machine connectivity graph.
				The partitioning process is then repeated recursively on each of the
				two communication and connectivity graph pairs. The process halts when
				the graphs can no longer be partitioned at which point the vertices in
				the communication graph are placed on their associated node.
				
				\begin{figure}
					\center
					\buildfig{figures/partitioning.tex}
					
					\caption{Illustration of application placement by recursive
					partitioning.}
					\label{fig:partitioning}
				\end{figure}
				
				TODO: PARTITIONING IS GREAT AND ALL BUT QUALITY ISN'T ALWAYS GREAT AND
				IT DOESN'T DEAL WELL WITH MULTI-CONSTRAINT SCENARIOS E.G. PROCESSOR AND
				MEMORY RESTRICTIONS.
				
				Unfortunately, many of these simply aren't suited to the scale of
				neural applications running on SpiNNaker (e.g. only cope with tens of
				nodes while SpiNNaker may contain hundreds of thousands).
				
				Additionally, a number of algorithms have been developed which make
				assumptions about the topologies of the problem or network. Tree match
				for example attempts to map tree-shaped problems to tree-shaped
				networks. Such algorithms can be highly effective but again do not
				apply to SpiNNaker or its neural applications.
		
		\subsection{Chip placement algorithms}
			
			The chip-design industry has, for many years, dealt with problems
			analogous to the task of placing super computer jobs in a way suited to
			SpiNNaker. Modern CPUs have millions or billions of components with
			strictly fixed connectivity. CPU designers must place each of these onto
			a chip such that the connection lengths are controlled to reduce
			congestion and increase performance. As such, these algorithms are
			ideally suited to future super computer placement work since they already
			operate at the scales required \cite{nam07}.
			
			\subsubsection{Cost functions}
				
				HPWL is popular but a bit crap for high fan-outs. It is, however, quite
				simple.
				
				TODO: SELECT A BETTER COST FUNCTION...
			
			\subsubsection{Simulated annealing}
				
				One of the oldest techniques used for circuit placement is simulated
				annealing and this remains popular today thanks to its sheer
				versatility (see VPR, other open FPGA tools).
				
				SA works by analogy with the physical process of annealing.
				The simulated annealing algorithm works by selecting random pairs of
				components on a chip, swapping them and evaluating some cost function.
				If the swap reduces the cost function, it is kept, if not, depending on
				a function of the current temperature and the cost introduced by the
				swap.
				
				TODO: ILLUSTRATION OF SIMULATED ANNEALING SWAP OPERATION
				
				By occasionally allowing costly swaps, the annealing algorithm avoids
				becoming trapped in local minima. As the algorithm proceeds, the
				temperature is slowly reduced and with it the proportion of costly
				swaps which are retained. This causes the placement to move from
				exploration early on towards refinement later on.
				
				The temperature schedule of an annealing algorithm is critical to its
				success. In general these schedules are computed based on the
				performance of the algorithm as it runs. In VPR the following schedule
				is used.
				
				TODO: DESCRIBE VPR'S SCHEDULE
				
				TODO: FIND AND DESCRIBE ALTERNATIVE SCHEDULE?
				
				Unfortunately, SA is very difficult to parallelise, especially in the
				case of placement. As a result, its scalability has been limited and
				resulted in significantly reduced usage in recent work.
			
			\subsubsection{Partitioning placement}
				
				Partitioning based placement solves the placement problem using
				graph-partitioning recursively on the problem graph to assign each part
				of the circuit to some area in the super chip. Though a number of
				algorithms have proven successful in academic placement contests over
				the years, they are not popular in industrial settings.
			
			\subsubsection{Analytical placement}
				
				In analytical placement, cost function for the circuit graph is
				approximated in a form which is amenable to solutions with standard
				numerical or symbolic algebraic techniques. Using these techniques,
				exact minimum cost (in terms of the approximation) configurations can
				be obtained.
				
				Quadratic placement is a popular analytical placement technique which
				approximates the cost of a placement as the sum of the squares of the
				distances between connected circuit elements.
				
				TODO: FIGURE EXAMPLE QUADRATIC PLACEMENT PROBLEM AND SOLUTION
				
				As such this gives a quadratic cost function like so which we must
				minimise.
				
				TODO: QUADRATIC COST EQN
				
				To minimise the function we differentiate and solve using simple
				symbolic manipulation.
				
				TODO: QUADRATIC COST SOLUTION
				
				Unfortunately, quadratic placement doesn't contain any congestion
				relief by default so various schemes exist. For example, extra anchor
				nodes are inserted which gently pull the circuit components apart from
				each other. As a result, the algorithm generally proceeds by iterating,
				regenerating anchors each time.
				
				Other non-quadratic analytical methods exist too with numerical
				solutions. The approaches are often similar.
			
			\subsubsection{Hierarchical clustering}
				
				Many placement algorithms scale super-linearly with problem size and so
				larger problems become increasingly problematic to handle. To solve
				this problem clustering techniques are first applied to first simplify
				the placement problem. A solution is then found at the coarse level and
				then hierarchically fleshed out.
				
				Various clustering algorithms are in use.
				
				TODO: TALK ABOUT CLUSTERING IN PLACEMENT...
				
				TODO: DESCRIBE THE ALGORITHM I IMPLEMENTED.
	
	\section{Application placement by simulated annealing}
		
		\label{sec:placement-by-annealing}	
		
		I have implemented a simplified SA based application placement algorithm
		based on the approach used in the popular VPR place and route tool chain.
		The algorithm is written in C and is optimised for experimentation rather
		than performance but is production-ready. It has been integrated into the
		`Rig' SpiNNaker software tools and has been used to place very large
		simulations. More on that later.
		
		\subsection{Representation}
			
			Model each chip as having a quantity of various resources (e.g. Cores,
			SDRAM) available. The application graph consists of vertices which each
			consume some quantity of these resources. Vertices must be placed on a
			single chip such that the resources required on a given chip do not
			exceed those available. Vertices are then interconnected by 1:N nets with
			weights which act as hints. The nets are treated as a soft constraint:
			vertices connected via a net will, ideally, be placed near to each other,
			with priority being given to nets with higher weights. Additionally there
			will be a list of placement constraints (see later).
			
			A key observation is that while vertices in an application may frequently
			have a 1:1 correspondence with application cores, this need-not be the
			case. For example, a vertex may represent a block of SDRAM which is
			shared. A vertex may also represent some other resource, for example,
			external IO availability. By making these resource types user-defined,
			applications programmers can express flexible hard-constraints on their
			application.
			
			Another observation is that generic soft constraints can be expressed may
			be expressed using a net with an appropriate weight.
			
			As a result of these facilities, application programmers can easily
			express their own application-specific hard and soft placement
			constraints without having to modify the algorithm. This representation
			has become a de-facto standard for placement problem interchange for
			SpiNNaker applications.
		
		\subsection{Cost function}
			
			At present I've used HPWL despite this being really bad for high-fan-out
			multicast and totally ignorant to the hexagonal nature of SpiNNaker...
			
			To compute bounding boxes for tori I use the following approach. For each
			dimension, sort the points on that dimension and find the largest gap
			between them on a ring. The bounding box goes the other way.
			
			TODO: FIGURE ILLUSTRATING BOUNDING BOX COMPUTATION FOR TORI.
		
		\subsection{Annealing schedule}
			
			The annealing schedule is that used by VPR. Despite being for circuit
			placement, it seems to work jolly well.
			
			TODO: DESCRIBE AND RATIONALISE THE SCHEDULE
		
		\subsection{Constraint handling}
			
			Various hard and soft constraints may be expressed by software
			approaches. For each we explain how they may be handled by the placement
			algorithm:
			
			\subsubsection{Location Constraint}
				
				The vertex is placed on a chip and removed from the set of movement
				candidates.
			
			\subsubsection{Same-chip constraint}
				
				When two vertices must always be placed on the same chip they are
				simply combined into one vertex which consumes the sum of their
				resources. Placement then treats them as one chip and thus is forced to
				atomically place the vertices.
			
			\subsubsection{Reserve resource constraint}
				
				Simply reduce resource availability on that chip.
			
			\subsubsection{Keep near Ethernet}
				
				Simply add a net.
	
	\section{Evaluation}
		
		\label{sec:placement-results}
		
		Though benchmarks exist for super computer loads and chip placement tasks,
		such things don't exist for neural applications. As a result I use a
		selection of real applications for SpiNNaker along with some synthetic
		benchmarks based on biological data.
		
		\subsection{Benchmark networks}
			
			First some real networks.
			
			Some nengo networks: SPAUN: `The world's largest functional brain model'.
			Word-net network from Jamie: Example of some learning.
			
			TODO: DESCRIBE SHAPE OF NENGO NETWORKS
			
			Some PyNN networks: Microcortical column model from PyNN. Note almost
			broadcast connectivity but varying weights. Try and extract a vision
			netlist from Anna. Maybe try and get a netlist for Tom's barrel cortex.
			
			Now for some artificial networks. Pipeline, noisy pipeline, mesh,
			Gaussian 2D.
		
		\subsection{Experiments}
			
			We compare random, linear, greedy and annealing based placement
			approaches to placement. We compare static metrics (such as mean/max
			congestion, table usage) along with experiments based on simulated
			network traffic in real hardware. Network Tester generates artificial
			traffic in proportion with the weights given for each model. We compare
			the relative level of traffic sustainable. We also consider use of
			machines of various sizes.
		
		\subsection{Results}
			
			SA placement is slow but rather effective, especially for some networks.
			Generally worth doing. Will need to be sped up for very large machines...
			
			TODO: EXPERIMENTS!
	

	\chapter{Discussion}

\section{Suitability of the hexagonal torus topology}
	\subsection{Physical scalability}
	\subsection{Routability}
	\subsection{Placeability}

\section{Suitability of the SpiNNaker router}
	\subsection{Deadlock avoidance}
	\subsection{Routing table size}

\section{Suitability of circuit placers for application placement}
	\subsection{Quality}
	\subsection{Runtime}
	\subsection{Routing resources}
	\subsection{Flexibility}
	\subsection{Scalability}


	\chapter{Future research}
	
	In this thesis I have presented a number of new techniques which have made it
	possible to assemble and operate the SpiNNaker super computer. This work
	opens up a range of possibie lines of research to extend this work to future
	architectures and applications. In this chapter I focus on two anticipated
	challenges of future systems: growing scale and greater dynamicism in
	applications.
	
	\section{Scaling up}
		
		TODO: INTRO
		
		\subsection{Grid machine room layouts}
			
			In chapter XXX, I developed a machine room layout for hexagonal torus
			topologies which allowed machines occupying a row of standard
			machine-room cabinets to scale up without the need for long
			interconnecting cables. For larger installations, however, it will be
			necessary to employ multiple rows of cabinets in a 2D arrangement.
		
		\subsection{Routing congestion control}
		
		\subsection{Parallel place and route}
	
	\section{Structural plasticity and dynamic fault-tolerance}
		\subsection{Plasticity models}
		\subsection{Incremental placement}
		\subsection{Incremental routing}
		\subsection{Hot-spare routes}

	\chapter{Conclusions and future research}
	
	The SpiNNaker architecture was designed to tackle the challenges presented by
	the simulation of biologically realistic neural networks. One of its
	distinguishing features is its network architecture which employs both an
	unconventional network topology and multicast router architecture. The
	hexagonal torus topology used by SpiNNaker was chosen to enable greater
	performance while maintaining ease of construction and scalability compared
	with conventional network topologies. SpiNNaker's router design centres
	around packets mimicking the neural `spike' signals they are designed to
	convey by being small, multicast and not guaranteed to arrive at their
	destination.  This novel design, though largely complete before this work
	began, left a number of open problems which this thesis has attempted to
	address.
	
	In this concluding chapter I begin by summarising the answers to the research
	questions raised in chapter~\ref{sec:introduction}. This is followed by a
	discussion of new research topics which have been uncovered by this work.
	
	\section{Answers to research questions}
		
		Each of the three research questions are answered below.
		
		\subsubsection{1. Can the hexagonal torus topology be deployed and used in
		real, large-scale systems?}
		
		In chapter~\ref{sec:building}, I introduced a cabling scheme and assembly
		technique which has been used successfully to build a prototype SpiNNaker
		system with over half a million processor cores using the hexagonal torus
		topology. The techniques shown are expected to enable a final SpiNNaker
		machine of double this size to be built, filling a six metre long row of
		machine-room cabinets.
		
		Though SpiNNaker's processor-count places it amongst some of the world's
		largest supercomputers (see figure \ref{fig:top500-num-processors} on page
		\pageref{fig:top500-num-processors}), it is comparatively compact, filling
		one row of cabinets compared with the warehouse-scale installations found
		in commercial systems. In spite of this, the folding and interleaving
		techniques described allow hexagonal torus topologies to scale to
		arbitrarily large installations without cables which span the machine.
		
		Chapter~\ref{sec:shortestPaths} described an efficient and general
		technique for finding, and enumerating shortest path vectors in hexagonal
		torus topologies. These developments bring the hexagonal torus topology in
		line with other topologies by enabling routing algorithms to exploit all
		possible paths in a network. Further, chapter~\ref{sec:placement}
		demonstrated that placement algorithms are also adaptable to hexagonal
		torus topologies thanks to their similarity to 2D toruses.
		
		Though, as this thesis highlights, hexagonal toruses lack many of the
		intuitive properties enjoyed by other topologies, it is still possible to
		reason about them with only limited computational effort.  Now that the
		practicality and scalability of the topology has also been demonstrated in
		practice, it represents a credible option for future network architectures.
		
		\subsubsection{2. Does SpiNNaker's router architecture help, or hinder
		fault tolerance?}
		
		SpiNNaker's unconventional use of packet dropping to avoid deadlocks
		greatly simplifies the router architecture, part of the motivation for this
		design. In chapter~\ref{sec:routing} this feature is used to the advantage
		of PGS repair to add fault tolerance to existing routing algorithms.
		Compared with the often complex and wasteful methods used to tolerate
		faults in other networks, PGS repair incurs very little performance
		overhead in the presence of static faults.
		
		Routing table usage does increase in the presence of faults, however, which
		may be a concern for applications which already require many routing table
		entries. Routing table usage, as well as other overheads, were most
		significantly increased in the presence of contiguous groups of network
		faults. This is because the PGS repair algorithm produces routes which pass
		tightly around the corners of faults, resulting in concentrations of
		routing table entries in those areas.  Though the symptoms of this problem
		can be attributed to the design of SpiNNaker's multicast routing mechanism,
		the responsibility lies with the behaviour of the PGS repair algorithm.
		Potential improvements to the PGS repair algorithm are discussed later in
		\S\ref{sec:pgs-repair-improvements}.
		
		The overall answer to this research question, therefore, is that the
		flexibility provided to routing algorithms in SpiNNaker's architecture is
		of great benefit, enabling arbitrary fault patterns to be inexpensively
		avoided.
		
		\subsubsection{3. How can the parts of a neural simulation be placed onto a
		large hexagonal torus topology to reduce network load?}
		
		In chapter~\ref{sec:placement}, I explored a number of contemporary
		approaches to the problem of placing applications with irregular
		communication patterns into network topologies. I observed that researchers
		working on circuit placement for chips and FPGAs are tackling similar
		problems and working at scales as large, or larger than, those faced in
		application placement. Based on this I developed a
		simulated annealing based placement algorithm inspired by the techniques
		used in circuit placement, with specific adaptations for use in application
		placement and SpiNNaker's network topology.
		
		The simulated annealing based placement algorithm consistently outperforms
		pre-existing placement algorithms included in benchmarks in terms of
		placement quality.  In the case of one benchmark, simulated annealing based
		placement made it possible to run that neural simulation in real-time for
		the first time.  At larger scales, simulated annealing was also found to be
		able to produce good quality placements in benchmarks containing over one
		million processes -- the largest size supported by the SpiNNaker
		architecture.
		
		The major shortcoming of simulated annealing based placement is its
		execution speed. Though its execution time grows in proportion to the size
		of the problem, the implementation used took over 12~hours to place a
		synthetic problem for the largest planned SpiNNaker machine. Though
		tractable -- particularly given the relative output quality compared with
		the prior state-of-the-art -- the algorithm is unlikely to function
		comfortably as-is on larger problems.
		
		The conclusion to be drawn from this result, however, is not just that
		simulated annealing is a good solution for today's placement problems but
		that circuit placement techniques in general could be successfully adapted
		to fulfil this role. The placement problems faced by chip designers are
		growing at roughly the same exponential rate as the size of super computers
		but circuit designs hold the lead in terms of problem size. Consequently,
		as approaches are retired by chip placement researchers, they may find new
		life in the field of application placement.
		
	\section{Future research}
		
		Though the goals of this study have largely been met, there also remain
		some important limitations which future work may hope to address.
		Furthermore, this work has uncovered a number of new research areas
		warranting future enquiry. This section outlines a number of future lines
		of research.
		
		\subsection{Warehouse-scale cabling}
			
			In chapter~\ref{sec:building} I developed and implemented a number of
			cabling schemes for the SpiNNaker architecture spanning up to a six metre
			row of machine-room cabinets -- a relatively small installation by
			current standards. In SpiNNaker, the cabling exists in a 2D plane (i.e.
			across the faces of the cabinets) but as the system is scaled up, a
			single row of cabinets will tend towards a 1D line. Since embedding a 2D
			structure in a 1D space necessarily results in long connections, this
			cannot scale indefinitely.
			
			\begin{figure}
				\center
				\buildfig{figures/multi-row-cabling.tex}
				
				\caption{Multiple rows of interconnected cabinets.}
				\label{fig:multi-row-cabling}
			\end{figure}
			
			In conventional large-scale super computer installations, nodes are
			installed in rows of cabinets as illustrated in
			figure~\ref{fig:multi-row-cabling}.  From a `bird's-eye' view, the system
			approximates a 2D space, spread across the floor of a machine-room.
			Therefore, in principle, the folding and interleaving techniques
			described in chapter~\ref{sec:building} still apply. Unfortunately for
			SpiNNaker, cables connecting between rows of cabinets would be longer
			than the one metre limit imposed by its hardware because of the spacing
			between rows of cabinets.  Future SpiNNaker systems will need to consider
			alternative link technologies.  For example, a hybrid system could be
			used in which intra-cabinet connections continue to use the current HSS
			link technology while inter-cabinet links might use optical connections.
			This type of architecture could be supported by the use of pluggable
			`SFP+' transceiver modules~\cite{sff01}.
		
		\subsection{Cabling assistance for other architectures}
			
			A secondary result of the construction of prototype SpiNNaker systems in
			chapter~\ref{sec:building} was the use of real-time guidance and feedback
			to assist cable installation. I am not aware of this technique's use by
			existing architectures and, following the success experienced in this
			project, it is possible that the technique may also be useful in
			conventional systems.
			
			During the construction of prototype SpiNNaker machines, each cable took
			seconds to install compared with the minutes reported for existing
			systems~\cite{mudigonda11}. Part of this increase in efficiency appears
			to result from the immediate identification of mistakes made during
			cabling, saving time-consuming backtracking later on.
			
			In many real-world network installations, units are less densely packed
			than in SpiNNaker and so longer cables are often required. As a
			consequence, cabling errors may become more likely as cabling patterns
			are spread over a larger area making them more difficult to visually
			verify. Like SpiNNaker, conventional networking hardware is often
			equipped with a generous range of indicator LEDs and diagnostic
			facilities which might be used to implement real-time installation
			guidance. Future work could explore the use of this technique in the
			construction of other large-scale networks, such as data centres.
		
		\subsection{Congestion mitigation}
			
			\label{sec:wiggly-board-allocations}
			
			In chapter~\ref{sec:routing} I found that contiguous network faults cause
			hot-spots of congestion and routing table depletion where the PGS repair
			algorithm routed many paths around the edges of faults.  However, it is
			not just faults which can cause contiguous blockages in the network
			topology. In reality, researchers do not always require a full-sized
			SpiNNaker system to perform their experiments so large SpiNNaker systems
			are soft-partitioned on demand into many smaller
			machines~\cite{spalloc16}. To ensure isolation between partitioned
			sub-machines, HSS links between boards in different partitions are
			disabled. Because of SpiNNaker's `wrapped triple' partitioning scheme,
			the resulting sub-machines have hexagonal \emph{mesh} topologies (i.e.
			without wrap-around links) with irregular boundaries as in
			figure~\ref{fig:spalloc-mesh}.
			
			\begin{figure}
				\center
				\buildfig{figures/spalloc-mesh.tex}
				
				\caption[Irregular edges of a partitioned SpiNNaker system.]%
				{Irregular edges in a SpiNNaker system comprised of 24~boards
				partitioned from a larger machine.  Each hexagon represents a SpiNNaker
				chip. No wrap-around connections are present.}
				\label{fig:spalloc-mesh}
			\end{figure}
			
			In partitioned systems, the `tooth'-like gaps on the periphery of the
			network result in similar congestion to the HSS link failures considered
			in chapter~\ref{sec:routing}. When a route is generated between nodes on
			opposite sides of a gap, the PGS repair process will produce a
			shortest-path route around it. Since many routes may be blocked by a
			single gap, a hot-spot may develop around the corners of the gap.
			
			In chapter~\ref{sec:placement}, the `CConv' benchmark application was
			found to run correctly the majority of the time when placed by the
			simulated annealing algorithm but would occasionally fail by a
			significant margin. Preliminary experiments suggest these occasional
			failures are caused by placement solutions which place heavily
			communicating parts of the application on opposite sides of gaps along
			the perimeter of the network. Two possible approaches which future work
			may consider are presented below.
			
			\subsubsection{Avoiding hotspots with PGS repair}
				
				\label{sec:pgs-repair-improvements}	
				
				Network congestion around faults and network irregularities could be
				reduced by forcing the PGS repair process to take more varied routes
				around faults. For example, in circuit routing algorithms, routers
				avoid congestion by increasing the cost of routes which pass through
				congested areas~\cite{kahng11}. A similar technique could be used in
				PGS repair to spread the routes it produces.
				
				An alternative approach would be to adapt the base routing algorithms
				used prior to PGS repair to, for example, attempt alternative dimension
				order routes which may avoid blockages due to faulty links.
			
			\subsubsection{Fault and irregularity aware placement}
				
				One of the shortcomings of the simulated annealing based placer
				developed in chapter~\ref{sec:placement} is that it does not account
				for network faults, or irregularities, when estimating the cost of
				placement solutions.  Future work may exploit techniques used in
				congestion-aware circuit placement which could be adapted for
				application placement~\cite{viswanathan07}.
		
		\subsection{Reducing placement execution time}
			
			The simulated annealing based placer presented in
			chapter~\ref{sec:placement} produced good quality placements but its
			execution time limits its use beyond one million vertex placement
			problems. Future work should explore possibilities for improving the
			performance and scalability of this technique.
			
			In addition to considering alternative placement algorithms based on
			other methods, one possible approach is to attempt to reduce the execution
			time of simulated annealing based placement by shrinking the application
			graph being placed.
			
			For example, graph clustering~\cite{schaeffer07} may be used to group
			together strongly connected vertices which would then be placed as a
			single unit.  Unfortunately, clustering can suffer from the same problems
			as graph-partitioning-based placement: vertices may be grouped together
			in ways which, in practice, cannot be packed together into a given portion
			of a machine.  A possible solution to this problem is to use a two-phase
			placement approach~\cite{kahng11}. In a `global' placement phase,
			solutions are permitted which can slightly over-allocate resources but
			overall achieve good placement quality. In the `detailed' placement phase
			which follows, the solution is `legalised' by making small changes to the
			global placement to eliminate over allocation.
			
			An alternative approach suited to SpiNNaker could be to limit the
			clustering process to clusters which fit on a single SpiNNaker chip. In
			typical SpiNNaker application graphs, clustering to this level may reduce
			placement problem sizes by an order of magnitude and, consequently,
			reduce execution times by the same ratio. Preliminary experiments suggest
			that this approach might result in little placement quality loss for
			large placement problems whilst substantially reducing overall execution
			time.
		
		\subsection{Benchmarking}
			
			One of the most significant limitations of this study has been the
			unavailability of large-scale SpiNNaker applications for use as
			benchmarks. As a consequence, much of the scalability experimentation
			performed has relied on simple synthetic benchmarks based on projections
			of future application behaviour.
			
			In the short term, more sophisticated synthetic benchmark generation
			techniques used by the circuit placement community~\cite{nam07} may offer
			alternative benchmarks for future work. In the longer term, however, it
			is hoped that the availability of large SpiNNaker systems -- and
			placement and routing algorithms better suited to exploit them -- will
			lead to larger scale applications being developed. Hopefully these
			applications will lead to more interesting and representative benchmarks
			for use in future work.
	
	\section{Closing remarks}
		
		One of the primary outcomes of this work is that a number of the practical
		challenges faced in scaling up the SpiNNaker architecture have been
		addressed leading to the construction of large-scale SpiNNaker machines.
		The development of an effective placement algorithm for SpiNNaker
		applications has been shown to enable some neural simulations to exploit
		SpiNNaker's architecture for the first time. The availability of larger
		SpiNNaker machines paves the way for future large-scale neural modelling
		work built on much larger models such as Spaun, `the world's largest
		functional brain model'~\cite{eliasmith12}.
		
		Beyond the SpiNNaker project, the hexagonal torus topology has also been
		validated as a scalable and practical candidate for future network
		architectures. As super computers become ever larger, the physical
		scalability afforded by the 2D nature of the hexagonal torus topology may
		make it a compelling option. In addition, the finding that circuit
		placement techniques can be adapted to support placement of SpiNNaker
		software indicates that these algorithms may also be applicable to other
		applications. Indeed, if this is the case, circuit placement may offer a
		long-term source of placement algorithms able to handle the demands of
		future applications.
		
		% This thesis has explored and tackled a number of the challenges posed in
		% scaling up the unconventional SpiNNaker architecture. Along the way I have
		% demonstrated that the hexagonal torus topology may be a practical choice in
		% future applications which can scale up to the physical dimensions expected
		% of future super computers. I have also developed new efficient and
		% effective methods of placing and routing neural simulation software on
		% SpiNNaker which -- it is hoped -- will enable a new generation of large
		% scale neural simulations on spinnaker.
		
		Although this work stops short of demonstrating truly large-scale
		neuroscientific simulations running at the scale of newly completed
		SpiNNaker machines (largely because such simulations do not yet exist) a
		number of smaller-scale neural simulations have been made possible for the
		first time. The algorithms and techniques devised in this work have
		subsequently been incorporated into various software libraries and tools
		now being used by researchers building SpiNNaker applications, vindicating
		the efforts of this thesis (see appendix~\ref{sec:software}). A successor
		to the SpiNNaker architecture is also in the early stages of design and is
		building on experience of the existing architecture. The current intention
		is to retain the hexagonal torus topology used by SpiNNaker, a decision
		supported by the findings of this thesis.
		
		With SpiNNaker's hardware architecture now operating at scales close to its
		architectural limits, it is hoped that the contributions of this work will
		enable researchers to develop larger and more detailed neural models for
		this unique architecture.

	
	% Bibliography
	\bibliography{references}
	\bibliographystyle{alpha}
	
\end{document}
}
		}
		
	\end{center}
	
\end{titlepage}

	
	% Make preliminary non-numbered chapters start on the right-hand side
	\makeatletter\@openrightfalse\makeatother
	
	% The table of contents which, per university regulations, is followed by a
	% total wordcount.
	\cleardoublepage
	\tableofcontents
	\vfill
	\noindent\immediate\write18{./wordcount.sh > thesis.wordcount}%
		\documentclass[12pt,twoside]{report}

\newcommand{\thesistitle}{Building and operating large-scale SpiNNaker machines}
\newcommand{\thesisauthor}{Jonathan Heathcote}
\newcommand{\thesisyear}{2016}

%%%%%%%%%%%%%%%%%%%%%%%%%%%%%%%%%%%%%%%%%%%%%%%%%%%%%%%%%%%%%%%%%%%%%%%%%%%%%%%%
% Used packages
%%%%%%%%%%%%%%%%%%%%%%%%%%%%%%%%%%%%%%%%%%%%%%%%%%%%%%%%%%%%%%%%%%%%%%%%%%%%%%%%

% Nice printing of URLs
\usepackage{url}

% Actually not tear-your-eyes-out-ugly tables
\usepackage{booktabs}

% Adjust linespacing for localised parts of the paper (e.g. abstract)
\usepackage{setspace}

% For the \ifthenelse macro
\usepackage{ifthen}

% For the \degree macro
\usepackage{gensymb}

% For subfigure support
\usepackage{caption}
\usepackage{subcaption}

% SI unit and number formatting
\usepackage{siunitx}

% Used to draw labels with a white outline to make them stand-out in diagrams
\usepackage[outline]{contour}

% TikZ + PGF Plots for diagram/plot drawing
\usepackage{tikz}
\usepackage{tikz3d}
\usepackage{pgfplots}
\usetikzlibrary{ hexagon
               , calc
               , backgrounds
               , positioning
               , decorations.pathreplacing
               , decorations.markings
               , arrows
               , positioning
               , automata
               , shadows
               , fit
               , shapes
               , arrows
               , patterns
               , spy
               }
\usepgfplotslibrary{statistics}

%%%%%%%%%%%%%%%%%%%%%%%%%%%%%%%%%%%%%%%%%%%%%%%%%%%%%%%%%%%%%%%%%%%%%%%%%%%%%%%%
% Environment settings
%%%%%%%%%%%%%%%%%%%%%%%%%%%%%%%%%%%%%%%%%%%%%%%%%%%%%%%%%%%%%%%%%%%%%%%%%%%%%%%%

% 1.5 linespacing (as required by university)
\renewcommand{\baselinestretch}{1.5}


% Specifies the thickness of the contour added by the \contour macro.
\contourlength{1.5pt}

% Define a few layers for TikZ to allow easier layering
\pgfdeclarelayer{bg}
\pgfdeclarelayer{fg}
\pgfsetlayers{bg,main,fg}

%%%%%%%%%%%%%%%%%%%%%%%%%%%%%%%%%%%%%%%%%%%%%%%%%%%%%%%%%%%%%%%%%%%%%%%%%%%%%%%%
% Definitions
%%%%%%%%%%%%%%%%%%%%%%%%%%%%%%%%%%%%%%%%%%%%%%%%%%%%%%%%%%%%%%%%%%%%%%%%%%%%%%%%

% Used in place of \chapter for preface sections. Prevents numbering but
% includes the chapter in the ToC
\newcommand{\prefacesection}[1]{
	\chapter*{#1}
	\addcontentsline{toc}{chapter}{#1}
}

% Adds 'discard if' and  'discard if not' options for \addplot to enable
% filtering of data. Taken from
% http://tex.stackexchange.com/questions/58548/is-it-possible-to-change-the-color-of-a-single-bar-when-the-bar-plot-is-based-on
\pgfplotsset{
    discard if/.style 2 args={
        x filter/.code={
            \edef\tempa{\thisrow{#1}}
            \edef\tempb{#2}
            \ifx\tempa\tempb
                \def\pgfmathresult{inf}
            \fi
        }
    },
    discard if not/.style 2 args={
        x filter/.code={
            \edef\tempa{\thisrow{#1}}
            \edef\tempb{#2}
            \ifx\tempa\tempb
            \else
                \def\pgfmathresult{inf}
            \fi
        }
    }
}

% Make PGFplots Treat "NA" (regardless of letter case) as "nan". From:
% http://tex.stackexchange.com/questions/110441/skip-specific-string-in-a-numeric-column-while-using-pgfplots
\makeatletter
\expandafter\def\csname pgffltA@N\endcsname{\pgfflt@readundef}
\expandafter\def\csname pgffltA@n\endcsname{\pgfflt@readundef}
\def\pgfflt@readundef #1{%
    \def\pgfflt@readnan@ok{1}%
    \if#1a\else\if#1A\else\def\pgfflt@readnan@ok{0}\fi\fi
    \if\pgfflt@readnan@ok1%
        \pgfmathfloat@a@S=3\relax%
        \pgfmathfloat@a@Mtok={0.0}%
        \pgfmathfloat@a@E=0%
        \expandafter\pgfflt@finish
    \else
        \def\pgfflt@readnan@{\pgfflt@error #1}%
        \expandafter\pgfflt@readnan@
    \fi
}
\makeatother

%%%%%%%%%%%%%%%%%%%%%%%%%%%%%%%%%%%%%%%%%%%%%%%%%%%%%%%%%%%%%%%%%%%%%%%%%%%%%%%%
% Document body
%%%%%%%%%%%%%%%%%%%%%%%%%%%%%%%%%%%%%%%%%%%%%%%%%%%%%%%%%%%%%%%%%%%%%%%%%%%%%%%%
\begin{document}
	
	% The title page
	\begin{titlepage}
	
	
	\begin{center}
		
		\vspace*{1.0in}
		
		{\LARGE\textbf{\thesistitle}}
		
		\vfill
		
		\textsc{A thesis submitted to the University of Manchester\\for the degree of Doctor
		of Philosophy\\in the Faculty of Science and Engineering.}
		
		\vfill
		
		\thesisyear
		
		\vfill
		
		\thesisauthor
		
		\vfill
		
		School of Computer Science
		
		\vfill
		
		\color{gray}{
			{\tiny{}Revision \texttt{\documentclass[12pt,twoside]{report}

\newcommand{\thesistitle}{Building and operating large-scale SpiNNaker machines}
\newcommand{\thesisauthor}{Jonathan Heathcote}
\newcommand{\thesisyear}{2016}

%%%%%%%%%%%%%%%%%%%%%%%%%%%%%%%%%%%%%%%%%%%%%%%%%%%%%%%%%%%%%%%%%%%%%%%%%%%%%%%%
% Used packages
%%%%%%%%%%%%%%%%%%%%%%%%%%%%%%%%%%%%%%%%%%%%%%%%%%%%%%%%%%%%%%%%%%%%%%%%%%%%%%%%

% Nice printing of URLs
\usepackage{url}

% Actually not tear-your-eyes-out-ugly tables
\usepackage{booktabs}

% Adjust linespacing for localised parts of the paper (e.g. abstract)
\usepackage{setspace}

% For the \ifthenelse macro
\usepackage{ifthen}

% For the \degree macro
\usepackage{gensymb}

% For subfigure support
\usepackage{caption}
\usepackage{subcaption}

% SI unit and number formatting
\usepackage{siunitx}

% Used to draw labels with a white outline to make them stand-out in diagrams
\usepackage[outline]{contour}

% TikZ + PGF Plots for diagram/plot drawing
\usepackage{tikz}
\usepackage{tikz3d}
\usepackage{pgfplots}
\usetikzlibrary{ hexagon
               , calc
               , backgrounds
               , positioning
               , decorations.pathreplacing
               , decorations.markings
               , arrows
               , positioning
               , automata
               , shadows
               , fit
               , shapes
               , arrows
               , patterns
               , spy
               }
\usepgfplotslibrary{statistics}

%%%%%%%%%%%%%%%%%%%%%%%%%%%%%%%%%%%%%%%%%%%%%%%%%%%%%%%%%%%%%%%%%%%%%%%%%%%%%%%%
% Environment settings
%%%%%%%%%%%%%%%%%%%%%%%%%%%%%%%%%%%%%%%%%%%%%%%%%%%%%%%%%%%%%%%%%%%%%%%%%%%%%%%%

% 1.5 linespacing (as required by university)
\renewcommand{\baselinestretch}{1.5}


% Specifies the thickness of the contour added by the \contour macro.
\contourlength{1.5pt}

% Define a few layers for TikZ to allow easier layering
\pgfdeclarelayer{bg}
\pgfdeclarelayer{fg}
\pgfsetlayers{bg,main,fg}

%%%%%%%%%%%%%%%%%%%%%%%%%%%%%%%%%%%%%%%%%%%%%%%%%%%%%%%%%%%%%%%%%%%%%%%%%%%%%%%%
% Definitions
%%%%%%%%%%%%%%%%%%%%%%%%%%%%%%%%%%%%%%%%%%%%%%%%%%%%%%%%%%%%%%%%%%%%%%%%%%%%%%%%

% Used in place of \chapter for preface sections. Prevents numbering but
% includes the chapter in the ToC
\newcommand{\prefacesection}[1]{
	\chapter*{#1}
	\addcontentsline{toc}{chapter}{#1}
}

% Adds 'discard if' and  'discard if not' options for \addplot to enable
% filtering of data. Taken from
% http://tex.stackexchange.com/questions/58548/is-it-possible-to-change-the-color-of-a-single-bar-when-the-bar-plot-is-based-on
\pgfplotsset{
    discard if/.style 2 args={
        x filter/.code={
            \edef\tempa{\thisrow{#1}}
            \edef\tempb{#2}
            \ifx\tempa\tempb
                \def\pgfmathresult{inf}
            \fi
        }
    },
    discard if not/.style 2 args={
        x filter/.code={
            \edef\tempa{\thisrow{#1}}
            \edef\tempb{#2}
            \ifx\tempa\tempb
            \else
                \def\pgfmathresult{inf}
            \fi
        }
    }
}

% Make PGFplots Treat "NA" (regardless of letter case) as "nan". From:
% http://tex.stackexchange.com/questions/110441/skip-specific-string-in-a-numeric-column-while-using-pgfplots
\makeatletter
\expandafter\def\csname pgffltA@N\endcsname{\pgfflt@readundef}
\expandafter\def\csname pgffltA@n\endcsname{\pgfflt@readundef}
\def\pgfflt@readundef #1{%
    \def\pgfflt@readnan@ok{1}%
    \if#1a\else\if#1A\else\def\pgfflt@readnan@ok{0}\fi\fi
    \if\pgfflt@readnan@ok1%
        \pgfmathfloat@a@S=3\relax%
        \pgfmathfloat@a@Mtok={0.0}%
        \pgfmathfloat@a@E=0%
        \expandafter\pgfflt@finish
    \else
        \def\pgfflt@readnan@{\pgfflt@error #1}%
        \expandafter\pgfflt@readnan@
    \fi
}
\makeatother

%%%%%%%%%%%%%%%%%%%%%%%%%%%%%%%%%%%%%%%%%%%%%%%%%%%%%%%%%%%%%%%%%%%%%%%%%%%%%%%%
% Document body
%%%%%%%%%%%%%%%%%%%%%%%%%%%%%%%%%%%%%%%%%%%%%%%%%%%%%%%%%%%%%%%%%%%%%%%%%%%%%%%%
\begin{document}
	
	% The title page
	\input{titlepage}
	
	% The table of contents which, per university regulations, is followed by a
	% total wordcount.
	\tableofcontents
	\vfill
	\noindent This thesis contains
		\immediate\write18{texcount -1 -sum -inc thesis.tex > thesis.wordcount}%
		\input{thesis.wordcount}words.
	
	\clearpage
	\listoffigures
	
	\clearpage
	\listoftables
	
	% Abstract
	\input{abstract}
	
	% Declaration of non-submission elsewhere
	\input{declaration}
	
	% University-prescribed copyright statement...
	\input{copyright}
	
	% Acknowledgements
	\input{acknowledgements}
	
	% Main body
	\input{introduction.tex}
	\input{background.tex}
	\input{building.tex}
	\input{shortestPaths.tex}
	\input{routing.tex}
	\input{placement.tex}
	\input{discussion.tex}
	\input{future.tex}
	\input{conclusions.tex}
	
	% Bibliography
	\bibliography{references}
	\bibliographystyle{alpha}
	
\end{document}
}\documentclass[12pt,twoside]{report}

\newcommand{\thesistitle}{Building and operating large-scale SpiNNaker machines}
\newcommand{\thesisauthor}{Jonathan Heathcote}
\newcommand{\thesisyear}{2016}

%%%%%%%%%%%%%%%%%%%%%%%%%%%%%%%%%%%%%%%%%%%%%%%%%%%%%%%%%%%%%%%%%%%%%%%%%%%%%%%%
% Used packages
%%%%%%%%%%%%%%%%%%%%%%%%%%%%%%%%%%%%%%%%%%%%%%%%%%%%%%%%%%%%%%%%%%%%%%%%%%%%%%%%

% Nice printing of URLs
\usepackage{url}

% Actually not tear-your-eyes-out-ugly tables
\usepackage{booktabs}

% Adjust linespacing for localised parts of the paper (e.g. abstract)
\usepackage{setspace}

% For the \ifthenelse macro
\usepackage{ifthen}

% For the \degree macro
\usepackage{gensymb}

% For subfigure support
\usepackage{caption}
\usepackage{subcaption}

% SI unit and number formatting
\usepackage{siunitx}

% Used to draw labels with a white outline to make them stand-out in diagrams
\usepackage[outline]{contour}

% TikZ + PGF Plots for diagram/plot drawing
\usepackage{tikz}
\usepackage{tikz3d}
\usepackage{pgfplots}
\usetikzlibrary{ hexagon
               , calc
               , backgrounds
               , positioning
               , decorations.pathreplacing
               , decorations.markings
               , arrows
               , positioning
               , automata
               , shadows
               , fit
               , shapes
               , arrows
               , patterns
               , spy
               }
\usepgfplotslibrary{statistics}

%%%%%%%%%%%%%%%%%%%%%%%%%%%%%%%%%%%%%%%%%%%%%%%%%%%%%%%%%%%%%%%%%%%%%%%%%%%%%%%%
% Environment settings
%%%%%%%%%%%%%%%%%%%%%%%%%%%%%%%%%%%%%%%%%%%%%%%%%%%%%%%%%%%%%%%%%%%%%%%%%%%%%%%%

% 1.5 linespacing (as required by university)
\renewcommand{\baselinestretch}{1.5}


% Specifies the thickness of the contour added by the \contour macro.
\contourlength{1.5pt}

% Define a few layers for TikZ to allow easier layering
\pgfdeclarelayer{bg}
\pgfdeclarelayer{fg}
\pgfsetlayers{bg,main,fg}

%%%%%%%%%%%%%%%%%%%%%%%%%%%%%%%%%%%%%%%%%%%%%%%%%%%%%%%%%%%%%%%%%%%%%%%%%%%%%%%%
% Definitions
%%%%%%%%%%%%%%%%%%%%%%%%%%%%%%%%%%%%%%%%%%%%%%%%%%%%%%%%%%%%%%%%%%%%%%%%%%%%%%%%

% Used in place of \chapter for preface sections. Prevents numbering but
% includes the chapter in the ToC
\newcommand{\prefacesection}[1]{
	\chapter*{#1}
	\addcontentsline{toc}{chapter}{#1}
}

% Adds 'discard if' and  'discard if not' options for \addplot to enable
% filtering of data. Taken from
% http://tex.stackexchange.com/questions/58548/is-it-possible-to-change-the-color-of-a-single-bar-when-the-bar-plot-is-based-on
\pgfplotsset{
    discard if/.style 2 args={
        x filter/.code={
            \edef\tempa{\thisrow{#1}}
            \edef\tempb{#2}
            \ifx\tempa\tempb
                \def\pgfmathresult{inf}
            \fi
        }
    },
    discard if not/.style 2 args={
        x filter/.code={
            \edef\tempa{\thisrow{#1}}
            \edef\tempb{#2}
            \ifx\tempa\tempb
            \else
                \def\pgfmathresult{inf}
            \fi
        }
    }
}

% Make PGFplots Treat "NA" (regardless of letter case) as "nan". From:
% http://tex.stackexchange.com/questions/110441/skip-specific-string-in-a-numeric-column-while-using-pgfplots
\makeatletter
\expandafter\def\csname pgffltA@N\endcsname{\pgfflt@readundef}
\expandafter\def\csname pgffltA@n\endcsname{\pgfflt@readundef}
\def\pgfflt@readundef #1{%
    \def\pgfflt@readnan@ok{1}%
    \if#1a\else\if#1A\else\def\pgfflt@readnan@ok{0}\fi\fi
    \if\pgfflt@readnan@ok1%
        \pgfmathfloat@a@S=3\relax%
        \pgfmathfloat@a@Mtok={0.0}%
        \pgfmathfloat@a@E=0%
        \expandafter\pgfflt@finish
    \else
        \def\pgfflt@readnan@{\pgfflt@error #1}%
        \expandafter\pgfflt@readnan@
    \fi
}
\makeatother

%%%%%%%%%%%%%%%%%%%%%%%%%%%%%%%%%%%%%%%%%%%%%%%%%%%%%%%%%%%%%%%%%%%%%%%%%%%%%%%%
% Document body
%%%%%%%%%%%%%%%%%%%%%%%%%%%%%%%%%%%%%%%%%%%%%%%%%%%%%%%%%%%%%%%%%%%%%%%%%%%%%%%%
\begin{document}
	
	% The title page
	\input{titlepage}
	
	% The table of contents which, per university regulations, is followed by a
	% total wordcount.
	\tableofcontents
	\vfill
	\noindent This thesis contains
		\immediate\write18{texcount -1 -sum -inc thesis.tex > thesis.wordcount}%
		\input{thesis.wordcount}words.
	
	\clearpage
	\listoffigures
	
	\clearpage
	\listoftables
	
	% Abstract
	\input{abstract}
	
	% Declaration of non-submission elsewhere
	\input{declaration}
	
	% University-prescribed copyright statement...
	\input{copyright}
	
	% Acknowledgements
	\input{acknowledgements}
	
	% Main body
	\input{introduction.tex}
	\input{background.tex}
	\input{building.tex}
	\input{shortestPaths.tex}
	\input{routing.tex}
	\input{placement.tex}
	\input{discussion.tex}
	\input{future.tex}
	\input{conclusions.tex}
	
	% Bibliography
	\bibliography{references}
	\bibliographystyle{alpha}
	
\end{document}
}
		}
		
	\end{center}
	
\end{titlepage}

	
	% The table of contents which, per university regulations, is followed by a
	% total wordcount.
	\tableofcontents
	\vfill
	\noindent This thesis contains
		\immediate\write18{texcount -1 -sum -inc thesis.tex > thesis.wordcount}%
		\documentclass[12pt,twoside]{report}

\newcommand{\thesistitle}{Building and operating large-scale SpiNNaker machines}
\newcommand{\thesisauthor}{Jonathan Heathcote}
\newcommand{\thesisyear}{2016}

%%%%%%%%%%%%%%%%%%%%%%%%%%%%%%%%%%%%%%%%%%%%%%%%%%%%%%%%%%%%%%%%%%%%%%%%%%%%%%%%
% Used packages
%%%%%%%%%%%%%%%%%%%%%%%%%%%%%%%%%%%%%%%%%%%%%%%%%%%%%%%%%%%%%%%%%%%%%%%%%%%%%%%%

% Nice printing of URLs
\usepackage{url}

% Actually not tear-your-eyes-out-ugly tables
\usepackage{booktabs}

% Adjust linespacing for localised parts of the paper (e.g. abstract)
\usepackage{setspace}

% For the \ifthenelse macro
\usepackage{ifthen}

% For the \degree macro
\usepackage{gensymb}

% For subfigure support
\usepackage{caption}
\usepackage{subcaption}

% SI unit and number formatting
\usepackage{siunitx}

% Used to draw labels with a white outline to make them stand-out in diagrams
\usepackage[outline]{contour}

% TikZ + PGF Plots for diagram/plot drawing
\usepackage{tikz}
\usepackage{tikz3d}
\usepackage{pgfplots}
\usetikzlibrary{ hexagon
               , calc
               , backgrounds
               , positioning
               , decorations.pathreplacing
               , decorations.markings
               , arrows
               , positioning
               , automata
               , shadows
               , fit
               , shapes
               , arrows
               , patterns
               , spy
               }
\usepgfplotslibrary{statistics}

%%%%%%%%%%%%%%%%%%%%%%%%%%%%%%%%%%%%%%%%%%%%%%%%%%%%%%%%%%%%%%%%%%%%%%%%%%%%%%%%
% Environment settings
%%%%%%%%%%%%%%%%%%%%%%%%%%%%%%%%%%%%%%%%%%%%%%%%%%%%%%%%%%%%%%%%%%%%%%%%%%%%%%%%

% 1.5 linespacing (as required by university)
\renewcommand{\baselinestretch}{1.5}


% Specifies the thickness of the contour added by the \contour macro.
\contourlength{1.5pt}

% Define a few layers for TikZ to allow easier layering
\pgfdeclarelayer{bg}
\pgfdeclarelayer{fg}
\pgfsetlayers{bg,main,fg}

%%%%%%%%%%%%%%%%%%%%%%%%%%%%%%%%%%%%%%%%%%%%%%%%%%%%%%%%%%%%%%%%%%%%%%%%%%%%%%%%
% Definitions
%%%%%%%%%%%%%%%%%%%%%%%%%%%%%%%%%%%%%%%%%%%%%%%%%%%%%%%%%%%%%%%%%%%%%%%%%%%%%%%%

% Used in place of \chapter for preface sections. Prevents numbering but
% includes the chapter in the ToC
\newcommand{\prefacesection}[1]{
	\chapter*{#1}
	\addcontentsline{toc}{chapter}{#1}
}

% Adds 'discard if' and  'discard if not' options for \addplot to enable
% filtering of data. Taken from
% http://tex.stackexchange.com/questions/58548/is-it-possible-to-change-the-color-of-a-single-bar-when-the-bar-plot-is-based-on
\pgfplotsset{
    discard if/.style 2 args={
        x filter/.code={
            \edef\tempa{\thisrow{#1}}
            \edef\tempb{#2}
            \ifx\tempa\tempb
                \def\pgfmathresult{inf}
            \fi
        }
    },
    discard if not/.style 2 args={
        x filter/.code={
            \edef\tempa{\thisrow{#1}}
            \edef\tempb{#2}
            \ifx\tempa\tempb
            \else
                \def\pgfmathresult{inf}
            \fi
        }
    }
}

% Make PGFplots Treat "NA" (regardless of letter case) as "nan". From:
% http://tex.stackexchange.com/questions/110441/skip-specific-string-in-a-numeric-column-while-using-pgfplots
\makeatletter
\expandafter\def\csname pgffltA@N\endcsname{\pgfflt@readundef}
\expandafter\def\csname pgffltA@n\endcsname{\pgfflt@readundef}
\def\pgfflt@readundef #1{%
    \def\pgfflt@readnan@ok{1}%
    \if#1a\else\if#1A\else\def\pgfflt@readnan@ok{0}\fi\fi
    \if\pgfflt@readnan@ok1%
        \pgfmathfloat@a@S=3\relax%
        \pgfmathfloat@a@Mtok={0.0}%
        \pgfmathfloat@a@E=0%
        \expandafter\pgfflt@finish
    \else
        \def\pgfflt@readnan@{\pgfflt@error #1}%
        \expandafter\pgfflt@readnan@
    \fi
}
\makeatother

%%%%%%%%%%%%%%%%%%%%%%%%%%%%%%%%%%%%%%%%%%%%%%%%%%%%%%%%%%%%%%%%%%%%%%%%%%%%%%%%
% Document body
%%%%%%%%%%%%%%%%%%%%%%%%%%%%%%%%%%%%%%%%%%%%%%%%%%%%%%%%%%%%%%%%%%%%%%%%%%%%%%%%
\begin{document}
	
	% The title page
	\begin{titlepage}
	
	
	\begin{center}
		
		\vspace*{1.0in}
		
		{\LARGE\textbf{\thesistitle}}
		
		\vfill
		
		\textsc{A thesis submitted to the University of Manchester\\for the degree of Doctor
		of Philosophy\\in the Faculty of Science and Engineering.}
		
		\vfill
		
		\thesisyear
		
		\vfill
		
		\thesisauthor
		
		\vfill
		
		School of Computer Science
		
		\vfill
		
		\color{gray}{
			{\tiny{}Revision \texttt{\input{thesis.fullhash}}\input{thesis.date}}
		}
		
	\end{center}
	
\end{titlepage}

	
	% The table of contents which, per university regulations, is followed by a
	% total wordcount.
	\tableofcontents
	\vfill
	\noindent This thesis contains
		\immediate\write18{texcount -1 -sum -inc thesis.tex > thesis.wordcount}%
		\documentclass[12pt,twoside]{report}

\newcommand{\thesistitle}{Building and operating large-scale SpiNNaker machines}
\newcommand{\thesisauthor}{Jonathan Heathcote}
\newcommand{\thesisyear}{2016}

%%%%%%%%%%%%%%%%%%%%%%%%%%%%%%%%%%%%%%%%%%%%%%%%%%%%%%%%%%%%%%%%%%%%%%%%%%%%%%%%
% Used packages
%%%%%%%%%%%%%%%%%%%%%%%%%%%%%%%%%%%%%%%%%%%%%%%%%%%%%%%%%%%%%%%%%%%%%%%%%%%%%%%%

% Nice printing of URLs
\usepackage{url}

% Actually not tear-your-eyes-out-ugly tables
\usepackage{booktabs}

% Adjust linespacing for localised parts of the paper (e.g. abstract)
\usepackage{setspace}

% For the \ifthenelse macro
\usepackage{ifthen}

% For the \degree macro
\usepackage{gensymb}

% For subfigure support
\usepackage{caption}
\usepackage{subcaption}

% SI unit and number formatting
\usepackage{siunitx}

% Used to draw labels with a white outline to make them stand-out in diagrams
\usepackage[outline]{contour}

% TikZ + PGF Plots for diagram/plot drawing
\usepackage{tikz}
\usepackage{tikz3d}
\usepackage{pgfplots}
\usetikzlibrary{ hexagon
               , calc
               , backgrounds
               , positioning
               , decorations.pathreplacing
               , decorations.markings
               , arrows
               , positioning
               , automata
               , shadows
               , fit
               , shapes
               , arrows
               , patterns
               , spy
               }
\usepgfplotslibrary{statistics}

%%%%%%%%%%%%%%%%%%%%%%%%%%%%%%%%%%%%%%%%%%%%%%%%%%%%%%%%%%%%%%%%%%%%%%%%%%%%%%%%
% Environment settings
%%%%%%%%%%%%%%%%%%%%%%%%%%%%%%%%%%%%%%%%%%%%%%%%%%%%%%%%%%%%%%%%%%%%%%%%%%%%%%%%

% 1.5 linespacing (as required by university)
\renewcommand{\baselinestretch}{1.5}


% Specifies the thickness of the contour added by the \contour macro.
\contourlength{1.5pt}

% Define a few layers for TikZ to allow easier layering
\pgfdeclarelayer{bg}
\pgfdeclarelayer{fg}
\pgfsetlayers{bg,main,fg}

%%%%%%%%%%%%%%%%%%%%%%%%%%%%%%%%%%%%%%%%%%%%%%%%%%%%%%%%%%%%%%%%%%%%%%%%%%%%%%%%
% Definitions
%%%%%%%%%%%%%%%%%%%%%%%%%%%%%%%%%%%%%%%%%%%%%%%%%%%%%%%%%%%%%%%%%%%%%%%%%%%%%%%%

% Used in place of \chapter for preface sections. Prevents numbering but
% includes the chapter in the ToC
\newcommand{\prefacesection}[1]{
	\chapter*{#1}
	\addcontentsline{toc}{chapter}{#1}
}

% Adds 'discard if' and  'discard if not' options for \addplot to enable
% filtering of data. Taken from
% http://tex.stackexchange.com/questions/58548/is-it-possible-to-change-the-color-of-a-single-bar-when-the-bar-plot-is-based-on
\pgfplotsset{
    discard if/.style 2 args={
        x filter/.code={
            \edef\tempa{\thisrow{#1}}
            \edef\tempb{#2}
            \ifx\tempa\tempb
                \def\pgfmathresult{inf}
            \fi
        }
    },
    discard if not/.style 2 args={
        x filter/.code={
            \edef\tempa{\thisrow{#1}}
            \edef\tempb{#2}
            \ifx\tempa\tempb
            \else
                \def\pgfmathresult{inf}
            \fi
        }
    }
}

% Make PGFplots Treat "NA" (regardless of letter case) as "nan". From:
% http://tex.stackexchange.com/questions/110441/skip-specific-string-in-a-numeric-column-while-using-pgfplots
\makeatletter
\expandafter\def\csname pgffltA@N\endcsname{\pgfflt@readundef}
\expandafter\def\csname pgffltA@n\endcsname{\pgfflt@readundef}
\def\pgfflt@readundef #1{%
    \def\pgfflt@readnan@ok{1}%
    \if#1a\else\if#1A\else\def\pgfflt@readnan@ok{0}\fi\fi
    \if\pgfflt@readnan@ok1%
        \pgfmathfloat@a@S=3\relax%
        \pgfmathfloat@a@Mtok={0.0}%
        \pgfmathfloat@a@E=0%
        \expandafter\pgfflt@finish
    \else
        \def\pgfflt@readnan@{\pgfflt@error #1}%
        \expandafter\pgfflt@readnan@
    \fi
}
\makeatother

%%%%%%%%%%%%%%%%%%%%%%%%%%%%%%%%%%%%%%%%%%%%%%%%%%%%%%%%%%%%%%%%%%%%%%%%%%%%%%%%
% Document body
%%%%%%%%%%%%%%%%%%%%%%%%%%%%%%%%%%%%%%%%%%%%%%%%%%%%%%%%%%%%%%%%%%%%%%%%%%%%%%%%
\begin{document}
	
	% The title page
	\input{titlepage}
	
	% The table of contents which, per university regulations, is followed by a
	% total wordcount.
	\tableofcontents
	\vfill
	\noindent This thesis contains
		\immediate\write18{texcount -1 -sum -inc thesis.tex > thesis.wordcount}%
		\input{thesis.wordcount}words.
	
	\clearpage
	\listoffigures
	
	\clearpage
	\listoftables
	
	% Abstract
	\input{abstract}
	
	% Declaration of non-submission elsewhere
	\input{declaration}
	
	% University-prescribed copyright statement...
	\input{copyright}
	
	% Acknowledgements
	\input{acknowledgements}
	
	% Main body
	\input{introduction.tex}
	\input{background.tex}
	\input{building.tex}
	\input{shortestPaths.tex}
	\input{routing.tex}
	\input{placement.tex}
	\input{discussion.tex}
	\input{future.tex}
	\input{conclusions.tex}
	
	% Bibliography
	\bibliography{references}
	\bibliographystyle{alpha}
	
\end{document}
words.
	
	\clearpage
	\listoffigures
	
	\clearpage
	\listoftables
	
	% Abstract
	{
	\prefacesection{Abstract}
	
	% Single line spacing for the abstract page
	\setstretch{1.0}
	
	
	\vfill
	
	% Standard thesis information
	\begin{center}
		\textsc{\large\thesistitle}
		
		\vspace{0.5em}
		
		\thesisauthor
		
		\vspace{0.5em}
		
		A thesis submitted to the University of Manchester\\
		for the degree of Doctor of Philosophy, 2016
	\end{center}
	
	\vfill
	
	% The abstract
	
	SpiNNaker is an unconventional super computer architecture designed to
	simulate up to one billion biologically realistic neurons in real-time. To
	achieve this goal, SpiNNaker employs a novel network architecture which poses
	a number of practical problems in scaling up from desktop prototypes to
	machine room filling installations.
	
	SpiNNaker's hexagonal torus network topology has received mostly theoretical
	treatment in the literature. This thesis tackles some of the challenges
	encountered when building `real-world' systems.  Firstly, a scheme is devised
	for physically laying out hexagonal torus topologies in machine rooms which
	avoids long cables; this is demonstrated on a half-million core SpiNNaker
	prototype.  Secondly, to improve the performance of existing routing
	algorithms, a more efficient process is proposed for finding (logically)
	short paths through hexagonal torus topologies. This is complemented by a
	formula which provides routing algorithms greater flexibility when finding
	paths, potentially resulting in a more balanced network utilisation.
	
	The scale of SpiNNaker's network and the models intended for it also present
	their own challenges. Placement and routing algorithms are developed which
	assign processes to nodes and generate paths through SpiNNaker's network.
	These algorithms reduce congestion and tolerate network faults. The proposed
	placement algorithm is inspired by techniques used in chip design and is
	shown to enable larger applications to run on SpiNNaker -- with good
	performance -- than the previous state-of-the-art. Likewise the routing
	algorithm developed is able to tolerate network faults, inevitably present in
	large scale systems, with little performance overhead.
	
	
	% Required to ensure single line spacing is used for this whole block
	\par%
}

	
	% Declaration of non-submission elsewhere
	\prefacesection{Declaration}

% Single line spacing for the declaration
{
	\setstretch{1.0}
	No portion of the work referred to in this thesis has been submitted in support
	of an application for another degree or qualification of this or any other
	university or other institute of learning.
	
	\par%
}


	
	% University-prescribed copyright statement...
	\input{copyright}
	
	% Acknowledgements
	{
	\prefacesection{Acknowledgements}
	
	% Single line spacing
	\setstretch{1.0}
	
	It is often said that it is not \emph{what} you know but \emph{who} you know.
	Throughout the course of my PhD I've been exceptionally lucky to have been
	helped along by a great number of people.
	
	Both my supervisor, Jim Garside, and co-supervisor, Steve Furber, have each
	spent countless hours patiently discussing and describing all manner of
	things with me while giving me great freedom in my project. Jim's office door
	has always been open to my unexpected interruptions be it about work, writing
	or walking.  Likewise, Steve has always managed to find time for both
	technical and frivolous endeavours alike. I'm also hugely grateful to Luis
	Plana who has been a rich source of sage advice, insightful questions
	patiently suffered many a foolish question.
	
	Various parts of the work in this thesis (and numerous side projects) would
	not have been possible if not for the multitude of discussions,
	collaborations and even sheer physical hard work of Steve Temple, Javier
	Navaridas, Simon Davidson and Dave Clark. I'm also indebted to Andrew Mundy
	and Jamie Knight, both of whom have donated so much time and effort towards
	verifying and using software implementations of the ideas in this thesis.
	
	The injection of lunchtime silliness by Andrew and Jamie, along with Amanieu
	d'Antras and Andrew Webb and the other CDT members has always brightened my
	day. So to has the friendly and stimulating environment of the School of
	Computer Science and its many staff and students. Of course, I am also very
	grateful for the funding the school has provided for my research.
	
	I cannot thank my wonderful wife, Ann-Marie, enough for being by my side. She
	has given me so much kindness, love and patience and endured a lifetime's
	quota of conversations about hexagons. Finally, thanks too to rest of my
	family, especially my parents, who are to blame for starting me down this
	path and co-suffering drafts and endless rants about this document.
	
	% Required to ensure single line spacing is used for this whole block
	\par%
}

	
	% Main body
	\chapter{Introduction}

\label{sec:introduction}

%Problem area
%
%* Network construction and exploitation
%  * Cabling: Build it cheaply in terms of tech cost and install cost
%  * Routing: Get around it cheaply and reliably
%  * Placement: Use it efficiently

The Spiking Neural Network Architecture (SpiNNaker) is a novel super computer
architecture designed to simulate biologically realistic models of brains in
real time \cite{furber07}. Though neurons, the building blocks of the brain,
are relatively well understood, their complex interactions remain mysterious.
Just as understanding the workings of a transistor is insufficient to
understand a modern microprocessor, neuroscientists believe that understanding
the neurons in isolation cannot explain the brain and that understanding their
connectivity is key \cite{eliasmith13,eliasmith14}. Experiments on real brains,
however, are fraught with difficulty. Variations between individuals can be
significant and it is only possible to record tens or hundreds of the trillions
of signals in the brain, and even then only with limited control over which
signals are recorded. Computer simulations of models of large neural networks,
however, enable researchers to develop repeatable experiments and gain complete
visibility of any signal and any neuron. Models such as SPAUN
\cite{eliasmith12}, built from millions of simulated neurons, have shown great
promise in expanding our understanding of higher level brain functions such as
memory and simple problem solving.  Unfortunately these neural models are
expensive to simulate, requiring hours of compute time to simulate each second
of neural activity. As well as being inconvenient, this precludes the use of
robotics to immerse these models in real world environments and also limits
studies of long-term behaviours such as learning.

SpiNNaker is designed to enable the real time simulation of models containing
up to one billion neurons -- approximately \SI{1}{\percent} of a human brain or
ten mouse brains \cite{furber06}. To achieve this goal, the largest planned
SpiNNaker machine will contain over one million low-powered computer processors
interconnected by a bespoke network architecture.

SpiNNaker's large processor count matches the current trend in super computers
where processor counts are growing exponentially \cite{meuer16j}, mimicking the
growth of the number of components in the processors themselves predicted by
Gordon Moore's famous `law' \cite{moore75}. As a result of this growth, the
interconnection networks which enable these processors to work together have
grown in importance \cite{dally04}.  Network designers must carefully balance
performance against practicality and financial cost.  SpiNNaker's network is no
exception to this rule and, as the systems scale up from desktop prototypes to
machine-room scale installations, the reality of building and exploiting these
machines presents an array of challenges.

As in all super computers, SpiNNaker's network interconnects its processors in
a particular network topology which defines how different processors may
communicate with each other. Unlike the tree and $N$-dimensional torus
topologies found in contemporary super computers \cite{dally04}, SpiNNaker
employs a `hexagonal torus topology'. In this topology, nodes in SpiNNaker's
network fit together in a honeycomb-like pattern where messages may `hop' from
node to node to reach their destination. As we will see in
chapter~\ref{sec:background}, the hexagonal torus topology, in theory, sits at
a `sweet spot' in terms of network performance and practicality. As the first
known large-scale installation of the hexagonal torus topology, however, there
remain a number of practical challenges for large spinnaker machines arising
from this choice.

As super computer networks have grown in scale to millions of processors the
task of dealing with previously rare faults has grown.  Though fault rates in
networks remain consistently low, architectures such as SpiNNaker may have
hundreds of thousands of links meaning even fault rates of a fraction of a
percent will impact tens or hundreds of links. To enable reliable operation,
networks must be able to adapt the routes taken by messages through the network
to avoid faulty links and nodes. The techniques employed are often closely tied
to a particular network architecture and consequently SpiNNaker's novel network
architecture demands its own approach.

Another challenge introduced by the growing scale of super computers is making
\emph{efficient} use of network resources. Communicating processes should be
located on logically `nearby' nodes to reduce network load. The neural models
for which SpiNNaker is designed are often described abstractly, rather than
geometrically, using modelling languages such as PyNN~\cite{davison08} and
Nengo~\cite{eliasmith04}.  Because of this, the communication requirements of
simulations can be highly irregular making an efficient placement of processes
onto processors in the machine non-trivial.

%Contributions
%
%* Cabling scheme for hexagonal toruses without long cables
%* Efficient installation technique for dense systems
%* Exhaustive and efficient route calculation in hex toruses
%* Fault tolerant routing scheme exploiting SpiNNaker's odd router
%* Placement based on SA a: works very well and b: suggests circuit placement is
%  a good source of inspiration.

This thesis addresses the practical challenges of scaling up the SpiNNaker
architecture in a real-world setting summarised by these research questions:

\begin{enumerate}
	
	\item Can the hexagonal torus topology be deployed and used in real, large
	scale systems?
	
	\item Does SpiNNaker's router architecture help, or hinder fault tolerance?
	
	\item How can the parts of a neural simulation be placed onto a large
	hexagonal torus topology to reduce network load?
	
\end{enumerate}

%Structure
%
%* Chapter 2: Background: detailed dive into what's in SpiNNaker, why its
%  really so unusual. Also looks at what applications run on SpiNNaker and how
%  they work.
%* Chapter 3: How to build a really big SpiNNaker machine.
%* Chapter 4: How to find your way around that machine.
%* Chapter 5: How to find your way around that machine even when its broken.
%* Chapter 6: Now you can walk, time to run.
%* Chapter 7: Wrapping up.
%* Appendices: Hard-to-come-by theoretical and practical details useful if
%  you're about to continue where this research left off but be useful but
%  otherwise hard to come by, especially in one place.

Chapter~\ref{sec:background} introduces the SpiNNaker architecture and, in
particular, describes its hexagonal torus topology and network architecture.

In chapter~\ref{sec:building}, I develop a cabling scheme for large hexagonal
torus topologies which enables arbitrarily large networks to be constructed
using only short, inexpensive cables. This theoretical work is then evaluated
through the construction of a range of prototype SpiNNaker systems. The largest
of these prototypes contains over half a million processor cores and spans
several machine room cabinets. In addition, I propose the use of built-in
diagnostic facilities to assist technicians performing network installation and
maintenance. This technique is found to greatly reduce the effort required and
the number of mistakes made.

In chapters~\ref{sec:shortestPaths}~and~\ref{sec:routing} I develop new routing
techniques for SpiNNaker's network. Chapter~\ref{sec:shortestPaths} develops a
new approach to finding the shortest paths through hexagonal torus topologies,
an integral part of many routing algorithms. This newly proposed approach is
cheaper to compute than the state of the art and, unlike previous efforts, is
able to discover all valid short paths through the topology. This theoretical
advance brings hexagonal torus topologies in line with conventional topologies
by providing routing algorithms with complete information about the paths
available to them. In chapter \ref{sec:routing} I propose a fault tolerant
routing algorithm for SpiNNaker which is able to avoid arbitrary static fault
patterns with minimal performance overhead. A key finding of this chapter is
that the flexibility afforded to fault tolerant routing algorithms by
SpiNNaker's unconventional router architecture is what facilities the low
overheads reported in this chapter.

Finally, in chapter~\ref{sec:placement}, I explore the problem of application
placement in SpiNNaker's network. As in other networks and applications, neural
simulations should be arranged such that communication occurs primarily between
processors close together in the network to control network load. Due to the
irregular connectivity and large scale of the neural models expected to run on
SpiNNaker, an automated approach is necessary. I develop a novel placement
algorithm based on algorithms used for circuit layout in computer chips. My
algorithm is found to allow some larger neural models to run on SpiNNaker for
the first time while enabling other applications to run at greater speeds. In
addition, synthetic benchmarks containing over one million processes indicate
that this algorithm should handle the anticipated demands of the neural models
expected to run on large-scale SpiNNaker installations.

	\chapter{The SpiNNaker Architecture}
	
	\label{sec:background}
	
	SpiNNaker is a massively parallel computer architecture designed to simulate
	biologically realistic neural models \cite{furber07}. In this chapter we will
	explore this unconventional architecture in detail, starting with its purpose
	before focusing on its most unconventional feature: its network.
	
	% * Purpose
	%   * Spiking neural simulations
	%     * Neural modelling: PyNN, Nengo...
	%     * Parallelisation + communication
	
	\section{Neural simulation}
		
		Human brains contain billions of neurons connected together by trillions of
		`synapses'. Neurons communicate by transmitting and receiving `spikes'
		through their synapses. Each spike is `valueless' in that a spike's only
		significant features are when it arrives and where it has come from.
		
		\begin{figure}
			\center
			\buildfig{figures/lif-neuron.tex}
			
			\caption{A Leaky Integrate-and-Fire (LIF) neuron.}
			\label{fig:lif-neuron}
		\end{figure}
		
		Though some detailed models of the electrochemical processes occurring
		inside neurons are computationally intensive, simplified models such as the
		Leaky Integrate-and-Fire (LIF) model can be implemented in just a handful
		of CPU instructions \cite{vainbrand11}. Figure~\ref{fig:lif-neuron}
		illustrates a simple LIF neuron in which incoming spikes cause charge to
		build up (integrated) which over time, leaks away. If an incoming spike
		causes the charge to rise above a certain threshold, the neuron `fires'
		producing an outgoing spike. Despite the simplicity of this model, large
		neural networks such as Spaun \cite{eliasmith12} -- built entirely from LIF
		neurons -- exhibit complex behaviours such as fine motor control and
		problem solving.
		
		The computational expense of large scale neural simulations does not arise
		from the cost of modelling neurons but instead from distributing spikes. In
		biology, neurons produce spikes at an average rate of \SI{10}{\hertz} and
		synapses connect each neuron's output to (order) \num{1000}~neurons
		\cite{navaridas09}. Consider an example neural model with $7\times10^7$
		neurons, approximately the number in a house mouse and
		$\nicefrac{1}{10}^\textrm{th}$ of the design target of SpiNNaker. This
		network might produce $7\times10^8$~spikes per second. Because each neuron
		connects to many others, this equates to $7\times10^{11}$ spikes being
		received per second. If each spike were transmitted as a UDP datagram
		containing a single \SI{32}{\bit} payload, the total network throughput
		required for this simulation would be \SI{179.2}{\tera\bit\per\second}. At
		the time of writing, this is more than double the bisection bandwidth (the
		theoretical worst-case throughput) of the world's most powerful super
		computer \cite{dongarra16}.
	
	\section{Network architecture}
		
		Architectures such as IBM's Blue Gene \cite{chiu11} and Cray's XK7
		\cite{ornl16} employ powerful compute nodes connected together using
		networks designed to transfer multi-kilobyte blocks of data between nodes.
		Since neural models have relatively light computational requirements and
		communications are based on small pieces of data (spikes), this type of
		architecture is poorly suited to the task.
		
		SpiNNaker's architectural target is to support realtime simulations of up
		to one billion neurons. Since neural models such as LIF are inexpensive to
		model and many neurons can be simulated independently in parallel,
		SpiNNaker employs many small, energy efficient ARM processors
		\cite{furber07}. To support the unusual communication requirements of
		neural simulations, a bespoke interconnection network is used which is the
		background to this thesis.
		
	%   * SpiNNaker chip
	%     * Cores
	%     * SDRAM
	%     * NoC
	%     * Router
		
		\begin{figure}
			\center
			%\includegraphics[width=19mm]{figures/spinnakerChip.jpg}
			\buildfig{figures/hex-chips.tex}
			
			\caption[SpiNNaker chips connected to their six neighbours.]%
			{SpiNNaker chips (actual size) connected to their six neighbours.}
			\label{fig:spinnakerChip}
		\end{figure}
		
		The fundamental building block of the SpiNNaker architecture is the
		SpiNNaker chip (figure \ref{fig:spinnakerChip}) \cite{furber13}. Each chip
		contains eighteen low power ARM 968 processor cores each capable of
		simulating between \num{200} and \num{2000} LIF neurons in real time
		\cite{mundy15}.  Each core has a total of \SI{96}{\kilo\byte} of private
		Tightly-Coupled Memory (TCM) and shares access to \SI{128}{\mega\byte} of
		on-chip SDRAM with other cores on the same chip. Finally, each chip
		contains a programmable router which routes network packets to and from the
		local cores and six neighbouring SpiNNaker chips. SpiNNaker machines are
		constructed by combining many SpiNNaker chips.
		
		\begin{figure}
			\center
			\buildfig{figures/spinnaker-packet.tex}
			
			\caption{SpiNNaker's \SI{40}{\bit} and \SI{72}{\bit} multicast packet
			format.}
			\label{fig:spinnaker-packet}
		\end{figure}
		
		Processor cores can communicate by sending and receiving network packets
		forwarded by routers through the network. Since SpiNNaker's network is
		designed to transmit neural spike events efficiently, individual network
		packets are small, either \SI{40}{\bit} or \SI{72}{\bit} compared with tens
		or hundreds of byte packets in typical network architectures.
		
		In a real-time simulation, the time at which a spike is produced is
		implicitly indicated by the time it is received -- since at biological
		timescales a computer network delivers packets `instantaneously'.
		Consequently, the only information which must be explicitly encoded is the
		identity of the neuron which produced the spike. In SpiNNaker, a spike may
		be encoded by using a single \SI{40}{\bit} `multicast packet' whose format
		is illustrated in figure~\ref{fig:spinnaker-packet}.  The \SI{8}{\bit}
		header is used by SpiNNaker's routers to determine the type of packet and
		the \SI{32}{\bit} `routing key' is used to identify the neuron which
		produced the packet. The routing key is also used by SpiNNaker's routers to
		determine how the packet should be directed through the network.
		
		The optional \SI{32}{\bit} payload is not used by conventional spiking
		neural simulations \cite{galluppi10} but has been exploited to enable more
		efficient simulation of a particular class of neural models \cite{mundy15}.
	
	\section{The SpiNNaker router}
		
		The SpiNNaker router employs an unconventional design which, despite its
		compact size and small energy requirements, implements a flexible multicast
		routing scheme. Unlike conventional routers which often employ hard-coded
		routing rules \cite[chapter~8]{dally04}, the SpiNNaker router uses a
		programmable `routing table' to determine how packets should be forwarded.
		In addition, to avoid deadlocks, SpiNNaker's router employs a simple,
		timeout-based mechanism which exploits the ability of neural networks to
		tolerate occasional missing packets. As we will see in chapter
		\ref{sec:routing}, this mechanism greatly simplifies the task of routing in
		SpiNNaker's network. In this section we'll look at these features in
		greater detail.
		
		\subsection{Routing tables}
		
			When a multicast packet arrives at a SpiNNaker router (either from a
			local core or a neighbouring chip), the router looks up the routing key
			in its routing table. This table consists of \num{1024} programmable
			table entries, each specifying a routing key bit pattern and mask to
			match and a set of routes.  When a multicast packet's key is matched by a
			routing entry the packet is forwarded along every route specified by that
			entry, potentially duplicating the packet. This `multicast' technique
			allows packets to be transmitted once but received in a number of places
			while making efficient use of the network \cite{navaridas12}.
			
			Though routing table entries are in finite supply (\num{1024} entries per
			router), it is still possible for many thousands of traffic flows to be
			routed through a single router. The bit pattern and mask in each routing
			entry matches against the 32~bits of a routing key as either
			`\texttt{1}', `\texttt{0}' or `\texttt{X}' (don't care).  This means that
			a single routing entry may, for example, be used to match all routing
			keys with a certain prefix. If a routing key is not matched by any entry
			in the routing table then the packet is `default routed' in a straight
			line. For example if a packet with an unmatched key is received from the
			chip to the left, the packet will be default routed to the chip on the
			right. By assigning routing keys such that neurons whose spikes are sent
			to similar destinations share a similar prefix, the number of routing
			entries required by a simulation is greatly reduced \cite{davies12}.
			
			\begin{figure}
				\center
				\buildfig{figures/routing-example.tex}
				
				\caption[Multicast routing example.]%
				{Multicast routing example with \SI{4}{\bit} routing keys. Each
				box represents a SpiNNaker chip whose router has been programmed with
				the routing entries shown. Grey lines mark connections between chips.}
				\label{fig:routing-example}
			\end{figure}
			
			Consider the simplified example in figure~\ref{fig:routing-example} in
			which a number of (\SI{4}{\bit}) routing table entries have been
			configured in the routers of a small SpiNNaker network. If a packet with
			the routing key \texttt{1011} is transmitted by a core in the chip
			labelled $(0, 0, 0)$, this will match the first routing table entry on
			that chip and will be routed to chip $(1, 0, 0)$. On chip $(1, 0, 0)$,
			the packet once again matches the first routing entry and is routed to
			chip $(1, 0, -1)$. On $(1, 0, -1)$, no match is made so the packet is
			default routed to $(1, 0, -2)$. On this chip, the packet matches a
			routing entry which routes the packet to core~7. In this example, default
			routing allows only three routing table entries to direct a packet
			through four chips.
			
			As a second example, if a packet with the routing key \texttt{0010} is
			transmitted by a core on chip $(0, 0, 0)$, this key will be matched by
			the second routing entry since \texttt{X}s in the table entry will match
			both \texttt{1}s and \texttt{0}s in the corresponding bits of the routing
			key. When the packet arrives at chip $(0, 0, -1)$ the matching routing
			entry forwards the packet to both $(0, 1, -1)$ and $(1, 0, -1)$
			simultaneously. The copy of the packet arriving at $(0, 1, -1)$ is routed
			to core~5 on that chip.  Meanwhile, the copy forwarded to $(1, 0, -1)$ is
			duplicated again with one copy being routed to core~11 and another being
			routed to chip $(1, 0, -2)$. Here the packet is finally delivered to
			core~6. In this example, the ability of the router to multicast
			(duplicate) packets as they pass through the network meant that sending
			one copy of the packet was sufficient to reach three destination cores.
			In addition, by using \texttt{X}s in the routing table entry, the same
			routing entries are sufficient to route packets with the keys
			\texttt{0000}, \texttt{0001}, \texttt{0010} and \texttt{0011}.
			
			In spite of these mechanisms, it is still possible for an application to
			run out of routing table entries. As we will see in
			chapter~\ref{sec:placement} by arranging applications appropriately
			within SpiNNaker's network, routing table usage can be reduced. In
			addition, other behaviours of SpiNNaker's router may be exploited to
			compress an applications routing tables further, however the techniques
			employed are beyond the scope of this thesis \cite{mundy16}.
		
		\subsection{Timeouts}
			
			SpiNNaker's router is built on a pipeline architecture. As shown in
			figure~\ref{fig:router-architecture}, the router is fed packets by an
			arbiter which serialises packets arriving from other chips and local
			cores. Every (\SI{100}{\mega\hertz}) clock cycle, the router pipeline
			accepts one packet from the arbiter and routes a packet to one or several
			output links. If any of the required output ports are busy then the
			packet is not forwarded to any output link and the pipeline stalls. Once
			a packet has been blocked for a programmable timeout, it is dropped
			(discarded) and routing continues as usual for next packet in the
			pipeline. Links become blocked while transmitting packets or waiting for
			the remote receiver to become ready. For example, a receiving processor
			core may be busy performing some computation or a receiving router may be
			blocked waiting for some of its outputs to become ready.
			
			\begin{figure}
				\center
				\buildfig{figures/router-architecture.tex}
				
				\caption{SpiNNaker router architecture}
				\label{fig:router-architecture}
			\end{figure}
			
			The timeout-based packet dropping mechanism is designed to defuse
			deadlocks in the network. For example, if two routers are trying to send
			each other a packet at the same time they may become deadlocked, each
			waiting for the other router to accept a packet before continuing.
			SpiNNaker's timeout mechanism breaks deadlocks by dropping packets which
			have been blocked for some time and therefore may be in a deadlock.  Once
			a packet has been dropped it is left to software to either tolerate the
			missing packet or trigger a retransmission. In neural simulations, as in
			biology, the loss of a single spike is unlikely to have a significant
			impact on the behaviour of a neural model and therefore these simulations
			are inherently tolerant of occasional dropped packets. During application
			loading and other system tasks, a higher level, software driven protocol
			based on acknowledgements and retransmissions is used to ensure
			guaranteed delivery.
			
			% TODO: MENTION TIMEOUT VALUE USED?
			% Router timeouts must be configured to be long enough that delays in
			% packet transmission, for example due to the time taken for packets to
			% traverse a link, do not trigger packet dropping. Conversely, the timeout
			% should be as short as possible to reduce the time the router is
			% blocked and maximise network throughput.
	
	\section{The hexagonal torus topology}
		
		Each SpiNNaker chip is a node in a `hexagonal torus topology' as
		illustrated in figure~\ref{fig:hexagonalTorusTopology}. Network packets
		sent by SpiNNaker's processor cores may `hop' through several nodes in the
		network to reach their intended destination. In each hop, a packet may
		advance one node along one of the three axes of the topology. For example,
		a packet sent by the node labelled $\alpha$ (in the bottom-left corner) to
		the node labelled $\beta$, might take the following sequence of hops:
		X$^+$, X$^+$, Z$^-$. Packets sent from $\alpha$ to $\gamma$ might take the
		route: X$^-$, X$^-$, Y$^+$, Y$^+$. The first hop of this route `wraps
		around' from the bottom-left node to the bottom-right node in a single hop.
		
		\begin{figure}
			\center
			\buildfig{figures/hexagonalTorusTopology.tex}
			
			\caption[A hexagonal torus topology.]%
			{A hexagonal torus topology. Each hexagon represents a node (i.e.
			a SpiNNaker chip). Touching nodes are directly connected. Nodes on edges
			$a$, $b$ and $c$ are also directly connected to the corresponding nodes
			on edges $a'$, $b'$ and $c'$, respectively. The three axes of the
			hexagonal torus topology, `X', `Y' and `Z' are also shown.}
			\label{fig:hexagonalTorusTopology}
		\end{figure}
		
		\begin{figure}
			\center
			\begin{subfigure}{0.39\linewidth}
				\center
				\includegraphics[width=\linewidth]{figures/torus-3d-flat.pdf}
				\caption{}
				\label{fig:torus-3d-flat}
			\end{subfigure}
			~~
			\begin{subfigure}{0.26\linewidth}
				\center
				\includegraphics[width=\linewidth]{figures/torus-3d-tube.pdf}
				\caption{}
				\label{fig:torus-3d-tube}
			\end{subfigure}
			~~
			\begin{subfigure}{0.23\linewidth}
				\center
				\includegraphics[width=\linewidth]{figures/torus-3d-torus.pdf}
				\caption{}
				\label{fig:torus-3d-torus}
			\end{subfigure}
			
			\caption{Visualisation of a hexagonal torus topology as a torus.}
			\label{fig:torus-3d}
		\end{figure}
		
		The wrap around connections in the topology are what give it the `torus'
		part of its name. Figure~\ref{fig:torus-3d-flat} shows a hexagonal torus
		topology drawn flat as in the previous figure. If the topology is rolled up
		into a tube such that the top and bottom nodes become directly adjacent, a
		tube is formed as in figure~\ref{fig:torus-3d-tube}. This tube can then be
		bent to bring together the nodes at the ends of the tube to form a torus as
		shown in figure~\ref{fig:torus-3d-torus}.
		
		A hexagonal torus topology is typically defined in terms of its width and
		height along the X and Y axes respectively. For example,
		figure~\ref{fig:hexagonalTorusTopology} shows a $10\times10$ hexagonal
		torus.  The nodes in a hexagonal torus topology are addressed using
		hexagonal coordinates of the form $(x, y, z)$ \cite{patel15}. The bottom
		left node (labelled $\alpha$ in the figure) has the coordinate $(0, 0, 0)$
		and other nodes are assigned coordinates according to the number of hops
		along each dimension from $(0, 0, 0)$, for example node $\beta$ has the
		coordinate $(2, 0, -1)$. Because the hexagonal torus topology's axes are
		non-orthogonal, it is possible to define several coordinates for the same
		location. For example $(3, 1, 0)$ and $(1, -1, -2)$ are also valid
		coordinates for node $\beta$. These dual coordinates emerge from the fact
		that adding $(1, 1, 1)$ to a coordinate produces an equivalent, but
		different, coordinate. This phenomenon is explained in detail in
		appendix~\ref{app:minimal-hex-coordinates} and related phenomena will be
		discussed in chapter~\ref{sec:shortestPaths}.
		
		The hexagonal torus topology was chosen over a more conventional network
		topology -- such as a 2D or 3D torus (sometimes known as a 2-ary $N$-cube
		or 3-ary $N$-cube respectively) \cite[chapters~3~and~5]{dally04} -- due to
		its balance of theoretical performance and practicality. The bisection
		bandwidth of a topology indicates the theoretical worst-case total
		throughput the network is able to sustain \cite[chapter~1]{dally04}.  In
		networks with homogeneous link throughput, bisection bandwidth is
		determined by the number of links cut by a balanced bisection of the
		network.  Figure~\ref{fig:bisection-bandwidth} illustrates the bisections
		of several torus topologies.
		
		\begin{figure}
			\center
			\begin{subfigure}[b]{0.3\linewidth}
				\center
				\buildfig{figures/bisection-bandwidth-2d.tex}
				
				\caption{2D Torus}
				\label{fig:bisection-bandwidth-2d}
			\end{subfigure}
			\begin{subfigure}[b]{0.3\linewidth}
				\center
				\buildfig{figures/bisection-bandwidth-hex.tex}
				
				\caption{Hexagonal Torus}
				\label{fig:bisection-bandwidth-hex}
			\end{subfigure}
			\begin{subfigure}[b]{0.3\linewidth}
				\center
				\buildfig{figures/bisection-bandwidth-3d.tex}
				
				\caption{3D Torus}
				\label{fig:bisection-bandwidth-3d}
			\end{subfigure}
			
			\caption[Bisections of torus topologies.]%
			{Bisections of torus topologies. Connections cut by the bisection
			are drawn as lines.}
			\label{fig:bisection-bandwidth}
		\end{figure}
		
		In a $N \times N$ 2D torus topology, the bisection bandwidth is $2N$~links
		and each node requires four links. The hexagonal torus topology requires
		six links per node but provides double bisection bandwidth ($4N$~links).
		The 3D torus topology also requires six links per node but by connecting
		the nodes differently achieves a bisection bandwidth of $8N$~links.  The 3D
		torus topology, however, comes at a price -- when immersed into the
		(approximately) 2D space provided by a large machine room or row of server
		cabinets, some connections require long cables. By contrast, the 2D and
		hexagonal torus topologies are both inherently two dimensional and
		consequently do not suffer from this effect. The hexagonal torus topology,
		therefore, shares the practicality of construction of a 2D torus while
		still gaining some of the performance of a 3D torus topology. In addition,
		because nodes in a hexagonal torus topology have a greater number of links,
		greater redundancy is available in the network to tolerate faults.
		
		Most torus topologies, including hexagonal, 2D and 3D toruses, have a
		related `mesh' topology. These mesh topologies maintain the same general
		connectivity structure as their torus topologies but omit wrap-around
		links. In practice, this saves a small number of links at the expense of
		halving the network's bisection bandwidth.  Because of their poorer
		performance, mesh networks are rarely used as the basis of a network
		architecture. Mesh networks, however, are occasionally formed when a
		network is partitioned into several smaller sub-networks to allow multiple
		users to share a system \cite{spalloc16}.
		
		\begin{figure}
			\center
			\begin{subfigure}[b]{0.45\linewidth}
				\center
				\buildfig{figures/hexagonal-torus.tex}
				\caption{Hexagonal torus}
				\label{fig:topo-compare-hexagonal-torus}
			\end{subfigure}
			\begin{subfigure}[b]{0.45\linewidth}
				\center
				\buildfig{figures/h-torus.tex}
				\caption{H-torus}
				\label{fig:topo-compare-h-torus}
			\end{subfigure}
			
			\caption[Hexagonal torus vs. H-torus topology.]%
			{Hexagonal torus vs. H-torus topology. Each numbered hexagon
			represents a node. The thick outline indicates the bounds of the
			topology after which the network repeats. In each topology, the path
			taken by advancing in the Y$^+$ direction from the node labelled `0' is
			shown.}
			\label{fig:topo-compare}
		\end{figure}
		
		\label{sec:hex-vs-h-torus}
		
		The hexagonal torus topology is not to be confused with the `H-torus'
		topology. This topology also uses a hexagonal tiling of nodes and even
		wraps this tiling into a torus-like topology \cite{zhao08}. However,
		H-torus topologies have very different characteristics to the hexagonal
		torus topology and are related to `twisted torus' topologies
		\cite{camara10}. For example, figure~\ref{fig:topo-compare} illustrates one
		major difference in the way paths wrap around the peripheries of both
		topologies.
	
	\section{Scaling-up SpiNNaker machines}
		
		To build large SpiNNaker systems comprising of tens of thousands of
		SpiNNaker chips, groups of forty-eight chips are mounted onto circuit
		boards as illustrated in figure~\ref{fig:spinnakerBoard}. These boards may
		be connected together to form larger systems.  Figure~\ref{fig:threeboard}
		shows a prototype three board system. Though the chips are
		\emph{physically} arranged in a (nearly) $7\times7$ grid on each SpiNNaker
		board, they logically form a hexagonal `wrapped triple'
		\cite{davidsonWiring} (see appendix~\ref{sec:partitioning}) which logically
		fit together as illustrated in figure~\ref{fig:threeboard-separate}. The
		labelled exposed corners of the three forty-eight chip boards connect
		together to form a $12\times12$ hexagonal torus topology as illustrated in
		figure~\ref{fig:threeboard-wrapped}. Larger SpiNNaker machines are
		assembled by combining more boards.
		
		\begin{figure}
			\center
			\begin{subfigure}[b]{0.45\linewidth}
				\center
				\includegraphics[width=\linewidth]{figures/spinnakerBoard.jpg}
				
				\caption{A SpiNNaker board}
				\label{fig:spinnakerBoard}
			\end{subfigure}
			~~~
			\begin{subfigure}[b]{0.45\linewidth}
				\center
				\includegraphics[width=\linewidth]{figures/threeboard.jpg}
				
				\caption{Three board prototype}
				\label{fig:threeboard}
			\end{subfigure}
			
			\vspace*{1em}
			
			\begin{subfigure}[b]{0.45\linewidth}
				\center
				\buildfig{figures/threeboard-separate.tex}
				
				\caption{Three board topology}
				\label{fig:threeboard-separate}
			\end{subfigure}
			~~~
			\begin{subfigure}[b]{0.45\linewidth}
				\center
				\buildfig{figures/threeboard-wrapped.tex}
				
				\caption{\ldots{}as a parallelogram}
				\label{fig:threeboard-wrapped}
			\end{subfigure}
			
			\caption{SpiNNaker boards and their topology.}
			\label{fig:spinnaker-boards}
		\end{figure}
		
		
		SpiNNaker chips on the same circuit board connect using low power links
		requiring sixteen wires each.  If this link technology were used to connect
		chips on neighbouring boards, each pair of boards would need to be
		connected with a 128~wire cable.  Cables and connectors supporting this
		many signals are expensive, unreliable and physically large. Instead,
		chip-to-chip connections between boards are multiplexed and demultiplexed
		onto a single High-Speed Serial (HSS) link \cite{athavale05} carried via
		commodity S-ATA cables which are often used to connect hard disks in
		desktop computers and servers \cite{sata3spec}. The six high-speed links
		are implemented by three onboard FPGAs (the three large chips at the top of
		the SpiNNaker board) and are logically transparent to the underlying
		network. The underlying technology and the choice of S-ATA cables limits
		each board-to-board connection to spanning at most one metre gaps. In
		chapter~\ref{sec:building} I present a cabling scheme for hexagonal torus
		topologies which enable large SpiNNaker systems to be assembled using only
		short cables between boards.
		
	\section{Conclusions}
		
		The SpiNNaker architecture has been designed to enable the simulation of
		large biologically realistic neural models in real time. To support this,
		its network architecture takes on an unconventional design based on a
		custom router and hexagonal torus topology. In the remainder of this
		thesis, I will tackle a number of the challenges in scaling up the
		SpiNNaker architecture outlined in this chapter.

	\chapter{Building large SpiNNaker machines}
	
	Like any super computer, physically putting together a large SpiNNaker
	machine poses many challenges in terms of organisation, assembly and
	maintainance. One of the key tasks in this process is the installation of
	network cables such that a desired overall network topology is constructed.
	The largest planned SpiNNaker machine will use \num{3600} S-ATA
	\cite{sata3spec} cables to interconnect its \num{1200} circuit boards,
	creating a hexagonal torus topology. Since the machine will be installed
	within standard server room cabinets (which are not available in a
	giant-doughnut form-factor) a mapping from a board's logical location in the
	network topology to its physical location must be constructed. In addition,
	the interconnect technology employed by SpiNNaker restricts the length of
	S-ATA cables used to $\le$~\SI{1}{\meter}, constraining the possible mappings
	used. In addition the practical issues of installation complexity and
	maintainance must be considered since all \num{3600} cables must ultimately
	be installed and maintained by human operators.
	
	In this chapter I describe a novel technique for physically laying out
	machines configured in hexagonal torus topologies, such as SpiNNaker, in
	commercial machine rooms, building on the techniques used in more
	conventional torus topologies. In addition, I also propose a new methodology
	for installing and maintaining super computer cabling which which exploits
	existing diagnostic features of the SpiNNaker hardware to interactively guide
	and validate cable installation. Finally, I demonstrate how these new
	techniques have been used successfully to interconnect a prototype
	\num{518400} core SpiNNaker machine in substantially less time than the
	industry norm.
	
	In this chapter, the term \emph{unit} refers to the smallest physical
	ecomponent between which connections connections are to be made. For example,
	in a SpiNNaker machine a unit is a 48-chip board while in data center, a unit
	might be a server blade.
	
	\section{Related work}
		
		In this section I describe the techniques conventionally employed when
		laying out and interconnecting the units within super computers. Due to
		SpiNNaker's hexagonal torus topology and dense physical packing of units,
		these existing techniques are found to be insufficient. In the remainder of
		the chapter we will explore solutions to the limitations exposed below.
		
		\subsection{Avoiding long cables}
			
			Na\"ive arrangements of torus topologies, including hexagonal torus
			topologies, feature long `wrap-around' connections which connect units at
			the peripheries of the system. These connections can be problematic for
			numerous reasons:
			
			\begin{description}
				
				\item[Performance] Signal quality diminishes as cables get longer,
				requiring the use of slower signalling speeds, increased error
				correction overhead or more complex hardware.
				
				\item[Energy] Longer cables require higher drive strengths and/or
				buffering to maintain signal integrity.
				
				\item[Cost] Cost Shorter cables are cheaper than long ones.  Longer
				cables imply more wire in a given space making the tasks of routing or
				cable installation more difficult increasing labour costs by as much as
				$5\times$ \cite{curtis12}.
				
			\end{description}
			
			In conventional torus topologies the need for long cables is eliminated
			by folding and interleaving units of the network \cite{dally04}. For
			example, for a 1D torus topology (a ring network), one long connection
			exists to connect the two opposite sides of the system. To remove these
			long connections, half the units are `folded' on top of the others and
			then this arrangement of units is interleaved as illustrated in figure
			\ref{fig:ring-folding}.
			
			\begin{figure}
				\center
				\begin{subfigure}[b]{0.39\linewidth}
					\center
					\buildfig{figures/ring-folding-row.tex}
					\caption{A ring network}
					\label{fig:ring-folding-row}
				\end{subfigure}
				\begin{subfigure}[b]{0.24\linewidth}
					\center
					\buildfig{figures/ring-folding-folded.tex}
					\caption{Folded}
					\label{fig:ring-folding-folded}
				\end{subfigure}
				\begin{subfigure}[b]{0.35\linewidth}
					\center
					\buildfig{figures/ring-folding-interleaved.tex}
					\caption{Folded and interleaved}
					\label{fig:ring-folding-interleaved}
				\end{subfigure}
				
				\caption{Folding and interleaving a ring network to reduce maximum wire
				length.}
				\label{fig:ring-folding}
			\end{figure}
			
			Folding and interleaving has the effect of approximately doubling the
			average cable length but also eliminates the need for a cable spanning
			the entire system. Since the mean cable length is typically already
			short, doubling it in exchange for a substantially reduced maximum cable
			length is often preferable.
			
			The folding and interleaving process may be extended to $N$-dimensional
			torus topologies by folding each dimension in turn. Since all dimensions
			are orthogonal, the folding process only moves units in the dimension
			being folded. In the hexagonal torus topology, however, the three
			dimensions are non-orthogonal and thus folding in one dimension also
			moves units in other dimensions, preventing the edges of the torus
			meeting as illustrated in figure \ref{fig:failing-to-fold-hex-toruses}.
			
			\begin{figure}
				\center
				\begin{subfigure}[b]{0.24\linewidth}
					\center
					\buildfig{figures/failing-to-fold-hex-toruses-none.tex}
					\caption{Not folded}
					\label{fig:failing-to-fold-hex-toruses-none}
				\end{subfigure}
				\begin{subfigure}[b]{0.24\linewidth}
					\center
					\buildfig{figures/failing-to-fold-hex-toruses-x.tex}
					\caption{X}
					\label{fig:failing-to-fold-hex-toruses-x}
				\end{subfigure}
				\begin{subfigure}[b]{0.24\linewidth}
					\center
					\buildfig{figures/failing-to-fold-hex-toruses-y.tex}
					\caption{Y}
					\label{fig:failing-to-fold-hex-toruses-y}
				\end{subfigure}
				\begin{subfigure}[b]{0.24\linewidth}
					\center
					\buildfig{figures/failing-to-fold-hex-toruses-z.tex}
					\caption{Z}
					\label{fig:failing-to-fold-hex-toruses-z}
				\end{subfigure}
				
				\caption{Schematics showing hexagonal torus topologies folded along
				each of their non-orthogonal dimensions. Note that folding along
				the Z axis brings the \emph{wrong} edges closer together.}
				\label{fig:failing-to-fold-hex-toruses}
			\end{figure}
		
		\subsection{Cabling installation}
			
			Existing machine room installations feature very repetitive cabling
			patterns which can easily be memorised by a human technician. For example
			in BlueGene super computers the connectivity between units is highly
			regular \cite{lakner07} while in data centre networks cabling often
			centres around a small number of high-port-count switches
			\cite{cisco07,csernai15}. Cable installation is usually only aided by
			the labelling of connectors and sockets in a standardised manner
			\cite{tia2006} such as in figure \ref{fig:bgWiring}.
			
			\begin{figure}
				\center
				\begin{subfigure}[t]{0.5\textwidth}
					\begin{tikzpicture}
						\node (cables) [inner sep=0]
						      {\includegraphics[width=\textwidth]{figures/bgCables.png}};
						\node (sockets) [inner sep=0, below=1.0em of cables]
						      {\includegraphics[width=\textwidth]{figures/bgSockets.png}};
						
						% Point at label on cable
						\draw [white, <-, line width=0.4em]
						      ([shift={(0.7cm, -0.3cm)}]cables.center)
						      -- ++(45:1cm);
						
						% Point at label on socket
						\draw [white, <-, line width=0.4em]
						      ([shift={(-1.0cm, 1.1cm)}]sockets.center)
						      -- ++(-45:1cm);
					\end{tikzpicture}
					
					\caption{Pre-labelled cables and sockets}
					\label{fig:bgWiringLabels}
				\end{subfigure}
				~
				\begin{subfigure}[t]{0.30\textwidth}
					\includegraphics[height=6.15cm]{figures/bgWiring.jpg}
					
					\caption{Installation of cables}
					\label{fig:bgWiringInstallation}
				\end{subfigure}
				
				\caption{BlueGene/Q cable installation \cite{cscs13}}
				\label{fig:bgWiring}
			\end{figure}
			
			Despite the regularity and careful labelling of cables, the cost of
			installation and maintenance alone can be significant with costs in the
			range of \$45-95 per \SI{1}{\meter} cable run and \$100-400 for runs of
			\SI{10}{\meter} reported in the literature \cite{mudigonda11}. Much of
			this cost is due to the care required during installation to avoid
			miswiring and ensure that cooling airflow is not hampered by cable runs
			\cite{cisco07}.
			
			Many researchers have attempted to control cable installation costs by
			trying to reduce the number or length of cables required by developing
			alternative network topologies \cite{curtis12, popa10, mudigonda11}.
			Unfortunately, these techniques do not apply to SpiNNaker since its
			network topology is fixed.
			
			Some super computers make use of large custom `midplane` PCBs in place of
			cables to interconnect units within a cabinet and thus simplify the task
			of cable installation \cite{prickett10}. This scheme can greatly reduce
			wiring complexity since only coarser-grain cabinet-to-cabinet
			connectivity is provided by cables. Unfortunately this technique is
			expensive and also constrains the dimensions of the network topology
			supported by the machine. Since the SpiNNaker platform is designed to
			scale from desktop machines to machine-room installations, this scheme is
			not practical.
	
	\section{Folding \& interleaving hexagonal toruses}
		
		The first step towards a practical machine-room installation of a large
		machine using a hexagonal torus topology is to find an arrangement of
		boards between which cable lengths are minimised. In this section I
		describe a sequence of transformations which map the positions of units in
		a hexagonal torus topology onto a regular rectangular grid which may be
		folded and interleaved to eliminate long wires. It is worth emphasising
		that this transformation only affects the \emph{physical} positions of
		units and \emph{not} their connectivity.
		
		As described earlier in \S\ref{sec:parititioning} (page
		\pageref{sec:parititioning}), hexagonal torus topologies may be partitioned
		into units containing wrapped-triples of nodes. For example, in SpiNNaker,
		chips (nodes) are partitioned into circuit boards (units) containing 48
		chips. For completeness, this section describes the process of folding both
		systems whose units are individual nodes and those whose units are
		wrapped-triples.
		
		The transformation process is divided into two parts, each described
		separately in this section.
		
		\begin{description}
			
			\item[Parallelogram to rectangle] Units of the system are transformed
			from a parallelogram shape to a rectangular shape.
			
			\item[Uncrinkle] Units within the rectangle are moved such that they all
			lie on a regular (and fully packed) 2D grid.
			
		\end{description}
		
		\subsection{Parallelogram to rectangle}
			
			The hexagonal torus topology is most naturally drawn as a parallelogram
			as illustrated in figures \ref{fig:hex-to-plane-node-native} and
			\ref{fig:hex-to-plane-native}. Two transformations are presented which
			transform these arangements of units into a rectangular form: shearing
			and slicing.
			
			A \SI{30}{\degree} shear transformation distorts networks such that the X
			and Y axes become orthogonal leading to a rectangular arrangement of
			units as illustrated in figures \ref{fig:hex-to-plane-node-shear} and
			\ref{fig:hex-to-plane-shear}.
			
			The slice transformation slices the units protruding from the
			left-hand-side of the parallelogram and moves them into the matching gap
			on the opposite side of the parallelogram as illustrated in figures
			\ref{fig:hex-to-plane-node-slice} and \ref{fig:hex-to-plane-slice}.
			 
			While the shear transformation introduces some distortion causing cables
			in the Z dimension to become $\sqrt{2}\times$ longer it leaves the
			pattern of wrap-around connections remains unchanged. By contrast, the
			slice transformation does not elongate any cables but changes the pattern
			of wrap-around connections. The exact pattern wrap-around connections
			produced when slicing depends on the aspect ratio of the network as
			illustrated in \ref{fig:slicing-examples} and influences the choice of
			folding technique applied as described later.
			
			\begin{figure}
				\center
				\begin{subfigure}[b]{0.32\linewidth}
					\center
					\buildfig{figures/hex-to-plane-node-native.tex}
					
					\caption{$7 \times 7$ node torus}
					\label{fig:hex-to-plane-node-native}
				\end{subfigure}
				\begin{subfigure}[b]{0.32\linewidth}
					\center
					\buildfig{figures/hex-to-plane-node-shear.tex}
					
					\caption{Sheared}
					\label{fig:hex-to-plane-node-shear}
				\end{subfigure}
				\begin{subfigure}[b]{0.32\linewidth}
					\center
					\buildfig{figures/hex-to-plane-node-slice.tex}
					
					\caption{Sliced}
					\label{fig:hex-to-plane-node-slice}
				\end{subfigure}
				
				\caption{Transformations of hexagonal toruses of nodes into a
				rectangular form. Thin lines show wrap-around links. Pointy-topped
				hexagons represent individual nodes.}
				\label{fig:hex-to-plane-node}
			\end{figure}
			
			\begin{figure}
				
				\begin{subfigure}[b]{0.32\linewidth}
					\center
					\buildfig{figures/hex-to-plane-native.tex}
					
					\caption{$4 \times 4$ triad torus}
					\label{fig:hex-to-plane-native}
				\end{subfigure}
				\begin{subfigure}[b]{0.32\linewidth}
					\center
					\buildfig{figures/hex-to-plane-shear.tex}
					
					\caption{Sheared}
					\label{fig:hex-to-plane-shear}
				\end{subfigure}
				\begin{subfigure}[b]{0.32\linewidth}
					\center
					\buildfig{figures/hex-to-plane-slice.tex}
					
					\caption{Sliced}
					\label{fig:hex-to-plane-slice}
				\end{subfigure}
				
				\caption{Transformations of hexagonal toruses of wrapped triples into a
				rectangular form.  Thin lines show wrap-around links. Flat-topped
				hexagons represent a wrapped triple of nodes.}
				\label{fig:hex-to-plane}
			\end{figure}
			
			\begin{figure}
				\center
				\buildfig{figures/slicing-examples.tex}
				\caption{Patterns of wiring in sliced systems of various sizes.}
				\label{fig:slicing-examples}
			\end{figure}
			
		\subsection{Uncrinkling}
			
			Though the transformmation step yields rectangular arrangements of units,
			these arrangements do not fall onto a regular 2D grid, with the exception
			of the shear transform on individual nodes. Figure \ref{fig:uncrinkling}
			illustrates how the various arrangements of hexagons may be moved to
			`uncrinkle' the units into a regular grid.
			
			\begin{figure}
				\center
				\begin{subfigure}[b]{0.44\linewidth}
					\center
					\buildfig{figures/uncrinkling-node-sheared.tex}
					
					\caption{$7 \times 7$ nodes, sheared}
					\label{fig:uncrinkling-node-sheared}
				\end{subfigure}
				\begin{subfigure}[b]{0.44\linewidth}
					\center
					\buildfig{figures/uncrinkling-node-sliced.tex}
					
					\caption{$7 \times 7$ nodes, sliced}
					\label{fig:uncrinkling-node-sliced}
				\end{subfigure}
				
				\vspace{1cm}
				
				\begin{subfigure}[b]{0.44\linewidth}
					\center
					\buildfig{figures/uncrinkling-sheared.tex}
					
					\caption{$4 \times 4$ triples, sheared}
					\label{fig:uncrinkling-sheared}
				\end{subfigure}
				\begin{subfigure}[b]{0.44\linewidth}
					\center
					\buildfig{figures/uncrinkling-sliced.tex}
					
					\caption{$4 \times 4$ triples, sliced}
					\label{fig:uncrinkling-sliced}
				\end{subfigure}
				
				\vspace{1em}
				
				\caption{Mapping rectangular arrangements of units into a square grid.
				Thick lines show how layers of units are uncrinkled.  Annotations show
				how the relative positions of nodes and wrapped triples change after
				uncrinkling.}
				\label{fig:uncrinkling}
			\end{figure}
			
			In the figure, the numbered units enumerate the different positions on
			the crinkle and those labelled alphabetically are those that immediately
			surround them. From this we can observe that uncrinkling largely
			preserves spatial locality but some distortion is introduced, separating
			previously neighbouring units. For example, in figure
			\ref{fig:uncrinkling-sheared}, the units labelled `1' and `i' are
			neighbours before uncrinkling but are separated by a (Euclidean) distance
			of $\sqrt{5}$ afterwards. Note that the distortion introduced depends on
			what part of the crinkle is considered, for example `2' and `a' have
			distance 2 but are logically connected in the same way.
		
		\subsection{Folding and Interleaving}
			
			Once a regular grid of units has been formed, this may be folded in the
			conventional way, eliminating long cables crossing from left-to-right and
			top-to-bottom as illustrated in \ref{fig:folding-sheared}.
			
			Unfortunately, for sliced systems whose dimensions are not of the ratio
			$1:2$, the pattern of wrap-around cables may also include some cables
			which do not cross directly to the opposite side of the system (refer
			back to figure \ref{fig:slicing-examples}). As a result of these
			connections, folding does not successfully eliminate all long
			connections. An exception to this rule is sliced systems whose dimensions
			are in the ratio $1:1$ where folding twice along the Y axis may
			successfully eliminate all wrap-around connections as illustrated in
			\ref{fig:folding-sliced}.
			
			\begin{figure}
				\begin{subfigure}{\linewidth}
					\center
					\buildfig{figures/folding-sheared.tex}
					\caption{$N \times M$ sheared systems and $N \times 2N$ sliced systems}
					\label{fig:folding-sheared}
				\end{subfigure}
				
				\vspace{1em}
				
				\begin{subfigure}{\linewidth}
					\center
					\buildfig{figures/folding-sliced.tex}
					\caption{$N \times N$ sliced systems}
					\label{fig:folding-sliced}
				\end{subfigure}
				
				\caption{Schematic illustrating elimination of long wrap-around links
				during folding. In each example a single link has been highlighted to
				aid in following the process.}
				\label{fig:folding}
			\end{figure}
			
			Once folded, the 2D grid is straight-forwardly interleaved as illustrated
			previously in figure \ref{fig:ring-folding}. The interleaving process
			introduces some additional distortion to the layout of units and causes
			most connections to become twice as long. For sliced $1:1$ systems, the
			additional fold results in additional overhead during interleaving since
			four layers of the system are interleaved.
		
		\subsection{Mapping to Cabinets}
			
			In the final step of the process is to map the 2D grid of units into
			positions in machine room cabinets as illustrated in figure
			\ref{fig:million-core-machine}. As illustrated in figure
			\ref{fig:cabinetisation}, first the grid of units is partitioned into
			groups of columns, one per cabinet, then groups of rows one per frame per
			cabinet. The units in each group are then allocated to slots within a
			frame, interleaving the rows of the groups as shown.
			
			\begin{figure}
				\center
				\buildfig{figures/cabinet-units.tex}
				
				\caption{An illustration of the physical construction of a
				multi-cabinet SpiNNaker system. (Note: network cables \emph{not}
				installed.)}
				\label{fig:cabinet-units}
			\end{figure}
			
			\begin{figure}
				\center
				\buildfig{figures/cabinetisation.tex}
				
				\caption{Mapping from 2D space to cabinets, frames and boards.}
				\label{fig:cabinetisation}
			\end{figure}
		
	\section{Cable installation}
		
		Cable installation is performed by a team of (human) technicians who must
		ensure that all network cables are correctly installed. As illustrated in
		previously in figure \ref{fig:cabinet-units}, the density of SpiNNaker's
		units, combined with the nature of the hexagonal torus topology, poses a
		challenge. To address this challenge I propose a semi-automated approach to
		cable installation which exploits diagnostic facilities available in the
		majority of super computers in order to guide technicians through the
		cabling process, interactively guiding installation and maintenance.
		
		\subsection{Interactive technician guidance and validation}
			
			While automated systems for validating cabling correctness are
			commonplace, these systems are typically used only after cabling has been
			completed \cite{lakner07}. As with other large-scale machines, SpiNNaker
			includes a low-bandwidth system management bus which may be used to
			interrogate network hardware and control diagnostic LEDs prior to the
			installation of the main SpiNNaker network interconnect.  Using these
			facilities I have constructed a tool called SpiNNer which interactively
			guides a technician, or team of technicians, through the cable
			installation process, validating each connection in real-time.
			
			Diagnostic LEDs mounted on each SpiNNaker board (figure
			\ref{fig:interactive-wiring-guide-leds}) are used to indicate the
			endpoints of the cable currently being installed. Simultaneously a
			Text-To-Speech (TTS) system gives an audible indication of which cable
			type is to be used and location of each connection.  Additionally, a GUI
			via a computer display (figure \ref{fig:interactive-wiring-guide-gui}).
			The centre of the display shows a `big-picture' perspective of the
			locations of the boards to be connected. The detailed views on the left
			and right indicate which of the six sockets on each board the cables
			should connect.
			
			\begin{figure}
				\center
				\begin{subfigure}[b]{0.40\textwidth}
					\begin{tikzpicture}
						\node (leds) [inner sep=0]
						      {\includegraphics[width=\textwidth]{figures/leds.jpg}};
						% Point at left LED
						\draw [white, <-, line width=0.4em]
						      ([shift={(-0.0cm, -0.6cm)}]leds.center)
						      -- ++(225:1cm);
						% Point at right LED
						\draw [white, <-, line width=0.4em]
						      ([shift={(1.1cm, -1.1cm)}]leds.center)
						      -- ++(225:1cm);
					\end{tikzpicture}
					
					\caption{Diagnostic LEDs}
					\label{fig:interactive-wiring-guide-leds}
				\end{subfigure}
				~
				\begin{subfigure}[b]{0.546\textwidth}
					\begin{tikzpicture}[thin, black!20!white]
						\node (screen) [inner sep=0]
						      {\includegraphics[width=\textwidth]{figures/wiring_guide_screenshot.png}};
						\draw (screen.south west) rectangle (screen.north east);
					\end{tikzpicture}
					
					\caption{Interactive wiring guide GUI}
					\label{fig:interactive-wiring-guide-gui}
				\end{subfigure}
				
				\caption{The SpiNNer interactive wiring guide uses a GUI,
				text-to-speech and diagnostic LEDs to assist during cable
				installation.}
				\label{fig:interactive-wiring-guide}
			\end{figure}
			
			SpiNNer also validates the connectivity of the system in real-time by
			polling the diagnostic interfaces of the network hardware at the
			endpoints of the cable being installed to determine if they are correctly
			connected. If a miswiring occurs, this is immediately detected and
			announced via TTS enabling the technician to immediately correct the
			error. Once a cable has been installed correctly, the software
			automatically advances to the next cable meaning direct interaction with
			the software by the technician is minimal. In practice, it is rarely
			necessary to refer to the GUI.
		
			SpiNNer presents the cables in an order intended to maximise ease of
			installation. Cables are installed in three groups with intra-frame
			cables being installed first, followed by intra-cabinet cables and
			inter-cabinet cables. Within each group, the tightest cables are
			installed first resulting in slacker cables naturally being installed
			over the top of already installed cables. By grouping cables in this
			manner, multiple technicians may work independently on the wiring within
			individual frames and cabinets.
			
			SpiNNer may also be used to repair or replace cables in the system.
			During maintenance, obstructing cables may be blindly removed alongside
			any cable being replaced. At the conclusion of the process, the wiring
			guide may be used to interactively guide re-installation of all removed
			cables.
		
		\subsection{Cable selection}
			
			Controlling slack is critical to ensuring reliable and maintainable
			cabling installations. If cables are too tight, cables and connectors can
			become easily damaged and when too slack, the excess cable obstructs
			other cables and can easily become tangled and damaged \cite{cisco07}. It
			has been observed that when ready-made cables are employed technicians
			frequently over-estimate the cable lengths required preferring to use
			overly long cables for all connections \cite{mazaris97}. To solve this
			problem, the SpiNNer wiring guide software dictates the cable lengths to
			be used by an installer based the rule of (three-)thumbs according to
			Mazaris \cite{mazaris97}. This rule suggests that an ideal amount of
			slack is approximately that which can be wrapped around three fingers.
			Specifically, the shortest available cable is selected which ensures at
			least \SI{5}{\centi\meter} of slack.
			
			The SpiNNer tool allocates cables assuming all cables take a Euclidean
			straight-line path between the endpoints of the connection. The result is
			that wiring is not routed through dedicated cable management structures
			but is simply suspended by its connectors in front of the machine. As
			demonstrated later, this unconventional approach leads neither to cooling
			problems nor increased maintenance effort.
	
	\section{Results and Evaluation}
		
		This stuff has been used and works. In this section I'll go over the
		overheads introduced by the various transformations and
		folding/interleaving steps and show a wiring scheme for a large machine
		which uses only short cables. I'll then show how SpiNNer was used to
		install this wiring plan into a very large machine without foobaring the
		cooling and in very little time. I'll also report on difficulty of
		maintenance.
		
		\subsection{Cable length}
			
			The transformation from regular hexagonal torus to a folded and
			interleaved form introduces some overhead to the cable lengths required.
			Using figure \ref{fig:uncrinkling} (page \pageref{fig:uncrinkling}), it
			is possible to compute the exact overhead introduced when each type of
			transformation proposed.
			
			For example, to compute the mean overhead introduced by the slicing
			technique when applied to units composed of wrapped triples, consider
			figure \ref{fig:uncrinkling-sliced}. The uncrinkling pattern used to
			transform this topology is a repeating pattern of two units, a pair of
			which have been labelled $1$ and $2$ respectively. Unit $1$ is
			immediately surrounded by six units labelled $a$, $b$, $c$, $2$, $g$ and
			$h$. Similarly, unit $2$ is surrounded by units $1$, $c$, $d$, $e$, $f$
			and $g$. Before the transformation, the distances, $D$, to each of these
			units is $1$ but after the transformation is applied, this is not always
			the case. Additionally, folding and interleaving introduce additional
			overhead. In this example, if the system is folded into $f_x$ columns and
			$f_y$ rows, the distances between previously neighbouring units become:
			
			\begin{equation*}
				\begin{aligned}[c]
					D_{1\,\leftrightarrow{}\,a} &= \sqrt{f_x^2 + f_y^2} \\
					D_{1\,\leftrightarrow{}\,b} &= f_y \\
					D_{1\,\leftrightarrow{}\,c} &= \sqrt{f_x^2 + f_y^2} \\
					D_{1\,\leftrightarrow{}\,2} &= f_x \\
					D_{1\,\leftrightarrow{}\,g} &= f_y \\
					D_{1\,\leftrightarrow{}\,h} &= f_x
				\end{aligned}
				\hspace{2cm}
				\begin{aligned}[c]
					D_{2\,\leftrightarrow{}\,1} &= f_x \\
					D_{2\,\leftrightarrow{}\,c} &= f_y \\
					D_{2\,\leftrightarrow{}\,d} &= f_x \\
					D_{2\,\leftrightarrow{}\,e} &= \sqrt{f_x^2 + f_y^2} \\
					D_{2\,\leftrightarrow{}\,f} &= f_y \\
					D_{2\,\leftrightarrow{}\,g} &= \sqrt{f_x^2 + f_y^2}
				\end{aligned}
			\end{equation*}
			
			From these values, the mean and maximum connection distances after
			folding and interleaving may be computed. Table
			\ref{tab:transform-overhead} gives the mean and maximum connection
			distances for each of the four transformations described in this chapter.
			
			\begin{table}
				\begin{subtable}[b]{\linewidth}
					\center
					\begin{tabular}{l c c}
						\toprule
						& Shear & Slice \\
						\addlinespace
						Nodes &
							$\frac{f_x + f_y + \sqrt{f_x^2 + f_y^2}}{3}$ &
							$\frac{f_x + f_y + \sqrt{f_x^2 + f_y^2}}{3}$ \\
						\addlinespace
						Triples &
							$\frac{7f_x + 3\sqrt{f_x^2 + f_y^2} + \sqrt{(2f_x)^2 + f_y^2}}{9}$ &
							$\frac{f_x + f_y + \sqrt{f_x^2 + f_y^2}}{3}$ \\
						\bottomrule
					\end{tabular}
					
					\caption{Mean}
					\label{tab:transform-overhead-mean}
				\end{subtable}
				
				\vspace{1em}
				
				\begin{subtable}[b]{\linewidth}
					\center
					\begin{tabular}{l c c}
						\toprule
						& Shear & Slice \\
						\addlinespace
						Nodes &
							$\sqrt{f_x^2 + f_y^2}$ &
							$\sqrt{f_x^2 + f_y^2}$ \\
						\addlinespace
						Triples &
							$\sqrt{(2f_x)^2 + f_y^2}$ &
							$\sqrt{f_x^2 + f_y^2}$ \\
						\bottomrule
					\end{tabular}
					
					\caption{Maximum}
					\label{tab:transform-overhead-max}
				\end{subtable}
				
				\caption{Overheads introduced when transforming unit positions onto a
				regular grid.}
				\label{tab:transform-overhead}
			\end{table}
			
			From these results it is evident that shearing and slicing networks
			whose units are nodes result in identical mean and maximum overhead in
			cable length when folded similarly. Since sliced networks may require
			folding more than once along each axis the shearing approach is
			preferable in general.
			
			For networks constructed from units of wrapped triples, the slicing
			approach suffers the same mean and maximum overhead has networks of
			nodes, and less overhead than shearing for the same number of folds. For
			systems with an aspect ratio of $1:2$ (where both slicing and shearing
			require $f_x = f_y = 2$), the slicing transformation yields lower mean
			and maximum overhead than shearing. For all other aspect ratios (where
			slicing requires a greater degree of folding) the shearing technique
			produces lower overhead. The recommended transformations for a given
			machine are thus given in table \ref{tab:transform-recommended}.
			
			\begin{table}
				\center
				\begin{tabular}{lcc}
					\toprule
					                         & $1:2$  & Other \\
					\addlinespace
					\multirow{2}{*}{Nodes}   & Either & Shear\\
					                         & \footnotesize $\mu\approx2.28 \quad \vee\approx2.83$
					                         & \footnotesize $\mu\approx2.28 \quad \vee\approx2.83$\\
					\addlinespace
					\multirow{2}{*}{Triples} & Slice  & Shear\\
					                         & \footnotesize $\mu\approx2.28 \quad \vee\approx2.83$
					                         & \footnotesize $\mu\approx3.00 \quad \vee\approx4.47$\\
					\bottomrule
				\end{tabular}
				
				\caption{Recommended transformation and folding scheme for different
				system types. $\mu$ and $\vee$ give the mean and maximum wire
				distortion introduced, respectively.}
				\label{tab:transform-recommended}
			\end{table}
			
			\begin{figure}
				\center
				\buildfig{figures/million-core-machine.tex}
				
				\caption{Cabling plan for a \num{1036800} core SpiNNaker
				machine's \num{3600} cables.}
				\label{fig:million-core-machine}
			\end{figure}
			
			Following folding and mapping to physical locations, the cabling plans
			for large machines require no large gaps to be spanned.  The largest
			planned SpiNNaker machine, illustrated in figure
			\ref{fig:million-core-machine}, will be \SI{6}{\meter} wide but the
			largest gap any cable must span is \SI{66}{\centi\meter}. This distance
			is well within the \SI{1}{\meter} allowed by the hardware and cables.
			
		\subsection{Installation practicality}
			
			\begin{table}
				\center
				\begin{tabular}{lrr@{$\,$}l}
					\toprule
						System & Number of Cables & \multicolumn{2}{r}{Installation time} \\
					\midrule
						24 boards  & \num{72}   & \num{10} & \si{\minute}         \\
						1 cabinet  & \num{360}  & \num{4}  & \si{\hour}$^\dagger$ \\
						2 cabinets & \num{720}  & \num{2}  & \si{\hour}           \\
						5 cabinets & \num{1800} & ?        &                      \\
					\bottomrule
				\end{tabular}
				
				\caption{Installation times for various sizes of machine.
				$\dagger$~This machine was installed without real-time validation of
				connectivity.}
				\label{tab:install-time}
			\end{table}
			
			A number of SpiNNaker machines of various scales have been assembled
			using the techniques described in this chapter ranging from single frames
			of 24 boards to a half-scale 5 cabinet machine. Table
			\ref{tab:install-time} gives the reported installation times of each of
			these machines.
			
			The single cabinet machine's installation time is notably
			disproportionate to its size. When this system was assembled, real-time
			connection validation was not yet available. As a result, though cable
			installation was rapid correcting errors was extremely costly, requiring
			careful retracing of many installation steps.
			
			TODO: TALK ABOUT MULTI-PERSON-WIRING IN PRACTICE ON FIVE CABINET MACHINE.
			
			\begin{figure}
				
				\center
				\buildfig{figures/wire-length-histogram.tex}
				
				\caption{Histogram of connection distances in a ten-cabinet,
				one-million core SpiNNaker machine annotated with the suggested cable
				length.}
				\label{fig:wire-length-histogram}
				
			\end{figure}
			
			FIGURE \ref{fig:wire-length-histogram} SHOWS THE DISTRIBUTION OF CABLE
			LENGTHS REQUIRED. IN PRACTICE THE SLACK ALLOCATED PROVED ADEQUATE. AS
			SHOWN IN FIGURE \ref{fig:install-histogram}, THE MOST IMPORTANT FACTOR IS
			WHETHER LEAVING THE FRAME OR NOT. LEAVING THE FRAME TAKES THE LONGEST.
			
			\begin{figure}
				\builddata{data/build_connection_log.tex}
				\buildfig{figures/install-histogram.tex}
				
				\caption{Histogram of cable installation times}
				\label{fig:install-histogram}
			\end{figure}
			
			TODO: COMPARE DIRECTLY WITH INSTALL TIMES REPORTED IN LITERATURE.
		
		\subsection{Thermal Impact}
			
			TODO: SHOW HOW TEMPERATURE IS CHANGED
			
		\subsection{Maintenance}
			
			TOOD: QUANTIFY CABLE REMOVALS REQUIRED. EXPERIMENT: REMOVE/REPLACE RANDOM
			BOARDS AND MEASURE TIME TAKEN, CABLES REMOVED. COMPARE WITH STANDARD DATA
			CENTRE WIRING

	\chapter{Finding shortest path vectors in SpiNNaker's network}
	
	Once a SpiNNaker machine has been constructed as described in the previous
	chapter, its network forms a large hexagonal torus topology. To exploit this
	network routing algorithms must be used to generate routes for packets to
	follow between nodes. As well as ensuring that packets arrive at the correct
	destination, routing algorithms often attempt to produce routes which make
	efficient use of the network. This often involves attempting to reduce
	congestion by ensuring packets do not travel further through the network than
	absolutely necessary.
	
	Many popular routing algorithms for torus topologies, including all published
	algorithms designed for SpiNNaker's hexagonal torus topology
	\cite{davies12,navaridas14}, internally function by computing shortest path
	vectors and generating routes from them. Existing methods of calculating
	shortest path vectors in hexagonal torus topologies are unable to generate
	all possible shortest path vectors and, as a result, reduces the diversity of
	routes produced by routing algorithms, potentially worsening network
	contention.
	
	In this chapter I describe a novel technique for computing shortest path
	vectors in hexagonal torus topologies which yields \emph{all} possible
	shortest path vectors for any pair of nodes. Further, implementations of this
	new technique execute an order of magnitude faster than the existing
	alternatives.
	
	\section{Related work}
		
		TODO: INTRODUCE SECTION
		
		\begin{figure}
			\center
			
			\begin{subfigure}{\linewidth}
				\center
				\buildfig{figures/distance-map-mesh.tex}
				\caption{2D mesh topology}
				\label{fig:distance-map-mesh}
			\end{subfigure}
			
			\vspace{1em}
			
			\begin{subfigure}{\linewidth}
				\center
				\buildfig{figures/distance-map-torus.tex}
				\caption{2D torus topology}
				\label{fig:distance-map-torus}
			\end{subfigure}
			
			\vspace{1em}
			
			\begin{subfigure}{\linewidth}
				\center
				\buildfig{figures/distance-map-hex-mesh.tex}
				\caption{Hexagonal mesh topology}
				\label{fig:distance-map-hex-mesh}
			\end{subfigure}
			
			\vspace{1em}
			
			\begin{subfigure}{\linewidth}
				\center
				\buildfig{figures/distance-map-hex-torus.tex}
				\caption{Hexagonal torus topology}
				\label{fig:distance-map-hex-torus}
			\end{subfigure}
			
			\caption{Plots showing distance from various locations marked
			         {\color{red}$\times$}. Darker areas are further away. Contour
			         lines show equidistant points.}
			\label{fig:distance-map}
		\end{figure}
		
		\subsection{Mesh Networks}
			
			In a (non-hexagonal) mesh network topology, shortest path vectors are
			computed by taking the element-wise difference between the source and
			destination nodes' coordinates.
			
			\begin{figure}
				\center
				\buildfig{figures/mesh-topology-coordinates.tex}
				\caption{An example 2D mesh network with example shortest-path routes
				from `A' to `B' and `B' to `C'.}
				\label{fig:mesh-topology-coordinates}
			\end{figure}
			
			For example, figure \ref{fig:mesh-topology-coordinates} illustrates a 2D
			mesh topology. In this topology, the nodes labelled `A', `B' and `C' have
			position vectors $(1, 2)$, $(4, 5)$ and $(6, 1)$ respectively. The
			shortest path vector from node `A' to `B' is thus simply $(4, 5) - (1, 2)
			= (3, 3)$ and from `B' to `C' is $(6, 1) - (4, 5) = (2, -4)$.
			
			A route may be produced from a shortest path vector by advancing the
			number of hops specified for each dimension in the vector. For example
			any permutation of the hops X$^+\,$X$^+\,$X$^+\,$Y$^+\,$Y$^+\,$Y$^+$, an
			example of which is included in the figure. Likewise a route from `B' to
			`C' may be constructed from any permutation of
			X$^+\,$X$^+\,$Y$^-\,$Y$^-\,$Y$^-\,$Y$^-$.
			
			Many popular routing algorithms such as Dimension Order Routing (DOR),
			Right-Turn Only Routing (RTOR) and Longest Dimension First Routing (LDFR)
			\cite{dally04,davies12} directly follow the above procedure and just
			prescribe a specific permutation of hop order. For example, DOR produces
			routes with X hops first, Y hops second and so on.
			
			The length of routes produced from a shortest path vector have a number
			of hops proportional to the magnitude of the vector, thus a shortest path
			vector yields a route with the minimum number of hops. For a two
			dimensional vector $(a, b)$ the magnitude is given as:
			%
			\begin{equation}
				\| (a, b) \| = \lvert a \rvert + \lvert b \rvert
			\end{equation}
		
		\subsection{Torus Networks}
			
			Computing shortest path vectors in (non-hexagonal) torus topologies is
			also straight forward. As an example, lets find the shortest path vector
			from node `A' to `B' in the 2D torus topology shown in figure
			\ref{fig:torus-shortest-path-example}. First, both nodes are translated
			such that the source node, `A', is at the centre of the network (figure
			\ref{fig:torus-shortest-path-translate}). Note that this translation may
			result in the destination node `wrapping around' the network. After
			translation, the shortest path vector is computed as in a mesh topology.
			As illustrated in \ref{fig:torus-shortest-path-routed}, the computed
			shortest path vector may be used to produce routes between the two nodes
			in their original positions.
			
			\begin{figure}
				\center
				\begin{subfigure}{0.3\linewidth}
					\center
					\buildfig{figures/torus-shortest-path-example.tex}
					\caption{Original}
					\label{fig:torus-shortest-path-example}
				\end{subfigure}
				\begin{subfigure}{0.3\linewidth}
					\center
					\buildfig{figures/torus-shortest-path-translate.tex}
					\caption{Translated}
					\label{fig:torus-shortest-path-translate}
				\end{subfigure}
				\begin{subfigure}{0.3\linewidth}
					\center
					\buildfig{figures/torus-shortest-path-routed.tex}
					\caption{Routed}
					\label{fig:torus-shortest-path-routed}
				\end{subfigure}
				
				\caption{Finding shortest paths in a 2D torus topology.}
				\label{fig:torus-shortest-path}
			\end{figure}
			
			This process works because vectors from the centre (though not other
			locations) of a torus topology are identical to those in mesh topologies
			(see figures \ref{fig:distance-map-mesh} and
			\ref{fig:distance-map-torus}).
		
		\subsection{Hexagonal Mesh Networks}
			
			In hexagonal mesh topologies it is conventional to define three `axes' X,
			Y and Z as shown in figure \ref{fig:hex-mesh-topology-coordinates}
			\cite{patel15}. In this example, the three labelled nodes `A', `B' and
			`C' may be given position vectors such as $(1, 1, 0)$, $(3, 2, 0)$ and
			$(0, 0, -7)$ respectively. As in other mesh networks, a vector between
			two nodes is found by subtracting the nodes' vectors. For example, a
			vector from `A' to `B' is $(3, 2, 0) - (1, 1, 0) = (2, 1, 0)$. This
			vector can then be converted into a route in the same way as a mesh
			network by taking any permutation of the three hops  X$^+\,$X$^+\,$Y$^+$.
			
			\begin{figure}
				\center
				\buildfig{figures/hex-mesh-topology-coordinates.tex}
				\caption{An example hexagonal mesh network topology.}
				\label{fig:hex-mesh-topology-coordinates}
			\end{figure}
			
			As explained in detail in appendix \ref{app:minimal-hex-coordinates},
			there are an infinite number of vectors between any two points. For
			example, the vectors $(1, 0, -1)$ and $(3, 2, 1)$ also reach node `B'
			from `A' in the example. However, for a given pair of nodes, there is
			always a single, unique vector whose magnitude is minimal which is
			given by the function:
			%
			\begin{equation}
				\operatorname{minimiseVector}(x,y,z)
					= (x,y,z) - \operatorname{median}(x,y,z) \cdot (1,1,1)
			\end{equation}
			%
			An important side-effect of this function is that a minimised vector will
			always contain at least one zero element meaning that shortest path
			routes will use at most two of the three available dimensions.
			
			To aid the reader's intuition, figure \ref{fig:distance-map-hex-mesh}
			illustrates how distances grow in a hexagonal mesh topology.
		
		\subsection{Hexagonal Torus Networks}
			
			Unfortunately, unlike non-hexagonal torus topologies, the translation
			technique cannot be used to compute shortest path vectors. As illustrated
			in figures \ref{fig:distance-map-hex-mesh} and
			\ref{fig:distance-map-hex-torus}, shortest path vectors from the center
			of a hexagonal mesh network are not equivalent to those of a hexagonal
			torus network.
			
			Prior research into routing in SpiNNaker's network has been based on the
			INSEE \cite{navaridas09,ghasempour15} interconnect simulator. Internally
			INSEE tries a set of twelve candidate vectors and picks the shortest as
			the shortest path vector to use for routing.
			
			\begin{figure}
				\center
				\begin{subfigure}{0.45\linewidth}
					\center
					\buildfig{figures/insee-vector-candidates-no-wrap.tex}
					\caption{$(\Delta_\textrm{X}, \Delta_\textrm{Y}) = (5,3)$}
					\label{fig:insee-vector-candidates-no-wrap}
				\end{subfigure}
				\begin{subfigure}{0.45\linewidth}
					\center
					\buildfig{figures/insee-vector-candidates-wrap-x.tex}
					\caption{$(\Delta'_\textrm{X}, \Delta_\textrm{Y}) = (-3,3)$}
					\label{fig:insee-vector-candidates-wrap-x}
				\end{subfigure}
				
				\vspace{1em}
				
				\begin{subfigure}{0.45\linewidth}
					\center
					\buildfig{figures/insee-vector-candidates-wrap-y.tex}
					\caption{$(\Delta_\textrm{X}, \Delta'_\textrm{Y}) = (5,-5)$}
					\label{fig:insee-vector-candidates-wrap-y}
				\end{subfigure}
				\begin{subfigure}{0.45\linewidth}
					\center
					\buildfig{figures/insee-vector-candidates-wrap.tex}
					\caption{$(\Delta'_\textrm{X}, \Delta'_\textrm{Y}) = (-3,-5)$}
					\label{fig:insee-vector-candidates-wrap}
				\end{subfigure}
				
				\vspace{1em}
				
				% Key
				\begin{tikzpicture}[thick]
					\coordinate (last);
					
					% #1 colour
					% #2 label
					\newcommand{\colourkeyentry}[2]{
						\node [#1] [right=of last, fill, rectangle, minimum size=1em] (last) {};
						\node [right=0 of last] (last) {#2};
					}
					
					\colourkeyentry{cb3class0}{$(\textrm{X}, \textrm{Y}, 0)$}
					\colourkeyentry{cb3class1}{$(\textrm{X} - \textrm{Y}, 0, - \textrm{Y})$}
					\colourkeyentry{cb3class2}{$(0, \textrm{Y} - \textrm{X}, - \textrm{X})$}
					
				\end{tikzpicture}
				
				\caption{The twelve candidate shortest-path vectors considered by INSEE
				represented as dimension-order routes. $W=H=8$,
				$(\Delta_\textrm{X},\Delta_\textrm{Y}) = (5, 3)$ and
				$(\Delta'_\textrm{X},\Delta'_\textrm{Y}) = (-3, -5)$.}
				\label{fig:insee-vector-candidates}
			\end{figure}
			
			The twelve vectors considered are constructed as follows.
			
			First a shortest path vector from the source to target node are
			constructed as if the network was a 2D mesh yielding a vector
			$(\Delta_\textrm{X},\Delta_\textrm{Y})$. From this, another vector
			$(\Delta'_\textrm{X},\Delta'_\textrm{Y})$, is defined:
			%
			\begin{align}
				\Delta'_\textrm{X} &= \Delta_\textrm{X} - \operatorname{sign}(\Delta_\textrm{X})W
				\\
				\Delta'_\textrm{Y} &= \Delta_\textrm{Y} - \operatorname{sign}(\Delta_\textrm{Y})H
			\end{align}
			%
			Where $W$ and $H$ are the width and height of the network respectively
			(in nodes). This new vector yields routes from the source to destination
			node that in a torus topology that \emph{always} wrap around the `X' and
			`Y' dimensions.
			
			From the pair of vectors defined, four possible 2D vectors can be
			produced: $(\Delta_\textrm{X},\Delta_\textrm{Y})$,
			$(\Delta'_\textrm{X},\Delta_\textrm{Y})$,
			$(\Delta_\textrm{X},\Delta'_\textrm{Y})$ and
			$(\Delta'_\textrm{X},\Delta'_\textrm{Y})$. Further, each 2D vector may be
			converted into one of three 3D vectors where either X, Y or Z are zero
			for a total of twelve candidate vectors.  Figure
			\ref{fig:insee-vector-candidates} illustrates all twelve candidate
			vectors for an example pair of nodes.
			
			\begin{figure}
				\center
				\buildfig{figures/xyz-protocol-regions.tex}
				
				\caption{The four regions defined by the XYZ-protocol.}
				\label{fig:xyz-protocol-regions}
			\end{figure}
			
			A more efficient technique is proposed by Hoffmann and D\'es\'erable
			called the XYZ-Protocol \cite{hoffmann15,hoffmann11}. If the source and
			destination nodes are translated such that the source node lies at the
			center of the topolgoy, the destination will lie in one of four regions
			illustrated in figure \ref{fig:xyz-protocol-regions}.
			
			If the destination lies in regions 1 or 4, a route may be constructed as
			if in a hexagonal mesh topology.
			
			Alternatively, if the destination lies in regions 2 or 3, the algorithm
			tests whether taking a mesh-like route within the region or
			wrapping-around either the X or Y dimension yields the shorter vector.
			The shortest of these vectors is then chosen.
			
			TODO DESCRIBE SPIRAL ROUTES.
			
			TODO DESCRIBE RTOR AND LDFR.
		
	\section{Dimension order routing in hexagonal torus topologies}
		
		So, existing solutions have two problems: trying 12 options and picking one
		is a bit kludgey and there are actually more options than that.
		
		\subsection{Simpler minimal hexagonal torus vectors}
			
			If you redraw your route such that it is sourced from bottom left corner
			(which we'll now call (0, 0)), there are four possible ways this route
			could wrap.
			
			TODO: DESCRIBE WAYS OF WRAPPING
			
			For each of these wrappings, all the possible routes we can take are
			strictly limited in terms of the dimensions used since we're stuck in a
			corner.
			
			In each case, the function computing the minimal hex vector function
			simplifies to a much simpler operation.
			
			TODO: DESCRIBE MINIMUM VECTOR LENGTH FUNCTIONS FOR EACH CASE
			
			This gives us a cheap way to compute which of the four possible wrappings
			are shortest. Based on this we can pick one of at most two (is this
			easily provable?) vectors in some fair manner to reduce load. This vector
			can then be minimised in the usual way.
			
			This also leads to confirming a theoretical result giving the length of a
			shortest path in a hexagonal torus topology.
			
			TODO: DESCRIBE HOW TO GET LENGTH FUNCTION AND COMPARE WITH \cite{xiao04}
		
		\subsection{Generating spiralling routes}
			
			In non-hexagonal torus topologies the previous technique would reveal all
			possible shortest vectors (e.g. in cases where you can wrap more than one
			way). Unfortunately, due to the addition of a non-orthogonal axes,
			hexagonal toruses also have other cases when the width and height do not
			match.
			
			TODO: TORUS SPIRALLING EXAMPLE
			
			It is possible to calculate the maximum number of spirals thus:
			
			TODO: DESCRIBE HOW MAX NUMBER OF SPIRALS IS COMPUTED
			
			Given a number of spirals, the vector can be updated this (note that the
			change does not add a multiple of (1, 1, 1) but also does not result in
			the vector changing length and thus becoming non-minimal!).
			
			TODO: DESCRIBE TRANSFORMATION
			
			TODO: PROVE THAT MINIMALITY IS MAINTAINED
		
		\subsection{Proof of completeness}
		
			TODO: PROOF OF COMPLETENESS BY EXHAUSTIVE SEARCH
	
		\subsection{Conclusions}
			
			This approach is simpler, smaller, has sounder theoretical basis, and
			finds more routes than alternatives. This is good for load balancing and
			fault avoidance and also good for completeness.


	\chapter{Routing packets in large SpiNNaker machines}
	
	\label{sec:routing}
	
	So far, this thesis has focused on tackling the practical challenges
	resulting from SpiNNaker's hexagonal torus network topology. In this chapter,
	I adjust my focus towards the practical challenges resulting from SpiNNaker's
	large scale. Faults in large systems are inevitable and in the half-million
	core, \num{28800} chip SpiNNaker machine recently completed at the University
	of Manchester, around \SI{1}{\percent} of chips exhibited faults\footnote{Of
	the faulty chips discovered, the vast majority of faults are attributed to a
	currently unknown SDRAM failure}. These faults result in gaps and broken
	links in the network topology which routing algorithms must avoid in order to
	ensure correct system operation.
	
	In this chapter I tackle the problem of extending existing routing algorithms
	for SpiNNaker's network to enable them to route-around known, static faults.
	Though dynamic or transient faults may also occur, in this work such faults
	are ignored and other techniques, such as protocol-level fault tolerance, are
	relied on instead.
	
	Numerous heuristic-based fault-tolerant routing algorithms exist which target
	different network topologies and router architectures. Unfortunately as I
	will show, these algorithms are not portable and rely on or attempt to work
	around specific features of their target network architecture. In particular,
	existing work is dominated by the challenge of developing routing schemes
	which avoid or defuse network deadlocks. Due to SpiNNaker's unconventional
	use of timeout-based flow-control, it is not subject to the routing
	restrictions present in other architectures intended to cope with deadlocks.
	
	In this chapter I introduce a graph-search based post-processing step for
	non-fault-tolerant routing algorithms which guarantees routability in
	SpiNNaker systems without disconnected subregions. I also demonstrate that
	this technique introduces both negligible computational overhead to the
	routing algorithm runtime and resulting network performance.
	
	TODO: NOTE THE FAULT RATES ENCOUNTERED IN PRACTICE...
	
	\section{Related work}
		
		Existing work on routing in SpiNNaker's network has ignored the challenge
		of avoiding faults and instead focused on producing efficient multicast
		routes. As a result this section is broken into two halves. In the first
		half I survey the existing non-fault-tolerant approaches to routing used in
		SpiNNaker to-date. In the second I discuss the approaches to fault tolerant
		routing taken in other systems.
		
		\subsection{Multicast routing in SpiNNaker}
			
			Various fault-intolerant multicast routing algorithms exist for many
			networks and a number have been proposed and evaluated specifically in the
			context of SpiNNaker.
			
			In 2012, Davies \emph{et al.} evaluated the use of three common torus
			routing algorithms in SpiNNaker and found that simple oblivious routing is
			suitable for typical neural applications \cite{davies12}. The three
			routing techniques are:
			
			\begin{description}
				
				\item[Dimension Order Routing (DOR)] Packets are routed along each
				dimension (e.g. $X$, $Y$ and $Z$) in turn until no further hops are
				available in that direction.  The order in which the dimensions are
				traversed is fixed.
				
				\item[Right Turn Only Routing (RTOR)] As in DOR except the dimension
				order is chosen such that routes only contain right-turns.
				
				\item[Longest Dimension First Routing (LDFR)] As in DOR except the
				dimension order is chosen in descending order of number of hops in each
				dimension.
				
			\end{description}
			
			These unicast routing techniques are converted into a multicast routing
			algorithm by merging together the routes produced between the source node
			and each destination node as illustrated in figure
			\ref{fig:simple-routers}.
			
			\begin{figure}
				\center
				\begin{subfigure}{0.3\linewidth}
					\center
					\buildfig{figures/simple-routers-dor.tex}
					
					\caption{DOR}
					\label{fig:simple-routers-dor}
				\end{subfigure}
				\begin{subfigure}{0.3\linewidth}
					\center
					\buildfig{figures/simple-routers-rtor.tex}
					
					\caption{RTOR}
					\label{fig:simple-routers-dor}
				\end{subfigure}
				\begin{subfigure}{0.3\linewidth}
					\center
					\buildfig{figures/simple-routers-ldfr.tex}
					
					\caption{LDFR}
					\label{fig:simple-routers-dor}
				\end{subfigure}
				
				\caption{Example multicast routes produced by merging together unicast
				routes from a central source node to each destination node.}
				\label{fig:simple-routers}
			\end{figure}
			
			In 2014, Navaridas \emph{et al.} introduced two new algorithms, `Enhanced
			Shortest Path Routing' (ESPR) and `Neighbourhood Exploring Routing' (NER)
			which produce multicast routing trees with fewer total hops
			\cite{navaridas14}. In both algorithms, routes are generated sequentially
			for each of the destinations of a route using LDFR. Unlike LDFR, however,
			these algorithms search a limited area of the network for other,
			already-connected destination nodes to which LDFR routes may be
			constructed. If no suitable destination is found, a LDFR route is
			constructed to the source node. Figure \ref{fig:search-regions} illustrates
			the shape of the searched regions of each algorithm. ESPR searches the
			trapezoidal region between the source and destination nodes while NER
			searches a fixed radius out from the destination node.
			
			\begin{figure}
				\center
				\begin{subfigure}{0.45\linewidth}
					\center
					\buildfig{figures/search-regions-espr.tex}
					
					\caption{ESPR}
					\label{fig:search-regions-espr}
				\end{subfigure}
				\begin{subfigure}{0.45\linewidth}
					\center
					\buildfig{figures/search-regions-ner.tex}
					
					\caption{NER}
					\label{fig:search-regions-espr}
				\end{subfigure}
				
				\caption{The ESPR and NER algorithms attempt to connect the node marked
				`D' to the closest node in the shaded region which is connected to the
				source node, `S'. If no connected node is found in the shaded region, the
				LDFR route is taken to `S'. The dotted line indicates the route chosen
				from `D'.}
				\label{fig:search-regions}
			\end{figure}
			
			Unfortunately none of these routing algorithms make any allowance for the
			avoidance of network faults. As a result their utility in real-world
			systems is limited.
		
		\subsection{Fault-tolerant routing}
			
			Numerous fault-tolerant routing algorithms have been proposed for
			super-computer networks. These algorithms are largely constrained by the
			need to maintain deadlock freedom. Since SpiNNaker's routers employ a
			timeout based deadlock-breaking strategy, much of this effort is
			unnecessary in SpiNNaker. As described below, this frequently renders
			existing fault-tolerant routing algorithms unnecessarily complex and
			inflexible.
			
			Deadlocks occur in a network if a cyclic dependency exists on any limited
			resource in the network. For example, as illustrated in figure
			\ref{fig:ring-deadlock}, in a ring network a deadlock may form when every
			node is waiting on the next node to accept a packet before accepting new
			packets from the previous node.
			
			\begin{figure}
				\center
				\buildfig{figures/ring-deadlock.tex}
				
				\caption{A deadlock in a ring network where each node is waiting for
				the next to accept a packet before accepting any further packets.}
				\label{fig:ring-deadlock}
			\end{figure}
			
			To prevent deadlocks, combinations of router microarchitectural features
			and routing restrictions may be employed. For example, a simple
			deadlock-free routing algorithm for mesh and torus networks mandates the
			use of DOR \cite{dally93}. Packets travelling in a -ve direction along
			each axis take priority over those travelling in a +ve direction. Packets
			travelling along the Y axis take priority over those travelling along the
			X dimension. Given these rules it is possible to define a total ordering
			on all hops in the network. Figure \ref{fig:deadlock-free-dor}
			illustrates a $3\times3$ mesh network whose hops have been numbered
			according to the total ordering defined above.  Any `X-then-Y' DOR route
			through this network results in the use of hops labelled with strictly
			increasing numbers. As a result, no cyclic dependencies (and thus no
			deadlocks) may occur.
			
			\begin{figure}
				\center
				\buildfig{figures/deadlock-free-dor.tex}
			
				\caption{Deadlock-free routing of two example routes using DOR in a 2D
				mesh topology. The numbers of the hops taken by each route are given on
				the right.}
				\label{fig:deadlock-free-dor}
			\end{figure}
			
			Unfortunately, the routing restrictions imposed to ensure deadlock
			freedom can result in fault-intolerant routing. In the example above, if
			the node at the bottom-right corner of the figure was faulty, the dotted
			example route would be blocked as no alternative routes are allowed.
			
			In practice, the routing rules used may be more relaxed, for example
			requiring that any route whose length is equal to a DOR must exist to
			guarantee routability \cite{rodrigo09}.
			
			Alternative routing strategies take a hybrid approach whereby an
			efficient but fault-intollerant routing algorithm is used where possible
			and in the presence of faults a less efficient but more robust strategy
			is employed. For example, the Immucube network architecture employs three
			virtual networks which operate independently over the same physical links
			\cite{puente07}. Initially messages are routed using a high-performance
			but potentially-deadlockable routing scheme in the first virtual network.
			If a deadlock is occurs, the deadlocked packet is dropped into the second
			virtual network in which packets are routed using a less efficient but
			deadlock-free but fault-intolerant routing algorithm. Finally, upon
			encountering a fault, packets are dropped onto the third virtual network
			which forms a ring network routing packets to every node in the network.
			
			Releated approaches \cite{mejia06,boppana95} divide the network into
			regions in which different routing rules are enforced to ensure deadlock
			freedom and, when required, fault tolerance.
			
			TODO FIGURE?
			
			The BlueGene/L supercomputer \cite{adiga02} uses DOR for its torus
			network and implements fault-tolerance by sacrificing otherwise
			functioning `lamb' nodes to ensure no route passes through a known dead
			link \cite{ho04}. In figure \ref{fig:lamb-nodes} an example scenario is
			shown where a single dead node is present and all nodes in the same row
			or column as the dead node have been made into lamb nodes. The lamb nodes
			may not be used in an application except as a through-route for other
			traffic. This pattern of lamb nodes guarantees that all dimension-order
			routes between all pairs of non-lamb nodes are not obstructed by the
			faulty node. This approach trades use of higher performance routing
			logic for wasted resources. This type of approach is most appropriate
			when algorithmic routing is used and routing rules are inflexible.
			
			\begin{figure}
				\center
				\buildfig{figures/lamb-nodes.tex}
				
				\caption{`Lamb' nodes may be disabled to ensure DOR will never
				encounter a fault.}
				\label{fig:lamb-nodes}
			\end{figure}
			
			Other algorithms proposed for the BlueGene architecture attempt to avoid
			the need for lamb nodes by generating routes which reach their destination
			via a `proxy' node \cite{gomez04}. By appropriately selecting the location
			of such a proxy, the existing routing algorithm used by the system can be
			guaranteed to select a route free of faults.
			
			TODO: EXAMPLE OF PROXY ROUTING TO AVOID FAULT
			
			Finally, many algorithms in in the field are distributed and use only local
			information along with limited information from their peers to generate
			routes \cite{fick09b}. In SpiNNaker, route generation is conventionally
			carried out centrally since no special on-chip hardware facilities exist
			for route generation. Centralised route generation also enables the routing
			algorithm to consider all available routes. As a result, there is little
			incentive for the use of distributed routing algorithms on SpiNNaker since
			global system information could be compactly shared for one-off routing
			passes.
			
			Algorithms for other architectures such as IP networks tend to be poor fits
			for static, regular network topologies since they use expensive graph-based
			algorithms for route discovery which aren't necessary here. They also tend
			to heavily feature graph topology discovery etc. which aren't needed here.
			
			Work on fault-tolerance in data centre networks does exploit the regularity
			of the network topology in routing algorithms \cite{guo08,liao12}.
			Unfortunately, the approaches used are not general enough to be applied to
			mesh-like topologies such as the one in SpiNNaker.
			
			Outside the field of computer networks, routing algorithms used to route
			wires across the surfaces of chips are required to solve similar problems
			to fault-tolerant network routing problems in mesh networks. Like mesh
			networks, the routes are defined within a regular Manhattan geometry and
			congested areas, rather than faults must be avoided by the algorithms
			\cite{kahng11}.  Unfortunately, these algorithms are designed for
			occasional batch operation prior to the multi-month process of chip
			manufacturing and so runtimes of hours or days are commonplace
			\cite{nam08}. As such these algorithms would be inappropriate for use
			with applications such as SpiNNaker where users' applications tend to be
			short-lived and thus routing should not be allowed to dominate runtime.
	
	\section{Partial graph search repair}
		
		In this section I introduce a novel post-processing algorithm, Partial
		Graph Search (PGS) repair, for routes produced by non-fault-tolerant
		routing algorithms.
		
		PGS repair guarantees routability for networks with no disconnected
		subregions by using a graph search algorithm to route around faults in the
		original route.  General-purpose graph search algorithms such as Breadth
		First Search (BFS), Dijkstra's Algorithm and A* are guaranteed to find
		shortest-path routes between pairs of points in arbitrary graphs. Such
		algorithms are generally a poor choice in highly regular network topologies
		such as meshes and toruses due to their high computational cost. In PGS
		repair, graph searching is only used for \emph{part} of the routing
		problem: to repair gaps in routes generated by more efficient routing
		algorithms.
		
		Real world super computer architectures are designed to ensure that faults
		are isolated \cite{gara05,alverson12} and thus tend to only impact a
		localised region of the network. Since PGS repair is only needed to route
		around these isolated faults, the space searched by the graph search
		algorithm should be very small in practice resulting in only short
		runtimes. In addition since faults are rare in real-world systems, the
		graph search process will only rarely be invoked.
		
		The PGS repair post-processing technique starts with a route produced by a
		non-fault-tolerant routing algorithm such as ESPR or NER. If this route is
		not obstructed by a fault, the algorithm terminates immediately without
		modifying the route. If the route attempts to use a faulty link, the
		algorithm proceeds as follows.
		
		The routing tree produced by the underlying routing algorithm is broken
		into subtrees wherever it attempts to route through a broken link and
		each subtree is assigned a unique colour, as illustrated in figure
		\ref{fig:pgs-repair-colouring}. From each disconnected subtree's root
		node in turn, a graph search is performed to find a short, fault-free
		route to a subtree node of a different colour. The subtree is then
		attached to the tree discovered by the graph search and re-coloured to
		match the tree it is connected to.
		
		\begin{figure}
			\center
			\begin{subfigure}{0.32\linewidth}
				\hspace*{-1.5em}
				\buildfig{figures/pgs-repair-colouring.tex}
				
				\caption{}
				\label{fig:pgs-repair-colouring}
			\end{subfigure}
			\begin{subfigure}{0.32\linewidth}
				\hspace*{-1.5em}
				\buildfig{figures/pgs-repair-colouring-fix1.tex}
				
				\caption{}
				\label{fig:pgs-repair-colouring-fix1}
			\end{subfigure}
			\begin{subfigure}{0.32\linewidth}
				\hspace*{-1.5em}
				\buildfig{figures/pgs-repair-colouring-fix2.tex}
				
				\caption{}
				\label{fig:pgs-repair-colouring-fix2}
			\end{subfigure}
			
			\caption{PGS repair process example showing a disconnected multicast
			route from A to B, C, D, E and F. $\times$ indicates a broken link.}
			\label{fig:pgs-repair-colouring-steps}
		\end{figure}
		
		For example in figure \ref{fig:pgs-repair-colouring-fix1} a path from the
		root of the subtree containing nodes E and F is found which connects it to
		the subtree rooted at A. Similarly in figure
		\ref{fig:pgs-repair-colouring-fix2} a path is also found connecting the
		subtree containing nodes C and D back to the subtree rooted at node A.
		
		If the routing tree was broken into $N+1$ subtrees by faults there will be
		$N$ subtrees disconnected from the root node. Each of the $N$ iterations of
		the algorithm connect a disconnected subtree to another subtree reducing
		the number of subtrees by $1$ each time. After $N$ iterations, therefore,
		exactly $1$ subtree remains which connects every node in the original
		routing tree without traversing faulty links.
		
		TODO: EXPLAIN THE FIDDLINESS HERE TO ENSURE WE DON'T CREATE LOOPS.
		
	\section{Evaluation \& Results}
		
		The PGS repair technique, by design, is able to work around all possible
		fault patterns which don't completely disconnect parts of the network. This
		result this evaluation focuses on the impact on performance PGS repair
		imposes. The metrics of interest in this evaluation are:
		
		\begin{itemize}
			\item Algorithm runtime
			\item Network congestion
			\item Routing table utilisation
		\end{itemize}
		
		\subsection{Traffic Patterns}
			
			In this evaluation, two standard benchmark multicast traffic patterns are
			used which have been used in previous research into SpiNNaker's network:
			
			\begin{figure}
				\center
				\buildfig{figures/traffic-distribution-centroids.tex}
				
				\caption{An example 4-centroid distribution with four centroids. The
				$\times$ marks the location of the origin node. Lighter colours
				indicate greater likelihood of a connection.}
				\label{fig:traffic-distribution-centroids}
			\end{figure}
			
			\begin{description}
				
				\item[Uniform] Destinations are chosen with uniform probability
				anywhere in the machine.
				
				\item[$N$-Centroids] Destinations are clustered around one of $N$
				randomly chosen `centroids' as illustrated in figure
				\ref{fig:traffic-distribution-centroids}.
				
			\end{description}
			
			The uniform traffic pattern is widely used in networks research
			\cite{dally04,davies12} while the centroids model was developed
			specifically to reproduce the traffic patterns found in the neural
			applications SpiNNaker is designed for \cite{navaridas14}. In this work
			we consider 3 centroids.
		
		\subsection{Fault model}
			
			In addition two different fault models are used which are representative of
			the faults found in real SpiNNaker systems:
			
			\begin{figure}
				\center
				\begin{subfigure}{0.48\linewidth}
					\hspace*{-1.5cm}
					\buildfig{figures/fault-example-uniform.tex}
					
					\caption{Uniform}
					\label{fig:fault-example-uniform}
				\end{subfigure}
				\begin{subfigure}{0.48\linewidth}
					\hspace*{-1.5cm}
					\buildfig{figures/fault-example-hss.tex}
					
					\caption{HSS Link}
					\label{fig:fault-example-hss}
				\end{subfigure}
				
				\caption{The two link fault models considered.}
				\label{fig:fault-example}
			\end{figure}
			
			\begin{description}
				
				\item[Uniform] Links are selected and disabled at random (figure
				\ref{fig:fault-example-uniform}).
				
				\item[HSS Link] Groups of links corresponding with randomly selected
				single High-Speed Serial (HSS) link between SpiNNaker boards are disabled
				together (figure \ref{fig:fault-example-uniform}).
				
			\end{description}
			
			The uniform link failure model models isolated failures resulting from
			isolated manufacturing defects in individual links. The HSS Link failure
			model models faults arising from failing or disconnected board-to-board
			links which carry several chip-to-chip traffic flows via a single cable in
			SpiNNaker systems. Though SpiNNaker-specific, the later fault model is
			analogous to failure modes arising in other architectures where a single
			fault may render several links impassable in a single area.
			
			A range of failure rates are explored in this section. My measurements of
			current large-scale SpiNNaker installations the link failure rate is about
			\SI{0.03}{\percent} with failures due to both individual chip-to-chip links
			and board-to-board HSS links. Exact link failure statistics for commercial
			super computer installations are not widely available, however, published
			Mean-Time-Between-Failure (MTBF) statistics place an upper bound on link
			failure rates at a similar \SI{0.03}{\percent} in one-year-old BlueGene/Q
			systems \cite{chiu11}.
			
			Unfortunately presently undiagnosed problem with the SDRAM packaged with
			approximately \SI{1}{\percent} of SpiNNaker chips has rendered these chips
			unusable for most applications. The gaps in the network resulting from the
			loss of these chips currently dominate true link failures leaving just over
			\SI{1}{\percent} of links inoperable.
			
			Surprisingly, research into fault tolerant routing in super computers
			appears to focus on benchmarks with even higher fault rates ranging from
			\SI{3}{\percent} to as high as \SI{7}{\percent}
			\cite{ho04,gomez04,mejia06}.
			
			In this evaluation, fault rates ranging from \SI{0.01}{\percent} to
			\SI{5}{\percent} are considered to cover both realistic fault levels
			along with the more extreme cases considered in related work.
		
		\subsection{Base routing algorithm}
			
			Since the PGS repair process is routing algorithm agnostic all
			experiments use the NER algorithm which has been found to be appropriate
			for SpiNNaker applications \cite{navaridas14}.
		
		\subsection{Algorithm runtime}
			
			To assess the impact of the PGS repair process on routing algorithm
			runtime, the algorithm was used to process a large number of randomly
			generated routing problems and the runtime recorded.
			
			\num{10000} one-to-sixteen multicast routing problems were generated in a
			$256\times256$ hexagonal torus topology, the largest size possible for a
			SpiNNaker system. Other quantities of multicast destinations were also
			evaluated but are omitted for brevity since the pattern of results are
			similar to those outlined here.
			
			TODO: APPENDIX WITH OTHER RUNS?
			
			The NER and PGS repair algorithms were written in C and compiled with GCC
			4.8.3 with \verb|-O2| level optimisations and executed on a cluster of
			idle workstations with 3.10 GHz Intel Core-i5-2400 CPUs.
			
			\begin{figure}
				\center
				\buildrplot{figures/routing-runtimes.R}
				
				\caption{Mean runtime of routing and PGS repair overhead. PGS repair
				overhead is stacked above the routing runtime (i.e. bars do not
				overlap). Error bars indicate 95\% confidence interval. Note different
				Y-scale for HSS link and uniform fault models.}
				\label{fig:routing-runtimes}
			\end{figure}
			
			Figure \ref{fig:routing-runtimes} shows the average runtimes recorded for
			both the NER and PGS repair algorithms. In fault-free networks the
			PGS-repair post-processing step is not required and incurs no penalty
			while the runtime of the algorithm grows with the fault rate for both
			fault and traffic models.
			
			Notably the HSS fault model results in longer runtimes for the PGS repair
			process compared with an equivalent fault-density of uniform faults.
			Because the HSS fault model produces contiguous lines of faults the PGS
			repair algorithm must construct a longer path to avoid the fault.  Since
			the space explored by a graph algorithm typically grows with $O(H^2)$
			with respect to the hops in the discovered route, $H$, this increase in
			search distance has a large impact on the runtime of the PGS repair
			process.
			
			The runtime of the PGS repair algorithm remains roughly in proportion to
			the runtime of the underlying routing algorithm with respect to different
			traffic models. The centroid traffic pattern tends to result in routes
			with fewer hops than a uniform traffic pattern with the same number of
			destination nodes as segments of routes are often shared between
			destination nodes. Since the NER algorithm's runtime is strongly related
			to the number of hops in the output route the runtime of the algorithm is
			greater for uniform traffic. Likewise the probability of PGS repair being
			required increases with the number of hops in route and hence the runtime
			of the PGS repair algorithm increases roughly in proportion.
		
		\subsection{Routing table usage}
			
			In order to gain a realistic measure of routing table usage it is
			necessary to determine the effect of many routes being generated for a
			single set of faults. To enable a sufficiently large number of sample to
			be collected the experimental setup considered previously is reduced to a
			network containing $48\times48$ nodes.
			
			\num{1000} $48\times48$ node network models are produced according to the
			HSS link and uniform fault models. For each of these models
			$48\times48\times16=$~\num{36864} one-to-sixteen routes are generated using
			the centroid and uniform traffic models. This corresponds to one
			multicast route per application core. As is convention in SpiNNaker,
			routing table entries are inserted for each route at the source of the
			route, at each destination and at each corner or fork. The number of
			routing table entries at each node in the model is counted and the
			maximum number of entries in a single node is reported for each network
			model.  The \emph{maximum} number of routing entries of any router was
			chosen since the number of entries available per SpiNNaker router is
			bounded by hardware.
			
			\begin{figure}
				\center
				\buildrplot{figures/routing-entries.R}
				
				\caption{Violin plot showing the distribution of maximum table sizes
				for \num{1000} random networks. The red line at \num{1024} entries
				indicates the size of SpiNNaker's routing tables.}
				\label{fig:routing-entries}
			\end{figure}
			
			
			Figure \ref{fig:routing-entries} shows the distributions of the largest
			routing table sizes for each fault and traffic model.
			
			\begin{figure}
				\center
				\begin{subfigure}{0.48\linewidth}
					\center
					\buildfig{figures/hss-link-routing-table-usage.tex}
					
					\caption{Routing table entries}
					\label{fig:hss-link-routing-table-usage}
				\end{subfigure}
				\begin{subfigure}{0.48\linewidth}
					\center
					\buildfig{figures/hss-link-resource-usage.tex}
					
					\caption{Routes passing through chip}
					\label{fig:hss-link-resource-usage}
				\end{subfigure}
				
				\caption{The impact of a HSS link fault on routing table usage and
				congestion. Each hexagon represents a single chip, the red line
				indicates the chip-to-chip connections broken by the HSS link fault.}
				\label{fig:hss-link-usage}
			\end{figure}
			
			The HSS link failure model has a much greater impact on peak routing
			table resource usage than uniform link failures for a given fault rate.
			This is because HSS link faults result in a large concentration of routes
			being disrupted and then re-routed around the same obstacle in a single
			location. Figure \ref{fig:hss-link-routing-table-usage} shows how routing
			table usage varies around a HSS link fault in one instance of the
			experiment. There are clear peaks in routing table usage around the ends
			of the line of faults which result from routes produced by PGS repair
			finding shortest paths around the edge of the faults.
		
		\subsection{Network congestion}
			
			To measure the impact of PGS repair on network congestion, two
			experiments were performed, one using the same model used to measure
			routing table usage and one based on tests run on SpiNNaker hardware.
			
			For each of the network fault and traffic pattern described previously,
			the paths taken for the \num{36864} one-to-sixteen multicast routes
			generated are used to compute the number of times each link in the
			network is used. The number of routes passing through the most-used link
			is then recorded, giving an indication of the level of congestion in the
			network.
			
			\begin{figure}
				\center
				\buildrplot{figures/routing-resource.R}
				
				\caption{Violin plot showing the distribution of maximum
				routes-per-chip for \num{1000} random networks.}
				\label{fig:routing-resource}
			\end{figure}
			
			The results are presented in figure \ref{fig:routing-resource} and follow
			the same trends as the results previously shown for routing table usage.
			Again, HSS link faults result in routes with the greatest congestion due
			to the concentration of routes finding shortest paths around an obstacle
			(see \ref{fig:hss-link-resource-usage}).
			
			To verify that the results above, an additional experiment has been
			carried out which attempts to mimic the model used previously in actual
			SpiNNaker hardware. In these experiments a large SpiNNaker machine is
			divided into independent 48-board (2304-chip) sections. Because the
			48-board systems used in these experiments are cut out of a larger
			machine, they lack wrap-around links and thus form hexagonal mesh
			topologies, rather than hexagonal toruses.
			
			Due to the SDRAM issue described above, fault rates below
			\SI{1}{\percent} cannot be modelled.  To simulate higher fault rates,
			additional links are disabled in software according to the fault models
			described used previously. Since some faults are due to genuine hardware
			faults, these faults cannot be placed randomly in each experiment. To
			reduce, bias each combination of fault rate, fault model and traffic
			pattern is repeated XXX times across randomly chosen physical machines.
			
			XXX 1-to-XXX routes are generated in both uniform and XXX-centroid
			distributions as used throughout this evaluation. Synthetic network
			traffic is generated at the source of each route following a Bernoulli
			distribution. Traffic consumers running on all destination nodes accept
			packets as quickly as possible from the network and log their arrival.
			The Bernoulli probability is set the same for every route's traffic
			generator and increased in steps of XXX and the number of packets dropped
			in an XXX second period logged. The network is considered saturated once
			less than \SI{99}{\percent} of packets successfully arrive at their
			destination.
			
			Figure \ref{XXX} shows the distributions of the saturation points for
			each experimental configuration.
			
			TODO: ANALYSIS
		
	\section{Conclusions}
		
		In this chapter I described how SpiNNaker's unconventional network and
		router architecture render existing fault tolerant routing algorithms
		unsuitable. I introduced PGS repair, a post-processing technique for
		existing non-fault tolerant routing algorithms designed for SpiNNaker such
		as NER.
		
		Unlike some other fault tolerant routing algorithms for other
		architectures, PGS repair is able to work-around arbitrary fault patterns
		by exploiting SpiNNaker's inbuilt deadlock avoidance mechanisms. In the
		presence of realistic failure rates of up to \SI{1}{\percent}, only small
		overheads of up to XXX, XXX and XXX for in algorithm runtime, routing table
		usage and network performance are incurred respectively. This low
		performance overhead makes PGS repair appropriate for use in real
		applications. At the time of writing the algorithm has been successfully
		used in a number of neural and non-neural SpiNNaker applications.
		
		At more extreme fault rates not expected in real-world systems, the
		algorithm still functions correctly but the results incur much greater
		routing table and congestion overheads, particularly when faults are
		concentrated. Future extensions to this algorithm might aim to reduce this
		overhead by producing longer and more varied routes around faults to even
		out the load.

	\chapter{Placing applications in large SpiNNaker machines}
	
	In the previous chapter I tackled the problem of scale in generating routes
	for very large networks such as SpiNNaker. In this work the centroid traffic
	pattern was used as an approximation of the expected network traffic
	generated by `well behaved' neural network simulation software running on
	SpiNNaker. The traffic produced largely exhibits strong locality, that is
	most communication occurs between either nearby nodes or clusters of nodes.
	In reality, neural simulation applications are not specified geometrically
	but rather as abstract graphs of communicating neurons
	\cite{davison08,eliasmith13}. Applications must then \emph{place} these
	neurons onto nodes in a SpiNNaker system, attempting maximise communication
	locality.
	
	In this chapter I re-evaluate the suitability of simulated annealing as a
	technique for finding high quality placements for large parallel
	applications. Though this technique had fallen out of fashion in the field of
	application placement by the early 1990s, it has found wide use for placing
	components in computer chip and FPGA designs. In the intervening years,
	placement problems in super computers have grown in size from tens or
	hundreds of nodes to millions, a scale at which chip placement techniques
	were operating in the mid 1990s. I adapt the simulated annealing algorithm
	used by the VPR academic circuit placement software to produce placements for
	applications running on SpiNNaker. In that in a range of real and synthetic
	benchmarks simulated annealing produces high quality placements enabling
	efficient use of SpiNNaker's network resources.
	
	
	%In the field of chip design, Moore's `Law' \cite{moore65,moore75} observes a
	%similar exponential growth in the number of components within a single chip.
	%Today modern processors contain billions of components and an analagous
	%placement problem exists in attempting to place interconnected components
	%near to eachother. In this chapter I explore the techniques used for circuit
	%placement and adapt one such technique, Simulated Annealing (SA)
	%\cite{kirkpatrick83}, for use in application placement. Despite some early
	%interest in SA for application placement in the 1980s and early 1990s, the
	%technique has since fallen out of favour. I find that at the scales of modern
	%placement problems SA-based placement is able to produce solutions of
	%superiour quality to contemporary methods.
	%
	%TODO: SUMMARISE RESULTS...
	
	\section{Related work}
		
		The placement problem has been tackled independently in the literature by
		researchers in both the application and chip placement communities. In this
		survey I cover application and chip placement separately as these two
		communities have remained largely isolated from one another. First I
		explore the techniques applied to application placement before moving on to
		contrast this with the techniques used in circuit placement.
		
		In the application placement literature, the placement problem is often
		referred under the umbrella term `mapping'. Unfortunately term is often
		used more broadly to include other tasks such as routing and application
		partitioning. To avoid ambiguity I use the term `placement', as preferred
		by the chip and FPGA design communities, to refer specifically to the
		problem of assigning nodes in an application's communication graph to nodes
		in a machine's connectivity graph.
		
		\subsection{Application placement algorithms}
			
			TODO: GENERAL INTRO
			
			\subsubsection{Application-specific approaches (manual placement)}
				
				In the case of some applications such as finite element modelling
				\cite{bermejo13}, the structure of the problem itself leads to a
				natural placement of the computation on nodes in a machine. For example
				when simulating a 3D volume in an node super computer with a $3 \times
				4 \times 2$ 3D torus or mesh topology network, the modelled volume
				might be divided into as in figure \ref{fig:fem-partitioning}. Each
				cuboid in the model is then assigned to the corresponding node in the
				network topology.
				
				\begin{figure}
					\center
					\buildfig{figures/fem-partitioning.tex}
					
					\caption{Example partitioning of a 3D space to fit into a super
					computer with a $3\times4\times2$ torus or mesh topology.}
					\label{fig:fem-partitioning}
				\end{figure}
				
				When the number of dimensions in a problem do not match that of the
				underlying network architecture, the common solution is to either
				divide only along a subset of the axes or to divide into additional
				pieces on the existing axes \cite{gilge14}.
			
			\subsubsection{Sequential placement}
				
				In the case where a placement solution is non-obvious one of the
				simplest and most popular strategies is to apply a simple sequential
				placement algorithm. Sequential placement algorithms function by
				iterating over the vertices in the application's communication graph
				and assigning them to a free node in the target machine. Sequential
				placement algorithms are differentiated by the order in which they
				iterate over vertices in the communication graph and fill nodes in the
				target machine. A number of widely used orderings are described below.
				
				\begin{figure}
					\center
					\begin{subfigure}{0.32\linewidth}
						\center
						\buildfig{figures/sequential-row-order.tex}
						\caption{Row-order}
						\label{fig:sequential-row-order}
					\end{subfigure}
					\begin{subfigure}{0.32\linewidth}
						\center
						\buildfig{figures/sequential-alternating.tex}
						\caption{Alternating}
						\label{fig:sequential-alternating}
					\end{subfigure}
					\begin{subfigure}{0.32\linewidth}
						\center
						\buildfig{figures/sequential-hilbert.tex}
						\caption{Hilbert curve}
						\label{fig:sequential-hilbert}
					\end{subfigure}
					
					\caption{Space-filling curves in 2D mesh and torus topologies.}
					\label{fig:sequential}
				\end{figure}
				
				Super computer management software such as SLURM \cite{yoo03} and Blue
				Gene's system software \cite{gilge14} by default na\"ively iterate over
				vertices in an application communication graph in the order they are
				provided. The nodes in the target machine are then iterated over in a
				simple space-filling curve through the network topology. Figure
				\ref{fig:hilbert-placement} illustrates the default patterns available
				in these software packages. The row-order (figure
				\ref{fig:sequential-row-order}) and alternating (figure
				\ref{fig:sequential-alternating}) curves correspond with 2D versions of
				the default node assignment orders used in SLURM and BlueGene systems.
				
				\begin{figure}
					\center
					\buildfig{figures/hilbert-placement.tex}
					
					\caption{A Hilbert curve, coloured from blue to red.}
					\label{fig:hilbert-placement}
				\end{figure}
				
				The Cray extensions to SLURM software provide a Hilbert curve
				\cite{hilbert91} (figure \ref{fig:sequential-hilbert}) node assignment
				order. Unlike the row-order and alternating space filling curves the
				Hilbert curve ensures that pairs of vertices close together in the node
				iteration order are also close together in the target machine's network
				\cite{moon01, zumbusch99}. Figure \ref{fig:hilbert-placement} shows a
				5$^\textrm{th}$-order Hilbert curve where each point in the curve is
				coloured according to its position along the curve. In this figure it
				is possible to see that nearby positions in the curve (which share
				similar colours) are also close in 2D space.
				
				When the proximity of vertices in the vertex-ordering supplied by an
				application is a good estimator of those vertices communication
				requirements, the sequential assignment schemes discussed above can be
				very effective. These techniques have also proven adequate in
				small-scale and densely connected applications such as early neural
				simulations running on prototype SpiNNaker machines with tens of nodes
				\cite{galluppi10} but growing beyond this scale has proven problematic.
				
				\begin{figure}
					\center
					\begin{subfigure}{0.45\linewidth}
						\center
						\buildfig{figures/rcm-initial.tex}
						
						\caption{Original permutation}
						\label{fig:rcm-initial}
					\end{subfigure}
					\begin{subfigure}{0.45\linewidth}
						\center
						\buildfig{figures/rcm-sorted.tex}
						
						\caption{RCM permutation}
						\label{fig:rcm-sorted}
					\end{subfigure}
					
					\caption{Adjacency matrix representation of a graph before and after
					permutation by the RCM algorithm.}
					\label{fig:rcm}
				\end{figure}
				
				A number of algorithms have been proposed for automatically selecting
				good vertex iteration orders, typically using a graph-traversal based
				heuristic. A typical method, described by Hoefler \emph{et al.}
				\cite{hoefler11} exploits the Reverse-Cuthill-McKee (RCM) algorithm
				\cite{cuthill69}. An application's communication matrix is represented
				as an adjacency matrix, $M$, where $M_{i,j}$ is 1 if node $i$ is
				connected by an edge to node $j$ and 0 otherwise. An example matrix is
				illustrated in figure \ref{fig:rcm-initial}. The RCM algorithm uses a
				simple heuristic to permute the matrix (i.e. renumber the nodes in the
				graph) in order to reduce the bandwidth of the matrix. Figure
				\ref{fig:rcm-sorted} shows the RCM-permuted version of the example
				adjacency matrix. When a graph's vertices are ordered as in a
				bandwidth-reduced sparse matrix, vertices close together in the
				ordering are likely to communicate while those further apart tend not
				to communicate.
				
			\subsubsection{Optimisation-based Placement}
				
				% Citations from short report about optimisation in placement...
				% \cite{chen06,jeannot14} and \cite{jeannot10} ("subsets of apps")
				
				In the academic community, a number of attempts have been made to use
				more sophisticated optimisation algorithms for the placement of
				applications. In 1985, Steele \cite{steele85} proposed the use of
				simulated annealing for placing applications in the 6D torus topology
				of the 64 node `Caltech Cosmic Cube' machine. Simulated annealing,
				originally developed by Kirkpatrick \emph{et al.} \cite{kirkpatrick83},
				is a general-purpose optimisation algorithm which works by analogy to
				the physical process of annealing. In brief simulated annealing
				functions by randomly swapping vertices in a candidate placement
				solution, accepting swaps which move connected vertices closer together
				and rejecting some proportion of swaps which move connected vertices
				further apart. The simulated annealing algorithm is described in detail
				later in this chapter.
				
				Towards the end of the 1980s, application placement appeared to be
				becoming less important as super computer network architectures
				improved:
				%
				\begin{displayquote}
					``Careful placement was necessary because of the slow communication
					and non-uniform addressing of early concurrent computers. However,
					the development of message passing machines with fast communications
					and a uniform global address space  has made placement less of an
					issue. In such machines a random placement performs nearly as well as
					an optimum placement.''
					
					\noindent --- W. Dally, 1987 \cite{dally87}
				\end{displayquote}
				%
				In addition, network and problem sizes remained small, so small in fact
				that linear-programming based optimal placement still appeared in
				benchmarks comparing placement algorithms \cite{xu91}. In this
				environment, simpler sequential placement algorithms gained favour over
				more computationally expensive algorithms such as simulated annealing.
				
				As problem and machine sizes have grown and network utilisation has
				once again become an important factor in application performance
				\cite{navaridas09b} more complex optimisation algorithms have
				reappeared in the literature. One popular approach employs graph
				partitioning algorithms such as METIS \cite{karypis98} to perform
				recursive bipartitioning based placement
				\cite{phillips14,hoefler11,pellegrini96}.  This placement process is
				illustrated in figure \ref{fig:partitioning}.
				
				In the first step, the application communication graph and machine
				connectivity graph are bipartitioned such that the number of edges
				between partitions is minimised. Each half of the communication graph
				is associated with one of the halves of the machine connectivity graph.
				The partitioning process is then repeated recursively on each of the
				two communication and connectivity graph pairs. The process halts when
				the graphs can no longer be partitioned at which point the vertices in
				the communication graph are placed on their associated node.
				
				\begin{figure}
					\center
					\buildfig{figures/partitioning.tex}
					
					\caption{Illustration of application placement by recursive
					partitioning.}
					\label{fig:partitioning}
				\end{figure}
				
				TODO: PARTITIONING IS GREAT AND ALL BUT QUALITY ISN'T ALWAYS GREAT AND
				IT DOESN'T DEAL WELL WITH MULTI-CONSTRAINT SCENARIOS E.G. PROCESSOR AND
				MEMORY RESTRICTIONS.
				
				Unfortunately, many of these simply aren't suited to the scale of
				neural applications running on SpiNNaker (e.g. only cope with tens of
				nodes while SpiNNaker may contain hundreds of thousands).
				
				Additionally, a number of algorithms have been developed which make
				assumptions about the topologies of the problem or network. Tree match
				for example attempts to map tree-shaped problems to tree-shaped
				networks. Such algorithms can be highly effective but again do not
				apply to SpiNNaker or its neural applications.
		
		\subsection{Chip placement algorithms}
			
			The chip-design industry has, for many years, dealt with problems
			analogous to the task of placing super computer jobs in a way suited to
			SpiNNaker. Modern CPUs have millions or billions of components with
			strictly fixed connectivity. CPU designers must place each of these onto
			a chip such that the connection lengths are controlled to reduce
			congestion and increase performance. As such, these algorithms are
			ideally suited to future super computer placement work since they already
			operate at the scales required \cite{nam07}.
			
			\subsubsection{Cost functions}
				
				HPWL is popular but a bit crap for high fan-outs. It is, however, quite
				simple.
				
				TODO: SELECT A BETTER COST FUNCTION...
			
			\subsubsection{Simulated annealing}
				
				One of the oldest techniques used for circuit placement is simulated
				annealing and this remains popular today thanks to its sheer
				versatility (see VPR, other open FPGA tools).
				
				SA works by analogy with the physical process of annealing.
				The simulated annealing algorithm works by selecting random pairs of
				components on a chip, swapping them and evaluating some cost function.
				If the swap reduces the cost function, it is kept, if not, depending on
				a function of the current temperature and the cost introduced by the
				swap.
				
				TODO: ILLUSTRATION OF SIMULATED ANNEALING SWAP OPERATION
				
				By occasionally allowing costly swaps, the annealing algorithm avoids
				becoming trapped in local minima. As the algorithm proceeds, the
				temperature is slowly reduced and with it the proportion of costly
				swaps which are retained. This causes the placement to move from
				exploration early on towards refinement later on.
				
				The temperature schedule of an annealing algorithm is critical to its
				success. In general these schedules are computed based on the
				performance of the algorithm as it runs. In VPR the following schedule
				is used.
				
				TODO: DESCRIBE VPR'S SCHEDULE
				
				TODO: FIND AND DESCRIBE ALTERNATIVE SCHEDULE?
				
				Unfortunately, SA is very difficult to parallelise, especially in the
				case of placement. As a result, its scalability has been limited and
				resulted in significantly reduced usage in recent work.
			
			\subsubsection{Partitioning placement}
				
				Partitioning based placement solves the placement problem using
				graph-partitioning recursively on the problem graph to assign each part
				of the circuit to some area in the super chip. Though a number of
				algorithms have proven successful in academic placement contests over
				the years, they are not popular in industrial settings.
			
			\subsubsection{Analytical placement}
				
				In analytical placement, cost function for the circuit graph is
				approximated in a form which is amenable to solutions with standard
				numerical or symbolic algebraic techniques. Using these techniques,
				exact minimum cost (in terms of the approximation) configurations can
				be obtained.
				
				Quadratic placement is a popular analytical placement technique which
				approximates the cost of a placement as the sum of the squares of the
				distances between connected circuit elements.
				
				TODO: FIGURE EXAMPLE QUADRATIC PLACEMENT PROBLEM AND SOLUTION
				
				As such this gives a quadratic cost function like so which we must
				minimise.
				
				TODO: QUADRATIC COST EQN
				
				To minimise the function we differentiate and solve using simple
				symbolic manipulation.
				
				TODO: QUADRATIC COST SOLUTION
				
				Unfortunately, quadratic placement doesn't contain any congestion
				relief by default so various schemes exist. For example, extra anchor
				nodes are inserted which gently pull the circuit components apart from
				each other. As a result, the algorithm generally proceeds by iterating,
				regenerating anchors each time.
				
				Other non-quadratic analytical methods exist too with numerical
				solutions. The approaches are often similar.
			
			\subsubsection{Hierarchical clustering}
				
				Many placement algorithms scale super-linearly with problem size and so
				larger problems become increasingly problematic to handle. To solve
				this problem clustering techniques are first applied to first simplify
				the placement problem. A solution is then found at the coarse level and
				then hierarchically fleshed out.
				
				Various clustering algorithms are in use.
				
				TODO: TALK ABOUT CLUSTERING IN PLACEMENT...
				
				TODO: DESCRIBE THE ALGORITHM I IMPLEMENTED.
	
	\section{Application placement by simulated annealing}
		
		\label{sec:placement-by-annealing}	
		
		I have implemented a simplified SA based application placement algorithm
		based on the approach used in the popular VPR place and route tool chain.
		The algorithm is written in C and is optimised for experimentation rather
		than performance but is production-ready. It has been integrated into the
		`Rig' SpiNNaker software tools and has been used to place very large
		simulations. More on that later.
		
		\subsection{Representation}
			
			Model each chip as having a quantity of various resources (e.g. Cores,
			SDRAM) available. The application graph consists of vertices which each
			consume some quantity of these resources. Vertices must be placed on a
			single chip such that the resources required on a given chip do not
			exceed those available. Vertices are then interconnected by 1:N nets with
			weights which act as hints. The nets are treated as a soft constraint:
			vertices connected via a net will, ideally, be placed near to each other,
			with priority being given to nets with higher weights. Additionally there
			will be a list of placement constraints (see later).
			
			A key observation is that while vertices in an application may frequently
			have a 1:1 correspondence with application cores, this need-not be the
			case. For example, a vertex may represent a block of SDRAM which is
			shared. A vertex may also represent some other resource, for example,
			external IO availability. By making these resource types user-defined,
			applications programmers can express flexible hard-constraints on their
			application.
			
			Another observation is that generic soft constraints can be expressed may
			be expressed using a net with an appropriate weight.
			
			As a result of these facilities, application programmers can easily
			express their own application-specific hard and soft placement
			constraints without having to modify the algorithm. This representation
			has become a de-facto standard for placement problem interchange for
			SpiNNaker applications.
		
		\subsection{Cost function}
			
			At present I've used HPWL despite this being really bad for high-fan-out
			multicast and totally ignorant to the hexagonal nature of SpiNNaker...
			
			To compute bounding boxes for tori I use the following approach. For each
			dimension, sort the points on that dimension and find the largest gap
			between them on a ring. The bounding box goes the other way.
			
			TODO: FIGURE ILLUSTRATING BOUNDING BOX COMPUTATION FOR TORI.
		
		\subsection{Annealing schedule}
			
			The annealing schedule is that used by VPR. Despite being for circuit
			placement, it seems to work jolly well.
			
			TODO: DESCRIBE AND RATIONALISE THE SCHEDULE
		
		\subsection{Constraint handling}
			
			Various hard and soft constraints may be expressed by software
			approaches. For each we explain how they may be handled by the placement
			algorithm:
			
			\subsubsection{Location Constraint}
				
				The vertex is placed on a chip and removed from the set of movement
				candidates.
			
			\subsubsection{Same-chip constraint}
				
				When two vertices must always be placed on the same chip they are
				simply combined into one vertex which consumes the sum of their
				resources. Placement then treats them as one chip and thus is forced to
				atomically place the vertices.
			
			\subsubsection{Reserve resource constraint}
				
				Simply reduce resource availability on that chip.
			
			\subsubsection{Keep near Ethernet}
				
				Simply add a net.
	
	\section{Evaluation}
		
		\label{sec:placement-results}
		
		Though benchmarks exist for super computer loads and chip placement tasks,
		such things don't exist for neural applications. As a result I use a
		selection of real applications for SpiNNaker along with some synthetic
		benchmarks based on biological data.
		
		\subsection{Benchmark networks}
			
			First some real networks.
			
			Some nengo networks: SPAUN: `The world's largest functional brain model'.
			Word-net network from Jamie: Example of some learning.
			
			TODO: DESCRIBE SHAPE OF NENGO NETWORKS
			
			Some PyNN networks: Microcortical column model from PyNN. Note almost
			broadcast connectivity but varying weights. Try and extract a vision
			netlist from Anna. Maybe try and get a netlist for Tom's barrel cortex.
			
			Now for some artificial networks. Pipeline, noisy pipeline, mesh,
			Gaussian 2D.
		
		\subsection{Experiments}
			
			We compare random, linear, greedy and annealing based placement
			approaches to placement. We compare static metrics (such as mean/max
			congestion, table usage) along with experiments based on simulated
			network traffic in real hardware. Network Tester generates artificial
			traffic in proportion with the weights given for each model. We compare
			the relative level of traffic sustainable. We also consider use of
			machines of various sizes.
		
		\subsection{Results}
			
			SA placement is slow but rather effective, especially for some networks.
			Generally worth doing. Will need to be sped up for very large machines...
			
			TODO: EXPERIMENTS!
	

	\chapter{Discussion}

\section{Suitability of the hexagonal torus topology}
	\subsection{Physical scalability}
	\subsection{Routability}
	\subsection{Placeability}

\section{Suitability of the SpiNNaker router}
	\subsection{Deadlock avoidance}
	\subsection{Routing table size}

\section{Suitability of circuit placers for application placement}
	\subsection{Quality}
	\subsection{Runtime}
	\subsection{Routing resources}
	\subsection{Flexibility}
	\subsection{Scalability}


	\chapter{Future research}
	
	In this thesis I have presented a number of new techniques which have made it
	possible to assemble and operate the SpiNNaker super computer. This work
	opens up a range of possibie lines of research to extend this work to future
	architectures and applications. In this chapter I focus on two anticipated
	challenges of future systems: growing scale and greater dynamicism in
	applications.
	
	\section{Scaling up}
		
		TODO: INTRO
		
		\subsection{Grid machine room layouts}
			
			In chapter XXX, I developed a machine room layout for hexagonal torus
			topologies which allowed machines occupying a row of standard
			machine-room cabinets to scale up without the need for long
			interconnecting cables. For larger installations, however, it will be
			necessary to employ multiple rows of cabinets in a 2D arrangement.
		
		\subsection{Routing congestion control}
		
		\subsection{Parallel place and route}
	
	\section{Structural plasticity and dynamic fault-tolerance}
		\subsection{Plasticity models}
		\subsection{Incremental placement}
		\subsection{Incremental routing}
		\subsection{Hot-spare routes}

	\chapter{Conclusions and future research}
	
	The SpiNNaker architecture was designed to tackle the challenges presented by
	the simulation of biologically realistic neural networks. One of its
	distinguishing features is its network architecture which employs both an
	unconventional network topology and multicast router architecture. The
	hexagonal torus topology used by SpiNNaker was chosen to enable greater
	performance while maintaining ease of construction and scalability compared
	with conventional network topologies. SpiNNaker's router design centres
	around packets mimicking the neural `spike' signals they are designed to
	convey by being small, multicast and not guaranteed to arrive at their
	destination.  This novel design, though largely complete before this work
	began, left a number of open problems which this thesis has attempted to
	address.
	
	In this concluding chapter I begin by summarising the answers to the research
	questions raised in chapter~\ref{sec:introduction}. This is followed by a
	discussion of new research topics which have been uncovered by this work.
	
	\section{Answers to research questions}
		
		Each of the three research questions are answered below.
		
		\subsubsection{1. Can the hexagonal torus topology be deployed and used in
		real, large-scale systems?}
		
		In chapter~\ref{sec:building}, I introduced a cabling scheme and assembly
		technique which has been used successfully to build a prototype SpiNNaker
		system with over half a million processor cores using the hexagonal torus
		topology. The techniques shown are expected to enable a final SpiNNaker
		machine of double this size to be built, filling a six metre long row of
		machine-room cabinets.
		
		Though SpiNNaker's processor-count places it amongst some of the world's
		largest supercomputers (see figure \ref{fig:top500-num-processors} on page
		\pageref{fig:top500-num-processors}), it is comparatively compact, filling
		one row of cabinets compared with the warehouse-scale installations found
		in commercial systems. In spite of this, the folding and interleaving
		techniques described allow hexagonal torus topologies to scale to
		arbitrarily large installations without cables which span the machine.
		
		Chapter~\ref{sec:shortestPaths} described an efficient and general
		technique for finding, and enumerating shortest path vectors in hexagonal
		torus topologies. These developments bring the hexagonal torus topology in
		line with other topologies by enabling routing algorithms to exploit all
		possible paths in a network. Further, chapter~\ref{sec:placement}
		demonstrated that placement algorithms are also adaptable to hexagonal
		torus topologies thanks to their similarity to 2D toruses.
		
		Though, as this thesis highlights, hexagonal toruses lack many of the
		intuitive properties enjoyed by other topologies, it is still possible to
		reason about them with only limited computational effort.  Now that the
		practicality and scalability of the topology has also been demonstrated in
		practice, it represents a credible option for future network architectures.
		
		\subsubsection{2. Does SpiNNaker's router architecture help, or hinder
		fault tolerance?}
		
		SpiNNaker's unconventional use of packet dropping to avoid deadlocks
		greatly simplifies the router architecture, part of the motivation for this
		design. In chapter~\ref{sec:routing} this feature is used to the advantage
		of PGS repair to add fault tolerance to existing routing algorithms.
		Compared with the often complex and wasteful methods used to tolerate
		faults in other networks, PGS repair incurs very little performance
		overhead in the presence of static faults.
		
		Routing table usage does increase in the presence of faults, however, which
		may be a concern for applications which already require many routing table
		entries. Routing table usage, as well as other overheads, were most
		significantly increased in the presence of contiguous groups of network
		faults. This is because the PGS repair algorithm produces routes which pass
		tightly around the corners of faults, resulting in concentrations of
		routing table entries in those areas.  Though the symptoms of this problem
		can be attributed to the design of SpiNNaker's multicast routing mechanism,
		the responsibility lies with the behaviour of the PGS repair algorithm.
		Potential improvements to the PGS repair algorithm are discussed later in
		\S\ref{sec:pgs-repair-improvements}.
		
		The overall answer to this research question, therefore, is that the
		flexibility provided to routing algorithms in SpiNNaker's architecture is
		of great benefit, enabling arbitrary fault patterns to be inexpensively
		avoided.
		
		\subsubsection{3. How can the parts of a neural simulation be placed onto a
		large hexagonal torus topology to reduce network load?}
		
		In chapter~\ref{sec:placement}, I explored a number of contemporary
		approaches to the problem of placing applications with irregular
		communication patterns into network topologies. I observed that researchers
		working on circuit placement for chips and FPGAs are tackling similar
		problems and working at scales as large, or larger than, those faced in
		application placement. Based on this I developed a
		simulated annealing based placement algorithm inspired by the techniques
		used in circuit placement, with specific adaptations for use in application
		placement and SpiNNaker's network topology.
		
		The simulated annealing based placement algorithm consistently outperforms
		pre-existing placement algorithms included in benchmarks in terms of
		placement quality.  In the case of one benchmark, simulated annealing based
		placement made it possible to run that neural simulation in real-time for
		the first time.  At larger scales, simulated annealing was also found to be
		able to produce good quality placements in benchmarks containing over one
		million processes -- the largest size supported by the SpiNNaker
		architecture.
		
		The major shortcoming of simulated annealing based placement is its
		execution speed. Though its execution time grows in proportion to the size
		of the problem, the implementation used took over 12~hours to place a
		synthetic problem for the largest planned SpiNNaker machine. Though
		tractable -- particularly given the relative output quality compared with
		the prior state-of-the-art -- the algorithm is unlikely to function
		comfortably as-is on larger problems.
		
		The conclusion to be drawn from this result, however, is not just that
		simulated annealing is a good solution for today's placement problems but
		that circuit placement techniques in general could be successfully adapted
		to fulfil this role. The placement problems faced by chip designers are
		growing at roughly the same exponential rate as the size of super computers
		but circuit designs hold the lead in terms of problem size. Consequently,
		as approaches are retired by chip placement researchers, they may find new
		life in the field of application placement.
		
	\section{Future research}
		
		Though the goals of this study have largely been met, there also remain
		some important limitations which future work may hope to address.
		Furthermore, this work has uncovered a number of new research areas
		warranting future enquiry. This section outlines a number of future lines
		of research.
		
		\subsection{Warehouse-scale cabling}
			
			In chapter~\ref{sec:building} I developed and implemented a number of
			cabling schemes for the SpiNNaker architecture spanning up to a six metre
			row of machine-room cabinets -- a relatively small installation by
			current standards. In SpiNNaker, the cabling exists in a 2D plane (i.e.
			across the faces of the cabinets) but as the system is scaled up, a
			single row of cabinets will tend towards a 1D line. Since embedding a 2D
			structure in a 1D space necessarily results in long connections, this
			cannot scale indefinitely.
			
			\begin{figure}
				\center
				\buildfig{figures/multi-row-cabling.tex}
				
				\caption{Multiple rows of interconnected cabinets.}
				\label{fig:multi-row-cabling}
			\end{figure}
			
			In conventional large-scale super computer installations, nodes are
			installed in rows of cabinets as illustrated in
			figure~\ref{fig:multi-row-cabling}.  From a `bird's-eye' view, the system
			approximates a 2D space, spread across the floor of a machine-room.
			Therefore, in principle, the folding and interleaving techniques
			described in chapter~\ref{sec:building} still apply. Unfortunately for
			SpiNNaker, cables connecting between rows of cabinets would be longer
			than the one metre limit imposed by its hardware because of the spacing
			between rows of cabinets.  Future SpiNNaker systems will need to consider
			alternative link technologies.  For example, a hybrid system could be
			used in which intra-cabinet connections continue to use the current HSS
			link technology while inter-cabinet links might use optical connections.
			This type of architecture could be supported by the use of pluggable
			`SFP+' transceiver modules~\cite{sff01}.
		
		\subsection{Cabling assistance for other architectures}
			
			A secondary result of the construction of prototype SpiNNaker systems in
			chapter~\ref{sec:building} was the use of real-time guidance and feedback
			to assist cable installation. I am not aware of this technique's use by
			existing architectures and, following the success experienced in this
			project, it is possible that the technique may also be useful in
			conventional systems.
			
			During the construction of prototype SpiNNaker machines, each cable took
			seconds to install compared with the minutes reported for existing
			systems~\cite{mudigonda11}. Part of this increase in efficiency appears
			to result from the immediate identification of mistakes made during
			cabling, saving time-consuming backtracking later on.
			
			In many real-world network installations, units are less densely packed
			than in SpiNNaker and so longer cables are often required. As a
			consequence, cabling errors may become more likely as cabling patterns
			are spread over a larger area making them more difficult to visually
			verify. Like SpiNNaker, conventional networking hardware is often
			equipped with a generous range of indicator LEDs and diagnostic
			facilities which might be used to implement real-time installation
			guidance. Future work could explore the use of this technique in the
			construction of other large-scale networks, such as data centres.
		
		\subsection{Congestion mitigation}
			
			\label{sec:wiggly-board-allocations}
			
			In chapter~\ref{sec:routing} I found that contiguous network faults cause
			hot-spots of congestion and routing table depletion where the PGS repair
			algorithm routed many paths around the edges of faults.  However, it is
			not just faults which can cause contiguous blockages in the network
			topology. In reality, researchers do not always require a full-sized
			SpiNNaker system to perform their experiments so large SpiNNaker systems
			are soft-partitioned on demand into many smaller
			machines~\cite{spalloc16}. To ensure isolation between partitioned
			sub-machines, HSS links between boards in different partitions are
			disabled. Because of SpiNNaker's `wrapped triple' partitioning scheme,
			the resulting sub-machines have hexagonal \emph{mesh} topologies (i.e.
			without wrap-around links) with irregular boundaries as in
			figure~\ref{fig:spalloc-mesh}.
			
			\begin{figure}
				\center
				\buildfig{figures/spalloc-mesh.tex}
				
				\caption[Irregular edges of a partitioned SpiNNaker system.]%
				{Irregular edges in a SpiNNaker system comprised of 24~boards
				partitioned from a larger machine.  Each hexagon represents a SpiNNaker
				chip. No wrap-around connections are present.}
				\label{fig:spalloc-mesh}
			\end{figure}
			
			In partitioned systems, the `tooth'-like gaps on the periphery of the
			network result in similar congestion to the HSS link failures considered
			in chapter~\ref{sec:routing}. When a route is generated between nodes on
			opposite sides of a gap, the PGS repair process will produce a
			shortest-path route around it. Since many routes may be blocked by a
			single gap, a hot-spot may develop around the corners of the gap.
			
			In chapter~\ref{sec:placement}, the `CConv' benchmark application was
			found to run correctly the majority of the time when placed by the
			simulated annealing algorithm but would occasionally fail by a
			significant margin. Preliminary experiments suggest these occasional
			failures are caused by placement solutions which place heavily
			communicating parts of the application on opposite sides of gaps along
			the perimeter of the network. Two possible approaches which future work
			may consider are presented below.
			
			\subsubsection{Avoiding hotspots with PGS repair}
				
				\label{sec:pgs-repair-improvements}	
				
				Network congestion around faults and network irregularities could be
				reduced by forcing the PGS repair process to take more varied routes
				around faults. For example, in circuit routing algorithms, routers
				avoid congestion by increasing the cost of routes which pass through
				congested areas~\cite{kahng11}. A similar technique could be used in
				PGS repair to spread the routes it produces.
				
				An alternative approach would be to adapt the base routing algorithms
				used prior to PGS repair to, for example, attempt alternative dimension
				order routes which may avoid blockages due to faulty links.
			
			\subsubsection{Fault and irregularity aware placement}
				
				One of the shortcomings of the simulated annealing based placer
				developed in chapter~\ref{sec:placement} is that it does not account
				for network faults, or irregularities, when estimating the cost of
				placement solutions.  Future work may exploit techniques used in
				congestion-aware circuit placement which could be adapted for
				application placement~\cite{viswanathan07}.
		
		\subsection{Reducing placement execution time}
			
			The simulated annealing based placer presented in
			chapter~\ref{sec:placement} produced good quality placements but its
			execution time limits its use beyond one million vertex placement
			problems. Future work should explore possibilities for improving the
			performance and scalability of this technique.
			
			In addition to considering alternative placement algorithms based on
			other methods, one possible approach is to attempt to reduce the execution
			time of simulated annealing based placement by shrinking the application
			graph being placed.
			
			For example, graph clustering~\cite{schaeffer07} may be used to group
			together strongly connected vertices which would then be placed as a
			single unit.  Unfortunately, clustering can suffer from the same problems
			as graph-partitioning-based placement: vertices may be grouped together
			in ways which, in practice, cannot be packed together into a given portion
			of a machine.  A possible solution to this problem is to use a two-phase
			placement approach~\cite{kahng11}. In a `global' placement phase,
			solutions are permitted which can slightly over-allocate resources but
			overall achieve good placement quality. In the `detailed' placement phase
			which follows, the solution is `legalised' by making small changes to the
			global placement to eliminate over allocation.
			
			An alternative approach suited to SpiNNaker could be to limit the
			clustering process to clusters which fit on a single SpiNNaker chip. In
			typical SpiNNaker application graphs, clustering to this level may reduce
			placement problem sizes by an order of magnitude and, consequently,
			reduce execution times by the same ratio. Preliminary experiments suggest
			that this approach might result in little placement quality loss for
			large placement problems whilst substantially reducing overall execution
			time.
		
		\subsection{Benchmarking}
			
			One of the most significant limitations of this study has been the
			unavailability of large-scale SpiNNaker applications for use as
			benchmarks. As a consequence, much of the scalability experimentation
			performed has relied on simple synthetic benchmarks based on projections
			of future application behaviour.
			
			In the short term, more sophisticated synthetic benchmark generation
			techniques used by the circuit placement community~\cite{nam07} may offer
			alternative benchmarks for future work. In the longer term, however, it
			is hoped that the availability of large SpiNNaker systems -- and
			placement and routing algorithms better suited to exploit them -- will
			lead to larger scale applications being developed. Hopefully these
			applications will lead to more interesting and representative benchmarks
			for use in future work.
	
	\section{Closing remarks}
		
		One of the primary outcomes of this work is that a number of the practical
		challenges faced in scaling up the SpiNNaker architecture have been
		addressed leading to the construction of large-scale SpiNNaker machines.
		The development of an effective placement algorithm for SpiNNaker
		applications has been shown to enable some neural simulations to exploit
		SpiNNaker's architecture for the first time. The availability of larger
		SpiNNaker machines paves the way for future large-scale neural modelling
		work built on much larger models such as Spaun, `the world's largest
		functional brain model'~\cite{eliasmith12}.
		
		Beyond the SpiNNaker project, the hexagonal torus topology has also been
		validated as a scalable and practical candidate for future network
		architectures. As super computers become ever larger, the physical
		scalability afforded by the 2D nature of the hexagonal torus topology may
		make it a compelling option. In addition, the finding that circuit
		placement techniques can be adapted to support placement of SpiNNaker
		software indicates that these algorithms may also be applicable to other
		applications. Indeed, if this is the case, circuit placement may offer a
		long-term source of placement algorithms able to handle the demands of
		future applications.
		
		% This thesis has explored and tackled a number of the challenges posed in
		% scaling up the unconventional SpiNNaker architecture. Along the way I have
		% demonstrated that the hexagonal torus topology may be a practical choice in
		% future applications which can scale up to the physical dimensions expected
		% of future super computers. I have also developed new efficient and
		% effective methods of placing and routing neural simulation software on
		% SpiNNaker which -- it is hoped -- will enable a new generation of large
		% scale neural simulations on spinnaker.
		
		Although this work stops short of demonstrating truly large-scale
		neuroscientific simulations running at the scale of newly completed
		SpiNNaker machines (largely because such simulations do not yet exist) a
		number of smaller-scale neural simulations have been made possible for the
		first time. The algorithms and techniques devised in this work have
		subsequently been incorporated into various software libraries and tools
		now being used by researchers building SpiNNaker applications, vindicating
		the efforts of this thesis (see appendix~\ref{sec:software}). A successor
		to the SpiNNaker architecture is also in the early stages of design and is
		building on experience of the existing architecture. The current intention
		is to retain the hexagonal torus topology used by SpiNNaker, a decision
		supported by the findings of this thesis.
		
		With SpiNNaker's hardware architecture now operating at scales close to its
		architectural limits, it is hoped that the contributions of this work will
		enable researchers to develop larger and more detailed neural models for
		this unique architecture.

	
	% Bibliography
	\bibliography{references}
	\bibliographystyle{alpha}
	
\end{document}
words.
	
	\clearpage
	\listoffigures
	
	\clearpage
	\listoftables
	
	% Abstract
	{
	\prefacesection{Abstract}
	
	% Single line spacing for the abstract page
	\setstretch{1.0}
	
	
	\vfill
	
	% Standard thesis information
	\begin{center}
		\textsc{\large\thesistitle}
		
		\vspace{0.5em}
		
		\thesisauthor
		
		\vspace{0.5em}
		
		A thesis submitted to the University of Manchester\\
		for the degree of Doctor of Philosophy, 2016
	\end{center}
	
	\vfill
	
	% The abstract
	
	SpiNNaker is an unconventional super computer architecture designed to
	simulate up to one billion biologically realistic neurons in real-time. To
	achieve this goal, SpiNNaker employs a novel network architecture which poses
	a number of practical problems in scaling up from desktop prototypes to
	machine room filling installations.
	
	SpiNNaker's hexagonal torus network topology has received mostly theoretical
	treatment in the literature. This thesis tackles some of the challenges
	encountered when building `real-world' systems.  Firstly, a scheme is devised
	for physically laying out hexagonal torus topologies in machine rooms which
	avoids long cables; this is demonstrated on a half-million core SpiNNaker
	prototype.  Secondly, to improve the performance of existing routing
	algorithms, a more efficient process is proposed for finding (logically)
	short paths through hexagonal torus topologies. This is complemented by a
	formula which provides routing algorithms greater flexibility when finding
	paths, potentially resulting in a more balanced network utilisation.
	
	The scale of SpiNNaker's network and the models intended for it also present
	their own challenges. Placement and routing algorithms are developed which
	assign processes to nodes and generate paths through SpiNNaker's network.
	These algorithms reduce congestion and tolerate network faults. The proposed
	placement algorithm is inspired by techniques used in chip design and is
	shown to enable larger applications to run on SpiNNaker -- with good
	performance -- than the previous state-of-the-art. Likewise the routing
	algorithm developed is able to tolerate network faults, inevitably present in
	large scale systems, with little performance overhead.
	
	
	% Required to ensure single line spacing is used for this whole block
	\par%
}

	
	% Declaration of non-submission elsewhere
	\prefacesection{Declaration}

% Single line spacing for the declaration
{
	\setstretch{1.0}
	No portion of the work referred to in this thesis has been submitted in support
	of an application for another degree or qualification of this or any other
	university or other institute of learning.
	
	\par%
}


	
	% University-prescribed copyright statement...
	\input{copyright}
	
	% Acknowledgements
	{
	\prefacesection{Acknowledgements}
	
	% Single line spacing
	\setstretch{1.0}
	
	It is often said that it is not \emph{what} you know but \emph{who} you know.
	Throughout the course of my PhD I've been exceptionally lucky to have been
	helped along by a great number of people.
	
	Both my supervisor, Jim Garside, and co-supervisor, Steve Furber, have each
	spent countless hours patiently discussing and describing all manner of
	things with me while giving me great freedom in my project. Jim's office door
	has always been open to my unexpected interruptions be it about work, writing
	or walking.  Likewise, Steve has always managed to find time for both
	technical and frivolous endeavours alike. I'm also hugely grateful to Luis
	Plana who has been a rich source of sage advice, insightful questions
	patiently suffered many a foolish question.
	
	Various parts of the work in this thesis (and numerous side projects) would
	not have been possible if not for the multitude of discussions,
	collaborations and even sheer physical hard work of Steve Temple, Javier
	Navaridas, Simon Davidson and Dave Clark. I'm also indebted to Andrew Mundy
	and Jamie Knight, both of whom have donated so much time and effort towards
	verifying and using software implementations of the ideas in this thesis.
	
	The injection of lunchtime silliness by Andrew and Jamie, along with Amanieu
	d'Antras and Andrew Webb and the other CDT members has always brightened my
	day. So to has the friendly and stimulating environment of the School of
	Computer Science and its many staff and students. Of course, I am also very
	grateful for the funding the school has provided for my research.
	
	I cannot thank my wonderful wife, Ann-Marie, enough for being by my side. She
	has given me so much kindness, love and patience and endured a lifetime's
	quota of conversations about hexagons. Finally, thanks too to rest of my
	family, especially my parents, who are to blame for starting me down this
	path and co-suffering drafts and endless rants about this document.
	
	% Required to ensure single line spacing is used for this whole block
	\par%
}

	
	% Main body
	\chapter{Introduction}

\label{sec:introduction}

%Problem area
%
%* Network construction and exploitation
%  * Cabling: Build it cheaply in terms of tech cost and install cost
%  * Routing: Get around it cheaply and reliably
%  * Placement: Use it efficiently

The Spiking Neural Network Architecture (SpiNNaker) is a novel super computer
architecture designed to simulate biologically realistic models of brains in
real time \cite{furber07}. Though neurons, the building blocks of the brain,
are relatively well understood, their complex interactions remain mysterious.
Just as understanding the workings of a transistor is insufficient to
understand a modern microprocessor, neuroscientists believe that understanding
the neurons in isolation cannot explain the brain and that understanding their
connectivity is key \cite{eliasmith13,eliasmith14}. Experiments on real brains,
however, are fraught with difficulty. Variations between individuals can be
significant and it is only possible to record tens or hundreds of the trillions
of signals in the brain, and even then only with limited control over which
signals are recorded. Computer simulations of models of large neural networks,
however, enable researchers to develop repeatable experiments and gain complete
visibility of any signal and any neuron. Models such as SPAUN
\cite{eliasmith12}, built from millions of simulated neurons, have shown great
promise in expanding our understanding of higher level brain functions such as
memory and simple problem solving.  Unfortunately these neural models are
expensive to simulate, requiring hours of compute time to simulate each second
of neural activity. As well as being inconvenient, this precludes the use of
robotics to immerse these models in real world environments and also limits
studies of long-term behaviours such as learning.

SpiNNaker is designed to enable the real time simulation of models containing
up to one billion neurons -- approximately \SI{1}{\percent} of a human brain or
ten mouse brains \cite{furber06}. To achieve this goal, the largest planned
SpiNNaker machine will contain over one million low-powered computer processors
interconnected by a bespoke network architecture.

SpiNNaker's large processor count matches the current trend in super computers
where processor counts are growing exponentially \cite{meuer16j}, mimicking the
growth of the number of components in the processors themselves predicted by
Gordon Moore's famous `law' \cite{moore75}. As a result of this growth, the
interconnection networks which enable these processors to work together have
grown in importance \cite{dally04}.  Network designers must carefully balance
performance against practicality and financial cost.  SpiNNaker's network is no
exception to this rule and, as the systems scale up from desktop prototypes to
machine-room scale installations, the reality of building and exploiting these
machines presents an array of challenges.

As in all super computers, SpiNNaker's network interconnects its processors in
a particular network topology which defines how different processors may
communicate with each other. Unlike the tree and $N$-dimensional torus
topologies found in contemporary super computers \cite{dally04}, SpiNNaker
employs a `hexagonal torus topology'. In this topology, nodes in SpiNNaker's
network fit together in a honeycomb-like pattern where messages may `hop' from
node to node to reach their destination. As we will see in
chapter~\ref{sec:background}, the hexagonal torus topology, in theory, sits at
a `sweet spot' in terms of network performance and practicality. As the first
known large-scale installation of the hexagonal torus topology, however, there
remain a number of practical challenges for large spinnaker machines arising
from this choice.

As super computer networks have grown in scale to millions of processors the
task of dealing with previously rare faults has grown.  Though fault rates in
networks remain consistently low, architectures such as SpiNNaker may have
hundreds of thousands of links meaning even fault rates of a fraction of a
percent will impact tens or hundreds of links. To enable reliable operation,
networks must be able to adapt the routes taken by messages through the network
to avoid faulty links and nodes. The techniques employed are often closely tied
to a particular network architecture and consequently SpiNNaker's novel network
architecture demands its own approach.

Another challenge introduced by the growing scale of super computers is making
\emph{efficient} use of network resources. Communicating processes should be
located on logically `nearby' nodes to reduce network load. The neural models
for which SpiNNaker is designed are often described abstractly, rather than
geometrically, using modelling languages such as PyNN~\cite{davison08} and
Nengo~\cite{eliasmith04}.  Because of this, the communication requirements of
simulations can be highly irregular making an efficient placement of processes
onto processors in the machine non-trivial.

%Contributions
%
%* Cabling scheme for hexagonal toruses without long cables
%* Efficient installation technique for dense systems
%* Exhaustive and efficient route calculation in hex toruses
%* Fault tolerant routing scheme exploiting SpiNNaker's odd router
%* Placement based on SA a: works very well and b: suggests circuit placement is
%  a good source of inspiration.

This thesis addresses the practical challenges of scaling up the SpiNNaker
architecture in a real-world setting summarised by these research questions:

\begin{enumerate}
	
	\item Can the hexagonal torus topology be deployed and used in real, large
	scale systems?
	
	\item Does SpiNNaker's router architecture help, or hinder fault tolerance?
	
	\item How can the parts of a neural simulation be placed onto a large
	hexagonal torus topology to reduce network load?
	
\end{enumerate}

%Structure
%
%* Chapter 2: Background: detailed dive into what's in SpiNNaker, why its
%  really so unusual. Also looks at what applications run on SpiNNaker and how
%  they work.
%* Chapter 3: How to build a really big SpiNNaker machine.
%* Chapter 4: How to find your way around that machine.
%* Chapter 5: How to find your way around that machine even when its broken.
%* Chapter 6: Now you can walk, time to run.
%* Chapter 7: Wrapping up.
%* Appendices: Hard-to-come-by theoretical and practical details useful if
%  you're about to continue where this research left off but be useful but
%  otherwise hard to come by, especially in one place.

Chapter~\ref{sec:background} introduces the SpiNNaker architecture and, in
particular, describes its hexagonal torus topology and network architecture.

In chapter~\ref{sec:building}, I develop a cabling scheme for large hexagonal
torus topologies which enables arbitrarily large networks to be constructed
using only short, inexpensive cables. This theoretical work is then evaluated
through the construction of a range of prototype SpiNNaker systems. The largest
of these prototypes contains over half a million processor cores and spans
several machine room cabinets. In addition, I propose the use of built-in
diagnostic facilities to assist technicians performing network installation and
maintenance. This technique is found to greatly reduce the effort required and
the number of mistakes made.

In chapters~\ref{sec:shortestPaths}~and~\ref{sec:routing} I develop new routing
techniques for SpiNNaker's network. Chapter~\ref{sec:shortestPaths} develops a
new approach to finding the shortest paths through hexagonal torus topologies,
an integral part of many routing algorithms. This newly proposed approach is
cheaper to compute than the state of the art and, unlike previous efforts, is
able to discover all valid short paths through the topology. This theoretical
advance brings hexagonal torus topologies in line with conventional topologies
by providing routing algorithms with complete information about the paths
available to them. In chapter \ref{sec:routing} I propose a fault tolerant
routing algorithm for SpiNNaker which is able to avoid arbitrary static fault
patterns with minimal performance overhead. A key finding of this chapter is
that the flexibility afforded to fault tolerant routing algorithms by
SpiNNaker's unconventional router architecture is what facilities the low
overheads reported in this chapter.

Finally, in chapter~\ref{sec:placement}, I explore the problem of application
placement in SpiNNaker's network. As in other networks and applications, neural
simulations should be arranged such that communication occurs primarily between
processors close together in the network to control network load. Due to the
irregular connectivity and large scale of the neural models expected to run on
SpiNNaker, an automated approach is necessary. I develop a novel placement
algorithm based on algorithms used for circuit layout in computer chips. My
algorithm is found to allow some larger neural models to run on SpiNNaker for
the first time while enabling other applications to run at greater speeds. In
addition, synthetic benchmarks containing over one million processes indicate
that this algorithm should handle the anticipated demands of the neural models
expected to run on large-scale SpiNNaker installations.

	\chapter{The SpiNNaker Architecture}
	
	\label{sec:background}
	
	SpiNNaker is a massively parallel computer architecture designed to simulate
	biologically realistic neural models \cite{furber07}. In this chapter we will
	explore this unconventional architecture in detail, starting with its purpose
	before focusing on its most unconventional feature: its network.
	
	% * Purpose
	%   * Spiking neural simulations
	%     * Neural modelling: PyNN, Nengo...
	%     * Parallelisation + communication
	
	\section{Neural simulation}
		
		Human brains contain billions of neurons connected together by trillions of
		`synapses'. Neurons communicate by transmitting and receiving `spikes'
		through their synapses. Each spike is `valueless' in that a spike's only
		significant features are when it arrives and where it has come from.
		
		\begin{figure}
			\center
			\buildfig{figures/lif-neuron.tex}
			
			\caption{A Leaky Integrate-and-Fire (LIF) neuron.}
			\label{fig:lif-neuron}
		\end{figure}
		
		Though some detailed models of the electrochemical processes occurring
		inside neurons are computationally intensive, simplified models such as the
		Leaky Integrate-and-Fire (LIF) model can be implemented in just a handful
		of CPU instructions \cite{vainbrand11}. Figure~\ref{fig:lif-neuron}
		illustrates a simple LIF neuron in which incoming spikes cause charge to
		build up (integrated) which over time, leaks away. If an incoming spike
		causes the charge to rise above a certain threshold, the neuron `fires'
		producing an outgoing spike. Despite the simplicity of this model, large
		neural networks such as Spaun \cite{eliasmith12} -- built entirely from LIF
		neurons -- exhibit complex behaviours such as fine motor control and
		problem solving.
		
		The computational expense of large scale neural simulations does not arise
		from the cost of modelling neurons but instead from distributing spikes. In
		biology, neurons produce spikes at an average rate of \SI{10}{\hertz} and
		synapses connect each neuron's output to (order) \num{1000}~neurons
		\cite{navaridas09}. Consider an example neural model with $7\times10^7$
		neurons, approximately the number in a house mouse and
		$\nicefrac{1}{10}^\textrm{th}$ of the design target of SpiNNaker. This
		network might produce $7\times10^8$~spikes per second. Because each neuron
		connects to many others, this equates to $7\times10^{11}$ spikes being
		received per second. If each spike were transmitted as a UDP datagram
		containing a single \SI{32}{\bit} payload, the total network throughput
		required for this simulation would be \SI{179.2}{\tera\bit\per\second}. At
		the time of writing, this is more than double the bisection bandwidth (the
		theoretical worst-case throughput) of the world's most powerful super
		computer \cite{dongarra16}.
	
	\section{Network architecture}
		
		Architectures such as IBM's Blue Gene \cite{chiu11} and Cray's XK7
		\cite{ornl16} employ powerful compute nodes connected together using
		networks designed to transfer multi-kilobyte blocks of data between nodes.
		Since neural models have relatively light computational requirements and
		communications are based on small pieces of data (spikes), this type of
		architecture is poorly suited to the task.
		
		SpiNNaker's architectural target is to support realtime simulations of up
		to one billion neurons. Since neural models such as LIF are inexpensive to
		model and many neurons can be simulated independently in parallel,
		SpiNNaker employs many small, energy efficient ARM processors
		\cite{furber07}. To support the unusual communication requirements of
		neural simulations, a bespoke interconnection network is used which is the
		background to this thesis.
		
	%   * SpiNNaker chip
	%     * Cores
	%     * SDRAM
	%     * NoC
	%     * Router
		
		\begin{figure}
			\center
			%\includegraphics[width=19mm]{figures/spinnakerChip.jpg}
			\buildfig{figures/hex-chips.tex}
			
			\caption[SpiNNaker chips connected to their six neighbours.]%
			{SpiNNaker chips (actual size) connected to their six neighbours.}
			\label{fig:spinnakerChip}
		\end{figure}
		
		The fundamental building block of the SpiNNaker architecture is the
		SpiNNaker chip (figure \ref{fig:spinnakerChip}) \cite{furber13}. Each chip
		contains eighteen low power ARM 968 processor cores each capable of
		simulating between \num{200} and \num{2000} LIF neurons in real time
		\cite{mundy15}.  Each core has a total of \SI{96}{\kilo\byte} of private
		Tightly-Coupled Memory (TCM) and shares access to \SI{128}{\mega\byte} of
		on-chip SDRAM with other cores on the same chip. Finally, each chip
		contains a programmable router which routes network packets to and from the
		local cores and six neighbouring SpiNNaker chips. SpiNNaker machines are
		constructed by combining many SpiNNaker chips.
		
		\begin{figure}
			\center
			\buildfig{figures/spinnaker-packet.tex}
			
			\caption{SpiNNaker's \SI{40}{\bit} and \SI{72}{\bit} multicast packet
			format.}
			\label{fig:spinnaker-packet}
		\end{figure}
		
		Processor cores can communicate by sending and receiving network packets
		forwarded by routers through the network. Since SpiNNaker's network is
		designed to transmit neural spike events efficiently, individual network
		packets are small, either \SI{40}{\bit} or \SI{72}{\bit} compared with tens
		or hundreds of byte packets in typical network architectures.
		
		In a real-time simulation, the time at which a spike is produced is
		implicitly indicated by the time it is received -- since at biological
		timescales a computer network delivers packets `instantaneously'.
		Consequently, the only information which must be explicitly encoded is the
		identity of the neuron which produced the spike. In SpiNNaker, a spike may
		be encoded by using a single \SI{40}{\bit} `multicast packet' whose format
		is illustrated in figure~\ref{fig:spinnaker-packet}.  The \SI{8}{\bit}
		header is used by SpiNNaker's routers to determine the type of packet and
		the \SI{32}{\bit} `routing key' is used to identify the neuron which
		produced the packet. The routing key is also used by SpiNNaker's routers to
		determine how the packet should be directed through the network.
		
		The optional \SI{32}{\bit} payload is not used by conventional spiking
		neural simulations \cite{galluppi10} but has been exploited to enable more
		efficient simulation of a particular class of neural models \cite{mundy15}.
	
	\section{The SpiNNaker router}
		
		The SpiNNaker router employs an unconventional design which, despite its
		compact size and small energy requirements, implements a flexible multicast
		routing scheme. Unlike conventional routers which often employ hard-coded
		routing rules \cite[chapter~8]{dally04}, the SpiNNaker router uses a
		programmable `routing table' to determine how packets should be forwarded.
		In addition, to avoid deadlocks, SpiNNaker's router employs a simple,
		timeout-based mechanism which exploits the ability of neural networks to
		tolerate occasional missing packets. As we will see in chapter
		\ref{sec:routing}, this mechanism greatly simplifies the task of routing in
		SpiNNaker's network. In this section we'll look at these features in
		greater detail.
		
		\subsection{Routing tables}
		
			When a multicast packet arrives at a SpiNNaker router (either from a
			local core or a neighbouring chip), the router looks up the routing key
			in its routing table. This table consists of \num{1024} programmable
			table entries, each specifying a routing key bit pattern and mask to
			match and a set of routes.  When a multicast packet's key is matched by a
			routing entry the packet is forwarded along every route specified by that
			entry, potentially duplicating the packet. This `multicast' technique
			allows packets to be transmitted once but received in a number of places
			while making efficient use of the network \cite{navaridas12}.
			
			Though routing table entries are in finite supply (\num{1024} entries per
			router), it is still possible for many thousands of traffic flows to be
			routed through a single router. The bit pattern and mask in each routing
			entry matches against the 32~bits of a routing key as either
			`\texttt{1}', `\texttt{0}' or `\texttt{X}' (don't care).  This means that
			a single routing entry may, for example, be used to match all routing
			keys with a certain prefix. If a routing key is not matched by any entry
			in the routing table then the packet is `default routed' in a straight
			line. For example if a packet with an unmatched key is received from the
			chip to the left, the packet will be default routed to the chip on the
			right. By assigning routing keys such that neurons whose spikes are sent
			to similar destinations share a similar prefix, the number of routing
			entries required by a simulation is greatly reduced \cite{davies12}.
			
			\begin{figure}
				\center
				\buildfig{figures/routing-example.tex}
				
				\caption[Multicast routing example.]%
				{Multicast routing example with \SI{4}{\bit} routing keys. Each
				box represents a SpiNNaker chip whose router has been programmed with
				the routing entries shown. Grey lines mark connections between chips.}
				\label{fig:routing-example}
			\end{figure}
			
			Consider the simplified example in figure~\ref{fig:routing-example} in
			which a number of (\SI{4}{\bit}) routing table entries have been
			configured in the routers of a small SpiNNaker network. If a packet with
			the routing key \texttt{1011} is transmitted by a core in the chip
			labelled $(0, 0, 0)$, this will match the first routing table entry on
			that chip and will be routed to chip $(1, 0, 0)$. On chip $(1, 0, 0)$,
			the packet once again matches the first routing entry and is routed to
			chip $(1, 0, -1)$. On $(1, 0, -1)$, no match is made so the packet is
			default routed to $(1, 0, -2)$. On this chip, the packet matches a
			routing entry which routes the packet to core~7. In this example, default
			routing allows only three routing table entries to direct a packet
			through four chips.
			
			As a second example, if a packet with the routing key \texttt{0010} is
			transmitted by a core on chip $(0, 0, 0)$, this key will be matched by
			the second routing entry since \texttt{X}s in the table entry will match
			both \texttt{1}s and \texttt{0}s in the corresponding bits of the routing
			key. When the packet arrives at chip $(0, 0, -1)$ the matching routing
			entry forwards the packet to both $(0, 1, -1)$ and $(1, 0, -1)$
			simultaneously. The copy of the packet arriving at $(0, 1, -1)$ is routed
			to core~5 on that chip.  Meanwhile, the copy forwarded to $(1, 0, -1)$ is
			duplicated again with one copy being routed to core~11 and another being
			routed to chip $(1, 0, -2)$. Here the packet is finally delivered to
			core~6. In this example, the ability of the router to multicast
			(duplicate) packets as they pass through the network meant that sending
			one copy of the packet was sufficient to reach three destination cores.
			In addition, by using \texttt{X}s in the routing table entry, the same
			routing entries are sufficient to route packets with the keys
			\texttt{0000}, \texttt{0001}, \texttt{0010} and \texttt{0011}.
			
			In spite of these mechanisms, it is still possible for an application to
			run out of routing table entries. As we will see in
			chapter~\ref{sec:placement} by arranging applications appropriately
			within SpiNNaker's network, routing table usage can be reduced. In
			addition, other behaviours of SpiNNaker's router may be exploited to
			compress an applications routing tables further, however the techniques
			employed are beyond the scope of this thesis \cite{mundy16}.
		
		\subsection{Timeouts}
			
			SpiNNaker's router is built on a pipeline architecture. As shown in
			figure~\ref{fig:router-architecture}, the router is fed packets by an
			arbiter which serialises packets arriving from other chips and local
			cores. Every (\SI{100}{\mega\hertz}) clock cycle, the router pipeline
			accepts one packet from the arbiter and routes a packet to one or several
			output links. If any of the required output ports are busy then the
			packet is not forwarded to any output link and the pipeline stalls. Once
			a packet has been blocked for a programmable timeout, it is dropped
			(discarded) and routing continues as usual for next packet in the
			pipeline. Links become blocked while transmitting packets or waiting for
			the remote receiver to become ready. For example, a receiving processor
			core may be busy performing some computation or a receiving router may be
			blocked waiting for some of its outputs to become ready.
			
			\begin{figure}
				\center
				\buildfig{figures/router-architecture.tex}
				
				\caption{SpiNNaker router architecture}
				\label{fig:router-architecture}
			\end{figure}
			
			The timeout-based packet dropping mechanism is designed to defuse
			deadlocks in the network. For example, if two routers are trying to send
			each other a packet at the same time they may become deadlocked, each
			waiting for the other router to accept a packet before continuing.
			SpiNNaker's timeout mechanism breaks deadlocks by dropping packets which
			have been blocked for some time and therefore may be in a deadlock.  Once
			a packet has been dropped it is left to software to either tolerate the
			missing packet or trigger a retransmission. In neural simulations, as in
			biology, the loss of a single spike is unlikely to have a significant
			impact on the behaviour of a neural model and therefore these simulations
			are inherently tolerant of occasional dropped packets. During application
			loading and other system tasks, a higher level, software driven protocol
			based on acknowledgements and retransmissions is used to ensure
			guaranteed delivery.
			
			% TODO: MENTION TIMEOUT VALUE USED?
			% Router timeouts must be configured to be long enough that delays in
			% packet transmission, for example due to the time taken for packets to
			% traverse a link, do not trigger packet dropping. Conversely, the timeout
			% should be as short as possible to reduce the time the router is
			% blocked and maximise network throughput.
	
	\section{The hexagonal torus topology}
		
		Each SpiNNaker chip is a node in a `hexagonal torus topology' as
		illustrated in figure~\ref{fig:hexagonalTorusTopology}. Network packets
		sent by SpiNNaker's processor cores may `hop' through several nodes in the
		network to reach their intended destination. In each hop, a packet may
		advance one node along one of the three axes of the topology. For example,
		a packet sent by the node labelled $\alpha$ (in the bottom-left corner) to
		the node labelled $\beta$, might take the following sequence of hops:
		X$^+$, X$^+$, Z$^-$. Packets sent from $\alpha$ to $\gamma$ might take the
		route: X$^-$, X$^-$, Y$^+$, Y$^+$. The first hop of this route `wraps
		around' from the bottom-left node to the bottom-right node in a single hop.
		
		\begin{figure}
			\center
			\buildfig{figures/hexagonalTorusTopology.tex}
			
			\caption[A hexagonal torus topology.]%
			{A hexagonal torus topology. Each hexagon represents a node (i.e.
			a SpiNNaker chip). Touching nodes are directly connected. Nodes on edges
			$a$, $b$ and $c$ are also directly connected to the corresponding nodes
			on edges $a'$, $b'$ and $c'$, respectively. The three axes of the
			hexagonal torus topology, `X', `Y' and `Z' are also shown.}
			\label{fig:hexagonalTorusTopology}
		\end{figure}
		
		\begin{figure}
			\center
			\begin{subfigure}{0.39\linewidth}
				\center
				\includegraphics[width=\linewidth]{figures/torus-3d-flat.pdf}
				\caption{}
				\label{fig:torus-3d-flat}
			\end{subfigure}
			~~
			\begin{subfigure}{0.26\linewidth}
				\center
				\includegraphics[width=\linewidth]{figures/torus-3d-tube.pdf}
				\caption{}
				\label{fig:torus-3d-tube}
			\end{subfigure}
			~~
			\begin{subfigure}{0.23\linewidth}
				\center
				\includegraphics[width=\linewidth]{figures/torus-3d-torus.pdf}
				\caption{}
				\label{fig:torus-3d-torus}
			\end{subfigure}
			
			\caption{Visualisation of a hexagonal torus topology as a torus.}
			\label{fig:torus-3d}
		\end{figure}
		
		The wrap around connections in the topology are what give it the `torus'
		part of its name. Figure~\ref{fig:torus-3d-flat} shows a hexagonal torus
		topology drawn flat as in the previous figure. If the topology is rolled up
		into a tube such that the top and bottom nodes become directly adjacent, a
		tube is formed as in figure~\ref{fig:torus-3d-tube}. This tube can then be
		bent to bring together the nodes at the ends of the tube to form a torus as
		shown in figure~\ref{fig:torus-3d-torus}.
		
		A hexagonal torus topology is typically defined in terms of its width and
		height along the X and Y axes respectively. For example,
		figure~\ref{fig:hexagonalTorusTopology} shows a $10\times10$ hexagonal
		torus.  The nodes in a hexagonal torus topology are addressed using
		hexagonal coordinates of the form $(x, y, z)$ \cite{patel15}. The bottom
		left node (labelled $\alpha$ in the figure) has the coordinate $(0, 0, 0)$
		and other nodes are assigned coordinates according to the number of hops
		along each dimension from $(0, 0, 0)$, for example node $\beta$ has the
		coordinate $(2, 0, -1)$. Because the hexagonal torus topology's axes are
		non-orthogonal, it is possible to define several coordinates for the same
		location. For example $(3, 1, 0)$ and $(1, -1, -2)$ are also valid
		coordinates for node $\beta$. These dual coordinates emerge from the fact
		that adding $(1, 1, 1)$ to a coordinate produces an equivalent, but
		different, coordinate. This phenomenon is explained in detail in
		appendix~\ref{app:minimal-hex-coordinates} and related phenomena will be
		discussed in chapter~\ref{sec:shortestPaths}.
		
		The hexagonal torus topology was chosen over a more conventional network
		topology -- such as a 2D or 3D torus (sometimes known as a 2-ary $N$-cube
		or 3-ary $N$-cube respectively) \cite[chapters~3~and~5]{dally04} -- due to
		its balance of theoretical performance and practicality. The bisection
		bandwidth of a topology indicates the theoretical worst-case total
		throughput the network is able to sustain \cite[chapter~1]{dally04}.  In
		networks with homogeneous link throughput, bisection bandwidth is
		determined by the number of links cut by a balanced bisection of the
		network.  Figure~\ref{fig:bisection-bandwidth} illustrates the bisections
		of several torus topologies.
		
		\begin{figure}
			\center
			\begin{subfigure}[b]{0.3\linewidth}
				\center
				\buildfig{figures/bisection-bandwidth-2d.tex}
				
				\caption{2D Torus}
				\label{fig:bisection-bandwidth-2d}
			\end{subfigure}
			\begin{subfigure}[b]{0.3\linewidth}
				\center
				\buildfig{figures/bisection-bandwidth-hex.tex}
				
				\caption{Hexagonal Torus}
				\label{fig:bisection-bandwidth-hex}
			\end{subfigure}
			\begin{subfigure}[b]{0.3\linewidth}
				\center
				\buildfig{figures/bisection-bandwidth-3d.tex}
				
				\caption{3D Torus}
				\label{fig:bisection-bandwidth-3d}
			\end{subfigure}
			
			\caption[Bisections of torus topologies.]%
			{Bisections of torus topologies. Connections cut by the bisection
			are drawn as lines.}
			\label{fig:bisection-bandwidth}
		\end{figure}
		
		In a $N \times N$ 2D torus topology, the bisection bandwidth is $2N$~links
		and each node requires four links. The hexagonal torus topology requires
		six links per node but provides double bisection bandwidth ($4N$~links).
		The 3D torus topology also requires six links per node but by connecting
		the nodes differently achieves a bisection bandwidth of $8N$~links.  The 3D
		torus topology, however, comes at a price -- when immersed into the
		(approximately) 2D space provided by a large machine room or row of server
		cabinets, some connections require long cables. By contrast, the 2D and
		hexagonal torus topologies are both inherently two dimensional and
		consequently do not suffer from this effect. The hexagonal torus topology,
		therefore, shares the practicality of construction of a 2D torus while
		still gaining some of the performance of a 3D torus topology. In addition,
		because nodes in a hexagonal torus topology have a greater number of links,
		greater redundancy is available in the network to tolerate faults.
		
		Most torus topologies, including hexagonal, 2D and 3D toruses, have a
		related `mesh' topology. These mesh topologies maintain the same general
		connectivity structure as their torus topologies but omit wrap-around
		links. In practice, this saves a small number of links at the expense of
		halving the network's bisection bandwidth.  Because of their poorer
		performance, mesh networks are rarely used as the basis of a network
		architecture. Mesh networks, however, are occasionally formed when a
		network is partitioned into several smaller sub-networks to allow multiple
		users to share a system \cite{spalloc16}.
		
		\begin{figure}
			\center
			\begin{subfigure}[b]{0.45\linewidth}
				\center
				\buildfig{figures/hexagonal-torus.tex}
				\caption{Hexagonal torus}
				\label{fig:topo-compare-hexagonal-torus}
			\end{subfigure}
			\begin{subfigure}[b]{0.45\linewidth}
				\center
				\buildfig{figures/h-torus.tex}
				\caption{H-torus}
				\label{fig:topo-compare-h-torus}
			\end{subfigure}
			
			\caption[Hexagonal torus vs. H-torus topology.]%
			{Hexagonal torus vs. H-torus topology. Each numbered hexagon
			represents a node. The thick outline indicates the bounds of the
			topology after which the network repeats. In each topology, the path
			taken by advancing in the Y$^+$ direction from the node labelled `0' is
			shown.}
			\label{fig:topo-compare}
		\end{figure}
		
		\label{sec:hex-vs-h-torus}
		
		The hexagonal torus topology is not to be confused with the `H-torus'
		topology. This topology also uses a hexagonal tiling of nodes and even
		wraps this tiling into a torus-like topology \cite{zhao08}. However,
		H-torus topologies have very different characteristics to the hexagonal
		torus topology and are related to `twisted torus' topologies
		\cite{camara10}. For example, figure~\ref{fig:topo-compare} illustrates one
		major difference in the way paths wrap around the peripheries of both
		topologies.
	
	\section{Scaling-up SpiNNaker machines}
		
		To build large SpiNNaker systems comprising of tens of thousands of
		SpiNNaker chips, groups of forty-eight chips are mounted onto circuit
		boards as illustrated in figure~\ref{fig:spinnakerBoard}. These boards may
		be connected together to form larger systems.  Figure~\ref{fig:threeboard}
		shows a prototype three board system. Though the chips are
		\emph{physically} arranged in a (nearly) $7\times7$ grid on each SpiNNaker
		board, they logically form a hexagonal `wrapped triple'
		\cite{davidsonWiring} (see appendix~\ref{sec:partitioning}) which logically
		fit together as illustrated in figure~\ref{fig:threeboard-separate}. The
		labelled exposed corners of the three forty-eight chip boards connect
		together to form a $12\times12$ hexagonal torus topology as illustrated in
		figure~\ref{fig:threeboard-wrapped}. Larger SpiNNaker machines are
		assembled by combining more boards.
		
		\begin{figure}
			\center
			\begin{subfigure}[b]{0.45\linewidth}
				\center
				\includegraphics[width=\linewidth]{figures/spinnakerBoard.jpg}
				
				\caption{A SpiNNaker board}
				\label{fig:spinnakerBoard}
			\end{subfigure}
			~~~
			\begin{subfigure}[b]{0.45\linewidth}
				\center
				\includegraphics[width=\linewidth]{figures/threeboard.jpg}
				
				\caption{Three board prototype}
				\label{fig:threeboard}
			\end{subfigure}
			
			\vspace*{1em}
			
			\begin{subfigure}[b]{0.45\linewidth}
				\center
				\buildfig{figures/threeboard-separate.tex}
				
				\caption{Three board topology}
				\label{fig:threeboard-separate}
			\end{subfigure}
			~~~
			\begin{subfigure}[b]{0.45\linewidth}
				\center
				\buildfig{figures/threeboard-wrapped.tex}
				
				\caption{\ldots{}as a parallelogram}
				\label{fig:threeboard-wrapped}
			\end{subfigure}
			
			\caption{SpiNNaker boards and their topology.}
			\label{fig:spinnaker-boards}
		\end{figure}
		
		
		SpiNNaker chips on the same circuit board connect using low power links
		requiring sixteen wires each.  If this link technology were used to connect
		chips on neighbouring boards, each pair of boards would need to be
		connected with a 128~wire cable.  Cables and connectors supporting this
		many signals are expensive, unreliable and physically large. Instead,
		chip-to-chip connections between boards are multiplexed and demultiplexed
		onto a single High-Speed Serial (HSS) link \cite{athavale05} carried via
		commodity S-ATA cables which are often used to connect hard disks in
		desktop computers and servers \cite{sata3spec}. The six high-speed links
		are implemented by three onboard FPGAs (the three large chips at the top of
		the SpiNNaker board) and are logically transparent to the underlying
		network. The underlying technology and the choice of S-ATA cables limits
		each board-to-board connection to spanning at most one metre gaps. In
		chapter~\ref{sec:building} I present a cabling scheme for hexagonal torus
		topologies which enable large SpiNNaker systems to be assembled using only
		short cables between boards.
		
	\section{Conclusions}
		
		The SpiNNaker architecture has been designed to enable the simulation of
		large biologically realistic neural models in real time. To support this,
		its network architecture takes on an unconventional design based on a
		custom router and hexagonal torus topology. In the remainder of this
		thesis, I will tackle a number of the challenges in scaling up the
		SpiNNaker architecture outlined in this chapter.

	\chapter{Building large SpiNNaker machines}
	
	Like any super computer, physically putting together a large SpiNNaker
	machine poses many challenges in terms of organisation, assembly and
	maintainance. One of the key tasks in this process is the installation of
	network cables such that a desired overall network topology is constructed.
	The largest planned SpiNNaker machine will use \num{3600} S-ATA
	\cite{sata3spec} cables to interconnect its \num{1200} circuit boards,
	creating a hexagonal torus topology. Since the machine will be installed
	within standard server room cabinets (which are not available in a
	giant-doughnut form-factor) a mapping from a board's logical location in the
	network topology to its physical location must be constructed. In addition,
	the interconnect technology employed by SpiNNaker restricts the length of
	S-ATA cables used to $\le$~\SI{1}{\meter}, constraining the possible mappings
	used. In addition the practical issues of installation complexity and
	maintainance must be considered since all \num{3600} cables must ultimately
	be installed and maintained by human operators.
	
	In this chapter I describe a novel technique for physically laying out
	machines configured in hexagonal torus topologies, such as SpiNNaker, in
	commercial machine rooms, building on the techniques used in more
	conventional torus topologies. In addition, I also propose a new methodology
	for installing and maintaining super computer cabling which which exploits
	existing diagnostic features of the SpiNNaker hardware to interactively guide
	and validate cable installation. Finally, I demonstrate how these new
	techniques have been used successfully to interconnect a prototype
	\num{518400} core SpiNNaker machine in substantially less time than the
	industry norm.
	
	In this chapter, the term \emph{unit} refers to the smallest physical
	ecomponent between which connections connections are to be made. For example,
	in a SpiNNaker machine a unit is a 48-chip board while in data center, a unit
	might be a server blade.
	
	\section{Related work}
		
		In this section I describe the techniques conventionally employed when
		laying out and interconnecting the units within super computers. Due to
		SpiNNaker's hexagonal torus topology and dense physical packing of units,
		these existing techniques are found to be insufficient. In the remainder of
		the chapter we will explore solutions to the limitations exposed below.
		
		\subsection{Avoiding long cables}
			
			Na\"ive arrangements of torus topologies, including hexagonal torus
			topologies, feature long `wrap-around' connections which connect units at
			the peripheries of the system. These connections can be problematic for
			numerous reasons:
			
			\begin{description}
				
				\item[Performance] Signal quality diminishes as cables get longer,
				requiring the use of slower signalling speeds, increased error
				correction overhead or more complex hardware.
				
				\item[Energy] Longer cables require higher drive strengths and/or
				buffering to maintain signal integrity.
				
				\item[Cost] Cost Shorter cables are cheaper than long ones.  Longer
				cables imply more wire in a given space making the tasks of routing or
				cable installation more difficult increasing labour costs by as much as
				$5\times$ \cite{curtis12}.
				
			\end{description}
			
			In conventional torus topologies the need for long cables is eliminated
			by folding and interleaving units of the network \cite{dally04}. For
			example, for a 1D torus topology (a ring network), one long connection
			exists to connect the two opposite sides of the system. To remove these
			long connections, half the units are `folded' on top of the others and
			then this arrangement of units is interleaved as illustrated in figure
			\ref{fig:ring-folding}.
			
			\begin{figure}
				\center
				\begin{subfigure}[b]{0.39\linewidth}
					\center
					\buildfig{figures/ring-folding-row.tex}
					\caption{A ring network}
					\label{fig:ring-folding-row}
				\end{subfigure}
				\begin{subfigure}[b]{0.24\linewidth}
					\center
					\buildfig{figures/ring-folding-folded.tex}
					\caption{Folded}
					\label{fig:ring-folding-folded}
				\end{subfigure}
				\begin{subfigure}[b]{0.35\linewidth}
					\center
					\buildfig{figures/ring-folding-interleaved.tex}
					\caption{Folded and interleaved}
					\label{fig:ring-folding-interleaved}
				\end{subfigure}
				
				\caption{Folding and interleaving a ring network to reduce maximum wire
				length.}
				\label{fig:ring-folding}
			\end{figure}
			
			Folding and interleaving has the effect of approximately doubling the
			average cable length but also eliminates the need for a cable spanning
			the entire system. Since the mean cable length is typically already
			short, doubling it in exchange for a substantially reduced maximum cable
			length is often preferable.
			
			The folding and interleaving process may be extended to $N$-dimensional
			torus topologies by folding each dimension in turn. Since all dimensions
			are orthogonal, the folding process only moves units in the dimension
			being folded. In the hexagonal torus topology, however, the three
			dimensions are non-orthogonal and thus folding in one dimension also
			moves units in other dimensions, preventing the edges of the torus
			meeting as illustrated in figure \ref{fig:failing-to-fold-hex-toruses}.
			
			\begin{figure}
				\center
				\begin{subfigure}[b]{0.24\linewidth}
					\center
					\buildfig{figures/failing-to-fold-hex-toruses-none.tex}
					\caption{Not folded}
					\label{fig:failing-to-fold-hex-toruses-none}
				\end{subfigure}
				\begin{subfigure}[b]{0.24\linewidth}
					\center
					\buildfig{figures/failing-to-fold-hex-toruses-x.tex}
					\caption{X}
					\label{fig:failing-to-fold-hex-toruses-x}
				\end{subfigure}
				\begin{subfigure}[b]{0.24\linewidth}
					\center
					\buildfig{figures/failing-to-fold-hex-toruses-y.tex}
					\caption{Y}
					\label{fig:failing-to-fold-hex-toruses-y}
				\end{subfigure}
				\begin{subfigure}[b]{0.24\linewidth}
					\center
					\buildfig{figures/failing-to-fold-hex-toruses-z.tex}
					\caption{Z}
					\label{fig:failing-to-fold-hex-toruses-z}
				\end{subfigure}
				
				\caption{Schematics showing hexagonal torus topologies folded along
				each of their non-orthogonal dimensions. Note that folding along
				the Z axis brings the \emph{wrong} edges closer together.}
				\label{fig:failing-to-fold-hex-toruses}
			\end{figure}
		
		\subsection{Cabling installation}
			
			Existing machine room installations feature very repetitive cabling
			patterns which can easily be memorised by a human technician. For example
			in BlueGene super computers the connectivity between units is highly
			regular \cite{lakner07} while in data centre networks cabling often
			centres around a small number of high-port-count switches
			\cite{cisco07,csernai15}. Cable installation is usually only aided by
			the labelling of connectors and sockets in a standardised manner
			\cite{tia2006} such as in figure \ref{fig:bgWiring}.
			
			\begin{figure}
				\center
				\begin{subfigure}[t]{0.5\textwidth}
					\begin{tikzpicture}
						\node (cables) [inner sep=0]
						      {\includegraphics[width=\textwidth]{figures/bgCables.png}};
						\node (sockets) [inner sep=0, below=1.0em of cables]
						      {\includegraphics[width=\textwidth]{figures/bgSockets.png}};
						
						% Point at label on cable
						\draw [white, <-, line width=0.4em]
						      ([shift={(0.7cm, -0.3cm)}]cables.center)
						      -- ++(45:1cm);
						
						% Point at label on socket
						\draw [white, <-, line width=0.4em]
						      ([shift={(-1.0cm, 1.1cm)}]sockets.center)
						      -- ++(-45:1cm);
					\end{tikzpicture}
					
					\caption{Pre-labelled cables and sockets}
					\label{fig:bgWiringLabels}
				\end{subfigure}
				~
				\begin{subfigure}[t]{0.30\textwidth}
					\includegraphics[height=6.15cm]{figures/bgWiring.jpg}
					
					\caption{Installation of cables}
					\label{fig:bgWiringInstallation}
				\end{subfigure}
				
				\caption{BlueGene/Q cable installation \cite{cscs13}}
				\label{fig:bgWiring}
			\end{figure}
			
			Despite the regularity and careful labelling of cables, the cost of
			installation and maintenance alone can be significant with costs in the
			range of \$45-95 per \SI{1}{\meter} cable run and \$100-400 for runs of
			\SI{10}{\meter} reported in the literature \cite{mudigonda11}. Much of
			this cost is due to the care required during installation to avoid
			miswiring and ensure that cooling airflow is not hampered by cable runs
			\cite{cisco07}.
			
			Many researchers have attempted to control cable installation costs by
			trying to reduce the number or length of cables required by developing
			alternative network topologies \cite{curtis12, popa10, mudigonda11}.
			Unfortunately, these techniques do not apply to SpiNNaker since its
			network topology is fixed.
			
			Some super computers make use of large custom `midplane` PCBs in place of
			cables to interconnect units within a cabinet and thus simplify the task
			of cable installation \cite{prickett10}. This scheme can greatly reduce
			wiring complexity since only coarser-grain cabinet-to-cabinet
			connectivity is provided by cables. Unfortunately this technique is
			expensive and also constrains the dimensions of the network topology
			supported by the machine. Since the SpiNNaker platform is designed to
			scale from desktop machines to machine-room installations, this scheme is
			not practical.
	
	\section{Folding \& interleaving hexagonal toruses}
		
		The first step towards a practical machine-room installation of a large
		machine using a hexagonal torus topology is to find an arrangement of
		boards between which cable lengths are minimised. In this section I
		describe a sequence of transformations which map the positions of units in
		a hexagonal torus topology onto a regular rectangular grid which may be
		folded and interleaved to eliminate long wires. It is worth emphasising
		that this transformation only affects the \emph{physical} positions of
		units and \emph{not} their connectivity.
		
		As described earlier in \S\ref{sec:parititioning} (page
		\pageref{sec:parititioning}), hexagonal torus topologies may be partitioned
		into units containing wrapped-triples of nodes. For example, in SpiNNaker,
		chips (nodes) are partitioned into circuit boards (units) containing 48
		chips. For completeness, this section describes the process of folding both
		systems whose units are individual nodes and those whose units are
		wrapped-triples.
		
		The transformation process is divided into two parts, each described
		separately in this section.
		
		\begin{description}
			
			\item[Parallelogram to rectangle] Units of the system are transformed
			from a parallelogram shape to a rectangular shape.
			
			\item[Uncrinkle] Units within the rectangle are moved such that they all
			lie on a regular (and fully packed) 2D grid.
			
		\end{description}
		
		\subsection{Parallelogram to rectangle}
			
			The hexagonal torus topology is most naturally drawn as a parallelogram
			as illustrated in figures \ref{fig:hex-to-plane-node-native} and
			\ref{fig:hex-to-plane-native}. Two transformations are presented which
			transform these arangements of units into a rectangular form: shearing
			and slicing.
			
			A \SI{30}{\degree} shear transformation distorts networks such that the X
			and Y axes become orthogonal leading to a rectangular arrangement of
			units as illustrated in figures \ref{fig:hex-to-plane-node-shear} and
			\ref{fig:hex-to-plane-shear}.
			
			The slice transformation slices the units protruding from the
			left-hand-side of the parallelogram and moves them into the matching gap
			on the opposite side of the parallelogram as illustrated in figures
			\ref{fig:hex-to-plane-node-slice} and \ref{fig:hex-to-plane-slice}.
			 
			While the shear transformation introduces some distortion causing cables
			in the Z dimension to become $\sqrt{2}\times$ longer it leaves the
			pattern of wrap-around connections remains unchanged. By contrast, the
			slice transformation does not elongate any cables but changes the pattern
			of wrap-around connections. The exact pattern wrap-around connections
			produced when slicing depends on the aspect ratio of the network as
			illustrated in \ref{fig:slicing-examples} and influences the choice of
			folding technique applied as described later.
			
			\begin{figure}
				\center
				\begin{subfigure}[b]{0.32\linewidth}
					\center
					\buildfig{figures/hex-to-plane-node-native.tex}
					
					\caption{$7 \times 7$ node torus}
					\label{fig:hex-to-plane-node-native}
				\end{subfigure}
				\begin{subfigure}[b]{0.32\linewidth}
					\center
					\buildfig{figures/hex-to-plane-node-shear.tex}
					
					\caption{Sheared}
					\label{fig:hex-to-plane-node-shear}
				\end{subfigure}
				\begin{subfigure}[b]{0.32\linewidth}
					\center
					\buildfig{figures/hex-to-plane-node-slice.tex}
					
					\caption{Sliced}
					\label{fig:hex-to-plane-node-slice}
				\end{subfigure}
				
				\caption{Transformations of hexagonal toruses of nodes into a
				rectangular form. Thin lines show wrap-around links. Pointy-topped
				hexagons represent individual nodes.}
				\label{fig:hex-to-plane-node}
			\end{figure}
			
			\begin{figure}
				
				\begin{subfigure}[b]{0.32\linewidth}
					\center
					\buildfig{figures/hex-to-plane-native.tex}
					
					\caption{$4 \times 4$ triad torus}
					\label{fig:hex-to-plane-native}
				\end{subfigure}
				\begin{subfigure}[b]{0.32\linewidth}
					\center
					\buildfig{figures/hex-to-plane-shear.tex}
					
					\caption{Sheared}
					\label{fig:hex-to-plane-shear}
				\end{subfigure}
				\begin{subfigure}[b]{0.32\linewidth}
					\center
					\buildfig{figures/hex-to-plane-slice.tex}
					
					\caption{Sliced}
					\label{fig:hex-to-plane-slice}
				\end{subfigure}
				
				\caption{Transformations of hexagonal toruses of wrapped triples into a
				rectangular form.  Thin lines show wrap-around links. Flat-topped
				hexagons represent a wrapped triple of nodes.}
				\label{fig:hex-to-plane}
			\end{figure}
			
			\begin{figure}
				\center
				\buildfig{figures/slicing-examples.tex}
				\caption{Patterns of wiring in sliced systems of various sizes.}
				\label{fig:slicing-examples}
			\end{figure}
			
		\subsection{Uncrinkling}
			
			Though the transformmation step yields rectangular arrangements of units,
			these arrangements do not fall onto a regular 2D grid, with the exception
			of the shear transform on individual nodes. Figure \ref{fig:uncrinkling}
			illustrates how the various arrangements of hexagons may be moved to
			`uncrinkle' the units into a regular grid.
			
			\begin{figure}
				\center
				\begin{subfigure}[b]{0.44\linewidth}
					\center
					\buildfig{figures/uncrinkling-node-sheared.tex}
					
					\caption{$7 \times 7$ nodes, sheared}
					\label{fig:uncrinkling-node-sheared}
				\end{subfigure}
				\begin{subfigure}[b]{0.44\linewidth}
					\center
					\buildfig{figures/uncrinkling-node-sliced.tex}
					
					\caption{$7 \times 7$ nodes, sliced}
					\label{fig:uncrinkling-node-sliced}
				\end{subfigure}
				
				\vspace{1cm}
				
				\begin{subfigure}[b]{0.44\linewidth}
					\center
					\buildfig{figures/uncrinkling-sheared.tex}
					
					\caption{$4 \times 4$ triples, sheared}
					\label{fig:uncrinkling-sheared}
				\end{subfigure}
				\begin{subfigure}[b]{0.44\linewidth}
					\center
					\buildfig{figures/uncrinkling-sliced.tex}
					
					\caption{$4 \times 4$ triples, sliced}
					\label{fig:uncrinkling-sliced}
				\end{subfigure}
				
				\vspace{1em}
				
				\caption{Mapping rectangular arrangements of units into a square grid.
				Thick lines show how layers of units are uncrinkled.  Annotations show
				how the relative positions of nodes and wrapped triples change after
				uncrinkling.}
				\label{fig:uncrinkling}
			\end{figure}
			
			In the figure, the numbered units enumerate the different positions on
			the crinkle and those labelled alphabetically are those that immediately
			surround them. From this we can observe that uncrinkling largely
			preserves spatial locality but some distortion is introduced, separating
			previously neighbouring units. For example, in figure
			\ref{fig:uncrinkling-sheared}, the units labelled `1' and `i' are
			neighbours before uncrinkling but are separated by a (Euclidean) distance
			of $\sqrt{5}$ afterwards. Note that the distortion introduced depends on
			what part of the crinkle is considered, for example `2' and `a' have
			distance 2 but are logically connected in the same way.
		
		\subsection{Folding and Interleaving}
			
			Once a regular grid of units has been formed, this may be folded in the
			conventional way, eliminating long cables crossing from left-to-right and
			top-to-bottom as illustrated in \ref{fig:folding-sheared}.
			
			Unfortunately, for sliced systems whose dimensions are not of the ratio
			$1:2$, the pattern of wrap-around cables may also include some cables
			which do not cross directly to the opposite side of the system (refer
			back to figure \ref{fig:slicing-examples}). As a result of these
			connections, folding does not successfully eliminate all long
			connections. An exception to this rule is sliced systems whose dimensions
			are in the ratio $1:1$ where folding twice along the Y axis may
			successfully eliminate all wrap-around connections as illustrated in
			\ref{fig:folding-sliced}.
			
			\begin{figure}
				\begin{subfigure}{\linewidth}
					\center
					\buildfig{figures/folding-sheared.tex}
					\caption{$N \times M$ sheared systems and $N \times 2N$ sliced systems}
					\label{fig:folding-sheared}
				\end{subfigure}
				
				\vspace{1em}
				
				\begin{subfigure}{\linewidth}
					\center
					\buildfig{figures/folding-sliced.tex}
					\caption{$N \times N$ sliced systems}
					\label{fig:folding-sliced}
				\end{subfigure}
				
				\caption{Schematic illustrating elimination of long wrap-around links
				during folding. In each example a single link has been highlighted to
				aid in following the process.}
				\label{fig:folding}
			\end{figure}
			
			Once folded, the 2D grid is straight-forwardly interleaved as illustrated
			previously in figure \ref{fig:ring-folding}. The interleaving process
			introduces some additional distortion to the layout of units and causes
			most connections to become twice as long. For sliced $1:1$ systems, the
			additional fold results in additional overhead during interleaving since
			four layers of the system are interleaved.
		
		\subsection{Mapping to Cabinets}
			
			In the final step of the process is to map the 2D grid of units into
			positions in machine room cabinets as illustrated in figure
			\ref{fig:million-core-machine}. As illustrated in figure
			\ref{fig:cabinetisation}, first the grid of units is partitioned into
			groups of columns, one per cabinet, then groups of rows one per frame per
			cabinet. The units in each group are then allocated to slots within a
			frame, interleaving the rows of the groups as shown.
			
			\begin{figure}
				\center
				\buildfig{figures/cabinet-units.tex}
				
				\caption{An illustration of the physical construction of a
				multi-cabinet SpiNNaker system. (Note: network cables \emph{not}
				installed.)}
				\label{fig:cabinet-units}
			\end{figure}
			
			\begin{figure}
				\center
				\buildfig{figures/cabinetisation.tex}
				
				\caption{Mapping from 2D space to cabinets, frames and boards.}
				\label{fig:cabinetisation}
			\end{figure}
		
	\section{Cable installation}
		
		Cable installation is performed by a team of (human) technicians who must
		ensure that all network cables are correctly installed. As illustrated in
		previously in figure \ref{fig:cabinet-units}, the density of SpiNNaker's
		units, combined with the nature of the hexagonal torus topology, poses a
		challenge. To address this challenge I propose a semi-automated approach to
		cable installation which exploits diagnostic facilities available in the
		majority of super computers in order to guide technicians through the
		cabling process, interactively guiding installation and maintenance.
		
		\subsection{Interactive technician guidance and validation}
			
			While automated systems for validating cabling correctness are
			commonplace, these systems are typically used only after cabling has been
			completed \cite{lakner07}. As with other large-scale machines, SpiNNaker
			includes a low-bandwidth system management bus which may be used to
			interrogate network hardware and control diagnostic LEDs prior to the
			installation of the main SpiNNaker network interconnect.  Using these
			facilities I have constructed a tool called SpiNNer which interactively
			guides a technician, or team of technicians, through the cable
			installation process, validating each connection in real-time.
			
			Diagnostic LEDs mounted on each SpiNNaker board (figure
			\ref{fig:interactive-wiring-guide-leds}) are used to indicate the
			endpoints of the cable currently being installed. Simultaneously a
			Text-To-Speech (TTS) system gives an audible indication of which cable
			type is to be used and location of each connection.  Additionally, a GUI
			via a computer display (figure \ref{fig:interactive-wiring-guide-gui}).
			The centre of the display shows a `big-picture' perspective of the
			locations of the boards to be connected. The detailed views on the left
			and right indicate which of the six sockets on each board the cables
			should connect.
			
			\begin{figure}
				\center
				\begin{subfigure}[b]{0.40\textwidth}
					\begin{tikzpicture}
						\node (leds) [inner sep=0]
						      {\includegraphics[width=\textwidth]{figures/leds.jpg}};
						% Point at left LED
						\draw [white, <-, line width=0.4em]
						      ([shift={(-0.0cm, -0.6cm)}]leds.center)
						      -- ++(225:1cm);
						% Point at right LED
						\draw [white, <-, line width=0.4em]
						      ([shift={(1.1cm, -1.1cm)}]leds.center)
						      -- ++(225:1cm);
					\end{tikzpicture}
					
					\caption{Diagnostic LEDs}
					\label{fig:interactive-wiring-guide-leds}
				\end{subfigure}
				~
				\begin{subfigure}[b]{0.546\textwidth}
					\begin{tikzpicture}[thin, black!20!white]
						\node (screen) [inner sep=0]
						      {\includegraphics[width=\textwidth]{figures/wiring_guide_screenshot.png}};
						\draw (screen.south west) rectangle (screen.north east);
					\end{tikzpicture}
					
					\caption{Interactive wiring guide GUI}
					\label{fig:interactive-wiring-guide-gui}
				\end{subfigure}
				
				\caption{The SpiNNer interactive wiring guide uses a GUI,
				text-to-speech and diagnostic LEDs to assist during cable
				installation.}
				\label{fig:interactive-wiring-guide}
			\end{figure}
			
			SpiNNer also validates the connectivity of the system in real-time by
			polling the diagnostic interfaces of the network hardware at the
			endpoints of the cable being installed to determine if they are correctly
			connected. If a miswiring occurs, this is immediately detected and
			announced via TTS enabling the technician to immediately correct the
			error. Once a cable has been installed correctly, the software
			automatically advances to the next cable meaning direct interaction with
			the software by the technician is minimal. In practice, it is rarely
			necessary to refer to the GUI.
		
			SpiNNer presents the cables in an order intended to maximise ease of
			installation. Cables are installed in three groups with intra-frame
			cables being installed first, followed by intra-cabinet cables and
			inter-cabinet cables. Within each group, the tightest cables are
			installed first resulting in slacker cables naturally being installed
			over the top of already installed cables. By grouping cables in this
			manner, multiple technicians may work independently on the wiring within
			individual frames and cabinets.
			
			SpiNNer may also be used to repair or replace cables in the system.
			During maintenance, obstructing cables may be blindly removed alongside
			any cable being replaced. At the conclusion of the process, the wiring
			guide may be used to interactively guide re-installation of all removed
			cables.
		
		\subsection{Cable selection}
			
			Controlling slack is critical to ensuring reliable and maintainable
			cabling installations. If cables are too tight, cables and connectors can
			become easily damaged and when too slack, the excess cable obstructs
			other cables and can easily become tangled and damaged \cite{cisco07}. It
			has been observed that when ready-made cables are employed technicians
			frequently over-estimate the cable lengths required preferring to use
			overly long cables for all connections \cite{mazaris97}. To solve this
			problem, the SpiNNer wiring guide software dictates the cable lengths to
			be used by an installer based the rule of (three-)thumbs according to
			Mazaris \cite{mazaris97}. This rule suggests that an ideal amount of
			slack is approximately that which can be wrapped around three fingers.
			Specifically, the shortest available cable is selected which ensures at
			least \SI{5}{\centi\meter} of slack.
			
			The SpiNNer tool allocates cables assuming all cables take a Euclidean
			straight-line path between the endpoints of the connection. The result is
			that wiring is not routed through dedicated cable management structures
			but is simply suspended by its connectors in front of the machine. As
			demonstrated later, this unconventional approach leads neither to cooling
			problems nor increased maintenance effort.
	
	\section{Results and Evaluation}
		
		This stuff has been used and works. In this section I'll go over the
		overheads introduced by the various transformations and
		folding/interleaving steps and show a wiring scheme for a large machine
		which uses only short cables. I'll then show how SpiNNer was used to
		install this wiring plan into a very large machine without foobaring the
		cooling and in very little time. I'll also report on difficulty of
		maintenance.
		
		\subsection{Cable length}
			
			The transformation from regular hexagonal torus to a folded and
			interleaved form introduces some overhead to the cable lengths required.
			Using figure \ref{fig:uncrinkling} (page \pageref{fig:uncrinkling}), it
			is possible to compute the exact overhead introduced when each type of
			transformation proposed.
			
			For example, to compute the mean overhead introduced by the slicing
			technique when applied to units composed of wrapped triples, consider
			figure \ref{fig:uncrinkling-sliced}. The uncrinkling pattern used to
			transform this topology is a repeating pattern of two units, a pair of
			which have been labelled $1$ and $2$ respectively. Unit $1$ is
			immediately surrounded by six units labelled $a$, $b$, $c$, $2$, $g$ and
			$h$. Similarly, unit $2$ is surrounded by units $1$, $c$, $d$, $e$, $f$
			and $g$. Before the transformation, the distances, $D$, to each of these
			units is $1$ but after the transformation is applied, this is not always
			the case. Additionally, folding and interleaving introduce additional
			overhead. In this example, if the system is folded into $f_x$ columns and
			$f_y$ rows, the distances between previously neighbouring units become:
			
			\begin{equation*}
				\begin{aligned}[c]
					D_{1\,\leftrightarrow{}\,a} &= \sqrt{f_x^2 + f_y^2} \\
					D_{1\,\leftrightarrow{}\,b} &= f_y \\
					D_{1\,\leftrightarrow{}\,c} &= \sqrt{f_x^2 + f_y^2} \\
					D_{1\,\leftrightarrow{}\,2} &= f_x \\
					D_{1\,\leftrightarrow{}\,g} &= f_y \\
					D_{1\,\leftrightarrow{}\,h} &= f_x
				\end{aligned}
				\hspace{2cm}
				\begin{aligned}[c]
					D_{2\,\leftrightarrow{}\,1} &= f_x \\
					D_{2\,\leftrightarrow{}\,c} &= f_y \\
					D_{2\,\leftrightarrow{}\,d} &= f_x \\
					D_{2\,\leftrightarrow{}\,e} &= \sqrt{f_x^2 + f_y^2} \\
					D_{2\,\leftrightarrow{}\,f} &= f_y \\
					D_{2\,\leftrightarrow{}\,g} &= \sqrt{f_x^2 + f_y^2}
				\end{aligned}
			\end{equation*}
			
			From these values, the mean and maximum connection distances after
			folding and interleaving may be computed. Table
			\ref{tab:transform-overhead} gives the mean and maximum connection
			distances for each of the four transformations described in this chapter.
			
			\begin{table}
				\begin{subtable}[b]{\linewidth}
					\center
					\begin{tabular}{l c c}
						\toprule
						& Shear & Slice \\
						\addlinespace
						Nodes &
							$\frac{f_x + f_y + \sqrt{f_x^2 + f_y^2}}{3}$ &
							$\frac{f_x + f_y + \sqrt{f_x^2 + f_y^2}}{3}$ \\
						\addlinespace
						Triples &
							$\frac{7f_x + 3\sqrt{f_x^2 + f_y^2} + \sqrt{(2f_x)^2 + f_y^2}}{9}$ &
							$\frac{f_x + f_y + \sqrt{f_x^2 + f_y^2}}{3}$ \\
						\bottomrule
					\end{tabular}
					
					\caption{Mean}
					\label{tab:transform-overhead-mean}
				\end{subtable}
				
				\vspace{1em}
				
				\begin{subtable}[b]{\linewidth}
					\center
					\begin{tabular}{l c c}
						\toprule
						& Shear & Slice \\
						\addlinespace
						Nodes &
							$\sqrt{f_x^2 + f_y^2}$ &
							$\sqrt{f_x^2 + f_y^2}$ \\
						\addlinespace
						Triples &
							$\sqrt{(2f_x)^2 + f_y^2}$ &
							$\sqrt{f_x^2 + f_y^2}$ \\
						\bottomrule
					\end{tabular}
					
					\caption{Maximum}
					\label{tab:transform-overhead-max}
				\end{subtable}
				
				\caption{Overheads introduced when transforming unit positions onto a
				regular grid.}
				\label{tab:transform-overhead}
			\end{table}
			
			From these results it is evident that shearing and slicing networks
			whose units are nodes result in identical mean and maximum overhead in
			cable length when folded similarly. Since sliced networks may require
			folding more than once along each axis the shearing approach is
			preferable in general.
			
			For networks constructed from units of wrapped triples, the slicing
			approach suffers the same mean and maximum overhead has networks of
			nodes, and less overhead than shearing for the same number of folds. For
			systems with an aspect ratio of $1:2$ (where both slicing and shearing
			require $f_x = f_y = 2$), the slicing transformation yields lower mean
			and maximum overhead than shearing. For all other aspect ratios (where
			slicing requires a greater degree of folding) the shearing technique
			produces lower overhead. The recommended transformations for a given
			machine are thus given in table \ref{tab:transform-recommended}.
			
			\begin{table}
				\center
				\begin{tabular}{lcc}
					\toprule
					                         & $1:2$  & Other \\
					\addlinespace
					\multirow{2}{*}{Nodes}   & Either & Shear\\
					                         & \footnotesize $\mu\approx2.28 \quad \vee\approx2.83$
					                         & \footnotesize $\mu\approx2.28 \quad \vee\approx2.83$\\
					\addlinespace
					\multirow{2}{*}{Triples} & Slice  & Shear\\
					                         & \footnotesize $\mu\approx2.28 \quad \vee\approx2.83$
					                         & \footnotesize $\mu\approx3.00 \quad \vee\approx4.47$\\
					\bottomrule
				\end{tabular}
				
				\caption{Recommended transformation and folding scheme for different
				system types. $\mu$ and $\vee$ give the mean and maximum wire
				distortion introduced, respectively.}
				\label{tab:transform-recommended}
			\end{table}
			
			\begin{figure}
				\center
				\buildfig{figures/million-core-machine.tex}
				
				\caption{Cabling plan for a \num{1036800} core SpiNNaker
				machine's \num{3600} cables.}
				\label{fig:million-core-machine}
			\end{figure}
			
			Following folding and mapping to physical locations, the cabling plans
			for large machines require no large gaps to be spanned.  The largest
			planned SpiNNaker machine, illustrated in figure
			\ref{fig:million-core-machine}, will be \SI{6}{\meter} wide but the
			largest gap any cable must span is \SI{66}{\centi\meter}. This distance
			is well within the \SI{1}{\meter} allowed by the hardware and cables.
			
		\subsection{Installation practicality}
			
			\begin{table}
				\center
				\begin{tabular}{lrr@{$\,$}l}
					\toprule
						System & Number of Cables & \multicolumn{2}{r}{Installation time} \\
					\midrule
						24 boards  & \num{72}   & \num{10} & \si{\minute}         \\
						1 cabinet  & \num{360}  & \num{4}  & \si{\hour}$^\dagger$ \\
						2 cabinets & \num{720}  & \num{2}  & \si{\hour}           \\
						5 cabinets & \num{1800} & ?        &                      \\
					\bottomrule
				\end{tabular}
				
				\caption{Installation times for various sizes of machine.
				$\dagger$~This machine was installed without real-time validation of
				connectivity.}
				\label{tab:install-time}
			\end{table}
			
			A number of SpiNNaker machines of various scales have been assembled
			using the techniques described in this chapter ranging from single frames
			of 24 boards to a half-scale 5 cabinet machine. Table
			\ref{tab:install-time} gives the reported installation times of each of
			these machines.
			
			The single cabinet machine's installation time is notably
			disproportionate to its size. When this system was assembled, real-time
			connection validation was not yet available. As a result, though cable
			installation was rapid correcting errors was extremely costly, requiring
			careful retracing of many installation steps.
			
			TODO: TALK ABOUT MULTI-PERSON-WIRING IN PRACTICE ON FIVE CABINET MACHINE.
			
			\begin{figure}
				
				\center
				\buildfig{figures/wire-length-histogram.tex}
				
				\caption{Histogram of connection distances in a ten-cabinet,
				one-million core SpiNNaker machine annotated with the suggested cable
				length.}
				\label{fig:wire-length-histogram}
				
			\end{figure}
			
			FIGURE \ref{fig:wire-length-histogram} SHOWS THE DISTRIBUTION OF CABLE
			LENGTHS REQUIRED. IN PRACTICE THE SLACK ALLOCATED PROVED ADEQUATE. AS
			SHOWN IN FIGURE \ref{fig:install-histogram}, THE MOST IMPORTANT FACTOR IS
			WHETHER LEAVING THE FRAME OR NOT. LEAVING THE FRAME TAKES THE LONGEST.
			
			\begin{figure}
				\builddata{data/build_connection_log.tex}
				\buildfig{figures/install-histogram.tex}
				
				\caption{Histogram of cable installation times}
				\label{fig:install-histogram}
			\end{figure}
			
			TODO: COMPARE DIRECTLY WITH INSTALL TIMES REPORTED IN LITERATURE.
		
		\subsection{Thermal Impact}
			
			TODO: SHOW HOW TEMPERATURE IS CHANGED
			
		\subsection{Maintenance}
			
			TOOD: QUANTIFY CABLE REMOVALS REQUIRED. EXPERIMENT: REMOVE/REPLACE RANDOM
			BOARDS AND MEASURE TIME TAKEN, CABLES REMOVED. COMPARE WITH STANDARD DATA
			CENTRE WIRING

	\chapter{Finding shortest path vectors in SpiNNaker's network}
	
	Once a SpiNNaker machine has been constructed as described in the previous
	chapter, its network forms a large hexagonal torus topology. To exploit this
	network routing algorithms must be used to generate routes for packets to
	follow between nodes. As well as ensuring that packets arrive at the correct
	destination, routing algorithms often attempt to produce routes which make
	efficient use of the network. This often involves attempting to reduce
	congestion by ensuring packets do not travel further through the network than
	absolutely necessary.
	
	Many popular routing algorithms for torus topologies, including all published
	algorithms designed for SpiNNaker's hexagonal torus topology
	\cite{davies12,navaridas14}, internally function by computing shortest path
	vectors and generating routes from them. Existing methods of calculating
	shortest path vectors in hexagonal torus topologies are unable to generate
	all possible shortest path vectors and, as a result, reduces the diversity of
	routes produced by routing algorithms, potentially worsening network
	contention.
	
	In this chapter I describe a novel technique for computing shortest path
	vectors in hexagonal torus topologies which yields \emph{all} possible
	shortest path vectors for any pair of nodes. Further, implementations of this
	new technique execute an order of magnitude faster than the existing
	alternatives.
	
	\section{Related work}
		
		TODO: INTRODUCE SECTION
		
		\begin{figure}
			\center
			
			\begin{subfigure}{\linewidth}
				\center
				\buildfig{figures/distance-map-mesh.tex}
				\caption{2D mesh topology}
				\label{fig:distance-map-mesh}
			\end{subfigure}
			
			\vspace{1em}
			
			\begin{subfigure}{\linewidth}
				\center
				\buildfig{figures/distance-map-torus.tex}
				\caption{2D torus topology}
				\label{fig:distance-map-torus}
			\end{subfigure}
			
			\vspace{1em}
			
			\begin{subfigure}{\linewidth}
				\center
				\buildfig{figures/distance-map-hex-mesh.tex}
				\caption{Hexagonal mesh topology}
				\label{fig:distance-map-hex-mesh}
			\end{subfigure}
			
			\vspace{1em}
			
			\begin{subfigure}{\linewidth}
				\center
				\buildfig{figures/distance-map-hex-torus.tex}
				\caption{Hexagonal torus topology}
				\label{fig:distance-map-hex-torus}
			\end{subfigure}
			
			\caption{Plots showing distance from various locations marked
			         {\color{red}$\times$}. Darker areas are further away. Contour
			         lines show equidistant points.}
			\label{fig:distance-map}
		\end{figure}
		
		\subsection{Mesh Networks}
			
			In a (non-hexagonal) mesh network topology, shortest path vectors are
			computed by taking the element-wise difference between the source and
			destination nodes' coordinates.
			
			\begin{figure}
				\center
				\buildfig{figures/mesh-topology-coordinates.tex}
				\caption{An example 2D mesh network with example shortest-path routes
				from `A' to `B' and `B' to `C'.}
				\label{fig:mesh-topology-coordinates}
			\end{figure}
			
			For example, figure \ref{fig:mesh-topology-coordinates} illustrates a 2D
			mesh topology. In this topology, the nodes labelled `A', `B' and `C' have
			position vectors $(1, 2)$, $(4, 5)$ and $(6, 1)$ respectively. The
			shortest path vector from node `A' to `B' is thus simply $(4, 5) - (1, 2)
			= (3, 3)$ and from `B' to `C' is $(6, 1) - (4, 5) = (2, -4)$.
			
			A route may be produced from a shortest path vector by advancing the
			number of hops specified for each dimension in the vector. For example
			any permutation of the hops X$^+\,$X$^+\,$X$^+\,$Y$^+\,$Y$^+\,$Y$^+$, an
			example of which is included in the figure. Likewise a route from `B' to
			`C' may be constructed from any permutation of
			X$^+\,$X$^+\,$Y$^-\,$Y$^-\,$Y$^-\,$Y$^-$.
			
			Many popular routing algorithms such as Dimension Order Routing (DOR),
			Right-Turn Only Routing (RTOR) and Longest Dimension First Routing (LDFR)
			\cite{dally04,davies12} directly follow the above procedure and just
			prescribe a specific permutation of hop order. For example, DOR produces
			routes with X hops first, Y hops second and so on.
			
			The length of routes produced from a shortest path vector have a number
			of hops proportional to the magnitude of the vector, thus a shortest path
			vector yields a route with the minimum number of hops. For a two
			dimensional vector $(a, b)$ the magnitude is given as:
			%
			\begin{equation}
				\| (a, b) \| = \lvert a \rvert + \lvert b \rvert
			\end{equation}
		
		\subsection{Torus Networks}
			
			Computing shortest path vectors in (non-hexagonal) torus topologies is
			also straight forward. As an example, lets find the shortest path vector
			from node `A' to `B' in the 2D torus topology shown in figure
			\ref{fig:torus-shortest-path-example}. First, both nodes are translated
			such that the source node, `A', is at the centre of the network (figure
			\ref{fig:torus-shortest-path-translate}). Note that this translation may
			result in the destination node `wrapping around' the network. After
			translation, the shortest path vector is computed as in a mesh topology.
			As illustrated in \ref{fig:torus-shortest-path-routed}, the computed
			shortest path vector may be used to produce routes between the two nodes
			in their original positions.
			
			\begin{figure}
				\center
				\begin{subfigure}{0.3\linewidth}
					\center
					\buildfig{figures/torus-shortest-path-example.tex}
					\caption{Original}
					\label{fig:torus-shortest-path-example}
				\end{subfigure}
				\begin{subfigure}{0.3\linewidth}
					\center
					\buildfig{figures/torus-shortest-path-translate.tex}
					\caption{Translated}
					\label{fig:torus-shortest-path-translate}
				\end{subfigure}
				\begin{subfigure}{0.3\linewidth}
					\center
					\buildfig{figures/torus-shortest-path-routed.tex}
					\caption{Routed}
					\label{fig:torus-shortest-path-routed}
				\end{subfigure}
				
				\caption{Finding shortest paths in a 2D torus topology.}
				\label{fig:torus-shortest-path}
			\end{figure}
			
			This process works because vectors from the centre (though not other
			locations) of a torus topology are identical to those in mesh topologies
			(see figures \ref{fig:distance-map-mesh} and
			\ref{fig:distance-map-torus}).
		
		\subsection{Hexagonal Mesh Networks}
			
			In hexagonal mesh topologies it is conventional to define three `axes' X,
			Y and Z as shown in figure \ref{fig:hex-mesh-topology-coordinates}
			\cite{patel15}. In this example, the three labelled nodes `A', `B' and
			`C' may be given position vectors such as $(1, 1, 0)$, $(3, 2, 0)$ and
			$(0, 0, -7)$ respectively. As in other mesh networks, a vector between
			two nodes is found by subtracting the nodes' vectors. For example, a
			vector from `A' to `B' is $(3, 2, 0) - (1, 1, 0) = (2, 1, 0)$. This
			vector can then be converted into a route in the same way as a mesh
			network by taking any permutation of the three hops  X$^+\,$X$^+\,$Y$^+$.
			
			\begin{figure}
				\center
				\buildfig{figures/hex-mesh-topology-coordinates.tex}
				\caption{An example hexagonal mesh network topology.}
				\label{fig:hex-mesh-topology-coordinates}
			\end{figure}
			
			As explained in detail in appendix \ref{app:minimal-hex-coordinates},
			there are an infinite number of vectors between any two points. For
			example, the vectors $(1, 0, -1)$ and $(3, 2, 1)$ also reach node `B'
			from `A' in the example. However, for a given pair of nodes, there is
			always a single, unique vector whose magnitude is minimal which is
			given by the function:
			%
			\begin{equation}
				\operatorname{minimiseVector}(x,y,z)
					= (x,y,z) - \operatorname{median}(x,y,z) \cdot (1,1,1)
			\end{equation}
			%
			An important side-effect of this function is that a minimised vector will
			always contain at least one zero element meaning that shortest path
			routes will use at most two of the three available dimensions.
			
			To aid the reader's intuition, figure \ref{fig:distance-map-hex-mesh}
			illustrates how distances grow in a hexagonal mesh topology.
		
		\subsection{Hexagonal Torus Networks}
			
			Unfortunately, unlike non-hexagonal torus topologies, the translation
			technique cannot be used to compute shortest path vectors. As illustrated
			in figures \ref{fig:distance-map-hex-mesh} and
			\ref{fig:distance-map-hex-torus}, shortest path vectors from the center
			of a hexagonal mesh network are not equivalent to those of a hexagonal
			torus network.
			
			Prior research into routing in SpiNNaker's network has been based on the
			INSEE \cite{navaridas09,ghasempour15} interconnect simulator. Internally
			INSEE tries a set of twelve candidate vectors and picks the shortest as
			the shortest path vector to use for routing.
			
			\begin{figure}
				\center
				\begin{subfigure}{0.45\linewidth}
					\center
					\buildfig{figures/insee-vector-candidates-no-wrap.tex}
					\caption{$(\Delta_\textrm{X}, \Delta_\textrm{Y}) = (5,3)$}
					\label{fig:insee-vector-candidates-no-wrap}
				\end{subfigure}
				\begin{subfigure}{0.45\linewidth}
					\center
					\buildfig{figures/insee-vector-candidates-wrap-x.tex}
					\caption{$(\Delta'_\textrm{X}, \Delta_\textrm{Y}) = (-3,3)$}
					\label{fig:insee-vector-candidates-wrap-x}
				\end{subfigure}
				
				\vspace{1em}
				
				\begin{subfigure}{0.45\linewidth}
					\center
					\buildfig{figures/insee-vector-candidates-wrap-y.tex}
					\caption{$(\Delta_\textrm{X}, \Delta'_\textrm{Y}) = (5,-5)$}
					\label{fig:insee-vector-candidates-wrap-y}
				\end{subfigure}
				\begin{subfigure}{0.45\linewidth}
					\center
					\buildfig{figures/insee-vector-candidates-wrap.tex}
					\caption{$(\Delta'_\textrm{X}, \Delta'_\textrm{Y}) = (-3,-5)$}
					\label{fig:insee-vector-candidates-wrap}
				\end{subfigure}
				
				\vspace{1em}
				
				% Key
				\begin{tikzpicture}[thick]
					\coordinate (last);
					
					% #1 colour
					% #2 label
					\newcommand{\colourkeyentry}[2]{
						\node [#1] [right=of last, fill, rectangle, minimum size=1em] (last) {};
						\node [right=0 of last] (last) {#2};
					}
					
					\colourkeyentry{cb3class0}{$(\textrm{X}, \textrm{Y}, 0)$}
					\colourkeyentry{cb3class1}{$(\textrm{X} - \textrm{Y}, 0, - \textrm{Y})$}
					\colourkeyentry{cb3class2}{$(0, \textrm{Y} - \textrm{X}, - \textrm{X})$}
					
				\end{tikzpicture}
				
				\caption{The twelve candidate shortest-path vectors considered by INSEE
				represented as dimension-order routes. $W=H=8$,
				$(\Delta_\textrm{X},\Delta_\textrm{Y}) = (5, 3)$ and
				$(\Delta'_\textrm{X},\Delta'_\textrm{Y}) = (-3, -5)$.}
				\label{fig:insee-vector-candidates}
			\end{figure}
			
			The twelve vectors considered are constructed as follows.
			
			First a shortest path vector from the source to target node are
			constructed as if the network was a 2D mesh yielding a vector
			$(\Delta_\textrm{X},\Delta_\textrm{Y})$. From this, another vector
			$(\Delta'_\textrm{X},\Delta'_\textrm{Y})$, is defined:
			%
			\begin{align}
				\Delta'_\textrm{X} &= \Delta_\textrm{X} - \operatorname{sign}(\Delta_\textrm{X})W
				\\
				\Delta'_\textrm{Y} &= \Delta_\textrm{Y} - \operatorname{sign}(\Delta_\textrm{Y})H
			\end{align}
			%
			Where $W$ and $H$ are the width and height of the network respectively
			(in nodes). This new vector yields routes from the source to destination
			node that in a torus topology that \emph{always} wrap around the `X' and
			`Y' dimensions.
			
			From the pair of vectors defined, four possible 2D vectors can be
			produced: $(\Delta_\textrm{X},\Delta_\textrm{Y})$,
			$(\Delta'_\textrm{X},\Delta_\textrm{Y})$,
			$(\Delta_\textrm{X},\Delta'_\textrm{Y})$ and
			$(\Delta'_\textrm{X},\Delta'_\textrm{Y})$. Further, each 2D vector may be
			converted into one of three 3D vectors where either X, Y or Z are zero
			for a total of twelve candidate vectors.  Figure
			\ref{fig:insee-vector-candidates} illustrates all twelve candidate
			vectors for an example pair of nodes.
			
			\begin{figure}
				\center
				\buildfig{figures/xyz-protocol-regions.tex}
				
				\caption{The four regions defined by the XYZ-protocol.}
				\label{fig:xyz-protocol-regions}
			\end{figure}
			
			A more efficient technique is proposed by Hoffmann and D\'es\'erable
			called the XYZ-Protocol \cite{hoffmann15,hoffmann11}. If the source and
			destination nodes are translated such that the source node lies at the
			center of the topolgoy, the destination will lie in one of four regions
			illustrated in figure \ref{fig:xyz-protocol-regions}.
			
			If the destination lies in regions 1 or 4, a route may be constructed as
			if in a hexagonal mesh topology.
			
			Alternatively, if the destination lies in regions 2 or 3, the algorithm
			tests whether taking a mesh-like route within the region or
			wrapping-around either the X or Y dimension yields the shorter vector.
			The shortest of these vectors is then chosen.
			
			TODO DESCRIBE SPIRAL ROUTES.
			
			TODO DESCRIBE RTOR AND LDFR.
		
	\section{Dimension order routing in hexagonal torus topologies}
		
		So, existing solutions have two problems: trying 12 options and picking one
		is a bit kludgey and there are actually more options than that.
		
		\subsection{Simpler minimal hexagonal torus vectors}
			
			If you redraw your route such that it is sourced from bottom left corner
			(which we'll now call (0, 0)), there are four possible ways this route
			could wrap.
			
			TODO: DESCRIBE WAYS OF WRAPPING
			
			For each of these wrappings, all the possible routes we can take are
			strictly limited in terms of the dimensions used since we're stuck in a
			corner.
			
			In each case, the function computing the minimal hex vector function
			simplifies to a much simpler operation.
			
			TODO: DESCRIBE MINIMUM VECTOR LENGTH FUNCTIONS FOR EACH CASE
			
			This gives us a cheap way to compute which of the four possible wrappings
			are shortest. Based on this we can pick one of at most two (is this
			easily provable?) vectors in some fair manner to reduce load. This vector
			can then be minimised in the usual way.
			
			This also leads to confirming a theoretical result giving the length of a
			shortest path in a hexagonal torus topology.
			
			TODO: DESCRIBE HOW TO GET LENGTH FUNCTION AND COMPARE WITH \cite{xiao04}
		
		\subsection{Generating spiralling routes}
			
			In non-hexagonal torus topologies the previous technique would reveal all
			possible shortest vectors (e.g. in cases where you can wrap more than one
			way). Unfortunately, due to the addition of a non-orthogonal axes,
			hexagonal toruses also have other cases when the width and height do not
			match.
			
			TODO: TORUS SPIRALLING EXAMPLE
			
			It is possible to calculate the maximum number of spirals thus:
			
			TODO: DESCRIBE HOW MAX NUMBER OF SPIRALS IS COMPUTED
			
			Given a number of spirals, the vector can be updated this (note that the
			change does not add a multiple of (1, 1, 1) but also does not result in
			the vector changing length and thus becoming non-minimal!).
			
			TODO: DESCRIBE TRANSFORMATION
			
			TODO: PROVE THAT MINIMALITY IS MAINTAINED
		
		\subsection{Proof of completeness}
		
			TODO: PROOF OF COMPLETENESS BY EXHAUSTIVE SEARCH
	
		\subsection{Conclusions}
			
			This approach is simpler, smaller, has sounder theoretical basis, and
			finds more routes than alternatives. This is good for load balancing and
			fault avoidance and also good for completeness.


	\chapter{Routing packets in large SpiNNaker machines}
	
	\label{sec:routing}
	
	So far, this thesis has focused on tackling the practical challenges
	resulting from SpiNNaker's hexagonal torus network topology. In this chapter,
	I adjust my focus towards the practical challenges resulting from SpiNNaker's
	large scale. Faults in large systems are inevitable and in the half-million
	core, \num{28800} chip SpiNNaker machine recently completed at the University
	of Manchester, around \SI{1}{\percent} of chips exhibited faults\footnote{Of
	the faulty chips discovered, the vast majority of faults are attributed to a
	currently unknown SDRAM failure}. These faults result in gaps and broken
	links in the network topology which routing algorithms must avoid in order to
	ensure correct system operation.
	
	In this chapter I tackle the problem of extending existing routing algorithms
	for SpiNNaker's network to enable them to route-around known, static faults.
	Though dynamic or transient faults may also occur, in this work such faults
	are ignored and other techniques, such as protocol-level fault tolerance, are
	relied on instead.
	
	Numerous heuristic-based fault-tolerant routing algorithms exist which target
	different network topologies and router architectures. Unfortunately as I
	will show, these algorithms are not portable and rely on or attempt to work
	around specific features of their target network architecture. In particular,
	existing work is dominated by the challenge of developing routing schemes
	which avoid or defuse network deadlocks. Due to SpiNNaker's unconventional
	use of timeout-based flow-control, it is not subject to the routing
	restrictions present in other architectures intended to cope with deadlocks.
	
	In this chapter I introduce a graph-search based post-processing step for
	non-fault-tolerant routing algorithms which guarantees routability in
	SpiNNaker systems without disconnected subregions. I also demonstrate that
	this technique introduces both negligible computational overhead to the
	routing algorithm runtime and resulting network performance.
	
	TODO: NOTE THE FAULT RATES ENCOUNTERED IN PRACTICE...
	
	\section{Related work}
		
		Existing work on routing in SpiNNaker's network has ignored the challenge
		of avoiding faults and instead focused on producing efficient multicast
		routes. As a result this section is broken into two halves. In the first
		half I survey the existing non-fault-tolerant approaches to routing used in
		SpiNNaker to-date. In the second I discuss the approaches to fault tolerant
		routing taken in other systems.
		
		\subsection{Multicast routing in SpiNNaker}
			
			Various fault-intolerant multicast routing algorithms exist for many
			networks and a number have been proposed and evaluated specifically in the
			context of SpiNNaker.
			
			In 2012, Davies \emph{et al.} evaluated the use of three common torus
			routing algorithms in SpiNNaker and found that simple oblivious routing is
			suitable for typical neural applications \cite{davies12}. The three
			routing techniques are:
			
			\begin{description}
				
				\item[Dimension Order Routing (DOR)] Packets are routed along each
				dimension (e.g. $X$, $Y$ and $Z$) in turn until no further hops are
				available in that direction.  The order in which the dimensions are
				traversed is fixed.
				
				\item[Right Turn Only Routing (RTOR)] As in DOR except the dimension
				order is chosen such that routes only contain right-turns.
				
				\item[Longest Dimension First Routing (LDFR)] As in DOR except the
				dimension order is chosen in descending order of number of hops in each
				dimension.
				
			\end{description}
			
			These unicast routing techniques are converted into a multicast routing
			algorithm by merging together the routes produced between the source node
			and each destination node as illustrated in figure
			\ref{fig:simple-routers}.
			
			\begin{figure}
				\center
				\begin{subfigure}{0.3\linewidth}
					\center
					\buildfig{figures/simple-routers-dor.tex}
					
					\caption{DOR}
					\label{fig:simple-routers-dor}
				\end{subfigure}
				\begin{subfigure}{0.3\linewidth}
					\center
					\buildfig{figures/simple-routers-rtor.tex}
					
					\caption{RTOR}
					\label{fig:simple-routers-dor}
				\end{subfigure}
				\begin{subfigure}{0.3\linewidth}
					\center
					\buildfig{figures/simple-routers-ldfr.tex}
					
					\caption{LDFR}
					\label{fig:simple-routers-dor}
				\end{subfigure}
				
				\caption{Example multicast routes produced by merging together unicast
				routes from a central source node to each destination node.}
				\label{fig:simple-routers}
			\end{figure}
			
			In 2014, Navaridas \emph{et al.} introduced two new algorithms, `Enhanced
			Shortest Path Routing' (ESPR) and `Neighbourhood Exploring Routing' (NER)
			which produce multicast routing trees with fewer total hops
			\cite{navaridas14}. In both algorithms, routes are generated sequentially
			for each of the destinations of a route using LDFR. Unlike LDFR, however,
			these algorithms search a limited area of the network for other,
			already-connected destination nodes to which LDFR routes may be
			constructed. If no suitable destination is found, a LDFR route is
			constructed to the source node. Figure \ref{fig:search-regions} illustrates
			the shape of the searched regions of each algorithm. ESPR searches the
			trapezoidal region between the source and destination nodes while NER
			searches a fixed radius out from the destination node.
			
			\begin{figure}
				\center
				\begin{subfigure}{0.45\linewidth}
					\center
					\buildfig{figures/search-regions-espr.tex}
					
					\caption{ESPR}
					\label{fig:search-regions-espr}
				\end{subfigure}
				\begin{subfigure}{0.45\linewidth}
					\center
					\buildfig{figures/search-regions-ner.tex}
					
					\caption{NER}
					\label{fig:search-regions-espr}
				\end{subfigure}
				
				\caption{The ESPR and NER algorithms attempt to connect the node marked
				`D' to the closest node in the shaded region which is connected to the
				source node, `S'. If no connected node is found in the shaded region, the
				LDFR route is taken to `S'. The dotted line indicates the route chosen
				from `D'.}
				\label{fig:search-regions}
			\end{figure}
			
			Unfortunately none of these routing algorithms make any allowance for the
			avoidance of network faults. As a result their utility in real-world
			systems is limited.
		
		\subsection{Fault-tolerant routing}
			
			Numerous fault-tolerant routing algorithms have been proposed for
			super-computer networks. These algorithms are largely constrained by the
			need to maintain deadlock freedom. Since SpiNNaker's routers employ a
			timeout based deadlock-breaking strategy, much of this effort is
			unnecessary in SpiNNaker. As described below, this frequently renders
			existing fault-tolerant routing algorithms unnecessarily complex and
			inflexible.
			
			Deadlocks occur in a network if a cyclic dependency exists on any limited
			resource in the network. For example, as illustrated in figure
			\ref{fig:ring-deadlock}, in a ring network a deadlock may form when every
			node is waiting on the next node to accept a packet before accepting new
			packets from the previous node.
			
			\begin{figure}
				\center
				\buildfig{figures/ring-deadlock.tex}
				
				\caption{A deadlock in a ring network where each node is waiting for
				the next to accept a packet before accepting any further packets.}
				\label{fig:ring-deadlock}
			\end{figure}
			
			To prevent deadlocks, combinations of router microarchitectural features
			and routing restrictions may be employed. For example, a simple
			deadlock-free routing algorithm for mesh and torus networks mandates the
			use of DOR \cite{dally93}. Packets travelling in a -ve direction along
			each axis take priority over those travelling in a +ve direction. Packets
			travelling along the Y axis take priority over those travelling along the
			X dimension. Given these rules it is possible to define a total ordering
			on all hops in the network. Figure \ref{fig:deadlock-free-dor}
			illustrates a $3\times3$ mesh network whose hops have been numbered
			according to the total ordering defined above.  Any `X-then-Y' DOR route
			through this network results in the use of hops labelled with strictly
			increasing numbers. As a result, no cyclic dependencies (and thus no
			deadlocks) may occur.
			
			\begin{figure}
				\center
				\buildfig{figures/deadlock-free-dor.tex}
			
				\caption{Deadlock-free routing of two example routes using DOR in a 2D
				mesh topology. The numbers of the hops taken by each route are given on
				the right.}
				\label{fig:deadlock-free-dor}
			\end{figure}
			
			Unfortunately, the routing restrictions imposed to ensure deadlock
			freedom can result in fault-intolerant routing. In the example above, if
			the node at the bottom-right corner of the figure was faulty, the dotted
			example route would be blocked as no alternative routes are allowed.
			
			In practice, the routing rules used may be more relaxed, for example
			requiring that any route whose length is equal to a DOR must exist to
			guarantee routability \cite{rodrigo09}.
			
			Alternative routing strategies take a hybrid approach whereby an
			efficient but fault-intollerant routing algorithm is used where possible
			and in the presence of faults a less efficient but more robust strategy
			is employed. For example, the Immucube network architecture employs three
			virtual networks which operate independently over the same physical links
			\cite{puente07}. Initially messages are routed using a high-performance
			but potentially-deadlockable routing scheme in the first virtual network.
			If a deadlock is occurs, the deadlocked packet is dropped into the second
			virtual network in which packets are routed using a less efficient but
			deadlock-free but fault-intolerant routing algorithm. Finally, upon
			encountering a fault, packets are dropped onto the third virtual network
			which forms a ring network routing packets to every node in the network.
			
			Releated approaches \cite{mejia06,boppana95} divide the network into
			regions in which different routing rules are enforced to ensure deadlock
			freedom and, when required, fault tolerance.
			
			TODO FIGURE?
			
			The BlueGene/L supercomputer \cite{adiga02} uses DOR for its torus
			network and implements fault-tolerance by sacrificing otherwise
			functioning `lamb' nodes to ensure no route passes through a known dead
			link \cite{ho04}. In figure \ref{fig:lamb-nodes} an example scenario is
			shown where a single dead node is present and all nodes in the same row
			or column as the dead node have been made into lamb nodes. The lamb nodes
			may not be used in an application except as a through-route for other
			traffic. This pattern of lamb nodes guarantees that all dimension-order
			routes between all pairs of non-lamb nodes are not obstructed by the
			faulty node. This approach trades use of higher performance routing
			logic for wasted resources. This type of approach is most appropriate
			when algorithmic routing is used and routing rules are inflexible.
			
			\begin{figure}
				\center
				\buildfig{figures/lamb-nodes.tex}
				
				\caption{`Lamb' nodes may be disabled to ensure DOR will never
				encounter a fault.}
				\label{fig:lamb-nodes}
			\end{figure}
			
			Other algorithms proposed for the BlueGene architecture attempt to avoid
			the need for lamb nodes by generating routes which reach their destination
			via a `proxy' node \cite{gomez04}. By appropriately selecting the location
			of such a proxy, the existing routing algorithm used by the system can be
			guaranteed to select a route free of faults.
			
			TODO: EXAMPLE OF PROXY ROUTING TO AVOID FAULT
			
			Finally, many algorithms in in the field are distributed and use only local
			information along with limited information from their peers to generate
			routes \cite{fick09b}. In SpiNNaker, route generation is conventionally
			carried out centrally since no special on-chip hardware facilities exist
			for route generation. Centralised route generation also enables the routing
			algorithm to consider all available routes. As a result, there is little
			incentive for the use of distributed routing algorithms on SpiNNaker since
			global system information could be compactly shared for one-off routing
			passes.
			
			Algorithms for other architectures such as IP networks tend to be poor fits
			for static, regular network topologies since they use expensive graph-based
			algorithms for route discovery which aren't necessary here. They also tend
			to heavily feature graph topology discovery etc. which aren't needed here.
			
			Work on fault-tolerance in data centre networks does exploit the regularity
			of the network topology in routing algorithms \cite{guo08,liao12}.
			Unfortunately, the approaches used are not general enough to be applied to
			mesh-like topologies such as the one in SpiNNaker.
			
			Outside the field of computer networks, routing algorithms used to route
			wires across the surfaces of chips are required to solve similar problems
			to fault-tolerant network routing problems in mesh networks. Like mesh
			networks, the routes are defined within a regular Manhattan geometry and
			congested areas, rather than faults must be avoided by the algorithms
			\cite{kahng11}.  Unfortunately, these algorithms are designed for
			occasional batch operation prior to the multi-month process of chip
			manufacturing and so runtimes of hours or days are commonplace
			\cite{nam08}. As such these algorithms would be inappropriate for use
			with applications such as SpiNNaker where users' applications tend to be
			short-lived and thus routing should not be allowed to dominate runtime.
	
	\section{Partial graph search repair}
		
		In this section I introduce a novel post-processing algorithm, Partial
		Graph Search (PGS) repair, for routes produced by non-fault-tolerant
		routing algorithms.
		
		PGS repair guarantees routability for networks with no disconnected
		subregions by using a graph search algorithm to route around faults in the
		original route.  General-purpose graph search algorithms such as Breadth
		First Search (BFS), Dijkstra's Algorithm and A* are guaranteed to find
		shortest-path routes between pairs of points in arbitrary graphs. Such
		algorithms are generally a poor choice in highly regular network topologies
		such as meshes and toruses due to their high computational cost. In PGS
		repair, graph searching is only used for \emph{part} of the routing
		problem: to repair gaps in routes generated by more efficient routing
		algorithms.
		
		Real world super computer architectures are designed to ensure that faults
		are isolated \cite{gara05,alverson12} and thus tend to only impact a
		localised region of the network. Since PGS repair is only needed to route
		around these isolated faults, the space searched by the graph search
		algorithm should be very small in practice resulting in only short
		runtimes. In addition since faults are rare in real-world systems, the
		graph search process will only rarely be invoked.
		
		The PGS repair post-processing technique starts with a route produced by a
		non-fault-tolerant routing algorithm such as ESPR or NER. If this route is
		not obstructed by a fault, the algorithm terminates immediately without
		modifying the route. If the route attempts to use a faulty link, the
		algorithm proceeds as follows.
		
		The routing tree produced by the underlying routing algorithm is broken
		into subtrees wherever it attempts to route through a broken link and
		each subtree is assigned a unique colour, as illustrated in figure
		\ref{fig:pgs-repair-colouring}. From each disconnected subtree's root
		node in turn, a graph search is performed to find a short, fault-free
		route to a subtree node of a different colour. The subtree is then
		attached to the tree discovered by the graph search and re-coloured to
		match the tree it is connected to.
		
		\begin{figure}
			\center
			\begin{subfigure}{0.32\linewidth}
				\hspace*{-1.5em}
				\buildfig{figures/pgs-repair-colouring.tex}
				
				\caption{}
				\label{fig:pgs-repair-colouring}
			\end{subfigure}
			\begin{subfigure}{0.32\linewidth}
				\hspace*{-1.5em}
				\buildfig{figures/pgs-repair-colouring-fix1.tex}
				
				\caption{}
				\label{fig:pgs-repair-colouring-fix1}
			\end{subfigure}
			\begin{subfigure}{0.32\linewidth}
				\hspace*{-1.5em}
				\buildfig{figures/pgs-repair-colouring-fix2.tex}
				
				\caption{}
				\label{fig:pgs-repair-colouring-fix2}
			\end{subfigure}
			
			\caption{PGS repair process example showing a disconnected multicast
			route from A to B, C, D, E and F. $\times$ indicates a broken link.}
			\label{fig:pgs-repair-colouring-steps}
		\end{figure}
		
		For example in figure \ref{fig:pgs-repair-colouring-fix1} a path from the
		root of the subtree containing nodes E and F is found which connects it to
		the subtree rooted at A. Similarly in figure
		\ref{fig:pgs-repair-colouring-fix2} a path is also found connecting the
		subtree containing nodes C and D back to the subtree rooted at node A.
		
		If the routing tree was broken into $N+1$ subtrees by faults there will be
		$N$ subtrees disconnected from the root node. Each of the $N$ iterations of
		the algorithm connect a disconnected subtree to another subtree reducing
		the number of subtrees by $1$ each time. After $N$ iterations, therefore,
		exactly $1$ subtree remains which connects every node in the original
		routing tree without traversing faulty links.
		
		TODO: EXPLAIN THE FIDDLINESS HERE TO ENSURE WE DON'T CREATE LOOPS.
		
	\section{Evaluation \& Results}
		
		The PGS repair technique, by design, is able to work around all possible
		fault patterns which don't completely disconnect parts of the network. This
		result this evaluation focuses on the impact on performance PGS repair
		imposes. The metrics of interest in this evaluation are:
		
		\begin{itemize}
			\item Algorithm runtime
			\item Network congestion
			\item Routing table utilisation
		\end{itemize}
		
		\subsection{Traffic Patterns}
			
			In this evaluation, two standard benchmark multicast traffic patterns are
			used which have been used in previous research into SpiNNaker's network:
			
			\begin{figure}
				\center
				\buildfig{figures/traffic-distribution-centroids.tex}
				
				\caption{An example 4-centroid distribution with four centroids. The
				$\times$ marks the location of the origin node. Lighter colours
				indicate greater likelihood of a connection.}
				\label{fig:traffic-distribution-centroids}
			\end{figure}
			
			\begin{description}
				
				\item[Uniform] Destinations are chosen with uniform probability
				anywhere in the machine.
				
				\item[$N$-Centroids] Destinations are clustered around one of $N$
				randomly chosen `centroids' as illustrated in figure
				\ref{fig:traffic-distribution-centroids}.
				
			\end{description}
			
			The uniform traffic pattern is widely used in networks research
			\cite{dally04,davies12} while the centroids model was developed
			specifically to reproduce the traffic patterns found in the neural
			applications SpiNNaker is designed for \cite{navaridas14}. In this work
			we consider 3 centroids.
		
		\subsection{Fault model}
			
			In addition two different fault models are used which are representative of
			the faults found in real SpiNNaker systems:
			
			\begin{figure}
				\center
				\begin{subfigure}{0.48\linewidth}
					\hspace*{-1.5cm}
					\buildfig{figures/fault-example-uniform.tex}
					
					\caption{Uniform}
					\label{fig:fault-example-uniform}
				\end{subfigure}
				\begin{subfigure}{0.48\linewidth}
					\hspace*{-1.5cm}
					\buildfig{figures/fault-example-hss.tex}
					
					\caption{HSS Link}
					\label{fig:fault-example-hss}
				\end{subfigure}
				
				\caption{The two link fault models considered.}
				\label{fig:fault-example}
			\end{figure}
			
			\begin{description}
				
				\item[Uniform] Links are selected and disabled at random (figure
				\ref{fig:fault-example-uniform}).
				
				\item[HSS Link] Groups of links corresponding with randomly selected
				single High-Speed Serial (HSS) link between SpiNNaker boards are disabled
				together (figure \ref{fig:fault-example-uniform}).
				
			\end{description}
			
			The uniform link failure model models isolated failures resulting from
			isolated manufacturing defects in individual links. The HSS Link failure
			model models faults arising from failing or disconnected board-to-board
			links which carry several chip-to-chip traffic flows via a single cable in
			SpiNNaker systems. Though SpiNNaker-specific, the later fault model is
			analogous to failure modes arising in other architectures where a single
			fault may render several links impassable in a single area.
			
			A range of failure rates are explored in this section. My measurements of
			current large-scale SpiNNaker installations the link failure rate is about
			\SI{0.03}{\percent} with failures due to both individual chip-to-chip links
			and board-to-board HSS links. Exact link failure statistics for commercial
			super computer installations are not widely available, however, published
			Mean-Time-Between-Failure (MTBF) statistics place an upper bound on link
			failure rates at a similar \SI{0.03}{\percent} in one-year-old BlueGene/Q
			systems \cite{chiu11}.
			
			Unfortunately presently undiagnosed problem with the SDRAM packaged with
			approximately \SI{1}{\percent} of SpiNNaker chips has rendered these chips
			unusable for most applications. The gaps in the network resulting from the
			loss of these chips currently dominate true link failures leaving just over
			\SI{1}{\percent} of links inoperable.
			
			Surprisingly, research into fault tolerant routing in super computers
			appears to focus on benchmarks with even higher fault rates ranging from
			\SI{3}{\percent} to as high as \SI{7}{\percent}
			\cite{ho04,gomez04,mejia06}.
			
			In this evaluation, fault rates ranging from \SI{0.01}{\percent} to
			\SI{5}{\percent} are considered to cover both realistic fault levels
			along with the more extreme cases considered in related work.
		
		\subsection{Base routing algorithm}
			
			Since the PGS repair process is routing algorithm agnostic all
			experiments use the NER algorithm which has been found to be appropriate
			for SpiNNaker applications \cite{navaridas14}.
		
		\subsection{Algorithm runtime}
			
			To assess the impact of the PGS repair process on routing algorithm
			runtime, the algorithm was used to process a large number of randomly
			generated routing problems and the runtime recorded.
			
			\num{10000} one-to-sixteen multicast routing problems were generated in a
			$256\times256$ hexagonal torus topology, the largest size possible for a
			SpiNNaker system. Other quantities of multicast destinations were also
			evaluated but are omitted for brevity since the pattern of results are
			similar to those outlined here.
			
			TODO: APPENDIX WITH OTHER RUNS?
			
			The NER and PGS repair algorithms were written in C and compiled with GCC
			4.8.3 with \verb|-O2| level optimisations and executed on a cluster of
			idle workstations with 3.10 GHz Intel Core-i5-2400 CPUs.
			
			\begin{figure}
				\center
				\buildrplot{figures/routing-runtimes.R}
				
				\caption{Mean runtime of routing and PGS repair overhead. PGS repair
				overhead is stacked above the routing runtime (i.e. bars do not
				overlap). Error bars indicate 95\% confidence interval. Note different
				Y-scale for HSS link and uniform fault models.}
				\label{fig:routing-runtimes}
			\end{figure}
			
			Figure \ref{fig:routing-runtimes} shows the average runtimes recorded for
			both the NER and PGS repair algorithms. In fault-free networks the
			PGS-repair post-processing step is not required and incurs no penalty
			while the runtime of the algorithm grows with the fault rate for both
			fault and traffic models.
			
			Notably the HSS fault model results in longer runtimes for the PGS repair
			process compared with an equivalent fault-density of uniform faults.
			Because the HSS fault model produces contiguous lines of faults the PGS
			repair algorithm must construct a longer path to avoid the fault.  Since
			the space explored by a graph algorithm typically grows with $O(H^2)$
			with respect to the hops in the discovered route, $H$, this increase in
			search distance has a large impact on the runtime of the PGS repair
			process.
			
			The runtime of the PGS repair algorithm remains roughly in proportion to
			the runtime of the underlying routing algorithm with respect to different
			traffic models. The centroid traffic pattern tends to result in routes
			with fewer hops than a uniform traffic pattern with the same number of
			destination nodes as segments of routes are often shared between
			destination nodes. Since the NER algorithm's runtime is strongly related
			to the number of hops in the output route the runtime of the algorithm is
			greater for uniform traffic. Likewise the probability of PGS repair being
			required increases with the number of hops in route and hence the runtime
			of the PGS repair algorithm increases roughly in proportion.
		
		\subsection{Routing table usage}
			
			In order to gain a realistic measure of routing table usage it is
			necessary to determine the effect of many routes being generated for a
			single set of faults. To enable a sufficiently large number of sample to
			be collected the experimental setup considered previously is reduced to a
			network containing $48\times48$ nodes.
			
			\num{1000} $48\times48$ node network models are produced according to the
			HSS link and uniform fault models. For each of these models
			$48\times48\times16=$~\num{36864} one-to-sixteen routes are generated using
			the centroid and uniform traffic models. This corresponds to one
			multicast route per application core. As is convention in SpiNNaker,
			routing table entries are inserted for each route at the source of the
			route, at each destination and at each corner or fork. The number of
			routing table entries at each node in the model is counted and the
			maximum number of entries in a single node is reported for each network
			model.  The \emph{maximum} number of routing entries of any router was
			chosen since the number of entries available per SpiNNaker router is
			bounded by hardware.
			
			\begin{figure}
				\center
				\buildrplot{figures/routing-entries.R}
				
				\caption{Violin plot showing the distribution of maximum table sizes
				for \num{1000} random networks. The red line at \num{1024} entries
				indicates the size of SpiNNaker's routing tables.}
				\label{fig:routing-entries}
			\end{figure}
			
			
			Figure \ref{fig:routing-entries} shows the distributions of the largest
			routing table sizes for each fault and traffic model.
			
			\begin{figure}
				\center
				\begin{subfigure}{0.48\linewidth}
					\center
					\buildfig{figures/hss-link-routing-table-usage.tex}
					
					\caption{Routing table entries}
					\label{fig:hss-link-routing-table-usage}
				\end{subfigure}
				\begin{subfigure}{0.48\linewidth}
					\center
					\buildfig{figures/hss-link-resource-usage.tex}
					
					\caption{Routes passing through chip}
					\label{fig:hss-link-resource-usage}
				\end{subfigure}
				
				\caption{The impact of a HSS link fault on routing table usage and
				congestion. Each hexagon represents a single chip, the red line
				indicates the chip-to-chip connections broken by the HSS link fault.}
				\label{fig:hss-link-usage}
			\end{figure}
			
			The HSS link failure model has a much greater impact on peak routing
			table resource usage than uniform link failures for a given fault rate.
			This is because HSS link faults result in a large concentration of routes
			being disrupted and then re-routed around the same obstacle in a single
			location. Figure \ref{fig:hss-link-routing-table-usage} shows how routing
			table usage varies around a HSS link fault in one instance of the
			experiment. There are clear peaks in routing table usage around the ends
			of the line of faults which result from routes produced by PGS repair
			finding shortest paths around the edge of the faults.
		
		\subsection{Network congestion}
			
			To measure the impact of PGS repair on network congestion, two
			experiments were performed, one using the same model used to measure
			routing table usage and one based on tests run on SpiNNaker hardware.
			
			For each of the network fault and traffic pattern described previously,
			the paths taken for the \num{36864} one-to-sixteen multicast routes
			generated are used to compute the number of times each link in the
			network is used. The number of routes passing through the most-used link
			is then recorded, giving an indication of the level of congestion in the
			network.
			
			\begin{figure}
				\center
				\buildrplot{figures/routing-resource.R}
				
				\caption{Violin plot showing the distribution of maximum
				routes-per-chip for \num{1000} random networks.}
				\label{fig:routing-resource}
			\end{figure}
			
			The results are presented in figure \ref{fig:routing-resource} and follow
			the same trends as the results previously shown for routing table usage.
			Again, HSS link faults result in routes with the greatest congestion due
			to the concentration of routes finding shortest paths around an obstacle
			(see \ref{fig:hss-link-resource-usage}).
			
			To verify that the results above, an additional experiment has been
			carried out which attempts to mimic the model used previously in actual
			SpiNNaker hardware. In these experiments a large SpiNNaker machine is
			divided into independent 48-board (2304-chip) sections. Because the
			48-board systems used in these experiments are cut out of a larger
			machine, they lack wrap-around links and thus form hexagonal mesh
			topologies, rather than hexagonal toruses.
			
			Due to the SDRAM issue described above, fault rates below
			\SI{1}{\percent} cannot be modelled.  To simulate higher fault rates,
			additional links are disabled in software according to the fault models
			described used previously. Since some faults are due to genuine hardware
			faults, these faults cannot be placed randomly in each experiment. To
			reduce, bias each combination of fault rate, fault model and traffic
			pattern is repeated XXX times across randomly chosen physical machines.
			
			XXX 1-to-XXX routes are generated in both uniform and XXX-centroid
			distributions as used throughout this evaluation. Synthetic network
			traffic is generated at the source of each route following a Bernoulli
			distribution. Traffic consumers running on all destination nodes accept
			packets as quickly as possible from the network and log their arrival.
			The Bernoulli probability is set the same for every route's traffic
			generator and increased in steps of XXX and the number of packets dropped
			in an XXX second period logged. The network is considered saturated once
			less than \SI{99}{\percent} of packets successfully arrive at their
			destination.
			
			Figure \ref{XXX} shows the distributions of the saturation points for
			each experimental configuration.
			
			TODO: ANALYSIS
		
	\section{Conclusions}
		
		In this chapter I described how SpiNNaker's unconventional network and
		router architecture render existing fault tolerant routing algorithms
		unsuitable. I introduced PGS repair, a post-processing technique for
		existing non-fault tolerant routing algorithms designed for SpiNNaker such
		as NER.
		
		Unlike some other fault tolerant routing algorithms for other
		architectures, PGS repair is able to work-around arbitrary fault patterns
		by exploiting SpiNNaker's inbuilt deadlock avoidance mechanisms. In the
		presence of realistic failure rates of up to \SI{1}{\percent}, only small
		overheads of up to XXX, XXX and XXX for in algorithm runtime, routing table
		usage and network performance are incurred respectively. This low
		performance overhead makes PGS repair appropriate for use in real
		applications. At the time of writing the algorithm has been successfully
		used in a number of neural and non-neural SpiNNaker applications.
		
		At more extreme fault rates not expected in real-world systems, the
		algorithm still functions correctly but the results incur much greater
		routing table and congestion overheads, particularly when faults are
		concentrated. Future extensions to this algorithm might aim to reduce this
		overhead by producing longer and more varied routes around faults to even
		out the load.

	\chapter{Placing applications in large SpiNNaker machines}
	
	In the previous chapter I tackled the problem of scale in generating routes
	for very large networks such as SpiNNaker. In this work the centroid traffic
	pattern was used as an approximation of the expected network traffic
	generated by `well behaved' neural network simulation software running on
	SpiNNaker. The traffic produced largely exhibits strong locality, that is
	most communication occurs between either nearby nodes or clusters of nodes.
	In reality, neural simulation applications are not specified geometrically
	but rather as abstract graphs of communicating neurons
	\cite{davison08,eliasmith13}. Applications must then \emph{place} these
	neurons onto nodes in a SpiNNaker system, attempting maximise communication
	locality.
	
	In this chapter I re-evaluate the suitability of simulated annealing as a
	technique for finding high quality placements for large parallel
	applications. Though this technique had fallen out of fashion in the field of
	application placement by the early 1990s, it has found wide use for placing
	components in computer chip and FPGA designs. In the intervening years,
	placement problems in super computers have grown in size from tens or
	hundreds of nodes to millions, a scale at which chip placement techniques
	were operating in the mid 1990s. I adapt the simulated annealing algorithm
	used by the VPR academic circuit placement software to produce placements for
	applications running on SpiNNaker. In that in a range of real and synthetic
	benchmarks simulated annealing produces high quality placements enabling
	efficient use of SpiNNaker's network resources.
	
	
	%In the field of chip design, Moore's `Law' \cite{moore65,moore75} observes a
	%similar exponential growth in the number of components within a single chip.
	%Today modern processors contain billions of components and an analagous
	%placement problem exists in attempting to place interconnected components
	%near to eachother. In this chapter I explore the techniques used for circuit
	%placement and adapt one such technique, Simulated Annealing (SA)
	%\cite{kirkpatrick83}, for use in application placement. Despite some early
	%interest in SA for application placement in the 1980s and early 1990s, the
	%technique has since fallen out of favour. I find that at the scales of modern
	%placement problems SA-based placement is able to produce solutions of
	%superiour quality to contemporary methods.
	%
	%TODO: SUMMARISE RESULTS...
	
	\section{Related work}
		
		The placement problem has been tackled independently in the literature by
		researchers in both the application and chip placement communities. In this
		survey I cover application and chip placement separately as these two
		communities have remained largely isolated from one another. First I
		explore the techniques applied to application placement before moving on to
		contrast this with the techniques used in circuit placement.
		
		In the application placement literature, the placement problem is often
		referred under the umbrella term `mapping'. Unfortunately term is often
		used more broadly to include other tasks such as routing and application
		partitioning. To avoid ambiguity I use the term `placement', as preferred
		by the chip and FPGA design communities, to refer specifically to the
		problem of assigning nodes in an application's communication graph to nodes
		in a machine's connectivity graph.
		
		\subsection{Application placement algorithms}
			
			TODO: GENERAL INTRO
			
			\subsubsection{Application-specific approaches (manual placement)}
				
				In the case of some applications such as finite element modelling
				\cite{bermejo13}, the structure of the problem itself leads to a
				natural placement of the computation on nodes in a machine. For example
				when simulating a 3D volume in an node super computer with a $3 \times
				4 \times 2$ 3D torus or mesh topology network, the modelled volume
				might be divided into as in figure \ref{fig:fem-partitioning}. Each
				cuboid in the model is then assigned to the corresponding node in the
				network topology.
				
				\begin{figure}
					\center
					\buildfig{figures/fem-partitioning.tex}
					
					\caption{Example partitioning of a 3D space to fit into a super
					computer with a $3\times4\times2$ torus or mesh topology.}
					\label{fig:fem-partitioning}
				\end{figure}
				
				When the number of dimensions in a problem do not match that of the
				underlying network architecture, the common solution is to either
				divide only along a subset of the axes or to divide into additional
				pieces on the existing axes \cite{gilge14}.
			
			\subsubsection{Sequential placement}
				
				In the case where a placement solution is non-obvious one of the
				simplest and most popular strategies is to apply a simple sequential
				placement algorithm. Sequential placement algorithms function by
				iterating over the vertices in the application's communication graph
				and assigning them to a free node in the target machine. Sequential
				placement algorithms are differentiated by the order in which they
				iterate over vertices in the communication graph and fill nodes in the
				target machine. A number of widely used orderings are described below.
				
				\begin{figure}
					\center
					\begin{subfigure}{0.32\linewidth}
						\center
						\buildfig{figures/sequential-row-order.tex}
						\caption{Row-order}
						\label{fig:sequential-row-order}
					\end{subfigure}
					\begin{subfigure}{0.32\linewidth}
						\center
						\buildfig{figures/sequential-alternating.tex}
						\caption{Alternating}
						\label{fig:sequential-alternating}
					\end{subfigure}
					\begin{subfigure}{0.32\linewidth}
						\center
						\buildfig{figures/sequential-hilbert.tex}
						\caption{Hilbert curve}
						\label{fig:sequential-hilbert}
					\end{subfigure}
					
					\caption{Space-filling curves in 2D mesh and torus topologies.}
					\label{fig:sequential}
				\end{figure}
				
				Super computer management software such as SLURM \cite{yoo03} and Blue
				Gene's system software \cite{gilge14} by default na\"ively iterate over
				vertices in an application communication graph in the order they are
				provided. The nodes in the target machine are then iterated over in a
				simple space-filling curve through the network topology. Figure
				\ref{fig:hilbert-placement} illustrates the default patterns available
				in these software packages. The row-order (figure
				\ref{fig:sequential-row-order}) and alternating (figure
				\ref{fig:sequential-alternating}) curves correspond with 2D versions of
				the default node assignment orders used in SLURM and BlueGene systems.
				
				\begin{figure}
					\center
					\buildfig{figures/hilbert-placement.tex}
					
					\caption{A Hilbert curve, coloured from blue to red.}
					\label{fig:hilbert-placement}
				\end{figure}
				
				The Cray extensions to SLURM software provide a Hilbert curve
				\cite{hilbert91} (figure \ref{fig:sequential-hilbert}) node assignment
				order. Unlike the row-order and alternating space filling curves the
				Hilbert curve ensures that pairs of vertices close together in the node
				iteration order are also close together in the target machine's network
				\cite{moon01, zumbusch99}. Figure \ref{fig:hilbert-placement} shows a
				5$^\textrm{th}$-order Hilbert curve where each point in the curve is
				coloured according to its position along the curve. In this figure it
				is possible to see that nearby positions in the curve (which share
				similar colours) are also close in 2D space.
				
				When the proximity of vertices in the vertex-ordering supplied by an
				application is a good estimator of those vertices communication
				requirements, the sequential assignment schemes discussed above can be
				very effective. These techniques have also proven adequate in
				small-scale and densely connected applications such as early neural
				simulations running on prototype SpiNNaker machines with tens of nodes
				\cite{galluppi10} but growing beyond this scale has proven problematic.
				
				\begin{figure}
					\center
					\begin{subfigure}{0.45\linewidth}
						\center
						\buildfig{figures/rcm-initial.tex}
						
						\caption{Original permutation}
						\label{fig:rcm-initial}
					\end{subfigure}
					\begin{subfigure}{0.45\linewidth}
						\center
						\buildfig{figures/rcm-sorted.tex}
						
						\caption{RCM permutation}
						\label{fig:rcm-sorted}
					\end{subfigure}
					
					\caption{Adjacency matrix representation of a graph before and after
					permutation by the RCM algorithm.}
					\label{fig:rcm}
				\end{figure}
				
				A number of algorithms have been proposed for automatically selecting
				good vertex iteration orders, typically using a graph-traversal based
				heuristic. A typical method, described by Hoefler \emph{et al.}
				\cite{hoefler11} exploits the Reverse-Cuthill-McKee (RCM) algorithm
				\cite{cuthill69}. An application's communication matrix is represented
				as an adjacency matrix, $M$, where $M_{i,j}$ is 1 if node $i$ is
				connected by an edge to node $j$ and 0 otherwise. An example matrix is
				illustrated in figure \ref{fig:rcm-initial}. The RCM algorithm uses a
				simple heuristic to permute the matrix (i.e. renumber the nodes in the
				graph) in order to reduce the bandwidth of the matrix. Figure
				\ref{fig:rcm-sorted} shows the RCM-permuted version of the example
				adjacency matrix. When a graph's vertices are ordered as in a
				bandwidth-reduced sparse matrix, vertices close together in the
				ordering are likely to communicate while those further apart tend not
				to communicate.
				
			\subsubsection{Optimisation-based Placement}
				
				% Citations from short report about optimisation in placement...
				% \cite{chen06,jeannot14} and \cite{jeannot10} ("subsets of apps")
				
				In the academic community, a number of attempts have been made to use
				more sophisticated optimisation algorithms for the placement of
				applications. In 1985, Steele \cite{steele85} proposed the use of
				simulated annealing for placing applications in the 6D torus topology
				of the 64 node `Caltech Cosmic Cube' machine. Simulated annealing,
				originally developed by Kirkpatrick \emph{et al.} \cite{kirkpatrick83},
				is a general-purpose optimisation algorithm which works by analogy to
				the physical process of annealing. In brief simulated annealing
				functions by randomly swapping vertices in a candidate placement
				solution, accepting swaps which move connected vertices closer together
				and rejecting some proportion of swaps which move connected vertices
				further apart. The simulated annealing algorithm is described in detail
				later in this chapter.
				
				Towards the end of the 1980s, application placement appeared to be
				becoming less important as super computer network architectures
				improved:
				%
				\begin{displayquote}
					``Careful placement was necessary because of the slow communication
					and non-uniform addressing of early concurrent computers. However,
					the development of message passing machines with fast communications
					and a uniform global address space  has made placement less of an
					issue. In such machines a random placement performs nearly as well as
					an optimum placement.''
					
					\noindent --- W. Dally, 1987 \cite{dally87}
				\end{displayquote}
				%
				In addition, network and problem sizes remained small, so small in fact
				that linear-programming based optimal placement still appeared in
				benchmarks comparing placement algorithms \cite{xu91}. In this
				environment, simpler sequential placement algorithms gained favour over
				more computationally expensive algorithms such as simulated annealing.
				
				As problem and machine sizes have grown and network utilisation has
				once again become an important factor in application performance
				\cite{navaridas09b} more complex optimisation algorithms have
				reappeared in the literature. One popular approach employs graph
				partitioning algorithms such as METIS \cite{karypis98} to perform
				recursive bipartitioning based placement
				\cite{phillips14,hoefler11,pellegrini96}.  This placement process is
				illustrated in figure \ref{fig:partitioning}.
				
				In the first step, the application communication graph and machine
				connectivity graph are bipartitioned such that the number of edges
				between partitions is minimised. Each half of the communication graph
				is associated with one of the halves of the machine connectivity graph.
				The partitioning process is then repeated recursively on each of the
				two communication and connectivity graph pairs. The process halts when
				the graphs can no longer be partitioned at which point the vertices in
				the communication graph are placed on their associated node.
				
				\begin{figure}
					\center
					\buildfig{figures/partitioning.tex}
					
					\caption{Illustration of application placement by recursive
					partitioning.}
					\label{fig:partitioning}
				\end{figure}
				
				TODO: PARTITIONING IS GREAT AND ALL BUT QUALITY ISN'T ALWAYS GREAT AND
				IT DOESN'T DEAL WELL WITH MULTI-CONSTRAINT SCENARIOS E.G. PROCESSOR AND
				MEMORY RESTRICTIONS.
				
				Unfortunately, many of these simply aren't suited to the scale of
				neural applications running on SpiNNaker (e.g. only cope with tens of
				nodes while SpiNNaker may contain hundreds of thousands).
				
				Additionally, a number of algorithms have been developed which make
				assumptions about the topologies of the problem or network. Tree match
				for example attempts to map tree-shaped problems to tree-shaped
				networks. Such algorithms can be highly effective but again do not
				apply to SpiNNaker or its neural applications.
		
		\subsection{Chip placement algorithms}
			
			The chip-design industry has, for many years, dealt with problems
			analogous to the task of placing super computer jobs in a way suited to
			SpiNNaker. Modern CPUs have millions or billions of components with
			strictly fixed connectivity. CPU designers must place each of these onto
			a chip such that the connection lengths are controlled to reduce
			congestion and increase performance. As such, these algorithms are
			ideally suited to future super computer placement work since they already
			operate at the scales required \cite{nam07}.
			
			\subsubsection{Cost functions}
				
				HPWL is popular but a bit crap for high fan-outs. It is, however, quite
				simple.
				
				TODO: SELECT A BETTER COST FUNCTION...
			
			\subsubsection{Simulated annealing}
				
				One of the oldest techniques used for circuit placement is simulated
				annealing and this remains popular today thanks to its sheer
				versatility (see VPR, other open FPGA tools).
				
				SA works by analogy with the physical process of annealing.
				The simulated annealing algorithm works by selecting random pairs of
				components on a chip, swapping them and evaluating some cost function.
				If the swap reduces the cost function, it is kept, if not, depending on
				a function of the current temperature and the cost introduced by the
				swap.
				
				TODO: ILLUSTRATION OF SIMULATED ANNEALING SWAP OPERATION
				
				By occasionally allowing costly swaps, the annealing algorithm avoids
				becoming trapped in local minima. As the algorithm proceeds, the
				temperature is slowly reduced and with it the proportion of costly
				swaps which are retained. This causes the placement to move from
				exploration early on towards refinement later on.
				
				The temperature schedule of an annealing algorithm is critical to its
				success. In general these schedules are computed based on the
				performance of the algorithm as it runs. In VPR the following schedule
				is used.
				
				TODO: DESCRIBE VPR'S SCHEDULE
				
				TODO: FIND AND DESCRIBE ALTERNATIVE SCHEDULE?
				
				Unfortunately, SA is very difficult to parallelise, especially in the
				case of placement. As a result, its scalability has been limited and
				resulted in significantly reduced usage in recent work.
			
			\subsubsection{Partitioning placement}
				
				Partitioning based placement solves the placement problem using
				graph-partitioning recursively on the problem graph to assign each part
				of the circuit to some area in the super chip. Though a number of
				algorithms have proven successful in academic placement contests over
				the years, they are not popular in industrial settings.
			
			\subsubsection{Analytical placement}
				
				In analytical placement, cost function for the circuit graph is
				approximated in a form which is amenable to solutions with standard
				numerical or symbolic algebraic techniques. Using these techniques,
				exact minimum cost (in terms of the approximation) configurations can
				be obtained.
				
				Quadratic placement is a popular analytical placement technique which
				approximates the cost of a placement as the sum of the squares of the
				distances between connected circuit elements.
				
				TODO: FIGURE EXAMPLE QUADRATIC PLACEMENT PROBLEM AND SOLUTION
				
				As such this gives a quadratic cost function like so which we must
				minimise.
				
				TODO: QUADRATIC COST EQN
				
				To minimise the function we differentiate and solve using simple
				symbolic manipulation.
				
				TODO: QUADRATIC COST SOLUTION
				
				Unfortunately, quadratic placement doesn't contain any congestion
				relief by default so various schemes exist. For example, extra anchor
				nodes are inserted which gently pull the circuit components apart from
				each other. As a result, the algorithm generally proceeds by iterating,
				regenerating anchors each time.
				
				Other non-quadratic analytical methods exist too with numerical
				solutions. The approaches are often similar.
			
			\subsubsection{Hierarchical clustering}
				
				Many placement algorithms scale super-linearly with problem size and so
				larger problems become increasingly problematic to handle. To solve
				this problem clustering techniques are first applied to first simplify
				the placement problem. A solution is then found at the coarse level and
				then hierarchically fleshed out.
				
				Various clustering algorithms are in use.
				
				TODO: TALK ABOUT CLUSTERING IN PLACEMENT...
				
				TODO: DESCRIBE THE ALGORITHM I IMPLEMENTED.
	
	\section{Application placement by simulated annealing}
		
		\label{sec:placement-by-annealing}	
		
		I have implemented a simplified SA based application placement algorithm
		based on the approach used in the popular VPR place and route tool chain.
		The algorithm is written in C and is optimised for experimentation rather
		than performance but is production-ready. It has been integrated into the
		`Rig' SpiNNaker software tools and has been used to place very large
		simulations. More on that later.
		
		\subsection{Representation}
			
			Model each chip as having a quantity of various resources (e.g. Cores,
			SDRAM) available. The application graph consists of vertices which each
			consume some quantity of these resources. Vertices must be placed on a
			single chip such that the resources required on a given chip do not
			exceed those available. Vertices are then interconnected by 1:N nets with
			weights which act as hints. The nets are treated as a soft constraint:
			vertices connected via a net will, ideally, be placed near to each other,
			with priority being given to nets with higher weights. Additionally there
			will be a list of placement constraints (see later).
			
			A key observation is that while vertices in an application may frequently
			have a 1:1 correspondence with application cores, this need-not be the
			case. For example, a vertex may represent a block of SDRAM which is
			shared. A vertex may also represent some other resource, for example,
			external IO availability. By making these resource types user-defined,
			applications programmers can express flexible hard-constraints on their
			application.
			
			Another observation is that generic soft constraints can be expressed may
			be expressed using a net with an appropriate weight.
			
			As a result of these facilities, application programmers can easily
			express their own application-specific hard and soft placement
			constraints without having to modify the algorithm. This representation
			has become a de-facto standard for placement problem interchange for
			SpiNNaker applications.
		
		\subsection{Cost function}
			
			At present I've used HPWL despite this being really bad for high-fan-out
			multicast and totally ignorant to the hexagonal nature of SpiNNaker...
			
			To compute bounding boxes for tori I use the following approach. For each
			dimension, sort the points on that dimension and find the largest gap
			between them on a ring. The bounding box goes the other way.
			
			TODO: FIGURE ILLUSTRATING BOUNDING BOX COMPUTATION FOR TORI.
		
		\subsection{Annealing schedule}
			
			The annealing schedule is that used by VPR. Despite being for circuit
			placement, it seems to work jolly well.
			
			TODO: DESCRIBE AND RATIONALISE THE SCHEDULE
		
		\subsection{Constraint handling}
			
			Various hard and soft constraints may be expressed by software
			approaches. For each we explain how they may be handled by the placement
			algorithm:
			
			\subsubsection{Location Constraint}
				
				The vertex is placed on a chip and removed from the set of movement
				candidates.
			
			\subsubsection{Same-chip constraint}
				
				When two vertices must always be placed on the same chip they are
				simply combined into one vertex which consumes the sum of their
				resources. Placement then treats them as one chip and thus is forced to
				atomically place the vertices.
			
			\subsubsection{Reserve resource constraint}
				
				Simply reduce resource availability on that chip.
			
			\subsubsection{Keep near Ethernet}
				
				Simply add a net.
	
	\section{Evaluation}
		
		\label{sec:placement-results}
		
		Though benchmarks exist for super computer loads and chip placement tasks,
		such things don't exist for neural applications. As a result I use a
		selection of real applications for SpiNNaker along with some synthetic
		benchmarks based on biological data.
		
		\subsection{Benchmark networks}
			
			First some real networks.
			
			Some nengo networks: SPAUN: `The world's largest functional brain model'.
			Word-net network from Jamie: Example of some learning.
			
			TODO: DESCRIBE SHAPE OF NENGO NETWORKS
			
			Some PyNN networks: Microcortical column model from PyNN. Note almost
			broadcast connectivity but varying weights. Try and extract a vision
			netlist from Anna. Maybe try and get a netlist for Tom's barrel cortex.
			
			Now for some artificial networks. Pipeline, noisy pipeline, mesh,
			Gaussian 2D.
		
		\subsection{Experiments}
			
			We compare random, linear, greedy and annealing based placement
			approaches to placement. We compare static metrics (such as mean/max
			congestion, table usage) along with experiments based on simulated
			network traffic in real hardware. Network Tester generates artificial
			traffic in proportion with the weights given for each model. We compare
			the relative level of traffic sustainable. We also consider use of
			machines of various sizes.
		
		\subsection{Results}
			
			SA placement is slow but rather effective, especially for some networks.
			Generally worth doing. Will need to be sped up for very large machines...
			
			TODO: EXPERIMENTS!
	

	\chapter{Discussion}

\section{Suitability of the hexagonal torus topology}
	\subsection{Physical scalability}
	\subsection{Routability}
	\subsection{Placeability}

\section{Suitability of the SpiNNaker router}
	\subsection{Deadlock avoidance}
	\subsection{Routing table size}

\section{Suitability of circuit placers for application placement}
	\subsection{Quality}
	\subsection{Runtime}
	\subsection{Routing resources}
	\subsection{Flexibility}
	\subsection{Scalability}


	\chapter{Future research}
	
	In this thesis I have presented a number of new techniques which have made it
	possible to assemble and operate the SpiNNaker super computer. This work
	opens up a range of possibie lines of research to extend this work to future
	architectures and applications. In this chapter I focus on two anticipated
	challenges of future systems: growing scale and greater dynamicism in
	applications.
	
	\section{Scaling up}
		
		TODO: INTRO
		
		\subsection{Grid machine room layouts}
			
			In chapter XXX, I developed a machine room layout for hexagonal torus
			topologies which allowed machines occupying a row of standard
			machine-room cabinets to scale up without the need for long
			interconnecting cables. For larger installations, however, it will be
			necessary to employ multiple rows of cabinets in a 2D arrangement.
		
		\subsection{Routing congestion control}
		
		\subsection{Parallel place and route}
	
	\section{Structural plasticity and dynamic fault-tolerance}
		\subsection{Plasticity models}
		\subsection{Incremental placement}
		\subsection{Incremental routing}
		\subsection{Hot-spare routes}

	\chapter{Conclusions and future research}
	
	The SpiNNaker architecture was designed to tackle the challenges presented by
	the simulation of biologically realistic neural networks. One of its
	distinguishing features is its network architecture which employs both an
	unconventional network topology and multicast router architecture. The
	hexagonal torus topology used by SpiNNaker was chosen to enable greater
	performance while maintaining ease of construction and scalability compared
	with conventional network topologies. SpiNNaker's router design centres
	around packets mimicking the neural `spike' signals they are designed to
	convey by being small, multicast and not guaranteed to arrive at their
	destination.  This novel design, though largely complete before this work
	began, left a number of open problems which this thesis has attempted to
	address.
	
	In this concluding chapter I begin by summarising the answers to the research
	questions raised in chapter~\ref{sec:introduction}. This is followed by a
	discussion of new research topics which have been uncovered by this work.
	
	\section{Answers to research questions}
		
		Each of the three research questions are answered below.
		
		\subsubsection{1. Can the hexagonal torus topology be deployed and used in
		real, large-scale systems?}
		
		In chapter~\ref{sec:building}, I introduced a cabling scheme and assembly
		technique which has been used successfully to build a prototype SpiNNaker
		system with over half a million processor cores using the hexagonal torus
		topology. The techniques shown are expected to enable a final SpiNNaker
		machine of double this size to be built, filling a six metre long row of
		machine-room cabinets.
		
		Though SpiNNaker's processor-count places it amongst some of the world's
		largest supercomputers (see figure \ref{fig:top500-num-processors} on page
		\pageref{fig:top500-num-processors}), it is comparatively compact, filling
		one row of cabinets compared with the warehouse-scale installations found
		in commercial systems. In spite of this, the folding and interleaving
		techniques described allow hexagonal torus topologies to scale to
		arbitrarily large installations without cables which span the machine.
		
		Chapter~\ref{sec:shortestPaths} described an efficient and general
		technique for finding, and enumerating shortest path vectors in hexagonal
		torus topologies. These developments bring the hexagonal torus topology in
		line with other topologies by enabling routing algorithms to exploit all
		possible paths in a network. Further, chapter~\ref{sec:placement}
		demonstrated that placement algorithms are also adaptable to hexagonal
		torus topologies thanks to their similarity to 2D toruses.
		
		Though, as this thesis highlights, hexagonal toruses lack many of the
		intuitive properties enjoyed by other topologies, it is still possible to
		reason about them with only limited computational effort.  Now that the
		practicality and scalability of the topology has also been demonstrated in
		practice, it represents a credible option for future network architectures.
		
		\subsubsection{2. Does SpiNNaker's router architecture help, or hinder
		fault tolerance?}
		
		SpiNNaker's unconventional use of packet dropping to avoid deadlocks
		greatly simplifies the router architecture, part of the motivation for this
		design. In chapter~\ref{sec:routing} this feature is used to the advantage
		of PGS repair to add fault tolerance to existing routing algorithms.
		Compared with the often complex and wasteful methods used to tolerate
		faults in other networks, PGS repair incurs very little performance
		overhead in the presence of static faults.
		
		Routing table usage does increase in the presence of faults, however, which
		may be a concern for applications which already require many routing table
		entries. Routing table usage, as well as other overheads, were most
		significantly increased in the presence of contiguous groups of network
		faults. This is because the PGS repair algorithm produces routes which pass
		tightly around the corners of faults, resulting in concentrations of
		routing table entries in those areas.  Though the symptoms of this problem
		can be attributed to the design of SpiNNaker's multicast routing mechanism,
		the responsibility lies with the behaviour of the PGS repair algorithm.
		Potential improvements to the PGS repair algorithm are discussed later in
		\S\ref{sec:pgs-repair-improvements}.
		
		The overall answer to this research question, therefore, is that the
		flexibility provided to routing algorithms in SpiNNaker's architecture is
		of great benefit, enabling arbitrary fault patterns to be inexpensively
		avoided.
		
		\subsubsection{3. How can the parts of a neural simulation be placed onto a
		large hexagonal torus topology to reduce network load?}
		
		In chapter~\ref{sec:placement}, I explored a number of contemporary
		approaches to the problem of placing applications with irregular
		communication patterns into network topologies. I observed that researchers
		working on circuit placement for chips and FPGAs are tackling similar
		problems and working at scales as large, or larger than, those faced in
		application placement. Based on this I developed a
		simulated annealing based placement algorithm inspired by the techniques
		used in circuit placement, with specific adaptations for use in application
		placement and SpiNNaker's network topology.
		
		The simulated annealing based placement algorithm consistently outperforms
		pre-existing placement algorithms included in benchmarks in terms of
		placement quality.  In the case of one benchmark, simulated annealing based
		placement made it possible to run that neural simulation in real-time for
		the first time.  At larger scales, simulated annealing was also found to be
		able to produce good quality placements in benchmarks containing over one
		million processes -- the largest size supported by the SpiNNaker
		architecture.
		
		The major shortcoming of simulated annealing based placement is its
		execution speed. Though its execution time grows in proportion to the size
		of the problem, the implementation used took over 12~hours to place a
		synthetic problem for the largest planned SpiNNaker machine. Though
		tractable -- particularly given the relative output quality compared with
		the prior state-of-the-art -- the algorithm is unlikely to function
		comfortably as-is on larger problems.
		
		The conclusion to be drawn from this result, however, is not just that
		simulated annealing is a good solution for today's placement problems but
		that circuit placement techniques in general could be successfully adapted
		to fulfil this role. The placement problems faced by chip designers are
		growing at roughly the same exponential rate as the size of super computers
		but circuit designs hold the lead in terms of problem size. Consequently,
		as approaches are retired by chip placement researchers, they may find new
		life in the field of application placement.
		
	\section{Future research}
		
		Though the goals of this study have largely been met, there also remain
		some important limitations which future work may hope to address.
		Furthermore, this work has uncovered a number of new research areas
		warranting future enquiry. This section outlines a number of future lines
		of research.
		
		\subsection{Warehouse-scale cabling}
			
			In chapter~\ref{sec:building} I developed and implemented a number of
			cabling schemes for the SpiNNaker architecture spanning up to a six metre
			row of machine-room cabinets -- a relatively small installation by
			current standards. In SpiNNaker, the cabling exists in a 2D plane (i.e.
			across the faces of the cabinets) but as the system is scaled up, a
			single row of cabinets will tend towards a 1D line. Since embedding a 2D
			structure in a 1D space necessarily results in long connections, this
			cannot scale indefinitely.
			
			\begin{figure}
				\center
				\buildfig{figures/multi-row-cabling.tex}
				
				\caption{Multiple rows of interconnected cabinets.}
				\label{fig:multi-row-cabling}
			\end{figure}
			
			In conventional large-scale super computer installations, nodes are
			installed in rows of cabinets as illustrated in
			figure~\ref{fig:multi-row-cabling}.  From a `bird's-eye' view, the system
			approximates a 2D space, spread across the floor of a machine-room.
			Therefore, in principle, the folding and interleaving techniques
			described in chapter~\ref{sec:building} still apply. Unfortunately for
			SpiNNaker, cables connecting between rows of cabinets would be longer
			than the one metre limit imposed by its hardware because of the spacing
			between rows of cabinets.  Future SpiNNaker systems will need to consider
			alternative link technologies.  For example, a hybrid system could be
			used in which intra-cabinet connections continue to use the current HSS
			link technology while inter-cabinet links might use optical connections.
			This type of architecture could be supported by the use of pluggable
			`SFP+' transceiver modules~\cite{sff01}.
		
		\subsection{Cabling assistance for other architectures}
			
			A secondary result of the construction of prototype SpiNNaker systems in
			chapter~\ref{sec:building} was the use of real-time guidance and feedback
			to assist cable installation. I am not aware of this technique's use by
			existing architectures and, following the success experienced in this
			project, it is possible that the technique may also be useful in
			conventional systems.
			
			During the construction of prototype SpiNNaker machines, each cable took
			seconds to install compared with the minutes reported for existing
			systems~\cite{mudigonda11}. Part of this increase in efficiency appears
			to result from the immediate identification of mistakes made during
			cabling, saving time-consuming backtracking later on.
			
			In many real-world network installations, units are less densely packed
			than in SpiNNaker and so longer cables are often required. As a
			consequence, cabling errors may become more likely as cabling patterns
			are spread over a larger area making them more difficult to visually
			verify. Like SpiNNaker, conventional networking hardware is often
			equipped with a generous range of indicator LEDs and diagnostic
			facilities which might be used to implement real-time installation
			guidance. Future work could explore the use of this technique in the
			construction of other large-scale networks, such as data centres.
		
		\subsection{Congestion mitigation}
			
			\label{sec:wiggly-board-allocations}
			
			In chapter~\ref{sec:routing} I found that contiguous network faults cause
			hot-spots of congestion and routing table depletion where the PGS repair
			algorithm routed many paths around the edges of faults.  However, it is
			not just faults which can cause contiguous blockages in the network
			topology. In reality, researchers do not always require a full-sized
			SpiNNaker system to perform their experiments so large SpiNNaker systems
			are soft-partitioned on demand into many smaller
			machines~\cite{spalloc16}. To ensure isolation between partitioned
			sub-machines, HSS links between boards in different partitions are
			disabled. Because of SpiNNaker's `wrapped triple' partitioning scheme,
			the resulting sub-machines have hexagonal \emph{mesh} topologies (i.e.
			without wrap-around links) with irregular boundaries as in
			figure~\ref{fig:spalloc-mesh}.
			
			\begin{figure}
				\center
				\buildfig{figures/spalloc-mesh.tex}
				
				\caption[Irregular edges of a partitioned SpiNNaker system.]%
				{Irregular edges in a SpiNNaker system comprised of 24~boards
				partitioned from a larger machine.  Each hexagon represents a SpiNNaker
				chip. No wrap-around connections are present.}
				\label{fig:spalloc-mesh}
			\end{figure}
			
			In partitioned systems, the `tooth'-like gaps on the periphery of the
			network result in similar congestion to the HSS link failures considered
			in chapter~\ref{sec:routing}. When a route is generated between nodes on
			opposite sides of a gap, the PGS repair process will produce a
			shortest-path route around it. Since many routes may be blocked by a
			single gap, a hot-spot may develop around the corners of the gap.
			
			In chapter~\ref{sec:placement}, the `CConv' benchmark application was
			found to run correctly the majority of the time when placed by the
			simulated annealing algorithm but would occasionally fail by a
			significant margin. Preliminary experiments suggest these occasional
			failures are caused by placement solutions which place heavily
			communicating parts of the application on opposite sides of gaps along
			the perimeter of the network. Two possible approaches which future work
			may consider are presented below.
			
			\subsubsection{Avoiding hotspots with PGS repair}
				
				\label{sec:pgs-repair-improvements}	
				
				Network congestion around faults and network irregularities could be
				reduced by forcing the PGS repair process to take more varied routes
				around faults. For example, in circuit routing algorithms, routers
				avoid congestion by increasing the cost of routes which pass through
				congested areas~\cite{kahng11}. A similar technique could be used in
				PGS repair to spread the routes it produces.
				
				An alternative approach would be to adapt the base routing algorithms
				used prior to PGS repair to, for example, attempt alternative dimension
				order routes which may avoid blockages due to faulty links.
			
			\subsubsection{Fault and irregularity aware placement}
				
				One of the shortcomings of the simulated annealing based placer
				developed in chapter~\ref{sec:placement} is that it does not account
				for network faults, or irregularities, when estimating the cost of
				placement solutions.  Future work may exploit techniques used in
				congestion-aware circuit placement which could be adapted for
				application placement~\cite{viswanathan07}.
		
		\subsection{Reducing placement execution time}
			
			The simulated annealing based placer presented in
			chapter~\ref{sec:placement} produced good quality placements but its
			execution time limits its use beyond one million vertex placement
			problems. Future work should explore possibilities for improving the
			performance and scalability of this technique.
			
			In addition to considering alternative placement algorithms based on
			other methods, one possible approach is to attempt to reduce the execution
			time of simulated annealing based placement by shrinking the application
			graph being placed.
			
			For example, graph clustering~\cite{schaeffer07} may be used to group
			together strongly connected vertices which would then be placed as a
			single unit.  Unfortunately, clustering can suffer from the same problems
			as graph-partitioning-based placement: vertices may be grouped together
			in ways which, in practice, cannot be packed together into a given portion
			of a machine.  A possible solution to this problem is to use a two-phase
			placement approach~\cite{kahng11}. In a `global' placement phase,
			solutions are permitted which can slightly over-allocate resources but
			overall achieve good placement quality. In the `detailed' placement phase
			which follows, the solution is `legalised' by making small changes to the
			global placement to eliminate over allocation.
			
			An alternative approach suited to SpiNNaker could be to limit the
			clustering process to clusters which fit on a single SpiNNaker chip. In
			typical SpiNNaker application graphs, clustering to this level may reduce
			placement problem sizes by an order of magnitude and, consequently,
			reduce execution times by the same ratio. Preliminary experiments suggest
			that this approach might result in little placement quality loss for
			large placement problems whilst substantially reducing overall execution
			time.
		
		\subsection{Benchmarking}
			
			One of the most significant limitations of this study has been the
			unavailability of large-scale SpiNNaker applications for use as
			benchmarks. As a consequence, much of the scalability experimentation
			performed has relied on simple synthetic benchmarks based on projections
			of future application behaviour.
			
			In the short term, more sophisticated synthetic benchmark generation
			techniques used by the circuit placement community~\cite{nam07} may offer
			alternative benchmarks for future work. In the longer term, however, it
			is hoped that the availability of large SpiNNaker systems -- and
			placement and routing algorithms better suited to exploit them -- will
			lead to larger scale applications being developed. Hopefully these
			applications will lead to more interesting and representative benchmarks
			for use in future work.
	
	\section{Closing remarks}
		
		One of the primary outcomes of this work is that a number of the practical
		challenges faced in scaling up the SpiNNaker architecture have been
		addressed leading to the construction of large-scale SpiNNaker machines.
		The development of an effective placement algorithm for SpiNNaker
		applications has been shown to enable some neural simulations to exploit
		SpiNNaker's architecture for the first time. The availability of larger
		SpiNNaker machines paves the way for future large-scale neural modelling
		work built on much larger models such as Spaun, `the world's largest
		functional brain model'~\cite{eliasmith12}.
		
		Beyond the SpiNNaker project, the hexagonal torus topology has also been
		validated as a scalable and practical candidate for future network
		architectures. As super computers become ever larger, the physical
		scalability afforded by the 2D nature of the hexagonal torus topology may
		make it a compelling option. In addition, the finding that circuit
		placement techniques can be adapted to support placement of SpiNNaker
		software indicates that these algorithms may also be applicable to other
		applications. Indeed, if this is the case, circuit placement may offer a
		long-term source of placement algorithms able to handle the demands of
		future applications.
		
		% This thesis has explored and tackled a number of the challenges posed in
		% scaling up the unconventional SpiNNaker architecture. Along the way I have
		% demonstrated that the hexagonal torus topology may be a practical choice in
		% future applications which can scale up to the physical dimensions expected
		% of future super computers. I have also developed new efficient and
		% effective methods of placing and routing neural simulation software on
		% SpiNNaker which -- it is hoped -- will enable a new generation of large
		% scale neural simulations on spinnaker.
		
		Although this work stops short of demonstrating truly large-scale
		neuroscientific simulations running at the scale of newly completed
		SpiNNaker machines (largely because such simulations do not yet exist) a
		number of smaller-scale neural simulations have been made possible for the
		first time. The algorithms and techniques devised in this work have
		subsequently been incorporated into various software libraries and tools
		now being used by researchers building SpiNNaker applications, vindicating
		the efforts of this thesis (see appendix~\ref{sec:software}). A successor
		to the SpiNNaker architecture is also in the early stages of design and is
		building on experience of the existing architecture. The current intention
		is to retain the hexagonal torus topology used by SpiNNaker, a decision
		supported by the findings of this thesis.
		
		With SpiNNaker's hardware architecture now operating at scales close to its
		architectural limits, it is hoped that the contributions of this work will
		enable researchers to develop larger and more detailed neural models for
		this unique architecture.

	
	% Bibliography
	\bibliography{references}
	\bibliographystyle{alpha}
	
\end{document}

	
	
	\clearpage
	\listoffigures
	
	\clearpage
	\listoftables
	
	% Abstract
	{
	\prefacesection{Abstract}
	
	% Single line spacing for the abstract page
	\setstretch{1.0}
	
	
	\vfill
	
	% Standard thesis information
	\begin{center}
		\textsc{\large\thesistitle}
		
		\vspace{0.5em}
		
		\thesisauthor
		
		\vspace{0.5em}
		
		A thesis submitted to the University of Manchester\\
		for the degree of Doctor of Philosophy, 2016
	\end{center}
	
	\vfill
	
	% The abstract
	
	SpiNNaker is an unconventional super computer architecture designed to
	simulate up to one billion biologically realistic neurons in real-time. To
	achieve this goal, SpiNNaker employs a novel network architecture which poses
	a number of practical problems in scaling up from desktop prototypes to
	machine room filling installations.
	
	SpiNNaker's hexagonal torus network topology has received mostly theoretical
	treatment in the literature. This thesis tackles some of the challenges
	encountered when building `real-world' systems.  Firstly, a scheme is devised
	for physically laying out hexagonal torus topologies in machine rooms which
	avoids long cables; this is demonstrated on a half-million core SpiNNaker
	prototype.  Secondly, to improve the performance of existing routing
	algorithms, a more efficient process is proposed for finding (logically)
	short paths through hexagonal torus topologies. This is complemented by a
	formula which provides routing algorithms greater flexibility when finding
	paths, potentially resulting in a more balanced network utilisation.
	
	The scale of SpiNNaker's network and the models intended for it also present
	their own challenges. Placement and routing algorithms are developed which
	assign processes to nodes and generate paths through SpiNNaker's network.
	These algorithms reduce congestion and tolerate network faults. The proposed
	placement algorithm is inspired by techniques used in chip design and is
	shown to enable larger applications to run on SpiNNaker -- with good
	performance -- than the previous state-of-the-art. Likewise the routing
	algorithm developed is able to tolerate network faults, inevitably present in
	large scale systems, with little performance overhead.
	
	
	% Required to ensure single line spacing is used for this whole block
	\par%
}

	
	% Lay abstract
	{
	\prefacesection{Lay abstract}
	
	% Single line spacing for the lay abstract page
	\setstretch{1.0}
	
	
	\vfill
	
	% Standard thesis information
	\begin{center}
		\textsc{\large\thesistitle}
		
		\vspace{0.5em}
		
		\thesisauthor
		
		\vspace{0.5em}
		
		A thesis submitted to the University of Manchester\\
		for the degree of Doctor of Philosophy, 2016
	\end{center}
	
	\vfill
	
	% The lay abstract
	
	SpiNNaker is a super computer designed to simulate neural networks like those
	in the brain. Unlike biological experiments, computer simulations make it
	easy for neuroscientists to see what's going on in a neural network and
	easily test new theories. Though it is extremely unlikely that SpiNNaker will
	ever `think', it may give researchers a better understanding of how different
	parts of the brain work, and what makes them go wrong.
	
	Like most super computers, SpiNNaker is made up of a many smaller computers
	which all work together on the same problem. When completed SpiNNaker, will
	contain over one million computer processors, each responsible for simulating
	several hundred neurons at once.
	
	This thesis makes three contributions towards tackling the challenge of
	building a full-sized SpiNNaker machine. Firstly I devised a new way of
	organising the many computers that make up SpiNNaker such that only short
	cables are required to connect everything together, even in big machines.
	These techniques make SpiNNaker cheaper and easier to build.  Secondly, I
	developed a method that allows the individual computers in SpiNNaker to
	communicate reliably, even if some of the connections between them break --
	an unavoidable situation in super computers. Finally, I adapted a technique
	normally used to design computer chips to decide how to assign neurons to
	SpiNNaker's processors. This method keeps neurons which are connected to each
	other close together inside SpiNNaker. By doing this, the signals exchanged
	between neurons have to travel less distance through SpiNNaker's computer
	network and don't get in each other's way. This makes it possible to simulate
	bigger and more complex neural networks.
	
	% Required to ensure single line spacing is used for this whole block
	\par%
}

	
	% Declaration of non-submission elsewhere
	\prefacesection{Declaration}

% Single line spacing for the declaration
{
	\setstretch{1.0}
	No portion of the work referred to in this thesis has been submitted in support
	of an application for another degree or qualification of this or any other
	university or other institute of learning.
	
	\par%
}


	
	% University-prescribed copyright statement...
	\input{copyright}
	
	% Acknowledgements
	{
	\prefacesection{Acknowledgements}
	
	% Single line spacing
	\setstretch{1.0}
	
	It is often said that it is not \emph{what} you know but \emph{who} you know.
	Throughout the course of my PhD I've been exceptionally lucky to have been
	helped along by a great number of people.
	
	Both my supervisor, Jim Garside, and co-supervisor, Steve Furber, have each
	spent countless hours patiently discussing and describing all manner of
	things with me while giving me great freedom in my project. Jim's office door
	has always been open to my unexpected interruptions be it about work, writing
	or walking.  Likewise, Steve has always managed to find time for both
	technical and frivolous endeavours alike. I'm also hugely grateful to Luis
	Plana who has been a rich source of sage advice, insightful questions
	patiently suffered many a foolish question.
	
	Various parts of the work in this thesis (and numerous side projects) would
	not have been possible if not for the multitude of discussions,
	collaborations and even sheer physical hard work of Steve Temple, Javier
	Navaridas, Simon Davidson and Dave Clark. I'm also indebted to Andrew Mundy
	and Jamie Knight, both of whom have donated so much time and effort towards
	verifying and using software implementations of the ideas in this thesis.
	
	The injection of lunchtime silliness by Andrew and Jamie, along with Amanieu
	d'Antras and Andrew Webb and the other CDT members has always brightened my
	day. So to has the friendly and stimulating environment of the School of
	Computer Science and its many staff and students. Of course, I am also very
	grateful for the funding the school has provided for my research.
	
	I cannot thank my wonderful wife, Ann-Marie, enough for being by my side. She
	has given me so much kindness, love and patience and endured a lifetime's
	quota of conversations about hexagons. Finally, thanks too to rest of my
	family, especially my parents, who are to blame for starting me down this
	path and co-suffering drafts and endless rants about this document.
	
	% Required to ensure single line spacing is used for this whole block
	\par%
}

	
	% Make chapter pages start on the right-hand side
	\makeatletter\@openrighttrue\makeatother
	
	% Main body
	\chapter{Introduction}

\label{sec:introduction}

%Problem area
%
%* Network construction and exploitation
%  * Cabling: Build it cheaply in terms of tech cost and install cost
%  * Routing: Get around it cheaply and reliably
%  * Placement: Use it efficiently

The Spiking Neural Network Architecture (SpiNNaker) is a novel super computer
architecture designed to simulate biologically realistic models of brains in
real time \cite{furber07}. Though neurons, the building blocks of the brain,
are relatively well understood, their complex interactions remain mysterious.
Just as understanding the workings of a transistor is insufficient to
understand a modern microprocessor, neuroscientists believe that understanding
the neurons in isolation cannot explain the brain and that understanding their
connectivity is key \cite{eliasmith13,eliasmith14}. Experiments on real brains,
however, are fraught with difficulty. Variations between individuals can be
significant and it is only possible to record tens or hundreds of the trillions
of signals in the brain, and even then only with limited control over which
signals are recorded. Computer simulations of models of large neural networks,
however, enable researchers to develop repeatable experiments and gain complete
visibility of any signal and any neuron. Models such as SPAUN
\cite{eliasmith12}, built from millions of simulated neurons, have shown great
promise in expanding our understanding of higher level brain functions such as
memory and simple problem solving.  Unfortunately these neural models are
expensive to simulate, requiring hours of compute time to simulate each second
of neural activity. As well as being inconvenient, this precludes the use of
robotics to immerse these models in real world environments and also limits
studies of long-term behaviours such as learning.

SpiNNaker is designed to enable the real time simulation of models containing
up to one billion neurons -- approximately \SI{1}{\percent} of a human brain or
ten mouse brains \cite{furber06}. To achieve this goal, the largest planned
SpiNNaker machine will contain over one million low-powered computer processors
interconnected by a bespoke network architecture.

SpiNNaker's large processor count matches the current trend in super computers
where processor counts are growing exponentially \cite{meuer16j}, mimicking the
growth of the number of components in the processors themselves predicted by
Gordon Moore's famous `law' \cite{moore75}. As a result of this growth, the
interconnection networks which enable these processors to work together have
grown in importance \cite{dally04}.  Network designers must carefully balance
performance against practicality and financial cost.  SpiNNaker's network is no
exception to this rule and, as the systems scale up from desktop prototypes to
machine-room scale installations, the reality of building and exploiting these
machines presents an array of challenges.

As in all super computers, SpiNNaker's network interconnects its processors in
a particular network topology which defines how different processors may
communicate with each other. Unlike the tree and $N$-dimensional torus
topologies found in contemporary super computers \cite{dally04}, SpiNNaker
employs a `hexagonal torus topology'. In this topology, nodes in SpiNNaker's
network fit together in a honeycomb-like pattern where messages may `hop' from
node to node to reach their destination. As we will see in
chapter~\ref{sec:background}, the hexagonal torus topology, in theory, sits at
a `sweet spot' in terms of network performance and practicality. As the first
known large-scale installation of the hexagonal torus topology, however, there
remain a number of practical challenges for large spinnaker machines arising
from this choice.

As super computer networks have grown in scale to millions of processors the
task of dealing with previously rare faults has grown.  Though fault rates in
networks remain consistently low, architectures such as SpiNNaker may have
hundreds of thousands of links meaning even fault rates of a fraction of a
percent will impact tens or hundreds of links. To enable reliable operation,
networks must be able to adapt the routes taken by messages through the network
to avoid faulty links and nodes. The techniques employed are often closely tied
to a particular network architecture and consequently SpiNNaker's novel network
architecture demands its own approach.

Another challenge introduced by the growing scale of super computers is making
\emph{efficient} use of network resources. Communicating processes should be
located on logically `nearby' nodes to reduce network load. The neural models
for which SpiNNaker is designed are often described abstractly, rather than
geometrically, using modelling languages such as PyNN~\cite{davison08} and
Nengo~\cite{eliasmith04}.  Because of this, the communication requirements of
simulations can be highly irregular making an efficient placement of processes
onto processors in the machine non-trivial.

%Contributions
%
%* Cabling scheme for hexagonal toruses without long cables
%* Efficient installation technique for dense systems
%* Exhaustive and efficient route calculation in hex toruses
%* Fault tolerant routing scheme exploiting SpiNNaker's odd router
%* Placement based on SA a: works very well and b: suggests circuit placement is
%  a good source of inspiration.

This thesis addresses the practical challenges of scaling up the SpiNNaker
architecture in a real-world setting summarised by these research questions:

\begin{enumerate}
	
	\item Can the hexagonal torus topology be deployed and used in real, large
	scale systems?
	
	\item Does SpiNNaker's router architecture help, or hinder fault tolerance?
	
	\item How can the parts of a neural simulation be placed onto a large
	hexagonal torus topology to reduce network load?
	
\end{enumerate}

%Structure
%
%* Chapter 2: Background: detailed dive into what's in SpiNNaker, why its
%  really so unusual. Also looks at what applications run on SpiNNaker and how
%  they work.
%* Chapter 3: How to build a really big SpiNNaker machine.
%* Chapter 4: How to find your way around that machine.
%* Chapter 5: How to find your way around that machine even when its broken.
%* Chapter 6: Now you can walk, time to run.
%* Chapter 7: Wrapping up.
%* Appendices: Hard-to-come-by theoretical and practical details useful if
%  you're about to continue where this research left off but be useful but
%  otherwise hard to come by, especially in one place.

Chapter~\ref{sec:background} introduces the SpiNNaker architecture and, in
particular, describes its hexagonal torus topology and network architecture.

In chapter~\ref{sec:building}, I develop a cabling scheme for large hexagonal
torus topologies which enables arbitrarily large networks to be constructed
using only short, inexpensive cables. This theoretical work is then evaluated
through the construction of a range of prototype SpiNNaker systems. The largest
of these prototypes contains over half a million processor cores and spans
several machine room cabinets. In addition, I propose the use of built-in
diagnostic facilities to assist technicians performing network installation and
maintenance. This technique is found to greatly reduce the effort required and
the number of mistakes made.

In chapters~\ref{sec:shortestPaths}~and~\ref{sec:routing} I develop new routing
techniques for SpiNNaker's network. Chapter~\ref{sec:shortestPaths} develops a
new approach to finding the shortest paths through hexagonal torus topologies,
an integral part of many routing algorithms. This newly proposed approach is
cheaper to compute than the state of the art and, unlike previous efforts, is
able to discover all valid short paths through the topology. This theoretical
advance brings hexagonal torus topologies in line with conventional topologies
by providing routing algorithms with complete information about the paths
available to them. In chapter \ref{sec:routing} I propose a fault tolerant
routing algorithm for SpiNNaker which is able to avoid arbitrary static fault
patterns with minimal performance overhead. A key finding of this chapter is
that the flexibility afforded to fault tolerant routing algorithms by
SpiNNaker's unconventional router architecture is what facilities the low
overheads reported in this chapter.

Finally, in chapter~\ref{sec:placement}, I explore the problem of application
placement in SpiNNaker's network. As in other networks and applications, neural
simulations should be arranged such that communication occurs primarily between
processors close together in the network to control network load. Due to the
irregular connectivity and large scale of the neural models expected to run on
SpiNNaker, an automated approach is necessary. I develop a novel placement
algorithm based on algorithms used for circuit layout in computer chips. My
algorithm is found to allow some larger neural models to run on SpiNNaker for
the first time while enabling other applications to run at greater speeds. In
addition, synthetic benchmarks containing over one million processes indicate
that this algorithm should handle the anticipated demands of the neural models
expected to run on large-scale SpiNNaker installations.

	\chapter{The SpiNNaker Architecture}
	
	\label{sec:background}
	
	SpiNNaker is a massively parallel computer architecture designed to simulate
	biologically realistic neural models \cite{furber07}. In this chapter we will
	explore this unconventional architecture in detail, starting with its purpose
	before focusing on its most unconventional feature: its network.
	
	% * Purpose
	%   * Spiking neural simulations
	%     * Neural modelling: PyNN, Nengo...
	%     * Parallelisation + communication
	
	\section{Neural simulation}
		
		Human brains contain billions of neurons connected together by trillions of
		`synapses'. Neurons communicate by transmitting and receiving `spikes'
		through their synapses. Each spike is `valueless' in that a spike's only
		significant features are when it arrives and where it has come from.
		
		\begin{figure}
			\center
			\buildfig{figures/lif-neuron.tex}
			
			\caption{A Leaky Integrate-and-Fire (LIF) neuron.}
			\label{fig:lif-neuron}
		\end{figure}
		
		Though some detailed models of the electrochemical processes occurring
		inside neurons are computationally intensive, simplified models such as the
		Leaky Integrate-and-Fire (LIF) model can be implemented in just a handful
		of CPU instructions \cite{vainbrand11}. Figure~\ref{fig:lif-neuron}
		illustrates a simple LIF neuron in which incoming spikes cause charge to
		build up (integrated) which over time, leaks away. If an incoming spike
		causes the charge to rise above a certain threshold, the neuron `fires'
		producing an outgoing spike. Despite the simplicity of this model, large
		neural networks such as Spaun \cite{eliasmith12} -- built entirely from LIF
		neurons -- exhibit complex behaviours such as fine motor control and
		problem solving.
		
		The computational expense of large scale neural simulations does not arise
		from the cost of modelling neurons but instead from distributing spikes. In
		biology, neurons produce spikes at an average rate of \SI{10}{\hertz} and
		synapses connect each neuron's output to (order) \num{1000}~neurons
		\cite{navaridas09}. Consider an example neural model with $7\times10^7$
		neurons, approximately the number in a house mouse and
		$\nicefrac{1}{10}^\textrm{th}$ of the design target of SpiNNaker. This
		network might produce $7\times10^8$~spikes per second. Because each neuron
		connects to many others, this equates to $7\times10^{11}$ spikes being
		received per second. If each spike were transmitted as a UDP datagram
		containing a single \SI{32}{\bit} payload, the total network throughput
		required for this simulation would be \SI{179.2}{\tera\bit\per\second}. At
		the time of writing, this is more than double the bisection bandwidth (the
		theoretical worst-case throughput) of the world's most powerful super
		computer \cite{dongarra16}.
	
	\section{Network architecture}
		
		Architectures such as IBM's Blue Gene \cite{chiu11} and Cray's XK7
		\cite{ornl16} employ powerful compute nodes connected together using
		networks designed to transfer multi-kilobyte blocks of data between nodes.
		Since neural models have relatively light computational requirements and
		communications are based on small pieces of data (spikes), this type of
		architecture is poorly suited to the task.
		
		SpiNNaker's architectural target is to support realtime simulations of up
		to one billion neurons. Since neural models such as LIF are inexpensive to
		model and many neurons can be simulated independently in parallel,
		SpiNNaker employs many small, energy efficient ARM processors
		\cite{furber07}. To support the unusual communication requirements of
		neural simulations, a bespoke interconnection network is used which is the
		background to this thesis.
		
	%   * SpiNNaker chip
	%     * Cores
	%     * SDRAM
	%     * NoC
	%     * Router
		
		\begin{figure}
			\center
			%\includegraphics[width=19mm]{figures/spinnakerChip.jpg}
			\buildfig{figures/hex-chips.tex}
			
			\caption[SpiNNaker chips connected to their six neighbours.]%
			{SpiNNaker chips (actual size) connected to their six neighbours.}
			\label{fig:spinnakerChip}
		\end{figure}
		
		The fundamental building block of the SpiNNaker architecture is the
		SpiNNaker chip (figure \ref{fig:spinnakerChip}) \cite{furber13}. Each chip
		contains eighteen low power ARM 968 processor cores each capable of
		simulating between \num{200} and \num{2000} LIF neurons in real time
		\cite{mundy15}.  Each core has a total of \SI{96}{\kilo\byte} of private
		Tightly-Coupled Memory (TCM) and shares access to \SI{128}{\mega\byte} of
		on-chip SDRAM with other cores on the same chip. Finally, each chip
		contains a programmable router which routes network packets to and from the
		local cores and six neighbouring SpiNNaker chips. SpiNNaker machines are
		constructed by combining many SpiNNaker chips.
		
		\begin{figure}
			\center
			\buildfig{figures/spinnaker-packet.tex}
			
			\caption{SpiNNaker's \SI{40}{\bit} and \SI{72}{\bit} multicast packet
			format.}
			\label{fig:spinnaker-packet}
		\end{figure}
		
		Processor cores can communicate by sending and receiving network packets
		forwarded by routers through the network. Since SpiNNaker's network is
		designed to transmit neural spike events efficiently, individual network
		packets are small, either \SI{40}{\bit} or \SI{72}{\bit} compared with tens
		or hundreds of byte packets in typical network architectures.
		
		In a real-time simulation, the time at which a spike is produced is
		implicitly indicated by the time it is received -- since at biological
		timescales a computer network delivers packets `instantaneously'.
		Consequently, the only information which must be explicitly encoded is the
		identity of the neuron which produced the spike. In SpiNNaker, a spike may
		be encoded by using a single \SI{40}{\bit} `multicast packet' whose format
		is illustrated in figure~\ref{fig:spinnaker-packet}.  The \SI{8}{\bit}
		header is used by SpiNNaker's routers to determine the type of packet and
		the \SI{32}{\bit} `routing key' is used to identify the neuron which
		produced the packet. The routing key is also used by SpiNNaker's routers to
		determine how the packet should be directed through the network.
		
		The optional \SI{32}{\bit} payload is not used by conventional spiking
		neural simulations \cite{galluppi10} but has been exploited to enable more
		efficient simulation of a particular class of neural models \cite{mundy15}.
	
	\section{The SpiNNaker router}
		
		The SpiNNaker router employs an unconventional design which, despite its
		compact size and small energy requirements, implements a flexible multicast
		routing scheme. Unlike conventional routers which often employ hard-coded
		routing rules \cite[chapter~8]{dally04}, the SpiNNaker router uses a
		programmable `routing table' to determine how packets should be forwarded.
		In addition, to avoid deadlocks, SpiNNaker's router employs a simple,
		timeout-based mechanism which exploits the ability of neural networks to
		tolerate occasional missing packets. As we will see in chapter
		\ref{sec:routing}, this mechanism greatly simplifies the task of routing in
		SpiNNaker's network. In this section we'll look at these features in
		greater detail.
		
		\subsection{Routing tables}
		
			When a multicast packet arrives at a SpiNNaker router (either from a
			local core or a neighbouring chip), the router looks up the routing key
			in its routing table. This table consists of \num{1024} programmable
			table entries, each specifying a routing key bit pattern and mask to
			match and a set of routes.  When a multicast packet's key is matched by a
			routing entry the packet is forwarded along every route specified by that
			entry, potentially duplicating the packet. This `multicast' technique
			allows packets to be transmitted once but received in a number of places
			while making efficient use of the network \cite{navaridas12}.
			
			Though routing table entries are in finite supply (\num{1024} entries per
			router), it is still possible for many thousands of traffic flows to be
			routed through a single router. The bit pattern and mask in each routing
			entry matches against the 32~bits of a routing key as either
			`\texttt{1}', `\texttt{0}' or `\texttt{X}' (don't care).  This means that
			a single routing entry may, for example, be used to match all routing
			keys with a certain prefix. If a routing key is not matched by any entry
			in the routing table then the packet is `default routed' in a straight
			line. For example if a packet with an unmatched key is received from the
			chip to the left, the packet will be default routed to the chip on the
			right. By assigning routing keys such that neurons whose spikes are sent
			to similar destinations share a similar prefix, the number of routing
			entries required by a simulation is greatly reduced \cite{davies12}.
			
			\begin{figure}
				\center
				\buildfig{figures/routing-example.tex}
				
				\caption[Multicast routing example.]%
				{Multicast routing example with \SI{4}{\bit} routing keys. Each
				box represents a SpiNNaker chip whose router has been programmed with
				the routing entries shown. Grey lines mark connections between chips.}
				\label{fig:routing-example}
			\end{figure}
			
			Consider the simplified example in figure~\ref{fig:routing-example} in
			which a number of (\SI{4}{\bit}) routing table entries have been
			configured in the routers of a small SpiNNaker network. If a packet with
			the routing key \texttt{1011} is transmitted by a core in the chip
			labelled $(0, 0, 0)$, this will match the first routing table entry on
			that chip and will be routed to chip $(1, 0, 0)$. On chip $(1, 0, 0)$,
			the packet once again matches the first routing entry and is routed to
			chip $(1, 0, -1)$. On $(1, 0, -1)$, no match is made so the packet is
			default routed to $(1, 0, -2)$. On this chip, the packet matches a
			routing entry which routes the packet to core~7. In this example, default
			routing allows only three routing table entries to direct a packet
			through four chips.
			
			As a second example, if a packet with the routing key \texttt{0010} is
			transmitted by a core on chip $(0, 0, 0)$, this key will be matched by
			the second routing entry since \texttt{X}s in the table entry will match
			both \texttt{1}s and \texttt{0}s in the corresponding bits of the routing
			key. When the packet arrives at chip $(0, 0, -1)$ the matching routing
			entry forwards the packet to both $(0, 1, -1)$ and $(1, 0, -1)$
			simultaneously. The copy of the packet arriving at $(0, 1, -1)$ is routed
			to core~5 on that chip.  Meanwhile, the copy forwarded to $(1, 0, -1)$ is
			duplicated again with one copy being routed to core~11 and another being
			routed to chip $(1, 0, -2)$. Here the packet is finally delivered to
			core~6. In this example, the ability of the router to multicast
			(duplicate) packets as they pass through the network meant that sending
			one copy of the packet was sufficient to reach three destination cores.
			In addition, by using \texttt{X}s in the routing table entry, the same
			routing entries are sufficient to route packets with the keys
			\texttt{0000}, \texttt{0001}, \texttt{0010} and \texttt{0011}.
			
			In spite of these mechanisms, it is still possible for an application to
			run out of routing table entries. As we will see in
			chapter~\ref{sec:placement} by arranging applications appropriately
			within SpiNNaker's network, routing table usage can be reduced. In
			addition, other behaviours of SpiNNaker's router may be exploited to
			compress an applications routing tables further, however the techniques
			employed are beyond the scope of this thesis \cite{mundy16}.
		
		\subsection{Timeouts}
			
			SpiNNaker's router is built on a pipeline architecture. As shown in
			figure~\ref{fig:router-architecture}, the router is fed packets by an
			arbiter which serialises packets arriving from other chips and local
			cores. Every (\SI{100}{\mega\hertz}) clock cycle, the router pipeline
			accepts one packet from the arbiter and routes a packet to one or several
			output links. If any of the required output ports are busy then the
			packet is not forwarded to any output link and the pipeline stalls. Once
			a packet has been blocked for a programmable timeout, it is dropped
			(discarded) and routing continues as usual for next packet in the
			pipeline. Links become blocked while transmitting packets or waiting for
			the remote receiver to become ready. For example, a receiving processor
			core may be busy performing some computation or a receiving router may be
			blocked waiting for some of its outputs to become ready.
			
			\begin{figure}
				\center
				\buildfig{figures/router-architecture.tex}
				
				\caption{SpiNNaker router architecture}
				\label{fig:router-architecture}
			\end{figure}
			
			The timeout-based packet dropping mechanism is designed to defuse
			deadlocks in the network. For example, if two routers are trying to send
			each other a packet at the same time they may become deadlocked, each
			waiting for the other router to accept a packet before continuing.
			SpiNNaker's timeout mechanism breaks deadlocks by dropping packets which
			have been blocked for some time and therefore may be in a deadlock.  Once
			a packet has been dropped it is left to software to either tolerate the
			missing packet or trigger a retransmission. In neural simulations, as in
			biology, the loss of a single spike is unlikely to have a significant
			impact on the behaviour of a neural model and therefore these simulations
			are inherently tolerant of occasional dropped packets. During application
			loading and other system tasks, a higher level, software driven protocol
			based on acknowledgements and retransmissions is used to ensure
			guaranteed delivery.
			
			% TODO: MENTION TIMEOUT VALUE USED?
			% Router timeouts must be configured to be long enough that delays in
			% packet transmission, for example due to the time taken for packets to
			% traverse a link, do not trigger packet dropping. Conversely, the timeout
			% should be as short as possible to reduce the time the router is
			% blocked and maximise network throughput.
	
	\section{The hexagonal torus topology}
		
		Each SpiNNaker chip is a node in a `hexagonal torus topology' as
		illustrated in figure~\ref{fig:hexagonalTorusTopology}. Network packets
		sent by SpiNNaker's processor cores may `hop' through several nodes in the
		network to reach their intended destination. In each hop, a packet may
		advance one node along one of the three axes of the topology. For example,
		a packet sent by the node labelled $\alpha$ (in the bottom-left corner) to
		the node labelled $\beta$, might take the following sequence of hops:
		X$^+$, X$^+$, Z$^-$. Packets sent from $\alpha$ to $\gamma$ might take the
		route: X$^-$, X$^-$, Y$^+$, Y$^+$. The first hop of this route `wraps
		around' from the bottom-left node to the bottom-right node in a single hop.
		
		\begin{figure}
			\center
			\buildfig{figures/hexagonalTorusTopology.tex}
			
			\caption[A hexagonal torus topology.]%
			{A hexagonal torus topology. Each hexagon represents a node (i.e.
			a SpiNNaker chip). Touching nodes are directly connected. Nodes on edges
			$a$, $b$ and $c$ are also directly connected to the corresponding nodes
			on edges $a'$, $b'$ and $c'$, respectively. The three axes of the
			hexagonal torus topology, `X', `Y' and `Z' are also shown.}
			\label{fig:hexagonalTorusTopology}
		\end{figure}
		
		\begin{figure}
			\center
			\begin{subfigure}{0.39\linewidth}
				\center
				\includegraphics[width=\linewidth]{figures/torus-3d-flat.pdf}
				\caption{}
				\label{fig:torus-3d-flat}
			\end{subfigure}
			~~
			\begin{subfigure}{0.26\linewidth}
				\center
				\includegraphics[width=\linewidth]{figures/torus-3d-tube.pdf}
				\caption{}
				\label{fig:torus-3d-tube}
			\end{subfigure}
			~~
			\begin{subfigure}{0.23\linewidth}
				\center
				\includegraphics[width=\linewidth]{figures/torus-3d-torus.pdf}
				\caption{}
				\label{fig:torus-3d-torus}
			\end{subfigure}
			
			\caption{Visualisation of a hexagonal torus topology as a torus.}
			\label{fig:torus-3d}
		\end{figure}
		
		The wrap around connections in the topology are what give it the `torus'
		part of its name. Figure~\ref{fig:torus-3d-flat} shows a hexagonal torus
		topology drawn flat as in the previous figure. If the topology is rolled up
		into a tube such that the top and bottom nodes become directly adjacent, a
		tube is formed as in figure~\ref{fig:torus-3d-tube}. This tube can then be
		bent to bring together the nodes at the ends of the tube to form a torus as
		shown in figure~\ref{fig:torus-3d-torus}.
		
		A hexagonal torus topology is typically defined in terms of its width and
		height along the X and Y axes respectively. For example,
		figure~\ref{fig:hexagonalTorusTopology} shows a $10\times10$ hexagonal
		torus.  The nodes in a hexagonal torus topology are addressed using
		hexagonal coordinates of the form $(x, y, z)$ \cite{patel15}. The bottom
		left node (labelled $\alpha$ in the figure) has the coordinate $(0, 0, 0)$
		and other nodes are assigned coordinates according to the number of hops
		along each dimension from $(0, 0, 0)$, for example node $\beta$ has the
		coordinate $(2, 0, -1)$. Because the hexagonal torus topology's axes are
		non-orthogonal, it is possible to define several coordinates for the same
		location. For example $(3, 1, 0)$ and $(1, -1, -2)$ are also valid
		coordinates for node $\beta$. These dual coordinates emerge from the fact
		that adding $(1, 1, 1)$ to a coordinate produces an equivalent, but
		different, coordinate. This phenomenon is explained in detail in
		appendix~\ref{app:minimal-hex-coordinates} and related phenomena will be
		discussed in chapter~\ref{sec:shortestPaths}.
		
		The hexagonal torus topology was chosen over a more conventional network
		topology -- such as a 2D or 3D torus (sometimes known as a 2-ary $N$-cube
		or 3-ary $N$-cube respectively) \cite[chapters~3~and~5]{dally04} -- due to
		its balance of theoretical performance and practicality. The bisection
		bandwidth of a topology indicates the theoretical worst-case total
		throughput the network is able to sustain \cite[chapter~1]{dally04}.  In
		networks with homogeneous link throughput, bisection bandwidth is
		determined by the number of links cut by a balanced bisection of the
		network.  Figure~\ref{fig:bisection-bandwidth} illustrates the bisections
		of several torus topologies.
		
		\begin{figure}
			\center
			\begin{subfigure}[b]{0.3\linewidth}
				\center
				\buildfig{figures/bisection-bandwidth-2d.tex}
				
				\caption{2D Torus}
				\label{fig:bisection-bandwidth-2d}
			\end{subfigure}
			\begin{subfigure}[b]{0.3\linewidth}
				\center
				\buildfig{figures/bisection-bandwidth-hex.tex}
				
				\caption{Hexagonal Torus}
				\label{fig:bisection-bandwidth-hex}
			\end{subfigure}
			\begin{subfigure}[b]{0.3\linewidth}
				\center
				\buildfig{figures/bisection-bandwidth-3d.tex}
				
				\caption{3D Torus}
				\label{fig:bisection-bandwidth-3d}
			\end{subfigure}
			
			\caption[Bisections of torus topologies.]%
			{Bisections of torus topologies. Connections cut by the bisection
			are drawn as lines.}
			\label{fig:bisection-bandwidth}
		\end{figure}
		
		In a $N \times N$ 2D torus topology, the bisection bandwidth is $2N$~links
		and each node requires four links. The hexagonal torus topology requires
		six links per node but provides double bisection bandwidth ($4N$~links).
		The 3D torus topology also requires six links per node but by connecting
		the nodes differently achieves a bisection bandwidth of $8N$~links.  The 3D
		torus topology, however, comes at a price -- when immersed into the
		(approximately) 2D space provided by a large machine room or row of server
		cabinets, some connections require long cables. By contrast, the 2D and
		hexagonal torus topologies are both inherently two dimensional and
		consequently do not suffer from this effect. The hexagonal torus topology,
		therefore, shares the practicality of construction of a 2D torus while
		still gaining some of the performance of a 3D torus topology. In addition,
		because nodes in a hexagonal torus topology have a greater number of links,
		greater redundancy is available in the network to tolerate faults.
		
		Most torus topologies, including hexagonal, 2D and 3D toruses, have a
		related `mesh' topology. These mesh topologies maintain the same general
		connectivity structure as their torus topologies but omit wrap-around
		links. In practice, this saves a small number of links at the expense of
		halving the network's bisection bandwidth.  Because of their poorer
		performance, mesh networks are rarely used as the basis of a network
		architecture. Mesh networks, however, are occasionally formed when a
		network is partitioned into several smaller sub-networks to allow multiple
		users to share a system \cite{spalloc16}.
		
		\begin{figure}
			\center
			\begin{subfigure}[b]{0.45\linewidth}
				\center
				\buildfig{figures/hexagonal-torus.tex}
				\caption{Hexagonal torus}
				\label{fig:topo-compare-hexagonal-torus}
			\end{subfigure}
			\begin{subfigure}[b]{0.45\linewidth}
				\center
				\buildfig{figures/h-torus.tex}
				\caption{H-torus}
				\label{fig:topo-compare-h-torus}
			\end{subfigure}
			
			\caption[Hexagonal torus vs. H-torus topology.]%
			{Hexagonal torus vs. H-torus topology. Each numbered hexagon
			represents a node. The thick outline indicates the bounds of the
			topology after which the network repeats. In each topology, the path
			taken by advancing in the Y$^+$ direction from the node labelled `0' is
			shown.}
			\label{fig:topo-compare}
		\end{figure}
		
		\label{sec:hex-vs-h-torus}
		
		The hexagonal torus topology is not to be confused with the `H-torus'
		topology. This topology also uses a hexagonal tiling of nodes and even
		wraps this tiling into a torus-like topology \cite{zhao08}. However,
		H-torus topologies have very different characteristics to the hexagonal
		torus topology and are related to `twisted torus' topologies
		\cite{camara10}. For example, figure~\ref{fig:topo-compare} illustrates one
		major difference in the way paths wrap around the peripheries of both
		topologies.
	
	\section{Scaling-up SpiNNaker machines}
		
		To build large SpiNNaker systems comprising of tens of thousands of
		SpiNNaker chips, groups of forty-eight chips are mounted onto circuit
		boards as illustrated in figure~\ref{fig:spinnakerBoard}. These boards may
		be connected together to form larger systems.  Figure~\ref{fig:threeboard}
		shows a prototype three board system. Though the chips are
		\emph{physically} arranged in a (nearly) $7\times7$ grid on each SpiNNaker
		board, they logically form a hexagonal `wrapped triple'
		\cite{davidsonWiring} (see appendix~\ref{sec:partitioning}) which logically
		fit together as illustrated in figure~\ref{fig:threeboard-separate}. The
		labelled exposed corners of the three forty-eight chip boards connect
		together to form a $12\times12$ hexagonal torus topology as illustrated in
		figure~\ref{fig:threeboard-wrapped}. Larger SpiNNaker machines are
		assembled by combining more boards.
		
		\begin{figure}
			\center
			\begin{subfigure}[b]{0.45\linewidth}
				\center
				\includegraphics[width=\linewidth]{figures/spinnakerBoard.jpg}
				
				\caption{A SpiNNaker board}
				\label{fig:spinnakerBoard}
			\end{subfigure}
			~~~
			\begin{subfigure}[b]{0.45\linewidth}
				\center
				\includegraphics[width=\linewidth]{figures/threeboard.jpg}
				
				\caption{Three board prototype}
				\label{fig:threeboard}
			\end{subfigure}
			
			\vspace*{1em}
			
			\begin{subfigure}[b]{0.45\linewidth}
				\center
				\buildfig{figures/threeboard-separate.tex}
				
				\caption{Three board topology}
				\label{fig:threeboard-separate}
			\end{subfigure}
			~~~
			\begin{subfigure}[b]{0.45\linewidth}
				\center
				\buildfig{figures/threeboard-wrapped.tex}
				
				\caption{\ldots{}as a parallelogram}
				\label{fig:threeboard-wrapped}
			\end{subfigure}
			
			\caption{SpiNNaker boards and their topology.}
			\label{fig:spinnaker-boards}
		\end{figure}
		
		
		SpiNNaker chips on the same circuit board connect using low power links
		requiring sixteen wires each.  If this link technology were used to connect
		chips on neighbouring boards, each pair of boards would need to be
		connected with a 128~wire cable.  Cables and connectors supporting this
		many signals are expensive, unreliable and physically large. Instead,
		chip-to-chip connections between boards are multiplexed and demultiplexed
		onto a single High-Speed Serial (HSS) link \cite{athavale05} carried via
		commodity S-ATA cables which are often used to connect hard disks in
		desktop computers and servers \cite{sata3spec}. The six high-speed links
		are implemented by three onboard FPGAs (the three large chips at the top of
		the SpiNNaker board) and are logically transparent to the underlying
		network. The underlying technology and the choice of S-ATA cables limits
		each board-to-board connection to spanning at most one metre gaps. In
		chapter~\ref{sec:building} I present a cabling scheme for hexagonal torus
		topologies which enable large SpiNNaker systems to be assembled using only
		short cables between boards.
		
	\section{Conclusions}
		
		The SpiNNaker architecture has been designed to enable the simulation of
		large biologically realistic neural models in real time. To support this,
		its network architecture takes on an unconventional design based on a
		custom router and hexagonal torus topology. In the remainder of this
		thesis, I will tackle a number of the challenges in scaling up the
		SpiNNaker architecture outlined in this chapter.

	\chapter{Building large SpiNNaker machines}
	
	Like any super computer, physically putting together a large SpiNNaker
	machine poses many challenges in terms of organisation, assembly and
	maintainance. One of the key tasks in this process is the installation of
	network cables such that a desired overall network topology is constructed.
	The largest planned SpiNNaker machine will use \num{3600} S-ATA
	\cite{sata3spec} cables to interconnect its \num{1200} circuit boards,
	creating a hexagonal torus topology. Since the machine will be installed
	within standard server room cabinets (which are not available in a
	giant-doughnut form-factor) a mapping from a board's logical location in the
	network topology to its physical location must be constructed. In addition,
	the interconnect technology employed by SpiNNaker restricts the length of
	S-ATA cables used to $\le$~\SI{1}{\meter}, constraining the possible mappings
	used. In addition the practical issues of installation complexity and
	maintainance must be considered since all \num{3600} cables must ultimately
	be installed and maintained by human operators.
	
	In this chapter I describe a novel technique for physically laying out
	machines configured in hexagonal torus topologies, such as SpiNNaker, in
	commercial machine rooms, building on the techniques used in more
	conventional torus topologies. In addition, I also propose a new methodology
	for installing and maintaining super computer cabling which which exploits
	existing diagnostic features of the SpiNNaker hardware to interactively guide
	and validate cable installation. Finally, I demonstrate how these new
	techniques have been used successfully to interconnect a prototype
	\num{518400} core SpiNNaker machine in substantially less time than the
	industry norm.
	
	In this chapter, the term \emph{unit} refers to the smallest physical
	ecomponent between which connections connections are to be made. For example,
	in a SpiNNaker machine a unit is a 48-chip board while in data center, a unit
	might be a server blade.
	
	\section{Related work}
		
		In this section I describe the techniques conventionally employed when
		laying out and interconnecting the units within super computers. Due to
		SpiNNaker's hexagonal torus topology and dense physical packing of units,
		these existing techniques are found to be insufficient. In the remainder of
		the chapter we will explore solutions to the limitations exposed below.
		
		\subsection{Avoiding long cables}
			
			Na\"ive arrangements of torus topologies, including hexagonal torus
			topologies, feature long `wrap-around' connections which connect units at
			the peripheries of the system. These connections can be problematic for
			numerous reasons:
			
			\begin{description}
				
				\item[Performance] Signal quality diminishes as cables get longer,
				requiring the use of slower signalling speeds, increased error
				correction overhead or more complex hardware.
				
				\item[Energy] Longer cables require higher drive strengths and/or
				buffering to maintain signal integrity.
				
				\item[Cost] Cost Shorter cables are cheaper than long ones.  Longer
				cables imply more wire in a given space making the tasks of routing or
				cable installation more difficult increasing labour costs by as much as
				$5\times$ \cite{curtis12}.
				
			\end{description}
			
			In conventional torus topologies the need for long cables is eliminated
			by folding and interleaving units of the network \cite{dally04}. For
			example, for a 1D torus topology (a ring network), one long connection
			exists to connect the two opposite sides of the system. To remove these
			long connections, half the units are `folded' on top of the others and
			then this arrangement of units is interleaved as illustrated in figure
			\ref{fig:ring-folding}.
			
			\begin{figure}
				\center
				\begin{subfigure}[b]{0.39\linewidth}
					\center
					\buildfig{figures/ring-folding-row.tex}
					\caption{A ring network}
					\label{fig:ring-folding-row}
				\end{subfigure}
				\begin{subfigure}[b]{0.24\linewidth}
					\center
					\buildfig{figures/ring-folding-folded.tex}
					\caption{Folded}
					\label{fig:ring-folding-folded}
				\end{subfigure}
				\begin{subfigure}[b]{0.35\linewidth}
					\center
					\buildfig{figures/ring-folding-interleaved.tex}
					\caption{Folded and interleaved}
					\label{fig:ring-folding-interleaved}
				\end{subfigure}
				
				\caption{Folding and interleaving a ring network to reduce maximum wire
				length.}
				\label{fig:ring-folding}
			\end{figure}
			
			Folding and interleaving has the effect of approximately doubling the
			average cable length but also eliminates the need for a cable spanning
			the entire system. Since the mean cable length is typically already
			short, doubling it in exchange for a substantially reduced maximum cable
			length is often preferable.
			
			The folding and interleaving process may be extended to $N$-dimensional
			torus topologies by folding each dimension in turn. Since all dimensions
			are orthogonal, the folding process only moves units in the dimension
			being folded. In the hexagonal torus topology, however, the three
			dimensions are non-orthogonal and thus folding in one dimension also
			moves units in other dimensions, preventing the edges of the torus
			meeting as illustrated in figure \ref{fig:failing-to-fold-hex-toruses}.
			
			\begin{figure}
				\center
				\begin{subfigure}[b]{0.24\linewidth}
					\center
					\buildfig{figures/failing-to-fold-hex-toruses-none.tex}
					\caption{Not folded}
					\label{fig:failing-to-fold-hex-toruses-none}
				\end{subfigure}
				\begin{subfigure}[b]{0.24\linewidth}
					\center
					\buildfig{figures/failing-to-fold-hex-toruses-x.tex}
					\caption{X}
					\label{fig:failing-to-fold-hex-toruses-x}
				\end{subfigure}
				\begin{subfigure}[b]{0.24\linewidth}
					\center
					\buildfig{figures/failing-to-fold-hex-toruses-y.tex}
					\caption{Y}
					\label{fig:failing-to-fold-hex-toruses-y}
				\end{subfigure}
				\begin{subfigure}[b]{0.24\linewidth}
					\center
					\buildfig{figures/failing-to-fold-hex-toruses-z.tex}
					\caption{Z}
					\label{fig:failing-to-fold-hex-toruses-z}
				\end{subfigure}
				
				\caption{Schematics showing hexagonal torus topologies folded along
				each of their non-orthogonal dimensions. Note that folding along
				the Z axis brings the \emph{wrong} edges closer together.}
				\label{fig:failing-to-fold-hex-toruses}
			\end{figure}
		
		\subsection{Cabling installation}
			
			Existing machine room installations feature very repetitive cabling
			patterns which can easily be memorised by a human technician. For example
			in BlueGene super computers the connectivity between units is highly
			regular \cite{lakner07} while in data centre networks cabling often
			centres around a small number of high-port-count switches
			\cite{cisco07,csernai15}. Cable installation is usually only aided by
			the labelling of connectors and sockets in a standardised manner
			\cite{tia2006} such as in figure \ref{fig:bgWiring}.
			
			\begin{figure}
				\center
				\begin{subfigure}[t]{0.5\textwidth}
					\begin{tikzpicture}
						\node (cables) [inner sep=0]
						      {\includegraphics[width=\textwidth]{figures/bgCables.png}};
						\node (sockets) [inner sep=0, below=1.0em of cables]
						      {\includegraphics[width=\textwidth]{figures/bgSockets.png}};
						
						% Point at label on cable
						\draw [white, <-, line width=0.4em]
						      ([shift={(0.7cm, -0.3cm)}]cables.center)
						      -- ++(45:1cm);
						
						% Point at label on socket
						\draw [white, <-, line width=0.4em]
						      ([shift={(-1.0cm, 1.1cm)}]sockets.center)
						      -- ++(-45:1cm);
					\end{tikzpicture}
					
					\caption{Pre-labelled cables and sockets}
					\label{fig:bgWiringLabels}
				\end{subfigure}
				~
				\begin{subfigure}[t]{0.30\textwidth}
					\includegraphics[height=6.15cm]{figures/bgWiring.jpg}
					
					\caption{Installation of cables}
					\label{fig:bgWiringInstallation}
				\end{subfigure}
				
				\caption{BlueGene/Q cable installation \cite{cscs13}}
				\label{fig:bgWiring}
			\end{figure}
			
			Despite the regularity and careful labelling of cables, the cost of
			installation and maintenance alone can be significant with costs in the
			range of \$45-95 per \SI{1}{\meter} cable run and \$100-400 for runs of
			\SI{10}{\meter} reported in the literature \cite{mudigonda11}. Much of
			this cost is due to the care required during installation to avoid
			miswiring and ensure that cooling airflow is not hampered by cable runs
			\cite{cisco07}.
			
			Many researchers have attempted to control cable installation costs by
			trying to reduce the number or length of cables required by developing
			alternative network topologies \cite{curtis12, popa10, mudigonda11}.
			Unfortunately, these techniques do not apply to SpiNNaker since its
			network topology is fixed.
			
			Some super computers make use of large custom `midplane` PCBs in place of
			cables to interconnect units within a cabinet and thus simplify the task
			of cable installation \cite{prickett10}. This scheme can greatly reduce
			wiring complexity since only coarser-grain cabinet-to-cabinet
			connectivity is provided by cables. Unfortunately this technique is
			expensive and also constrains the dimensions of the network topology
			supported by the machine. Since the SpiNNaker platform is designed to
			scale from desktop machines to machine-room installations, this scheme is
			not practical.
	
	\section{Folding \& interleaving hexagonal toruses}
		
		The first step towards a practical machine-room installation of a large
		machine using a hexagonal torus topology is to find an arrangement of
		boards between which cable lengths are minimised. In this section I
		describe a sequence of transformations which map the positions of units in
		a hexagonal torus topology onto a regular rectangular grid which may be
		folded and interleaved to eliminate long wires. It is worth emphasising
		that this transformation only affects the \emph{physical} positions of
		units and \emph{not} their connectivity.
		
		As described earlier in \S\ref{sec:parititioning} (page
		\pageref{sec:parititioning}), hexagonal torus topologies may be partitioned
		into units containing wrapped-triples of nodes. For example, in SpiNNaker,
		chips (nodes) are partitioned into circuit boards (units) containing 48
		chips. For completeness, this section describes the process of folding both
		systems whose units are individual nodes and those whose units are
		wrapped-triples.
		
		The transformation process is divided into two parts, each described
		separately in this section.
		
		\begin{description}
			
			\item[Parallelogram to rectangle] Units of the system are transformed
			from a parallelogram shape to a rectangular shape.
			
			\item[Uncrinkle] Units within the rectangle are moved such that they all
			lie on a regular (and fully packed) 2D grid.
			
		\end{description}
		
		\subsection{Parallelogram to rectangle}
			
			The hexagonal torus topology is most naturally drawn as a parallelogram
			as illustrated in figures \ref{fig:hex-to-plane-node-native} and
			\ref{fig:hex-to-plane-native}. Two transformations are presented which
			transform these arangements of units into a rectangular form: shearing
			and slicing.
			
			A \SI{30}{\degree} shear transformation distorts networks such that the X
			and Y axes become orthogonal leading to a rectangular arrangement of
			units as illustrated in figures \ref{fig:hex-to-plane-node-shear} and
			\ref{fig:hex-to-plane-shear}.
			
			The slice transformation slices the units protruding from the
			left-hand-side of the parallelogram and moves them into the matching gap
			on the opposite side of the parallelogram as illustrated in figures
			\ref{fig:hex-to-plane-node-slice} and \ref{fig:hex-to-plane-slice}.
			 
			While the shear transformation introduces some distortion causing cables
			in the Z dimension to become $\sqrt{2}\times$ longer it leaves the
			pattern of wrap-around connections remains unchanged. By contrast, the
			slice transformation does not elongate any cables but changes the pattern
			of wrap-around connections. The exact pattern wrap-around connections
			produced when slicing depends on the aspect ratio of the network as
			illustrated in \ref{fig:slicing-examples} and influences the choice of
			folding technique applied as described later.
			
			\begin{figure}
				\center
				\begin{subfigure}[b]{0.32\linewidth}
					\center
					\buildfig{figures/hex-to-plane-node-native.tex}
					
					\caption{$7 \times 7$ node torus}
					\label{fig:hex-to-plane-node-native}
				\end{subfigure}
				\begin{subfigure}[b]{0.32\linewidth}
					\center
					\buildfig{figures/hex-to-plane-node-shear.tex}
					
					\caption{Sheared}
					\label{fig:hex-to-plane-node-shear}
				\end{subfigure}
				\begin{subfigure}[b]{0.32\linewidth}
					\center
					\buildfig{figures/hex-to-plane-node-slice.tex}
					
					\caption{Sliced}
					\label{fig:hex-to-plane-node-slice}
				\end{subfigure}
				
				\caption{Transformations of hexagonal toruses of nodes into a
				rectangular form. Thin lines show wrap-around links. Pointy-topped
				hexagons represent individual nodes.}
				\label{fig:hex-to-plane-node}
			\end{figure}
			
			\begin{figure}
				
				\begin{subfigure}[b]{0.32\linewidth}
					\center
					\buildfig{figures/hex-to-plane-native.tex}
					
					\caption{$4 \times 4$ triad torus}
					\label{fig:hex-to-plane-native}
				\end{subfigure}
				\begin{subfigure}[b]{0.32\linewidth}
					\center
					\buildfig{figures/hex-to-plane-shear.tex}
					
					\caption{Sheared}
					\label{fig:hex-to-plane-shear}
				\end{subfigure}
				\begin{subfigure}[b]{0.32\linewidth}
					\center
					\buildfig{figures/hex-to-plane-slice.tex}
					
					\caption{Sliced}
					\label{fig:hex-to-plane-slice}
				\end{subfigure}
				
				\caption{Transformations of hexagonal toruses of wrapped triples into a
				rectangular form.  Thin lines show wrap-around links. Flat-topped
				hexagons represent a wrapped triple of nodes.}
				\label{fig:hex-to-plane}
			\end{figure}
			
			\begin{figure}
				\center
				\buildfig{figures/slicing-examples.tex}
				\caption{Patterns of wiring in sliced systems of various sizes.}
				\label{fig:slicing-examples}
			\end{figure}
			
		\subsection{Uncrinkling}
			
			Though the transformmation step yields rectangular arrangements of units,
			these arrangements do not fall onto a regular 2D grid, with the exception
			of the shear transform on individual nodes. Figure \ref{fig:uncrinkling}
			illustrates how the various arrangements of hexagons may be moved to
			`uncrinkle' the units into a regular grid.
			
			\begin{figure}
				\center
				\begin{subfigure}[b]{0.44\linewidth}
					\center
					\buildfig{figures/uncrinkling-node-sheared.tex}
					
					\caption{$7 \times 7$ nodes, sheared}
					\label{fig:uncrinkling-node-sheared}
				\end{subfigure}
				\begin{subfigure}[b]{0.44\linewidth}
					\center
					\buildfig{figures/uncrinkling-node-sliced.tex}
					
					\caption{$7 \times 7$ nodes, sliced}
					\label{fig:uncrinkling-node-sliced}
				\end{subfigure}
				
				\vspace{1cm}
				
				\begin{subfigure}[b]{0.44\linewidth}
					\center
					\buildfig{figures/uncrinkling-sheared.tex}
					
					\caption{$4 \times 4$ triples, sheared}
					\label{fig:uncrinkling-sheared}
				\end{subfigure}
				\begin{subfigure}[b]{0.44\linewidth}
					\center
					\buildfig{figures/uncrinkling-sliced.tex}
					
					\caption{$4 \times 4$ triples, sliced}
					\label{fig:uncrinkling-sliced}
				\end{subfigure}
				
				\vspace{1em}
				
				\caption{Mapping rectangular arrangements of units into a square grid.
				Thick lines show how layers of units are uncrinkled.  Annotations show
				how the relative positions of nodes and wrapped triples change after
				uncrinkling.}
				\label{fig:uncrinkling}
			\end{figure}
			
			In the figure, the numbered units enumerate the different positions on
			the crinkle and those labelled alphabetically are those that immediately
			surround them. From this we can observe that uncrinkling largely
			preserves spatial locality but some distortion is introduced, separating
			previously neighbouring units. For example, in figure
			\ref{fig:uncrinkling-sheared}, the units labelled `1' and `i' are
			neighbours before uncrinkling but are separated by a (Euclidean) distance
			of $\sqrt{5}$ afterwards. Note that the distortion introduced depends on
			what part of the crinkle is considered, for example `2' and `a' have
			distance 2 but are logically connected in the same way.
		
		\subsection{Folding and Interleaving}
			
			Once a regular grid of units has been formed, this may be folded in the
			conventional way, eliminating long cables crossing from left-to-right and
			top-to-bottom as illustrated in \ref{fig:folding-sheared}.
			
			Unfortunately, for sliced systems whose dimensions are not of the ratio
			$1:2$, the pattern of wrap-around cables may also include some cables
			which do not cross directly to the opposite side of the system (refer
			back to figure \ref{fig:slicing-examples}). As a result of these
			connections, folding does not successfully eliminate all long
			connections. An exception to this rule is sliced systems whose dimensions
			are in the ratio $1:1$ where folding twice along the Y axis may
			successfully eliminate all wrap-around connections as illustrated in
			\ref{fig:folding-sliced}.
			
			\begin{figure}
				\begin{subfigure}{\linewidth}
					\center
					\buildfig{figures/folding-sheared.tex}
					\caption{$N \times M$ sheared systems and $N \times 2N$ sliced systems}
					\label{fig:folding-sheared}
				\end{subfigure}
				
				\vspace{1em}
				
				\begin{subfigure}{\linewidth}
					\center
					\buildfig{figures/folding-sliced.tex}
					\caption{$N \times N$ sliced systems}
					\label{fig:folding-sliced}
				\end{subfigure}
				
				\caption{Schematic illustrating elimination of long wrap-around links
				during folding. In each example a single link has been highlighted to
				aid in following the process.}
				\label{fig:folding}
			\end{figure}
			
			Once folded, the 2D grid is straight-forwardly interleaved as illustrated
			previously in figure \ref{fig:ring-folding}. The interleaving process
			introduces some additional distortion to the layout of units and causes
			most connections to become twice as long. For sliced $1:1$ systems, the
			additional fold results in additional overhead during interleaving since
			four layers of the system are interleaved.
		
		\subsection{Mapping to Cabinets}
			
			In the final step of the process is to map the 2D grid of units into
			positions in machine room cabinets as illustrated in figure
			\ref{fig:million-core-machine}. As illustrated in figure
			\ref{fig:cabinetisation}, first the grid of units is partitioned into
			groups of columns, one per cabinet, then groups of rows one per frame per
			cabinet. The units in each group are then allocated to slots within a
			frame, interleaving the rows of the groups as shown.
			
			\begin{figure}
				\center
				\buildfig{figures/cabinet-units.tex}
				
				\caption{An illustration of the physical construction of a
				multi-cabinet SpiNNaker system. (Note: network cables \emph{not}
				installed.)}
				\label{fig:cabinet-units}
			\end{figure}
			
			\begin{figure}
				\center
				\buildfig{figures/cabinetisation.tex}
				
				\caption{Mapping from 2D space to cabinets, frames and boards.}
				\label{fig:cabinetisation}
			\end{figure}
		
	\section{Cable installation}
		
		Cable installation is performed by a team of (human) technicians who must
		ensure that all network cables are correctly installed. As illustrated in
		previously in figure \ref{fig:cabinet-units}, the density of SpiNNaker's
		units, combined with the nature of the hexagonal torus topology, poses a
		challenge. To address this challenge I propose a semi-automated approach to
		cable installation which exploits diagnostic facilities available in the
		majority of super computers in order to guide technicians through the
		cabling process, interactively guiding installation and maintenance.
		
		\subsection{Interactive technician guidance and validation}
			
			While automated systems for validating cabling correctness are
			commonplace, these systems are typically used only after cabling has been
			completed \cite{lakner07}. As with other large-scale machines, SpiNNaker
			includes a low-bandwidth system management bus which may be used to
			interrogate network hardware and control diagnostic LEDs prior to the
			installation of the main SpiNNaker network interconnect.  Using these
			facilities I have constructed a tool called SpiNNer which interactively
			guides a technician, or team of technicians, through the cable
			installation process, validating each connection in real-time.
			
			Diagnostic LEDs mounted on each SpiNNaker board (figure
			\ref{fig:interactive-wiring-guide-leds}) are used to indicate the
			endpoints of the cable currently being installed. Simultaneously a
			Text-To-Speech (TTS) system gives an audible indication of which cable
			type is to be used and location of each connection.  Additionally, a GUI
			via a computer display (figure \ref{fig:interactive-wiring-guide-gui}).
			The centre of the display shows a `big-picture' perspective of the
			locations of the boards to be connected. The detailed views on the left
			and right indicate which of the six sockets on each board the cables
			should connect.
			
			\begin{figure}
				\center
				\begin{subfigure}[b]{0.40\textwidth}
					\begin{tikzpicture}
						\node (leds) [inner sep=0]
						      {\includegraphics[width=\textwidth]{figures/leds.jpg}};
						% Point at left LED
						\draw [white, <-, line width=0.4em]
						      ([shift={(-0.0cm, -0.6cm)}]leds.center)
						      -- ++(225:1cm);
						% Point at right LED
						\draw [white, <-, line width=0.4em]
						      ([shift={(1.1cm, -1.1cm)}]leds.center)
						      -- ++(225:1cm);
					\end{tikzpicture}
					
					\caption{Diagnostic LEDs}
					\label{fig:interactive-wiring-guide-leds}
				\end{subfigure}
				~
				\begin{subfigure}[b]{0.546\textwidth}
					\begin{tikzpicture}[thin, black!20!white]
						\node (screen) [inner sep=0]
						      {\includegraphics[width=\textwidth]{figures/wiring_guide_screenshot.png}};
						\draw (screen.south west) rectangle (screen.north east);
					\end{tikzpicture}
					
					\caption{Interactive wiring guide GUI}
					\label{fig:interactive-wiring-guide-gui}
				\end{subfigure}
				
				\caption{The SpiNNer interactive wiring guide uses a GUI,
				text-to-speech and diagnostic LEDs to assist during cable
				installation.}
				\label{fig:interactive-wiring-guide}
			\end{figure}
			
			SpiNNer also validates the connectivity of the system in real-time by
			polling the diagnostic interfaces of the network hardware at the
			endpoints of the cable being installed to determine if they are correctly
			connected. If a miswiring occurs, this is immediately detected and
			announced via TTS enabling the technician to immediately correct the
			error. Once a cable has been installed correctly, the software
			automatically advances to the next cable meaning direct interaction with
			the software by the technician is minimal. In practice, it is rarely
			necessary to refer to the GUI.
		
			SpiNNer presents the cables in an order intended to maximise ease of
			installation. Cables are installed in three groups with intra-frame
			cables being installed first, followed by intra-cabinet cables and
			inter-cabinet cables. Within each group, the tightest cables are
			installed first resulting in slacker cables naturally being installed
			over the top of already installed cables. By grouping cables in this
			manner, multiple technicians may work independently on the wiring within
			individual frames and cabinets.
			
			SpiNNer may also be used to repair or replace cables in the system.
			During maintenance, obstructing cables may be blindly removed alongside
			any cable being replaced. At the conclusion of the process, the wiring
			guide may be used to interactively guide re-installation of all removed
			cables.
		
		\subsection{Cable selection}
			
			Controlling slack is critical to ensuring reliable and maintainable
			cabling installations. If cables are too tight, cables and connectors can
			become easily damaged and when too slack, the excess cable obstructs
			other cables and can easily become tangled and damaged \cite{cisco07}. It
			has been observed that when ready-made cables are employed technicians
			frequently over-estimate the cable lengths required preferring to use
			overly long cables for all connections \cite{mazaris97}. To solve this
			problem, the SpiNNer wiring guide software dictates the cable lengths to
			be used by an installer based the rule of (three-)thumbs according to
			Mazaris \cite{mazaris97}. This rule suggests that an ideal amount of
			slack is approximately that which can be wrapped around three fingers.
			Specifically, the shortest available cable is selected which ensures at
			least \SI{5}{\centi\meter} of slack.
			
			The SpiNNer tool allocates cables assuming all cables take a Euclidean
			straight-line path between the endpoints of the connection. The result is
			that wiring is not routed through dedicated cable management structures
			but is simply suspended by its connectors in front of the machine. As
			demonstrated later, this unconventional approach leads neither to cooling
			problems nor increased maintenance effort.
	
	\section{Results and Evaluation}
		
		This stuff has been used and works. In this section I'll go over the
		overheads introduced by the various transformations and
		folding/interleaving steps and show a wiring scheme for a large machine
		which uses only short cables. I'll then show how SpiNNer was used to
		install this wiring plan into a very large machine without foobaring the
		cooling and in very little time. I'll also report on difficulty of
		maintenance.
		
		\subsection{Cable length}
			
			The transformation from regular hexagonal torus to a folded and
			interleaved form introduces some overhead to the cable lengths required.
			Using figure \ref{fig:uncrinkling} (page \pageref{fig:uncrinkling}), it
			is possible to compute the exact overhead introduced when each type of
			transformation proposed.
			
			For example, to compute the mean overhead introduced by the slicing
			technique when applied to units composed of wrapped triples, consider
			figure \ref{fig:uncrinkling-sliced}. The uncrinkling pattern used to
			transform this topology is a repeating pattern of two units, a pair of
			which have been labelled $1$ and $2$ respectively. Unit $1$ is
			immediately surrounded by six units labelled $a$, $b$, $c$, $2$, $g$ and
			$h$. Similarly, unit $2$ is surrounded by units $1$, $c$, $d$, $e$, $f$
			and $g$. Before the transformation, the distances, $D$, to each of these
			units is $1$ but after the transformation is applied, this is not always
			the case. Additionally, folding and interleaving introduce additional
			overhead. In this example, if the system is folded into $f_x$ columns and
			$f_y$ rows, the distances between previously neighbouring units become:
			
			\begin{equation*}
				\begin{aligned}[c]
					D_{1\,\leftrightarrow{}\,a} &= \sqrt{f_x^2 + f_y^2} \\
					D_{1\,\leftrightarrow{}\,b} &= f_y \\
					D_{1\,\leftrightarrow{}\,c} &= \sqrt{f_x^2 + f_y^2} \\
					D_{1\,\leftrightarrow{}\,2} &= f_x \\
					D_{1\,\leftrightarrow{}\,g} &= f_y \\
					D_{1\,\leftrightarrow{}\,h} &= f_x
				\end{aligned}
				\hspace{2cm}
				\begin{aligned}[c]
					D_{2\,\leftrightarrow{}\,1} &= f_x \\
					D_{2\,\leftrightarrow{}\,c} &= f_y \\
					D_{2\,\leftrightarrow{}\,d} &= f_x \\
					D_{2\,\leftrightarrow{}\,e} &= \sqrt{f_x^2 + f_y^2} \\
					D_{2\,\leftrightarrow{}\,f} &= f_y \\
					D_{2\,\leftrightarrow{}\,g} &= \sqrt{f_x^2 + f_y^2}
				\end{aligned}
			\end{equation*}
			
			From these values, the mean and maximum connection distances after
			folding and interleaving may be computed. Table
			\ref{tab:transform-overhead} gives the mean and maximum connection
			distances for each of the four transformations described in this chapter.
			
			\begin{table}
				\begin{subtable}[b]{\linewidth}
					\center
					\begin{tabular}{l c c}
						\toprule
						& Shear & Slice \\
						\addlinespace
						Nodes &
							$\frac{f_x + f_y + \sqrt{f_x^2 + f_y^2}}{3}$ &
							$\frac{f_x + f_y + \sqrt{f_x^2 + f_y^2}}{3}$ \\
						\addlinespace
						Triples &
							$\frac{7f_x + 3\sqrt{f_x^2 + f_y^2} + \sqrt{(2f_x)^2 + f_y^2}}{9}$ &
							$\frac{f_x + f_y + \sqrt{f_x^2 + f_y^2}}{3}$ \\
						\bottomrule
					\end{tabular}
					
					\caption{Mean}
					\label{tab:transform-overhead-mean}
				\end{subtable}
				
				\vspace{1em}
				
				\begin{subtable}[b]{\linewidth}
					\center
					\begin{tabular}{l c c}
						\toprule
						& Shear & Slice \\
						\addlinespace
						Nodes &
							$\sqrt{f_x^2 + f_y^2}$ &
							$\sqrt{f_x^2 + f_y^2}$ \\
						\addlinespace
						Triples &
							$\sqrt{(2f_x)^2 + f_y^2}$ &
							$\sqrt{f_x^2 + f_y^2}$ \\
						\bottomrule
					\end{tabular}
					
					\caption{Maximum}
					\label{tab:transform-overhead-max}
				\end{subtable}
				
				\caption{Overheads introduced when transforming unit positions onto a
				regular grid.}
				\label{tab:transform-overhead}
			\end{table}
			
			From these results it is evident that shearing and slicing networks
			whose units are nodes result in identical mean and maximum overhead in
			cable length when folded similarly. Since sliced networks may require
			folding more than once along each axis the shearing approach is
			preferable in general.
			
			For networks constructed from units of wrapped triples, the slicing
			approach suffers the same mean and maximum overhead has networks of
			nodes, and less overhead than shearing for the same number of folds. For
			systems with an aspect ratio of $1:2$ (where both slicing and shearing
			require $f_x = f_y = 2$), the slicing transformation yields lower mean
			and maximum overhead than shearing. For all other aspect ratios (where
			slicing requires a greater degree of folding) the shearing technique
			produces lower overhead. The recommended transformations for a given
			machine are thus given in table \ref{tab:transform-recommended}.
			
			\begin{table}
				\center
				\begin{tabular}{lcc}
					\toprule
					                         & $1:2$  & Other \\
					\addlinespace
					\multirow{2}{*}{Nodes}   & Either & Shear\\
					                         & \footnotesize $\mu\approx2.28 \quad \vee\approx2.83$
					                         & \footnotesize $\mu\approx2.28 \quad \vee\approx2.83$\\
					\addlinespace
					\multirow{2}{*}{Triples} & Slice  & Shear\\
					                         & \footnotesize $\mu\approx2.28 \quad \vee\approx2.83$
					                         & \footnotesize $\mu\approx3.00 \quad \vee\approx4.47$\\
					\bottomrule
				\end{tabular}
				
				\caption{Recommended transformation and folding scheme for different
				system types. $\mu$ and $\vee$ give the mean and maximum wire
				distortion introduced, respectively.}
				\label{tab:transform-recommended}
			\end{table}
			
			\begin{figure}
				\center
				\buildfig{figures/million-core-machine.tex}
				
				\caption{Cabling plan for a \num{1036800} core SpiNNaker
				machine's \num{3600} cables.}
				\label{fig:million-core-machine}
			\end{figure}
			
			Following folding and mapping to physical locations, the cabling plans
			for large machines require no large gaps to be spanned.  The largest
			planned SpiNNaker machine, illustrated in figure
			\ref{fig:million-core-machine}, will be \SI{6}{\meter} wide but the
			largest gap any cable must span is \SI{66}{\centi\meter}. This distance
			is well within the \SI{1}{\meter} allowed by the hardware and cables.
			
		\subsection{Installation practicality}
			
			\begin{table}
				\center
				\begin{tabular}{lrr@{$\,$}l}
					\toprule
						System & Number of Cables & \multicolumn{2}{r}{Installation time} \\
					\midrule
						24 boards  & \num{72}   & \num{10} & \si{\minute}         \\
						1 cabinet  & \num{360}  & \num{4}  & \si{\hour}$^\dagger$ \\
						2 cabinets & \num{720}  & \num{2}  & \si{\hour}           \\
						5 cabinets & \num{1800} & ?        &                      \\
					\bottomrule
				\end{tabular}
				
				\caption{Installation times for various sizes of machine.
				$\dagger$~This machine was installed without real-time validation of
				connectivity.}
				\label{tab:install-time}
			\end{table}
			
			A number of SpiNNaker machines of various scales have been assembled
			using the techniques described in this chapter ranging from single frames
			of 24 boards to a half-scale 5 cabinet machine. Table
			\ref{tab:install-time} gives the reported installation times of each of
			these machines.
			
			The single cabinet machine's installation time is notably
			disproportionate to its size. When this system was assembled, real-time
			connection validation was not yet available. As a result, though cable
			installation was rapid correcting errors was extremely costly, requiring
			careful retracing of many installation steps.
			
			TODO: TALK ABOUT MULTI-PERSON-WIRING IN PRACTICE ON FIVE CABINET MACHINE.
			
			\begin{figure}
				
				\center
				\buildfig{figures/wire-length-histogram.tex}
				
				\caption{Histogram of connection distances in a ten-cabinet,
				one-million core SpiNNaker machine annotated with the suggested cable
				length.}
				\label{fig:wire-length-histogram}
				
			\end{figure}
			
			FIGURE \ref{fig:wire-length-histogram} SHOWS THE DISTRIBUTION OF CABLE
			LENGTHS REQUIRED. IN PRACTICE THE SLACK ALLOCATED PROVED ADEQUATE. AS
			SHOWN IN FIGURE \ref{fig:install-histogram}, THE MOST IMPORTANT FACTOR IS
			WHETHER LEAVING THE FRAME OR NOT. LEAVING THE FRAME TAKES THE LONGEST.
			
			\begin{figure}
				\builddata{data/build_connection_log.tex}
				\buildfig{figures/install-histogram.tex}
				
				\caption{Histogram of cable installation times}
				\label{fig:install-histogram}
			\end{figure}
			
			TODO: COMPARE DIRECTLY WITH INSTALL TIMES REPORTED IN LITERATURE.
		
		\subsection{Thermal Impact}
			
			TODO: SHOW HOW TEMPERATURE IS CHANGED
			
		\subsection{Maintenance}
			
			TOOD: QUANTIFY CABLE REMOVALS REQUIRED. EXPERIMENT: REMOVE/REPLACE RANDOM
			BOARDS AND MEASURE TIME TAKEN, CABLES REMOVED. COMPARE WITH STANDARD DATA
			CENTRE WIRING

	\chapter{Finding shortest path vectors in SpiNNaker's network}
	
	Once a SpiNNaker machine has been constructed as described in the previous
	chapter, its network forms a large hexagonal torus topology. To exploit this
	network routing algorithms must be used to generate routes for packets to
	follow between nodes. As well as ensuring that packets arrive at the correct
	destination, routing algorithms often attempt to produce routes which make
	efficient use of the network. This often involves attempting to reduce
	congestion by ensuring packets do not travel further through the network than
	absolutely necessary.
	
	Many popular routing algorithms for torus topologies, including all published
	algorithms designed for SpiNNaker's hexagonal torus topology
	\cite{davies12,navaridas14}, internally function by computing shortest path
	vectors and generating routes from them. Existing methods of calculating
	shortest path vectors in hexagonal torus topologies are unable to generate
	all possible shortest path vectors and, as a result, reduces the diversity of
	routes produced by routing algorithms, potentially worsening network
	contention.
	
	In this chapter I describe a novel technique for computing shortest path
	vectors in hexagonal torus topologies which yields \emph{all} possible
	shortest path vectors for any pair of nodes. Further, implementations of this
	new technique execute an order of magnitude faster than the existing
	alternatives.
	
	\section{Related work}
		
		TODO: INTRODUCE SECTION
		
		\begin{figure}
			\center
			
			\begin{subfigure}{\linewidth}
				\center
				\buildfig{figures/distance-map-mesh.tex}
				\caption{2D mesh topology}
				\label{fig:distance-map-mesh}
			\end{subfigure}
			
			\vspace{1em}
			
			\begin{subfigure}{\linewidth}
				\center
				\buildfig{figures/distance-map-torus.tex}
				\caption{2D torus topology}
				\label{fig:distance-map-torus}
			\end{subfigure}
			
			\vspace{1em}
			
			\begin{subfigure}{\linewidth}
				\center
				\buildfig{figures/distance-map-hex-mesh.tex}
				\caption{Hexagonal mesh topology}
				\label{fig:distance-map-hex-mesh}
			\end{subfigure}
			
			\vspace{1em}
			
			\begin{subfigure}{\linewidth}
				\center
				\buildfig{figures/distance-map-hex-torus.tex}
				\caption{Hexagonal torus topology}
				\label{fig:distance-map-hex-torus}
			\end{subfigure}
			
			\caption{Plots showing distance from various locations marked
			         {\color{red}$\times$}. Darker areas are further away. Contour
			         lines show equidistant points.}
			\label{fig:distance-map}
		\end{figure}
		
		\subsection{Mesh Networks}
			
			In a (non-hexagonal) mesh network topology, shortest path vectors are
			computed by taking the element-wise difference between the source and
			destination nodes' coordinates.
			
			\begin{figure}
				\center
				\buildfig{figures/mesh-topology-coordinates.tex}
				\caption{An example 2D mesh network with example shortest-path routes
				from `A' to `B' and `B' to `C'.}
				\label{fig:mesh-topology-coordinates}
			\end{figure}
			
			For example, figure \ref{fig:mesh-topology-coordinates} illustrates a 2D
			mesh topology. In this topology, the nodes labelled `A', `B' and `C' have
			position vectors $(1, 2)$, $(4, 5)$ and $(6, 1)$ respectively. The
			shortest path vector from node `A' to `B' is thus simply $(4, 5) - (1, 2)
			= (3, 3)$ and from `B' to `C' is $(6, 1) - (4, 5) = (2, -4)$.
			
			A route may be produced from a shortest path vector by advancing the
			number of hops specified for each dimension in the vector. For example
			any permutation of the hops X$^+\,$X$^+\,$X$^+\,$Y$^+\,$Y$^+\,$Y$^+$, an
			example of which is included in the figure. Likewise a route from `B' to
			`C' may be constructed from any permutation of
			X$^+\,$X$^+\,$Y$^-\,$Y$^-\,$Y$^-\,$Y$^-$.
			
			Many popular routing algorithms such as Dimension Order Routing (DOR),
			Right-Turn Only Routing (RTOR) and Longest Dimension First Routing (LDFR)
			\cite{dally04,davies12} directly follow the above procedure and just
			prescribe a specific permutation of hop order. For example, DOR produces
			routes with X hops first, Y hops second and so on.
			
			The length of routes produced from a shortest path vector have a number
			of hops proportional to the magnitude of the vector, thus a shortest path
			vector yields a route with the minimum number of hops. For a two
			dimensional vector $(a, b)$ the magnitude is given as:
			%
			\begin{equation}
				\| (a, b) \| = \lvert a \rvert + \lvert b \rvert
			\end{equation}
		
		\subsection{Torus Networks}
			
			Computing shortest path vectors in (non-hexagonal) torus topologies is
			also straight forward. As an example, lets find the shortest path vector
			from node `A' to `B' in the 2D torus topology shown in figure
			\ref{fig:torus-shortest-path-example}. First, both nodes are translated
			such that the source node, `A', is at the centre of the network (figure
			\ref{fig:torus-shortest-path-translate}). Note that this translation may
			result in the destination node `wrapping around' the network. After
			translation, the shortest path vector is computed as in a mesh topology.
			As illustrated in \ref{fig:torus-shortest-path-routed}, the computed
			shortest path vector may be used to produce routes between the two nodes
			in their original positions.
			
			\begin{figure}
				\center
				\begin{subfigure}{0.3\linewidth}
					\center
					\buildfig{figures/torus-shortest-path-example.tex}
					\caption{Original}
					\label{fig:torus-shortest-path-example}
				\end{subfigure}
				\begin{subfigure}{0.3\linewidth}
					\center
					\buildfig{figures/torus-shortest-path-translate.tex}
					\caption{Translated}
					\label{fig:torus-shortest-path-translate}
				\end{subfigure}
				\begin{subfigure}{0.3\linewidth}
					\center
					\buildfig{figures/torus-shortest-path-routed.tex}
					\caption{Routed}
					\label{fig:torus-shortest-path-routed}
				\end{subfigure}
				
				\caption{Finding shortest paths in a 2D torus topology.}
				\label{fig:torus-shortest-path}
			\end{figure}
			
			This process works because vectors from the centre (though not other
			locations) of a torus topology are identical to those in mesh topologies
			(see figures \ref{fig:distance-map-mesh} and
			\ref{fig:distance-map-torus}).
		
		\subsection{Hexagonal Mesh Networks}
			
			In hexagonal mesh topologies it is conventional to define three `axes' X,
			Y and Z as shown in figure \ref{fig:hex-mesh-topology-coordinates}
			\cite{patel15}. In this example, the three labelled nodes `A', `B' and
			`C' may be given position vectors such as $(1, 1, 0)$, $(3, 2, 0)$ and
			$(0, 0, -7)$ respectively. As in other mesh networks, a vector between
			two nodes is found by subtracting the nodes' vectors. For example, a
			vector from `A' to `B' is $(3, 2, 0) - (1, 1, 0) = (2, 1, 0)$. This
			vector can then be converted into a route in the same way as a mesh
			network by taking any permutation of the three hops  X$^+\,$X$^+\,$Y$^+$.
			
			\begin{figure}
				\center
				\buildfig{figures/hex-mesh-topology-coordinates.tex}
				\caption{An example hexagonal mesh network topology.}
				\label{fig:hex-mesh-topology-coordinates}
			\end{figure}
			
			As explained in detail in appendix \ref{app:minimal-hex-coordinates},
			there are an infinite number of vectors between any two points. For
			example, the vectors $(1, 0, -1)$ and $(3, 2, 1)$ also reach node `B'
			from `A' in the example. However, for a given pair of nodes, there is
			always a single, unique vector whose magnitude is minimal which is
			given by the function:
			%
			\begin{equation}
				\operatorname{minimiseVector}(x,y,z)
					= (x,y,z) - \operatorname{median}(x,y,z) \cdot (1,1,1)
			\end{equation}
			%
			An important side-effect of this function is that a minimised vector will
			always contain at least one zero element meaning that shortest path
			routes will use at most two of the three available dimensions.
			
			To aid the reader's intuition, figure \ref{fig:distance-map-hex-mesh}
			illustrates how distances grow in a hexagonal mesh topology.
		
		\subsection{Hexagonal Torus Networks}
			
			Unfortunately, unlike non-hexagonal torus topologies, the translation
			technique cannot be used to compute shortest path vectors. As illustrated
			in figures \ref{fig:distance-map-hex-mesh} and
			\ref{fig:distance-map-hex-torus}, shortest path vectors from the center
			of a hexagonal mesh network are not equivalent to those of a hexagonal
			torus network.
			
			Prior research into routing in SpiNNaker's network has been based on the
			INSEE \cite{navaridas09,ghasempour15} interconnect simulator. Internally
			INSEE tries a set of twelve candidate vectors and picks the shortest as
			the shortest path vector to use for routing.
			
			\begin{figure}
				\center
				\begin{subfigure}{0.45\linewidth}
					\center
					\buildfig{figures/insee-vector-candidates-no-wrap.tex}
					\caption{$(\Delta_\textrm{X}, \Delta_\textrm{Y}) = (5,3)$}
					\label{fig:insee-vector-candidates-no-wrap}
				\end{subfigure}
				\begin{subfigure}{0.45\linewidth}
					\center
					\buildfig{figures/insee-vector-candidates-wrap-x.tex}
					\caption{$(\Delta'_\textrm{X}, \Delta_\textrm{Y}) = (-3,3)$}
					\label{fig:insee-vector-candidates-wrap-x}
				\end{subfigure}
				
				\vspace{1em}
				
				\begin{subfigure}{0.45\linewidth}
					\center
					\buildfig{figures/insee-vector-candidates-wrap-y.tex}
					\caption{$(\Delta_\textrm{X}, \Delta'_\textrm{Y}) = (5,-5)$}
					\label{fig:insee-vector-candidates-wrap-y}
				\end{subfigure}
				\begin{subfigure}{0.45\linewidth}
					\center
					\buildfig{figures/insee-vector-candidates-wrap.tex}
					\caption{$(\Delta'_\textrm{X}, \Delta'_\textrm{Y}) = (-3,-5)$}
					\label{fig:insee-vector-candidates-wrap}
				\end{subfigure}
				
				\vspace{1em}
				
				% Key
				\begin{tikzpicture}[thick]
					\coordinate (last);
					
					% #1 colour
					% #2 label
					\newcommand{\colourkeyentry}[2]{
						\node [#1] [right=of last, fill, rectangle, minimum size=1em] (last) {};
						\node [right=0 of last] (last) {#2};
					}
					
					\colourkeyentry{cb3class0}{$(\textrm{X}, \textrm{Y}, 0)$}
					\colourkeyentry{cb3class1}{$(\textrm{X} - \textrm{Y}, 0, - \textrm{Y})$}
					\colourkeyentry{cb3class2}{$(0, \textrm{Y} - \textrm{X}, - \textrm{X})$}
					
				\end{tikzpicture}
				
				\caption{The twelve candidate shortest-path vectors considered by INSEE
				represented as dimension-order routes. $W=H=8$,
				$(\Delta_\textrm{X},\Delta_\textrm{Y}) = (5, 3)$ and
				$(\Delta'_\textrm{X},\Delta'_\textrm{Y}) = (-3, -5)$.}
				\label{fig:insee-vector-candidates}
			\end{figure}
			
			The twelve vectors considered are constructed as follows.
			
			First a shortest path vector from the source to target node are
			constructed as if the network was a 2D mesh yielding a vector
			$(\Delta_\textrm{X},\Delta_\textrm{Y})$. From this, another vector
			$(\Delta'_\textrm{X},\Delta'_\textrm{Y})$, is defined:
			%
			\begin{align}
				\Delta'_\textrm{X} &= \Delta_\textrm{X} - \operatorname{sign}(\Delta_\textrm{X})W
				\\
				\Delta'_\textrm{Y} &= \Delta_\textrm{Y} - \operatorname{sign}(\Delta_\textrm{Y})H
			\end{align}
			%
			Where $W$ and $H$ are the width and height of the network respectively
			(in nodes). This new vector yields routes from the source to destination
			node that in a torus topology that \emph{always} wrap around the `X' and
			`Y' dimensions.
			
			From the pair of vectors defined, four possible 2D vectors can be
			produced: $(\Delta_\textrm{X},\Delta_\textrm{Y})$,
			$(\Delta'_\textrm{X},\Delta_\textrm{Y})$,
			$(\Delta_\textrm{X},\Delta'_\textrm{Y})$ and
			$(\Delta'_\textrm{X},\Delta'_\textrm{Y})$. Further, each 2D vector may be
			converted into one of three 3D vectors where either X, Y or Z are zero
			for a total of twelve candidate vectors.  Figure
			\ref{fig:insee-vector-candidates} illustrates all twelve candidate
			vectors for an example pair of nodes.
			
			\begin{figure}
				\center
				\buildfig{figures/xyz-protocol-regions.tex}
				
				\caption{The four regions defined by the XYZ-protocol.}
				\label{fig:xyz-protocol-regions}
			\end{figure}
			
			A more efficient technique is proposed by Hoffmann and D\'es\'erable
			called the XYZ-Protocol \cite{hoffmann15,hoffmann11}. If the source and
			destination nodes are translated such that the source node lies at the
			center of the topolgoy, the destination will lie in one of four regions
			illustrated in figure \ref{fig:xyz-protocol-regions}.
			
			If the destination lies in regions 1 or 4, a route may be constructed as
			if in a hexagonal mesh topology.
			
			Alternatively, if the destination lies in regions 2 or 3, the algorithm
			tests whether taking a mesh-like route within the region or
			wrapping-around either the X or Y dimension yields the shorter vector.
			The shortest of these vectors is then chosen.
			
			TODO DESCRIBE SPIRAL ROUTES.
			
			TODO DESCRIBE RTOR AND LDFR.
		
	\section{Dimension order routing in hexagonal torus topologies}
		
		So, existing solutions have two problems: trying 12 options and picking one
		is a bit kludgey and there are actually more options than that.
		
		\subsection{Simpler minimal hexagonal torus vectors}
			
			If you redraw your route such that it is sourced from bottom left corner
			(which we'll now call (0, 0)), there are four possible ways this route
			could wrap.
			
			TODO: DESCRIBE WAYS OF WRAPPING
			
			For each of these wrappings, all the possible routes we can take are
			strictly limited in terms of the dimensions used since we're stuck in a
			corner.
			
			In each case, the function computing the minimal hex vector function
			simplifies to a much simpler operation.
			
			TODO: DESCRIBE MINIMUM VECTOR LENGTH FUNCTIONS FOR EACH CASE
			
			This gives us a cheap way to compute which of the four possible wrappings
			are shortest. Based on this we can pick one of at most two (is this
			easily provable?) vectors in some fair manner to reduce load. This vector
			can then be minimised in the usual way.
			
			This also leads to confirming a theoretical result giving the length of a
			shortest path in a hexagonal torus topology.
			
			TODO: DESCRIBE HOW TO GET LENGTH FUNCTION AND COMPARE WITH \cite{xiao04}
		
		\subsection{Generating spiralling routes}
			
			In non-hexagonal torus topologies the previous technique would reveal all
			possible shortest vectors (e.g. in cases where you can wrap more than one
			way). Unfortunately, due to the addition of a non-orthogonal axes,
			hexagonal toruses also have other cases when the width and height do not
			match.
			
			TODO: TORUS SPIRALLING EXAMPLE
			
			It is possible to calculate the maximum number of spirals thus:
			
			TODO: DESCRIBE HOW MAX NUMBER OF SPIRALS IS COMPUTED
			
			Given a number of spirals, the vector can be updated this (note that the
			change does not add a multiple of (1, 1, 1) but also does not result in
			the vector changing length and thus becoming non-minimal!).
			
			TODO: DESCRIBE TRANSFORMATION
			
			TODO: PROVE THAT MINIMALITY IS MAINTAINED
		
		\subsection{Proof of completeness}
		
			TODO: PROOF OF COMPLETENESS BY EXHAUSTIVE SEARCH
	
		\subsection{Conclusions}
			
			This approach is simpler, smaller, has sounder theoretical basis, and
			finds more routes than alternatives. This is good for load balancing and
			fault avoidance and also good for completeness.


	\chapter{Routing packets in large SpiNNaker machines}
	
	\label{sec:routing}
	
	So far, this thesis has focused on tackling the practical challenges
	resulting from SpiNNaker's hexagonal torus network topology. In this chapter,
	I adjust my focus towards the practical challenges resulting from SpiNNaker's
	large scale. Faults in large systems are inevitable and in the half-million
	core, \num{28800} chip SpiNNaker machine recently completed at the University
	of Manchester, around \SI{1}{\percent} of chips exhibited faults\footnote{Of
	the faulty chips discovered, the vast majority of faults are attributed to a
	currently unknown SDRAM failure}. These faults result in gaps and broken
	links in the network topology which routing algorithms must avoid in order to
	ensure correct system operation.
	
	In this chapter I tackle the problem of extending existing routing algorithms
	for SpiNNaker's network to enable them to route-around known, static faults.
	Though dynamic or transient faults may also occur, in this work such faults
	are ignored and other techniques, such as protocol-level fault tolerance, are
	relied on instead.
	
	Numerous heuristic-based fault-tolerant routing algorithms exist which target
	different network topologies and router architectures. Unfortunately as I
	will show, these algorithms are not portable and rely on or attempt to work
	around specific features of their target network architecture. In particular,
	existing work is dominated by the challenge of developing routing schemes
	which avoid or defuse network deadlocks. Due to SpiNNaker's unconventional
	use of timeout-based flow-control, it is not subject to the routing
	restrictions present in other architectures intended to cope with deadlocks.
	
	In this chapter I introduce a graph-search based post-processing step for
	non-fault-tolerant routing algorithms which guarantees routability in
	SpiNNaker systems without disconnected subregions. I also demonstrate that
	this technique introduces both negligible computational overhead to the
	routing algorithm runtime and resulting network performance.
	
	TODO: NOTE THE FAULT RATES ENCOUNTERED IN PRACTICE...
	
	\section{Related work}
		
		Existing work on routing in SpiNNaker's network has ignored the challenge
		of avoiding faults and instead focused on producing efficient multicast
		routes. As a result this section is broken into two halves. In the first
		half I survey the existing non-fault-tolerant approaches to routing used in
		SpiNNaker to-date. In the second I discuss the approaches to fault tolerant
		routing taken in other systems.
		
		\subsection{Multicast routing in SpiNNaker}
			
			Various fault-intolerant multicast routing algorithms exist for many
			networks and a number have been proposed and evaluated specifically in the
			context of SpiNNaker.
			
			In 2012, Davies \emph{et al.} evaluated the use of three common torus
			routing algorithms in SpiNNaker and found that simple oblivious routing is
			suitable for typical neural applications \cite{davies12}. The three
			routing techniques are:
			
			\begin{description}
				
				\item[Dimension Order Routing (DOR)] Packets are routed along each
				dimension (e.g. $X$, $Y$ and $Z$) in turn until no further hops are
				available in that direction.  The order in which the dimensions are
				traversed is fixed.
				
				\item[Right Turn Only Routing (RTOR)] As in DOR except the dimension
				order is chosen such that routes only contain right-turns.
				
				\item[Longest Dimension First Routing (LDFR)] As in DOR except the
				dimension order is chosen in descending order of number of hops in each
				dimension.
				
			\end{description}
			
			These unicast routing techniques are converted into a multicast routing
			algorithm by merging together the routes produced between the source node
			and each destination node as illustrated in figure
			\ref{fig:simple-routers}.
			
			\begin{figure}
				\center
				\begin{subfigure}{0.3\linewidth}
					\center
					\buildfig{figures/simple-routers-dor.tex}
					
					\caption{DOR}
					\label{fig:simple-routers-dor}
				\end{subfigure}
				\begin{subfigure}{0.3\linewidth}
					\center
					\buildfig{figures/simple-routers-rtor.tex}
					
					\caption{RTOR}
					\label{fig:simple-routers-dor}
				\end{subfigure}
				\begin{subfigure}{0.3\linewidth}
					\center
					\buildfig{figures/simple-routers-ldfr.tex}
					
					\caption{LDFR}
					\label{fig:simple-routers-dor}
				\end{subfigure}
				
				\caption{Example multicast routes produced by merging together unicast
				routes from a central source node to each destination node.}
				\label{fig:simple-routers}
			\end{figure}
			
			In 2014, Navaridas \emph{et al.} introduced two new algorithms, `Enhanced
			Shortest Path Routing' (ESPR) and `Neighbourhood Exploring Routing' (NER)
			which produce multicast routing trees with fewer total hops
			\cite{navaridas14}. In both algorithms, routes are generated sequentially
			for each of the destinations of a route using LDFR. Unlike LDFR, however,
			these algorithms search a limited area of the network for other,
			already-connected destination nodes to which LDFR routes may be
			constructed. If no suitable destination is found, a LDFR route is
			constructed to the source node. Figure \ref{fig:search-regions} illustrates
			the shape of the searched regions of each algorithm. ESPR searches the
			trapezoidal region between the source and destination nodes while NER
			searches a fixed radius out from the destination node.
			
			\begin{figure}
				\center
				\begin{subfigure}{0.45\linewidth}
					\center
					\buildfig{figures/search-regions-espr.tex}
					
					\caption{ESPR}
					\label{fig:search-regions-espr}
				\end{subfigure}
				\begin{subfigure}{0.45\linewidth}
					\center
					\buildfig{figures/search-regions-ner.tex}
					
					\caption{NER}
					\label{fig:search-regions-espr}
				\end{subfigure}
				
				\caption{The ESPR and NER algorithms attempt to connect the node marked
				`D' to the closest node in the shaded region which is connected to the
				source node, `S'. If no connected node is found in the shaded region, the
				LDFR route is taken to `S'. The dotted line indicates the route chosen
				from `D'.}
				\label{fig:search-regions}
			\end{figure}
			
			Unfortunately none of these routing algorithms make any allowance for the
			avoidance of network faults. As a result their utility in real-world
			systems is limited.
		
		\subsection{Fault-tolerant routing}
			
			Numerous fault-tolerant routing algorithms have been proposed for
			super-computer networks. These algorithms are largely constrained by the
			need to maintain deadlock freedom. Since SpiNNaker's routers employ a
			timeout based deadlock-breaking strategy, much of this effort is
			unnecessary in SpiNNaker. As described below, this frequently renders
			existing fault-tolerant routing algorithms unnecessarily complex and
			inflexible.
			
			Deadlocks occur in a network if a cyclic dependency exists on any limited
			resource in the network. For example, as illustrated in figure
			\ref{fig:ring-deadlock}, in a ring network a deadlock may form when every
			node is waiting on the next node to accept a packet before accepting new
			packets from the previous node.
			
			\begin{figure}
				\center
				\buildfig{figures/ring-deadlock.tex}
				
				\caption{A deadlock in a ring network where each node is waiting for
				the next to accept a packet before accepting any further packets.}
				\label{fig:ring-deadlock}
			\end{figure}
			
			To prevent deadlocks, combinations of router microarchitectural features
			and routing restrictions may be employed. For example, a simple
			deadlock-free routing algorithm for mesh and torus networks mandates the
			use of DOR \cite{dally93}. Packets travelling in a -ve direction along
			each axis take priority over those travelling in a +ve direction. Packets
			travelling along the Y axis take priority over those travelling along the
			X dimension. Given these rules it is possible to define a total ordering
			on all hops in the network. Figure \ref{fig:deadlock-free-dor}
			illustrates a $3\times3$ mesh network whose hops have been numbered
			according to the total ordering defined above.  Any `X-then-Y' DOR route
			through this network results in the use of hops labelled with strictly
			increasing numbers. As a result, no cyclic dependencies (and thus no
			deadlocks) may occur.
			
			\begin{figure}
				\center
				\buildfig{figures/deadlock-free-dor.tex}
			
				\caption{Deadlock-free routing of two example routes using DOR in a 2D
				mesh topology. The numbers of the hops taken by each route are given on
				the right.}
				\label{fig:deadlock-free-dor}
			\end{figure}
			
			Unfortunately, the routing restrictions imposed to ensure deadlock
			freedom can result in fault-intolerant routing. In the example above, if
			the node at the bottom-right corner of the figure was faulty, the dotted
			example route would be blocked as no alternative routes are allowed.
			
			In practice, the routing rules used may be more relaxed, for example
			requiring that any route whose length is equal to a DOR must exist to
			guarantee routability \cite{rodrigo09}.
			
			Alternative routing strategies take a hybrid approach whereby an
			efficient but fault-intollerant routing algorithm is used where possible
			and in the presence of faults a less efficient but more robust strategy
			is employed. For example, the Immucube network architecture employs three
			virtual networks which operate independently over the same physical links
			\cite{puente07}. Initially messages are routed using a high-performance
			but potentially-deadlockable routing scheme in the first virtual network.
			If a deadlock is occurs, the deadlocked packet is dropped into the second
			virtual network in which packets are routed using a less efficient but
			deadlock-free but fault-intolerant routing algorithm. Finally, upon
			encountering a fault, packets are dropped onto the third virtual network
			which forms a ring network routing packets to every node in the network.
			
			Releated approaches \cite{mejia06,boppana95} divide the network into
			regions in which different routing rules are enforced to ensure deadlock
			freedom and, when required, fault tolerance.
			
			TODO FIGURE?
			
			The BlueGene/L supercomputer \cite{adiga02} uses DOR for its torus
			network and implements fault-tolerance by sacrificing otherwise
			functioning `lamb' nodes to ensure no route passes through a known dead
			link \cite{ho04}. In figure \ref{fig:lamb-nodes} an example scenario is
			shown where a single dead node is present and all nodes in the same row
			or column as the dead node have been made into lamb nodes. The lamb nodes
			may not be used in an application except as a through-route for other
			traffic. This pattern of lamb nodes guarantees that all dimension-order
			routes between all pairs of non-lamb nodes are not obstructed by the
			faulty node. This approach trades use of higher performance routing
			logic for wasted resources. This type of approach is most appropriate
			when algorithmic routing is used and routing rules are inflexible.
			
			\begin{figure}
				\center
				\buildfig{figures/lamb-nodes.tex}
				
				\caption{`Lamb' nodes may be disabled to ensure DOR will never
				encounter a fault.}
				\label{fig:lamb-nodes}
			\end{figure}
			
			Other algorithms proposed for the BlueGene architecture attempt to avoid
			the need for lamb nodes by generating routes which reach their destination
			via a `proxy' node \cite{gomez04}. By appropriately selecting the location
			of such a proxy, the existing routing algorithm used by the system can be
			guaranteed to select a route free of faults.
			
			TODO: EXAMPLE OF PROXY ROUTING TO AVOID FAULT
			
			Finally, many algorithms in in the field are distributed and use only local
			information along with limited information from their peers to generate
			routes \cite{fick09b}. In SpiNNaker, route generation is conventionally
			carried out centrally since no special on-chip hardware facilities exist
			for route generation. Centralised route generation also enables the routing
			algorithm to consider all available routes. As a result, there is little
			incentive for the use of distributed routing algorithms on SpiNNaker since
			global system information could be compactly shared for one-off routing
			passes.
			
			Algorithms for other architectures such as IP networks tend to be poor fits
			for static, regular network topologies since they use expensive graph-based
			algorithms for route discovery which aren't necessary here. They also tend
			to heavily feature graph topology discovery etc. which aren't needed here.
			
			Work on fault-tolerance in data centre networks does exploit the regularity
			of the network topology in routing algorithms \cite{guo08,liao12}.
			Unfortunately, the approaches used are not general enough to be applied to
			mesh-like topologies such as the one in SpiNNaker.
			
			Outside the field of computer networks, routing algorithms used to route
			wires across the surfaces of chips are required to solve similar problems
			to fault-tolerant network routing problems in mesh networks. Like mesh
			networks, the routes are defined within a regular Manhattan geometry and
			congested areas, rather than faults must be avoided by the algorithms
			\cite{kahng11}.  Unfortunately, these algorithms are designed for
			occasional batch operation prior to the multi-month process of chip
			manufacturing and so runtimes of hours or days are commonplace
			\cite{nam08}. As such these algorithms would be inappropriate for use
			with applications such as SpiNNaker where users' applications tend to be
			short-lived and thus routing should not be allowed to dominate runtime.
	
	\section{Partial graph search repair}
		
		In this section I introduce a novel post-processing algorithm, Partial
		Graph Search (PGS) repair, for routes produced by non-fault-tolerant
		routing algorithms.
		
		PGS repair guarantees routability for networks with no disconnected
		subregions by using a graph search algorithm to route around faults in the
		original route.  General-purpose graph search algorithms such as Breadth
		First Search (BFS), Dijkstra's Algorithm and A* are guaranteed to find
		shortest-path routes between pairs of points in arbitrary graphs. Such
		algorithms are generally a poor choice in highly regular network topologies
		such as meshes and toruses due to their high computational cost. In PGS
		repair, graph searching is only used for \emph{part} of the routing
		problem: to repair gaps in routes generated by more efficient routing
		algorithms.
		
		Real world super computer architectures are designed to ensure that faults
		are isolated \cite{gara05,alverson12} and thus tend to only impact a
		localised region of the network. Since PGS repair is only needed to route
		around these isolated faults, the space searched by the graph search
		algorithm should be very small in practice resulting in only short
		runtimes. In addition since faults are rare in real-world systems, the
		graph search process will only rarely be invoked.
		
		The PGS repair post-processing technique starts with a route produced by a
		non-fault-tolerant routing algorithm such as ESPR or NER. If this route is
		not obstructed by a fault, the algorithm terminates immediately without
		modifying the route. If the route attempts to use a faulty link, the
		algorithm proceeds as follows.
		
		The routing tree produced by the underlying routing algorithm is broken
		into subtrees wherever it attempts to route through a broken link and
		each subtree is assigned a unique colour, as illustrated in figure
		\ref{fig:pgs-repair-colouring}. From each disconnected subtree's root
		node in turn, a graph search is performed to find a short, fault-free
		route to a subtree node of a different colour. The subtree is then
		attached to the tree discovered by the graph search and re-coloured to
		match the tree it is connected to.
		
		\begin{figure}
			\center
			\begin{subfigure}{0.32\linewidth}
				\hspace*{-1.5em}
				\buildfig{figures/pgs-repair-colouring.tex}
				
				\caption{}
				\label{fig:pgs-repair-colouring}
			\end{subfigure}
			\begin{subfigure}{0.32\linewidth}
				\hspace*{-1.5em}
				\buildfig{figures/pgs-repair-colouring-fix1.tex}
				
				\caption{}
				\label{fig:pgs-repair-colouring-fix1}
			\end{subfigure}
			\begin{subfigure}{0.32\linewidth}
				\hspace*{-1.5em}
				\buildfig{figures/pgs-repair-colouring-fix2.tex}
				
				\caption{}
				\label{fig:pgs-repair-colouring-fix2}
			\end{subfigure}
			
			\caption{PGS repair process example showing a disconnected multicast
			route from A to B, C, D, E and F. $\times$ indicates a broken link.}
			\label{fig:pgs-repair-colouring-steps}
		\end{figure}
		
		For example in figure \ref{fig:pgs-repair-colouring-fix1} a path from the
		root of the subtree containing nodes E and F is found which connects it to
		the subtree rooted at A. Similarly in figure
		\ref{fig:pgs-repair-colouring-fix2} a path is also found connecting the
		subtree containing nodes C and D back to the subtree rooted at node A.
		
		If the routing tree was broken into $N+1$ subtrees by faults there will be
		$N$ subtrees disconnected from the root node. Each of the $N$ iterations of
		the algorithm connect a disconnected subtree to another subtree reducing
		the number of subtrees by $1$ each time. After $N$ iterations, therefore,
		exactly $1$ subtree remains which connects every node in the original
		routing tree without traversing faulty links.
		
		TODO: EXPLAIN THE FIDDLINESS HERE TO ENSURE WE DON'T CREATE LOOPS.
		
	\section{Evaluation \& Results}
		
		The PGS repair technique, by design, is able to work around all possible
		fault patterns which don't completely disconnect parts of the network. This
		result this evaluation focuses on the impact on performance PGS repair
		imposes. The metrics of interest in this evaluation are:
		
		\begin{itemize}
			\item Algorithm runtime
			\item Network congestion
			\item Routing table utilisation
		\end{itemize}
		
		\subsection{Traffic Patterns}
			
			In this evaluation, two standard benchmark multicast traffic patterns are
			used which have been used in previous research into SpiNNaker's network:
			
			\begin{figure}
				\center
				\buildfig{figures/traffic-distribution-centroids.tex}
				
				\caption{An example 4-centroid distribution with four centroids. The
				$\times$ marks the location of the origin node. Lighter colours
				indicate greater likelihood of a connection.}
				\label{fig:traffic-distribution-centroids}
			\end{figure}
			
			\begin{description}
				
				\item[Uniform] Destinations are chosen with uniform probability
				anywhere in the machine.
				
				\item[$N$-Centroids] Destinations are clustered around one of $N$
				randomly chosen `centroids' as illustrated in figure
				\ref{fig:traffic-distribution-centroids}.
				
			\end{description}
			
			The uniform traffic pattern is widely used in networks research
			\cite{dally04,davies12} while the centroids model was developed
			specifically to reproduce the traffic patterns found in the neural
			applications SpiNNaker is designed for \cite{navaridas14}. In this work
			we consider 3 centroids.
		
		\subsection{Fault model}
			
			In addition two different fault models are used which are representative of
			the faults found in real SpiNNaker systems:
			
			\begin{figure}
				\center
				\begin{subfigure}{0.48\linewidth}
					\hspace*{-1.5cm}
					\buildfig{figures/fault-example-uniform.tex}
					
					\caption{Uniform}
					\label{fig:fault-example-uniform}
				\end{subfigure}
				\begin{subfigure}{0.48\linewidth}
					\hspace*{-1.5cm}
					\buildfig{figures/fault-example-hss.tex}
					
					\caption{HSS Link}
					\label{fig:fault-example-hss}
				\end{subfigure}
				
				\caption{The two link fault models considered.}
				\label{fig:fault-example}
			\end{figure}
			
			\begin{description}
				
				\item[Uniform] Links are selected and disabled at random (figure
				\ref{fig:fault-example-uniform}).
				
				\item[HSS Link] Groups of links corresponding with randomly selected
				single High-Speed Serial (HSS) link between SpiNNaker boards are disabled
				together (figure \ref{fig:fault-example-uniform}).
				
			\end{description}
			
			The uniform link failure model models isolated failures resulting from
			isolated manufacturing defects in individual links. The HSS Link failure
			model models faults arising from failing or disconnected board-to-board
			links which carry several chip-to-chip traffic flows via a single cable in
			SpiNNaker systems. Though SpiNNaker-specific, the later fault model is
			analogous to failure modes arising in other architectures where a single
			fault may render several links impassable in a single area.
			
			A range of failure rates are explored in this section. My measurements of
			current large-scale SpiNNaker installations the link failure rate is about
			\SI{0.03}{\percent} with failures due to both individual chip-to-chip links
			and board-to-board HSS links. Exact link failure statistics for commercial
			super computer installations are not widely available, however, published
			Mean-Time-Between-Failure (MTBF) statistics place an upper bound on link
			failure rates at a similar \SI{0.03}{\percent} in one-year-old BlueGene/Q
			systems \cite{chiu11}.
			
			Unfortunately presently undiagnosed problem with the SDRAM packaged with
			approximately \SI{1}{\percent} of SpiNNaker chips has rendered these chips
			unusable for most applications. The gaps in the network resulting from the
			loss of these chips currently dominate true link failures leaving just over
			\SI{1}{\percent} of links inoperable.
			
			Surprisingly, research into fault tolerant routing in super computers
			appears to focus on benchmarks with even higher fault rates ranging from
			\SI{3}{\percent} to as high as \SI{7}{\percent}
			\cite{ho04,gomez04,mejia06}.
			
			In this evaluation, fault rates ranging from \SI{0.01}{\percent} to
			\SI{5}{\percent} are considered to cover both realistic fault levels
			along with the more extreme cases considered in related work.
		
		\subsection{Base routing algorithm}
			
			Since the PGS repair process is routing algorithm agnostic all
			experiments use the NER algorithm which has been found to be appropriate
			for SpiNNaker applications \cite{navaridas14}.
		
		\subsection{Algorithm runtime}
			
			To assess the impact of the PGS repair process on routing algorithm
			runtime, the algorithm was used to process a large number of randomly
			generated routing problems and the runtime recorded.
			
			\num{10000} one-to-sixteen multicast routing problems were generated in a
			$256\times256$ hexagonal torus topology, the largest size possible for a
			SpiNNaker system. Other quantities of multicast destinations were also
			evaluated but are omitted for brevity since the pattern of results are
			similar to those outlined here.
			
			TODO: APPENDIX WITH OTHER RUNS?
			
			The NER and PGS repair algorithms were written in C and compiled with GCC
			4.8.3 with \verb|-O2| level optimisations and executed on a cluster of
			idle workstations with 3.10 GHz Intel Core-i5-2400 CPUs.
			
			\begin{figure}
				\center
				\buildrplot{figures/routing-runtimes.R}
				
				\caption{Mean runtime of routing and PGS repair overhead. PGS repair
				overhead is stacked above the routing runtime (i.e. bars do not
				overlap). Error bars indicate 95\% confidence interval. Note different
				Y-scale for HSS link and uniform fault models.}
				\label{fig:routing-runtimes}
			\end{figure}
			
			Figure \ref{fig:routing-runtimes} shows the average runtimes recorded for
			both the NER and PGS repair algorithms. In fault-free networks the
			PGS-repair post-processing step is not required and incurs no penalty
			while the runtime of the algorithm grows with the fault rate for both
			fault and traffic models.
			
			Notably the HSS fault model results in longer runtimes for the PGS repair
			process compared with an equivalent fault-density of uniform faults.
			Because the HSS fault model produces contiguous lines of faults the PGS
			repair algorithm must construct a longer path to avoid the fault.  Since
			the space explored by a graph algorithm typically grows with $O(H^2)$
			with respect to the hops in the discovered route, $H$, this increase in
			search distance has a large impact on the runtime of the PGS repair
			process.
			
			The runtime of the PGS repair algorithm remains roughly in proportion to
			the runtime of the underlying routing algorithm with respect to different
			traffic models. The centroid traffic pattern tends to result in routes
			with fewer hops than a uniform traffic pattern with the same number of
			destination nodes as segments of routes are often shared between
			destination nodes. Since the NER algorithm's runtime is strongly related
			to the number of hops in the output route the runtime of the algorithm is
			greater for uniform traffic. Likewise the probability of PGS repair being
			required increases with the number of hops in route and hence the runtime
			of the PGS repair algorithm increases roughly in proportion.
		
		\subsection{Routing table usage}
			
			In order to gain a realistic measure of routing table usage it is
			necessary to determine the effect of many routes being generated for a
			single set of faults. To enable a sufficiently large number of sample to
			be collected the experimental setup considered previously is reduced to a
			network containing $48\times48$ nodes.
			
			\num{1000} $48\times48$ node network models are produced according to the
			HSS link and uniform fault models. For each of these models
			$48\times48\times16=$~\num{36864} one-to-sixteen routes are generated using
			the centroid and uniform traffic models. This corresponds to one
			multicast route per application core. As is convention in SpiNNaker,
			routing table entries are inserted for each route at the source of the
			route, at each destination and at each corner or fork. The number of
			routing table entries at each node in the model is counted and the
			maximum number of entries in a single node is reported for each network
			model.  The \emph{maximum} number of routing entries of any router was
			chosen since the number of entries available per SpiNNaker router is
			bounded by hardware.
			
			\begin{figure}
				\center
				\buildrplot{figures/routing-entries.R}
				
				\caption{Violin plot showing the distribution of maximum table sizes
				for \num{1000} random networks. The red line at \num{1024} entries
				indicates the size of SpiNNaker's routing tables.}
				\label{fig:routing-entries}
			\end{figure}
			
			
			Figure \ref{fig:routing-entries} shows the distributions of the largest
			routing table sizes for each fault and traffic model.
			
			\begin{figure}
				\center
				\begin{subfigure}{0.48\linewidth}
					\center
					\buildfig{figures/hss-link-routing-table-usage.tex}
					
					\caption{Routing table entries}
					\label{fig:hss-link-routing-table-usage}
				\end{subfigure}
				\begin{subfigure}{0.48\linewidth}
					\center
					\buildfig{figures/hss-link-resource-usage.tex}
					
					\caption{Routes passing through chip}
					\label{fig:hss-link-resource-usage}
				\end{subfigure}
				
				\caption{The impact of a HSS link fault on routing table usage and
				congestion. Each hexagon represents a single chip, the red line
				indicates the chip-to-chip connections broken by the HSS link fault.}
				\label{fig:hss-link-usage}
			\end{figure}
			
			The HSS link failure model has a much greater impact on peak routing
			table resource usage than uniform link failures for a given fault rate.
			This is because HSS link faults result in a large concentration of routes
			being disrupted and then re-routed around the same obstacle in a single
			location. Figure \ref{fig:hss-link-routing-table-usage} shows how routing
			table usage varies around a HSS link fault in one instance of the
			experiment. There are clear peaks in routing table usage around the ends
			of the line of faults which result from routes produced by PGS repair
			finding shortest paths around the edge of the faults.
		
		\subsection{Network congestion}
			
			To measure the impact of PGS repair on network congestion, two
			experiments were performed, one using the same model used to measure
			routing table usage and one based on tests run on SpiNNaker hardware.
			
			For each of the network fault and traffic pattern described previously,
			the paths taken for the \num{36864} one-to-sixteen multicast routes
			generated are used to compute the number of times each link in the
			network is used. The number of routes passing through the most-used link
			is then recorded, giving an indication of the level of congestion in the
			network.
			
			\begin{figure}
				\center
				\buildrplot{figures/routing-resource.R}
				
				\caption{Violin plot showing the distribution of maximum
				routes-per-chip for \num{1000} random networks.}
				\label{fig:routing-resource}
			\end{figure}
			
			The results are presented in figure \ref{fig:routing-resource} and follow
			the same trends as the results previously shown for routing table usage.
			Again, HSS link faults result in routes with the greatest congestion due
			to the concentration of routes finding shortest paths around an obstacle
			(see \ref{fig:hss-link-resource-usage}).
			
			To verify that the results above, an additional experiment has been
			carried out which attempts to mimic the model used previously in actual
			SpiNNaker hardware. In these experiments a large SpiNNaker machine is
			divided into independent 48-board (2304-chip) sections. Because the
			48-board systems used in these experiments are cut out of a larger
			machine, they lack wrap-around links and thus form hexagonal mesh
			topologies, rather than hexagonal toruses.
			
			Due to the SDRAM issue described above, fault rates below
			\SI{1}{\percent} cannot be modelled.  To simulate higher fault rates,
			additional links are disabled in software according to the fault models
			described used previously. Since some faults are due to genuine hardware
			faults, these faults cannot be placed randomly in each experiment. To
			reduce, bias each combination of fault rate, fault model and traffic
			pattern is repeated XXX times across randomly chosen physical machines.
			
			XXX 1-to-XXX routes are generated in both uniform and XXX-centroid
			distributions as used throughout this evaluation. Synthetic network
			traffic is generated at the source of each route following a Bernoulli
			distribution. Traffic consumers running on all destination nodes accept
			packets as quickly as possible from the network and log their arrival.
			The Bernoulli probability is set the same for every route's traffic
			generator and increased in steps of XXX and the number of packets dropped
			in an XXX second period logged. The network is considered saturated once
			less than \SI{99}{\percent} of packets successfully arrive at their
			destination.
			
			Figure \ref{XXX} shows the distributions of the saturation points for
			each experimental configuration.
			
			TODO: ANALYSIS
		
	\section{Conclusions}
		
		In this chapter I described how SpiNNaker's unconventional network and
		router architecture render existing fault tolerant routing algorithms
		unsuitable. I introduced PGS repair, a post-processing technique for
		existing non-fault tolerant routing algorithms designed for SpiNNaker such
		as NER.
		
		Unlike some other fault tolerant routing algorithms for other
		architectures, PGS repair is able to work-around arbitrary fault patterns
		by exploiting SpiNNaker's inbuilt deadlock avoidance mechanisms. In the
		presence of realistic failure rates of up to \SI{1}{\percent}, only small
		overheads of up to XXX, XXX and XXX for in algorithm runtime, routing table
		usage and network performance are incurred respectively. This low
		performance overhead makes PGS repair appropriate for use in real
		applications. At the time of writing the algorithm has been successfully
		used in a number of neural and non-neural SpiNNaker applications.
		
		At more extreme fault rates not expected in real-world systems, the
		algorithm still functions correctly but the results incur much greater
		routing table and congestion overheads, particularly when faults are
		concentrated. Future extensions to this algorithm might aim to reduce this
		overhead by producing longer and more varied routes around faults to even
		out the load.

	\chapter{Placing applications in large SpiNNaker machines}
	
	In the previous chapter I tackled the problem of scale in generating routes
	for very large networks such as SpiNNaker. In this work the centroid traffic
	pattern was used as an approximation of the expected network traffic
	generated by `well behaved' neural network simulation software running on
	SpiNNaker. The traffic produced largely exhibits strong locality, that is
	most communication occurs between either nearby nodes or clusters of nodes.
	In reality, neural simulation applications are not specified geometrically
	but rather as abstract graphs of communicating neurons
	\cite{davison08,eliasmith13}. Applications must then \emph{place} these
	neurons onto nodes in a SpiNNaker system, attempting maximise communication
	locality.
	
	In this chapter I re-evaluate the suitability of simulated annealing as a
	technique for finding high quality placements for large parallel
	applications. Though this technique had fallen out of fashion in the field of
	application placement by the early 1990s, it has found wide use for placing
	components in computer chip and FPGA designs. In the intervening years,
	placement problems in super computers have grown in size from tens or
	hundreds of nodes to millions, a scale at which chip placement techniques
	were operating in the mid 1990s. I adapt the simulated annealing algorithm
	used by the VPR academic circuit placement software to produce placements for
	applications running on SpiNNaker. In that in a range of real and synthetic
	benchmarks simulated annealing produces high quality placements enabling
	efficient use of SpiNNaker's network resources.
	
	
	%In the field of chip design, Moore's `Law' \cite{moore65,moore75} observes a
	%similar exponential growth in the number of components within a single chip.
	%Today modern processors contain billions of components and an analagous
	%placement problem exists in attempting to place interconnected components
	%near to eachother. In this chapter I explore the techniques used for circuit
	%placement and adapt one such technique, Simulated Annealing (SA)
	%\cite{kirkpatrick83}, for use in application placement. Despite some early
	%interest in SA for application placement in the 1980s and early 1990s, the
	%technique has since fallen out of favour. I find that at the scales of modern
	%placement problems SA-based placement is able to produce solutions of
	%superiour quality to contemporary methods.
	%
	%TODO: SUMMARISE RESULTS...
	
	\section{Related work}
		
		The placement problem has been tackled independently in the literature by
		researchers in both the application and chip placement communities. In this
		survey I cover application and chip placement separately as these two
		communities have remained largely isolated from one another. First I
		explore the techniques applied to application placement before moving on to
		contrast this with the techniques used in circuit placement.
		
		In the application placement literature, the placement problem is often
		referred under the umbrella term `mapping'. Unfortunately term is often
		used more broadly to include other tasks such as routing and application
		partitioning. To avoid ambiguity I use the term `placement', as preferred
		by the chip and FPGA design communities, to refer specifically to the
		problem of assigning nodes in an application's communication graph to nodes
		in a machine's connectivity graph.
		
		\subsection{Application placement algorithms}
			
			TODO: GENERAL INTRO
			
			\subsubsection{Application-specific approaches (manual placement)}
				
				In the case of some applications such as finite element modelling
				\cite{bermejo13}, the structure of the problem itself leads to a
				natural placement of the computation on nodes in a machine. For example
				when simulating a 3D volume in an node super computer with a $3 \times
				4 \times 2$ 3D torus or mesh topology network, the modelled volume
				might be divided into as in figure \ref{fig:fem-partitioning}. Each
				cuboid in the model is then assigned to the corresponding node in the
				network topology.
				
				\begin{figure}
					\center
					\buildfig{figures/fem-partitioning.tex}
					
					\caption{Example partitioning of a 3D space to fit into a super
					computer with a $3\times4\times2$ torus or mesh topology.}
					\label{fig:fem-partitioning}
				\end{figure}
				
				When the number of dimensions in a problem do not match that of the
				underlying network architecture, the common solution is to either
				divide only along a subset of the axes or to divide into additional
				pieces on the existing axes \cite{gilge14}.
			
			\subsubsection{Sequential placement}
				
				In the case where a placement solution is non-obvious one of the
				simplest and most popular strategies is to apply a simple sequential
				placement algorithm. Sequential placement algorithms function by
				iterating over the vertices in the application's communication graph
				and assigning them to a free node in the target machine. Sequential
				placement algorithms are differentiated by the order in which they
				iterate over vertices in the communication graph and fill nodes in the
				target machine. A number of widely used orderings are described below.
				
				\begin{figure}
					\center
					\begin{subfigure}{0.32\linewidth}
						\center
						\buildfig{figures/sequential-row-order.tex}
						\caption{Row-order}
						\label{fig:sequential-row-order}
					\end{subfigure}
					\begin{subfigure}{0.32\linewidth}
						\center
						\buildfig{figures/sequential-alternating.tex}
						\caption{Alternating}
						\label{fig:sequential-alternating}
					\end{subfigure}
					\begin{subfigure}{0.32\linewidth}
						\center
						\buildfig{figures/sequential-hilbert.tex}
						\caption{Hilbert curve}
						\label{fig:sequential-hilbert}
					\end{subfigure}
					
					\caption{Space-filling curves in 2D mesh and torus topologies.}
					\label{fig:sequential}
				\end{figure}
				
				Super computer management software such as SLURM \cite{yoo03} and Blue
				Gene's system software \cite{gilge14} by default na\"ively iterate over
				vertices in an application communication graph in the order they are
				provided. The nodes in the target machine are then iterated over in a
				simple space-filling curve through the network topology. Figure
				\ref{fig:hilbert-placement} illustrates the default patterns available
				in these software packages. The row-order (figure
				\ref{fig:sequential-row-order}) and alternating (figure
				\ref{fig:sequential-alternating}) curves correspond with 2D versions of
				the default node assignment orders used in SLURM and BlueGene systems.
				
				\begin{figure}
					\center
					\buildfig{figures/hilbert-placement.tex}
					
					\caption{A Hilbert curve, coloured from blue to red.}
					\label{fig:hilbert-placement}
				\end{figure}
				
				The Cray extensions to SLURM software provide a Hilbert curve
				\cite{hilbert91} (figure \ref{fig:sequential-hilbert}) node assignment
				order. Unlike the row-order and alternating space filling curves the
				Hilbert curve ensures that pairs of vertices close together in the node
				iteration order are also close together in the target machine's network
				\cite{moon01, zumbusch99}. Figure \ref{fig:hilbert-placement} shows a
				5$^\textrm{th}$-order Hilbert curve where each point in the curve is
				coloured according to its position along the curve. In this figure it
				is possible to see that nearby positions in the curve (which share
				similar colours) are also close in 2D space.
				
				When the proximity of vertices in the vertex-ordering supplied by an
				application is a good estimator of those vertices communication
				requirements, the sequential assignment schemes discussed above can be
				very effective. These techniques have also proven adequate in
				small-scale and densely connected applications such as early neural
				simulations running on prototype SpiNNaker machines with tens of nodes
				\cite{galluppi10} but growing beyond this scale has proven problematic.
				
				\begin{figure}
					\center
					\begin{subfigure}{0.45\linewidth}
						\center
						\buildfig{figures/rcm-initial.tex}
						
						\caption{Original permutation}
						\label{fig:rcm-initial}
					\end{subfigure}
					\begin{subfigure}{0.45\linewidth}
						\center
						\buildfig{figures/rcm-sorted.tex}
						
						\caption{RCM permutation}
						\label{fig:rcm-sorted}
					\end{subfigure}
					
					\caption{Adjacency matrix representation of a graph before and after
					permutation by the RCM algorithm.}
					\label{fig:rcm}
				\end{figure}
				
				A number of algorithms have been proposed for automatically selecting
				good vertex iteration orders, typically using a graph-traversal based
				heuristic. A typical method, described by Hoefler \emph{et al.}
				\cite{hoefler11} exploits the Reverse-Cuthill-McKee (RCM) algorithm
				\cite{cuthill69}. An application's communication matrix is represented
				as an adjacency matrix, $M$, where $M_{i,j}$ is 1 if node $i$ is
				connected by an edge to node $j$ and 0 otherwise. An example matrix is
				illustrated in figure \ref{fig:rcm-initial}. The RCM algorithm uses a
				simple heuristic to permute the matrix (i.e. renumber the nodes in the
				graph) in order to reduce the bandwidth of the matrix. Figure
				\ref{fig:rcm-sorted} shows the RCM-permuted version of the example
				adjacency matrix. When a graph's vertices are ordered as in a
				bandwidth-reduced sparse matrix, vertices close together in the
				ordering are likely to communicate while those further apart tend not
				to communicate.
				
			\subsubsection{Optimisation-based Placement}
				
				% Citations from short report about optimisation in placement...
				% \cite{chen06,jeannot14} and \cite{jeannot10} ("subsets of apps")
				
				In the academic community, a number of attempts have been made to use
				more sophisticated optimisation algorithms for the placement of
				applications. In 1985, Steele \cite{steele85} proposed the use of
				simulated annealing for placing applications in the 6D torus topology
				of the 64 node `Caltech Cosmic Cube' machine. Simulated annealing,
				originally developed by Kirkpatrick \emph{et al.} \cite{kirkpatrick83},
				is a general-purpose optimisation algorithm which works by analogy to
				the physical process of annealing. In brief simulated annealing
				functions by randomly swapping vertices in a candidate placement
				solution, accepting swaps which move connected vertices closer together
				and rejecting some proportion of swaps which move connected vertices
				further apart. The simulated annealing algorithm is described in detail
				later in this chapter.
				
				Towards the end of the 1980s, application placement appeared to be
				becoming less important as super computer network architectures
				improved:
				%
				\begin{displayquote}
					``Careful placement was necessary because of the slow communication
					and non-uniform addressing of early concurrent computers. However,
					the development of message passing machines with fast communications
					and a uniform global address space  has made placement less of an
					issue. In such machines a random placement performs nearly as well as
					an optimum placement.''
					
					\noindent --- W. Dally, 1987 \cite{dally87}
				\end{displayquote}
				%
				In addition, network and problem sizes remained small, so small in fact
				that linear-programming based optimal placement still appeared in
				benchmarks comparing placement algorithms \cite{xu91}. In this
				environment, simpler sequential placement algorithms gained favour over
				more computationally expensive algorithms such as simulated annealing.
				
				As problem and machine sizes have grown and network utilisation has
				once again become an important factor in application performance
				\cite{navaridas09b} more complex optimisation algorithms have
				reappeared in the literature. One popular approach employs graph
				partitioning algorithms such as METIS \cite{karypis98} to perform
				recursive bipartitioning based placement
				\cite{phillips14,hoefler11,pellegrini96}.  This placement process is
				illustrated in figure \ref{fig:partitioning}.
				
				In the first step, the application communication graph and machine
				connectivity graph are bipartitioned such that the number of edges
				between partitions is minimised. Each half of the communication graph
				is associated with one of the halves of the machine connectivity graph.
				The partitioning process is then repeated recursively on each of the
				two communication and connectivity graph pairs. The process halts when
				the graphs can no longer be partitioned at which point the vertices in
				the communication graph are placed on their associated node.
				
				\begin{figure}
					\center
					\buildfig{figures/partitioning.tex}
					
					\caption{Illustration of application placement by recursive
					partitioning.}
					\label{fig:partitioning}
				\end{figure}
				
				TODO: PARTITIONING IS GREAT AND ALL BUT QUALITY ISN'T ALWAYS GREAT AND
				IT DOESN'T DEAL WELL WITH MULTI-CONSTRAINT SCENARIOS E.G. PROCESSOR AND
				MEMORY RESTRICTIONS.
				
				Unfortunately, many of these simply aren't suited to the scale of
				neural applications running on SpiNNaker (e.g. only cope with tens of
				nodes while SpiNNaker may contain hundreds of thousands).
				
				Additionally, a number of algorithms have been developed which make
				assumptions about the topologies of the problem or network. Tree match
				for example attempts to map tree-shaped problems to tree-shaped
				networks. Such algorithms can be highly effective but again do not
				apply to SpiNNaker or its neural applications.
		
		\subsection{Chip placement algorithms}
			
			The chip-design industry has, for many years, dealt with problems
			analogous to the task of placing super computer jobs in a way suited to
			SpiNNaker. Modern CPUs have millions or billions of components with
			strictly fixed connectivity. CPU designers must place each of these onto
			a chip such that the connection lengths are controlled to reduce
			congestion and increase performance. As such, these algorithms are
			ideally suited to future super computer placement work since they already
			operate at the scales required \cite{nam07}.
			
			\subsubsection{Cost functions}
				
				HPWL is popular but a bit crap for high fan-outs. It is, however, quite
				simple.
				
				TODO: SELECT A BETTER COST FUNCTION...
			
			\subsubsection{Simulated annealing}
				
				One of the oldest techniques used for circuit placement is simulated
				annealing and this remains popular today thanks to its sheer
				versatility (see VPR, other open FPGA tools).
				
				SA works by analogy with the physical process of annealing.
				The simulated annealing algorithm works by selecting random pairs of
				components on a chip, swapping them and evaluating some cost function.
				If the swap reduces the cost function, it is kept, if not, depending on
				a function of the current temperature and the cost introduced by the
				swap.
				
				TODO: ILLUSTRATION OF SIMULATED ANNEALING SWAP OPERATION
				
				By occasionally allowing costly swaps, the annealing algorithm avoids
				becoming trapped in local minima. As the algorithm proceeds, the
				temperature is slowly reduced and with it the proportion of costly
				swaps which are retained. This causes the placement to move from
				exploration early on towards refinement later on.
				
				The temperature schedule of an annealing algorithm is critical to its
				success. In general these schedules are computed based on the
				performance of the algorithm as it runs. In VPR the following schedule
				is used.
				
				TODO: DESCRIBE VPR'S SCHEDULE
				
				TODO: FIND AND DESCRIBE ALTERNATIVE SCHEDULE?
				
				Unfortunately, SA is very difficult to parallelise, especially in the
				case of placement. As a result, its scalability has been limited and
				resulted in significantly reduced usage in recent work.
			
			\subsubsection{Partitioning placement}
				
				Partitioning based placement solves the placement problem using
				graph-partitioning recursively on the problem graph to assign each part
				of the circuit to some area in the super chip. Though a number of
				algorithms have proven successful in academic placement contests over
				the years, they are not popular in industrial settings.
			
			\subsubsection{Analytical placement}
				
				In analytical placement, cost function for the circuit graph is
				approximated in a form which is amenable to solutions with standard
				numerical or symbolic algebraic techniques. Using these techniques,
				exact minimum cost (in terms of the approximation) configurations can
				be obtained.
				
				Quadratic placement is a popular analytical placement technique which
				approximates the cost of a placement as the sum of the squares of the
				distances between connected circuit elements.
				
				TODO: FIGURE EXAMPLE QUADRATIC PLACEMENT PROBLEM AND SOLUTION
				
				As such this gives a quadratic cost function like so which we must
				minimise.
				
				TODO: QUADRATIC COST EQN
				
				To minimise the function we differentiate and solve using simple
				symbolic manipulation.
				
				TODO: QUADRATIC COST SOLUTION
				
				Unfortunately, quadratic placement doesn't contain any congestion
				relief by default so various schemes exist. For example, extra anchor
				nodes are inserted which gently pull the circuit components apart from
				each other. As a result, the algorithm generally proceeds by iterating,
				regenerating anchors each time.
				
				Other non-quadratic analytical methods exist too with numerical
				solutions. The approaches are often similar.
			
			\subsubsection{Hierarchical clustering}
				
				Many placement algorithms scale super-linearly with problem size and so
				larger problems become increasingly problematic to handle. To solve
				this problem clustering techniques are first applied to first simplify
				the placement problem. A solution is then found at the coarse level and
				then hierarchically fleshed out.
				
				Various clustering algorithms are in use.
				
				TODO: TALK ABOUT CLUSTERING IN PLACEMENT...
				
				TODO: DESCRIBE THE ALGORITHM I IMPLEMENTED.
	
	\section{Application placement by simulated annealing}
		
		\label{sec:placement-by-annealing}	
		
		I have implemented a simplified SA based application placement algorithm
		based on the approach used in the popular VPR place and route tool chain.
		The algorithm is written in C and is optimised for experimentation rather
		than performance but is production-ready. It has been integrated into the
		`Rig' SpiNNaker software tools and has been used to place very large
		simulations. More on that later.
		
		\subsection{Representation}
			
			Model each chip as having a quantity of various resources (e.g. Cores,
			SDRAM) available. The application graph consists of vertices which each
			consume some quantity of these resources. Vertices must be placed on a
			single chip such that the resources required on a given chip do not
			exceed those available. Vertices are then interconnected by 1:N nets with
			weights which act as hints. The nets are treated as a soft constraint:
			vertices connected via a net will, ideally, be placed near to each other,
			with priority being given to nets with higher weights. Additionally there
			will be a list of placement constraints (see later).
			
			A key observation is that while vertices in an application may frequently
			have a 1:1 correspondence with application cores, this need-not be the
			case. For example, a vertex may represent a block of SDRAM which is
			shared. A vertex may also represent some other resource, for example,
			external IO availability. By making these resource types user-defined,
			applications programmers can express flexible hard-constraints on their
			application.
			
			Another observation is that generic soft constraints can be expressed may
			be expressed using a net with an appropriate weight.
			
			As a result of these facilities, application programmers can easily
			express their own application-specific hard and soft placement
			constraints without having to modify the algorithm. This representation
			has become a de-facto standard for placement problem interchange for
			SpiNNaker applications.
		
		\subsection{Cost function}
			
			At present I've used HPWL despite this being really bad for high-fan-out
			multicast and totally ignorant to the hexagonal nature of SpiNNaker...
			
			To compute bounding boxes for tori I use the following approach. For each
			dimension, sort the points on that dimension and find the largest gap
			between them on a ring. The bounding box goes the other way.
			
			TODO: FIGURE ILLUSTRATING BOUNDING BOX COMPUTATION FOR TORI.
		
		\subsection{Annealing schedule}
			
			The annealing schedule is that used by VPR. Despite being for circuit
			placement, it seems to work jolly well.
			
			TODO: DESCRIBE AND RATIONALISE THE SCHEDULE
		
		\subsection{Constraint handling}
			
			Various hard and soft constraints may be expressed by software
			approaches. For each we explain how they may be handled by the placement
			algorithm:
			
			\subsubsection{Location Constraint}
				
				The vertex is placed on a chip and removed from the set of movement
				candidates.
			
			\subsubsection{Same-chip constraint}
				
				When two vertices must always be placed on the same chip they are
				simply combined into one vertex which consumes the sum of their
				resources. Placement then treats them as one chip and thus is forced to
				atomically place the vertices.
			
			\subsubsection{Reserve resource constraint}
				
				Simply reduce resource availability on that chip.
			
			\subsubsection{Keep near Ethernet}
				
				Simply add a net.
	
	\section{Evaluation}
		
		\label{sec:placement-results}
		
		Though benchmarks exist for super computer loads and chip placement tasks,
		such things don't exist for neural applications. As a result I use a
		selection of real applications for SpiNNaker along with some synthetic
		benchmarks based on biological data.
		
		\subsection{Benchmark networks}
			
			First some real networks.
			
			Some nengo networks: SPAUN: `The world's largest functional brain model'.
			Word-net network from Jamie: Example of some learning.
			
			TODO: DESCRIBE SHAPE OF NENGO NETWORKS
			
			Some PyNN networks: Microcortical column model from PyNN. Note almost
			broadcast connectivity but varying weights. Try and extract a vision
			netlist from Anna. Maybe try and get a netlist for Tom's barrel cortex.
			
			Now for some artificial networks. Pipeline, noisy pipeline, mesh,
			Gaussian 2D.
		
		\subsection{Experiments}
			
			We compare random, linear, greedy and annealing based placement
			approaches to placement. We compare static metrics (such as mean/max
			congestion, table usage) along with experiments based on simulated
			network traffic in real hardware. Network Tester generates artificial
			traffic in proportion with the weights given for each model. We compare
			the relative level of traffic sustainable. We also consider use of
			machines of various sizes.
		
		\subsection{Results}
			
			SA placement is slow but rather effective, especially for some networks.
			Generally worth doing. Will need to be sped up for very large machines...
			
			TODO: EXPERIMENTS!
	

	\chapter{Conclusions and future research}
	
	The SpiNNaker architecture was designed to tackle the challenges presented by
	the simulation of biologically realistic neural networks. One of its
	distinguishing features is its network architecture which employs both an
	unconventional network topology and multicast router architecture. The
	hexagonal torus topology used by SpiNNaker was chosen to enable greater
	performance while maintaining ease of construction and scalability compared
	with conventional network topologies. SpiNNaker's router design centres
	around packets mimicking the neural `spike' signals they are designed to
	convey by being small, multicast and not guaranteed to arrive at their
	destination.  This novel design, though largely complete before this work
	began, left a number of open problems which this thesis has attempted to
	address.
	
	In this concluding chapter I begin by summarising the answers to the research
	questions raised in chapter~\ref{sec:introduction}. This is followed by a
	discussion of new research topics which have been uncovered by this work.
	
	\section{Answers to research questions}
		
		Each of the three research questions are answered below.
		
		\subsubsection{1. Can the hexagonal torus topology be deployed and used in
		real, large-scale systems?}
		
		In chapter~\ref{sec:building}, I introduced a cabling scheme and assembly
		technique which has been used successfully to build a prototype SpiNNaker
		system with over half a million processor cores using the hexagonal torus
		topology. The techniques shown are expected to enable a final SpiNNaker
		machine of double this size to be built, filling a six metre long row of
		machine-room cabinets.
		
		Though SpiNNaker's processor-count places it amongst some of the world's
		largest supercomputers (see figure \ref{fig:top500-num-processors} on page
		\pageref{fig:top500-num-processors}), it is comparatively compact, filling
		one row of cabinets compared with the warehouse-scale installations found
		in commercial systems. In spite of this, the folding and interleaving
		techniques described allow hexagonal torus topologies to scale to
		arbitrarily large installations without cables which span the machine.
		
		Chapter~\ref{sec:shortestPaths} described an efficient and general
		technique for finding, and enumerating shortest path vectors in hexagonal
		torus topologies. These developments bring the hexagonal torus topology in
		line with other topologies by enabling routing algorithms to exploit all
		possible paths in a network. Further, chapter~\ref{sec:placement}
		demonstrated that placement algorithms are also adaptable to hexagonal
		torus topologies thanks to their similarity to 2D toruses.
		
		Though, as this thesis highlights, hexagonal toruses lack many of the
		intuitive properties enjoyed by other topologies, it is still possible to
		reason about them with only limited computational effort.  Now that the
		practicality and scalability of the topology has also been demonstrated in
		practice, it represents a credible option for future network architectures.
		
		\subsubsection{2. Does SpiNNaker's router architecture help, or hinder
		fault tolerance?}
		
		SpiNNaker's unconventional use of packet dropping to avoid deadlocks
		greatly simplifies the router architecture, part of the motivation for this
		design. In chapter~\ref{sec:routing} this feature is used to the advantage
		of PGS repair to add fault tolerance to existing routing algorithms.
		Compared with the often complex and wasteful methods used to tolerate
		faults in other networks, PGS repair incurs very little performance
		overhead in the presence of static faults.
		
		Routing table usage does increase in the presence of faults, however, which
		may be a concern for applications which already require many routing table
		entries. Routing table usage, as well as other overheads, were most
		significantly increased in the presence of contiguous groups of network
		faults. This is because the PGS repair algorithm produces routes which pass
		tightly around the corners of faults, resulting in concentrations of
		routing table entries in those areas.  Though the symptoms of this problem
		can be attributed to the design of SpiNNaker's multicast routing mechanism,
		the responsibility lies with the behaviour of the PGS repair algorithm.
		Potential improvements to the PGS repair algorithm are discussed later in
		\S\ref{sec:pgs-repair-improvements}.
		
		The overall answer to this research question, therefore, is that the
		flexibility provided to routing algorithms in SpiNNaker's architecture is
		of great benefit, enabling arbitrary fault patterns to be inexpensively
		avoided.
		
		\subsubsection{3. How can the parts of a neural simulation be placed onto a
		large hexagonal torus topology to reduce network load?}
		
		In chapter~\ref{sec:placement}, I explored a number of contemporary
		approaches to the problem of placing applications with irregular
		communication patterns into network topologies. I observed that researchers
		working on circuit placement for chips and FPGAs are tackling similar
		problems and working at scales as large, or larger than, those faced in
		application placement. Based on this I developed a
		simulated annealing based placement algorithm inspired by the techniques
		used in circuit placement, with specific adaptations for use in application
		placement and SpiNNaker's network topology.
		
		The simulated annealing based placement algorithm consistently outperforms
		pre-existing placement algorithms included in benchmarks in terms of
		placement quality.  In the case of one benchmark, simulated annealing based
		placement made it possible to run that neural simulation in real-time for
		the first time.  At larger scales, simulated annealing was also found to be
		able to produce good quality placements in benchmarks containing over one
		million processes -- the largest size supported by the SpiNNaker
		architecture.
		
		The major shortcoming of simulated annealing based placement is its
		execution speed. Though its execution time grows in proportion to the size
		of the problem, the implementation used took over 12~hours to place a
		synthetic problem for the largest planned SpiNNaker machine. Though
		tractable -- particularly given the relative output quality compared with
		the prior state-of-the-art -- the algorithm is unlikely to function
		comfortably as-is on larger problems.
		
		The conclusion to be drawn from this result, however, is not just that
		simulated annealing is a good solution for today's placement problems but
		that circuit placement techniques in general could be successfully adapted
		to fulfil this role. The placement problems faced by chip designers are
		growing at roughly the same exponential rate as the size of super computers
		but circuit designs hold the lead in terms of problem size. Consequently,
		as approaches are retired by chip placement researchers, they may find new
		life in the field of application placement.
		
	\section{Future research}
		
		Though the goals of this study have largely been met, there also remain
		some important limitations which future work may hope to address.
		Furthermore, this work has uncovered a number of new research areas
		warranting future enquiry. This section outlines a number of future lines
		of research.
		
		\subsection{Warehouse-scale cabling}
			
			In chapter~\ref{sec:building} I developed and implemented a number of
			cabling schemes for the SpiNNaker architecture spanning up to a six metre
			row of machine-room cabinets -- a relatively small installation by
			current standards. In SpiNNaker, the cabling exists in a 2D plane (i.e.
			across the faces of the cabinets) but as the system is scaled up, a
			single row of cabinets will tend towards a 1D line. Since embedding a 2D
			structure in a 1D space necessarily results in long connections, this
			cannot scale indefinitely.
			
			\begin{figure}
				\center
				\buildfig{figures/multi-row-cabling.tex}
				
				\caption{Multiple rows of interconnected cabinets.}
				\label{fig:multi-row-cabling}
			\end{figure}
			
			In conventional large-scale super computer installations, nodes are
			installed in rows of cabinets as illustrated in
			figure~\ref{fig:multi-row-cabling}.  From a `bird's-eye' view, the system
			approximates a 2D space, spread across the floor of a machine-room.
			Therefore, in principle, the folding and interleaving techniques
			described in chapter~\ref{sec:building} still apply. Unfortunately for
			SpiNNaker, cables connecting between rows of cabinets would be longer
			than the one metre limit imposed by its hardware because of the spacing
			between rows of cabinets.  Future SpiNNaker systems will need to consider
			alternative link technologies.  For example, a hybrid system could be
			used in which intra-cabinet connections continue to use the current HSS
			link technology while inter-cabinet links might use optical connections.
			This type of architecture could be supported by the use of pluggable
			`SFP+' transceiver modules~\cite{sff01}.
		
		\subsection{Cabling assistance for other architectures}
			
			A secondary result of the construction of prototype SpiNNaker systems in
			chapter~\ref{sec:building} was the use of real-time guidance and feedback
			to assist cable installation. I am not aware of this technique's use by
			existing architectures and, following the success experienced in this
			project, it is possible that the technique may also be useful in
			conventional systems.
			
			During the construction of prototype SpiNNaker machines, each cable took
			seconds to install compared with the minutes reported for existing
			systems~\cite{mudigonda11}. Part of this increase in efficiency appears
			to result from the immediate identification of mistakes made during
			cabling, saving time-consuming backtracking later on.
			
			In many real-world network installations, units are less densely packed
			than in SpiNNaker and so longer cables are often required. As a
			consequence, cabling errors may become more likely as cabling patterns
			are spread over a larger area making them more difficult to visually
			verify. Like SpiNNaker, conventional networking hardware is often
			equipped with a generous range of indicator LEDs and diagnostic
			facilities which might be used to implement real-time installation
			guidance. Future work could explore the use of this technique in the
			construction of other large-scale networks, such as data centres.
		
		\subsection{Congestion mitigation}
			
			\label{sec:wiggly-board-allocations}
			
			In chapter~\ref{sec:routing} I found that contiguous network faults cause
			hot-spots of congestion and routing table depletion where the PGS repair
			algorithm routed many paths around the edges of faults.  However, it is
			not just faults which can cause contiguous blockages in the network
			topology. In reality, researchers do not always require a full-sized
			SpiNNaker system to perform their experiments so large SpiNNaker systems
			are soft-partitioned on demand into many smaller
			machines~\cite{spalloc16}. To ensure isolation between partitioned
			sub-machines, HSS links between boards in different partitions are
			disabled. Because of SpiNNaker's `wrapped triple' partitioning scheme,
			the resulting sub-machines have hexagonal \emph{mesh} topologies (i.e.
			without wrap-around links) with irregular boundaries as in
			figure~\ref{fig:spalloc-mesh}.
			
			\begin{figure}
				\center
				\buildfig{figures/spalloc-mesh.tex}
				
				\caption[Irregular edges of a partitioned SpiNNaker system.]%
				{Irregular edges in a SpiNNaker system comprised of 24~boards
				partitioned from a larger machine.  Each hexagon represents a SpiNNaker
				chip. No wrap-around connections are present.}
				\label{fig:spalloc-mesh}
			\end{figure}
			
			In partitioned systems, the `tooth'-like gaps on the periphery of the
			network result in similar congestion to the HSS link failures considered
			in chapter~\ref{sec:routing}. When a route is generated between nodes on
			opposite sides of a gap, the PGS repair process will produce a
			shortest-path route around it. Since many routes may be blocked by a
			single gap, a hot-spot may develop around the corners of the gap.
			
			In chapter~\ref{sec:placement}, the `CConv' benchmark application was
			found to run correctly the majority of the time when placed by the
			simulated annealing algorithm but would occasionally fail by a
			significant margin. Preliminary experiments suggest these occasional
			failures are caused by placement solutions which place heavily
			communicating parts of the application on opposite sides of gaps along
			the perimeter of the network. Two possible approaches which future work
			may consider are presented below.
			
			\subsubsection{Avoiding hotspots with PGS repair}
				
				\label{sec:pgs-repair-improvements}	
				
				Network congestion around faults and network irregularities could be
				reduced by forcing the PGS repair process to take more varied routes
				around faults. For example, in circuit routing algorithms, routers
				avoid congestion by increasing the cost of routes which pass through
				congested areas~\cite{kahng11}. A similar technique could be used in
				PGS repair to spread the routes it produces.
				
				An alternative approach would be to adapt the base routing algorithms
				used prior to PGS repair to, for example, attempt alternative dimension
				order routes which may avoid blockages due to faulty links.
			
			\subsubsection{Fault and irregularity aware placement}
				
				One of the shortcomings of the simulated annealing based placer
				developed in chapter~\ref{sec:placement} is that it does not account
				for network faults, or irregularities, when estimating the cost of
				placement solutions.  Future work may exploit techniques used in
				congestion-aware circuit placement which could be adapted for
				application placement~\cite{viswanathan07}.
		
		\subsection{Reducing placement execution time}
			
			The simulated annealing based placer presented in
			chapter~\ref{sec:placement} produced good quality placements but its
			execution time limits its use beyond one million vertex placement
			problems. Future work should explore possibilities for improving the
			performance and scalability of this technique.
			
			In addition to considering alternative placement algorithms based on
			other methods, one possible approach is to attempt to reduce the execution
			time of simulated annealing based placement by shrinking the application
			graph being placed.
			
			For example, graph clustering~\cite{schaeffer07} may be used to group
			together strongly connected vertices which would then be placed as a
			single unit.  Unfortunately, clustering can suffer from the same problems
			as graph-partitioning-based placement: vertices may be grouped together
			in ways which, in practice, cannot be packed together into a given portion
			of a machine.  A possible solution to this problem is to use a two-phase
			placement approach~\cite{kahng11}. In a `global' placement phase,
			solutions are permitted which can slightly over-allocate resources but
			overall achieve good placement quality. In the `detailed' placement phase
			which follows, the solution is `legalised' by making small changes to the
			global placement to eliminate over allocation.
			
			An alternative approach suited to SpiNNaker could be to limit the
			clustering process to clusters which fit on a single SpiNNaker chip. In
			typical SpiNNaker application graphs, clustering to this level may reduce
			placement problem sizes by an order of magnitude and, consequently,
			reduce execution times by the same ratio. Preliminary experiments suggest
			that this approach might result in little placement quality loss for
			large placement problems whilst substantially reducing overall execution
			time.
		
		\subsection{Benchmarking}
			
			One of the most significant limitations of this study has been the
			unavailability of large-scale SpiNNaker applications for use as
			benchmarks. As a consequence, much of the scalability experimentation
			performed has relied on simple synthetic benchmarks based on projections
			of future application behaviour.
			
			In the short term, more sophisticated synthetic benchmark generation
			techniques used by the circuit placement community~\cite{nam07} may offer
			alternative benchmarks for future work. In the longer term, however, it
			is hoped that the availability of large SpiNNaker systems -- and
			placement and routing algorithms better suited to exploit them -- will
			lead to larger scale applications being developed. Hopefully these
			applications will lead to more interesting and representative benchmarks
			for use in future work.
	
	\section{Closing remarks}
		
		One of the primary outcomes of this work is that a number of the practical
		challenges faced in scaling up the SpiNNaker architecture have been
		addressed leading to the construction of large-scale SpiNNaker machines.
		The development of an effective placement algorithm for SpiNNaker
		applications has been shown to enable some neural simulations to exploit
		SpiNNaker's architecture for the first time. The availability of larger
		SpiNNaker machines paves the way for future large-scale neural modelling
		work built on much larger models such as Spaun, `the world's largest
		functional brain model'~\cite{eliasmith12}.
		
		Beyond the SpiNNaker project, the hexagonal torus topology has also been
		validated as a scalable and practical candidate for future network
		architectures. As super computers become ever larger, the physical
		scalability afforded by the 2D nature of the hexagonal torus topology may
		make it a compelling option. In addition, the finding that circuit
		placement techniques can be adapted to support placement of SpiNNaker
		software indicates that these algorithms may also be applicable to other
		applications. Indeed, if this is the case, circuit placement may offer a
		long-term source of placement algorithms able to handle the demands of
		future applications.
		
		% This thesis has explored and tackled a number of the challenges posed in
		% scaling up the unconventional SpiNNaker architecture. Along the way I have
		% demonstrated that the hexagonal torus topology may be a practical choice in
		% future applications which can scale up to the physical dimensions expected
		% of future super computers. I have also developed new efficient and
		% effective methods of placing and routing neural simulation software on
		% SpiNNaker which -- it is hoped -- will enable a new generation of large
		% scale neural simulations on spinnaker.
		
		Although this work stops short of demonstrating truly large-scale
		neuroscientific simulations running at the scale of newly completed
		SpiNNaker machines (largely because such simulations do not yet exist) a
		number of smaller-scale neural simulations have been made possible for the
		first time. The algorithms and techniques devised in this work have
		subsequently been incorporated into various software libraries and tools
		now being used by researchers building SpiNNaker applications, vindicating
		the efforts of this thesis (see appendix~\ref{sec:software}). A successor
		to the SpiNNaker architecture is also in the early stages of design and is
		building on experience of the existing architecture. The current intention
		is to retain the hexagonal torus topology used by SpiNNaker, a decision
		supported by the findings of this thesis.
		
		With SpiNNaker's hardware architecture now operating at scales close to its
		architectural limits, it is hoped that the contributions of this work will
		enable researchers to develop larger and more detailed neural models for
		this unique architecture.

	
	\appendix
	\chapter{Partitioning hexagonal toruses}
	
	\label{sec:partitioning}
	
	The nodes in super computer networks are usually relatively small, for
	example in SpiNNaker each node is a single chip. To allow several nodes to
	share resources such as power supplies and simplify construction. Typically
	tens of nodes are packed together into a single unit such as a circuit board
	or server blade to simplify assembly and share common power and cooling
	resources \cite{gilge14,ajima12}. In commercial super computers built on
	non-hexagonal torus topologies, each unit's represents a hypercube partition
	of the overall topology as illustrated in figure
	\ref{fig:hypercube-partitioning} \cite{chen11,ajima12}.
	
	\begin{figure}
		\center
		\begin{subfigure}[b]{0.45\textwidth}
			\center
			\buildfig{figures/hypercube-partitioning.tex}
			\caption{2D hypercube partitioning}
			\label{fig:apdx-hypercube-partitioning}
		\end{subfigure}
		\begin{subfigure}[b]{0.45\textwidth}
			\center
			\buildfig{figures/parallelogram-partitioning.tex}
			\caption{Parallelogram partitioning}
			\label{fig:apdx-parallelogram-partitioning}
		\end{subfigure}
		
		\caption[Torus partitioning.]%
		{Conventional hypercube topology partitioning (a) and the
		hexagonal torus topology analogue (b).}
		\label{fig:apdx-partitioning-options}
	\end{figure}
	
	
	The analogue of this scheme in a hexagonal torus topology is a parallelogram
	as illustrated in figure~\ref{fig:parallelogram-partitioning}.  Each
	partition connects to six neighbouring partitions and, unlike hypercube
	partitions, the number of connections to each is imbalanced.  Specifically
	the partitions above-right and below-left are connected by only one link
	each. The consequence is potentially a need for multiple interconnect
	technologies if connections between partitions are concentrated into single
	connections as in SpiNNaker (see chapter~\ref{sec:background}). This adds
	both to design complexity and system cost.
	
	In this appendix, I describe how and why the `wrapped triple' partitioning
	scheme devised by Davidson works \cite{davidsonWiring}.
	
	\section{Tiling}
		
		For a particular configuration of nodes to form a valid partition, it must
		be possible to use this configuration to `tile' a hexagonal torus. In many
		of the figures in this thesis, a `pointy-topped' hexagon is used to
		represent chips in a hexagonal torus resulting in a parallelogram-shaped
		arrangement. Any partition which shares the translational symmetry of a
		pointy-topped hexagon must also tile a hexagonal torus.
		
		\begin{figure}
			\center
			\buildfig{figures/tiling-a-torus.tex}
			
			\caption{Tiling a hexagonal torus with parallelograms.}
			\label{fig:tiling-a-torus}
		 \end{figure}
		 
		 \begin{figure}
			\center
			\buildfig{figures/parallelogram-tiling.tex}
			
			\caption{Visual proof that a parallelogram shares translational symmetry
			with a pointy-topped hexagon.}
			\label{fig:parallelogram-tiling}
		\end{figure}
		
		For example, in figure~\ref{fig:tiling-a-torus} we can see that $2\times2$
		parallelograms can tile a $9\times9$ hexagonal torus topology. In
		figure~\ref{fig:parallelogram-tiling}, we will see how a parallelogram
		partition shares its translational symmetry with a pointy-topped hexagon.
		
		In figure~\ref{fig:parallelogram-tiling}a, the $2\times2$ partition is
		shown with each node shaded differently and in
		figure~\ref{fig:parallelogram-tiling}b, a pointy-topped hexagon has been
		superimposed. By tiling several copes of this partition
		(figure~\ref{fig:parallelogram-tiling}c), we can see that repeating pattern
		of the parallelogram matches the repeating pattern of the pointy-topped
		hexagon. We refer to this as the parallelogram sharing the translational
		symmetry of a pointy-topped hexagon.
		
		Notice that the parallelogram partition can be redrawn such that all parts
		protruding from the overlaid pointy-topped hexagon wrap around, filling the
		gaps on opposite sides to produce a pointy-topped hexagon shaped tile as
		shown in figure~\ref{fig:parallelogram-tiling}d.
		
	
	\section{How \emph{not} to tile a hexagonal torus}
		
		\begin{figure}
			\center
			\buildfig{figures/wrapped-hexagon-tiling.tex}
			
			\caption{Visual proof that a wrapped hexagon does not share translational
			symmetry with a pointy-topped hexagon.}
			\label{fig:wrapped-hexagon-tiling}
		\end{figure}
	
		An `obvious' partitioning scheme for a hexagonal topology which evenly
		distributes links between six sides is to use a hexagonal partition. Such a
		partition might na\"ively be formed by wrapping `layers' of hexagons around a
		single hexagon. Figure~\ref{fig:wrapped-hexagon-tiling}a illustrates a
		simple partition with a single layer of hexagons surrounding a central
		hexagon.
	
		While this type of partition exposes six equally-sized edges (satisfying
		the requirement that connections between boards should have a balanced
		number of connections) this partition does not share translational symmetry
		with a pointy-topped hexagon. Figure~\ref{fig:wrapped-hexagon-tiling}b
		shows a best-fitting pointy-topped hexagon superimposed on the partition.
		In figure~\ref{fig:wrapped-hexagon-tiling}c, we can see that when tiled,
		the partition leaves gaps between the superimposed pointy-topped hexagon
		indicating it does not share the same translational symmetry.
		
		Consequently, it is not possible to construct a hexagonal torus from
		partitions of this shape (though it is possible to construct a related, but
		different topology, the H-torus, see \S\ref{sec:hex-vs-h-torus}).
	
	\section{Triads of triples}
		
		\begin{figure}
			\center
			\buildfig{figures/wrapped-triple-tiling.tex}
			
			\caption{Visual proof that a triple shares the same translational
			symmetry as a flat-topped hexagon.}
			\label{fig:wrapped-triple-tiling}
		\end{figure}
		
		A `triple' is a partition made up of three nodes arranged as in
		figure~\ref{fig:wrapped-triple-tiling}a. This partition's edges may be
		broken into six groups with an equal number of connections, meeting the
		requirements set out at the beginning of this appendix. A triple partition,
		however, does not share translational symmetry with a pointy-topped hexagon
		but \emph{does} share it with a `flat-topped' hexagon (the
		\SI{30}{\degree}-rotated cousin of the pointy-topped hexagon) as
		demonstrated in figures~\ref{fig:wrapped-triple-tiling}b and
		\ref{fig:wrapped-triple-tiling}c.
		
		\begin{figure}
			\center
			\buildfig{figures/triad-tiling.tex}
			
			\caption[Triads tile a hexagonal torus topology.]%
			{Demonstration that a triad tiles a hexagonal torus topology.}
			\label{fig:triad-tiling}
		\end{figure}
		
		Because a triple made up of pointy-topped hexagons tiles like a flat-topped
		hexagon, a triple made up of flat-topped hexagons must share translational
		symmetry with a pointy-topped hexagon (turn
		figure~\ref{fig:wrapped-triple-tiling} \SI{30}{\degree} to visualise this).
		It follows that three triples arranged as in figure~\ref{fig:triad-tiling}
		-- a `triad' -- share translational symmetry with pointy-topped hexagons.
		
		Consequently, a triple may be used to tile a hexagonal torus topology 
	
	\section{Wrapped triples}
		
		\begin{figure}
			\center
			\buildfig{figures/wrapped-triple.tex}
			
			\caption{A wrapped triple with four layers, labelled by layer number.}
			\label{fig:wrapped-triple}
		\end{figure}
		
		A triple forms a partition which can be used to tile a hexagonal torus when
		tiled using triads of triples. To increase the number of nodes in the
		partition, layers of hexagons may be wrapped around a triple to form a
		`wrapped triple' as in figure~\ref{fig:wrapped-triple}. Wrapped triples
		have exactly balanced communication requirements to their neighbouring
		partitions.  Triads of wrapped-triples share translational symmetry with a
		pointy topped hexagon and thus may be used to tile a hexagonal torus
		topology.
		
		In SpiNNaker, the network is partitioned into four-layer wrapped triples
		containing forty~eight nodes each.

	\chapter{Minimising hexagonal mesh coordinates}
	\label{app:minimal-hex-coordinates}
	
	As in non-hexagonal mesh and torus topologies, coordinates in a hexagonal
	torus topology are given in terms of the offset of a point along each axis of
	the topology from an arbitrary origin. Hexagonal topologies, unusually, are
	considered as having three, non-orthogonal axes. In a conventional 2D
	topology, for example, moving along the X axis does not alter your position
	on the Y axis because the X and Y axes are orthogonal. In a hexagonal
	coordinate system, moving along any one of the three axes results in some
	movement in the other two.  A consequence of this is that multiple
	coordinates may exist for the same position.
	
	\begin{figure}
		\center
		\buildfig{figures/hex-mesh-loop.tex}
		\caption[$(1,1,1)$ in a hexagonal mesh or torus.]%
		{The vector $(1, 1, 1)$ in a hexagonal mesh topology results in
		no movement.}
		\label{fig:hex-mesh-loop}
	\end{figure}
	
	Consider the vector $(1,1,1)$ which takes one hop along each axis.
	Figure~\ref{fig:hex-mesh-loop} illustrates how this vector takes us back to
	our starting point after three hops. Because, like $(0,0,0)$, this vector
	results in no movement, it may be added or subtracted to any existing vector
	or coordinate without changing the meaning of the vector. Doing so, however,
	clearly has some impact on the magnitude of the vector. For example, the
	coordinate vector $(2, -3, -1)$ has a magnitude of 6. Adding $(1,1,1)$
	results in a new, equivalent vector, $(3, -2, 0)$, with a reduced magnitude
	of 5.  Adding $(1,1,1)$ once more produces the vector $(4, -1, 1)$ of
	magnitude 6 again.
	
	In a hexagonal mesh, every point has a unique coordinate vector whose
	magnitude is minimal. To demonstrate this by cases, consider:
	
	\begin{itemize}
	
		\item If a vector has three positive elements, subtracting $(1,1,1)$
		\emph{reduces} the magnitude of the vector by three overall.
		
		\item If a vector has two positive elements, subtracting $(1,1,1)$
		decreases the magnitude of those two elements and increases the magnitude
		of the remaining element resulting in a net \emph{reduction} in magnitude
		of one.
		
		\item If a vector has only one positive, non-zero element, subtracting
		$(1,1,1)$ decreases the magnitude of that element and increases the
		magnitude of the remaining two resulting in a net \emph{increase} in
		magnitude of one.
		
		\item Similar arguments may be made for vectors with negative elements and
		the \emph{addition} of $(1,1,1)$.
	
	\end{itemize}
	
	These cases indicate that vectors and coordinates containing at least one
	zero element with the remaining elements having opposite signs are minimal
	since adding or subtracting $(1,1,1)$ will always increase the magnitude of
	the vector.
	
	To minimise a coordinate vector in a hexagonal mesh topology, the following
	function may be used, producing a unique coordinate for any given point:
	%
	\begin{equation*}
		\operatorname{minimiseVector}(x,y,z) =
			(x,y,z) - \operatorname{median}(x,y,z) \cdot (1,1,1)
	\end{equation*}
	
	As an aside, minimising a hexagonal coordinate vector is not the only way to
	determine a unique coordinate for a given point. Given a vector of the form
	$(x, y, z)$, subtracting $(z,z,z)$ will result in a vector of the form $(x',
	y', 0)$. Because this form mimics the appearance of a standard 2D coordinate
	system, while uniquely identifying points, it is widely used as a convenient
	and intuitive addressing scheme by SpiNNaker's system software.

	
	% Bibliography
	\bibliography{references}
	\bibliographystyle{alpha}
	
\end{document}
