\chapter{Minimising hexagonal mesh coordinates}
	\label{app:minimal-hex-coordinates}
	
	TODO: THE WHOLE THING REALLY...
	
	\begin{figure}
		\center
		\buildfig{figures/hex-mesh-loop.tex}
		\caption{The vector $(1, 1, 1)$ in a hexagonal mesh topology results in
		no movement.}
		\label{fig:hex-mesh-loop}
	\end{figure}
	
	Unlike (non-hexagonal) mesh topologies, the dimensions of a hexagonal
	mesh are \emph{non-orthogonal}, the most notable side-effect of which is
	that there are an infinite number of vectors (of an infinite number of
	lengths) between a given pair of points. For example the vector $(3, 2,
	1)$ also reaches `B' from `A' but does so in six hops. The reason for
	this phenomenon is illustrated in figure \ref{fig:hex-mesh-loop} which
	shows how the vector $(1, 1, 1)$ results in no movement. As a result,
	adding or subtracting any multiple of $(1, 1, 1)$ from a vector yields an
	equivalent, yet different, vector. For example, given our original vector
	from `A' to `B', a new vector can be produced like so:
	%
	\begin{equation} (2, 1, 0) - (1, 1, 1) = (1, 0, -1) \end{equation}
	%
	This new vector yields routes requiring only two hops compared with three
	for the original vector. Subtracting $(1, 1, 1)$ again, however, produces
	$(0, -1, -2)$ with a greater magnitude of three.
