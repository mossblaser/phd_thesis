\chapter{Minimising hexagonal mesh coordinates}
	\label{app:minimal-hex-coordinates}
	
	As in non-hexagonal mesh and torus topologies, coordinates in a hexagonal
	torus topology are given in terms of the offset of a point along each axis of
	the topology from an arbitrary origin. Hexagonal topologies, unusually, are
	considered as having three, non-orthogonal axes. In a conventional 2D
	topology, for example, moving along the X axis does not alter your position
	on the Y axis because the X and Y axes are orthogonal. In a hexagonal
	coordinate system, moving along any one of the three axes results in some
	movement in the other two.  A consequence of this is that multiple
	coordinates may exist for the same position.
	
	\begin{figure}
		\center
		\buildfig{figures/hex-mesh-loop.tex}
		\caption[$(1,1,1)$ in a hexagonal mesh or torus.]%
		{The vector $(1, 1, 1)$ in a hexagonal mesh topology results in
		no movement.}
		\label{fig:hex-mesh-loop}
	\end{figure}
	
	Consider the vector $(1,1,1)$ which takes one hop along each axis.
	Figure~\ref{fig:hex-mesh-loop} illustrates how this vector takes us back to
	our starting point after three hops. Because, like $(0,0,0)$, this vector
	results in no movement, it may be added or subtracted to any existing vector
	or coordinate without changing the meaning of the vector. Doing so, however,
	clearly has some impact on the magnitude of the vector. For example, the
	coordinate vector $(2, -3, -1)$ has a magnitude of 6. Adding $(1,1,1)$
	results in a new, equivalent vector, $(3, -2, 0)$, with a reduced magnitude
	of 5.  Adding $(1,1,1)$ once more produces the vector $(4, -1, 1)$ of
	magnitude 6 again.
	
	In a hexagonal mesh, every point has a unique coordinate vector whose
	magnitude is minimal. To demonstrate this by cases, consider:
	
	\begin{itemize}
	
		\item If a vector has three positive elements, subtracting $(1,1,1)$
		\emph{reduces} the magnitude of the vector by three overall.
		
		\item If a vector has two positive elements, subtracting $(1,1,1)$
		decreases the magnitude of those two elements and increases the magnitude
		of the remaining element resulting in a net \emph{reduction} in magnitude
		of one.
		
		\item If a vector has only one positive, non-zero element, subtracting
		$(1,1,1)$ decreases the magnitude of that element and increases the
		magnitude of the remaining two resulting in a net \emph{increase} in
		magnitude of one.
		
		\item Similar arguments may be made for vectors with negative elements and
		the \emph{addition} of $(1,1,1)$.
	
	\end{itemize}
	
	These cases indicate that vectors and coordinates containing at least one
	zero element with the remaining elements having opposite signs are minimal
	since adding or subtracting $(1,1,1)$ will always increase the magnitude of
	the vector.
	
	To minimise a coordinate vector in a hexagonal mesh topology, the following
	function may be used, producing a unique coordinate for any given point:
	%
	\begin{equation*}
		\operatorname{minimiseVector}(x,y,z) =
			(x,y,z) - \operatorname{median}(x,y,z) \cdot (1,1,1)
	\end{equation*}
	
	As an aside, minimising a hexagonal coordinate vector is not the only way to
	determine a unique coordinate for a given point. Given a vector of the form
	$(x, y, z)$, subtracting $(z,z,z)$ will result in a vector of the form $(x',
	y', 0)$. Because this form mimics the appearance of a standard 2D coordinate
	system, while uniquely identifying points, it is widely used as a convenient
	and intuitive addressing scheme by SpiNNaker's system software.
