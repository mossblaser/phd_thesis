\begin{tikzpicture}[thick, font=\footnotesize, scale=1.0]
	% #1 location
	% #2 name
	\newcommand{\fixed}[2]{
		\node [ draw=black
		      , circle
		      , inner sep=0
		      , minimum width=1.5em
		      ]
		      at (#1) {};
		\node [ draw=black
		      , circle
		      , inner sep=0
		      , minimum width=1.8em
		      ]
		      (#2) at (#1) {$#2$};
	}
	
	% #1 location
	% #2 name
	\newcommand{\movable}[2]{
		\node [ draw=black
		      , circle
		      , inner sep=0
		      , minimum width=1.5em
		      ]
		      (#2) at (#1) {$#2$};
	}
	
	% #1 weight
	% #2 src
	% #3 dst
	\newcommand{\edge}[3]{
		\draw (#2) -- node [anchor=south] {#1} (#3);
	}
	
	\input{|"python figures/quadratic_placement.py"}
	
	% Ruler
	\begin{scope}[yshift=-2.0em, help lines]
		\draw (0, -0.2) --
		      (0, 0) --
		      (10, 0) --
		      (10, -0.2);
		\foreach \x in {1,...,9}{
			\draw (\x, 0) -- ++(0, -0.1);
		}
		\foreach \x in {0,...,10}{
			\node at (\x, -0.2) [anchor=north] {\x};
		}
		
		\foreach \vertex in {f_1,f_2,m_1,m_2}{
			\draw [dashed] (\vertex) -- (\vertex |- 0, 0);
		}
	\end{scope}
	
\end{tikzpicture}
