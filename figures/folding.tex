% Define the points on the quads which make up the folded sheet of the
% specified size.
\newcommand{\singleAxisFoldPoints}[4]{
	\pgfmathtruncatemacro{\width}{#1}
	\pgfmathtruncatemacro{\height}{#2}
	\pgfmathtruncatemacro{\numFaces}{#3}
	\pgfmathsetmacro{\foldAngle}{#4}
	
	
	\pgfmathtruncatemacro{\lastPoint}{\numFaces+1}
	
	\coordinate (cur) at (0,0);
	\foreach \x in {1,...,\lastPoint}{
		\pgfmathtruncatemacro{\parity}{Mod(\x,2)}
		\coordinate (px\x y1) at (cur);
		\coordinate (px\x y2) at ([shift={(0,-\height)}]cur);
		
		\ifthenelse{\parity = 1}{
			\coordinate (cur) at ([shift={(\width/\numFaces,0)}]cur);
		}{
			\coordinate (cur) at ([shift={(-180+\foldAngle:\width/\numFaces)}]cur);
		}
	}
}

% Define the points on the quads which make up the folded sheet folded
% once along the y axis and many times along the x axis.
\newcommand{\doubleAxisFoldPoints}[5]{
	\pgfmathtruncatemacro{\width}{#1}
	\pgfmathtruncatemacro{\height}{#2}
	\pgfmathtruncatemacro{\numFaces}{#3}
	\pgfmathsetmacro{\foldAngle}{#4}
	\pgfmathsetmacro{\perspectiveAngle}{#5}
	
	\pgfmathtruncatemacro{\lastPoint}{\numFaces+1}
	
	\pgfmathsetmacro{\foldSpace}{2*(\width/\numFaces)*sin(\foldAngle/2)}
	
	\foreach \x in {1,...,\lastPoint}{
		\pgfmathtruncatemacro{\parity}{Mod(\x,2)}
		
		% Common point where the folds meet on the Y-axis
		\pgfmathsetmacro{\xx}{(1-\parity)*\width/\numFaces}
		\coordinate (px\x y2) at (\xx, 0);
		
		% Edges of the paper fanned out
		\begin{scope}[shift={(-\perspectiveAngle:\height/2)}]
			
			\pgfmathsetmacro{\xx}{(1-\parity)*\width/\numFaces}
			\pgfmathsetmacro{\yy}{(\numFaces-\x)*\foldSpace}
			\coordinate (px\x y1) at (\xx,\yy);
			
			\pgfmathsetmacro{\xx}{\xx}
			\pgfmathsetmacro{\yy}{-\numFaces*\foldSpace-\yy-\foldSpace}
			\coordinate (px\x y3) at (\xx,\yy);
		\end{scope}
	}
}

% Draw the faces of the most recently defined surface of the specified
% size.
\newcommand{\drawSheets}[2]{
	\pgfmathtruncatemacro{\facesX}{#1}
	\pgfmathtruncatemacro{\facesY}{#2}
	
	% Doesn't work for PDFs
	\def\sheetOpacity{1.0}
	
	\foreach \y in {1,...,\facesY}{
		\pgfmathtruncatemacro{\rowparity}{Mod(\y,2)}
		\foreach \x in {1,...,\facesX}{
			% Reverse draw direction as appropriate
			\ifthenelse{\rowparity = 0}{
				\pgfmathtruncatemacro{\x}{\facesX + 1 - \x}
			}{}
			\pgfmathtruncatemacro{\y}{\facesY + 1 - \y}
			
			\pgfmathtruncatemacro{\xx}{\x+1}
			\pgfmathtruncatemacro{\yy}{\y+1}
			
			\pgfmathtruncatemacro{\parity}{Mod(\y+\x,2)}
			\ifthenelse{\parity = 0}{
				\draw [fill=white,fill opacity=\sheetOpacity]
				      (px\x y\y) -- (px\xx y\y) -- (px\xx y\yy) -- (px\x y\yy) -- cycle;
			}{
				\draw [fill=paperback,fill opacity=\sheetOpacity]
				      (px\x y\y) -- (px\xx y\y) -- (px\xx y\yy) -- (px\x y\yy) -- cycle;
			}
		}
	}
	% Draw feint outlines on top
	\foreach \y in {1,...,\facesY}{
		\pgfmathtruncatemacro{\rowparity}{Mod(\y,2)}
		\foreach \x in {1,...,\facesX}{
			% Reverse draw direction as appropriate
			\ifthenelse{\rowparity = 0}{
				\pgfmathtruncatemacro{\x}{\facesX + 1 - \x}
			}{}
			\pgfmathtruncatemacro{\y}{\facesY + 1 - \y}
			
			\pgfmathtruncatemacro{\xx}{\x+1}
			\pgfmathtruncatemacro{\yy}{\y+1}
			
			\draw [ultra thin, dotted]
			      (px\x y\y) -- (px\xx y\y) -- (px\xx y\yy) -- (px\x y\yy) -- cycle;
		}
	}
}

% Add coordinates of the form x1y1 for pins around the folded sheet
\newcommand{\definePins}[4]{
	\pgfmathtruncatemacro{\facesX}{#1}
	\pgfmathtruncatemacro{\facesY}{#2}
	\pgfmathtruncatemacro{\pinsX}{#3}
	\pgfmathtruncatemacro{\pinsY}{#4}
	
	\pgfmathtruncatemacro{\facesXX}{\facesX+1}
	\pgfmathtruncatemacro{\facesYY}{\facesY+1}
	
	\pgfmathtruncatemacro{\pinsXX}{\pinsX+1}
	\pgfmathtruncatemacro{\pinsYY}{\pinsY+1}
	
	\foreach \x in {1,...,\pinsX}{
		\pgfmathsetmacro{\faceX}{((\x-0.5)/\pinsX) * \facesX}
		% Points defining the x position
		\pgfmathtruncatemacro{\pX}{floor(\faceX)+1}
		\pgfmathtruncatemacro{\pXX}{\pX+1}
		
		% Proportion along the edge
		\pgfmathsetmacro{\oX}{\faceX-floor(\faceX)}
		
		\coordinate (x\x y0) at ($(px\pX y1)!\oX!(px\pXX y1)$);
		\coordinate (x\x y\pinsYY) at ($(px\pX y\facesYY)!\oX!(px\pXX y\facesYY)$);
	}
	
	\foreach \y in {1,...,\pinsY}{
		\pgfmathsetmacro{\faceY}{((\y-0.5)/\pinsY) * \facesY}
		% Points defining the y position
		\pgfmathtruncatemacro{\pY}{floor(\faceY)+1}
		\pgfmathtruncatemacro{\pYY}{\pY+1}
		
		% Proportion along the edge
		\pgfmathsetmacro{\oY}{\faceY-floor(\faceY)}
		
		\coordinate (x0y\y) at ($(px1y\pY)!\oY!(px1y\pYY)$);
		\coordinate (x\pinsXX y\y) at ($(px\facesXX y\pY)!\oY!(px\facesXX y\pYY)$);
	}
}

\newcommand{\drawSlicedWires}[2]{
	\def\width{#1}
	\def\height{#2}
	\def\wrapSize{0.2}
	
	\pgfmathtruncatemacro{\widthh}{\width+1}
	\pgfmathtruncatemacro{\heightt}{\height+1}
	\pgfmathtruncatemacro{\heightHalf}{\height/2}
	
	% Left-to-right wrap-around
	\foreach \y in {1,...,\height}{
		\draw (x0y\y) -- (x\widthh y\y);
	}
	
	% Bottom-to-top
	\foreach \x in {1,...,\width}{
		\pgfmathtruncatemacro{\xx}{Mod((\x-1) - floor(\height/2), \width)+1}
		\draw (x\x y\heightt) -- (x\xx y0);
	}
	
	% Diagonal wrap-around
	\ifthenelse{\width > \heightHalf}{
		% Do nothing, line already drawn by bottom-to-top
	}{
		\draw (x1y\heightt) -- (x\heightHalf y0);
	}
}

\newcommand{\drawShearedWires}[2]{
	\def\width{#1}
	\def\height{#2}
	\def\wrapSize{0.2}
	
	\pgfmathtruncatemacro{\widthh}{\width+1}
	\pgfmathtruncatemacro{\heightt}{\height+1}
	
	% Left-to-right wrap-around
	\foreach \y in {1,...,\height}{
		\draw (x0y\y) -- (x\widthh y\y);
	}
	
	% Top-to-bottom wrap-around
	\foreach \x in {1,...,\width}{
		\draw (x\x y0) -- (x\x y\heightt);
	}
	
	% Diagonal
	\draw (x1y\heightt) -- (x\height y0);
}

\newcommand{\foldExample}[9]{
	\def\exampleWidth{#1}
	\def\exampleHeight{#2}
	\def\exampleType{#3} % Either sliced or sheared
	\def\exampleFoldsX{#4}
	\def\exampleFoldsY{#5}
	\def\exampleHighlightStart{#6}
	\def\exampleHighlightEnd{#7}
	\def\exampleFoldLineX{#8}
	\def\exampleFoldLineY{#9}
	
	% Define the folding
	\ifthenelse{\exampleFoldsY = 1}{
		\singleAxisFoldPoints{\exampleWidth}{\exampleHeight}{\exampleFoldsX}{20}
	}{
		\doubleAxisFoldPoints{\exampleWidth}{\exampleHeight}{\exampleFoldsX}{10}{50}
	}
	
	% Draw the page in
	\drawSheets{\exampleFoldsX}{\exampleFoldsY}
	
	% Add connection coordinates for wires
	\definePins{\exampleFoldsX}{\exampleFoldsY}{\exampleWidth}{\exampleHeight}
	
	% Draw the wires on
	\begin{scope}[gray]
		\ifthenelse{\equal{\exampleType}{sheared}}{
			\drawShearedWires{\exampleWidth}{\exampleHeight}
		}{
			\drawSlicedWires{\exampleWidth}{\exampleHeight}
		}
	\end{scope}
	
	% Draw in fold lines
	\ifthenelse{\equal{\exampleFoldLineX}{}}{
		% Nothing to do
	}{
		\draw [dashed] (\exampleFoldLineX, 1) -- (\exampleFoldLineX,-1-\exampleHeight);
	}
	
	% Draw in fold lines
	\ifthenelse{\equal{\exampleFoldLineY}{}}{
		% Nothing to do
	}{
		\draw [dashed] (-1, -\exampleFoldLineY) -- (1+\exampleWidth/\exampleFoldsX,-\exampleFoldLineY);
	}
	
	% Draw in the highlighted wire
	\ifthenelse{\equal{\exampleHighlightStart}{}}{
		% Don't draw highlighted wire if none given...
	}{
		\draw [ultra thick] (\exampleHighlightStart) -- (\exampleHighlightEnd);
		\fill [] (\exampleHighlightStart) circle (0.4em);
		\fill [] (\exampleHighlightEnd) circle (0.4em);
	}
}

\newcommand{\myarrow}[2]{
	\fill [scale=0.5]
	      (2*#1-0.85,2*#2+1.5)
	   -- ++(0.7,0)
	   -- ++(0,0.5)
	   -- ++(1.0,-1.0)
	   -- ++(-1.0,-1.0)
	   -- ++(0,0.5)
	   -- ++(-0.7,0)
	   -- cycle
	   ;
}
