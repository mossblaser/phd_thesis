\pgfmathsetmacro{\scale}{0.5}%
\begin{tikzpicture}[scale=\scale, hexagonXYZ, line cap=round]
	
	% Size of torus (in parallelograms)
	\pgfmathtruncatemacro{\twidth}{3}
	\pgfmathtruncatemacro{\theight}{3}
	
	\foreach \tx in {1,...,\twidth}{
		\foreach \ty in {1,...,\theight}{
			
			% Size of parallelogram
			\pgfmathtruncatemacro{\width}{2}
			\pgfmathtruncatemacro{\height}{2}
			
			% Chips
			\foreach \x in {1,...,\width}{
				\foreach \y in {1,...,\height}{
					\node (hex-\x-\y)
					      [ hexagon
					      , draw
					      , inner sep=0
					      , minimum width=\scale*1cm
					      ]
					      at (\x + \tx*\width, \y + \ty*\height, 0)
					      {};
				}
			}
			
			% Parallelogram outline
			\begin{scope}[ultra thick]
				\foreach [count=\x] \xx in {2,...,\width}{
					% Bottom
					\draw (hex-\x-1.corner 4) --
					      (hex-\x-1.corner 5) --
					      (hex-\xx-1.corner 4);
					% Top
					\draw (hex-\x-\height.corner 1) --
					      (hex-\x-\height.corner 6) --
					      (hex-\xx-\height.corner 1);
				}
				\foreach [count=\y] \yy in {2,...,\height}{
					% Left
					\draw (hex-1-\y.corner 3) --
					      (hex-1-\y.corner 2) --
					      (hex-1-\yy.corner 3);
					% Right
					\draw (hex-\width-\y.corner 6) --
					      (hex-\width-\yy.corner 5) --
					      (hex-\width-\yy.corner 6);
				}
				% Top left
				\draw (hex-1-\height.corner 3) --
				      (hex-1-\height.corner 2) --
				      (hex-1-\height.corner 1);
				% Bottom right
				\draw (hex-\width-1.corner 4) --
				      (hex-\width-1.corner 5) --
				      (hex-\width-1.corner 6);
				% Bottom left
				\draw (hex-1-1.corner 4) --
				      (hex-1-1.corner 3);
				% Top right
				\draw (hex-\width-\height.corner 6) --
				      (hex-\width-\height.corner 1);
			\end{scope}
		}
	}
	
\end{tikzpicture}
