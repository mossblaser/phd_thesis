\pgfmathsetmacro{\scale}{0.50}%
\begin{tikzpicture}[thick, scale=\scale, line cap=round]
	
	% Drop a hexagon and advance by 'cur' the specified vector
	% #1 advance vector x
	% #2 advance vector y
	% #3 advance vector z
	% #4 label
	% #5 text
	\newcommand{\hex}[5]{
		% Draw hexagon
		\draw (center) ++([hexagonXYZ]\curchipx,\curchipy,\curchipz) coordinate (cur);
		\node (#4) at (cur)
		      [ hexagon
		      , draw
		      , thin
		      , inner sep=0
		      , minimum width=\scale*1cm
		      , font=\footnotesize
		      ]
		      {#5};
		
		% Move on
		\pgfmathtruncatemacro{\x}{\curchipx + #1}
		\pgfmathtruncatemacro{\y}{\curchipy + #2}
		\pgfmathtruncatemacro{\z}{\curchipz + #3}
		\global\let\curchipx=\x
		\global\let\curchipy=\y
		\global\let\curchipz=\z
	}
	
	% #1 origin
	% #2 Number of layers
	% #3 label
	\newcommand{\wrappedtriple}[3]{
		\pgfmathtruncatemacro{\nlayers}{#2-1}
		
		\begin{scope}[hexagonXYZ]
			\coordinate (center) at (#1);
			\pgfmathtruncatemacro{\zero}{0}
			\global\let\curchipx=\zero
			\global\let\curchipy=\zero
			\global\let\curchipz=\zero
			\foreach \layer in {0,...,\nlayers}{
				\ifthenelse{\layer < 1}{}{
					\foreach \n in {1,...,\layer}{
						\hex{0}{-1}{0}{#3-\layer--y-\n}{\layer}
					}
				}
				
				\ifthenelse{\layer < 1}{}{
					\foreach \n in {1,...,\layer}{
						\hex{0}{0}{1}{#3-\layer-+z-\n}{\layer}
					}
				}
				
				\foreach \n in {0,...,\layer}{
					\hex{-1}{0}{0}{#3-\layer--x-\n}{\layer}
				}
				
				\ifthenelse{\layer < 1}{}{
					\foreach \n in {1,...,\layer}{
						\hex{0}{1}{0}{#3-\layer-+y-\n}{\layer}
					}
				}
				
				\foreach \n in {0,...,\layer}{
					\hex{0}{0}{-1}{#3-\layer--z-\n}{\layer}
				}
				
				\foreach \n in {0,...,\layer}{
					\hex{1}{0}{0}{#3-\layer-+x-\n}{\layer}
				}
			}
		\end{scope}
	}
	
	\wrappedtriple{0,0,0}{4}{x}
	

\end{tikzpicture}
