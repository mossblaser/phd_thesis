\newcommand{\cliptorus}{
	\clip (1.5, 1.5) --
	    ++(\width, 0) --
	    ++(0, \height) --
	    ++(-\width, 0) --
	      cycle;
}

% #1 width
% #2 height
\newcommand{\drawtorus}[2]{
	\pgfmathtruncatemacro{\width}{#1}
	\pgfmathtruncatemacro{\height}{#2}
	\pgfmathtruncatemacro{\widthh}{\width+1}
	\pgfmathtruncatemacro{\heightt}{\height+1}
	\pgfmathtruncatemacro{\widthhh}{\width+2}
	\pgfmathtruncatemacro{\heighttt}{\height+2}
	
	\begin{scope}
		\cliptorus
		% Links
		\begin{scope}[help lines]
			\foreach \x in {1,...,\widthhh}{
				\foreach [count=\y] \yy in {2,...,\heighttt}{
					\draw (\x, \y) -- (\x, \yy);
				}
			}
			\foreach \y in {1,...,\heighttt}{
				\foreach [count=\x] \xx in {2,...,\widthhh}{
					\draw (\x, \y) -- (\xx, \y);
				}
			}
			\foreach [count=\x] \xx in {2,...,\widthhh}{
				\foreach [count=\y] \yy in {2,...,\heighttt}{
					\draw (\x, \y) -- (\xx, \yy);
				}
			}
		\end{scope}
		
		% Nodes
		\foreach \x in {1,...,\widthhh}{
			\foreach \y in {1,...,\heighttt}{
				\node (node\x\y)
				      [ fill
				      , circle
				      , minimum size=0.5em
				      , inner sep=0
				      ]
				      at (\x, \y)
				      {};
			}
		}
	\end{scope}
}

% #1 x/y/label,...
\newcommand{\drawnodes}[1]{
	% Annotations
	\foreach \x/\y/\lab in {#1}{
		\node [ draw=black
		      , fill=white
		      , font=\tiny
		      , circle
		      , inner sep=0.1em
		      ]
		      at (\x, \y)
		      {\lab};
	}
}

% #1 colour/style
% #2 \draw-like list of coordinates
\newcommand{\drawroute}[2]{
	\begin{scope}
		\cliptorus
		\foreach \dx in {-1,...,1}{
			\foreach \dy in {-1,...,1}{
				\begin{scope}[shift={(\width*\dx, \height*\dy)}]
					\draw [line width=0.3em] [#1] #2;
				\end{scope}
			}
		}
	\end{scope}
}

% #1 coordinate of middle of dead link segment
\newcommand{\drawdeadlink}[1]{
	\node [red] at (#1) {\contour{white}{$\times$}};
}

% #1 key entries
\newcommand{\key}[1]{
	\node [below=1em of $(node2\heightt|-node22)!0.5!(node\widthh2)$] {
		\begin{tikzpicture}
			\coordinate (last);
			#1
		\end{tikzpicture}
	};
}

% #1 colour
% #2 label
\newcommand{\keyentry}[2]{
	\node [#1] [right=1em of last, fill, rectangle, minimum size=1em] (last) {};
	\node [right=0 of last] (last) {#2};
}
