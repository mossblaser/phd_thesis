\pgfmathsetmacro{\scale}{0.54}%
\begin{tikzpicture}[thick, scale=\scale, line cap=round]
	
	% Drop a hexagon and advance by 'cur' the specified vector
	% #1 advance vector x
	% #2 advance vector y
	% #3 advance vector z
	% #4 label
	\newcommand{\hex}[4]{
		% Draw hexagon
		\draw (center) ++([hexagonXYZ]\curchipx,\curchipy,\curchipz) coordinate (cur);
		\node (#4) at (cur)
		      [ hexagon
		      , draw
		      , thin
		      , inner sep=0
		      , minimum width=\scale*1cm
		      ]
		      {};
		
		% Move on
		\pgfmathtruncatemacro{\x}{\curchipx + #1}
		\pgfmathtruncatemacro{\y}{\curchipy + #2}
		\pgfmathtruncatemacro{\z}{\curchipz + #3}
		\global\let\curchipx=\x
		\global\let\curchipy=\y
		\global\let\curchipz=\z
	}
	
	% #1 origin
	% #2 Number of layers
	% #3 label
	\newcommand{\wrappedtriple}[3]{
		\pgfmathtruncatemacro{\nlayers}{#2-1}
		
		\begin{scope}[hexagonXYZ]
			\coordinate (center) at (#1);
			\pgfmathtruncatemacro{\zero}{0}
			\global\let\curchipx=\zero
			\global\let\curchipy=\zero
			\global\let\curchipz=\zero
			\foreach \layer in {0,...,\nlayers}{
				\ifthenelse{\layer < 1}{}{
					\foreach \n in {1,...,\layer}{
						\hex{0}{-1}{0}{#3-\layer--y-\n}
					}
				}
				
				\ifthenelse{\layer < 1}{}{
					\foreach \n in {1,...,\layer}{
						\hex{0}{0}{1}{#3-\layer-+z-\n}
					}
				}
				
				\foreach \n in {0,...,\layer}{
					\hex{-1}{0}{0}{#3-\layer--x-\n}
				}
				
				\ifthenelse{\layer < 1}{}{
					\foreach \n in {1,...,\layer}{
						\hex{0}{1}{0}{#3-\layer-+y-\n}
					}
				}
				
				\foreach \n in {0,...,\layer}{
					\hex{0}{0}{-1}{#3-\layer--z-\n}
				}
				
				\foreach \n in {0,...,\layer}{
					\hex{1}{0}{0}{#3-\layer-+x-\n}
				}
			}
		\end{scope}
	}
	
	% Links connecting x+ and z-
	% #1 board id a
	% #2 board id b
	% #3 pin identifiers
	% #4 perpendicular angle
	\newcommand{\linkup}[4]{
		\pgfmathsetlengthmacro{\joindistance}{2em}
		\pgfmathsetlengthmacro{\joinpull}{1em}
		
		\foreach [count=\n] \ida/\idb in {#3}{
			\coordinate (pin a\n) at (#1-\ida);
			\coordinate (pin b\n) at (#2-\idb);
		}
		
		\coordinate (join a) at ([shift={(#4:\joindistance)}]$(pin a1)!0.5!(pin a8)$);
		\coordinate (join b) at ([shift={(#4:-\joindistance)}]$(pin b1)!0.5!(pin b8)$);
		
		\foreach \n in {1,...,8}{
			\draw (pin a\n)
			      .. controls +(#4:\joinpull) and +(#4:-\joinpull) ..
			      (join a) -- (join b)
			      .. controls +(#4:\joinpull) and +(#4:-\joinpull) ..
			      (pin b\n);
		}
		\draw [line width=0.3em] (join a) -- (join b);
	}
	
	% #1 bottom-left board id
	% #2 top-right board id
	\newcommand{\linkxz}[2]{
		\linkup{#1}{#2}{%
			3-+x-3.side east/3--z-0.side west,%
			3--y-1.side north east/3--z-0.side south west,%
			3--y-1.side east/3-+y-3.side west,%
			3--y-2.side north east/3-+y-3.side south west,%
			3--y-2.side east/3-+y-2.side west,%
			3--y-3.side north east/3-+y-2.side south west,%
			3--y-3.side east/3-+y-1.side west,%
			3-+z-1.side north east/3-+y-1.side south west%
		}{30}
	}
	
	
	% #1 bottom board id
	% #2 top board id
	\newcommand{\linkyz}[2]{
		\linkup{#1}{#2}{%
			3-+x-0.side north/3-+y-1.side south,%
			3-+x-0.side north east/3--x-3.side south west,%
			3-+x-1.side north/3--x-3.side south,%
			3-+x-1.side north east/3--x-2.side south west,%
			3-+x-2.side north/3--x-2.side south,%
			3-+x-2.side north east/3--x-1.side south west,%
			3-+x-3.side north/3--x-1.side south,%
			3-+x-3.side north east/3--x-0.side south west%
		}{90}
	}
	
	% #1 bottom-right board id
	% #2 top-left board id
	\newcommand{\linkxy}[2]{
		\linkup{#1}{#2}{%
			3--z-0.side north/3--x-0.side south,%
			3--z-1.side west/3--x-0.side east,%
			3--z-1.side north/3-+z-3.side south,%
			3--z-2.side west/3-+z-3.side east,%
			3--z-2.side north/3-+z-2.side south,%
			3--z-3.side west/3-+z-2.side east,%
			3--z-3.side north/3-+z-1.side south,%
			3-+x-0.side west/3-+z-1.side east%
		}{150}
	}
	
	%%%%%%%%%%%%%%%%%%%%%%%%%%%%%%%%%%%%%%%%%%%%%%%%%%%%%%%%%%%%%%%%%%%%%%%%%%%%%%%%
	
	% Mask
	\draw ([hexagonXYZ]21, 16, -8)
	      ++([hexagonBoardXYZ]0, 0, -0.5)
	      coordinate (poi);
	
	\pgfmathsetlengthmacro{\width}{24cm}
	\pgfmathsetlengthmacro{\height}{16cm}
	\clip (poi)
	      +(-0.5*\width, -0.5*\height)
	      rectangle
	      +(0.5*\width, 0.5*\height)
	      ;
	
	% Draw the triads of boards
	\pgfmathsetmacro{\spacing}{6}
	\pgfmathtruncatemacro{\width}{4}
	\pgfmathtruncatemacro{\height}{2}
	
	\foreach \y in {1,...,\height}{
		\foreach \x in {1,...,\width}{
			\wrappedtriple{3*\spacing*\x + 0        ,3*\spacing*\y + 0,       0}{4}{\x\y1}
			\wrappedtriple{3*\spacing*\x + -\spacing,3*\spacing*\y + 0,       \spacing}{4}{\x\y0}
			\wrappedtriple{3*\spacing*\x + -\spacing,3*\spacing*\y + \spacing,0}{4}{\x\y2}
		}
	}
	
	% Draw the wires between boards
	\foreach \y in {1,...,\height}{
		\foreach \x in {1,...,\width}{
			\linkxz{\x\y0}{\x\y1}
			\linkyz{\x\y0}{\x\y2}
			\linkxy{\x\y1}{\x\y2}
		}
	}
	
	\foreach [count=\y] \yy in {2,...,\height}{
		\foreach \x in {1,...,\width}{
			\linkyz{\x\y2}{\x\yy1}
			\linkxy{\x\y2}{\x\yy0}
		}
	}
	
	\foreach \y in {1,...,\height}{
		\foreach [count=\x] \xx in {2,...,\width}{
			\linkxz{\x\y1}{\xx\y2}
			\linkxy{\xx\y0}{\x\y1}
		}
	}
	
	\foreach [count=\y] \yy in {2,...,\height}{
		\foreach [count=\x] \xx in {2,...,\width}{
			\linkxz{\x\y2}{\xx\yy0}
			\linkyz{\x\y1}{\xx\yy0}
		}
	}
	
\end{tikzpicture}
