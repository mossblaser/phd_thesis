% This file is a library of drawing macros!
% \begin{tikzpicture}
%   % This file is a library of drawing macros!
% \begin{tikzpicture}
%   % This file is a library of drawing macros!
% \begin{tikzpicture}
%   % This file is a library of drawing macros!
% \begin{tikzpicture}
%   \input{figures/distance-map.tex}
%   \meshExample{x}{y}
%   \torusExample{x}{y}
%   \hexMeshExample{x}{y}
%   \hexTorusExample{x}{y}
% \end{tikzpicture}

\pgfmathsetmacro{\scale}{3.5}
\pgfmathtruncatemacro{\numrings}{20}

% X marks the origin...
% #1 tikz location
\newcommand{\centermark}[1]{
	\node [red, inner sep=0, font=\large]
	      at (#1)
	      {\contour{white}{$\times$}};
}

% 2D mesh with contour lines
% #1 origin x
% #2 origin y
\newcommand{\meshExample}[2]{
	\buildfile{python figures/distance_gradient.py {output} --2D --mesh --origin #1 #2}{.png}
	
	\begin{scope}[scale=\scale, anchor=center]
		% Image
		\pgfmathsetlengthmacro{\height}{\scale cm}
		\pgfmathsetlengthmacro{\width}{\scale cm}
		\node [inner sep=0, anchor=south west]
		      at (0,0)
		      {\includegraphics[width=\width,height=\height]{\filename}};
		
		% Contour Lines
		\begin{scope}[white]
			\clip (0,0) -- (1,0) -- (1,1) -- (0,1) -- cycle;
			\foreach \r in {0.5,1.0,...,20.0}{
				\node [ draw
				      , diamond
				      , inner sep=0
				      , minimum size=1.3*\r cm
				      ] at (#1, #2) {};
			}
		\end{scope}
		
		% Border
		\draw (0,0) -- (1,0) -- (1,1) -- (0,1) -- cycle;
		
		% Center
		\coordinate (center) at (#1, #2);
		\centermark{center}
	\end{scope}
}

% 2D torus with contour lines
% #1 origin x
% #2 origin y
\newcommand{\torusExample}[2]{
	\buildfile{python figures/distance_gradient.py {output} --2D --torus --origin #1 #2}{.png}
	
	
	\begin{scope}[scale=\scale, anchor=center]
		% Image
		\pgfmathsetlengthmacro{\height}{\scale cm}
		\pgfmathsetlengthmacro{\width}{\scale cm}
		\node [inner sep=0, anchor=south west]
		      at (0,0)
		      {\includegraphics[width=\width,height=\height]{\filename}};
		
		% Contour Lines
		\begin{scope}[white]
			\clip (0,0) -- (1,0) -- (1,1) -- (0,1) -- cycle;
			
			\foreach \dx in {-1, 0, 1}{
				\foreach \dy in {-1, 0, 1}{
					\foreach \r in {0.5,1.0,...,20.0}{
						\begin{scope}[shift={(\dx, \dy)}]
							\clip (#1, #2) ++(-0.5, -0.5) --
							      ++(1, 0) --
							      ++(0, 1) --
							      ++(-1, 0) --
							      cycle;
							\node [ draw
							      , diamond
							      , inner sep=0
							      , minimum size=1.3*\r cm
							      ] at (#1, #2) {};
						\end{scope}
					}
				}
			}
		\end{scope}
		
		% Border
		\draw (0,0) -- (1,0) -- (1,1) -- (0,1) -- cycle;
		
		% Center
		\coordinate (center) at (#1, #2);
		\centermark{center}
	\end{scope}
}

% Hexagonal mesh with contour lines
% #1 origin x
% #2 origin y
\newcommand{\hexMeshExample}[2]{
	\buildfile{python figures/distance_gradient.py {output} --hex --mesh --origin #1 #2}{.png}
	
	
	\begin{scope}[scale=\scale, anchor=center]
		% Image
		\pgfmathsetmacro{\xshift}{-cos(60)}
		\pgfmathsetlengthmacro{\height}{\scale cm * sin(60)}
		\pgfmathsetlengthmacro{\width}{\scale cm * (cos(60) + 1)}
		\begin{scope}
			\clip [hexagonXYZ] (0,0) -- (1,0) -- (1,1) -- (0,1) -- cycle;
			\node [inner sep=0] at ([hexagonXYZ]0.5, 0.5)
			      {\includegraphics[width=\width,height=\height]{\filename}};
		\end{scope}
		
		% Contour Lines
		\begin{scope}[white]
			\clip [hexagonXYZ] (0,0) -- (1,0) -- (1,1) -- (0,1) -- cycle;
			\foreach \r in {0.5,1.0,...,20.0}{
				\node [ draw
				      , hexagonXYZ
				      , hexagonBoard
				      , inner sep=0
				      , minimum size=0.9*\r cm
				      ] at (#1, #2) {};
			}
		\end{scope}
		
		% Border
		\draw [hexagonXYZ] (0,0) -- (1,0) -- (1,1) -- (0,1) -- cycle;
		
		% Center
		\coordinate (center) at ([hexagonXYZ]#1, #2);
		\centermark{center}
	\end{scope}
}

% Hexagonal torus with contour lines
% #1 origin x
% #2 origin y
\newcommand{\hexTorusExample}[2]{
	\buildfile{python figures/distance_gradient.py {output} --hex --torus --origin #1 #2}{.png}
	
	\pgfmathsetmacro{\boardtochip}{cos(30)*2/3}
	
	\begin{scope}[scale=\scale, anchor=center]
		% Image
		\pgfmathsetlengthmacro{\height}{\scale cm * sin(60)}
		\pgfmathsetlengthmacro{\width}{\scale cm * (cos(60) + 1)}
		\begin{scope}
			\clip [hexagonXYZ] (0,0) -- (1,0) -- (1,1) -- (0,1) -- cycle;
			\node [inner sep=0] at ([hexagonXYZ]0.5, 0.5)
			      {\includegraphics[width=\width,height=\height]{\filename}};
		\end{scope}
		
		% Contour Lines
		\begin{scope}[white]
			\clip [hexagonXYZ] (0,0) -- (1,0) -- (1,1) -- (0,1) -- cycle;
			\foreach \dx in {-1, 0, 1}{
				\foreach \dy in {-1, 0, 1}{
					\foreach \r in {0.5,1.0,...,20.0}{
						\begin{scope}[shift={([hexagonXYZ]#1+\dx, #2+\dy)}]
							\clip [hexagonBoardXYZ]
							      ++(0, \boardtochip, 0) --
							      ++(\boardtochip, 0, 0) --
							      ++(0, -\boardtochip, 0) --
							      ++(0, 0, \boardtochip) --
							      ++(-\boardtochip, 0, 0) --
							      ++(0, \boardtochip, 0) --
							      cycle;
							\node [ draw
							      , hexagonXYZ
							      , hexagonBoard
							      , inner sep=0
							      , minimum size=0.9*\r cm
							      ] {};
						\end{scope}
					}
				}
			}
		\end{scope}
		
		% Define coordinates of useful locations
		\coordinate (bottom left) at ([hexagonXYZ]0, 0, 0);
		\coordinate (bottom right) at ([hexagonXYZ]1, 0, 0);
		\coordinate (top right) at ([hexagonXYZ]1, 1, 0);
		\coordinate (top left) at ([hexagonXYZ]0, 1, 0);
		
		\coordinate (bottom) at ([hexagonXYZ]0.5, 0, 0);
		\coordinate (top) at ([hexagonXYZ]0.5, 1, 0);
		\coordinate (left) at ([hexagonXYZ]0, 0.5, 0);
		\coordinate (right) at ([hexagonXYZ]1, 0.5, 0);
		
		\coordinate (center) at ([hexagonXYZ]0.5, 0.5, 0);
		
		% Border
		\draw [hexagonXYZ] (0,0) -- (1,0) -- (1,1) -- (0,1) -- cycle;
		
		% Origin
		\coordinate (origin) at ([hexagonXYZ]#1, #2);
		\centermark{origin}
	\end{scope}
}

%   \meshExample{x}{y}
%   \torusExample{x}{y}
%   \hexMeshExample{x}{y}
%   \hexTorusExample{x}{y}
% \end{tikzpicture}

\pgfmathsetmacro{\scale}{3.5}
\pgfmathtruncatemacro{\numrings}{20}

% X marks the origin...
% #1 tikz location
\newcommand{\centermark}[1]{
	\node [red, inner sep=0, font=\large]
	      at (#1)
	      {\contour{white}{$\times$}};
}

% 2D mesh with contour lines
% #1 origin x
% #2 origin y
\newcommand{\meshExample}[2]{
	\buildfile{python figures/distance_gradient.py {output} --2D --mesh --origin #1 #2}{.png}
	
	\begin{scope}[scale=\scale, anchor=center]
		% Image
		\pgfmathsetlengthmacro{\height}{\scale cm}
		\pgfmathsetlengthmacro{\width}{\scale cm}
		\node [inner sep=0, anchor=south west]
		      at (0,0)
		      {\includegraphics[width=\width,height=\height]{\filename}};
		
		% Contour Lines
		\begin{scope}[white]
			\clip (0,0) -- (1,0) -- (1,1) -- (0,1) -- cycle;
			\foreach \r in {0.5,1.0,...,20.0}{
				\node [ draw
				      , diamond
				      , inner sep=0
				      , minimum size=1.3*\r cm
				      ] at (#1, #2) {};
			}
		\end{scope}
		
		% Border
		\draw (0,0) -- (1,0) -- (1,1) -- (0,1) -- cycle;
		
		% Center
		\coordinate (center) at (#1, #2);
		\centermark{center}
	\end{scope}
}

% 2D torus with contour lines
% #1 origin x
% #2 origin y
\newcommand{\torusExample}[2]{
	\buildfile{python figures/distance_gradient.py {output} --2D --torus --origin #1 #2}{.png}
	
	
	\begin{scope}[scale=\scale, anchor=center]
		% Image
		\pgfmathsetlengthmacro{\height}{\scale cm}
		\pgfmathsetlengthmacro{\width}{\scale cm}
		\node [inner sep=0, anchor=south west]
		      at (0,0)
		      {\includegraphics[width=\width,height=\height]{\filename}};
		
		% Contour Lines
		\begin{scope}[white]
			\clip (0,0) -- (1,0) -- (1,1) -- (0,1) -- cycle;
			
			\foreach \dx in {-1, 0, 1}{
				\foreach \dy in {-1, 0, 1}{
					\foreach \r in {0.5,1.0,...,20.0}{
						\begin{scope}[shift={(\dx, \dy)}]
							\clip (#1, #2) ++(-0.5, -0.5) --
							      ++(1, 0) --
							      ++(0, 1) --
							      ++(-1, 0) --
							      cycle;
							\node [ draw
							      , diamond
							      , inner sep=0
							      , minimum size=1.3*\r cm
							      ] at (#1, #2) {};
						\end{scope}
					}
				}
			}
		\end{scope}
		
		% Border
		\draw (0,0) -- (1,0) -- (1,1) -- (0,1) -- cycle;
		
		% Center
		\coordinate (center) at (#1, #2);
		\centermark{center}
	\end{scope}
}

% Hexagonal mesh with contour lines
% #1 origin x
% #2 origin y
\newcommand{\hexMeshExample}[2]{
	\buildfile{python figures/distance_gradient.py {output} --hex --mesh --origin #1 #2}{.png}
	
	
	\begin{scope}[scale=\scale, anchor=center]
		% Image
		\pgfmathsetmacro{\xshift}{-cos(60)}
		\pgfmathsetlengthmacro{\height}{\scale cm * sin(60)}
		\pgfmathsetlengthmacro{\width}{\scale cm * (cos(60) + 1)}
		\begin{scope}
			\clip [hexagonXYZ] (0,0) -- (1,0) -- (1,1) -- (0,1) -- cycle;
			\node [inner sep=0] at ([hexagonXYZ]0.5, 0.5)
			      {\includegraphics[width=\width,height=\height]{\filename}};
		\end{scope}
		
		% Contour Lines
		\begin{scope}[white]
			\clip [hexagonXYZ] (0,0) -- (1,0) -- (1,1) -- (0,1) -- cycle;
			\foreach \r in {0.5,1.0,...,20.0}{
				\node [ draw
				      , hexagonXYZ
				      , hexagonBoard
				      , inner sep=0
				      , minimum size=0.9*\r cm
				      ] at (#1, #2) {};
			}
		\end{scope}
		
		% Border
		\draw [hexagonXYZ] (0,0) -- (1,0) -- (1,1) -- (0,1) -- cycle;
		
		% Center
		\coordinate (center) at ([hexagonXYZ]#1, #2);
		\centermark{center}
	\end{scope}
}

% Hexagonal torus with contour lines
% #1 origin x
% #2 origin y
\newcommand{\hexTorusExample}[2]{
	\buildfile{python figures/distance_gradient.py {output} --hex --torus --origin #1 #2}{.png}
	
	\pgfmathsetmacro{\boardtochip}{cos(30)*2/3}
	
	\begin{scope}[scale=\scale, anchor=center]
		% Image
		\pgfmathsetlengthmacro{\height}{\scale cm * sin(60)}
		\pgfmathsetlengthmacro{\width}{\scale cm * (cos(60) + 1)}
		\begin{scope}
			\clip [hexagonXYZ] (0,0) -- (1,0) -- (1,1) -- (0,1) -- cycle;
			\node [inner sep=0] at ([hexagonXYZ]0.5, 0.5)
			      {\includegraphics[width=\width,height=\height]{\filename}};
		\end{scope}
		
		% Contour Lines
		\begin{scope}[white]
			\clip [hexagonXYZ] (0,0) -- (1,0) -- (1,1) -- (0,1) -- cycle;
			\foreach \dx in {-1, 0, 1}{
				\foreach \dy in {-1, 0, 1}{
					\foreach \r in {0.5,1.0,...,20.0}{
						\begin{scope}[shift={([hexagonXYZ]#1+\dx, #2+\dy)}]
							\clip [hexagonBoardXYZ]
							      ++(0, \boardtochip, 0) --
							      ++(\boardtochip, 0, 0) --
							      ++(0, -\boardtochip, 0) --
							      ++(0, 0, \boardtochip) --
							      ++(-\boardtochip, 0, 0) --
							      ++(0, \boardtochip, 0) --
							      cycle;
							\node [ draw
							      , hexagonXYZ
							      , hexagonBoard
							      , inner sep=0
							      , minimum size=0.9*\r cm
							      ] {};
						\end{scope}
					}
				}
			}
		\end{scope}
		
		% Define coordinates of useful locations
		\coordinate (bottom left) at ([hexagonXYZ]0, 0, 0);
		\coordinate (bottom right) at ([hexagonXYZ]1, 0, 0);
		\coordinate (top right) at ([hexagonXYZ]1, 1, 0);
		\coordinate (top left) at ([hexagonXYZ]0, 1, 0);
		
		\coordinate (bottom) at ([hexagonXYZ]0.5, 0, 0);
		\coordinate (top) at ([hexagonXYZ]0.5, 1, 0);
		\coordinate (left) at ([hexagonXYZ]0, 0.5, 0);
		\coordinate (right) at ([hexagonXYZ]1, 0.5, 0);
		
		\coordinate (center) at ([hexagonXYZ]0.5, 0.5, 0);
		
		% Border
		\draw [hexagonXYZ] (0,0) -- (1,0) -- (1,1) -- (0,1) -- cycle;
		
		% Origin
		\coordinate (origin) at ([hexagonXYZ]#1, #2);
		\centermark{origin}
	\end{scope}
}

%   \meshExample{x}{y}
%   \torusExample{x}{y}
%   \hexMeshExample{x}{y}
%   \hexTorusExample{x}{y}
% \end{tikzpicture}

\pgfmathsetmacro{\scale}{3.5}
\pgfmathtruncatemacro{\numrings}{20}

% X marks the origin...
% #1 tikz location
\newcommand{\centermark}[1]{
	\node [red, inner sep=0, font=\large]
	      at (#1)
	      {\contour{white}{$\times$}};
}

% 2D mesh with contour lines
% #1 origin x
% #2 origin y
\newcommand{\meshExample}[2]{
	\buildfile{python figures/distance_gradient.py {output} --2D --mesh --origin #1 #2}{.png}
	
	\begin{scope}[scale=\scale, anchor=center]
		% Image
		\pgfmathsetlengthmacro{\height}{\scale cm}
		\pgfmathsetlengthmacro{\width}{\scale cm}
		\node [inner sep=0, anchor=south west]
		      at (0,0)
		      {\includegraphics[width=\width,height=\height]{\filename}};
		
		% Contour Lines
		\begin{scope}[white]
			\clip (0,0) -- (1,0) -- (1,1) -- (0,1) -- cycle;
			\foreach \r in {0.5,1.0,...,20.0}{
				\node [ draw
				      , diamond
				      , inner sep=0
				      , minimum size=1.3*\r cm
				      ] at (#1, #2) {};
			}
		\end{scope}
		
		% Border
		\draw (0,0) -- (1,0) -- (1,1) -- (0,1) -- cycle;
		
		% Center
		\coordinate (center) at (#1, #2);
		\centermark{center}
	\end{scope}
}

% 2D torus with contour lines
% #1 origin x
% #2 origin y
\newcommand{\torusExample}[2]{
	\buildfile{python figures/distance_gradient.py {output} --2D --torus --origin #1 #2}{.png}
	
	
	\begin{scope}[scale=\scale, anchor=center]
		% Image
		\pgfmathsetlengthmacro{\height}{\scale cm}
		\pgfmathsetlengthmacro{\width}{\scale cm}
		\node [inner sep=0, anchor=south west]
		      at (0,0)
		      {\includegraphics[width=\width,height=\height]{\filename}};
		
		% Contour Lines
		\begin{scope}[white]
			\clip (0,0) -- (1,0) -- (1,1) -- (0,1) -- cycle;
			
			\foreach \dx in {-1, 0, 1}{
				\foreach \dy in {-1, 0, 1}{
					\foreach \r in {0.5,1.0,...,20.0}{
						\begin{scope}[shift={(\dx, \dy)}]
							\clip (#1, #2) ++(-0.5, -0.5) --
							      ++(1, 0) --
							      ++(0, 1) --
							      ++(-1, 0) --
							      cycle;
							\node [ draw
							      , diamond
							      , inner sep=0
							      , minimum size=1.3*\r cm
							      ] at (#1, #2) {};
						\end{scope}
					}
				}
			}
		\end{scope}
		
		% Border
		\draw (0,0) -- (1,0) -- (1,1) -- (0,1) -- cycle;
		
		% Center
		\coordinate (center) at (#1, #2);
		\centermark{center}
	\end{scope}
}

% Hexagonal mesh with contour lines
% #1 origin x
% #2 origin y
\newcommand{\hexMeshExample}[2]{
	\buildfile{python figures/distance_gradient.py {output} --hex --mesh --origin #1 #2}{.png}
	
	
	\begin{scope}[scale=\scale, anchor=center]
		% Image
		\pgfmathsetmacro{\xshift}{-cos(60)}
		\pgfmathsetlengthmacro{\height}{\scale cm * sin(60)}
		\pgfmathsetlengthmacro{\width}{\scale cm * (cos(60) + 1)}
		\begin{scope}
			\clip [hexagonXYZ] (0,0) -- (1,0) -- (1,1) -- (0,1) -- cycle;
			\node [inner sep=0] at ([hexagonXYZ]0.5, 0.5)
			      {\includegraphics[width=\width,height=\height]{\filename}};
		\end{scope}
		
		% Contour Lines
		\begin{scope}[white]
			\clip [hexagonXYZ] (0,0) -- (1,0) -- (1,1) -- (0,1) -- cycle;
			\foreach \r in {0.5,1.0,...,20.0}{
				\node [ draw
				      , hexagonXYZ
				      , hexagonBoard
				      , inner sep=0
				      , minimum size=0.9*\r cm
				      ] at (#1, #2) {};
			}
		\end{scope}
		
		% Border
		\draw [hexagonXYZ] (0,0) -- (1,0) -- (1,1) -- (0,1) -- cycle;
		
		% Center
		\coordinate (center) at ([hexagonXYZ]#1, #2);
		\centermark{center}
	\end{scope}
}

% Hexagonal torus with contour lines
% #1 origin x
% #2 origin y
\newcommand{\hexTorusExample}[2]{
	\buildfile{python figures/distance_gradient.py {output} --hex --torus --origin #1 #2}{.png}
	
	\pgfmathsetmacro{\boardtochip}{cos(30)*2/3}
	
	\begin{scope}[scale=\scale, anchor=center]
		% Image
		\pgfmathsetlengthmacro{\height}{\scale cm * sin(60)}
		\pgfmathsetlengthmacro{\width}{\scale cm * (cos(60) + 1)}
		\begin{scope}
			\clip [hexagonXYZ] (0,0) -- (1,0) -- (1,1) -- (0,1) -- cycle;
			\node [inner sep=0] at ([hexagonXYZ]0.5, 0.5)
			      {\includegraphics[width=\width,height=\height]{\filename}};
		\end{scope}
		
		% Contour Lines
		\begin{scope}[white]
			\clip [hexagonXYZ] (0,0) -- (1,0) -- (1,1) -- (0,1) -- cycle;
			\foreach \dx in {-1, 0, 1}{
				\foreach \dy in {-1, 0, 1}{
					\foreach \r in {0.5,1.0,...,20.0}{
						\begin{scope}[shift={([hexagonXYZ]#1+\dx, #2+\dy)}]
							\clip [hexagonBoardXYZ]
							      ++(0, \boardtochip, 0) --
							      ++(\boardtochip, 0, 0) --
							      ++(0, -\boardtochip, 0) --
							      ++(0, 0, \boardtochip) --
							      ++(-\boardtochip, 0, 0) --
							      ++(0, \boardtochip, 0) --
							      cycle;
							\node [ draw
							      , hexagonXYZ
							      , hexagonBoard
							      , inner sep=0
							      , minimum size=0.9*\r cm
							      ] {};
						\end{scope}
					}
				}
			}
		\end{scope}
		
		% Define coordinates of useful locations
		\coordinate (bottom left) at ([hexagonXYZ]0, 0, 0);
		\coordinate (bottom right) at ([hexagonXYZ]1, 0, 0);
		\coordinate (top right) at ([hexagonXYZ]1, 1, 0);
		\coordinate (top left) at ([hexagonXYZ]0, 1, 0);
		
		\coordinate (bottom) at ([hexagonXYZ]0.5, 0, 0);
		\coordinate (top) at ([hexagonXYZ]0.5, 1, 0);
		\coordinate (left) at ([hexagonXYZ]0, 0.5, 0);
		\coordinate (right) at ([hexagonXYZ]1, 0.5, 0);
		
		\coordinate (center) at ([hexagonXYZ]0.5, 0.5, 0);
		
		% Border
		\draw [hexagonXYZ] (0,0) -- (1,0) -- (1,1) -- (0,1) -- cycle;
		
		% Origin
		\coordinate (origin) at ([hexagonXYZ]#1, #2);
		\centermark{origin}
	\end{scope}
}

%   \meshExample{x}{y}
%   \torusExample{x}{y}
%   \hexMeshExample{x}{y}
%   \hexTorusExample{x}{y}
%   \hexTorusRatioExample{x}{y}{w}{h}{show boundaries}
% \end{tikzpicture}

\colorlet{boundry}{cb5class0}
\colorlet{contour}{cb5class1}

% Not used directly, hard coded into distance_gradient.py
\colorlet{near}{cb5class2}
\colorlet{far}{cb5class4}

\pgfmathsetmacro{\scale}{3.5}
\pgfmathtruncatemacro{\numrings}{20}

% X marks the origin...
% #1 tikz location
\newcommand{\centermark}[1]{
	\node [red, inner sep=0, font=\large]
	      at (#1)
	      {\contour{white}{$\times$}};
}

% 2D mesh with contour lines
% #1 origin x
% #2 origin y
\newcommand{\meshExample}[2]{
	\buildfile{python figures/distance_gradient.py {output} --2D --mesh --origin #1 #2}{.png}
	
	\begin{scope}[scale=\scale, anchor=center]
		% Image
		\pgfmathsetlengthmacro{\height}{\scale cm}
		\pgfmathsetlengthmacro{\width}{\scale cm}
		\node [inner sep=0, anchor=south west]
		      at (0,0)
		      {\includegraphics[width=\width,height=\height]{\filename}};
		
		% Contour Lines
		\begin{scope}[contour]
			\clip (0,0) -- (1,0) -- (1,1) -- (0,1) -- cycle;
			\foreach \r in {0.5,1.0,...,20.0}{
				\node [ draw
				      , diamond
				      , inner sep=0
				      , minimum size=1.3*\r cm
				      ] at (#1, #2) {};
			}
		\end{scope}
		
		% Border
		\draw (0,0) -- (1,0) -- (1,1) -- (0,1) -- cycle;
		
		% Center
		\coordinate (center) at (#1, #2);
		\centermark{center}
	\end{scope}
}

% 2D torus with contour lines
% #1 origin x
% #2 origin y
\newcommand{\torusExample}[2]{
	\buildfile{python figures/distance_gradient.py {output} --2D --torus --origin #1 #2}{.png}
	
	
	\begin{scope}[scale=\scale, anchor=center]
		% Image
		\pgfmathsetlengthmacro{\height}{\scale cm}
		\pgfmathsetlengthmacro{\width}{\scale cm}
		\node [inner sep=0, anchor=south west]
		      at (0,0)
		      {\includegraphics[width=\width,height=\height]{\filename}};
		
		% Contour Lines
		\begin{scope}[contour]
			\clip (0,0) -- (1,0) -- (1,1) -- (0,1) -- cycle;
			
			\foreach \dx in {-1, 0, 1}{
				\foreach \dy in {-1, 0, 1}{
					\foreach \r in {0.5,1.0,...,20.0}{
						\begin{scope}[shift={(\dx, \dy)}]
							\clip (#1, #2) ++(-0.5, -0.5) --
							      ++(1, 0) --
							      ++(0, 1) --
							      ++(-1, 0) --
							      cycle;
							\node [ draw
							      , diamond
							      , inner sep=0
							      , minimum size=1.3*\r cm
							      ] at (#1, #2) {};
						\end{scope}
					}
				}
			}
		\end{scope}
		
		% Border
		\draw (0,0) -- (1,0) -- (1,1) -- (0,1) -- cycle;
		
		% Center
		\coordinate (center) at (#1, #2);
		\centermark{center}
	\end{scope}
}

% Hexagonal mesh with contour lines
% #1 origin x
% #2 origin y
\newcommand{\hexMeshExample}[2]{
	\buildfile{python figures/distance_gradient.py {output} --hex --mesh --origin #1 #2}{.png}
	
	
	\begin{scope}[scale=\scale, anchor=center]
		% Image
		\pgfmathsetmacro{\xshift}{-cos(60)}
		\pgfmathsetlengthmacro{\height}{\scale cm * sin(60)}
		\pgfmathsetlengthmacro{\width}{\scale cm * (cos(60) + 1)}
		\begin{scope}
			\clip [hexagonXYZ] (0,0) -- (1,0) -- (1,1) -- (0,1) -- cycle;
			\node [inner sep=0] at ([hexagonXYZ]0.5, 0.5)
			      {\includegraphics[width=\width,height=\height]{\filename}};
		\end{scope}
		
		% Contour Lines
		\begin{scope}[contour]
			\clip [hexagonXYZ] (0,0) -- (1,0) -- (1,1) -- (0,1) -- cycle;
			\foreach \r in {0.5,1.0,...,20.0}{
				\node [ draw
				      , hexagonXYZ
				      , hexagonBoard
				      , inner sep=0
				      , minimum size=0.9*\r cm
				      ] at (#1, #2) {};
			}
		\end{scope}
		
		% Border
		\draw [hexagonXYZ] (0,0) -- (1,0) -- (1,1) -- (0,1) -- cycle;
		
		% Center
		\coordinate (center) at ([hexagonXYZ]#1, #2);
		\centermark{center}
	\end{scope}
}

% Non-square hexagonal torus with contour lines eminating from (0, 0)
% #1 origin x
% #2 origin y
% #3 width
% #4 height
% #5 draw regions (1) or just contour lines (0)
\newcommand{\hexTorusRatioExample}[5]{
	\buildfile{python figures/distance_gradient.py {output} --hex --torus --origin #1 #2 --dimensions #3 #4}{.png}
	
	\pgfmathsetmacro{\boardtochip}{cos(30)*2/3}
	
	\begin{scope}[scale=\scale, anchor=center]
		% Image
		\pgfmathsetlengthmacro{\height}{\scale cm * (#4 * sin(60))}
		\pgfmathsetlengthmacro{\width}{\scale cm * ((#4 * cos(60)) + #3)}
		\begin{scope}
			\clip [hexagonXYZ] (0,0) -- (#3,0) -- (#3,#4) -- (0,#4) -- cycle;
			\node [inner sep=0] at ([hexagonXYZ]0.5*#3, 0.5*#4)
			      {\includegraphics[width=\width,height=\height]{\filename}};
		\end{scope}
		
		% Contour Lines
		\begin{scope}
			\clip [hexagonXYZ] (0,0) -- (#3,0) -- (#3,#4) -- (0,#4) -- cycle;
			
			\pgfmathsetmacro{\a}{#4/2 * sin(30)}
			\pgfmathsetmacro{\b}{#3/2 - #4*cos(60)}
			\pgfmathsetmacro{\c}{(\a + \b) / cos(30)}
			\pgfmathsetmacro{\d}{#4/2 * cos(30)}
			\pgfmathsetmacro{\e}{\c * sin(30)}
			\pgfmathsetmacro{\f}{(\e + \d) * tan(30)}
			\pgfmathsetmacro{\g}{#3/2 - (\b + \f)}
			\pgfmathsetmacro{\i}{(#4 * cos(30)) - (\e + \d)}
			\pgfmathsetmacro{\j}{-(tan(30) * \i)/((tan(30)*tan(30)) - 1)}
			\pgfmathsetmacro{\k}{\j / cos(30)}
			\pgfmathsetmacro{\l}{\j / sin(30)}
			\pgfmathsetmacro{\m}{#4 - (2*\l)}
			\pgfmathsetmacro{\n}{#4 * cos(30)}
			\pgfmathsetmacro{\o}{(\n/2)/sin(30)}
			\pgfmathsetmacro{\p}{\o * cos(30)}
			\pgfmathsetmacro{\q}{#4 * sin(30)}
			\pgfmathsetmacro{\r}{\a + (#3/2) - \q}
			\pgfmathsetmacro{\s}{\r / cos(30)}
			\pgfmathsetmacro{\t}{#3 - (4*\r)}
			\pgfmathsetmacro{\u}{\s * sin(30)}
			\pgfmathsetmacro{\v}{\d-\u}
			\pgfmathsetmacro{\aa}{(#3/2) - (#3 - \a)}
			\pgfmathsetmacro{\ab}{\aa / sin(30)}
			\pgfmathsetmacro{\ac}{\ab * cos(30)}
			\pgfmathsetmacro{\ad}{\d-\ac}
			\pgfmathsetmacro{\ae}{\b-(\p-\a)}
			
			\coordinate (left middle) at ([hexagonXYZ]0, 0.5*#4, 0);
			\foreach \x in {-1,0,1}{
				\foreach \y in {-1,0,1}{
					\begin{scope}[ shift={(0:\x*#3 + #1)}
					             , shift={(120:\y*#4 + #2)}
					             ]
						\pgfmathsetmacro{\ratio}{#3/#4}
						\ifthenelse{\lengthtest{\ratio pt > 2pt}}{
							% Ratio > 2
							\newcommand{\thepath}{++(120:#4/2)
							                      ++(30:-\o)
							                   -- ++(30:2*\o)
							                   -- ++(0:\ae)
							                   -- ++(-60:#4)
							                   -- ++(180:\ae)
							                   -- ++(180+30:2*\o)
							                   -- ++(180:\ae)
							                   -- ++(180-60:#4)
							                   -- cycle}
						}{\ifthenelse{\lengthtest{\ratio pt > 1pt}}{
							% 1 < Ratio <= 2
							\newcommand{\thepath}{++(120:#4/2)
							                      ++(210:\c)
							                   -- ++(30:2*\c)
							                   -- ++(-30:\k)
							                   -- ++(-60:\m)
							                   -- ++(-30:\k)
							                   -- ++(-90:2*\i)
							                   -- ++(180+30:2*\c)
							                   -- ++(180-30:\k)
							                   -- ++(180-60:\m)
							                   -- ++(180-30:\k)
							                   -- cycle}
						}{\ifthenelse{\lengthtest{\ratio pt > 0.5pt}}{
							% 0.5 < Ratio <= 1
							\newcommand{\thepath}{++(120:#4/2)
							                      ++(210:\s)
							                   -- ++(30:2*\s)
							                   -- ++(-30:\s)
							                   -- ++(0:\t)
							                   -- ++(-30:\s)
							                   -- ++(-90:2*\v)
							                   -- ++(180+30:2*\s)
							                   -- ++(180-30:\s)
							                   -- ++(180:\t)
							                   -- ++(180-30:\s)
							                   -- cycle}
						}{
							% Ratio <= 0.5
							\newcommand{\thepath}{++(120:#4/2)
							                   -- ++(0:#3)
							                   -- ++(-60:\ab)
							                   -- ++(-90:2*\ad)
							                   -- ++(-60:\ab)
							                   -- ++(180:#3)
							                   -- ++(180-60:\ab)
							                   -- ++(180-90:2*\ad)
							                   -- cycle}
						}}}
						
						\begin{scope}
							\clip \thepath;
							\foreach \r in {0.5,1,...,30}{
								\node [ draw
								      , contour
								      , hexagonXYZ
								      , hexagonBoard
								      , inner sep=0
								      , minimum size=0.9*\r cm
								      ] {};
							}
						\end{scope}
						\ifthenelse{#5 > 0}{
							\draw [boundry] \thepath;
						}{}
					\end{scope}
				}
			}
			
		\end{scope}
		
		% Define coordinates of useful locations
		\coordinate (bottom left) at ([hexagonXYZ]0, 0, 0);
		\coordinate (bottom right) at ([hexagonXYZ]#3, 0, 0);
		\coordinate (top right) at ([hexagonXYZ]#3, #4, 0);
		\coordinate (top left) at ([hexagonXYZ]0, #4, 0);
		
		\coordinate (bottom) at ([hexagonXYZ]0.5*#3, 0, 0);
		\coordinate (top) at ([hexagonXYZ]0.5*#3, #4, 0);
		\coordinate (left) at ([hexagonXYZ]0, 0.5*#4, 0);
		\coordinate (right) at ([hexagonXYZ]1*#3, 0.5*#4, 0);
		
		\coordinate (center) at ([hexagonXYZ]0.5*#3, 0.5*#4, 0);
		
		% Border
		\draw [hexagonXYZ] (0,0) -- (#3,0) -- (#3,#4) -- (0,#4) -- cycle;
		
		% Origin
		\coordinate (origin) at ([hexagonXYZ]#1, #2);
		\centermark{origin}
	\end{scope}
}

% Hexagonal torus with contour lines
% #1 origin x
% #2 origin y
\newcommand{\hexTorusExample}[2]{
	\hexTorusRatioExample{#1}{#2}{1}{1}{0}
}
