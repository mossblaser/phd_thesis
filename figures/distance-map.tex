% This file is a library of drawing macros!
% \begin{tikzpicture}
%   % This file is a library of drawing macros!
% \begin{tikzpicture}
%   % This file is a library of drawing macros!
% \begin{tikzpicture}
%   % This file is a library of drawing macros!
% \begin{tikzpicture}
%   \input{figures/distance-map.tex}
%   \meshExample{x}{y}
%   \torusExample{x}{y}
%   \hexMeshExample{x}{y}
%   \hexTorusExample{x}{y}
% \end{tikzpicture}

\pgfmathsetmacro{\scale}{3.5}
\pgfmathtruncatemacro{\numrings}{20}

% X marks the origin...
% #1 tikz location
\newcommand{\centermark}[1]{
	\node [red, inner sep=0, font=\large]
	      at (#1)
	      {\contour{white}{$\times$}};
}

% 2D mesh with contour lines
% #1 origin x
% #2 origin y
\newcommand{\meshExample}[2]{
	\buildfile{python figures/distance_gradient.py {output} --2D --mesh --origin #1 #2}{.png}
	
	\begin{scope}[scale=\scale, anchor=center]
		% Image
		\pgfmathsetlengthmacro{\height}{\scale cm}
		\pgfmathsetlengthmacro{\width}{\scale cm}
		\node [inner sep=0, anchor=south west]
		      at (0,0)
		      {\includegraphics[width=\width,height=\height]{\filename}};
		
		% Contour Lines
		\begin{scope}[white]
			\clip (0,0) -- (1,0) -- (1,1) -- (0,1) -- cycle;
			\foreach \r in {0.5,1.0,...,20.0}{
				\node [ draw
				      , diamond
				      , inner sep=0
				      , minimum size=1.3*\r cm
				      ] at (#1, #2) {};
			}
		\end{scope}
		
		% Border
		\draw (0,0) -- (1,0) -- (1,1) -- (0,1) -- cycle;
		
		% Center
		\coordinate (center) at (#1, #2);
		\centermark{center}
	\end{scope}
}

% 2D torus with contour lines
% #1 origin x
% #2 origin y
\newcommand{\torusExample}[2]{
	\buildfile{python figures/distance_gradient.py {output} --2D --torus --origin #1 #2}{.png}
	
	
	\begin{scope}[scale=\scale, anchor=center]
		% Image
		\pgfmathsetlengthmacro{\height}{\scale cm}
		\pgfmathsetlengthmacro{\width}{\scale cm}
		\node [inner sep=0, anchor=south west]
		      at (0,0)
		      {\includegraphics[width=\width,height=\height]{\filename}};
		
		% Contour Lines
		\begin{scope}[white]
			\clip (0,0) -- (1,0) -- (1,1) -- (0,1) -- cycle;
			
			\foreach \dx in {-1, 0, 1}{
				\foreach \dy in {-1, 0, 1}{
					\foreach \r in {0.5,1.0,...,20.0}{
						\begin{scope}[shift={(\dx, \dy)}]
							\clip (#1, #2) ++(-0.5, -0.5) --
							      ++(1, 0) --
							      ++(0, 1) --
							      ++(-1, 0) --
							      cycle;
							\node [ draw
							      , diamond
							      , inner sep=0
							      , minimum size=1.3*\r cm
							      ] at (#1, #2) {};
						\end{scope}
					}
				}
			}
		\end{scope}
		
		% Border
		\draw (0,0) -- (1,0) -- (1,1) -- (0,1) -- cycle;
		
		% Center
		\coordinate (center) at (#1, #2);
		\centermark{center}
	\end{scope}
}

% Hexagonal mesh with contour lines
% #1 origin x
% #2 origin y
\newcommand{\hexMeshExample}[2]{
	\buildfile{python figures/distance_gradient.py {output} --hex --mesh --origin #1 #2}{.png}
	
	
	\begin{scope}[scale=\scale, anchor=center]
		% Image
		\pgfmathsetmacro{\xshift}{-cos(60)}
		\pgfmathsetlengthmacro{\height}{\scale cm * sin(60)}
		\pgfmathsetlengthmacro{\width}{\scale cm * (cos(60) + 1)}
		\begin{scope}
			\clip [hexagonXYZ] (0,0) -- (1,0) -- (1,1) -- (0,1) -- cycle;
			\node [inner sep=0] at ([hexagonXYZ]0.5, 0.5)
			      {\includegraphics[width=\width,height=\height]{\filename}};
		\end{scope}
		
		% Contour Lines
		\begin{scope}[white]
			\clip [hexagonXYZ] (0,0) -- (1,0) -- (1,1) -- (0,1) -- cycle;
			\foreach \r in {0.5,1.0,...,20.0}{
				\node [ draw
				      , hexagonXYZ
				      , hexagonBoard
				      , inner sep=0
				      , minimum size=0.9*\r cm
				      ] at (#1, #2) {};
			}
		\end{scope}
		
		% Border
		\draw [hexagonXYZ] (0,0) -- (1,0) -- (1,1) -- (0,1) -- cycle;
		
		% Center
		\coordinate (center) at ([hexagonXYZ]#1, #2);
		\centermark{center}
	\end{scope}
}

% Hexagonal torus with contour lines
% #1 origin x
% #2 origin y
\newcommand{\hexTorusExample}[2]{
	\buildfile{python figures/distance_gradient.py {output} --hex --torus --origin #1 #2}{.png}
	
	\pgfmathsetmacro{\boardtochip}{cos(30)*2/3}
	
	\begin{scope}[scale=\scale, anchor=center]
		% Image
		\pgfmathsetlengthmacro{\height}{\scale cm * sin(60)}
		\pgfmathsetlengthmacro{\width}{\scale cm * (cos(60) + 1)}
		\begin{scope}
			\clip [hexagonXYZ] (0,0) -- (1,0) -- (1,1) -- (0,1) -- cycle;
			\node [inner sep=0] at ([hexagonXYZ]0.5, 0.5)
			      {\includegraphics[width=\width,height=\height]{\filename}};
		\end{scope}
		
		% Contour Lines
		\begin{scope}[white]
			\clip [hexagonXYZ] (0,0) -- (1,0) -- (1,1) -- (0,1) -- cycle;
			\foreach \dx in {-1, 0, 1}{
				\foreach \dy in {-1, 0, 1}{
					\foreach \r in {0.5,1.0,...,20.0}{
						\begin{scope}[shift={([hexagonXYZ]#1+\dx, #2+\dy)}]
							\clip [hexagonBoardXYZ]
							      ++(0, \boardtochip, 0) --
							      ++(\boardtochip, 0, 0) --
							      ++(0, -\boardtochip, 0) --
							      ++(0, 0, \boardtochip) --
							      ++(-\boardtochip, 0, 0) --
							      ++(0, \boardtochip, 0) --
							      cycle;
							\node [ draw
							      , hexagonXYZ
							      , hexagonBoard
							      , inner sep=0
							      , minimum size=0.9*\r cm
							      ] {};
						\end{scope}
					}
				}
			}
		\end{scope}
		
		% Define coordinates of useful locations
		\coordinate (bottom left) at ([hexagonXYZ]0, 0, 0);
		\coordinate (bottom right) at ([hexagonXYZ]1, 0, 0);
		\coordinate (top right) at ([hexagonXYZ]1, 1, 0);
		\coordinate (top left) at ([hexagonXYZ]0, 1, 0);
		
		\coordinate (bottom) at ([hexagonXYZ]0.5, 0, 0);
		\coordinate (top) at ([hexagonXYZ]0.5, 1, 0);
		\coordinate (left) at ([hexagonXYZ]0, 0.5, 0);
		\coordinate (right) at ([hexagonXYZ]1, 0.5, 0);
		
		\coordinate (center) at ([hexagonXYZ]0.5, 0.5, 0);
		
		% Border
		\draw [hexagonXYZ] (0,0) -- (1,0) -- (1,1) -- (0,1) -- cycle;
		
		% Origin
		\coordinate (origin) at ([hexagonXYZ]#1, #2);
		\centermark{origin}
	\end{scope}
}

%   \meshExample{x}{y}
%   \torusExample{x}{y}
%   \hexMeshExample{x}{y}
%   \hexTorusExample{x}{y}
% \end{tikzpicture}

\pgfmathsetmacro{\scale}{3.5}
\pgfmathtruncatemacro{\numrings}{20}

% X marks the origin...
% #1 tikz location
\newcommand{\centermark}[1]{
	\node [red, inner sep=0, font=\large]
	      at (#1)
	      {\contour{white}{$\times$}};
}

% 2D mesh with contour lines
% #1 origin x
% #2 origin y
\newcommand{\meshExample}[2]{
	\buildfile{python figures/distance_gradient.py {output} --2D --mesh --origin #1 #2}{.png}
	
	\begin{scope}[scale=\scale, anchor=center]
		% Image
		\pgfmathsetlengthmacro{\height}{\scale cm}
		\pgfmathsetlengthmacro{\width}{\scale cm}
		\node [inner sep=0, anchor=south west]
		      at (0,0)
		      {\includegraphics[width=\width,height=\height]{\filename}};
		
		% Contour Lines
		\begin{scope}[white]
			\clip (0,0) -- (1,0) -- (1,1) -- (0,1) -- cycle;
			\foreach \r in {0.5,1.0,...,20.0}{
				\node [ draw
				      , diamond
				      , inner sep=0
				      , minimum size=1.3*\r cm
				      ] at (#1, #2) {};
			}
		\end{scope}
		
		% Border
		\draw (0,0) -- (1,0) -- (1,1) -- (0,1) -- cycle;
		
		% Center
		\coordinate (center) at (#1, #2);
		\centermark{center}
	\end{scope}
}

% 2D torus with contour lines
% #1 origin x
% #2 origin y
\newcommand{\torusExample}[2]{
	\buildfile{python figures/distance_gradient.py {output} --2D --torus --origin #1 #2}{.png}
	
	
	\begin{scope}[scale=\scale, anchor=center]
		% Image
		\pgfmathsetlengthmacro{\height}{\scale cm}
		\pgfmathsetlengthmacro{\width}{\scale cm}
		\node [inner sep=0, anchor=south west]
		      at (0,0)
		      {\includegraphics[width=\width,height=\height]{\filename}};
		
		% Contour Lines
		\begin{scope}[white]
			\clip (0,0) -- (1,0) -- (1,1) -- (0,1) -- cycle;
			
			\foreach \dx in {-1, 0, 1}{
				\foreach \dy in {-1, 0, 1}{
					\foreach \r in {0.5,1.0,...,20.0}{
						\begin{scope}[shift={(\dx, \dy)}]
							\clip (#1, #2) ++(-0.5, -0.5) --
							      ++(1, 0) --
							      ++(0, 1) --
							      ++(-1, 0) --
							      cycle;
							\node [ draw
							      , diamond
							      , inner sep=0
							      , minimum size=1.3*\r cm
							      ] at (#1, #2) {};
						\end{scope}
					}
				}
			}
		\end{scope}
		
		% Border
		\draw (0,0) -- (1,0) -- (1,1) -- (0,1) -- cycle;
		
		% Center
		\coordinate (center) at (#1, #2);
		\centermark{center}
	\end{scope}
}

% Hexagonal mesh with contour lines
% #1 origin x
% #2 origin y
\newcommand{\hexMeshExample}[2]{
	\buildfile{python figures/distance_gradient.py {output} --hex --mesh --origin #1 #2}{.png}
	
	
	\begin{scope}[scale=\scale, anchor=center]
		% Image
		\pgfmathsetmacro{\xshift}{-cos(60)}
		\pgfmathsetlengthmacro{\height}{\scale cm * sin(60)}
		\pgfmathsetlengthmacro{\width}{\scale cm * (cos(60) + 1)}
		\begin{scope}
			\clip [hexagonXYZ] (0,0) -- (1,0) -- (1,1) -- (0,1) -- cycle;
			\node [inner sep=0] at ([hexagonXYZ]0.5, 0.5)
			      {\includegraphics[width=\width,height=\height]{\filename}};
		\end{scope}
		
		% Contour Lines
		\begin{scope}[white]
			\clip [hexagonXYZ] (0,0) -- (1,0) -- (1,1) -- (0,1) -- cycle;
			\foreach \r in {0.5,1.0,...,20.0}{
				\node [ draw
				      , hexagonXYZ
				      , hexagonBoard
				      , inner sep=0
				      , minimum size=0.9*\r cm
				      ] at (#1, #2) {};
			}
		\end{scope}
		
		% Border
		\draw [hexagonXYZ] (0,0) -- (1,0) -- (1,1) -- (0,1) -- cycle;
		
		% Center
		\coordinate (center) at ([hexagonXYZ]#1, #2);
		\centermark{center}
	\end{scope}
}

% Hexagonal torus with contour lines
% #1 origin x
% #2 origin y
\newcommand{\hexTorusExample}[2]{
	\buildfile{python figures/distance_gradient.py {output} --hex --torus --origin #1 #2}{.png}
	
	\pgfmathsetmacro{\boardtochip}{cos(30)*2/3}
	
	\begin{scope}[scale=\scale, anchor=center]
		% Image
		\pgfmathsetlengthmacro{\height}{\scale cm * sin(60)}
		\pgfmathsetlengthmacro{\width}{\scale cm * (cos(60) + 1)}
		\begin{scope}
			\clip [hexagonXYZ] (0,0) -- (1,0) -- (1,1) -- (0,1) -- cycle;
			\node [inner sep=0] at ([hexagonXYZ]0.5, 0.5)
			      {\includegraphics[width=\width,height=\height]{\filename}};
		\end{scope}
		
		% Contour Lines
		\begin{scope}[white]
			\clip [hexagonXYZ] (0,0) -- (1,0) -- (1,1) -- (0,1) -- cycle;
			\foreach \dx in {-1, 0, 1}{
				\foreach \dy in {-1, 0, 1}{
					\foreach \r in {0.5,1.0,...,20.0}{
						\begin{scope}[shift={([hexagonXYZ]#1+\dx, #2+\dy)}]
							\clip [hexagonBoardXYZ]
							      ++(0, \boardtochip, 0) --
							      ++(\boardtochip, 0, 0) --
							      ++(0, -\boardtochip, 0) --
							      ++(0, 0, \boardtochip) --
							      ++(-\boardtochip, 0, 0) --
							      ++(0, \boardtochip, 0) --
							      cycle;
							\node [ draw
							      , hexagonXYZ
							      , hexagonBoard
							      , inner sep=0
							      , minimum size=0.9*\r cm
							      ] {};
						\end{scope}
					}
				}
			}
		\end{scope}
		
		% Define coordinates of useful locations
		\coordinate (bottom left) at ([hexagonXYZ]0, 0, 0);
		\coordinate (bottom right) at ([hexagonXYZ]1, 0, 0);
		\coordinate (top right) at ([hexagonXYZ]1, 1, 0);
		\coordinate (top left) at ([hexagonXYZ]0, 1, 0);
		
		\coordinate (bottom) at ([hexagonXYZ]0.5, 0, 0);
		\coordinate (top) at ([hexagonXYZ]0.5, 1, 0);
		\coordinate (left) at ([hexagonXYZ]0, 0.5, 0);
		\coordinate (right) at ([hexagonXYZ]1, 0.5, 0);
		
		\coordinate (center) at ([hexagonXYZ]0.5, 0.5, 0);
		
		% Border
		\draw [hexagonXYZ] (0,0) -- (1,0) -- (1,1) -- (0,1) -- cycle;
		
		% Origin
		\coordinate (origin) at ([hexagonXYZ]#1, #2);
		\centermark{origin}
	\end{scope}
}

%   \meshExample{x}{y}
%   \torusExample{x}{y}
%   \hexMeshExample{x}{y}
%   \hexTorusExample{x}{y}
% \end{tikzpicture}

\pgfmathsetmacro{\scale}{3.5}
\pgfmathtruncatemacro{\numrings}{20}

% X marks the origin...
% #1 tikz location
\newcommand{\centermark}[1]{
	\node [red, inner sep=0, font=\large]
	      at (#1)
	      {\contour{white}{$\times$}};
}

% 2D mesh with contour lines
% #1 origin x
% #2 origin y
\newcommand{\meshExample}[2]{
	\buildfile{python figures/distance_gradient.py {output} --2D --mesh --origin #1 #2}{.png}
	
	\begin{scope}[scale=\scale, anchor=center]
		% Image
		\pgfmathsetlengthmacro{\height}{\scale cm}
		\pgfmathsetlengthmacro{\width}{\scale cm}
		\node [inner sep=0, anchor=south west]
		      at (0,0)
		      {\includegraphics[width=\width,height=\height]{\filename}};
		
		% Contour Lines
		\begin{scope}[white]
			\clip (0,0) -- (1,0) -- (1,1) -- (0,1) -- cycle;
			\foreach \r in {0.5,1.0,...,20.0}{
				\node [ draw
				      , diamond
				      , inner sep=0
				      , minimum size=1.3*\r cm
				      ] at (#1, #2) {};
			}
		\end{scope}
		
		% Border
		\draw (0,0) -- (1,0) -- (1,1) -- (0,1) -- cycle;
		
		% Center
		\coordinate (center) at (#1, #2);
		\centermark{center}
	\end{scope}
}

% 2D torus with contour lines
% #1 origin x
% #2 origin y
\newcommand{\torusExample}[2]{
	\buildfile{python figures/distance_gradient.py {output} --2D --torus --origin #1 #2}{.png}
	
	
	\begin{scope}[scale=\scale, anchor=center]
		% Image
		\pgfmathsetlengthmacro{\height}{\scale cm}
		\pgfmathsetlengthmacro{\width}{\scale cm}
		\node [inner sep=0, anchor=south west]
		      at (0,0)
		      {\includegraphics[width=\width,height=\height]{\filename}};
		
		% Contour Lines
		\begin{scope}[white]
			\clip (0,0) -- (1,0) -- (1,1) -- (0,1) -- cycle;
			
			\foreach \dx in {-1, 0, 1}{
				\foreach \dy in {-1, 0, 1}{
					\foreach \r in {0.5,1.0,...,20.0}{
						\begin{scope}[shift={(\dx, \dy)}]
							\clip (#1, #2) ++(-0.5, -0.5) --
							      ++(1, 0) --
							      ++(0, 1) --
							      ++(-1, 0) --
							      cycle;
							\node [ draw
							      , diamond
							      , inner sep=0
							      , minimum size=1.3*\r cm
							      ] at (#1, #2) {};
						\end{scope}
					}
				}
			}
		\end{scope}
		
		% Border
		\draw (0,0) -- (1,0) -- (1,1) -- (0,1) -- cycle;
		
		% Center
		\coordinate (center) at (#1, #2);
		\centermark{center}
	\end{scope}
}

% Hexagonal mesh with contour lines
% #1 origin x
% #2 origin y
\newcommand{\hexMeshExample}[2]{
	\buildfile{python figures/distance_gradient.py {output} --hex --mesh --origin #1 #2}{.png}
	
	
	\begin{scope}[scale=\scale, anchor=center]
		% Image
		\pgfmathsetmacro{\xshift}{-cos(60)}
		\pgfmathsetlengthmacro{\height}{\scale cm * sin(60)}
		\pgfmathsetlengthmacro{\width}{\scale cm * (cos(60) + 1)}
		\begin{scope}
			\clip [hexagonXYZ] (0,0) -- (1,0) -- (1,1) -- (0,1) -- cycle;
			\node [inner sep=0] at ([hexagonXYZ]0.5, 0.5)
			      {\includegraphics[width=\width,height=\height]{\filename}};
		\end{scope}
		
		% Contour Lines
		\begin{scope}[white]
			\clip [hexagonXYZ] (0,0) -- (1,0) -- (1,1) -- (0,1) -- cycle;
			\foreach \r in {0.5,1.0,...,20.0}{
				\node [ draw
				      , hexagonXYZ
				      , hexagonBoard
				      , inner sep=0
				      , minimum size=0.9*\r cm
				      ] at (#1, #2) {};
			}
		\end{scope}
		
		% Border
		\draw [hexagonXYZ] (0,0) -- (1,0) -- (1,1) -- (0,1) -- cycle;
		
		% Center
		\coordinate (center) at ([hexagonXYZ]#1, #2);
		\centermark{center}
	\end{scope}
}

% Hexagonal torus with contour lines
% #1 origin x
% #2 origin y
\newcommand{\hexTorusExample}[2]{
	\buildfile{python figures/distance_gradient.py {output} --hex --torus --origin #1 #2}{.png}
	
	\pgfmathsetmacro{\boardtochip}{cos(30)*2/3}
	
	\begin{scope}[scale=\scale, anchor=center]
		% Image
		\pgfmathsetlengthmacro{\height}{\scale cm * sin(60)}
		\pgfmathsetlengthmacro{\width}{\scale cm * (cos(60) + 1)}
		\begin{scope}
			\clip [hexagonXYZ] (0,0) -- (1,0) -- (1,1) -- (0,1) -- cycle;
			\node [inner sep=0] at ([hexagonXYZ]0.5, 0.5)
			      {\includegraphics[width=\width,height=\height]{\filename}};
		\end{scope}
		
		% Contour Lines
		\begin{scope}[white]
			\clip [hexagonXYZ] (0,0) -- (1,0) -- (1,1) -- (0,1) -- cycle;
			\foreach \dx in {-1, 0, 1}{
				\foreach \dy in {-1, 0, 1}{
					\foreach \r in {0.5,1.0,...,20.0}{
						\begin{scope}[shift={([hexagonXYZ]#1+\dx, #2+\dy)}]
							\clip [hexagonBoardXYZ]
							      ++(0, \boardtochip, 0) --
							      ++(\boardtochip, 0, 0) --
							      ++(0, -\boardtochip, 0) --
							      ++(0, 0, \boardtochip) --
							      ++(-\boardtochip, 0, 0) --
							      ++(0, \boardtochip, 0) --
							      cycle;
							\node [ draw
							      , hexagonXYZ
							      , hexagonBoard
							      , inner sep=0
							      , minimum size=0.9*\r cm
							      ] {};
						\end{scope}
					}
				}
			}
		\end{scope}
		
		% Define coordinates of useful locations
		\coordinate (bottom left) at ([hexagonXYZ]0, 0, 0);
		\coordinate (bottom right) at ([hexagonXYZ]1, 0, 0);
		\coordinate (top right) at ([hexagonXYZ]1, 1, 0);
		\coordinate (top left) at ([hexagonXYZ]0, 1, 0);
		
		\coordinate (bottom) at ([hexagonXYZ]0.5, 0, 0);
		\coordinate (top) at ([hexagonXYZ]0.5, 1, 0);
		\coordinate (left) at ([hexagonXYZ]0, 0.5, 0);
		\coordinate (right) at ([hexagonXYZ]1, 0.5, 0);
		
		\coordinate (center) at ([hexagonXYZ]0.5, 0.5, 0);
		
		% Border
		\draw [hexagonXYZ] (0,0) -- (1,0) -- (1,1) -- (0,1) -- cycle;
		
		% Origin
		\coordinate (origin) at ([hexagonXYZ]#1, #2);
		\centermark{origin}
	\end{scope}
}

%   \meshExample{x}{y}
%   \torusExample{x}{y}
%   \hexMeshExample{x}{y}
%   \hexTorusExample{x}{y}
% \end{tikzpicture}

\pgfmathsetmacro{\scale}{3.5}
\pgfmathtruncatemacro{\numrings}{20}

% X marks the origin...
% #1 tikz location
\newcommand{\centermark}[1]{
	\node [red, inner sep=0, font=\large]
	      at (#1)
	      {\contour{white}{$\times$}};
}

% 2D mesh with contour lines
% #1 origin x
% #2 origin y
\newcommand{\meshExample}[2]{
	\buildfile{python figures/distance_gradient.py {output} --2D --mesh --origin #1 #2}{.png}
	
	\begin{scope}[scale=\scale, anchor=center]
		% Image
		\pgfmathsetlengthmacro{\height}{\scale cm}
		\pgfmathsetlengthmacro{\width}{\scale cm}
		\node [inner sep=0, anchor=south west]
		      at (0,0)
		      {\includegraphics[width=\width,height=\height]{\filename}};
		
		% Contour Lines
		\begin{scope}[white]
			\clip (0,0) -- (1,0) -- (1,1) -- (0,1) -- cycle;
			\foreach \r in {0.5,1.0,...,20.0}{
				\node [ draw
				      , diamond
				      , inner sep=0
				      , minimum size=1.3*\r cm
				      ] at (#1, #2) {};
			}
		\end{scope}
		
		% Border
		\draw (0,0) -- (1,0) -- (1,1) -- (0,1) -- cycle;
		
		% Center
		\coordinate (center) at (#1, #2);
		\centermark{center}
	\end{scope}
}

% 2D torus with contour lines
% #1 origin x
% #2 origin y
\newcommand{\torusExample}[2]{
	\buildfile{python figures/distance_gradient.py {output} --2D --torus --origin #1 #2}{.png}
	
	
	\begin{scope}[scale=\scale, anchor=center]
		% Image
		\pgfmathsetlengthmacro{\height}{\scale cm}
		\pgfmathsetlengthmacro{\width}{\scale cm}
		\node [inner sep=0, anchor=south west]
		      at (0,0)
		      {\includegraphics[width=\width,height=\height]{\filename}};
		
		% Contour Lines
		\begin{scope}[white]
			\clip (0,0) -- (1,0) -- (1,1) -- (0,1) -- cycle;
			
			\foreach \dx in {-1, 0, 1}{
				\foreach \dy in {-1, 0, 1}{
					\foreach \r in {0.5,1.0,...,20.0}{
						\begin{scope}[shift={(\dx, \dy)}]
							\clip (#1, #2) ++(-0.5, -0.5) --
							      ++(1, 0) --
							      ++(0, 1) --
							      ++(-1, 0) --
							      cycle;
							\node [ draw
							      , diamond
							      , inner sep=0
							      , minimum size=1.3*\r cm
							      ] at (#1, #2) {};
						\end{scope}
					}
				}
			}
		\end{scope}
		
		% Border
		\draw (0,0) -- (1,0) -- (1,1) -- (0,1) -- cycle;
		
		% Center
		\coordinate (center) at (#1, #2);
		\centermark{center}
	\end{scope}
}

% Hexagonal mesh with contour lines
% #1 origin x
% #2 origin y
\newcommand{\hexMeshExample}[2]{
	\buildfile{python figures/distance_gradient.py {output} --hex --mesh --origin #1 #2}{.png}
	
	
	\begin{scope}[scale=\scale, anchor=center]
		% Image
		\pgfmathsetmacro{\xshift}{-cos(60)}
		\pgfmathsetlengthmacro{\height}{\scale cm * sin(60)}
		\pgfmathsetlengthmacro{\width}{\scale cm * (cos(60) + 1)}
		\begin{scope}
			\clip [hexagonXYZ] (0,0) -- (1,0) -- (1,1) -- (0,1) -- cycle;
			\node [inner sep=0] at ([hexagonXYZ]0.5, 0.5)
			      {\includegraphics[width=\width,height=\height]{\filename}};
		\end{scope}
		
		% Contour Lines
		\begin{scope}[white]
			\clip [hexagonXYZ] (0,0) -- (1,0) -- (1,1) -- (0,1) -- cycle;
			\foreach \r in {0.5,1.0,...,20.0}{
				\node [ draw
				      , hexagonXYZ
				      , hexagonBoard
				      , inner sep=0
				      , minimum size=0.9*\r cm
				      ] at (#1, #2) {};
			}
		\end{scope}
		
		% Border
		\draw [hexagonXYZ] (0,0) -- (1,0) -- (1,1) -- (0,1) -- cycle;
		
		% Center
		\coordinate (center) at ([hexagonXYZ]#1, #2);
		\centermark{center}
	\end{scope}
}

% Hexagonal torus with contour lines
% #1 origin x
% #2 origin y
\newcommand{\hexTorusExample}[2]{
	\buildfile{python figures/distance_gradient.py {output} --hex --torus --origin #1 #2}{.png}
	
	\pgfmathsetmacro{\boardtochip}{cos(30)*2/3}
	
	\begin{scope}[scale=\scale, anchor=center]
		% Image
		\pgfmathsetlengthmacro{\height}{\scale cm * sin(60)}
		\pgfmathsetlengthmacro{\width}{\scale cm * (cos(60) + 1)}
		\begin{scope}
			\clip [hexagonXYZ] (0,0) -- (1,0) -- (1,1) -- (0,1) -- cycle;
			\node [inner sep=0] at ([hexagonXYZ]0.5, 0.5)
			      {\includegraphics[width=\width,height=\height]{\filename}};
		\end{scope}
		
		% Contour Lines
		\begin{scope}[white]
			\clip [hexagonXYZ] (0,0) -- (1,0) -- (1,1) -- (0,1) -- cycle;
			\foreach \dx in {-1, 0, 1}{
				\foreach \dy in {-1, 0, 1}{
					\foreach \r in {0.5,1.0,...,20.0}{
						\begin{scope}[shift={([hexagonXYZ]#1+\dx, #2+\dy)}]
							\clip [hexagonBoardXYZ]
							      ++(0, \boardtochip, 0) --
							      ++(\boardtochip, 0, 0) --
							      ++(0, -\boardtochip, 0) --
							      ++(0, 0, \boardtochip) --
							      ++(-\boardtochip, 0, 0) --
							      ++(0, \boardtochip, 0) --
							      cycle;
							\node [ draw
							      , hexagonXYZ
							      , hexagonBoard
							      , inner sep=0
							      , minimum size=0.9*\r cm
							      ] {};
						\end{scope}
					}
				}
			}
		\end{scope}
		
		% Define coordinates of useful locations
		\coordinate (bottom left) at ([hexagonXYZ]0, 0, 0);
		\coordinate (bottom right) at ([hexagonXYZ]1, 0, 0);
		\coordinate (top right) at ([hexagonXYZ]1, 1, 0);
		\coordinate (top left) at ([hexagonXYZ]0, 1, 0);
		
		\coordinate (bottom) at ([hexagonXYZ]0.5, 0, 0);
		\coordinate (top) at ([hexagonXYZ]0.5, 1, 0);
		\coordinate (left) at ([hexagonXYZ]0, 0.5, 0);
		\coordinate (right) at ([hexagonXYZ]1, 0.5, 0);
		
		\coordinate (center) at ([hexagonXYZ]0.5, 0.5, 0);
		
		% Border
		\draw [hexagonXYZ] (0,0) -- (1,0) -- (1,1) -- (0,1) -- cycle;
		
		% Origin
		\coordinate (origin) at ([hexagonXYZ]#1, #2);
		\centermark{origin}
	\end{scope}
}
