\begin{tikzpicture}[thick, hexagonXYZ, scale=4.0, font=\footnotesize]
	
	\tikzstyle{router node}=[ draw
	                        , rectangle
	                        , minimum height=2.5cm
	                        , text width=2.5cm
	                        , fill=white
	                        ]
	
	\clip (-0.75, -0.5, 0) rectangle (3.75, 2.5, 0);
	
	% Wires between nodes
	\begin{scope}[line width=1em, white!80!black]
		\foreach \x in {-1,...,4}{
			\foreach \y in {-1,...,4}{
				\draw (\x, \y, 0) -- ++(1, 0, 0);
				\draw (\x, \y, 0) -- ++(0, 1, 0);
				\draw (\x, \y, 0) -- ++(0, 0, 1);
			}
		}
	\end{scope}
	
	% #1 Coordinates
	% #2 Text
	\newcommand{\router}[2]{
		\node (router) at (#1)
		      [router node] {
			#2
		};
		\node [anchor=north west] at (router.north west)
		      [help lines]
		      {$(#1)$};
	}
	
	% #1 link no
	\newcommand{\link}[1]{%
		\begin{tikzpicture}[baseline=-0.25em]
			% Pad to make all arrows the same height
			\draw [<->, white](60: -0.5em) -- (60: 0.5em);
			 % The actual arrow
			\draw [->] (#1 * 60: -0.5em) -- (#1 * 60: 0.5em);
		\end{tikzpicture}%
	}
	% #1 core no
	\newcommand{\core}[1]{%
		$C_{#1}$%
	}
	
	\router{0, 0, 0}{
		\texttt{1011}:~\{\link{0}\}\\
		\texttt{00XX}:~\{\link{1}\}\\
	}
	\router{1, 0, 0}{
		\texttt{1011}:~\{\link{1}\}\\
	}
	\router{2, 0, 0}{}
	
	\router{0, 1, 0}{}
	\router{0, 0, -1}{
		\texttt{00XX}:~\{\link{2}, \link{0}\}\\
	}
	\router{1, 0, -1}{
		\texttt{00XX}:~\{\core{11}, \link{1}\}\\
	}
	\router{2, 0, -1}{}
	
	\router{0, 1, -1}{
		\texttt{00XX}:~\{\core{5}\}\\
	}
	\router{0, 0, -2}{}
	\router{1, 0, -2}{
		\texttt{1011}:~\{\core{7}\}\\
		\texttt{00XX}:~\{\core{6}\}\\
	}
	
\end{tikzpicture}
