\pgfmathtruncatemacro{\width}{8}
\pgfmathtruncatemacro{\height}{8}
\pgfmathtruncatemacro{\widthh}{\width+2}
\pgfmathtruncatemacro{\heightt}{\height+2}

% Clipping
\newcommand{\drawtorus}{
	\clip (1.5, 1.5) --
	    ++(\width, 0) --
	    ++(0, \height) --
	    ++(-\width, 0) --
	      cycle;
	
	% Links
	\begin{scope}[help lines]
		\foreach \x in {1,...,\widthh}{
			\foreach [count=\y] \yy in {2,...,\heightt}{
				\draw (\x, \y) -- (\x, \yy);
			}
		}
		\foreach \y in {1,...,\heightt}{
			\foreach [count=\x] \xx in {2,...,\widthh}{
				\draw (\x, \y) -- (\xx, \y);
			}
		}
		\foreach [count=\x] \xx in {2,...,\widthh}{
			\foreach [count=\y] \yy in {2,...,\heightt}{
				\draw (\x, \y) -- (\xx, \yy);
			}
		}
	\end{scope}
	
	% Nodes
	\foreach \x in {1,...,\widthh}{
		\foreach \y in {1,...,\heightt}{
			\node (node\x\y)
			      [ fill
			      , circle
			      , minimum size=0.5em
			      , inner sep=0
			      ]
			      at (\x, \y)
			      {};
		}
	}
}

\newcommand{\drawnodes}{
	% Annotations
	\foreach \x/\y/\lab in {3/4/A,8/7/B}{
		\node [ draw=black
		      , fill=white
		      , font=\tiny
		      , circle
		      , inner sep=0.1em
		      ]
		      at (node\x\y)
		      {\lab};
	}
}

% #1 colour/style
% #2 \draw-like list of coordinates
\newcommand{\drawroute}[2]{
	\foreach \dx in {-1, 0, 1}{
		\foreach \dy in {-1, 0, 1}{
			\begin{scope}[shift={(\width*\dx, \height*\dy)}]
				\draw [line width=0.3em] [#1] #2;
			\end{scope}
		}
	}
}
% #1 colour/style
% #2 \draw-like list of coordinates
\newcommand{\overdrawroute}[2]{
	\drawroute{#1] [line width=0.15em, shift={(0, 0, -0.18em/1cm)}}{#2}
}
% #1 colour/style
% #2 \draw-like list of coordinates
\newcommand{\underdrawroute}[2]{
	\drawroute{#1] [line width=0.15em, shift={(0, -0.18em/1cm, 0)}}{#2}
}
