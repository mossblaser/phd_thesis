\chapter{Finding shortest path vectors in SpiNNaker's network}
	
	\label{sec:shortestPaths}
	
	% XXX: Add note explaining shortest path between two points in non-torus
	% space.
	
	In the previous chapter we explored the practical challenges of building
	machines which use hexagonal torus topologies working at the scale of units
	containing several nodes. To exploit these machines, however, we must also be
	able to route packets efficiently through the nodes in the resulting network.
	In this chapter the problem of finding shortest path vectors in hexagonal
	torus topologies is tackled. Shortest path vectors are used by many routing
	algorithms as the basis for route generation. In non-hexagonal torus
	topologies, finding shortest path vectors is trivial and intuitive. In
	hexagonal toruses, this is not the case.  In this chapter I introduce the
	\emph{Irregular Quadrant (IQ) method}, a new technique for computing shortest
	path vectors in hexagonal torus topologies.  The IQ method is cheaper to
	compute and more general than pre-existing techniques, functioning correctly
	on hexagonal torus topologies of any aspect ratio.
	
	In some hexagonal torus topologies, many shortest path vectors may exist
	between a given pair of points. I propose a technique for discovering all
	possible shortest path vectors. With access to these alternative shortest
	path vectors, routing algorithms may be able to produce routes which load a
	network more evenly.
	
	In this chapter, I assume an idealised hexagonal torus topologies without
	faults or other irregularities. The challenge handling such artefacts in real
	world systems will be tackled in chapter~\ref{sec:routing}.
	
	\section{Shortest path vectors}
		
		Many popular routing algorithms for torus topologies, including all
		published algorithms designed for SpiNNaker~\cite{davies12,navaridas14},
		function by computing \emph{shortest path vectors} between the endpoints of
		a route and generating paths from these. A shortest path vector between two
		nodes is a vector, $\mathbf{v} = (v_1, v_2, v_3, \ldots)$, whose magnitude,
		$\| \mathbf{v} \| = \lvert v_1 \rvert + \lvert v_2 \rvert + \lvert v_3
		\rvert + \cdots$, is minimal with respect to all possible vectors between
		those nodes.
		
		\begin{figure}
			\center
			\buildfig{figures/mesh-topology-coordinates.tex}
			\caption[Shortest path routes in a 2D mesh network.]%
			{An example 2D mesh network with example shortest-path routes
			from `A' to `B' and `B' to `C'.}
			\label{fig:mesh-topology-coordinates}
		\end{figure}
		
		In a non-hexagonal mesh topology, shortest path vectors are computed by
		taking the element-wise difference between the source and destination
		nodes' coordinates. For example, figure~\ref{fig:mesh-topology-coordinates}
		illustrates a 2D mesh topology with three nodes labelled `A', `B' and `C'
		with position vectors $(1, 2)$, $(4, 5)$ and $(6, 1)$ respectively. The
		shortest path vector from node `A' to `B' is $(4, 5) - (1, 2) = (3, 3)$ and
		from `B' to `C' is $(6, 1) - (4, 5) = (2, -4)$. A route may be produced
		from a shortest path vector by advancing the number of hops specified for
		each dimension in the vector.  For example a route from `A' to `B' may be
		constructed from any permutation of the hops
		X$^+\,$X$^+\,$X$^+\,$Y$^+\,$Y$^+\,$Y$^+$, an example of which is included
		in the figure. Likewise a route from `B' to `C' may be constructed from any
		permutation of the hops X$^+\,$X$^+\,$Y$^-\,$Y$^-\,$Y$^-\,$Y$^-$.
		Regardless of the permutation of hops, the length of the route produced
		from a shortest path vector, $\mathbf{v}$, is given by its magnitude,
		$\|\mathbf{v}\|$.
		
		Many popular routing algorithms such as Dimension Order Routing, Right-Turn
		Only Routing and Longest Dimension First Routing~\cite{davies12} are simply
		defined as a rule for ordering the hops specified by a shortest path
		vector.
		
		\subsection{Torus Networks}
			
			\begin{figure}
				\center
				\begin{subfigure}{0.3\linewidth}
					\center
					\buildfig{figures/torus-shortest-path-example.tex}
					\caption{Original}
					\label{fig:torus-shortest-path-example}
				\end{subfigure}
				\begin{subfigure}{0.3\linewidth}
					\center
					\buildfig{figures/torus-shortest-path-translate.tex}
					\caption{Routed \& translated}
					\label{fig:torus-shortest-path-translate}
				\end{subfigure}
				\begin{subfigure}{0.3\linewidth}
					\center
					\buildfig{figures/torus-shortest-path-routed.tex}
					\caption{Routed original}
					\label{fig:torus-shortest-path-routed}
				\end{subfigure}
				
				\caption{Finding shortest paths in a 2D torus topology.}
				\label{fig:torus-shortest-path}
			\end{figure}
			
			Computing shortest path vectors in non-hexagonal torus topologies is also
			straight forward. For example, to find the shortest path vector from node
			`A' to `B' in the 2D torus topology shown in figure~\ref{fig:torus-shortest-path-example} both nodes are translated such that
			the source node, `A', is at the centre of the network. The shortest path
			vector is then computed in the same way as a mesh network (figure~\ref{fig:torus-shortest-path-translate}). Note that, as in this example,
			translation may cause the destination node to `wrap around' the network.
			As illustrated in figure~\ref{fig:torus-shortest-path-routed}, the
			computed shortest path vector is also valid for the two points prior to
			translation.
			
			\begin{figure}
				\center
				
				\begin{subfigure}{\linewidth}
					\center
					\buildfig{figures/distance-map-mesh.tex}
					\caption{2D mesh topology}
					\label{fig:distance-map-mesh}
				\end{subfigure}
				
				\vspace{1em}
				
				\begin{subfigure}{\linewidth}
					\center
					\buildfig{figures/distance-map-torus.tex}
					\caption{2D torus topology}
					\label{fig:distance-map-torus}
				\end{subfigure}
				
				\caption[Magnitudes of shortest path vectors in a 2D mesh.]%
				{Plots showing the magnitude of shortest path vectors in a 2D
				(non-hexagonal) topology from various locations marked
				{\color{red}$\times$}.  Darker areas are further away. Contour lines show
				equidistant points.}
				
				\label{fig:distance-map}
			\end{figure}
			
			This procedure works because vectors from the centre of a non-hexagonal
			torus topology to any other point are identical to those in a
			corresponding mesh topology. For example, in figures
			\ref{fig:distance-map-mesh} and~\ref{fig:distance-map-torus} we can see
			that the magnitude of the shortest path vectors from the centre of a mesh
			and torus grow identically. Conversely, the magnitudes of vectors from
			other locations in mesh and torus topologies do not match.
		
	\section{Related work}
		
		The problem of finding shortest path vectors in hexagonal mesh topologies
		has been widely considered and formulations may be found in a variety of
		applications, including computer games~\cite{patel15}. Hexagonal
		toruses, by contrast, have only received limited attention. In this section I
		briefly summarise the solutions proposed for hexagonal mesh topologies
		before more deeply examining existing solutions for torus topologies.
		
		\subsection{Hexagonal Mesh Networks}
			
			\begin{figure}
				\center
				\buildfig{figures/hex-mesh-topology-coordinates.tex}
				\caption{An example hexagonal mesh network topology.}
				\label{fig:hex-mesh-topology-coordinates}
			\end{figure}
			
			In hexagonal mesh topologies it is conventional to define three `axes' X,
			Y and Z as shown in
			figure~\ref{fig:hex-mesh-topology-coordinates}~\cite{patel15}. In this
			example, the three labelled nodes `A', `B' and `C' may be given position
			vectors such as $(1, 1, 0)$, $(3, 2, 0)$ and $(0, 0, -7)$ respectively.
			As in other mesh networks, a vector between two nodes is found by
			subtracting the nodes' vectors. For example, a vector from `A' to `B' is
			$(3, 2, 0) - (1, 1, 0) = (2, 1, 0)$. This vector can then be converted
			into a route in the same way as a mesh network by taking any permutation
			of the three hops  X$^+\,$X$^+\,$Y$^+$.
			
			As explained in detail in appendix~\ref{app:minimal-hex-coordinates},
			there are a multitude of vectors between any two points in a hexagonal
			mesh. For example, the vectors $(1, 0, -1)$ and $(3, 2, 1)$ also reach
			node `B' from `A'. However, for a given pair of nodes, there is always a
			single, unique vector whose magnitude is minimal which is given by the
			function:
			%
			\begin{equation*}
				\operatorname{minimiseVector}(x,y,z) =
					(x,y,z) - \operatorname{median}(x,y,z) \cdot (1,1,1)
			\end{equation*}
			%
			For example, given the vector $(3, 2, 1)$ from `A' to `B' would be
			minimised as follows:
			%
			\begin{align*}
				\operatorname{minimiseVector}(3,2,1) &=
					(3,2,1) - \operatorname{median}(3,2,1) \cdot (1,1,1) \\
				&=
					(3,2,1) - (2,2,2) \\
				&=
					(1,0,-1)
			\end{align*}
			%
			An important side-effect of this function is that a minimised vector will
			always contain at least one zero element meaning that shortest path
			routes will use at most two of the three available dimensions.
		
		\subsection{Hexagonal Torus Networks}
			
			\begin{figure}
				\center
				
				\begin{subfigure}{\linewidth}
					\center
					\buildfig{figures/distance-map-hex-mesh.tex}
					\caption{Hexagonal mesh topology}
					\label{fig:distance-map-hex-mesh}
				\end{subfigure}
				
				\vspace{1em}
				
				\begin{subfigure}{\linewidth}
					\center
					\buildfig{figures/distance-map-hex-torus.tex}
					\caption{Hexagonal torus topology}
					\label{fig:distance-map-hex-torus}
				\end{subfigure}
				
				\caption[Magnitudes of shortest path vectors in a hexagonal torus.]%
				{Plots showing the magnitude of shortest path vectors in a
				hexagonal torus topology from various locations marked
				{\color{red}$\times$}.  Darker areas are further away. Contour lines show
				equidistant points.}
				
				\label{fig:distance-map-hex}
			\end{figure}
			
			Unfortunately, the translation technique used for non-hexagonal toruses
			cannot be used in a hexagonal torus. As illustrated in figures
			\ref{fig:distance-map-hex-mesh} and \ref{fig:distance-map-hex-torus},
			shortest path vectors from the centre, or any other part of a hexagonal
			mesh network, do not grow in magnitude in the same way that those of a
			hexagonal torus network do.
			
			I am aware of two pre-existing approaches to computing shortest path
			vectors in hexagonal toruses in the literature. These are described
			below.
			
			\subsubsection{INSEE Method}
			
				The INSEE interconnect simulator has been used in all published
				research into SpiNNaker's hexagonal torus interconnect to
				date~\cite{navaridas09,ghasempour15}. Internally INSEE finds shortest
				path vectors by selecting the shortest of a set of twelve candidate
				vectors known to always contain a shortest path vector.
				
				\begin{figure}
					\center
					\begin{subfigure}{0.45\linewidth}
						\center
						\buildfig{figures/insee-vector-candidates-no-wrap.tex}
						\caption{$(\Delta_\textrm{X}, \Delta_\textrm{Y}) = (5,3)$}
						\label{fig:insee-vector-candidates-no-wrap}
					\end{subfigure}
					\begin{subfigure}{0.45\linewidth}
						\center
						\buildfig{figures/insee-vector-candidates-wrap-x.tex}
						\caption{$(\Delta'_\textrm{X}, \Delta_\textrm{Y}) = (-3,3)$}
						\label{fig:insee-vector-candidates-wrap-x}
					\end{subfigure}
					
					\vspace{1em}
					
					\begin{subfigure}{0.45\linewidth}
						\center
						\buildfig{figures/insee-vector-candidates-wrap-y.tex}
						\caption{$(\Delta_\textrm{X}, \Delta'_\textrm{Y}) = (5,-5)$}
						\label{fig:insee-vector-candidates-wrap-y}
					\end{subfigure}
					\begin{subfigure}{0.45\linewidth}
						\center
						\buildfig{figures/insee-vector-candidates-wrap.tex}
						\caption{$(\Delta'_\textrm{X}, \Delta'_\textrm{Y}) = (-3,-5)$}
						\label{fig:insee-vector-candidates-wrap}
					\end{subfigure}
					
					\vspace{1em}
					
					% Key
					\begin{tikzpicture}[thick]
						\coordinate (last);
						
						% #1 colour
						% #2 label
						\newcommand{\colourkeyentry}[2]{
							\node [#1] [right=of last, fill, rectangle, minimum size=1em] (last) {};
							\node [right=0 of last] (last) {#2};
						}
						
						\colourkeyentry{cb3class0}{$(\textrm{X}, \textrm{Y}, 0)$}
						\colourkeyentry{cb3class1}{$(\textrm{X} - \textrm{Y}, 0, - \textrm{Y})$}
						\colourkeyentry{cb3class2}{$(0, \textrm{Y} - \textrm{X}, - \textrm{X})$}
						
					\end{tikzpicture}
					
					\caption[The twelve candidate vectors considered by the INSEE method.]%
					{The twelve candidate shortest-path vectors considered by the INSEE
					method represented as dimension-order routes. $W=H=8$,
					$(\Delta_\textrm{X},\Delta_\textrm{Y}) = (5, 3)$ and
					$(\Delta'_\textrm{X},\Delta'_\textrm{Y}) = (-3, -5)$.}
					\label{fig:insee-vector-candidates}
				\end{figure}
				
				The twelve vectors considered are illustrated in
				figure~\ref{fig:insee-vector-candidates} and are constructed as
				follows.  First a shortest path vector from the source to target node
				is constructed as if the network was a 2D mesh producing a vector
				$(\Delta_\textrm{X},\Delta_\textrm{Y})$. A second 2D vector,
				$(\Delta'_\textrm{X},\Delta'_\textrm{Y})$, is also defined:
				%
				\begin{align*}
					\Delta'_\textrm{X} &= \Delta_\textrm{X} - \operatorname{sign}(\Delta_\textrm{X})W
					\\
					\Delta'_\textrm{Y} &= \Delta_\textrm{Y} - \operatorname{sign}(\Delta_\textrm{Y})H
				\end{align*}
				%
				Where $W$ and $H$ are the width and height of the network respectively
				(in nodes). This vector describes a route from the source to
				destination node that, in a torus topology, \emph{always} wraps around
				the peripheries of both the `X' and `Y' dimensions.
				
				Two further 2D vectors, $(\Delta'_\textrm{X},\Delta_\textrm{Y})$ and
				$(\Delta_\textrm{X},\Delta'_\textrm{Y})$ may be defined which wrap around
				just the X and Y axes only, respectively.
				
				Each of the four 2D vectors may be converted into a three hexagonal 3D
				vectors in which one element of the vector is zero. In total this
				results in twelve different vectors which cover all combinations of
				wrapping and non-wrapping routes and all combinations of axes used. The
				vector with the smallest magnitude must be the shortest path vector.
				
				This method can find shortest path vectors in hexagonal torus
				topologies of any aspect ratio but, compared with the XYZ-Protocol
				(described next), is relatively clumsy and slow to compute.
			
			\subsubsection{XYZ-Protocol}
				
				\begin{figure}
					\center
					\buildfig{figures/xyz-protocol-regions.tex}
					
					\caption{The four regions defined by the XYZ-protocol.}
					\label{fig:xyz-protocol-regions}
				\end{figure}
			
				Hoffmann and D\'es\'erable describe a technique for computing shortest
				path vectors in hexagonal toruses with equal width and height called
				the XYZ-Protocol~\cite{hoffmann15,hoffmann11}. First, the source and
				destination nodes are translated such that the source node lies at the
				centre of the topology. The authors observe that from the centre of the
				topology the pattern with which distances grow differs between the four
				quadrants outlined in figure~\ref{fig:xyz-protocol-regions}.
				
				If the destination lies in quadrants 1 or 4, a route may be constructed
				as if in a hexagonal mesh topology. If the destination lies in regions
				2 or 3, however, the algorithm tests whether taking a mesh-like vector
				within the region or wrapping-around either the X or Y dimensions
				yields the shortest vector.
				
				By comparison with the INSEE method, the number of vectors considered
				is much smaller, just one for destinations in quadrant 1 or 4 and in
				quadrant 2 or 3, no more than three. Unfortunately, though the
				XYZ-protocol can be computed more cheaply than the INSEE method, it
				does not produce valid shortest path vectors for hexagonal torus
				topologies with aspect ratios other than $1:1$.
		
	\section{The Irregular Quadrant (IQ) Method}
		
		In this section I propose a novel technique for finding shortest path
		vectors in hexagonal torus topologies called the Irregular Quadrant (IQ)
		method and compare its performance with existing techniques.
		
		\subsection{Computing shortest path vectors}
		
			Let us consider the problem of finding a shortest path vector from the node
			at (0, 0, 0), at the bottom-left, to another node somewhere else in a
			hexagonal torus topology.
			
			\begin{figure}
				\center
				\buildfig{figures/shortest-path-regions.tex}
				
				\caption[The four regions defined by the IQ method.]%
				{Hexagonal torus topologies of various aspect ratios divided
				into regions in which a particular pair of dimensions is used.}
				\label{fig:shortest-path-regions}
			\end{figure}
			
			Figure~\ref{fig:shortest-path-regions} illustrates how hexagonal torus
			topologies of various aspect ratios may be partitioned into four
			\emph{irregular quadrants}. These quadrants are defined according to which
			axes are wrapped around by the shortest path vectors reaching them. The
			irregular quadrants correspond to locations reachable by shortest path
			vectors which are:
			%
			\begin{enumerate}
				\item Non-wrapping
				\item Wrap around X only
				\item Wrap around Y only
				\item Wrap around both X and Y
			\end{enumerate}
			%
			Within each irregular quadrant, we observe that shortest path vectors are
			constrained to using only certain dimensions:
			%
			\begin{enumerate}
				\item Only X$^+$, Y$^+$ and Z$^-$.
				\item Only X$^-$ and Y$^+$
				\item Only X$^+$ and Y$^-$
				\item Only X$^-$, Y$^-$ and Z$^+$.
			\end{enumerate}
			%
			Given the topology is of width and height $W$ and $H$ respectively and the
			destination node's 2D mesh coordinates are $(\Delta_\textrm{X},
			\Delta_\textrm{Y})$ we can define the shortest path vector within each
			irregular quadrant as:
			%
			\begin{enumerate}
				\item $\operatorname{minimiseVector}(\Delta_\textrm{X},\Delta_\textrm{Y},0)$
				\item $\operatorname{minimiseVector}(-(W-\Delta_\textrm{X}),\Delta_\textrm{Y},0)$
				\item $\operatorname{minimiseVector}(\Delta_\textrm{X},-(H-\Delta_\textrm{Y}),0)$
				\item $\operatorname{minimiseVector}(-(W-\Delta_\textrm{X}),-(H-\Delta_\textrm{Y}),0)$
			\end{enumerate}
			%
			Since we know that $0 \le \Delta_\textrm{X} < W$ and $0 \le
			\Delta_\textrm{Y} < H$, we can simplify expressions for the magnitude of
			each of the above vectors. This yields four expressions giving the
			magnitude of a minimal vector which wraps around each combination of the
			X and Y axes:
			%
			\begin{enumerate}
				\item $\operatorname{max}(\Delta_\textrm{X}, \Delta_\textrm{Y})$
				\item $(W - \Delta_\textrm{X}) + \Delta_\textrm{Y}$
				\item $\Delta_\textrm{X} + (H - \Delta_\textrm{Y})$
				\item $\operatorname{max}(W-\Delta_\textrm{X}, H-\Delta_\textrm{Y})$
			\end{enumerate}
			%
			From these expressions we can determine which combination of axes the
			shortest path vector wraps around by find the expression with the minimum
			value. With this information, the shortest path can be found by
			minimising the vector associated with that irregular quadrant.  Figure~\ref{fig:iqmethod.py} shows a simple Python implementation of the IQ
			method.
			
			\begin{figure}
				\inputminted{python}{figures/iqmethod.py}
				
				\caption{A Python implementation of the IQ Method.}
				\label{fig:iqmethod.py}
			\end{figure}
			
			Compared with the INSEE method, this technique requires fewer cases to be
			considered (four rather than twelve) making it cheaper to compute.
			
			Unlike the four quadrants defined by the XYZ-protocol, the irregular
			quadrants defined by the IQ method correspond exactly to a particular
			shortest path vector. This means that, once the irregular quadrant a
			point lies in has been discovered, the shortest path vector can be
			calculated directly without considering multiple options. Though the
			boundaries between the four irregular quadrants are more complex than the
			quadrants of the XYZ-protocol, it is only marginally more expensive to
			discover which irregular quadrant a point lies in. In addition, the
			irregular quadrants retain their meaning across different aspect ratios
			making the IQ method suitable for any hexagonal torus topology.
		
		\subsection{Computing distances}
		
			The four vector magnitude expressions defined by the IQ method can also
			be combined to produce a compact expression of the distance between two
			points in a hexagonal torus topology:
			%
			\begin{align*}
				\operatorname{shortestPathVectorMagnitude}(\Delta_\textrm{X}, \Delta_\textrm{Y}, W, H) =
				\operatorname{min}(&\operatorname{max}(\Delta_\textrm{X}, \Delta_\textrm{Y}),\\
				                   &(W - \Delta_\textrm{X}) + \Delta_\textrm{Y},\\
				                   &\Delta_\textrm{X} + (H - \Delta_\textrm{Y}),\\
				                   &\operatorname{max}(W-\Delta_\textrm{X}, H-\Delta_\textrm{Y}))
			\end{align*}
			%
			This expression has also been derived independently from a
			graph-theoretic study of Cayley graphs, of which the hexagonal torus
			topology is a special case, by Xiao and Parhami~\cite{xiao04}.
		
		\subsection{Computational efficiency}
			
			Since computing shortest path vectors forms an integral part of the
			kernel of many routing algorithms, the running time of the function
			chosen used can be an important consideration.
			
			To compare the performance of the three shortest path vector functions
			considered, the runtime of a C implementation of each technique was
			measured. The C implementation of the INSEE method is taken directly from
			the INSEE source code~\cite{navaridas09}. The C implementation of the
			XYZ-Protocol is a straight translation of the published
			pseudocode~\cite{hoffmann15}.
			
			Each function was called approximately 3 billion times: once for every
			pair of source and destination nodes in a $240\times240$ node hexagonal
			torus topology. The total execution time is then divided by the number of
			calls to the shortest path vector function giving an average runtime.
			This experiment was repeated 50 times and the overall average runtime of
			each shortest path vector function was recorded. The experiments were
			conducted on a cluster of idle workstations with 3.10~GHz Intel
			Core-i5-2400 CPUs. The function implementations were compiled with GCC
			5.3.0 with optimisations enabled (\verb|-O2|).
			
			\begin{figure}
				\center
				\buildrplot{figures/shortest-path-vector-runtimes.R}
				
				\caption[Mean runtime of each shortest path vector function.]%
				{Mean runtime of each shortest path vector function. Error bars
				indicate the range of the mean runtimes of the fifty
				3-billion-combination experiments.}
				\label{fig:shortest-path-vector-runtimes}
			\end{figure}
			
			Figure~\ref{fig:shortest-path-vector-runtimes} shows the runtimes of each
			shortest path vector function. From these results we can see that the
			INSEE method is slower than the other techniques. Although the
			XYZ-protocol and IQ method have similar performance, since the IQ method
			works for any hexagonal torus topology it is the better candidate for use
			in new applications.
	
	\section{Generating all shortest path vectors}
			
			\begin{figure}
				\center
				\buildfig{figures/wrap-alternatives.tex}
				
				\caption{Two distinct shortest path vectors in a hexagonal torus.}
				\label{fig:wrap-alternatives}
			\end{figure}
			
			In odd-sized non-hexagonal and $1:1$ hexagonal torus topologies there is
			exactly one distinct shortest path vector between any two points (though
			many routes may be defined from it). In even-sized topologies of these
			types there may be two distinct shortest path vectors between two nodes
			exactly half the length of an axis away as in
			figure~\ref{fig:wrap-alternatives}. The INSEE and IQ methods may generate
			either vector depending on how ties are broken in their implementation.
			
			\begin{figure}
				\center
				\buildfig{figures/spiralling.tex}
				
				\caption[Distinct shortest path vectors in non-square topologies.]%
				{Distinct shortest path vectors between two points, all with
				magnitude 11.}
				\label{fig:spiralling}
			\end{figure}
			
			Unlike non-hexagonal topologies, when the aspect ratio of a hexagonal
			topology is not $1:1$, some points may have many distinct (but equal
			magnitude) shortest path vectors.  For example, figure~\ref{fig:spiralling} illustrates the three distinct shortest path vectors
			between a single pair of nodes. None of the shortest path vector
			functions discussed will generate all possible vectors in this situation,
			potentially limiting the choices available to routing algorithms.  To
			address this shortcoming, I propose a formula which enumerates every
			shortest path vector between a pair of points in a hexagonal torus.
			
			In a $W \times H$ hexagonal torus topology, the vector $(0, H, 0)$
			straight-forwardly wraps once around the Y axis arriving back where it
			started. The vector $(1,1,1)$ also arrives back at its starting point as
			described in appendix \ref{app:minimal-hex-coordinates}. Consequently
			$(0,H,0) - H\cdot(1,1,1) = (-H, 0, -H)$ must also be a vector resulting
			in no movement.  Adding this vector to shortest path vectors of the form
			$(x, 0, z)$ where $x\ge H$ and $z\le0$ results in a new, distinct,
			shortest path vector.
			
			For example the vector $(10, 0, -1)$ (magnitude 11) in
			figure~\ref{fig:spiralling} can be added with $(-4, 0, -4)$ yielding $(6,
			0, -5)$ (also magnitude 11) which is still a shortest path vector between
			the two labelled nodes.  Since $(6, 0, -5)$ still meets the criteria
			defined above ($6 \ge 4$ and $-5 \le 0$), we can add $(-4, 0, -4)$ again
			yielding another shortest path vector $(2, 0, -9)$.  This new vector does
			not meet the requirement that $x \ge H$ (since $2 \ngeq 4$) and so no
			further shortest path vectors can be produced.
			
			More formally, a shortest path vector $(x, 0, z)$ may be converted into
			another shortest path vector $(x', 0, z')$ as defined by the following
			formula:
			%
			\begin{equation*}
				(x', 0, z') = (x, 0, z) - \left(\operatorname{trunc}\left(\frac{z}{H}\right) + n\right)(H, 0, H)
			\end{equation*}
			%
			where
			%
			\begin{equation*}
				\left\{
					n \in \mathbb{Z}
				\;\Big|\;
					0 \le n \le
						\left\lfloor
							\frac{\left|x\right| + \left|z\right|}{H}
						\right\rfloor
				\right\}
			\end{equation*}
			%
			and $\operatorname{trunc}(\cdots)$ is the truncation operator which
			rounds towards zero to the nearest integer.
			
			A complementary formula may be derived based on the related observation
			that the vector $(W, 0, 0)$ results in no movement:
			%
			\begin{equation*}
				(0, y', z') = (0, y, z) - \left(\operatorname{trunc}\left(\frac{z}{W}\right) + n\right)(0, W, W)
			\end{equation*}
			%
			where
			%
			\begin{equation*}
				\left\{
					n \in \mathbb{Z}
				\;\Big|\;
					0 \le n \le
						\left\lfloor
							\frac{\left|y\right| + \left|z\right|}{W}
						\right\rfloor
				\right\}
			\end{equation*}
			
			Using these two expressions for wide and tall topologies, respectively,
			all possible distinct shortest path vectors between any two points may be
			found.
	
	\section{Conclusions}
		
		The calculation of shortest path vectors in mesh and torus topologies is at
		the heart of many routing algorithms. In this chapter I introduced a new
		technique for computing shortest path vectors in hexagonal toruses which
		generalises to any hexagonal torus topology while executing more quickly
		than existing approaches. In addition I described a formula for enumerating
		the distinct shortest path vectors between points in hexagonal torus
		topologies. Unlike previous work, this allows routing algorithms greater
		freedom by providing a choice of shortest path vectors between some points.
		These contributions demonstrate that key operations such as computing
		distances and vectors in hexagonal toruses can be efficient and exhaustive
		just as in non-hexagonal torus topologies.
