\chapter{Finding shortest path vectors in SpiNNaker's network}
	
	Once a SpiNNaker machine has been constructed as described in the previous
	chapter, its network forms a large hexagonal torus topology. To exploit this
	network routing algorithms must be used to generate routes for packets to
	follow between nodes. As well as ensuring that packets arrive at the correct
	destination, routing algorithms often attempt to produce routes which make
	efficient use of the network. This often involves attempting to reduce
	congestion by ensuring packets do not travel further through the network than
	absolutely necessary.
	
	Many popular routing algorithms for torus topologies, including all published
	algorithms designed for SpiNNaker's hexagonal torus topology
	\cite{davies12,navaridas14}, internally function by computing shortest path
	vectors and generating routes from them. Existing methods of calculating
	shortest path vectors in hexagonal torus topologies are unable to generate
	all possible shortest path vectors and, as a result, reduces the diversity of
	routes produced by routing algorithms, potentially worsening network
	contention.
	
	In this chapter I describe a novel technique for computing shortest path
	vectors in hexagonal torus topologies which yields \emph{all} possible
	shortest path vectors for any pair of nodes. Further, implementations of this
	new technique execute an order of magnitude faster than the existing
	alternatives.
	
	\section{Related work}
		
		TODO: INTRODUCE SECTION
		
		\begin{figure}
			\center
			
			\begin{subfigure}{\linewidth}
				\center
				\buildfig{figures/distance-map-mesh.tex}
				\caption{2D mesh topology}
				\label{fig:distance-map-mesh}
			\end{subfigure}
			
			\vspace{1em}
			
			\begin{subfigure}{\linewidth}
				\center
				\buildfig{figures/distance-map-torus.tex}
				\caption{2D torus topology}
				\label{fig:distance-map-torus}
			\end{subfigure}
			
			\vspace{1em}
			
			\begin{subfigure}{\linewidth}
				\center
				\buildfig{figures/distance-map-hex-mesh.tex}
				\caption{Hexagonal mesh topology}
				\label{fig:distance-map-hex-mesh}
			\end{subfigure}
			
			\vspace{1em}
			
			\begin{subfigure}{\linewidth}
				\center
				\buildfig{figures/distance-map-hex-torus.tex}
				\caption{Hexagonal torus topology}
				\label{fig:distance-map-hex-torus}
			\end{subfigure}
			
			\caption{Plots showing distance from various locations marked
			         {\color{red}$\times$}. Darker areas are further away. Contour
			         lines show equidistant points.}
			\label{fig:distance-map}
		\end{figure}
		
		\subsection{Mesh Networks}
			
			In a (non-hexagonal) mesh network topology, shortest path vectors are
			computed by taking the element-wise difference between the source and
			destination nodes' coordinates.
			
			\begin{figure}
				\center
				\buildfig{figures/mesh-topology-coordinates.tex}
				\caption{An example 2D mesh network with example shortest-path routes
				from `A' to `B' and `B' to `C'.}
				\label{fig:mesh-topology-coordinates}
			\end{figure}
			
			For example, figure \ref{fig:mesh-topology-coordinates} illustrates a 2D
			mesh topology. In this topology, the nodes labelled `A', `B' and `C' have
			position vectors $(1, 2)$, $(4, 5)$ and $(6, 1)$ respectively. The
			shortest path vector from node `A' to `B' is thus simply $(4, 5) - (1, 2)
			= (3, 3)$ and from `B' to `C' is $(6, 1) - (4, 5) = (2, -4)$.
			
			A route may be produced from a shortest path vector by advancing the
			number of hops specified for each dimension in the vector. For example
			any permutation of the hops X$^+\,$X$^+\,$X$^+\,$Y$^+\,$Y$^+\,$Y$^+$, an
			example of which is included in the figure. Likewise a route from `B' to
			`C' may be constructed from any permutation of
			X$^+\,$X$^+\,$Y$^-\,$Y$^-\,$Y$^-\,$Y$^-$.
			
			Many popular routing algorithms such as Dimension Order Routing (DOR),
			Right-Turn Only Routing (RTOR) and Longest Dimension First Routing (LDFR)
			\cite{dally04,davies12} directly follow the above procedure and just
			prescribe a specific permutation of hop order. For example, DOR produces
			routes with X hops first, Y hops second and so on.
			
			The length of routes produced from a shortest path vector have a number
			of hops proportional to the magnitude of the vector, thus a shortest path
			vector yields a route with the minimum number of hops. For a two
			dimensional vector $(a, b)$ the magnitude is given as:
			%
			\begin{equation}
				\| (a, b) \| = \lvert a \rvert + \lvert b \rvert
			\end{equation}
		
		\subsection{Torus Networks}
			
			Computing shortest path vectors in (non-hexagonal) torus topologies is
			also straight forward. As an example, lets find the shortest path vector
			from node `A' to `B' in the 2D torus topology shown in figure
			\ref{fig:torus-shortest-path-example}. First, both nodes are translated
			such that the source node, `A', is at the centre of the network (figure
			\ref{fig:torus-shortest-path-translate}). Note that this translation may
			result in the destination node `wrapping around' the network. After
			translation, the shortest path vector is computed as in a mesh topology.
			As illustrated in \ref{fig:torus-shortest-path-routed}, the computed
			shortest path vector may be used to produce routes between the two nodes
			in their original positions.
			
			\begin{figure}
				\center
				\begin{subfigure}{0.3\linewidth}
					\center
					\buildfig{figures/torus-shortest-path-example.tex}
					\caption{Original}
					\label{fig:torus-shortest-path-example}
				\end{subfigure}
				\begin{subfigure}{0.3\linewidth}
					\center
					\buildfig{figures/torus-shortest-path-translate.tex}
					\caption{Translated}
					\label{fig:torus-shortest-path-translate}
				\end{subfigure}
				\begin{subfigure}{0.3\linewidth}
					\center
					\buildfig{figures/torus-shortest-path-routed.tex}
					\caption{Routed}
					\label{fig:torus-shortest-path-routed}
				\end{subfigure}
				
				\caption{Finding shortest paths in a 2D torus topology.}
				\label{fig:torus-shortest-path}
			\end{figure}
			
			This process works because vectors from the centre (though not other
			locations) of a torus topology are identical to those in mesh topologies
			(see figures \ref{fig:distance-map-mesh} and
			\ref{fig:distance-map-torus}).
		
		\subsection{Hexagonal Mesh Networks}
			
			In hexagonal mesh topologies it is conventional to define three `axes' X,
			Y and Z as shown in figure \ref{fig:hex-mesh-topology-coordinates}
			\cite{patel15}. In this example, the three labelled nodes `A', `B' and
			`C' may be given position vectors such as $(1, 1, 0)$, $(3, 2, 0)$ and
			$(0, 0, -7)$ respectively. As in other mesh networks, a vector between
			two nodes is found by subtracting the nodes' vectors. For example, a
			vector from `A' to `B' is $(3, 2, 0) - (1, 1, 0) = (2, 1, 0)$. This
			vector can then be converted into a route in the same way as a mesh
			network by taking any permutation of the three hops  X$^+\,$X$^+\,$Y$^+$.
			
			\begin{figure}
				\center
				\buildfig{figures/hex-mesh-topology-coordinates.tex}
				\caption{An example hexagonal mesh network topology.}
				\label{fig:hex-mesh-topology-coordinates}
			\end{figure}
			
			As explained in detail in appendix \ref{app:minimal-hex-coordinates},
			there are an infinite number of vectors between any two points. For
			example, the vectors $(1, 0, -1)$ and $(3, 2, 1)$ also reach node `B'
			from `A' in the example. However, for a given pair of nodes, there is
			always a single, unique vector whose magnitude is minimal which is
			given by the function:
			%
			\begin{equation}
				\operatorname{minimiseVector}(x,y,z)
					= (x,y,z) - \operatorname{median}(x,y,z) \cdot (1,1,1)
			\end{equation}
			%
			An important side-effect of this function is that a minimised vector will
			always contain at least one zero element meaning that shortest path
			routes will use at most two of the three available dimensions.
			
			To aid the reader's intuition, figure \ref{fig:distance-map-hex-mesh}
			illustrates how distances grow in a hexagonal mesh topology.
		
		\subsection{Hexagonal Torus Networks}
			
			Unfortunately, unlike non-hexagonal torus topologies, the translation
			technique cannot be used to compute shortest path vectors. As illustrated
			in figures \ref{fig:distance-map-hex-mesh} and
			\ref{fig:distance-map-hex-torus}, shortest path vectors from the center
			of a hexagonal mesh network are not equivalent to those of a hexagonal
			torus network.
			
			Prior research into routing in SpiNNaker's network has been based on the
			INSEE \cite{navaridas09,ghasempour15} interconnect simulator. Internally
			INSEE tries a set of twelve candidate vectors and picks the shortest as
			the shortest path vector to use for routing.
			
			\begin{figure}
				\center
				\begin{subfigure}{0.45\linewidth}
					\center
					\buildfig{figures/insee-vector-candidates-no-wrap.tex}
					\caption{$(\Delta_\textrm{X}, \Delta_\textrm{Y}) = (5,3)$}
					\label{fig:insee-vector-candidates-no-wrap}
				\end{subfigure}
				\begin{subfigure}{0.45\linewidth}
					\center
					\buildfig{figures/insee-vector-candidates-wrap-x.tex}
					\caption{$(\Delta'_\textrm{X}, \Delta_\textrm{Y}) = (-3,3)$}
					\label{fig:insee-vector-candidates-wrap-x}
				\end{subfigure}
				
				\vspace{1em}
				
				\begin{subfigure}{0.45\linewidth}
					\center
					\buildfig{figures/insee-vector-candidates-wrap-y.tex}
					\caption{$(\Delta_\textrm{X}, \Delta'_\textrm{Y}) = (5,-5)$}
					\label{fig:insee-vector-candidates-wrap-y}
				\end{subfigure}
				\begin{subfigure}{0.45\linewidth}
					\center
					\buildfig{figures/insee-vector-candidates-wrap.tex}
					\caption{$(\Delta'_\textrm{X}, \Delta'_\textrm{Y}) = (-3,-5)$}
					\label{fig:insee-vector-candidates-wrap}
				\end{subfigure}
				
				\vspace{1em}
				
				% Key
				\begin{tikzpicture}[thick]
					\coordinate (last);
					
					% #1 colour
					% #2 label
					\newcommand{\colourkeyentry}[2]{
						\node [#1] [right=of last, fill, rectangle, minimum size=1em] (last) {};
						\node [right=0 of last] (last) {#2};
					}
					
					\colourkeyentry{cb3class0}{$(\textrm{X}, \textrm{Y}, 0)$}
					\colourkeyentry{cb3class1}{$(\textrm{X} - \textrm{Y}, 0, - \textrm{Y})$}
					\colourkeyentry{cb3class2}{$(0, \textrm{Y} - \textrm{X}, - \textrm{X})$}
					
				\end{tikzpicture}
				
				\caption{The twelve candidate shortest-path vectors considered by INSEE
				represented as dimension-order routes. $W=H=8$,
				$(\Delta_\textrm{X},\Delta_\textrm{Y}) = (5, 3)$ and
				$(\Delta'_\textrm{X},\Delta'_\textrm{Y}) = (-3, -5)$.}
				\label{fig:insee-vector-candidates}
			\end{figure}
			
			The twelve vectors considered are constructed as follows.
			
			First a shortest path vector from the source to target node are
			constructed as if the network was a 2D mesh yielding a vector
			$(\Delta_\textrm{X},\Delta_\textrm{Y})$. From this, another vector
			$(\Delta'_\textrm{X},\Delta'_\textrm{Y})$, is defined:
			%
			\begin{align}
				\Delta'_\textrm{X} &= \Delta_\textrm{X} - \operatorname{sign}(\Delta_\textrm{X})W
				\\
				\Delta'_\textrm{Y} &= \Delta_\textrm{Y} - \operatorname{sign}(\Delta_\textrm{Y})H
			\end{align}
			%
			Where $W$ and $H$ are the width and height of the network respectively
			(in nodes). This new vector yields routes from the source to destination
			node that in a torus topology that \emph{always} wrap around the `X' and
			`Y' dimensions.
			
			From the pair of vectors defined, four possible 2D vectors can be
			produced: $(\Delta_\textrm{X},\Delta_\textrm{Y})$,
			$(\Delta'_\textrm{X},\Delta_\textrm{Y})$,
			$(\Delta_\textrm{X},\Delta'_\textrm{Y})$ and
			$(\Delta'_\textrm{X},\Delta'_\textrm{Y})$. Further, each 2D vector may be
			converted into one of three 3D vectors where either X, Y or Z are zero
			for a total of twelve candidate vectors.  Figure
			\ref{fig:insee-vector-candidates} illustrates all twelve candidate
			vectors for an example pair of nodes.
			
			\begin{figure}
				\center
				\buildfig{figures/xyz-protocol-regions.tex}
				
				\caption{The four regions defined by the XYZ-protocol.}
				\label{fig:xyz-protocol-regions}
			\end{figure}
			
			A more efficient technique is proposed by Hoffmann and D\'es\'erable
			called the XYZ-Protocol \cite{hoffmann15,hoffmann11}. If the source and
			destination nodes are translated such that the source node lies at the
			center of the topolgoy, the destination will lie in one of four regions
			illustrated in figure \ref{fig:xyz-protocol-regions}.
			
			If the destination lies in regions 1 or 4, a route may be constructed as
			if in a hexagonal mesh topology.
			
			Alternatively, if the destination lies in regions 2 or 3, the algorithm
			tests whether taking a mesh-like route within the region or
			wrapping-around either the X or Y dimension yields the shorter vector.
			The shortest of these vectors is then chosen.
			
			TODO DESCRIBE SPIRAL ROUTES.
			
			TODO DESCRIBE RTOR AND LDFR.
		
	\section{Dimension order routing in hexagonal torus topologies}
		
		So, existing solutions have two problems: trying 12 options and picking one
		is a bit kludgey and there are actually more options than that.
		
		\subsection{Simpler minimal hexagonal torus vectors}
			
			If you redraw your route such that it is sourced from bottom left corner
			(which we'll now call (0, 0)), there are four possible ways this route
			could wrap.
			
			TODO: DESCRIBE WAYS OF WRAPPING
			
			For each of these wrappings, all the possible routes we can take are
			strictly limited in terms of the dimensions used since we're stuck in a
			corner.
			
			In each case, the function computing the minimal hex vector function
			simplifies to a much simpler operation.
			
			TODO: DESCRIBE MINIMUM VECTOR LENGTH FUNCTIONS FOR EACH CASE
			
			This gives us a cheap way to compute which of the four possible wrappings
			are shortest. Based on this we can pick one of at most two (is this
			easily provable?) vectors in some fair manner to reduce load. This vector
			can then be minimised in the usual way.
			
			This also leads to confirming a theoretical result giving the length of a
			shortest path in a hexagonal torus topology.
			
			TODO: DESCRIBE HOW TO GET LENGTH FUNCTION AND COMPARE WITH \cite{xiao04}
		
		\subsection{Generating spiralling routes}
			
			In non-hexagonal torus topologies the previous technique would reveal all
			possible shortest vectors (e.g. in cases where you can wrap more than one
			way). Unfortunately, due to the addition of a non-orthogonal axes,
			hexagonal toruses also have other cases when the width and height do not
			match.
			
			TODO: TORUS SPIRALLING EXAMPLE
			
			It is possible to calculate the maximum number of spirals thus:
			
			TODO: DESCRIBE HOW MAX NUMBER OF SPIRALS IS COMPUTED
			
			Given a number of spirals, the vector can be updated this (note that the
			change does not add a multiple of (1, 1, 1) but also does not result in
			the vector changing length and thus becoming non-minimal!).
			
			TODO: DESCRIBE TRANSFORMATION
			
			TODO: PROVE THAT MINIMALITY IS MAINTAINED
		
		\subsection{Proof of completeness}
		
			TODO: PROOF OF COMPLETENESS BY EXHAUSTIVE SEARCH
	
		\subsection{Conclusions}
			
			This approach is simpler, smaller, has sounder theoretical basis, and
			finds more routes than alternatives. This is good for load balancing and
			fault avoidance and also good for completeness.

