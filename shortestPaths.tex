\chapter{Finding shortest paths in SpiNNaker's network}
	
	TODO: INTRO MOTIVATING NEED FOR SHORTEST PATH CALCULATIONS
	
	\section{Related work}
		
		DOR routes follow a shortest-path from source to destination following one
		dimensions then the next. For mesh topologies this requires computing a
		vector from the source to the sink and is straightforward:
		
		TODO DESCRIBE MESH CALCULATION
		
		For hexagonal toruses, you have three dimensions but they aren't
		orthogonal... \cite{patel15}
		
		TODO DESCRIBE HEX MESH CALCULATION
		
		For non-hexagonal toruses you just recenter everything and take the normal
		mesh vector. For hexagonal toruses this doesn't work.
		
		TODO DESCRIBE HEX TORUS CALCULATION NOT WORKING WHEN RECENTERING
		
		All known existing solutions perform this calculation by trying 12-possible
		DOR routes and picking the shortest. TODO: CITE SOME IMPLEMENTATIONS
		
		TODO DESCRIBE THE 12 ROUTES
		
		As we'll show later this mechanism is not very elegant/efficient and more
		importantly does not reveal all possible routes, e.g. those that spiral.
		
		TODO DESCRIBE SPIRAL ROUTES.
		
		TODO DESCRIBE RTOR AND LDFR.
		
	\section{Dimension order routing in hexagonal torus topologies}
		
		So, existing solutions have two problems: trying 12 options and picking one
		is a bit kludgey and there are actually more options than that.
		
		\subsection{Simpler minimal hexagonal torus vectors}
			
			If you redraw your route such that it is sourced from bottom left corner
			(which we'll now call (0, 0)), there are four possible ways this route
			could wrap.
			
			TODO: DESCRIBE WAYS OF WRAPPING
			
			For each of these wrappings, all the possible routes we can take are
			strictly limited in terms of the dimensions used since we're stuck in a
			corner.
			
			In each case, the function computing the minimal hex vector function
			simplifies to a much simpler operation.
			
			TODO: DESCRIBE MINIMUM VECTOR LENGTH FUNCTIONS FOR EACH CASE
			
			This gives us a cheap way to compute which of the four possible wrappings
			are shortest. Based on this we can pick one of at most two (is this
			easily provable?) vectors in some fair manner to reduce load. This vector
			can then be minimised in the usual way.
			
			This also leads to confirming a theoretical result giving the length of a
			shortest path in a hexagonal torus topology.
			
			TODO: DESCRIBE HOW TO GET LENGTH FUNCTION AND COMPARE WITH \cite{xiao04}
		
		\subsection{Generating spiralling routes}
			
			In non-hexagonal torus topologies the previous technique would reveal all
			possible shortest vectors (e.g. in cases where you can wrap more than one
			way). Unfortunately, due to the addition of a non-orthogonal axes,
			hexagonal toruses also have other cases when the width and height do not
			match.
			
			TODO: TORUS SPIRALLING EXAMPLE
			
			It is possible to calculate the maximum number of spirals thus:
			
			TODO: DESCRIBE HOW MAX NUMBER OF SPIRALS IS COMPUTED
			
			Given a number of spirals, the vector can be updated this (note that the
			change does not add a multiple of (1, 1, 1) but also does not result in
			the vector changing length and thus becoming non-minimal!).
			
			TODO: DESCRIBE TRANSFORMATION
			
			TODO: PROVE THAT MINIMALITY IS MAINTAINED
		
		\subsection{Proof of completeness}
		
			TODO: PROOF OF COMPLETENESS BY EXHAUSTIVE SEARCH
	
		\subsection{Conclusions}
			
			This approach is simpler, smaller, has sounder theoretical basis, and
			finds more routes than alternatives. This is good for load balancing and
			fault avoidance and also good for completeness.

