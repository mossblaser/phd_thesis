\chapter{Software}
	\label{sec:software}
	
	The techniques described in this thesis have been implemented in various
	software systems now in use in the wider SpiNNaker ecosystem. These
	implementations may prove instructive to future researchers hoping to
	leverage or extend the techniques described by this work. With one exception,
	all software is written in the Python programming language.
	
	\begin{description}
		
		\item [SpiNNer:] A collection of tools for generating cabling plans, guiding
		installation and assisting with maintenance using the techniques described
		in chapter~\ref{sec:building}.
		\\
		\vspace*{-1.25em}
		\\
		\begin{tabular}{ll}
			Source & \url{https://github.com/SpiNNakerManchester/SpiNNer} \\
			Manual & \url{http://spinner.readthedocs.org/} \\
		\end{tabular}
		
		\item [Rig:] A library for SpiNNaker application writers which implements
		the shortest-path vector geometry functions introduced in
		chapter~\ref{sec:shortestPaths}, the NER/PGS repair routing algorithm
		described in chapter~\ref{sec:routing} and the simulated annealing based
		placement algorithm from chapter~\ref{sec:placement}.
		\\
		\vspace*{-1.25em}
		\\
		\begin{tabular}{ll}
			Source & \url{https://github.com/project-rig/rig} \\
			Manual & \url{http://rig.readthedocs.org/} \\
		\end{tabular}
		
		\item [Rig C SA:] A C implementation of the simulated annealing based
		placement algorithm from chapter~\ref{sec:placement}. An optional add-on
		for the Rig library.
		\\
		\vspace*{-1.25em}
		\\
		\begin{tabular}{ll}
			Source & \url{https://github.com/project-rig/rig_c_sa} \\
		\end{tabular}
		
		\item [Network Tester:] A library for quickly building and running
		experiments on SpiNNaker's network and used to perform experiments in
		chapters~\ref{sec:routing} and \ref{sec:placement}.
		\\
		\vspace*{-1.25em}
		\\
		\begin{tabular}{ll}
			Source & \url{https://github.com/project-rig/network_tester} \\
			Manual & \url{http://network-tester.readthedocs.org/} \\
		\end{tabular}
		
		\item [Spalloc:] A centralised system which partitions large SpiNNaker
		machines into smaller ones on demand. This system was used to allow many of
		the experiments performed to be carried out in parallel.
		\\
		\vspace*{-1.25em}
		\\
		\begin{tabular}{ll}
			Source & \url{https://github.com/project-rig/spalloc_server} \\
			Manual & \url{http://spalloc-server.readthedocs.org/} \\
		\end{tabular}
		
	\end{description}
