\chapter{Software}
	\label{sec:software}
	
	The techniques described in this thesis have been implemented by the author
	in various software systems which are now in use in the wider SpiNNaker
	ecosystem. These implementations may prove instructive to future researchers
	hoping to leverage or extend the techniques described by this work.
	
	\begin{description}
		
		\item [SpiNNer:] A collection of tools for generating cabling plans and
		guiding installation and maintenance using the techniques described in
		chapter~\ref{sec:building}.
		\\
		\vspace*{-1.25em}
		\\
		\begin{tabular}{ll}
			Source & \url{https://github.com/SpiNNakerManchester/SpiNNer} \\
			Manual & \url{http://spinner.readthedocs.org/} \\
		\end{tabular}
		
		\item [Rig:] A library for SpiNNaker application writers which, among other
		features, implements the shortest-path vector geometry functions introduced
		in chapter~\ref{sec:shortestPaths}, the NER/PGS repair routing algorithm
		described in chapter~\ref{sec:routing} and the simulated annealing based
		placement algorithm from chapter~\ref{sec:placement}.
		\\
		\vspace*{-1.25em}
		\\
		\begin{tabular}{ll}
			Source & \url{https://github.com/project-rig/rig} \\
			Manual & \url{http://rig.readthedocs.org/} \\
		\end{tabular}
		
		\item [Rig C SA:] A C implementation of the simulated annealing based
		placement algorithm from chapter~\ref{sec:placement}. An optional drop-in
		replacement for Python implementation of the algorithm in the Rig library.
		\\
		\vspace*{-1.25em}
		\\
		\begin{tabular}{ll}
			Source & \url{https://github.com/project-rig/rig_c_sa} \\
		\end{tabular}
		
		\item [Network Tester:] A library for quickly building and running
		experiments on SpiNNaker's network and used to perform some of the
		experiments in chapters~\ref{sec:routing} and \ref{sec:placement}.
		\\
		\vspace*{-1.25em}
		\\
		\begin{tabular}{ll}
			Source & \url{https://github.com/project-rig/network_tester} \\
			Manual & \url{http://network-tester.readthedocs.org/} \\
		\end{tabular}
		
		\item [Spalloc:] A centralised system which partitions large SpiNNaker
		machines into smaller ones on demand. This system was used to perform many
		experiments in parallel. This system is also widely used by researchers to
		share access to large SpiNNaker machines.
		\\
		\vspace*{-1.25em}
		\\
		\begin{tabular}{ll}
			Source & \url{https://github.com/project-rig/spalloc_server} \\
			Manual & \url{http://spalloc-server.readthedocs.org/} \\
		\end{tabular}
		
	\end{description}

In addition to these `stand-alone' software packages, the source code and raw
result data for the experiments in this thesis are available to download from
GitHub:

\url{https://github.com/mossblaser/phd_thesis}

\url{https://github.com/mossblaser/phd_thesis_experiments}
