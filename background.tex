\chapter{Background}

	\section{Super computers}
		\subsection{Applications and architecture}
		\subsection{Network topology and construction}
		\subsection{Routing}
	
	\section{Neural modelling and simulation}
		\subsection{Neural modelling}
		\subsection{Spikes}
		\subsection{High-level modelling tools}
	
	\section{SpiNNaker}
		\subsection{Architecture}
		\subsection{Network topology}
		
			% TODO: Introduce topology
			
			% TODO: Reword the following lifted straight from wiring paper.
			
			\begin{figure}
				\center
				\buildfig{figures/hexagonalTorusTopology.tex}
				
				\caption{A $10 \times 10$ hexagonal torus topology with wrap-around
				links between $a$, $b$, $c$ and $a'$, $b'$, $c'$, respectively, omitted
				for clarity. The hexagonal torus's three non-orthogonal axes are shown
				in grey.}
				\label{fig:hexagonalTorusTopology}
			\end{figure}
			
			Figure \ref{fig:hexagonalTorusTopology} illustrates a hexagonal torus
			topology where each hexagon represents a node in the torus and touching
			edges represent connections between nodes. The nodes at the periphery of
			the system connect via wrap-around links (not shown) to nodes on opposing
			sides. To illustrate how the topology gets its name, imagine rolling the
			network into a tube such that edges $a$ and $a'$ connect. Next, bend this
			tube around forming a torus (doughnut shape), connecting $b$ to $b'$ and
			$c$ to $c'$. Note that in this paper we draw nodes as pointy-topped
			hexagons without any loss of generality: for the case where nodes are
			drawn as flat-topped hexagons, rotate the paper by 30\degree{}.
			
			Hexagonal torus topologies share many features with 2D toruses making them
			easy to reason about and describe. For example, just as in 2D toruses, the
			X- and Y- axes wrap-around single rows and columns and they form an easily
			addressed, regularly shaped system. Note that other toroidal topologies
			can be constructed with hexagonal meshes, but these are more closely
			related to twisted-torus topologies where wrapping around the X- and Y-
			axes can result in paths passing through every node in the system
			\cite{camara10}.
			
			Hexagonal topologies can be addressed by a non-orthogonal coordinate
			system through which Manhattan paths approximate Euclidean distances much
			more closely than square topologies. It is this property that gives
			hexagonal toruses their improved bisection bandwidth over 2D toruses and
			makes them popular as grids in board and computer games \cite{patel15}.
		
		\subsection{Partitioning hexagonal toruses}
			
			\label{sec:parititioning}
			
			The nodes in most super computer networks are relatively fine-grained by
			comparison with their physical construction in order to make construction
			more practical. For example, a single circuit board may contain tens or
			even hundreds of nodes \cite{gilge14,ajima12}. The majority of commercial
			super computers use torus topologies partitioned into hypercubes as
			illustrated in figure \ref{fig:hypercube-partitioning}
			\cite{chen11,ajima12}.
			
			\begin{figure}
				\center
				\begin{subfigure}[b]{0.45\textwidth}
					\center
					\buildfig{figures/hypercube-partitioning.tex}
					\caption{2D hypercube partitioning}
					\label{fig:hypercube-partitioning}
				\end{subfigure}
				\begin{subfigure}[b]{0.45\textwidth}
					\center
					\buildfig{figures/parallelogram-partitioning.tex}
					\caption{Parallelogram partitioning}
					\label{fig:parallelogram-partitioning}
				\end{subfigure}
				
				\caption{Conventional hypercube topology partitioning (a) and the
				hexagonal torus topology analogue (b).}
				\label{fig:partitioning-options}
			\end{figure}
			
			One possible analogue in a hexagonal torus topology is a parallelogram as
			illustrated in figure \ref{fig:parallelogram-partitioning}.
			
			Each partition connects to six neighbouring partitions and, unlike
			hypercube partitions, the number of connections to each is imbalanced.
			Specifically the partitions above-right and below-left are connected by
			only one link each. The consequence is potentially a need for multiple
			types of interconnect for connecting partitioned pieces of the system,
			adding both to design complexity and cost.
			
			Another `obvious' choice of partition is that of a hexagon wrapped in
			concentric layers of hexagons as illustrated in figure
			\ref{fig:wrapped-hexagon-tiling}.
			
			\begin{figure}
				\center
				\buildfig{figures/wrapped-hexagon-tiling.tex}
				
				\caption{A single hexagon wrapped in layers of hexagons does not tile the
				same way as a pointy-topped hexagon.}
				\label{fig:wrapped-hexagon-tiling}
			\end{figure}
			
			While this partition presents six equally-sized edges, it cannot be used
			to build a hexagonal torus as described. A partition constructed from
			pointy-topped hexagons tiles a hexagonal torus topology constructed from
			pointy-topped hexagons iff the partition tiles with a translational
			symmetry shared with a hexagonal torus. As we can see from the figure,
			this is not the case. This tiles more like a twisted torus of some sort.
			To prove this test is legit, see figure \ref{fig:parallelogram-tiling}
			where we superimpose a pointy-topped hexagon on a parallelogram.
			
			\begin{figure}
				\center
				\buildfig{figures/parallelogram-tiling.tex}
				
				\caption{Demonstration that a parallelogram partition tiles a hexagonal
				torus.}
				\label{fig:parallelogram-tiling}
			\end{figure}
			
			Furber, Davison \emph{et al.} \cite{davidsonWiring} proposed an
			alternative based on wrapped triples where you take three hexagons and
			wrap more layers around them as required. This pattern tiles like
			flat-topped hexagons, which isn't quite right. But since we know a
			pointy-topped triple tiles like a flat-topped hexagon, a flat-topped
			triple must tile like a pointy-topped hexagon. Combining these facts
			gives that by tiling triads of wrapped triples we can tile a hexagonal
			torus.
			
			\begin{figure}
				\center
				\buildfig{figures/wrapped-triple-tiling.tex}
				
				\caption{Three pointy-topped hexagons tiles in the same way as flat-topped
				hexagons.}
				\label{fig:wrapped-triple-tiling}
			\end{figure}
			
			\begin{figure}
				\center
				\buildfig{figures/triad-tiling.tex}
				
				\caption{Demonstration that a triad tiles a hexagonal torus topology.}
				\label{fig:triad-tiling}
			\end{figure}
		
		\subsection{Router micro-architecture}
		\subsection{Application work-flow}
