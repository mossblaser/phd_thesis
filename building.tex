\chapter{Building large SpiNNaker machines}
	
	When physically constructing a large super computer the network topology used
	can have a huge impact on the cost, complexity and effort involved in the
	process. This chapter describes the approaches used to design conventional
	torus-based super computer installations which ensure that cable lengths
	remain manageable and ensure a technician is realistically able to connect
	them. These approaches, unfortunately, do not extend directly to hexagonal
	torus topologies and so a number of new techniques are described which allow
	large hexagonal-torus based systems to be constructed. The chapter concludes
	by describing the successful application of these techniques to very large
	SpiNNaker machines.
	
	\section{Related work}
		
		The process can be broken into three steps: partitioning, layout and
		installation. During partitioning the topology is broken up into pieces
		which can easily be manufactured and handled. During layout, the
		partitioned pieces of the topology are physically laid out in order to
		control cable length. During installation a technician must physically
		install cables connecting things together.
		
		\subsection{Partitioning hexagonal toruses}
			
			The nodes in most super computer networks are relatively fine-grained by
			comparison with their physical construction in order to make construction
			more practical. For example, a single circuit board may contain tens or
			even hundreds of nodes \cite{gilge14,ajima12}. The majority of commercial
			super computers use torus topologies partitioned into hypercubes
			\cite{chen11,ajima12}.
			
			TODO: FIGURE HYPERCUBE PARTITIONING
			
			One possible analogue in a hexagonal torus topology is a parallelogram as
			illustrated in figure \ref{fig:parallelogramPartitioning}.
			
			\begin{figure}
				\center
				\begin{tikzpicture}
	\def\size{4}
	\def\scale{0.5}
	
	\pgfmathtruncatemacro{\sizee}{\size-1}
	\pgfmathtruncatemacro{\spacing}{\size+1}
	\pgfmathtruncatemacro{\sizeTotal}{(\size*3)-1}
	
	\begin{scope}[scale=\scale,hexagonXYZ]
		% Set up coordinates
		\foreach \px in {0,1,2}{
			\foreach \py in {0,1,2}{
				\foreach \x in {0,...,\sizee}{
					\foreach \y in {0,...,\sizee}{
						\pgfmathtruncatemacro{\xx}{\x + (\size*\px)}
						\pgfmathtruncatemacro{\yy}{\y + (\size*\py)}
						\coordinate (node x\xx y\yy)
						            at ( \px*\spacing + \x
						               , \py*\spacing + \y
						               ) {};
					}
				}
			}
		}
	\end{scope}
	
	% Clip the view
	\pgfmathtruncatemacro{\x}{\size}
	\pgfmathtruncatemacro{\y}{\size*2 - 1}
	\pgfmathtruncatemacro{\xx}{\size*2 - 1}
	\pgfmathtruncatemacro{\yy}{\size}
	\clip ([shift={(-2*\scale cm, 2*\scale cm)}]node x\x y\y)
	      rectangle
	      ([shift={( 2*\scale cm,-2*\scale cm)}]node x\xx y\yy)
	      ;
	
	% Draw the nodes
	\foreach \thickness in {1.6pt, 0.6pt}{
		\foreach \x in {0,...,\sizeTotal}{
			\foreach \y in {0,...,\sizeTotal}{
				\node [ draw=black
				      , fill=white
				      , hexagon
				      , inner sep=0
				      , minimum width=\scale * 1cm
				      , line width=\thickness
				      ]
				      (node x\x y\y)
				      at (node x\x y\y)
				      {}
				      ;
			}
		}
	}
	
	% Draw interconnecting wires
	\begin{scope}
		\foreach \x in {1,...,\sizeTotal}{
			\foreach \y in {1,...,\sizeTotal}{
				\pgfmathtruncatemacro{\xx}{\x - 1}
				\pgfmathtruncatemacro{\yy}{\y - 1}
				\draw (node x\xx y\y.side east) -- (node x\x y\y.side west);
				\draw (node x\x y\yy.side north) -- (node x\x y\y.side south);
				\draw (node x\xx y\yy.side north east) -- (node x\x y\y.side south west);
			}
		}
	\end{scope}
\end{tikzpicture}


				
				\caption{Partitioning of a hexagonal torus into parallelograms.}
				\label{fig:parallelogramPartitioning}
			\end{figure}
			
			Each partition connects to six neighbouring partitions and, unlike
			hypercube partitions, the number of connections to each is imbalanced.
			Specifically the partitions above-right and below-left are connected by
			only one link each. The consequence is potentially a need for multiple
			types of interconnect for connecting partitioned pieces of the system,
			adding both to design complexity and cost.
			
			Another 'obvious' choice of partition is a hexagonal one like so:
			
			TODO: FIGURE SHOWING WRAPPED HEXAGON
			
			While this partition presents six equally-sized edges, it cannot be used
			to build a hexagonal torus as described. A partition constructed from
			pointy-topped hexagons tiles a hexagonal torus topology constructed from
			pointy-topped hexagons iff the partition tiles with a translational
			symmetry shared with a hexagonal torus. As we can see from the figure,
			this is not the case. This tiles more like a twisted torus of some sort.
			To prove this test is legit, see the figure below where we superimpose a
			pointy-topped hexagon on a parallelogram.
			
			TODO: FIGURE POINTY TOP SUPERIMPOSED PARALLELOGRAM TILING
			
			Furber, Davison \emph{et al.} \cite{davidsonWiring} proposed an
			alternative based on wrapped triples where you take three hexagons and
			wrap more layers around them as required. This pattern tiles like
			flat-topped hexagons, which isn't quite right. But since we know a
			pointy-topped triple
			tiles like a flat-topped hexagon, a flat-topped triple must tile like a
			pointy-topped hexagon. Combining these facts gives that by tiling triads
			of wrapped triples we can tile a hexagonal torus.
			
			TODO: FIGURES DEMONSTRATING TILING OF TRIADS OF WRAPPED TRIPLES
		
		\subsection{Folding regular toruses}
			
			Once partitioned, a na\"ive arrangements of torus topologies, including
			hexagonal torus topologies, feature long `wrap-around' connections.
			
			TODO: FIGURE SHOWING LONG CABLES DUE TO WRAP-AROUND
			
			In conventional torus topologies these long connections are eliminated by
			folding and interleaving nodes of the network. For example, for a 1D
			torus network (a.k.a. a ring network), one long connection exists to
			connect the two opposite sides of the system. To remove these long
			connections, half the nodes are `folded' on top of the others and then
			this arrangement of nodes is interleaved.
			
			TODO: FIGURE SHOWING RING NETWORK BEING FOLDED
			
			This approach substantially decreases the maximum wire length (in fact
			does so regardless of network size) at the expense of approximately
			doubling the average wire length. In practice, the maximum length is more
			important and so this approach is widely used in practice.
			
			Unfortunately, the non-orthogonality of hexagonal torus topologies means
			that folding does not work correctly. In this chapter I describe a few
			transformations that fix this in various settings.
		
		\subsection{Cabling installation}
			
			Conventional machine room installations often feature very repetitive
			cabling patterns which can easily be followed by a human technician.
			Such cabling patterns are more practical in data-center networks where
			cabling within a cabinet tends to concentrate on a small number of
			high-port-count switches. Unfortunately, such patterns do not appear as
			commonly in super computer cabling where point-to-point connections
			between compute nodes are the norm.
			
			TODO: MORE BACKGROUND READING FOR DATACENTER ETC HOPEFULLY CAN FIND SOME
			INFO ABOUT WIRE LENGTH SELECTION RATHER THAN MOSTLY CABLE MANAGEMENT. TIA
			STANDARD SEEMS GOOD.
			
			In BlueGene architectures, the 5D torus network topology places some of
			its torus wiring on fixed PCB backplanes which means no manual wiring is
			required at all. The rest of the cabling uses optical interconnect which
			must be manually installed. Groups of individual cables are custom-made
			to match the exact lengths requirede and factory assembled into labeled
			bundles of cables which connect pairs of cabinets. Technicians then use
			the labels to guide installation order into pre-labeled sockets.
			
			TODO: IBM BLUEGENE ASSEMBLY PHOTOS
			
			Manufacturing custom cables is an expensive proposition and is not
			practical for many applications. SpiNNaker's interconnect uses
			off-the-shelf S-ATA cables to reduce costs. Such cables are widely
			available in standardised lengths. SpiNNaker's packaging is also
			significantly finer-grained than BlueGene which comes in pre-assembled
			cabinets with most intra-cabinet connectivity provided by the backplane.
			SpiNNaker has 120 PCBs with 6 sockets each in every cabinet, all of which
			need connecting. The labelled approach does not scale.
			
			TODO: PHOTO OF SPINNAKER PCBS IN RACK.
	
	\section{Folding \& interleaving hexagonal toruses}
		
		In order to fold a hexagonal torus topology we must transform it into a
		regular rectangular grid. In this section we break this process down into
		two steps: transformation from a parallelogram to a rectangle and
		uncrinkling. We propose two alternative transformations which can be
		applied to hexagonal torus topologies constructed out of triads of triples
		or to the basic torus topology.
		
		\subsection{Transformation}
			
			The hexagonal torus topology is most naturally drawn as a parallelogram
			(see left of figures). As described earlier, folding this just doesn't
			work since the ends don't meet. To make this foldable we must transform
			the network into a rectangle. We present two transformations to do this.
			
			The first transformation we describe is shearing. When shearing you
			distort the network to cause the X and Y axes to become orthogonal and
			the whole shape to become a rectangle. The down-side of this approach is
			that the Z dimension has been stretched and this distortion means that
			cables going in the Z direction will need to be $\sqrt(2)$ times longer
			than those going in the X and Y directions.
			
			The second transformation is slicing where we translate the left triangle
			of the parallelogram onto the opposite side forming a generally
			rectangular shape. This transformation has the advantage that it
			maintains distances in hexagonal meshes without distortion but sadly with
			wires criss-crossing which will cause some headaches later.
			
			TODO: FIGURE SHOWING NATIVE VS SLICING VS SHEARING FOR TRIADS AND CHIPS
			
		\subsection{Uncrinkling}
			
			At this point we have two ways of getting into a rectangular arrangement
			but this still cannot be folded since the spacing of individual nodes
			does not evenly fill a regular 2D grid. For sheared systems of triads and
			for sliced systems of either kind, the arrays of hexagons contain
			crinkled columns and rows as illustrated below.
			
			TODO: UNCRINKLING FIGURE
			
			By uncrinkling these columns or rows, a regular 2D grid arrangement is
			formed. In the figure, the numbered hexagons enumerate the different
			positions on the crinkle and those labelled alphabetically are those that
			immediately surround them. From this we can observe that uncrinkling
			largely preserves spatial locality but some distortion is introduced
			separating previously neighbouring nodes. For example, in C, the wrapped
			triples labelled `1' and `i' are neighbours before uncrinkling but are
			separated by a (Euclidean) distance of $\sqrt{5}$ afterwards. Note that
			the distortion introduced depends on what part of the crinkle is
			considered, for example `2' and `a' have distance 2 but are logically
			connected in the same way.
		
		\subsection{Folding}
			
			We now have a square, 2D grid with the usual patterns of connectivity:
			all very local except for the connections crossing from side-to-side and
			normal folding techniques can be used.
			
			TODO: FIGURE SHOWING FOLDING WORKING
			
			That is unless you used the slicing technique. If you've done this then
			you get criss-crossing streams of cables which folding will not remove,
			depending on the dimensions of the network.
			
			TODO: FIGURE SHOWING BAD SLICING PRODUCING BAD FOLDS
			
			If you have Nx2N then you actually get normal crossing patterns and thus
			folding can be done as-per-normal.
			
			If you have NxN, though, it all goes wrong and you get criss-crossing.
			Such systems can be folded by doing a double vertical fold.
		
		\subsection{Interleaving}
			
			Interleaving is performed as illustrated earlier.
			
			TODO: MORE FULLY SPECIFY
		
		\subsection{Mapping to Cabinets}
			
			A simple linear allocation to cabinet locations.
			
			TODO: DESCRIBE ALGORITHM
			
			The result is a connection list like so...
		
	\section{Installation}
		
		Installation will be performed by a (team of) technicians and requires that
		all cables are connected correctly and connectivity tested.
		
		\subsection{Cable selection}
			
			The common length selection method of requiring a fixed slack is
			insufficient in practice due to the wide variation in cable lengths
			(centimetres to tens of centimetres): short cables become too tight and
			long cables become too slack. Based on the observation that short and
			medium length cables tend to form an arc between their endpoints we
			propose a wire length selection heuristic based on enforcing a minimum
			height of the arc. We then pick the available cable long enough to
			achieve the specified arc height.
			
			TODO: PHOTOGRAPHS OF CABLES FORMING ARCS AND SHORT LOOPS
			
			TODO: DIAGRAM SHOWING ARC/LOOP HEIGHT CONSTRAINT.
			
			When the wire forms the arc, the following eqn must be solved. Sadly this
			has no closed form solution so a simple numerical approximation is used
			to compute it.
			
			TODO: FORMULAE FOR ARC HEIGHT COMPUTATION
			
			In the case where the wire is too small to form an arc they form a loop
			whose height is calculated thus.
			
			TODO: FORMULAE FOR LOOP HEIGHT COMPUTATION
		
		\subsection{Interactive technician guidance and validation}
			
			By using diagnostic LEDs controlled via a low-bandwidth system-management
			bus provided by the backplane before cabling it is possible to provide
			interactive guidance to installing technicians. Rather than issuing paper
			guides and using cable labelling, these LEDs, augmented with a visual
			display and TTS can be used. This significantly reduces infrastructure
			requirements.
			
			This infrastructure can also be used to configure the high-speed serial
			links to enter a special diagnostic mode. In this mode unique IDs
			assigned by a host to each link are sent instead of sending random data
			during idle frames (used normally for EMC reasons). The unique IDs
			received at the opposing end of the connection can be used to determine
			cable correctness.
		
		\subsection{Installation ordering}
			
			The order of installation can have a significant impact on
			maintainability and installation effort. To make installation practical,
			wiring congestion should be avoided when possible.
			
			Thanks the locality of all connections, it is practical to install both
			ends at once (not often done in data centre installations) and doing so
			enables real-time validation of connections. This also means that cables
			need not be labelled.
			
			Based on the observation that cables tend to run in arcs, we want to
			install things in an order such that later cables can easily fit over the
			top of existing cables. Thanks to the arc model above, we order cable
			installation by arc height, installing those with wider arc height later.
			
			Additionally, cables are grouped by direction since SpiNNaker's cable
			sockets are also grouped by direction this means each direction does not
			obstruct the next until connections begin to run between cabinets.
			
			We also group cables by intra and inter rack and cabinet. As well as
			generally causing short cables are installed first, wires crossing
			between cabinets and racks are inherently going to obstruct cables within
			the rack so we want to do that second. Further since racks are
			conveniently wide enough for a single technician to comfortably work,
			saving intra-cabinet connections 'til last means that for most of the
			installation the technicians run no risk of colliding.
			
			Finally, ordering runs top-to-bottom, left-to-right as the final
			tie-breaker. Since hops with similar arc length etc. tend to follow
			similar patterns, this means that wires tend to follow in repeating
			patterns.
	
	\section{Results and Evaluation}
		
		\subsection{Cable length}
			
			TODO: CALCULATE AND DESCRIBE COSTS INTRODUCED AT EACH STAGE:
			TRANSFORMATION, UNCRINKLING, FOLDING + INTERLEAVING AND MAPPING INTO REAL
			HARDWARE FOR SPINNAKER.
			
			Given the approaches suggested, we find that in general shearing and 2x2
			folding works the best unless you have a Nx2N system in which case
			slicing and 2x2 folding works better if you have a triad-partitioned
			network.
			
			TODO: FIGURE OF LARGE SPINNAKER MACHINE WIRING PLAN
			
			When mapped to real hardware the largest planned SpiNNaker machine
			requires cables no longer than 70cm.
			
		\subsection{Installation practicality}
			
			TODO: LENGTH SUITABILITY (EXPERIMENTS?) ARE THE CABLES TOO TIGHT OR NOT.
			PROBABLY LIMITED TO INFORMAL MEASUREMENTS
			
			TODO: ANALYSIS OF INSTALL TIME DEPENDING ON CABLE LENGTH AND CROSSING
			TYPES
			
			TODO: GUIDANCE AND INSTALLATION. MAYBE INCLUDE TRIALS OF UNDERGRAD
			VOLUNTEERS. COMPARE INSTALL WITH AND WITHOUT VALIDATION.
			
			With in-line verification, all connectors were correctly seated in the
			correct sockets first time since everything is verified at build time.
			
		\subsection{Maintenance}
			
			TOOD: QUANTIFY CABLE REMOVALS REQUIRED. EXPERIMENT: REMOVE/REPLACE RANDOM
			BOARDS AND MEASURE TIME TAKEN, CABLES REMOVED. COMPARE WITH STANDARD DATA
			CENTER WIRING
