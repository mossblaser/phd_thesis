\chapter{Building large SpiNNaker machines}
	
	When physically constructing a large super computer the network topology used
	can have a huge impact on the cost, complexity and effort involved in the
	process. This chapter describes the approaches used to design conventional
	torus-based super computer installations which ensure that cable lengths
	remain manageable and ensure a technician is realistically able to connect
	them. These approaches, unfortunately, do not extend directly to hexagonal
	torus topologies and so a number of new techniques are described which allow
	large hexagonal-torus based systems to be constructed. The chapter concludes
	by describing the successful application of these techniques to very large
	SpiNNaker machines.
	
	\section{Related work}
		
		Conventionally, the design process for the physical arrangement of network
		nodes in datacenter racks is broken into two steps:
		
		\begin{enumerate}
			
			\item Partitioning the network into easily manufacturered components
			(e.g. circuit boards).
			
			\item Laying out the machine to avoid the need for (expensive) long
			cables.
		
		\end{enumerate}
		
		Finally, once the physical layout is designed, a technician is required to
		assemble the final machine. This process is often non-trivial and thus
		various techniques exist which attempt to simplify the task.
		
		\subsection{Partitioning hexagonal toruses}
			
			The nodes in most super computer networks are relatively fine-grained by
			comparison with their physical construction in order to make construction
			more practical. For example, a single circuit board may contain tens or
			even hundreds of nodes \cite{gilge14,ajima12}. The majority of commercial
			super computers use torus topologies partitioned into hypercubes
			\cite{chen11,ajima12} for which the analogue in a hexagonal torus
			topology is a parallelogram as illustrated in figure
			\ref{fig:parallelogramPartitioning}.
			
			\begin{figure}
				\center
				\begin{tikzpicture}
	\def\size{4}
	\def\scale{0.5}
	
	\pgfmathtruncatemacro{\sizee}{\size-1}
	\pgfmathtruncatemacro{\spacing}{\size+1}
	\pgfmathtruncatemacro{\sizeTotal}{(\size*3)-1}
	
	\begin{scope}[scale=\scale,hexagonXYZ]
		% Set up coordinates
		\foreach \px in {0,1,2}{
			\foreach \py in {0,1,2}{
				\foreach \x in {0,...,\sizee}{
					\foreach \y in {0,...,\sizee}{
						\pgfmathtruncatemacro{\xx}{\x + (\size*\px)}
						\pgfmathtruncatemacro{\yy}{\y + (\size*\py)}
						\coordinate (node x\xx y\yy)
						            at ( \px*\spacing + \x
						               , \py*\spacing + \y
						               ) {};
					}
				}
			}
		}
	\end{scope}
	
	% Clip the view
	\pgfmathtruncatemacro{\x}{\size}
	\pgfmathtruncatemacro{\y}{\size*2 - 1}
	\pgfmathtruncatemacro{\xx}{\size*2 - 1}
	\pgfmathtruncatemacro{\yy}{\size}
	\clip ([shift={(-2*\scale cm, 2*\scale cm)}]node x\x y\y)
	      rectangle
	      ([shift={( 2*\scale cm,-2*\scale cm)}]node x\xx y\yy)
	      ;
	
	% Draw the nodes
	\foreach \thickness in {1.6pt, 0.6pt}{
		\foreach \x in {0,...,\sizeTotal}{
			\foreach \y in {0,...,\sizeTotal}{
				\node [ draw=black
				      , fill=white
				      , hexagon
				      , inner sep=0
				      , minimum width=\scale * 1cm
				      , line width=\thickness
				      ]
				      (node x\x y\y)
				      at (node x\x y\y)
				      {}
				      ;
			}
		}
	}
	
	% Draw interconnecting wires
	\begin{scope}
		\foreach \x in {1,...,\sizeTotal}{
			\foreach \y in {1,...,\sizeTotal}{
				\pgfmathtruncatemacro{\xx}{\x - 1}
				\pgfmathtruncatemacro{\yy}{\y - 1}
				\draw (node x\xx y\y.side east) -- (node x\x y\y.side west);
				\draw (node x\x y\yy.side north) -- (node x\x y\y.side south);
				\draw (node x\xx y\yy.side north east) -- (node x\x y\y.side south west);
			}
		}
	\end{scope}
\end{tikzpicture}


				
				\caption{Partitioning of a hexagonal torus into parallelograms.}
				\label{fig:parallelogramPartitioning}
			\end{figure}
			
			Each partition connects to six neighbouring partitions and, unlike
			hypercube partitions, the number of connections to each is imbalanced.
			Specifically the partitions above-right and below-left are connected by
			only one link each. The consequence is potentially a need for multiple
			types of interconnect for connecting partitioned pieces of the system,
			adding both to design complexity and cost.
		
		\subsection{Folding regular toruses}
			
			
		
		\subsection{Cabling installation}
	
	\section{Folding hexagonal toruses}
		\subsection{Transformation}
		\subsection{Uncrinkling}
		\subsection{Folding}
	\section{Installation}
		\subsection{Cable selection}
		\subsection{Installation ordering}
		\subsection{Interactive technician guidance}
		\subsection{Validation}
	\section{Results}
		\subsection{Cable length}
		\subsection{Installation practicality}
