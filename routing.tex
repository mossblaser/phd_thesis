\chapter{Routing packets in large SpiNNaker machines}
	
	% In large networks, faults are inevitable and routing algorithms are
	% required to work around them. <Give some motivating fault numbers, e.g.
	% from bluegene or SpiNNaker>. This work hopes to extend the state of the art
	% by defining a new fault-tolerant routing algorithm for SpiNNaker which can
	% route around arbitrary permanent link failures.
	
	% Routing packets is NP complete and a key factor in the efficiency of a
	% super computer network. Numerous heuristic algorithms exist which target
	% different network technologies but many techniques are not portable.
	
	% In this work we care about SpiNNaker which has a table-based router, packet
	% dropping flow control, and applications which deal with multicast. So far
	% all proposed routing algorithms for SpiNNaker have been fault-intolerant
	% and in this chapter, a new fault-tolerant algorithm is proposed.
	
	% Various fault-intollerant multicast routing algorithms exist for many
	% networks and a number have been proposed for SpiNNaker. In davies12, three
	% routing strategies based on overlapping DOR, RTOR and LDFR routes are
	% suggested which have no ability to avoid faults.
	
	% TODO describe DOR, RTOR and LDFR.
	
	% In navaridas14 ESPR and NER are introduced which are able to generate more
	% efficient multicast routes but, again, do not have the ability to avoid
	% faults.
	
	% TODO describe ESPR and NER.
	
	% Work on fault-tolerant routing in super computer networks often focuses
	% heavily on the issue of deadlock avoidance. Super computers typically rely
	% on router micro architecture and restricted routing rules (dally93). In
	% addition to the costs imposed by this method of deadlock avoidance, such
	% restrictions can be fatal to fault tolerance: e.g. enforced DOR or no-turn
	% rules may prevent routing around a fault or non-optimal paths may be
	% required.
	
	To make efficient use of SpiNNaker's interconnection network packets must be
	routed reliably to their destinations such that network congestion is
	minimised. Routing algorithms are tightly constrained by the network topology
	and router micro-architecture on which they are used. As a result of this,
	exploiting SpiNNaker's unconventional topology and router architecture
	requires the development of new routing algorithms. The Neighbouhood
	Exploring Routing (NER) algorithm proposed by Navaridas \emph{et al.}
	\cite{navaridas14} is well suited to SpiNNaker's architecture and
	applications but lacks the ability to cope with faults, an inevetability in
	large machines. In addition, all known implementations of routing algorithms
	for hexagonal toruses cannot generate all valid Dimension Order Routes which
	in turn can result in increased network congestion.
	
	In this chapter an efficient algorithm for fair DOR on hexagonal toruses is
	presented and integrated into the NER routing algorithm. Fault-tolerance
	extensions to the NER algorithm are then described which ensure successful
	routing in the presence of any static fault condition. These extensions
	exploit the SpiNNaker router's timeout-based deadlock breaking behaviour
	enabling greater flexibility over routing decisions and avoiding the need for
	the complex routing strategies required by existing algorithms.
	
	\section{Related work}
		
		Routing algorithms continue to be a hot topic in networks research however
		many algorithms are tightly coupled with specific network technologies and
		thus unsuitable for use with SpiNNaker. For example, algorithms designed
		for non-torus topologies such as Internet routing and tree-like networks
		do not readily apply to torus networks.
		
		\subsection{Dimension Order Rotuing (DOR)}
			
			Dimension Order Routing is a simple unicast routing algorithm which forms
			the basis of many mesh and torus routing algorithms \cite{dally04}. In
			DOR, packets are first routed along each network dimension in turn in
			order to reach their destination. Figure \ref{fig:dimensionOrderRouting}
			shows two possible routes between a pair of nodes, A and B. In the red
			route, the Z dimension is used first followed by the X dimension. The
			blue route uses X then Z.
			
			\begin{figure}
				\center
				\begin{tikzpicture}[thick, scale=0.7]
	\def\width{8}
	\def\height{6}
	% Set node positions
	\foreach \y in {1,...,\height}{
		\foreach \x in {1,...,\width}{
			\node [hexagonXYZ]
			      at (\x, \y)
			      (node\x\y)
			      {}
			      ;
		}
	}
	\begin{scope}[help lines]
		\foreach \y in {1,...,\height}{
			\foreach \x in {1,...,\width}{
				\pgfmathtruncatemacro{\xx}{\x-1}
				\pgfmathtruncatemacro{\yy}{\y-1}
				\ifthenelse{\xx > 0}{
					\draw (node\x\y.center) -- (node\xx\y.center);
				}{ }
				\ifthenelse{\yy > 0}{
					\draw (node\x\y.center) -- (node\x\yy.center);
				}{ }
				\ifthenelse{\xx > 0 \AND \yy > 0}{
					\draw (node\x\y.center) -- (node\xx\yy.center);
				}{ }
			}
		}
		% Draw nodes
		\foreach \y in {1,...,\height}{
			\foreach \x in {1,...,\width}{
				\node [draw, fill=white, circle, inner sep=0, minimum size=0.7em]
				      at (node\x\y)
				      {}
				      ;
			}
		}
	\end{scope}
	
	% Highlight a pair of nodes
	\begin{scope}[font=\small]
		\node [fill=white, draw,circle, inner sep=0.1em]
		      (A)
		      at (node22)
		      {A}
		      ;
		\node [fill=white, draw,circle, inner sep=0.1em]
		      (B)
		      at (node75)
		      {B}
		      ;
	\end{scope}
	
	\begin{scope}[ultra thick]
		\draw [->,red] (A) -- (node55.center) -- (B);
		\draw [->,blue] (A) -- (node42.center) -- (B);
	\end{scope}
	
	
	% Axes
	\begin{scope}[ xshift=10cm
	             , yshift=3.2cm
	             , hexagonXYZ
	             , scale=1.5
	             , font=\footnotesize
	             , inner sep=0.1em
	             ]
		\draw [->]
		      (0,0,0) -- ++(1,0,0)
		      node [hexagon,anchor=side west] {X}
		      ;
		\draw [->]
		      (0,0,0) -- ++(0,1,0)
		      node [hexagon,anchor=side south] {Y}
		      ;
		\draw [->]
		      (0,0,0) -- ++(0,0,1)
		      node [hexagon,anchor=side north east] {Z}
		      ;
	\end{scope}
\end{tikzpicture}


				
				\caption{Two dimension order routes in a hexagonal mesh topology.}
				\label{fig:dimensionOrderRouting}
			\end{figure}
			
			The choice of dimension order does not impact the length of the routes
			produced but can be important in many systems. For example, some deadlock
			avoidance strategies rely on a fixed dimension ordering and will be
			described later.
			
		\subsection{Multicast Routing}
			
			In Longest Dimension First Routing (LDFR), a common variant of DOR, the
			dimension whose absolute value is greatest is used first which has a
			number of favourable properties.
		
		\subsection{Deadlock avoidance}
		\subsection{Fault-tolerance}
	
	\section{Dimension order routing in hexagonal torus topologies}
		\subsection{Hexagonal meshes}
		\subsection{Hexagonal toruses}
		\subsection{Ensuring fairness}
		\subsection{Generating spiralling routes}
	
	\section{Fault-tolerant multicast routing}
		\subsection{Graph search based repair}
		\subsection{Alternative routing heuristics}
		\subsection{Results}
			\subsubsection{Heuristic performance}
			\subsubsection{Runtime}
			\subsubsection{Route congestion}
	
