{
	\prefacesection{Abstract}
	
	% Single line spacing for the abstract page
	\setstretch{1.0}
	
	
	\vfill
	
	% Standard thesis information
	\begin{center}
		\textsc{\large\thesistitle}
		
		\vspace{1em}
		
		\thesisauthor
		
		\vspace{1em}
		
		A thesis submitted to the University of Manchester\\
		for the degree of Doctor of Philosophy, 2016
	\end{center}
	
	\vfill
	
	% The abstract
	Super computers consist of a large number of independent compute nodes
	(commonly CPU cores) interconnected by a high performance network. SpiNNaker is
	an unusual super computer designed for simulating large-scale neural models of
	the brain in biological realtime. This project addresses a number of the key
	practical challenges involved in the construction and operation of a full-sized
	SpiNNaker machine with over one million CPU cores.
	
	Due to its unconventional hexagonal torus network topology, conventional
	approaches to building large machine installations do not apply. We begin by
	proposing a novel machine-room layout for SpiNNaker systems which enables
	systems of arbitrary size to be constructed using only short cables, saving
	money and energy while also improving network performance. We follow this with
	a novel routing algorithm for hexagonal torus networks which is both more
	concise and more evenly utilises available resources. Since any large-scale
	system will inevetably have some faults we develop novel extensions to this
	algorithm which take advantage of unique features of SpiNNaker's network to
	tollerate arbitrary static network faults. Finally, to make the most
	efficient use of SpiNNaker's network we adapt techniques from the field of chip
	design to efficiently map applications across SpiNNakers million cores while
	making effective use of the network.

	% Required to ensure linespacing is set for this whole block
	\par%
}
