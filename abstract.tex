{
	\prefacesection{Abstract}
	
	% Single line spacing for the abstract page
	\setstretch{1.0}
	
	
	\vfill
	
	% Standard thesis information
	\begin{center}
		\textsc{\large\thesistitle}
		
		\vspace{0.5em}
		
		\thesisauthor
		
		\vspace{0.5em}
		
		A thesis submitted to the University of Manchester\\
		for the degree of Doctor of Philosophy, 2016
	\end{center}
	
	\vfill
	
	% The abstract
	
	SpiNNaker is an unconventional super computer architecture designed to
	simulate up to one billion biologically realistic neurons in real time. To
	achieve this goal, SpiNNaker employs an unconventional network architecture
	which poses a number of practical problems in scaling up from desktop
	prototypes to machine room filling installations.
	
	SpiNNaker's hexagonal torus network topology has received mostly theoretical
	treatment in the literature. This thesis tackles two chalenges encountered
	when building `real-world' systems.  Firstly, a scheme for laying out out
	hexagonal torus topologies in machine rooms while avoiding long cables is
	devised and demonstrated on a half-million core SpiNNaker prototype.
	Secondly, to improve the performance of existing routing algorithms, a more
	efficient process is proposed for finding short paths through hexagonal torus
	topologies. This is complemented by a formula enabling routing algorithms
	greater flexibility finding paths potentially resulting in more balanced
	network utilisation.
	
	The scale of SpiNNaker's networks and the models intended for it also present
	a practical challenge. A pair of placement and routing algorithms are
	developed which assign processes to nodes and generate paths through the
	network such that network congestion is reduced and network faults are
	tolerated.  The proposed placement algorithm is inspired by techniques used
	in chip design and is shown to allow larger applications to run -- with good
	performance -- on SpiNNaker than the previous state-of-the-art. Likewise the
	routing algorithm developed is able to tolerate network faults inevetably
	present in large scale systems with littler performance overhead.
	
	
	% Required to ensure single line spacing is used for this whole block
	\par%
}
