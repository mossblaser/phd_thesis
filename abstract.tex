{
	\prefacesection{Abstract}
	
	% Single line spacing for the abstract page
	\setstretch{1.0}
	
	
	\vfill
	
	% Standard thesis information
	\begin{center}
		\textsc{\large\thesistitle}
		
		\vspace{0.5em}
		
		\thesisauthor
		
		\vspace{0.5em}
		
		A thesis submitted to the University of Manchester\\
		for the degree of Doctor of Philosophy, 2016.
	\end{center}
	
	\vfill
	
	% The abstract
	
	SpiNNaker is an unconventional super computer architecture designed to
	simulate up to one billion biologically realistic neurons in real-time. To
	achieve this goal, SpiNNaker employs a novel network architecture which poses
	a number of practical problems in scaling up from desktop prototypes to
	machine room filling installations.
	
	SpiNNaker's hexagonal torus network topology has received mostly theoretical
	treatment in the literature. This thesis tackles some of the challenges
	encountered when building `real-world' systems.  Firstly, a scheme is devised
	for physically laying out hexagonal torus topologies in machine rooms which
	avoids long cables; this is demonstrated on a half-million core SpiNNaker
	prototype.  Secondly, to improve the performance of existing routing
	algorithms, a more efficient process is proposed for finding (logically)
	short paths through hexagonal torus topologies. This is complemented by a
	formula which provides routing algorithms greater flexibility when finding
	paths, potentially resulting in a more balanced network utilisation.
	
	The scale of SpiNNaker's network and the models intended for it also present
	their own challenges. Placement and routing algorithms are developed which
	assign processes to nodes and generate paths through SpiNNaker's network.
	These algorithms reduce congestion and tolerate network faults. The proposed
	placement algorithm is inspired by techniques used in chip design and is
	shown to enable larger applications to run on SpiNNaker, with good
	performance, than the previous state-of-the-art. Likewise the routing
	algorithm developed is able to tolerate network faults, inevitably present in
	large-scale systems, with little performance overhead.
	
	
	% Required to ensure single line spacing is used for this whole block
	\par%
}
