{
	\prefacesection{Acknowledgements}
	
	% Single line spacing
	\setstretch{1.0}
	
	It is often said that it is not \emph{what} you know but \emph{who} you know.
	Throughout the course of my PhD I've been exceptionally lucky to have been
	helped along (and sometimes simply tolerated) by a great number of people.
	
	Both my supervisor, Jim Garside, and co-supervisor, Steve Furber, have each
	spent countless hours patiently discussing and describing all manner of
	things with me while giving me great freedom in my project. Jim's office door
	has always been open to my unexpected interruptions be it about work, writing
	or walking.  Likewise, Steve has always managed to find time for both
	technical and frivolous endeavours alike. I'm also hugely grateful to Luis
	Plana who has been a rich source of sage advice, insightful questions
	patiently suffered many a stupid question.
	
	Various parts of the work in this thesis (and numerous side projects) would
	not have been possible if not for the multitude of discussions,
	collaborations and even sheer physical hard work of Steve Temple, Javier
	Navaridas, Simon Davidson and Dave Clark. I'm also indebted to Andrew Mundy
	and Jamie Knight, both of whom have donated so much time and effort towards
	verifying and using software implementations of the ideas in this thesis.
	
	A tribute must also be paid to Andrew and Jamie, along with Amanieu d'Antras
	and Andrew Webb and the other CDT members whose injection of lunchtime
	silliness always brightened my day. As well as the friendly and stimulating
	environment the School of Computer Science has provided me for the past seven
	years, I am also grateful for the funding it has provided for my research.
	
	My family, especially my parents have provided me with constant reassurance
	and many welcome distractions. Finally, I cannot thank my wonderful wife,
	Ann-Marie, enough. She has given so much kindness and love while enduring a
	lifetime's quota of conversations about hexagons.
	
	% Required to ensure single line spacing is used for this whole block
	\par%
}
